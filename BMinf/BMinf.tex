\documentclass[palatino,nochap]{apuntes}

\title{BMinf}
\author{Víctor de Juan}
\date{15/16 C2}

% Paquetes adicionales

% --------------------

\begin{document}
\pagestyle{plain}
\maketitle
\tableofcontents
\newpage

\section{Segundo parcial}

Este es un esquema de lo que hemos dado:


\[
\text{Recomendación} \to 
	\left\{
		\begin{array}{l}
		F.C \to
			\left\{ 
				\begin{array}{l}
					BU \to\left\{ 
						\begin{array}{l}
							\text{Normal}\\
							\text{Centrado en la media}
						\end{array}\right.\\
					BI \to \text{Normal}
				\end{array}\right.\\
		B.C. \to\left\{
			\begin{array}{l}
				\text{KNN}\\
				\text{Rocchio}
			\end{array}\right.
		\end{array}
	\right.
\]


Además, para cada uno de estos métodos tenemos diversas medidas de similitud. Típicamente \textbf{Pearson} y \textbf{coseno} para recomendadores no binarios y Jaccard para recomendadores binarios.

% Contenido.

\section{Redes sociales}

\concept{Closeness:} Inversa de la distancia mínima media.

\concept{Betweeness (nodo):} promedio de caminos de distancia mínima (CDM) del grafo que pasan por el nodo. 
Llamamos $ns_{v,w}$ al número de caminos de distancia mínima entre $v$ y $w$ y a $ns_{v,w}(u)$ al número de caminos de distancia mínima entre $v$ y $w$ que pasan por $u$. 
\[B(u) = \frac{1}{\rfrac{(n-2)(n-1)}{2}}\sum_{v,w≠u} \frac{ns_{v,w}(u)}{ns_{v,w}}\]

\concept{Betweeness (arco):} promedio de caminos de distancia mínima (CDM) del grafo que incluyen el arco. 
Llamamos $ns_{v,w}$ al número de caminos de distancia mínima entre $v$ y $w$ y a $ns_{v,w}(a,b)$ al número de caminos de distancia mínima entre $v$ y $w$ que incluyen el nodo que va de $a$ a $b$
\[B(a,b) = \frac{1}{\rfrac{(n-2)(n-1)}{2}}\sum_{v,w≠a,b} \frac{ns_{v,w}(a,b)}{ns_{v,w}}\]


%% Apéndices (ejercicios, exámenes)
\appendix

\chapter{---}
% -*- root: ../BMinf.tex -*-

\section{Primer parcial}

\section{Sistemas de recomendación}

\begin{table}[hbtp]
\centering
\begin{tabular}{c|ccccc}
&La feria & Origen & Rey León & Batman(B) & Club de la Lucha\\\hline
L&0&4&3&1&4\\
C&5&3&1&4&4\\
S&0&3&4&0&2\\
A&0&5&1&0&5
\end{tabular}
\caption{Matriz de recomendaciones del ejercicio 1.}
\label{data_ejer_2.1}
\end{table}

\begin{table}[hbtp]
\centering
\begin{tabular}{c|cccccc}
&La fiera & Origen & Rey León & Batman & C.\\\hline
Aventura 	&	0	&	0	&	1	&	1	&	0	\\
Animación	&	0	&	0	&	1	&	0	&	0	\\
Comedia		&	1	&	0	&	0	&	0	&	0	\\
Drama		&	0	&	0	&	1	&	0	&	1	\\
Romance		&	1	&	0	&	0	&	0	&	0	\\
Accion		&	0	&	1	&	0	&	1	&	0	\\
Sci-fi		&	0	&	1	&	0	&	1	&	0	\\
Thriller	&	0	&	1	&	0	&	0	&	0	
\end{tabular}
\caption{Matriz de características de los items del ejercicio \ref{ejer:2.1}}
\label{data_ejer_2.1_2}
\end{table}

\begin{problem}[?]
\label{ejer:2.1}

Tomando los datos de la matriz \ref{data_ejer_2.1}, nos preguntan:


\ppart $KNN$ centrado en la media, tomando $K=1$.

\ppart Recomendación basada en contenidos.

\ppart Evaluación.


\solution

\spart 
Calculamos las medias:

\begin{table}[hbtp]
\centering
\begin{tabular}{c|ccccc||c}
&La feria & Origen & Rey León & Batman & C. & Medias\\\hline
L&0&4&3&1&4&3\\
C&5&3&1&4&4&3.4\\
S&0&3&4&0&2&3\\
A&0&5&1&0&5&3.6
\end{tabular}
\caption{Matriz de recomendaciones.}
\end{table}

Calculamos la similitud de $S$ con la película \textit{Batman} (B) y para ello necesitamos la similitud con su vecino.

\[
	\hat{r}(S,B) = 3 + \frac{sim(S,L)·(1-2)}{sim(S,L)} = 1
\]

\spart Lo primero es definir qué características de la película vamos a utilizar como contenido. Las posibilidades son muchas: el director, la sinopsis haciendo tf-idf, los actores, género ...

\paragraph{Rocchio:}

\subparagraph*{Opción no binaria}
\[
s = 3 \begin{pmatrix}0\\0\\0\\0\\0\\1\\1\\1\end{pmatrix} +
4 \begin{pmatrix}1\\1\\1\\1\\0\\0\\0\\0\end{pmatrix}+
2 \begin{pmatrix}0\\0\\0\\1\\0\\0\\0\\0\end{pmatrix} =
\begin{pmatrix}4\\4\\4\\6\\0\\3\\3\\3\end{pmatrix}
\]

\[
cos(s,F) = ...
\]

\subparagraph*{Opción binaria}

\[
s = 
\begin{pmatrix}0\\0\\0\\0\\0\\1\\1\\1\end{pmatrix} +
\begin{pmatrix}1\\1\\1\\1\\0\\0\\0\\0\end{pmatrix}+
\begin{pmatrix}0\\0\\0\\1\\0\\0\\0\\0\end{pmatrix} =
\begin{pmatrix}1\\1\\1\\2\\0\\1\\1\\1\end{pmatrix}
\]

Al utilizar opción binaria, la similitud es Jaccard.


\[
	Jaccard(s,Fiera) = \frac{|s\cap F|}{|s\cup F|} = \frac{1}{8}
\]
\[
	Jaccard(s,Batman) = \frac{3}{7}
\]

\paragraph*{KNN}

Empezamos identificando los 2 vecinos más próximos a \textit{La Fiera}. Vamos a suponer que son \textit{Batman} y \textit{Rey León}.

Todo es exactamente igual que $KNN$ basado en contendio, salvo la similitud entre las películas. Para calcular la similitud entre las películas en un filtrado colaborativo se utilizan los vectores de recomendaciones de los usuarios para las películas. 
\footnote{Para calcular la similitud entre películas para una recomendación basada en contenidos se utilizan las etiquetas.}

\spart Evaluación:



\end{problem}
\printindex
\end{document}
