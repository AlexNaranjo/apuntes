% -*- root: ../EDP2016.tex -*-
\section{Hoja 1}


\begin{problem}[1]

	\ppart Utilizar el principio de Duhamel para resolver el problema
	\[ u_t + v u_x = f(x,t) \]
	con dato inicial $u(x,0) = 0$.

	Encontrar una solución explícita cuando $f(x,t) = e^{-t} \sin x$.

	\ppart Resolver el modelo de evolución de la contaminación en un río, incluyendo un término de depuración natural proporcional a la concentración que se rige por la ecuación:
	\[ u_t + vu_x = -\gamma u \]
	donde $\gamma$ es una constante positiva, con condiciones
	\[ u(x,0) = 0 \text{ si } x > 0, u(0,t) = \beta \text{ si } t > 0 \]

	\solution


\end{problem}











\begin{problem}[2] Resolver la ecuación de Burgers $u_t + uu_x = 0$ con dato inicial:
	\[ g(x) =
	\begin{cases}
		1 & x \leq 0 \\
		1-x & 0 < x < 1 \\
		0 & x \geq 1
	\end{cases} \]

	\solution

\end{problem}











\begin{problem}[3] Resolver la ecuación de Burgers $u_t + uu_x = 0$ con dato inicial:
	\[ g(x) =
	\begin{cases}
		0 & x < 0 \\
		1 & 0 < x < 1 \\
		0 & x > 1
	\end{cases} \]

	\solution


\end{problem}











\begin{problem}[4] Se considera la ecuación del transporte \[u_t + \frac{1}{u^2} u_x = 0\] con dato inicial \[u(x,0)=f(x) =
	\begin{cases}
	1 & x < 0 \\
	2 & x > 0
	\end{cases}\]
	
	Dibujar el diagrama $ X-T $ de la solución (características y valores de la solución). Dibujar las gráficas de las funciones $ u(1,t) $ y $ u(x,2) $.
	
\solution

\end{problem}









\begin{problem}[5]
	Resolver el problema de Cauchy $u_x + u_y = u^2, u(x,0) = 1$.

	\solution
\end{problem}











\begin{problem}[6]
	Resolver el problema de Cauchy $xu_x - yu_y = u - y, u(y^2,y)=y$. Estudiar si puede existir alguna solución definida en un entorno del origen.

	\solution
\end{problem}










\begin{problem}[7]
	Resolver la ecuación $u_x + 3y^{2/3} u_y = 2$ sujeta a la condición inicial $u(x,0) = φ(x)$.

	\solution

\end{problem}














\begin{problem}[8]
	Resolver la ecuación $(y + u)u_x + yu_y = x - y$ sujeta a la condición inicial $u(x,1) = 1 + x$.

	\solution
\end{problem}













\begin{problem}[9]
	Se considera la ecuación $yu_x - xu_y = 0 (y > 0)$. Para cada uno de los siguientes datos iniciales:

	\ppart $u(x,0) = x^2$
	\ppart $u(x,0) = x$
	\ppart $u(x,0) = x, x > 0$

	comprobar si el problema tiene solución. Si la tiene, encontrarla; si no la tiene, explicar por qué.

	\solution

\end{problem}











\begin{problem}[10]
	Resolver la ecuación $u_y + u^2 u_x = 0$ en $x > 0$ bajo la condición inicial $u(x,0) = \sqrt{x}$. ¿Cuál es el dominio de la existencia de la solución?

	\solution
\end{problem}













\begin{problem}[11]
	Resolver el problema de Cauchy $xu_x + yu_y = -u$, $ u(\cos s \sin s) = 1$, $ 0 \leq s \leq \pi$. ¿Está la solución definida en todas partes?

	\solution
\end{problem}








\begin{problem}[12]
	\ppart Encontrar una función $u = u(x,y)$ que resuelva el problema de Cauchy $(x + y^2)u_x + yu_y + (\frac{x}{y} - y) u = 1$, $u(x,1) = 0, x \in \real$.
	\ppart Comprobar si se cumple la condición de transversalidad.
	\ppart Dibujar las proyecciones sobre el plano $(x,y)$ de la condiciones inicial y de las características que emanan de los puntos $(2,1,0)$ y $(0,1,0)$.
	\ppart ¿Está la solución obtenida en $(a)$ definida en el origen $(x,y) = (0,0)$? ¿Contradice esto al teorema de la existencia-unicidad?

	\solution


\end{problem}


