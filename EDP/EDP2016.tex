\documentclass[palatino]{apuntes}

\title{Ecuaciones en Derivadas Parciales}
\author{Guillermo Guridi Mateos}
\date{15/16 C2}

% Paquetes adicionales
\usepackage{tikztools}
\usepackage{fancysprefs}
% --------------------

\bibliographystyle{plainnat}

\begin{document}
\pagestyle{plain}

% http://tex.stackexchange.com/a/14243
\relpenalty=10000
\binoppenalty=10000

\begin{abstract}
Estos son los apuntes del curso de Ecuaciones en Derivadas Parciales (2016), del profesor Jesús García Azorero. Es recomendable también mirar los apuntes del año 2015 \cite{ApuntesEDPRual}, que son del mismo curso aunque con un temario organizado de forma distinta.
\end{abstract}

\maketitle

\tableofcontents
\newpage
% Contenido.

\chapter{Introducción}

\begin{figure}[hbtp]
\centering
\inputtikz{TransmisionCalor}
\caption{Esquema de la varilla (en negro) aislada y conectada a dos bloques de hielo (temperatura 0)}.
\label{fig:TransmisionCalor}
\end{figure}

El primer ejemplo que vamos a ver tiene que ver con la ecuación del calor y la solución de Fourier. Consideraremos una varilla de longitud $L$ aislada, y conectada con dos bloques de hielo. Llamaremos $u(x,t)$ a la temperatura en el punto $x$ en el instante $t$. Sabemos que en $t = 0$ tenemos una distribución inicial de temperatura, y en $t \to ∞$ tenemos que la temperatura será cero (no llega a fundir completamente los bloques de hielo).

Como notación, escribiremos las derivadas parciales con subíndices: $u_t \equiv \pd{u}{t}$, $u_{xx} = \pd[2]{u}{x}$, y similar.

Para estudiar el problema, vamos a ver que ocurre en el intervalo $[x, x+h]$. La temperatura, al final nos da una medida de la energía de los puntos. Si la sumamos en ese intervalo obtendremos una cantidad de calor acumulada en ese intervalo. Es decir, $\int_x^{x+h} u(s,t) \dif s$ es la cantidad de calor en el instante $t$ en $[x,x+h]$. Lo interesante es que esa cantidad de calor varía a lo largo del tiempo. Como la varilla está rodeada de aislante, la cantidad de calor sólo se puede estar perdiendo a través de los extremos.

Suponemos que podemos meter la derivada dentro de la integral (habría que demostrarlo), así que podemos decir que \(  \od{}{t} \int_x^{x+h} u(s,t) \dif s = \int_x^{x+h} u_t(s,t) \dif s \label{eq:intro_1}\)

La difusión de la temperatura será, según la ley de Newton, proporcional a la temperatura en el punto que estemos considerando\footnote{Como no somos físicos, la constante es 1.}. Si en el extremo derecho tenemos una derivada positiva (con respecto a x), estamos aportando calor al intervalo. Si tenemos derivada positiva en el extremo izquierdo, estamos perdiendo calor en el intervalo.

Tenemos entonces que ese intercambio de calor es proporcional a \[u_x(x+h, t) - u_x(x,t)\]

Por la regla de Barrow\footnote{Corolario del Teorema Fundamental del Cálculo} podemos decir que \[ u_x(x+h, t) - u_x(x,t) = \int_x^{x+h} u_{xx}(s,t) \dif s\]
Con esto, hemos desarrollado el lado derecho de la ecuación \ref{eq:intro_1}, y pasando el lado derecho restando, combinando las integrales\footnote{Los límites de integración son los mismos así que lo podemos hacer.}, y multiplicando por $\frac{1}{h}$ (ahora veremos por qué) tenemos que
\( \frac{1}{h} \int_{x}^{x+h} u_t(s,t) - u_{xx}(s,t) \dif s = 0 \label{eq:intro_2}\)

Si definimos $$F(x) = \int_{0}^{x} u_t(s,t) - u_{xx}(s,t) \dif s $$ entonces \ref{eq:intro_2}, por definición de cociente incremental, es:
\( \ref{eq:intro_2} = \frac{F(x+h) - F(x)}{h} = 0  \label{eq:intro_3}\)

y aplicando el límite $h \to 0$ y por el TFC\footnote{Teorema Fundamental del Cálculo}, obtenemos que \ref{eq:intro_3} tiende a:
\[ F'(x) = u_t(s,t) - u_{xx}(s,t) \]
Luego como \ref{eq:intro_3} era igual a 0, tenemos que,
\[ u_t(x,t) - u_{xx} (x,t) = 0, \forall x \in (0,1), \forall t > 0\]

El problema que tendremos que resolver es entonces que
$$u(0,t) = u(L,t) = 0 \text{ para } t > 0$$

¿Cómo resolver esto? Buscaremos que $u(x,t) = X(x) · T(t)$, es decir, que podamos separar las variables. En ese caso, tenemos que la ecuación pasa a ser \[ u_t - u_{xx} = 0 \implies XT' - X''T = 0 \implies \frac{X''}{X} = \frac{T'}{T} \]

La igualdad se debe cumplir $∀x ∈ (0,L)$ y $∀t > 0$. Tenemos a dos ecuaciones que dependen de dos variables distintas y que son iguales, luego debe existir un $λ ∈ ℝ$ tal que $\frac{X''}{X} = \frac{T'}{T} = λ$, y nos movemos al mundo de las ecuaciones ordinarias, llegando al sistema  \[ \begin{matrix} T' - λT = 0 \\ X'' - λX = 0 \end{matrix}\]

El problema es que no sabemos qué es λ, así que tenemos que irnos al resto de condiciones del problema. Sabíamos que en los extremos la temperatura es 0, así que tenemos que añadir esas condiciones $X(0) = 0 = X(L)$. Sin embargo, lo que estudiábamos era el problema de Cauchy, en el que teníamos valores sobre la función y su derivada en el mismo punto. Este es un problema de contorno que se queda fuera del temario de EDO (\cite{ApuntesEDO}). Como tenemos una solución trivial ($X = 0$), tendremos que buscar para qué valores de λ no se cumple el teorema de existencia y unicidad.

Lo que sí sabemos resolver es ver qué tipo de soluciones tenemos para $X'' - λX = 0$. Para ello, resolvemos el polinomio característico. Para $λ = 0$, tenemos que $X(x) = a+bx$. Como $X(0) = 0 = a$, y $X(L) = bL = 0$, tenemos que $X = 0$ y no nos vale porque es una solución trivial.

Buscamos ahora ver qué pasa con $λ > 0$. Podemos decir perfectamente que $λ = μ^2$. En este caso, tendríamos que las soluciones son $X(x) = ae^{μx} + be^{-μx}$. Sin embargo, las condiciones de contorno nos dirán que $a = b = 0$.

Nos falta probar qué pasa con $λ < 0$. Igual que antes, ponemos $λ = -μ^2$. Repasando de nuevo el curso de EDO, tenemos que $X(x) = a \cos μx + b \sin μx$. Con las condiciones de contorno tenemos que $X(x) = b \sin μx$ y con la constante $μ = \frac{kπ}{L}$ para $k = 1,2,3,\dotsc$.

Las soluciones serán entonces de la forma \[ X_k = b_k \sin (\frac{kπ}{L} x)\] correspondientes a los autovalores $λ_k = -\left(\frac{kπ}{L}\right)^2$. % \lambda < 0

Por otra parte, miramos qué ocurre con $T$: resolviendo tenemos que $T_k(t) = c_k e^{λ_k t}$, así que las soluciones particulares de la ecuación del calor son \[ u_k(x,t) = a_k e^{-\left(\frac{kπ}{L}\right)^2t} \sin (\frac{kπ}{L} x) \]

Pero no hemos terminado el problema: nos falta meter el dato inicial $u(x,0) = f(x)$. Por ejemplo, si $f(x) = 5 \sin (\frac{2π}{L} x)$ entonces $u(x,t) = 5 e^{-\left(\frac{2π}{L}\right)^2t} \sin (\frac{2π}{L} x)$.

Como la ecuación es lineal, también podemos sacar la solución para funciones que sean sumas de senos; y utilizando la propiedad de que dadas dos soluciones particulares, si las sumamos también es solución, podemos llegar a la idea de Fourier.

¿Nos vale con sumas de senos? La respuesta es que sí: Fourier dice que cualquier función se puede escribir como suma infinita de senos, y por lo tanto siempre podemos encontrar la solución: \[ f(x) = \sum_{k=1}^∞ a_k \sin \frac{kπ}{L} x\]

La cuestión es que ahí nos encontramos con problemas:
\begin{itemize}
	\item La serie es alternada y no sabemos si converge.
	\item No sabemos cómo se calculan los $a_k$.
	\item No sabemos exactamente qué significa que una función sea igual a una serie infinita.
	\item No sabemos si la derivada de la función es igual a la derivada de la derecha.
\end{itemize}

Todos estos problemas son los que motivan el análisis funcional y las matemáticas del siglo XIX. Nosotros los veremos en este curso en el \fref{chap:EcuacionesSegundoOrden}.

\chapter{Ecuaciones de primer orden}

\section{Modelo de un atasco}

Vamos a empezar con un modelo sencillo. Observemos el fenómeno del embotellamiento fantasma, el cual tiene una ecuación bastante sencilla. Veamos como se generan esos atascos y como se disipan.

Estudiaremos una función que nos de la densidad de coches en un punto, en el tiempo.

$$u(x,t) = \text{densidad}$$

Observemos un modelo muy simple en el que solo hay un carril y los coches fluyen solo en una dirección.

(DIBUO DE LA RECTA CON FLECHA HACIA LA DERECHA)

La variación en el número de coches de un intervalo será:

$$ \frac{d}{dt} \int^{x+h}_{x} u(s,t) ds = \int^{x+h}_{x} u_t(s,t) ds $$

Esto equivale a al número de coches que entran por x menos los que salen por x + h. Es decir, tenemos un flujo:

$$\text{Flujo: } q(u, x, t)$$

Si suponemos que nos fijamos en un instante lo suficientemente pequeño en un tramo pequeño podemos asumir que depende solo de $u$. En ese caso, si $q > 0$ el flujo sería hacia la derecha.

$$ q(u(x,t)) - q(u(x+h,t)) = \int^{x+h}_{x} [q(u)]_{x} $$

Conclusión:

$$ u_t + [q(u)]_{x} = 0$$

con un dato: $u(x, 0) = F(x)$.

Ahora debemos probar distintas funciones q.


\subsection{Modelo básico}

$$q(u) = cu$$

$$u_t + cu_x = 0 \quad\quad u(x,0) = F(x)$$


Esta ecuación dice que el flujo es proporcional a la densidad. Implica que los coches se mueven con velocidad constante, a más velocidad más flujo y viceversa. Si en el instante inicial tenemos una función de la densidad respecto de las zonas del tramo, según pasa el tiempo la densidad se irá desplazando hacia la derecha manteniendo la misma forma.

(DIBUJO)

\textbf{Comprobación:}

$$ u(x,t) = F(x-ct) $$
$$ u_x(x,t) = F'(x-ct)$$
$$u_t(x,t) = F'(x-ct)*(-c)$$

$$ \{ u_t + cu_x = … = 0 \quad\quad u(x,0) = F(x) \} $$

\subsubsection{Representación gráfica}


$$u(x,t) = \text{Cte.}$$
$$F(x - ct) = \text{Cte.}$$

Conclusión: $x-ct = k$.


Las soluciones dependiendo de k se desplazan por el plano siguiendo una recta de inclinación $c$. Siguiendo una fórmula del tipo $x-ct = k$. Con esto se puede obtener como avanzan los datos (densidades) respecto de un instante inicial.

Esto son los conjuntos de nivel de las soluciones.

(DIBUJO)



\subsection{Otro ejemplo}

Imaginemos un río con una velocidad constante, en el que a partir de un punto realizamos un vertido. El río tiene velocidad $v$ y en $x=0$ realizaremos una contaminación $\beta{t}$. En $t=0$ consideraremos que está limpio.

Tenemos las ecuaciones:

$$u_t + vu_x = 0$$
$$u(x,0) = 0 \quad (x>0) $$
$$u(0,t) = \beta(t), t>0$$


Resolvemos el problema igual que antes, buscamos los conjuntos de nivel que serán rectas como nos ha salido antes. Vamos a buscar esas rectas.

\textbf{Buscamos $x(t)$ tal que $u(x(t),t)$ sea constante}

Derivando en t: $u_x x' + u_t = 0$ que junto con la ecuación $u_t + v u_x = 0$ nos da que $x' = v \Rightarrow x = x_0 + vt $.

Las características saldrían $x-vt = \text{Cte}(=X_0)$.

Esas son nuestras rectas conjuntos de nivel, que en este caso indicarían el frente de la contaminación, el límite a partir del cual el río sigue limpio.

$$u(x.t) = 0 \quad(x-vt) > 0 $$
$$u(x,t) = \beta{t - \frac{x}{v}} \quad (x-vt) < 0 $$

(esto se obtiene de lo que viene a continuación)

$$0 - vt^{*} \Rightarrow x-vt = -vt^{*}$$

$$U(x,t) = \beta{t^*} = \beta(t - \frac{x}{v})$$

cuando $x - vt < 0$.

De esto obtenemos una función de x en función del tiempo que nos permite saber cuando una parte del río se contamina.

\subsection{Ejemplo más avanzado}

Supongamos que existe descomposición biológica:

$$u_t + vu_x = -\gamma u$$
$$u(x,0) = \text{Cte}$$
$$u(0,t) = \beta$$

(Cambiamos de variable)

$$u_t + \gamma_u + vu_x = 0$$
$$e^{\gamma t} u_t + e^{\gamma t} u + v e^{ \gamma t} u_x = 0 $$

lo que es lo mismo

$$(e^{\gamma t}u)_t + v (e^{\gamma t} u)_x$$

Y definimos la función $W$:

$$W = e^{\gamma t}u$$.


Hay que comprobar cual es el efecto del término de descomposición biológica ($-\gamma u$). ¿ $W(x,0)$ , $ W(0,t)$ ?. Esto queda como ejercicio al lector.


\subsection{Modelo con aportación externa}

$$u_t + vu_x = f(x,t)$$
$$u(x,0) = F(x)$$

Ya al no ver constante, la solución no va a ser constante a lo largo de las características, que siguen existiendo. En el caso $f = 0$ estas daban la solución, pero ahora van a influir.

\subsubsection{Solución 1}

Cambio de variables:

$$z = x-vt$$

Lo cual hace más sencillo observar la variación temporal. Continuamos el cambio:

$$x = z + vt$$
$$W(z,t) = u(z+vtt)$$

Entonces queremos estudiar la variación de $W$ respecto del tiempo:

$$W_t(z,t) = u_x(z + vt, t)v + u_t(z+vt, t) = f(z + vt, t)$$

Integramos:

$$W(z,t) = W(0,t) + \int^{t}_{0} f(z+vz, z) dz $$

Lo que se entiende como el dato inicial más la suma de las aportaciones de esta función $f$ en el tiempo pasado. Ahora deshacemos el cambio de variable:

$$u(z + vt, t) = u(z,0) + \int^{b}_[0] f(z+vz, z) dz$$
$$z = x-vt$$
$$u(x,t) = u(x-vt,0)+ \int^{t}_[0] f(x-v(t-z),z) dz$$

(donde $u(x-vt,0) = F(x-vt)$)

Esto es un milagro ya que tiene una fórmula explícita, que es poco común en EDOs y EDPs.

\subsubsection{Solución 2}

Vamos a descomponer el sistema en 2. Dos soluciones en las que cada una va a ser solución de dos partes del problema que nos interesan. Una del dato inicial y otra del resto.

$$u = \phi + \psi$$

Por un lado:

$$\phi_t + v\phi_x = 0$$
$$\phi(x,0) = F(x) $$

$$\phi(x,t) = F(x-vt)$$

Y por otro:

$$\psi_t + v\psi_x = f(xt)$$
$$\psi(x,0) = 0$$

Este problema con DUHAMEL (la fuente externa se interpreta como una fuente de datos iniciales, se cambia el cero de la segunda ecuación de las anteriores) se puede interpretar como que tenemos una función que transporta el sistema.

$$\xi_t + v\xi_x = 0$$
$$\xi(x,s) + f(x,s)$$

Continuamos:

$$\xi(x,t) = f(x^*,s)$$
$$x-vt = k$$
$$x^* - vs = k$$
$$x-vt = x^* - vs$$
$$x-v(t-s) = x^{*}$$

$$\xi(x,t) = f(x - v(t-s), s)$$


Tenemos la solución correspondiente de un aporte instantáneo en el punto $t$. Por lo tanto la solución será la suma de todos los aportes instantáneos.

$$\Rightarrow \psi(x,t) = \int^{t}_{0} f(x-v(t-s),s) ds $$

Aunque hemos hecho un montón de asunciones que tendremos que comprobar en algún momento.

\subsection{Modelo de tráfico más realista}

No hemos tenido en cuenta que la velocidad de los coches no es totalmente proporcional a la densidad. Si esta baja mucho los coches llegan a pararse. Hay una densidad máxima donde el tráfico quedará totalmente estancado (los coches pegados).

Una función más realista sería una parábola. Si no hay coches no pasa ninguno, pero a partir de un punto cuantos más coches menos rápido van y menos pasan.

$$ q(u) = Au (B-u) = ABu - Au^{2} $$

Veamos su ecuación:

$$u_t + [q(u)]_x = 0$$
$$ … $$
$$ u_t + (AB - 2Au) u_x = 0 $$

La velocidad va a ser dependiente de la densidad.

Nos saldrán características cuya pendiente dependa de el valor en el tiempo inicial. Si divergen no habrá problema, pero si convergen habrá puntos donde rectas de densidad 2 (por ejemplo) cortarán con rectas de densidad 1. Esto se entiende como coches que van más rápido que se encuentran con coches que van más lentos y se ven obligados a frenar.






\chapter{Ecuaciones de segundo orden}
\label{chap:EcuacionesSegundoOrden}

\section{Método de separación de variables}

\section{Series de Fourier}

\chapter{Comportamiento cualitativo}

%% Apéndices (ejercicios, exámenes)
\appendix

\bibliography{../Apuntes}{}
\printindex
\end{document}

