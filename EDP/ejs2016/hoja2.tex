% -*- root: ../EDP2016.tex -*-
\section{Hoja 2}

\begin{problem} Determinar si se puede usar el método de separación de variables para cada una de las EDPs siguientes. En caso afirmativo, hallar las EDOs resultantes. En ningún caso se pide resolver.

\ppart $xu_{xx} + u_t = 0$.
\ppart $tu_{xx} + xu_t = 0$.
\ppart $u_{xx} + u_{xt} + u_t = 0$.
\ppart $(ρ(x) u_x)_x - r(x) u_{tt} = 0$.
\ppart $u_{xx} + (x + y) u_{yy} = 0$.

\solution

\spart Suponemos que $u = XT$, sustituimos y \[ 0 = xu_{xx} + u_t = xX''T + T'X \implies \frac{x · X''}{X} = \frac{-T'}{T} = λ \]

\spart \[ 0 = tu_{xx} + xu_t = t X'' T + x T' X \implies \frac{X''}{xX} = \frac{-T'}{tT} = λ\]

\spart \[ 0  = u_{xx} + u_{xt} + u_t = X''T + X'T' + XT' \implies \frac{X''}{X' + X} = \frac{T'}{T} \]

\spart \[ 0 = (ρ(x) u_x)_x - r(x) u_{tt} = ρ' X'T + ρ X''T - rX T'' \implies \frac{ρ'X' + ρX''}{-rX} = \frac{T''}{T} \]

\spart \[ 0 = u_{xx} + (x + y) u_{yy} = X''Y + (x + y) X Y'' \] y esto no da mucha sensación de poderse separar.

\end{problem}

\begin{problem} Consideramos la ecuación del calor en dos dimensiones espaciales, $u_t = k Δ u$, donde $u = u(x,y,t)$ y $Δ = \pd[2]{}{x} + \pd[2]{}{y}$ es el laplaciano respecto a las variales espaciales $x,y$.

Consideramos una solución de la forma $u(x,y,t) = X(x) Y(y) T(t)$. Hallar las EDOs satisfechas por $X$, $Y$ y $T$.

\solution

Vamos a sustituir en la ecuación: \[ 0 = u_t - k Δ u = XYT' - kT\left(X''Y + XY''\right) \implies \frac{T'}{T} = \frac{XY}{k(X''Y + XY'')} = λ \]

Trabajamos con el segundo cociente a ver si se puede hacer algo: \begin{align*}
λ &=  \frac{XY}{k(X''Y + XY'')} \\
kλ(X'' Y + XY'') &= XY \\
kλX''Y &= X(Y - Y'') \\
kλ\frac{X''}{X} &= \frac{Y - Y''}{Y} = μ
\end{align*} y listos.

\end{problem}

\begin{problem} El movimiento de una membrana circular está gobernado por la ecuación de ondas en dos dimensiones espaciales: \[ u_{tt} = c^2(u_{xx} + u_{yy})\qquad x^2 + y^2 ≤ R^2, \; t > 0 \]

\ppart Escribir la ecuación en coordenadas polares $(r,θ)$.
\ppart Consideramos una solución de la forma $u(r,θ,t) = R(r)Θ(θ)T(t)$. Encontrar las EDOs satisfechas por $R$, $Θ$ y $T$.

\solution

\end{problem}

\begin{problem}[4] Encontrar la solución del problema \[ \begin{cases}
u_{tt} = u_{xx} 	& 0 < x < π,\; t > 0 \\
u(0,t) = 0 = u(π,t) & t ≥ 0 \\
u(x,0) = \sin^3 x	& 0 ≤ x ≤ π \\
u_t(x,0) = \sin 2x 	& 0 ≤ x ≤ π
\end{cases} \]

\solution

\textbf{Nota de Azorero:}

Sabemos:

\[f(x) =  \frac{a_0}{2} +\sum_k a_k \cos (kx) + b_k \sin(kx)  \]

\[ x \in [-\pi,\pi]\]

\[ \gor{f} = \begin{cases}
	f & \text{ en }[0,\pi] \\
	\text{extensión } & \text{ en } [-R,0]
\end{cases} \]

\[a_k = \frac{1}{\pi}  \int^{\pi}_{-\pi} \underbrace{\gor{f}(\underbrace{\cos kx}_{\text{par}}) }_{\text{impar}} dx \qeq 0 \]

Solución: $\gor{f} = $ extensión impar de f.

\[ b_k = \frac{1}{\pi} \int_{-\pi}^{\pi} \underbrace{\underbrace{\gor{f}(x)}_{\text{impar}} \underbrace{\sin(kx)}_{\text{impar}}}_{\text{par}} dx = 2 \frac{1}{\pi} \int^\pi_{0} f(x) \sin(kx) dx \]

La idea es que nos quede un desarrollo en serie de senos, ya que esta es una ecuación de contorno Dirichlet. Hacemos separción de variables y tenemos que \[ T''X = X''T \implies \frac{T''}{T} = \frac{X''}{X} = λ \]

Si $λ = 0$, entonces $X'' = 0$, $X'$ constante y $X$ es polinomio de grado uno. Si $X(x) = a + bx$, como $X(0) = 0$, $a = 0$, y como $X(π) = 0$, $b = 0$ por lo que por aquí no llegamos a nada.

Si $λ > 0$, $λ = μ^2$, la solución será de la forma $X(x) = a e^{μx} + be^{-μx}$. Si $X(0) = 0$, entonces $a = -b$; y si $X(π) = 0$, entonces $0 = a(e^{μπ} - e^{-μπ})$, así que o bien $a = 0$ o bien $μ = 0$. Lo segundo es imposible, lo primero nos da una ecuación trivial.

Si $λ < 0$, $λ = -μ^2$, la solución será de la forma $X(x) = a \cos μx + b \sin μ x$. Si $X(0) = 0$, entonces $a = 0$. Si $X(π) = 0$, entonces $0 = b \sin μπ$ y por lo tanto $μ = k ∈ ℤ$.

Resolvemos ahora para $T$ sabiendo que $λ = λ_k = - k^2$, con $k ∈ ℤ$. Tenemos que \[ \frac{T''}{T} = -k^2 \], vamos a tener igualmente una solución de la forma $T(t) = a \cos k t + b \sin k t$. Si $X(x) T(0) = \sin^3 x$, entonces $T(0) = 1$ y $a = 1$. Si $u_t (x,0) = \sin 2x$, entonces $T'(0) = 1$. Derivando, \[ T'(t) = -a \sin k t + b \cos k t\], luego $b = 1$ y $T(t) = \sin k t + \cos k t $.

Para ajustar esto, tenemos que escribir $\sin^3 x$ y $\sin 2x$ como serie de cosenos. Usamos el truco de Azorero y tomamos la extensión impar de $f$, de tal forma que podrmeos calcular \[ b_k = \frac{2}{π} \int_0^π f(x) \sin kx \dif x \] siguiendo las ecuaciones de \eqref{eq:CoefsFourier}. Vamos a ello:
\begin{align*}
b_k &= \frac{2}{π} \int_0^π \underbracket{\sin^3 x}_{u} \underbracket{\sin k x \dif x}_{\dif v}
	= \frac{2}{π} \left(-k \sin^3 x \cos x + \int_{0}^π 3k \cos kx \sin^2 \cos x \right|_{x = 0}^π
\end{align*}

No me apetece seguir. Hágalo usted mismo.

\seprule

Otra posibilidad es usar la fórmula de D'Alembert \eqref{eq:DALEMBERT}, con $f(x) = \sin^3 x$ y $g(x) = \sin 2x$, de tal forma que \begin{multline*} u(x,t) = \frac{\sin^3 (x+t) + \sin^3(x-t)}{2} + \frac{1}{2} \int_{x-t}^{x+t} \sin 2s \dif s =  \\ =  \frac{\sin^3 (x+t) + \sin^3(x-t) - \cos 2(x+t) + \cos 2(x-t)}{2} \end{multline*}

\end{problem}

\begin{problem}[6] Resolver la siguiente ecuación: \[
\begin{cases}
u_t - u_{xx} = hu & x ∈ (0,π), t > 0 \\
u(0,t) = u(π,t) = 0 \\
u(x,0) = x(π-x)
\end{cases}\]

Usar el método de separación de variables. ¿Cuánto será $\lim_{t \to ∞} u(x,t)$?

\solution

Por searación de variables, buscamos $u(x,t) = X(x)·T(t)$, así que \begin{align*}
T'X-TX'' &= hXT \\
\frac{T'}{T} - \frac{X''}{X} &= h \\
\frac{T'}{T} &= \frac{X''}{X} + h = λ ∈ ℝ \\
\frac{T'}{T} - h &= \frac{X''}{X} = λ ∈ ℝ
\end{align*}

Además,la condición de contorno dice que $X(0) = X(π) = 0$. Podemos hacerlo de las dos formas que planteamos y resolviendo el sistema correspondiente. En nuestro caso resolveremos el siguiente sistema, aunque da lo mismo:
\begin{gather*}
\begin{cases}
X'' = λX \\
X(0) = X(L) = 0
\end{cases}
\\
\begin{cases} T' = (λ+h)T \end{cases} \end{gather*}

De nuevo haciendo las cuentas de toda la vida, tenemos que tener $λ_k = - k^2$ y las soluciones de la forma \[ X_k(x) = a_k \sin kx \], y con esto resolvemos en $T$ y tenemos \[ T_k(t) = α_k e^{-(k^2-h)t} \]

Las soluciones particulares serán entonces \[ u_k(x,t) = X_k(x) T_k(t) = A_k e^{-(k^2-h)t} \sin kx \] y la solución general de la forma \[ u(x,t) = \sum_{k=1}^{∞} A_k e^{-(k^2-h)t} \sin kx \], donde los $A_k$ serán los coeficientes del desarrollo en serie de senos del dato $x(π-x)$.

Una vez resuelto todo, lo que vemos es que la única diferencia con otros ejercicios es el coeficiente $h$ en el lado derecho de la ecuación. Así, si $h$ es positivo y muy grande podemos tener el problema de que esa suma no converja.

Si $h < 1$, todas las exponenciales tienen exponente negativo, así que $u(x,t) \convs[][t] 0$. Si $h = 1$, sólo la primera tiene exponente positivo, así que $u(x,t) \convs[][t] A_1 \sin x$. El problema ocurrirá cuando $h > 1$, el primer exponente es positivo y el límite no existe.

\end{problem}

\begin{problem}[11] Pruébese que \[ \lim_{n\to ∞} \int_0^π \log x \sin nx \dif x = 0\]

\solution

\end{problem}


\begin{problem}[9]

	\[
	\begin{array}{l}
		u_t - u_{xx} = 0, x \in (0,1), t>0
	\end{array}
	\]

	\solution

	El dato de contorno no es homogéneo ($\neq 0$). Intentaremos descomponer el problema. Encontrar una solución que sea suma de dos funciones $u=v+w$. COn la $v$ intentaremos ajustar el dato de contorno y con la $w$ hacer que el dato inicial se cumpla.

	\[ v_t - v_{xx} = 0 \]
	\[ v(0,t) = 0, v(1,t) = 1 \]

	Tenemos una solución trivial, estacionária que sería:

	\[ \left.\begin{array}{l}
		v''(x) = 0 \\
		v(0) = 0 \\
		v(1) = 1
	\end{array}\right\} \Rightarrow v(x) = x  \]

	Faltaría calcular la $w$:

	\[ \begin{array}{l}
		w_t - w_xx = 0 \\
		w(0,t) = 0 = w(1,t) \text{ dato de contorno homogéneo } \\
		w(x,0) = u(x,0) -v(x) = \Phi(x) - x
		\end{array}
	\]

	Y hemos llegado a un problema de los que ya sabemos resolver así que no tenemos problema. Se queda pendiente terminar el ejercicio.

\end{problem}

\begin{problem}[10]

	\[f(x) = e^x, x \in (-\pi,\pi)\]
	\[S_n f(x) = \text{N-ésima suma de Fourier de f}\]

	\[\text{¿} S_n f(\pi) \convs * \text{?}\]

	\solution

	\begin{center}
	\begin{tikzpicture}
	\draw[-] (-3,0) -- (3,0) node [below] {$\theta$};
	\draw[-] (0,-0.5) -- (0,2);

	\draw[-] (-1.5,0.1) -- (-1.5,-0.1) node[below] {$-\pi$};
	\draw[-] (1.5,0.1) -- (1.5,-0.1) node[below] {$\pi$};

	\draw[blue, thick] (-1.5,0.2) .. controls (0,0.6) .. (1.5,1.5) ;
	\draw[blue, thick] (-4.5,0.2) .. controls (-3,0.6) .. (-1.5,1.5) ;
	\draw[blue, thick] (1.5,0.2) .. controls (3,0.6) .. (4.5,1.5) ;

	\draw[dashed, green] (1.5,2) -- (1.5,0);

	\end{tikzpicture}
	\end{center}

	Representamos la extensión $2\pi$-periódica de la función. Y nos preguntan que pasa en el punto $\pi$. Tenemos dos teoremas de Dirichlet sobre convergencia puntual. En este caso usamos el segundo teorema de Dirichlet que nos dice:

	\[S_n f(\pi) \convs \frac{e^\pi+e^{-\pi}}{2}\]


\end{problem}



