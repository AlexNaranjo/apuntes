% -*- root: ../EDP2016.tex -*-
\section{Hoja 2}

\begin{problem} Determinar si se puede usar el método de separación de variables para cada una de las EDPs siguientes. En caso afirmativo, hallar las EDOs resultantes. En ningún caso se pide resolver.

\ppart $xu_{xx} + u_t = 0$.
\ppart $tu_{xx} + xu_t = 0$.
\ppart $u_{xx} + u_{xt} + u_t = 0$.
\ppart $(ρ(x) u_x)_x - r(x) u_{tt} = 0$.
\ppart $u_{xx} + (x + y) u_{yy} = 0$.

\solution

\spart Suponemos que $u = XT$, sustituimos y \[ 0 = xu_{xx} + u_t = xX''T + T'X \implies \frac{x · X''}{X} = \frac{-T'}{T} = λ \]

\spart \[ 0 = tu_{xx} + xu_t = t X'' T + x T' X \implies \frac{X''}{xX} = \frac{-T'}{tT} = λ\]

\spart \[ 0  = u_{xx} + u_{xt} + u_t = X''T + X'T' + XT' \implies \frac{X''}{X' + X} = \frac{T'}{T} \]

\spart \[ 0 = (ρ(x) u_x)_x - r(x) u_{tt} = ρ' X'T + ρ X''T - rX T'' \implies \frac{ρ'X' + ρX''}{-rX} = \frac{T''}{T} \]

\spart \[ 0 = u_{xx} + (x + y) u_{yy} = X''Y + (x + y) X Y'' \] y esto no da mucha sensación de poderse separar.

\end{problem}

\begin{problem} Consideramos la ecuación del calor en dos dimensiones espaciales, $u_t = k Δ u$, donde $u = u(x,y,t)$ y $Δ = \pd[2]{}{x} + \pd[2]{}{y}$ es el laplaciano respecto a las variales espaciales $x,y$.

Consideramos una solución de la forma $u(x,y,t) = X(x) Y(y) T(t)$. Hallar las EDOs satisfechas por $X$, $Y$ y $T$.

\solution

Vamos a sustituir en la ecuación: \[ 0 = u_t - k Δ u = XYT' - kT\left(X''Y + XY''\right) \implies \frac{T'}{T} = \frac{XY}{k(X''Y + XY'')} = λ \]

Trabajamos con el segundo cociente a ver si se puede hacer algo: \begin{align*}
λ &=  \frac{XY}{k(X''Y + XY'')} \\
kλ(X'' Y + XY'') &= XY \\
kλX''Y &= X(Y - Y'') \\
kλ\frac{X''}{X} &= \frac{Y - Y''}{Y} = μ
\end{align*} y listos.

\end{problem}

\begin{problem} El movimiento de una membrana circular está gobernado por la ecuación de ondas en dos dimensiones espaciales: \[ u_{tt} = c^2(u_{xx} + u_{yy})\qquad x^2 + y^2 ≤ R^2, \; t > 0 \]

\ppart Escribir la ecuación en coordenadas polares $(r,θ)$.
\ppart Consideramos una solución de la forma $u(r,θ,t) = R(r)Θ(θ)T(t)$. Encontrar las EDOs satisfechas por $R$, $Θ$ y $T$.

\solution

\end{problem}

\begin{problem}[4] Encontrar la solución del problema \[ \begin{cases}
u_{tt} = u_{xx} 	& 0 < x < π,\; t > 0 \\
u(0,t) = 0 = u(π,t) & t ≥ 0 \\
u(x,0) = \sin^3 x	& 0 ≤ x ≤ π \\
u_t(x,0) = \sin 2x 	& 0 ≤ x ≤ π
\end{cases} \]

\solution

\textbf{Nota de Azorero:}

Sabemos:

\[f(x) =  \frac{a_0}{2} +\sum_k a_k \cos (kx) + b_k \sin(kx)  \]

\[ x \in [-\pi,\pi]\]

\[ \gor{f} = \begin{cases}
	f & \text{ en }[0,\pi] \\
	\text{extensión } & \text{ en } [-R,0]
\end{cases} \]

\[a_k = \frac{1}{\pi}  \int^{\pi}_{-\pi} \underbrace{\gor{f}(\underbrace{\cos kx}_{\text{par}}) }_{\text{impar}} dx \qeq 0 \]

Solución: $\gor{f} = $ extensión impar de f.

\[ b_k = \frac{1}{\pi} \int_{-\pi}^{\pi} \underbrace{\underbrace{\gor{f}(x)}_{\text{impar}} \underbrace{\sin(kx)}_{\text{impar}}}_{\text{par}} dx = 2 \frac{1}{\pi} \int^\pi_{0} f(x) \sin(kx) dx \]

La idea es que nos quede un desarrollo en serie de senos, ya que esta es una ecuación de contorno Dirichlet. Hacemos separción de variables y tenemos que \[ T''X = X''T \implies \frac{T''}{T} = \frac{X''}{X} = λ \]

Si $λ = 0$, entonces $X'' = 0$, $X'$ constante y $X$ es polinomio de grado uno. Si $X(x) = a + bx$, como $X(0) = 0$, $a = 0$, y como $X(π) = 0$, $b = 0$ por lo que por aquí no llegamos a nada.

Si $λ > 0$, $λ = μ^2$, la solución será de la forma $X(x) = a e^{μx} + be^{-μx}$. Si $X(0) = 0$, entonces $a = -b$; y si $X(π) = 0$, entonces $0 = a(e^{μπ} - e^{-μπ})$, así que o bien $a = 0$ o bien $μ = 0$. Lo segundo es imposible, lo primero nos da una ecuación trivial.

Si $λ < 0$, $λ = -μ^2$, la solución será de la forma $X(x) = a \cos μx + b \sin μ x$. Si $X(0) = 0$, entonces $a = 0$. Si $X(π) = 0$, entonces $0 = b \sin μπ$ y por lo tanto $μ = k ∈ ℤ$.

Resolvemos ahora para $T$ sabiendo que $λ = λ_k = - k^2$, con $k ∈ ℤ$. Tenemos que \[ \frac{T''}{T} = -k^2 \], vamos a tener igualmente una solución de la forma $T(t) = a \cos k t + b \sin k t$. Si $X(x) T(0) = \sin^3 x$, entonces $T(0) = 1$ y $a = 1$. Si $u_t (x,0) = \sin 2x$, entonces $T'(0) = 1$. Derivando, \[ T'(t) = -a \sin k t + b \cos k t\], luego $b = 1$ y $T(t) = \sin k t + \cos k t $.

Para ajustar esto, tenemos que escribir $\sin^3 x$ y $\sin 2x$ como serie de cosenos. Usamos el truco de Azorero y tomamos la extensión impar de $f$, de tal forma que podrmeos calcular \[ b_k = \frac{2}{π} \int_0^π f(x) \sin kx \dif x \] siguiendo las ecuaciones de \eqref{eq:CoefsFourier}. Vamos a ello:
\begin{align*}
b_k &= \frac{2}{π} \int_0^π \underbracket{\sin^3 x}_{u} \underbracket{\sin k x \dif x}_{\dif v}
	= \frac{2}{π} \left(-k \sin^3 x \cos x + \int_{0}^π 3k \cos kx \sin^2 \cos x \right|_{x = 0}^π
\end{align*}

No me apetece seguir. Hágalo usted mismo.

\end{problem}

\begin{problem}[11] Pruébese que \[ \lim_{n\to ∞} \int_0^π \log x \sin nx \dif x = 0\]

\solution

\end{problem}
