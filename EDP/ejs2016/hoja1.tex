% -*- root: ../EDP2016.tex -*-
\section{Hoja 1}


\begin{problem}[1]

	\ppart Utilizar el principio de Duhamel para resolver el problema
	\[ u_t + v u_x = f(x,t) \]
	con dato inicial $u(x,0) = 0$.

	Encontrar una solución explícita cuando $f(x,t) = e^{-t} \sin x$.

	\ppart Resolver el modelo de evolución de la contaminación en un río, incluyendo un término de depuración natural proporcional a la concentración que se rige por la ecuación:
	\[ u_t + vu_x = -\gamma u \]
	con condiciones
	\[ u(x,0) = 0 \text{ si } x > 0, u(0,t) = \beta \text{ si } t > 0 \]

	\solution

	\spart

	\doneby{Guille}

	El \nref{sec:PrincipioDuhamel} nos dice que la solución al sistema viene dada por la siguiente ecuación: \[ u(x,t) = F(x-vt)+ \int^{t}_{0} f(x-v(t-\tau),\tau) \dif \tau \]

	En concreto, como $u(x,0) = F(x) = 0$, eso que nos quitamos. Para encontrar la solución explícita, integramos:
	\begin{gather*}
	\int_0^t f(x-v(t-τ), τ) \dif τ =
		\int_0^t e^τ \sin (x - v(t-τ)) \dif τ = \\
	= \int_0^t e^τ \frac{e^{i(x-v(t-τ))} - e^{-i(x-v(t-τ))}}{2i} \dif τ = \\
	= \int_0^t \frac{e^{τ + i(x-v(t-τ))} - e^{τ - i(x-v(t-τ))}}{2i} \dif τ = \\
	= \frac{1}{2i} \left(\frac{e^{τ + i(x-v(t-τ))}}{1 + iv} - \frac{ e^{τ - i(x-v(t-τ))}}{1 - iv}\right|_{τ = 0}^t = \\
	= \frac{1}{2i} \left(e^τ\frac{(1-iv) e^{i(x-v(t-τ))} - (1+iv) e^{- i(x-v(t-τ))}}{1 + v^2}\right|_{τ=0}^t = \\
	= \left(e^τ \frac{e^{i(x-v(t-τ))} - e^{- i(x-v(t-τ))}}{(1+v^2)2i} - ve^τ\frac{e^{i(x-v(t-τ))} + e^{- i(x-v(t-τ))}}{(1+v^2)2} \right|_{τ = 0}^t = \\
	= \left(\frac{e^τ}{1+v^2} \left(\sin (x-v(t-τ)) - v\cos (x-v(t-τ))\right) \right|_{τ = 0}^t = \\
	= \frac{e^t}{1+v^2}(\sin x - v\cos x) - \frac{1}{1+v^2}(\sin  (x-vt) - v\cos (x-vt))
	\end{gather*}

	\noteby{Guille}{Siento usar notación compleja pero estoy demasiado cansado como para hacer integrales de mierrrrrda. Tampoco estoy seguro de que esté bien.}

	\spart

	Resuelto en la \fref{sec:CurvaDatoRara}.

\end{problem}

\begin{problem}[2] Resolver la ecuación de Burgers $u_t + uu_x = 0$ con dato inicial:
	\[ g(x) =
	\begin{cases}
		1 & x \leq 0 \\
		1-x & 0 < x < 1 \\
		0 & x \geq 1
	\end{cases} \]

	\solution

	\begin{figure}[hbtp]
	\centering
	\inputtikz{Hoja1_E2}
	\caption{Características del problema 2.}
	\label{fig:Hoja1:E2}
	\end{figure}

	\textit{Nota: Resuelto por Kike.}

	En una ecuación de Burgers, las características siguen siendo $x-F(x_0) t = x_0$. Lo que parece que ocurre es lo de la \fref{fig:Hoja1:E2}, que todas las características dan a intersecar en $(x,t) = (1,1)$.

	Primero vemos qué ocurre en la región naranja, en la que las rectas son \begin{align*}
	x - (1 - x_0) t  &= x_0 \\
	t &= \frac{x - x_0}{1-x_0}
	\end{align*}

	En $(x,t)$, $u$ tendrá el valor dado por la característica que pase por ese punto. Simplemente despejando, tendremos que \[ x_0 = \frac{x-t}{1-t} \], por lo que en esa región tendremos \[ u(x,t) = g\left(\frac{x-t}{1-t}\right) \]

	Nos falta ver la onda de choque a partir del punto $(1,1)$. Obtenemos la ecuación según Rankine-Hugoniot \eqref{eq:DerivadaOndaChoque}, recordando que el flujo es $q(u) = \frac{u^2}{2}$, y que el salto será entre las zonas verde y roja. Tendremos que \begin{align*}
	s'(t) &= \frac{\frac{1}{2}}{1} = \frac{1}{2} \\
	s(t) &= \frac{1}{2}t + c
	\end{align*}

	Como $s(1) = 1$, tiene que ser $c = \frac{1}{2}$. Nos queda una pendiente de $2$ y tenemos el dibujo de la \fref{fig:Hoja1:E2} con las soluciones dadas por las características.

\end{problem}

\begin{problem}[3] Resolver la ecuación de Burgers $u_t + uu_x = 0$ con dato inicial:
	\[ g(x) =
	\begin{cases}
		0 & x < 0 \\
		1-x & 0 < x < 1 \\
		0 & x > 1
	\end{cases} \]

	\solution

	\doneby{Guille}

	\textit{Nota: El enunciado está mal copiado con respecto a lo que hay en las hojas, pero lo dejo así por pereza de cambiarlo.}

	Estamos en un caso parecido al del problema anterior (sólo cambia el dato inicial), aunque esta vez vamos a tener un pequeño problema con una zona sin definir bien, la que sale de $x = 0$. En esa región intermedia tendremos que usar el truco de las aproximaciones $F_ε$, de tal forma que tendremos características $x - k_ε t = x_ε$ y por lo tanto \[ k_ε = \frac{x - x_ε}{t} \convs[][ε][0] \frac{x}{t} \]

	Por lo tanto, en la región intermedia tenemos que el valor es \[ u(x,t) = \frac{x}{t} \]

	Esto ya nos permite calcular la ecuación la onda de choque con Rankine-Hugoniot. El flujo asociado es $q(u) = \frac{u^2}{2}$ de nuevo, así que \begin{align*}
	s'(t) &= \frac{\frac{s^2}{2t^2} - 0}{\frac{s}{t} - 0} = \frac{s}{2t} \\
	s(t) &= 2t· s'(t) \\
	s(t) &= c \sqrt{t}
	\end{align*}

	Como la onda de choque tiene que empezar en $s(t) = 1$, $c = 1$ y nuestra ecuación es $s(t) = \sqrt{t}$. El dibujo de características quedaría como en la \fref{fig:Hoja1:E3}.

	\begin{figure}[hbtp]
	\centering
	\inputtikz{Hoja1_E3}
	\caption{Características del problema 3.}
	\label{fig:Hoja1:E3}
	\end{figure}

\end{problem}

\begin{problem}[4] Un modelo bastante realista del tráfico en un túnel muy largo es el siguiente:
	$$v(\rho) =
	\begin{cases}
		v_m & 0 \leq \rho \leq \rho_c \\
		\lambda \log (\frac{\rho_m}{\rho}) & \rho_c \leq \rho \leq \rho_m
	\end{cases}$$

	donde $\rho(x,t)$ es la densidad del tráfico y $v(\rho)$ la velocidad de los coches, $v_m$ es la velocidad máxima, $\rho_m$ es la densidad máxima, y $\lambda = v_m /(\log (\rho_m / \rho_c))$.

	Observar que $v$ es una función continua, incluyendo el punto $\rho_c = \rho_m e^{-v_m / \lambda}$, y que $\rho_c$ representa una densidad crítica, pues si $\rho \leq \rho_c$ los coches pueden alcanzar una velocidad máxima $v_m$. Supongamos que el dato inicial del problema es $\rho = \rho_m$, $x<0$, y $\rho = 0$, $x > 0$ (pensar en un semáforo a la entrada del túnel, que se abre en el instante $t = 0$).

	\ppart Determinar densidad y velocidad como función del tiempo.
	\ppart Dibujar en el plano $x - t$ la trayectoria de un coche que en el instante inicial está en el punto $x_0 < 0$, y calcular el tiempo que tarda en entrar en el túnel.

	\solution

	\spart

	Aplicando los \href{https://moodle.uam.es/pluginfile.php/910610/mod_resource/content/1/Problema4.pdf}{comentarios de Azorero}, tenemos que el flujo es $q(ρ) = v(ρ) · ρ$. Nuestro sistema será
	\[
	\begin{cases}
		ρ_t + [q(ρ)]_x = 0 \\
		ρ(x,0) = F(x)
	\end{cases}\] con \[ F(x) = \begin{cases} ρ_m & x < 0 \\ 0 & x ≥ 0 \end{cases} \]

	Podemos calcular la derivada del flujo. El caso peculiar es cuando $ρ_c ≤ ρ ≤ ρ_m$, así que operamos \[ [q(ρ)]_x = [ρ λ \left(\log ρ_m - \log ρ\right)]_x = λ(ρ_x \log ρ_m - ρ · \frac{ρ_x}{ρ} - ρ_x \log ρ) \] de tal forma que la derivada final es \[ [q(ρ)]_x = \begin{cases}
	v_m ρ_x & 0 ≤ ρ ≤ ρ_c \\
	λρ_x\left(\log \frac{ρ_m}{ρ} - 1\right) & ρ_c ≤ ρ ≤ ρ_m \end{cases}
	\]

	Tratamos de resolverlo por características, así que vamos a tratar de ver qué ocurre cuando $ρ(x,t) = k$ constante. Derivando, $ρ_xx' + ρ_t = 0$, y viendo el sistema anterior tiene que ser $x' = v_m$ cuando $0 ≤ ρ ≤ ρ_c$ y $x' = λ(\log \frac{ρ_m}{ρ} - 1)$ cuando $ρ_c ≤ ρ ≤ ρ_m$. Por lo tanto, las rectas características estarán dadas por \begin{align*}
	x - v_m t = x_0 & \quad x_0 > 0 \\
	x + tλ = x_0 & \quad x_0 < 0
	\end{align*}

	Hay que tener en cuenta que $λ > 0$ (sólo hay que hacer las cuentas) así que tenemos rectas características que divergen. Tenemos entonces una zona en la que no sabemos qué pasa.

	Para eso aplicamos el método de las aproximaciones por $F_ε$ continuas para quitarnos el salto en $0$. Lo que nos sale es lo siguiente. Las características serán \[ x - \frac{[q(k_ε)]_x}{ρ_x} t = x_ε\]

	El problema es que esta ecuación depende del valor mismo de $k_ε$, tal y como veíamos en la ecuación anterior para $[q(ρ)]_x$. Lo que vamos a hacer es sacarlo en los casos y ver cuál nos vale. Primero, cuando $0 ≤ k_ε ≤ ρ_c$, tenemos que \[
	x - v_m t = x_ε \convs[][ε][0] x - v_m t = 0
	\]

	Esta no nos sirve de mucho porque nos da la característica del otro lado y no nos permite ver cuánto vale $ρ$, así que vamos a por la otra, cuando $ρ_c ≤ k_ε ≤ ρ_m$.
	\begin{align*}
	x - λ t (\log \frac{ρ_m}{k_ε} - 1) &= x_ε \\
	\log ρ_m - \log k_ε - 1 &= \frac{x_ε - x}{λt} \\
	\log k_ε &= \log ρ_m - 1 - \frac{x_ε - x}{λt} \\
	k_ε &= ρ_m e^{- \left(1 + \frac{x_ε - x}{λt}\right)}
	\end{align*}

	Haciendo converger $ε \to 0$ nos queda que \[ ρ(x,t) = \frac{ρ_m}{e^{1 + \frac{x}{λt}}} \]

	Creo, aunque no estoy del todo seguro, que habría que ver exactamente la zona en la que estamos trabajando, es decir, ver cuándo $ρ ≥ ρ_c$. Sustituyendo, \begin{align*}
	ρ_c &≥ \frac{ρ_m}{e^{1 + \frac{x}{λt}}} \\
	ρ_m e^{-\frac{v_m}{λ}} &≥ \frac{ρ_m}{e^{1 + \frac{x}{λt}}}\\
	e^{-\frac{v_m}{λ}} &≥ e^{-1 - \frac{x}{λt}} \\
	\frac{v_m}{λ} &≥ 1 + \frac{x}{λt} \\
	v_m &≥ λ + \frac{x}{t} \\
	v_m - λ &≥ \frac{x}{t} \\
	t &≥ \frac{x}{v_m - λ}
	\end{align*}

	Así, la zona en la que nos valdrá esta cosa será cuando se cumpla esa ecuación, y en el resto tendremos que hacer algo de magia que nos ha explicado hoy Azorero y que ya copiaré aquí en algún momento.

	\spart


\end{problem}

\begin{problem}[5]
	Resolver el problema de Cauchy $u_x + u_y = u^2, u(x,0) = 1$.

	\solution
\end{problem}

\begin{problem}[6]
	Resolver el problema de Cauchy $xu_x - yu_y = u - y, u(y^2,y)=y$. Estudiar si puede existir alguna solución definida en un entorno del origen.

	\solution
\end{problem}

\begin{problem}[7]
	Resolver la ecuación $u_x + 3y^{2/3} u_y = 2$ sujeta a la condición inicial $u(x,0) = φ(x)$.

	\solution

	Calculamos primero el determinante, y vemos que \[ \det \begin{pmatrix} a & b \\ α' & β' \end{pmatrix} = \det \begin{pmatrix} 1 & 3y^{\frac{2}{3}} \\ 1 & 0 \end{pmatrix}\], que es nulo en todos los puntos de la curva dato ($y = 0$). La única forma de que esto tenga solución es que la diferencial $\Dif Φ$ tenga rango uno. Entonces \[ \Dif Φ = \begin{pmatrix} 1 & 0 & 2 \\ 1 & 0 & φ' \end{pmatrix}\], luego la única posibilidad es que $φ' = 2$ y entonces $φ(x) = 2x + C$.

	Pasamos a resolver \begin{align*}
	x' = 1 					&\implies x(t) = t + s \\
	y' = 3y^{\frac{2}{3}} 	&\implies y(t) = t^3 \\
	z' = 2 					&\implies z(t) = 2t + 2s
	\end{align*}, y entonces la solución es \[ Φ(s,t) = (t+ s, t^3, 2t + 2s) \]
\end{problem}

\begin{problem}[8]
	Resolver la ecuación $(y + u)u_x + yu_y = x - y$ sujeta a la condición inicial $u(x,1) = 1 + x$.

	\solution
\end{problem}

\begin{problem}[9]
	Se considera la ecuación $yu_x - xu_y = 0 (y > 0)$. Para cada uno de los siguientes datos iniciales:

	\ppart $u(x, 0) = x^{2}$
	\ppart $u(x,0) = x$
	\ppart $u(x,0) = x, x > 0$

	comprobar si el problema tiene solución. Si la tiene, encontrarla; si no la tiene, explicar por qué.

	\solution

	\spart
	\spart
	\spart
\end{problem}

\begin{problem}[10]
	Resolver la ecuación $u_y + u^2 u_x = 0$ en $x > 0$ bajo la condición inicial $u(x,0) = \sqrt{x}$. ¿Cuál es el dominio de la existencia de la solución?

	\solution
\end{problem}

\begin{problem}[11]
	Resolver el problema de Cauchy $xu_x + yu_y = -u$, $ u(\cos s \sin s) = 1$, $ 0 \leq s \leq \pi$. ¿Está la solución definida en todas partes?

	\solution
\end{problem}

\begin{problem}[12]
	\ppart Encontrar una función $u = u(x,y)$ que resuelva el problema de Cauchy $(x + y^2)u_x + yu_y + (\frac{x}{y} - y) u = 1$, $u(x,1) = 0, x \in \mathbb{R}$.
	\ppart Comprobar si se cumple la condición de transversalidad.
	\ppart Dibujar las proyecciones sobre el plano $(x,y)$ de la condiciones inicial y de las características que emanan de los puntos $(2,1,0)$ y $(0,1,0)$.
	\ppart ¿Está la solución obtenida en (a) definida en el origen $(x,y) = (0,0)$? ¿Contradice esto al teorema de la existencia-unicidad?

	\solution

	\spart

	La condición de transversalidad la comprobamos en el siguiente apartado, aquí hacemos simplemente cuentas a ver qué sale. Resolvemos primero en $y$: \[ y'(t) = y(t) \implies y = Ce^t \]

	Cuando $t = 0$, $y = 1$ luego $C = 1$ y entonces $y(t) = e^t$. Podemos sustituir ahora para resolver $x$: \[ x'(t) = x(t) + e^{2t} \implies x(t) = c e^t+ e^{2t} \], y como $x(0) = s$, entonces la ecuación es $x(t) = (s-1) e^t + e^{2t}$.

	Sólo falta resolver la $z$, así que vamos a por ello: \[ z'(t) = 1 - z\left(\frac{x}{y} -y\right) = 1 - z((s-1) + e^t - e^t) = 1 - z(s-1) \]

	Resolvemos por separación de variables e integramos \[ t = \int_0^z \frac{1}{1 - λ(s-1)} \dif λ  + C = \eval{\frac{-1}{s-1} \log (1 - λ(s-1))}_{λ = 0}^z + C = \frac{-1}{s-1} \log (1 - z(s-1)) + C \]

	Como $z(s) = 0$, sustituimos y entonces $C = 0$. Ahora sólo nos queda despejar la $z$ (yuju): \begin{align*}
	t &= \frac{-1}{s-1} \log (1 - z(s-1)) \\
	e^{-(s-1)t} &= 1 - z(s-1) \\
	z(s-1) &= 1 - e^{-(s-1)t} \\
	z &= \frac{1 - e^{(1-s)t}}{s-1}
	\end{align*}

	\spart

	Comprobamos condición de transversalidad: \( \det \begin{pmatrix} (x+y^2) & y \\ 1 & 0 \end{pmatrix} = y ≠ 0 \label{eq:H1:E12:Trans} \) por ser nuestra curva dato $y = 1$.

	\spart

	El dibujo está en la \fref{fig:Hoja1:E12}.

	\begin{figure}
	\centering
	\inputtikz{Hoja1_E12}
	\caption{Dibujo de las características que salen de $(2,1,0)$ (naranja) y de $(0,1,0)$ (verde).}
	\label{fig:Hoja1:E12}
	\end{figure}

	\spart

	No están definidas en el origen, y no contradice a nada porque en $(0,0)$ no se cumpliría la condición de transversalidad de \eqref{eq:H1:E12:Trans}.

\end{problem}


