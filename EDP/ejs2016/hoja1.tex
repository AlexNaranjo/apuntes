% -*- root: ../EDP2016.tex -*-
\section{Hoja 1}


\begin{problem}[1]

	\ppart Utilizar el principio de Duhamel para resolver el problema
	\[ u_t + v u_x = f(x,t) \]
	con dato inicial $u(x,0) = 0$.

	Encontrar una solución explícita cuando $f(x,t) = e^{-t} \sin x$.

	\ppart Resolver el modelo de evolución de la contaminación en un río, incluyendo un término de depuración natural proporcional a la concentración que se rige por la ecuación:
	\[ u_t + vu_x = -\gamma u \]
	con condiciones
	\[ u(x,0) = 0 \text{ si } x > 0, u(0,t) = \beta \text{ si } t > 0 \]

	\solution

	\spart

	\doneby{Guille}

	El \nref{sec:PrincipioDuhamel} nos dice que la solución al sistema viene dada por la siguiente ecuación: \[ u(x,t) = F(x-vt)+ \int^{t}_{0} f(x-v(t-\tau),\tau) \dif \tau \]

	En concreto, como $u(x,0) = F(x) = 0$, eso que nos quitamos. Para encontrar la solución explícita, integramos:
	\begin{gather*}
	\int_0^t f(x-v(t-τ), τ) \dif τ =
		\int_0^t e^τ \sin (x - v(t-τ)) \dif τ = \\
	= \int_0^t e^τ \frac{e^{i(x-v(t-τ))} - e^{-i(x-v(t-τ))}}{2i} \dif τ = \\
	= \int_0^t \frac{e^{τ + i(x-v(t-τ))} - e^{τ - i(x-v(t-τ))}}{2i} \dif τ = \\
	= \frac{1}{2i} \left(\frac{e^{τ + i(x-v(t-τ))}}{1 + iv} - \frac{ e^{τ - i(x-v(t-τ))}}{1 - iv}\right|_{τ = 0}^t = \\
	= \frac{1}{2i} \left(e^τ\frac{(1-iv) e^{i(x-v(t-τ))} - (1+iv) e^{- i(x-v(t-τ))}}{1 + v^2}\right|_{τ=0}^t = \\
	= \left(e^τ \frac{e^{i(x-v(t-τ))} - e^{- i(x-v(t-τ))}}{(1+v^2)2i} - ve^τ\frac{e^{i(x-v(t-τ))} + e^{- i(x-v(t-τ))}}{(1+v^2)2} \right|_{τ = 0}^t = \\
	= \left(\frac{e^τ}{1+v^2} \left(\sin (x-v(t-τ)) - v\cos (x-v(t-τ))\right) \right|_{τ = 0}^t = \\
	= \frac{e^t}{1+v^2}(\sin x - v\cos x) - \frac{1}{1+v^2}(\sin  (x-vt) - v\cos (x-vt))
	\end{gather*}

	\noteby{Guille}{Siento usar notación compleja pero estoy demasiado cansado como para hacer integrales de mierrrrrda. Tampoco estoy seguro de que esté bien.}

	\spart

\end{problem}

\begin{problem}[2]

	Resolver la ecuación de Burgers $u_t + uu_x = 0$ con dato inicial:

	$$ g(x) =
	\begin{cases}
		1 & x \leq 0 \\
		1-x & 0 < x < 1 \\
		0 & x \geq 1
	\end{cases}$$

	\solution

\end{problem}

\begin{problem}[3]

	Resolver la ecuación de Burgers $u_t + uu_x = 0$ con dato inicial:

	$$ g(x) =
	\begin{cases}
		0 & x < 0 \\
		1-x & 0 < x < 1 \\
		0 & x > 1
	\end{cases}$$

	\solution

\end{problem}

\begin{problem}[4]

	Tráfico en un tunel. Un modelo bastante realista del tráfico en un tunel muy largo es el siguiente:

	$$v(\rho) =
	\begin{cases}
		v_m & 0 \leq \rho \leq \rho_c \\
		\lambda \log (\frac{\rho_m}{\rho}) & \rho_c \leq \rho \leq \rho_m
	\end{cases}$$

	donde $\rho(x,t)$ es la densidad del tráfico y $v(\rho)$ la velocidad, $v_m$ es la velocidad máxima, $\rho_m$ es la densidad máxima, y $\lambda = v_m /(\log (\rho_m / \rho_c))$.

	Observar que $v$ es una función continua, incluyendo el punto $\rho_c = \rho_m e^{-v_m / \lambda}$, y que $\rho_c$ representa una densidad crítica, pues si $\rho \leq \rho_c$ los coches pueden alcanzar una velocidad máxima $v_m$. Supongamos que el dato inicial del problema es $\rho = \rho_m$, $x<0$, y $\rho = 0$, $x > 0$ (pensar en un semáforo a la entrada del tunel, que se abre en el instante $t = 0$).

	\ppart Determinar densidad y velocidad como función del tiempo.
	\ppart Dibujar en el plano $x - t$ la trayectoria de un coche que en el instante inicial está en el punto $x_0 < 0$, y calcular el tiempo que tarda en entrar en el tunel.

	\solution

	\spart
	\spart


\end{problem}

\begin{problem}[5]
	Resolver el problema de Cauchy $u_x + u_y = u^2, u(x,0) = 1$.

	\solution
\end{problem}

\begin{problem}[6]
	Resolver el problema de Cauchy $xu_x - yu_y = u - y, u(y^2,y)=y$. Estudiar si puede existir alguna solución definida en un entorno del origen.

	\solution
\end{problem}

\begin{problem}[7]
	Resolver la ecuación $u_x + 3y^{2/3} u_y = 2$ sujeta a la condición inicial $u(x,0) = \phi(x)$.

	\solution
\end{problem}

\begin{problem}[8]
	Resolver la ecuación $(y + u)u_x + yu_y = x - y$ sujeta a la condición inicial $u(x,1) = 1 + x$.

	\solution
\end{problem}

\begin{problem}[9]
	Se considera la ecuación $yu_x - xu_y = 0 (y > 0)$. Para cada uno de los siguientes datos iniciales:

	\ppart $u(x, 0) = x^{2}$
	\ppart $u(x,0) = x$
	\ppart $u(x,0) = x, x > 0$

	comprobar si el problema tiene solución. Si la tiene, encontrarla; si no la tiene, explicar por qué.

	\solution

	\spart
	\spart
	\spart
\end{problem}

\begin{problem}[10]
	Resolver la ecuación $u_y + u^2 u_x = 0$ en $x > 0$ bajo la condición inicial $u(x,0) = \sqrt{x}$. ¿Cuál es el dominio de la existencia de la solución?

	\solution
\end{problem}

\begin{problem}[11]
	Resolver el problema de Cauchy $xu_x + yu_y = -u$, $ u(\cos s \sin s) = 1$, $ 0 \leq s \leq \pi$. ¿Está la solución definida en todas partes?

	\solution
\end{problem}

\begin{problem}[12]
	\ppart Encontrar una función $u = u(x,y)$ que resuelva el problema de Cauchy $(x + y^2)u_x + yu_y + (\frac{x}{y} - y) u = 1$, $u(x,1) = 0, x \in \mathbb{R}$.
	\ppart Comprobar si se cumple la condición de transversalidad.
	\ppart Dibujar las proyecciones sobre el plano $(x,y)$ de la condiciones inicial y de las características que emanan de los puntos $(2,1,0)$ y $(0,1,0)$.
	\ppart ¿Ésta la solución obtenida en (a) definida en el origen $(x,y) = (0,0)$? ¿Contradice esto al teorema de la existencia-unicidad?

	\solution

	\spart
	\spart
	\spart
	\spart
\end{problem}


