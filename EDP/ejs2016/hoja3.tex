% -*- root: ../EDP2016.tex -*-
\section{Hoja 3}

\begin{problem}[7] Demostrar que el problema \[
\begin{cases}
-Δ u = F & \text{en} Ω \\
u = f & \text{en }C_1 ⊂ ∂C \\
\dpd{u}{\vn} + αu = g &\text{en } C_2 = ∂Ω \setminus C_1
\end{cases}\] con $α > 0$ y $Ω ⊂ ℝ^N$ dominio regular, tiene a lo sumo una solución.

\solution

\doneby{Guille}

Procedimiento habitual: supongamos que existen dos soluciones $u_1, u_2$. Entonces $w = u_1 - u_2$ es solución del problema
\[ \begin{cases}
-Δ w = 0 & \text{en} Ω \\
w = 0 & \text{en }C_1 ⊂ ∂C \\
\dpd{w}{\vn} + αw = 0 &\text{en } C_2 = ∂Ω \setminus C_1
\end{cases}\]

Aquí querríamos aplicar el principio del máximo y ver que eso tiene que ser cero, pero no tenemos muy claro cuánto vale $w$ en $C_2$. Aquí se me ocurre tratar de aplicar la \nref{prop:IdGreen},

\end{problem}

\begin{problem}[12]

	\[\begin{cases}
		u_t - u_{xx} = 0, x \in (0,1), t > 0\\
		u(0,t) = 0 = u(1,t)\\
		u(x,0) = 4x(1-x)
	\end{cases}\]

	\[\text{¿} 0 < u(x,t) < 1 \text{?}\]

\solution

	El principio del máximo dice que la función alcanza máximo y mínimo en la frontera parabólica. Pero no nos da desigualdades estrictas, por lo que tendremos que trabajar más.

	\begin{itemize}
		\item Tomamos $g(x)$:
		\[g(x) = \begin{cases}
			4x(1-x) & \text{ si } x \in (0,1) \\
			0 & \text{ si } x \not\in (0,1)
		\end{cases}\]

		\[ \left. \begin{cases} v_t - v_{xx} = 0, x \in \real, t > 0 \\
			v(x,0) = g(x)
		\end{cases} \right| \Rightarrow \underbrace{v(x,t)}_{>0} = \int_{-\infty}^{+\infty} \underbrace{G(x-y,t)}_{>0} \underbrace{g(y)}_{\geq 0} dy = \int_0^1 G(x-y,t) 4y(1-y) dy \]

		En $t > 0$:

			\[ 0 < v(x,t) = \int_0^1  \frac{1}{\sqrt{4\pi t}} e^{-\frac{|x-y|^2}{4t}} \underbrace{4y(1-y)}_{\leq 1} \leq \int_0^1 \frac{1}{\sqrt{4\pi t}} e^{-\frac{|x-y|^2}{4t}} dy \]
			\[ < \int_{-\infty}^{\infty}  \frac{1}{\sqrt{4\pi t}} e^{-\frac{|x-y|^2}{4t}} dy = 1 \]

			Por lo que hemos llegado a $u \leq v < 1$. Nos falta hacer estricta la otra desigualdad.
	\end{itemize}

	Tomamos:

	\[W_t - W_{xx} = 0, x \in (x_0 - \epsilon, x_0 + \epsilon)\]
	\[W(x_0-\epsilon,t) = 0 = w(x_0+\epsilon, t)\]
	\[W(x,0) = \Psi(x)\]

	Realizamos una traslación y una dilatación:

	\[ \Psi = \delta \sin(\frac{x- (x_0 - \epsilon)}{2\epsilon/\pi}) \]

	Lo que nos lleva a $W = \Psi(x) e^{-xt}$.

\end{problem}
