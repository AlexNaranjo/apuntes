\chapter{Ecuaciones de primer orden}

\section{Modelo de un atasco}

	Vamos a empezar con un modelo sencillo. Observemos el fenómeno del embotellamiento fantasma, el cual tiene una ecuación bastante sencilla. Veamos como se generan esos atascos y como se disipan.

	Estudiaremos una función que nos de la densidad de coches en un punto, en el tiempo.

	$$u(x,t) = \text{densidad}$$

	Observemos un modelo muy simple en el que solo hay un carril y los coches fluyen solo en una dirección.

	(DIBUO DE LA RECTA CON FLECHA HACIA LA DERECHA)

	La variación en el número de coches de un intervalo será:
	\[ \od{}{t} \int^{x+h}_{x} u(s,t) \dif s = \int^{x+h}_{x} u_t(s,t) \dif s \]

	Esto equivale a al número de coches que entran por $x$ menos los que salen por $x + h$. Es decir, tenemos un flujo $q(u, x, t)$. Si suponemos que nos fijamos en un instante lo suficientemente pequeño en un tramo pequeño podemos asumir que depende solo de $u$. En ese caso, si $q > 0$ el flujo sería hacia la derecha.
	\[ q(u(x,t)) - q(u(x+h,t)) = -( q(u(x+h,t)) -  q(u(x,t)) ) \eqexpl{Barrow} -\int^{x+h}_{x} [q(u)]_{x}
	\]

	Por lo tanto, tendremos sistemas del tipo:
	\begin{equation*}
	\left\{
	\begin{array}{rl}
	u_t + [q(u)]_{x} =&\! 0 \\
	u(x, 0) =&\! F(x) \quad \quad \text{dato}
	\end{array}
	\right.
	\end{equation*}

	Ahora debemos probar distintas funciones $q$ y tratar de resolver el sistema.

	\subsection{Modelo básico}

		Tomando $q(u) = cu$.

		\begin{equation*}
		\left\{
		\begin{array}{l}
		u_t + cu_x = 0 \\
		u(x,0) = F(x)
		\end{array}
		\right.
		\end{equation*}


		Esta ecuación dice que el flujo es proporcional a la densidad. Implica que los coches se mueven con velocidad constante, a más velocidad más flujo y viceversa. Si en el instante inicial tenemos una función de la densidad respecto de las zonas del tramo, según pasa el tiempo la densidad se irá desplazando hacia la derecha manteniendo la misma forma.

		(DIBUJO)

		\textbf{Comprobación:}

		$$ u(x,t) = F(x-ct) $$
		$$ u_x(x,t) = F'(x-ct)$$
		$$u_t(x,t) = F'(x-ct)*(-c)$$

		$$ \{ u_t + cu_x = … = 0 \quad\quad u(x,0) = F(x) \} $$

		\subsubsection{Representación gráfica}


			$$u(x,t) = \text{Cte.}$$
			$$F(x - ct) = \text{Cte.}$$

			Conclusión: $x-ct = k$.


			Las soluciones dependiendo de k se desplazan por el plano siguiendo una recta de inclinación $c$. Siguiendo una fórmula del tipo $x-ct = k$. Con esto se puede obtener como avanzan los datos (densidades) respecto de un instante inicial.

			Esto son los conjuntos de nivel de las soluciones.

			(DIBUJO)



	\subsection{Otro ejemplo}

		Imaginemos un río con una velocidad constante, en el que a partir de un punto realizamos un vertido. El río tiene velocidad $v$ y en $x=0$ realizaremos una contaminación $\beta(t)$. En $t=0$ consideraremos que está limpio.

		Tenemos las ecuaciones:

		$$u_t + vu_x = 0$$
		$$u(x,0) = 0 \quad (x>0) $$
		$$u(0,t) = \beta(t), t>0$$


		Resolvemos el problema igual que antes, buscamos los conjuntos de nivel que serán rectas como nos ha salido antes. Vamos a buscar esas rectas.

		\textbf{Buscamos $x(t)$ tal que $u(x(t),t)$ sea constante}

		Derivando en t: $u_x x' + u_t = 0$ que junto con la ecuación $u_t + v u_x = 0$ nos da que $x' = v \Rightarrow x = x_0 + vt $.

		Las características saldrían \(x-vt = \text{Cte}(=x_0) \label{eq:rio_vcte}\).

		Esas son nuestras rectas conjuntos de nivel, que en este caso indicarían el frente de la contaminación, el límite a partir del cual el río sigue limpio.

		$$u(x,t) =
			\begin{cases}
				0                      & (x-vt) > 0 \\
				\beta(t - \frac{x}{v}) & (x-vt) < 0
			\end{cases}
		$$

		\begin{figure}[hbtp]
			\centering
			\inputtikz{ContaminacionRio}
			\caption{}.
			\label{fig:ContaminacionRio}
		\end{figure}

		El valor de $u(x,t)$ cuando $x-vt < 0$, se obtiene utilizando \ref{eq:rio_vcte} y tomando el punto de corte de la recta $x-c \cdot t=k$ que pasa por $(0,t^*)$:

		$$
		\begin{rcases}
			0 - vt^{*} = k \\
			x - vt = k
		\end{rcases}
		 \Rightarrow x-vt = -vt^{*} \iff t^* = \frac{x-vt}{-v} = t - \frac{x}{v}$$

		Por lo que
		$$u(x,t) = \beta(t^*) = \beta(t - \frac{x}{v}), \quad \text{ cuando } x - vt < 0$$

		De esto obtenemos una función de x en función del tiempo que nos permite saber cuándo una parte del río se contamina.

	\subsection{Ejemplo más avanzado}

		Supongamos que existe descomposición biológica:

		$$u_t + vu_x = -\gamma u$$
		$$u(x,0) = \text{Cte}$$
		$$u(0,t) = \beta$$

		(Cambiamos de variable)

		$$u_t + \gamma_u + vu_x = 0$$
		$$e^{\gamma t} u_t + e^{\gamma t} u + v e^{ \gamma t} u_x = 0 $$

		lo que es lo mismo

		$$(e^{\gamma t}u)_t + v (e^{\gamma t} u)_x$$

		Y definimos la función $W$:

		$$W = e^{\gamma t}u$$.


		Hay que comprobar cuál es el efecto del término de descomposición biológica ($-\gamma u$). ¿ $W(x,0)$ , $ W(0,t)$ ?. Esto queda como ejercicio al lector.


	\subsection{Modelo con aportación externa}

		$$u_t + vu_x = f(x,t)$$
		$$u(x,0) = F(x)$$

		Ya al no ver constante, la solución no va a ser constante a lo largo de las características, que siguen existiendo. En el caso $f = 0$ éstas daban la solución, pero ahora van a influir.


		\subsubsection{Solución 1}

			Cambio de variables:
			$$z = x-vt$$

			Lo cual hace más sencillo observar la variación temporal. Continuamos el cambio:
			$$x = z + vt$$
			$$W(z,t) = u(z+vt, t)$$

			Entonces queremos estudiar la variación de $W$ respecto del tiempo:

			$$W_t(z,t) = u_x(z + vt, t)v + u_t(z+vt, t) = f(z + vt, t)$$

			Integramos:
			$$W(z,t) = W(0,t) + \int^{t}_{0} f(z+v\tau, \tau) d\tau $$

			Lo que se entiende como el dato inicial más la suma de las aportaciones de esta función $f$ en el tiempo pasado. Ahora deshacemos el cambio de variable:

			$$u(z + vt, t) = u(z,0) + \int^{t}_{0} f(z+v\tau, \tau) d\tau$$
			$$z = x-vt$$
			$$u(x,t) = u(x-vt,0)+ \int^{t}_{0} f(x-v(t-\tau),\tau) d\tau$$

			(donde $u(x-vt,0) = F(x-vt)$)

			Esto es un milagro ya que tiene una fórmula explícita, que es poco común en EDOs y EDPs.

		\subsubsection{Solución 2}

			Vamos a descomponer el sistema en 2. Dos soluciones en las que cada una va a ser solución de dos partes del problema que nos interesan. Una del dato inicial y otra del resto.
			$$u = \phi + \psi$$

			Por un lado:
			$$
			\begin{rcases}
				\phi_t + v\phi_x = 0 \\
				\phi(x,0) = F(x)
			\end{rcases}
			\rightarrow \phi(x,t) = F(x-vt)
			$$

			Y por otro:
			$$\psi_t + v\psi_x = f(x,t)$$
			$$\psi(x,0) = 0$$

			Este problema con \concept{DUHAMEL} (la fuente externa se interpreta como una fuente de datos iniciales, se cambia el cero de la segunda ecuación de las anteriores) se puede interpretar como que tenemos una función que transporta el sistema.

			\begin{figure}[hbtp]
				\centering
				\inputtikz{interseccionCaracteristica}
				\caption{}
				\label{fig:interseccionCaracteristica}
			\end{figure}



			$$\xi_t + v\xi_x = 0$$
			$$\xi(x,s) = f(x,s)$$

			Continuamos:
			\(\xi(x,t) = f(x^*,s) \label{eq:modelo_aportacion_externa_1}\)
			$$
			\begin{rcases}
				x-vt = k \\
				x^* - vs = k
			\end{rcases}
			 \rightarrow x-vt = x^* - vs
			$$
			\( x-v(t-s) = x^{*} \label{eq:modelo_aportacion_externa_2} \)

			Y sustituimos \ref{eq:modelo_aportacion_externa_2} en \ref{eq:modelo_aportacion_externa_1}
			$$\xi(x,t) = f(x - v(t-s), s)$$


			Tenemos la solución correspondiente de un aporte instantáneo en el punto $t$. Por lo tanto la solución será la suma de todos los aportes instantáneos:

			$$\Rightarrow \psi(x,t) = \int^{t}_{0} f(x-v(t-s),s) \dif s $$

			Aunque hemos hecho un montón de asunciones que tendremos que comprobar en algún momento.

	\subsection{Modelo de tráfico más realista}

		No hemos tenido en cuenta que la velocidad de los coches no es totalmente proporcional a la densidad. Si ésta baja mucho, los coches llegan a pararse. Hay una densidad máxima donde el tráfico quedará totalmente estancado (los coches pegados).

		\begin{figure}[hbtp]
			\centering
			\inputtikz{parabola}
			\caption{}
			\label{fig:parabola}
		\end{figure}

		Una función más realista sería una parábola. Si no hay coches no pasa ninguno, pero a partir de un punto cuantos más coches, menos rápido van, y menos pasan.

		$$ q(u) = Au (B-u) = ABu - Au^{2} $$

		Veamos su ecuación:
		$$u_t + [q(u)]_x = 0$$
		$$ … $$
		$$ u_t + (AB - 2Au) u_x = 0 $$

		La velocidad va a ser dependiente de la densidad.

		Nos saldrán características cuya pendiente dependa del valor en el tiempo inicial. Si divergen no habrá problema, pero si convergen, habrá puntos donde rectas de densidad 2 (por ejemplo) cortarán con rectas de densidad 1. Esto se entiende como coches que van más rápido que se encuentran con coches que van más lentos y se ven obligados a frenar.


		Para simplificar los cálculos de este modelo asumamos que $A = B = 1$. Por lo tanto ahora estudiaremos:

		\begin{equation*}
		\left\{
		\begin{array}{l}
		u_t + (1-2u)u_{x} = 0 \\
		u(x, 0) = F(x) \quad \quad \text{dato}
		\end{array}
		\right.
		\end{equation*}

		Obtengamos las características (curvas de nivel para esa solución). Con la regla de la cadena tenemos:

		$$u_x x' + u_t = 0 \text{ a lo largo de la característica}$$

		$$x' = 1 - 2u \eqexpl{$u=k$} 1 - 2k$$

		$$x = x_0 + (1-2k)t$$

		Características:
		\( x - (1-2k)t = x_0\quad,\quad k=F(x_0)  \label{eq:caracteristicas_realistas} \)

		Me fijo en un punto en el instante inicial. Con el valor de ese punto calculamos la pendiente. A lo largo de las caracterísitcas el valor del dato se mantiene. En este caso las caracteristicas no son paralelas porque la pendiente es $1-2k$, es decir, depende de $k$.

		\begin{figure}[!htb]
			\centering
			\inputtikz{rectasDivergentes}
			\caption{}
			\label{fig:rectasDivergentes}
		\end{figure}

		\newpage

		Veamos algunos ejemplos:

		\begin{example}[Semáforo]

			$$F(x) =
			\begin{cases}
				1 & x < 0 \\
				0 & x > 0
			\end{cases}
			$$

			\begin{figure}[!htb]
				\centering
				\inputtikz{semaforoCerrado}
				\caption{Semáforo cerrado en $t=0$}
				\label{fig:semaforoCerrado}
			\end{figure}

			Y en $t=0$ el semáforo está cerrado, en $t>0$ abierto.

			En la sección izquierda: $ F = 1 \Rightarrow x - (1 - 2*1)t = x + t = k \Rightarrow x = k -t $, con $k$ constante.

			En la derecha: $F = 0 \Rightarrow x-t = k$.

			\begin{figure}[!htb]
				\centering
				\inputtikz{caracteristicasSemaforo}
				\caption{}
				\label{fig:caracteristicasSemaforo}
			\end{figure}

			Vemos que las características se abren y se separan desde $x=0$ hacia afuera.

			Por lo tanto las características no nos dan información en la región entre las rectas características que parten del $0$, ya que ninguna característica pasa por ese espacio.

			\begin{figure}[!htb]
				\centering
				\inputtikz{semaforoT}
				\caption{Instante $t=T$}
				\label{fig:semaforoT}
			\end{figure}

			Esto pasa porque no está definida la función en el $0$. Si lo estuviera podríamos intentar sacar infinitas características desde el 0 que rellenaran la zona en la que no tenemos información.

		\end{example}

		\begin{example}[Frenazo]

			Imaginemos ahora dos zonas, una con densidad de coches baja y otra con densidad alta (se encuentran con un atasco)

			\begin{figure}[!htb]
				\centering
				\inputtikz{modeloAtasco}
				\caption{}
				\label{fig:modeloAtasco}
			\end{figure}

			$$F(x) =
			\begin{cases}
				1/5 & x < 0 \\
				2/3 & x > 0
			\end{cases}
			$$

			Dibujamos las características con \ref{eq:caracteristicas_realistas}.

			\begin{figure}[!htb]
				\centering
				\inputtikz{ondaChoque}
				\caption{}
				\label{fig:ondaChoque}
			\end{figure}

			Vemos en este dibujo casos en los que cuando recorremos una recta característica con valor $\frac{1}{5}$ nos encontramos con otra de valor $\frac{2}{3}$. Momento en el cual se produce el frenazo.

			Fijando $T$ podemos mirar el valor de cada $x$ trazando su correspondiente característica. Ahí es donde nos encontraremos los puntos en los que las rectas cortan. La sucesión de puntos de corte en función de $t$ es la curva que nos interesa. Se corresponde a la función que dictamina como se van encendiendo las luces rojas de los coches según se van parando al llegar a un atasco.

		\end{example}

		Intentemos resolver ambos en general. Volvemos a la equación:

		$$u_t +m [q(u)]_x = 0$$

		Como tiene derivadas y nos impide trabajar con funciones con discontinuidades intentaremos ver el origen de esta hasta que no haya derivadas y no hayamos hecho hipótesis de regularidad. Esta ecuación venía de una integral:

		$$\frac{1}{h} \int^{x+h}_{x} u_t + [q(u)]_x = 0$$

		que a su vez venía de:

		$$\frac{d}{dt} \int^{x+h}_{x} u = q(u(x))-q(u(x+h))$$

		$u$ es $C^1$ a trozos con discontinuidad de salto en $x=s(t)$. Así que llegamos al problema:

		\textbf{Hallar $s(t)$}

		\begin{figure}[hbtp]
			\centering
			\inputtikz{discontinuidadAtasco}
			\caption{}
			\label{fig:discontinuidadAtasco}
		\end{figure}

		% A partir de aquí está bastante mal muy probablemente


		En el intervalo $[x_1,x_2]$:

		$$\frac{d}{dt} \left[ \int^{s(t)}_{x_1} u(x,t) dx + \int^{x^2}_{s(t)} u(x,t) dx \right] = q(u(x_1)) - q(u(x_2))$$


		$$ \text{¿ } \frac{d}{dt} \int^{s(t)}_{x_1} u(x,t) \dif x \text{ ?} $$

		Tomemos:

		$$G(z,t) = \int^z_{x_1}1 u(x,z) \dif x $$

		$$G_z = u(z,t) $$

		$$G_z = \int^{z}_{x_1} u_t(x,z) \dif x $$

		$$\int^{s(t)}_{x_1} u(x,t) dt = G(s(t),t)$$


		\textbf{Notación}

		$$[u(s(t),t)]^{-} = \lim_{x \to s(t)^{-}} u(x, s(t)) $$

		\textit{Queda para el lector hacer lo mismo con la otra integral.}

		Si realizamos lo mismo con la otra integral llegamos a esta \textbf{conclusión}:

		$$([u(s(t),t)]^{-} - [u(s(t),t)]^{+}) s'(t) +  \int^{s(t)}_{x_1} u_t(x,t) dx + \int^{x^2}_{s(t)} u_t(x,t) dx = q(u(x_1,t)) - q(u(x,t)) $$

		Ahora llevamos $x_1$ y $x_2$ tender a $s(t)$.

		Hacemos $x_1 \rightarrow s(t)^{-}$,$x_2 \rightarrow s(t)^{+}$.


		\textbf{Resultado:}

		$$(u(s(t),t)^{-} - u(s(t),t)^{+}) s'(t) = q(u(s(t),t)^{-}) - q(u(s(t),t)^{+})$$

		$$s'(t) = \frac{q(u(s(t),t)^{-}) - q(u(s(t),t)^{+})}{u(s(t),t)^{-} - u(s(t),t)^{+}} $$

		Lo cual resulta ser nuestra equación del shock.

		(FALTA NOMBRE TEOREMA EXTRAÑO)

		(DIBUJO DE LA ONDA DE SHOCK OTRA VEZ)

		Para obtener la curva calculamos:

		Salto en u: $2/3 - 1/5 = 7/15$

		Salto en q: $2/3(1-2/3) - 1/5(1-1/5)$

		$$q(u) = u (1-u)$$ = 2/9 - 4/25 = 14/225


		Entonces:

		$$s'(t) = \frac{\text{Salto q}}{\text{Salto u}} = \frac{\frac{14}{225}}{\frac{7}{15}} = \frac{14.15}{7.225} \equiv \alpha \quad(>0)$$

		$$s(0) = 0 \Rightarrow s(t) = \alpha t $$

		Por lo que la onda de choque sigue $x = s(t) = \alpha t$.

		(DIBUJO RECTA)


		(EMPIEZA CLASE IGUAL REPITO ALGO)


		\subsection{Soluciones generalizadas}

			\begin{itemize}

				\item $ u \in C^1 \text{ en } R^{-} \text{ y en } R^{+}. $
				\item $u$ tiene una discontinuidad de salto a lo largo de la surva $x = s(t)$.
				\item La curva satisvace la condición de Rankine-Hugonist: $s' = \frac{\text{Salto q}}{\text{Salto u}}$.

			\end{itemize}

			\begin{example}[pendiente de resolución]

				$$u_t + (1-2u)u_x = 0$$
				$$u(x,0) = F(x)$$

				Las características son las curvas $(x(t),t)$ tales que $u(x(t),t) = k$.

				Las calculamos:

				$u_x x' + u_t = 0 $
				$x' = 1-2u \eqexpl{$u=k$} 1-2k $
				$x = (1-2k)t + x_0$
				$x - (1-2k)t = x_0$. Siendo $x_0$ el punto de partida y $k$ el valor que se propaga.

				(DIBUJOS DE F con x y t con x con interrogaciones en el medio)

				Para calcular la región intermedia usaremos esta \textbf{idea}: Una aproximación y paso al límite. Imaginemos que $F_\epsilon$ continua y decreciente.

				\begin{figure}[hbtp]
					\centering
					\inputtikz{FEpsilon}
					\caption{}
					\label{fig:FEpsilon}
				\end{figure}

				Entonces en la región intermedia:

				$$u_\epsilon (x,t) = F_\epsilon (x_\epsilon) \equiv K_\epsilon $$

				Característica: $x - (1-2k_\epsilon) t = x_\epsilon$, con $F_\epsilon(x_\epsilon) = k_\epsilon$

				Y ahora queremos hallar $\lim_{\epsilon \rightarrow 0} u_\epsilon(x,t).$

				Por ser $x_\epsilon \in (0,\epsilon), \lim_{\epsilon \rightarrow 0} x_\epsilon = 0$.

				$$ \frac{x - x_\epsilon}{t} = 1-2_\epsilon$$
				$$k_\epsilon = \frac{1}{2} (1 - \frac{x-x_\epsilon}{t}) \lim_{\epsilon \rightarrow 0} = \frac{1}{2} (1 - \frac{x}{t})$$ % arreglar este limite!!

				Con lo que llegamos a una formula para la densidad en la región intermedia:

				$$u(x,t) = \frac{1}{2} (1 - \frac{x}{t}) \text{ si } (x,t) \text{ está en la región intermedia. }$$

				\begin{figure}[hbtp]
					\centering
					\inputtikz{SolucionTFijo}
					\caption{}
					\label{fig:SolucionTFijo}
				\end{figure}

				$$u(x,T) =
				\begin{cases}
					1 & x \leq -T \\
					\frac{1}{2} (1 - \frac{x}{T}) & -T < x < T \\
					0 & x \geq T
				\end{cases}
				$$

				A esto se le llama la \textbf{Onda de rarefacción}.

			\end{example}

	\subsection{Cambio de variables. Ecuación de Burgers}

		Dados:

		$$ u_t + [q(u)]_x = 0 \quad (q \in C^2) $$

		$$u_t + V(u)u_x = 0 $$

		Realizamos un cambio de variables:

		$$W = V(u)$$
		$$W_t = V'(u) u_t$$
		$$W_x = V'(u) * u_x$$

		Y vamos a ver que función cumple $W$. Lo que hacemos es pasar a escribir la ecuaciín en terminos de una variable que es la "velocidad" (CUIDADO CON LA INTUICIÓN), al contrario que antes, que era de la densidad. Esta velocidad es la velocidad de $u$. No tenemos que entenderla como la velocidad de los coches.

		Imaginemos un mapa de tráfico en el que se ve la densidad del tráfico con colores. La $W$ es la velocidad a las que esas manchas de color se desplazan, no los coches en si. Están relacionadas pero no son lo mismo.

		Completemos el sistema anterior con $W$:

		$$u_t = v(u) u_x = 0\quad\quad (u \equiv \text{densidad})$$

		$$v'(u) u_t + v(u) * v'(u) u_x = 0$$

		(INDICAR SECCIONES DE LA FORMULA BIEN)

		\( W_t = WW_x = 0 \label{eq:Ecuación de Burgers} \)

		(en este caso es homogénea al ser $0$ el término independiente)

		(ES POSIBLE QUE FALTE ALGO AQUÍ)


		Expresemos ahora $q$ en función de la ecuación de Burgers.

		\subsubsection{Flujo asociado a la equación de Burgers}

			$$q_{B} = \frac{W^2}{2}$$

			$$W_t + WW_x = W_t + [q_B(w)]_x = 0$$


		\subsubsection{Características en la ec. de Burgers}

			$$W(x,t),t) = k$$

			$W_t + W_x x' = 0 \Rightarrow x' = w = k$

			Por lo que las características serán:

			$$x-kt = x_0 \text{ , con }F(x_0) = k$$

			\begin{figure}[hbtp]
				\centering
				\inputtikz{CaracteristicasBurger}
				\caption{}
				\label{fig:CaracteristicasBurger}
			\end{figure}

			Con $W(x,0) = F(x)$.

		\begin{example}{Zona central de atasco}

			Dada la ecuación de Burgers:

			$$u_t + u u_x = 0$$

			Dato inicial:

			$u(x,0) = F(x)$ con $F(x)$ definida como:

			\begin{figure}[hbtp]
				\centering
				\inputtikz{FTresVelocidades}
				\caption{}
				\label{fig:FTresVelocidades}
			\end{figure}


			Vemos que los que vengan de la izquierda con velocidad 2 se encontrarán con la zona de velocidad 0 en la que tendrán que parar. Con lo que surgirá una onda de choque entre esas dos secciones. A su vez, entre la parte de velocidad 0 y la parte de velocidad 1 surgirá una onda de rarefacción ya que la parte con más velocidad "huirá" de la anterior. Esto ocurre para tiempos pequeños pero habrá que ver como evoluciona el sistema.

			Las características son:

			$$x - kt = x_0 \text{ con } F(x_0) = k$$.

			Va a ocurrir en algún punto que la onda de choque llegue hasta la zona delimitada por la onda de rarefacción. La onda de choque depende de los valores a ambos lados de esta. Cuando ocurra este suceso la onda de choque estará definida por otra ecuación. Y si en algúm momento llega a confluir con la primera caracterísitca de la sección de velocidad 1 obtendrá otra forma.


			Empecemos por \textbf{tiempos pequeños}:

			Onda de choque que parte de $x=0$.

			$$R-H \Rightarrow s' = \frac{\text{salto q}}{\text{salto u}} = \frac{\text{salto }\frac{u^2}{2}}{\text{salto }u} = \frac{\frac{4}{2} - 0}{2-0} = 1 $$

			Esa será la derivada de nuestra onda de choque, que al ser constante es la pendiente. La onda será:

			$$s(t) = t$$

			Nos queda entonces calcular la rarefación en $x=1$:

			$$x-k_\epsilon t = x_\epsilon$$

			$$u_\epsilon (x,t) = k_\epsilon = F_\epsilon (x_\epsilon)$$

			$$x_\epsilon \rightarrow_{\epsilon \rightarrow 0} 1 \Rightarrow x - kt = 1 \Rightarrow $$

			Con lo que obtenemos la onda de rarefación desde $x=1$:

			$$\Rightarrow 	\frac{x-1}{t} = k = u(x,t)$$

			Eso nos deja un modelo así:

			\begin{figure}[hbtp]
				\centering
				\inputtikz{ModeloTresVelocidades}
				\caption{}
				\label{fig:ModeloTresVelocidades}
			\end{figure}

		\end{example}

















