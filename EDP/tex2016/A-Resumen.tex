% -*- root: ../EDP2016.tex -*-
\chapter{Resumen rápido}

\section{Ecuaciones de primer orden}

En general, para las ecuaciones de primer orden trabajaremos sobre un sistema \[ \begin{cases}
u_t + [q(u)]_x = f(x,t) \\
u(x,0) = F(x)
\end{cases}\], donde $f$ es la aportación externa, $q$ el flujo (que depende de la densidad) y $F(x)$ el dato inicial.

\subsection{Flujo lineal. Principio de Duhamel}

Para entender las soluciones trabajaremos con las curvas características, curvas en el espacio $x,t$ a lo largo de las cuales la solución se mantiene constante. Con un flujo lineal y sin aportación externa ($q(u) = cu$, $f \equiv 0$) las características son rectas de la forma $x - ct = x_0$, con $x_0$ el punto inicial (en $t = 0$) de esa recta.

Cuando la aportación externa no es nula, el enfoque es separar la solución en una parte que depende del dato incial y otra que depende de la aportación externa. El desarrollo está en la \fref{sec:ModeloGeneral}, pero la parte que nos interesa es la ecuación \eqref{eq:ModeloCombinado}, a la que se llega igualmente con el \nref{sec:PrincipioDuhamel} \[ u(x,t) = F(x-vt)+ \int^{t}_{0} f(x-v(t-\tau),\tau) \dif \tau \]

\subsection{Flujo más complicado}

Si el flujo se hace más complicado, como en la \fref{sec:ModeloTraficoRealista} que lo hacemos parabólico, la cosa se complica aunque no demasiado. Esta vez las curvas características serán de la forma \[ x - (1 - 2F(x_0)) t = x_0 \], donde $x_0$ es el punto inicial de la recta. Lo interesante es que la pendiente depende del valor inicial, y eso nos llevará a apariciones de ondas de choque o rarefacción cuando tengamos discontinuidades en el dato inicial.

Esa onda tendrá una ecuación $s(t)$ que se puede resolver a partir de su derivada \eqref{eq:DerivadaOndaChoque}: \[ s'(t) = \frac{q(u(s(t),t)^{-}) - q(u(s(t),t)^{+})}{u(s(t),t)^{-} - u(s(t),t)^{+}} \], que aunque parece muy liosa en realidad es simplemente el cociente entre el salto del flujo y el de la densidad cuando nos acercamos a la discontinuidad.

En el caso en el que tengamos no sólo ondas de choque sino también zonas en las que no hay características, podemos suponer la existencia de funciones $F_ε$ continuas que aproximan la función dato discontinua. Sacando las caraterísticas de ese dato y pasando al límite $ε \to 0$ podremos sacar las características en esa zona. Un ejemplo es el resultado de \eqref{eq:SolucionRarefaccion} para el modelo del semáforo.

\subsection{Ecuación de Burgers}

Otro método para resolver sistemas con flujo no lineal es el de la \textbf{Ecuación de Burgers}, desarrollado en la \fref{sec:EcuacionBurgers}. Consideramos que la derivada del flujo es $[q(u)]_x = V(u)u_x$, y entonces se realiza el cambio de variables $W = V(u)$. Esto nos permite entonces pasar de ecuaciones de la forma \[ W_t + WW_x = 0 \] a ecuaciones con un flujo $q_B = \frac{W^2}{2}$ asociado que sabemos cómo resolver con el método de las características. Las caracterstícas serán de nuevo $x - kt = x_0$ con $F(x_0) = k$.
