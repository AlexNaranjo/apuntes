% -*- root: ../EDP2016.tex -*-
\chapter{Resumen rápido}

\section{Ecuaciones de primer orden}

En general, para las ecuaciones de primer orden trabajaremos sobre un sistema \[ \begin{cases}
u_t + [q(u)]_x = f(x,t) \\
u(x,0) = F(x)
\end{cases}\], donde $f$ es la aportación externa, $q$ el flujo (que depende de la densidad) y $F(x)$ el dato inicial.

\subsection{Flujo lineal. Principio de Duhamel}

Para entender las soluciones trabajaremos con las curvas características, curvas en el espacio $x,t$ a lo largo de las cuales la solución se mantiene constante. Con un flujo lineal y sin aportación externa ($q(u) = cu$, $f \equiv 0$) las características son rectas de la forma $x - ct = x_0$, con $x_0$ el punto inicial (en $t = 0$) de esa recta.

Cuando la aportación externa no es nula, el enfoque es separar la solución en una parte que depende del dato incial y otra que depende de la aportación externa. El desarrollo está en la \fref{sec:ModeloGeneral}, pero la parte que nos interesa es la ecuación \eqref{eq:ModeloCombinado}, a la que se llega igualmente con el \nref{sec:PrincipioDuhamel} \[ u(x,t) = F(x-vt)+ \int^{t}_{0} f(x-v(t-\tau),\tau) \dif \tau \]

\subsection{Flujo más complicado}

Si el flujo se hace más complicado, como en la \fref{sec:ModeloTraficoRealista} que lo hacemos parabólico, la cosa se complica aunque no demasiado. Esta vez las curvas características serán de la forma \[ x - (1 - 2F(x_0)) t = x_0 \], donde $x_0$ es el punto inicial de la recta. Lo interesante es que la pendiente depende del valor inicial, y eso nos llevará a apariciones de ondas de choque o rarefacción cuando tengamos discontinuidades en el dato inicial.

Esa onda tendrá una ecuación $s(t)$ que se puede resolver a partir de su derivada \eqref{eq:DerivadaOndaChoque}: \[ s'(t) = \frac{q(u(s(t),t)^{-}) - q(u(s(t),t)^{+})}{u(s(t),t)^{-} - u(s(t),t)^{+}} \], que aunque parece muy liosa en realidad es simplemente el cociente entre el salto del flujo y el de la densidad cuando nos acercamos a la discontinuidad.

En el caso en el que tengamos no sólo ondas de choque sino también zonas en las que no hay características, podemos suponer la existencia de funciones $F_ε$ continuas que aproximan la función dato discontinua. Sacando las caraterísticas de ese dato y pasando al límite $ε \to 0$ podremos sacar las características en esa zona. Un ejemplo es el resultado de \eqref{eq:SolucionRarefaccion} para el modelo del semáforo.

\subsection{Ecuación de Burgers}

Otro método para resolver sistemas con flujo no lineal es el de la \textbf{Ecuación de Burgers}, desarrollado en la \fref{sec:EcuacionBurgers}. Consideramos que la derivada del flujo es $[q(u)]_x = V(u)u_x$, y entonces se realiza el cambio de variables $W = V(u)$. Esto nos permite entonces pasar de ecuaciones de la forma \[ W_t + WW_x = 0 \] a ecuaciones con un flujo $q_B = \frac{W^2}{2}$ asociado que sabemos cómo resolver con el método de las características. Las caracterstícas serán de nuevo $x - kt = x_0$ con $F(x_0) = k$.

\subsection{Caso general: Problema de Cauchy}

El desarrollo completo está en la \fref{sec:ProblemaCauchy}, pero la parte que nos interesa se resume rápido. Partimos de una ecuación \[ a(x,y,u) u_x + b(x,y,u) u_y = c(x,y,u) \], y tenemos un dato dado por una curva \[ Γ \equiv (α(s), β(s), γ(s)) \]

Si se cumple la condición de transversalidad \eqref{eq:CondTransversalidad}, esto es, que la siguiente matriz tenga determinante no nulo \[ \left|\begin{matrix} a & b \\ α' & β' \end{matrix}\right| ≠ 0 \],
entonces (\fref{thm:Transversalidad}) tenemos una solución única en un entorno local de cada punto que se puede construir con las curvas características. Estas curvas $(x,y,z)$ se podrán obtener resolviendo el siguiente sistema de EDOs: \begin{align*}
 x'(t) &= a(x(t), y(t), z(t)) \\
 y'(t) &= b(x(t), y(t), z(t)) \\
 z'(t) &= c(x(t), y(t), z(t)) \\
 (x(0), y(0), z(0)) &= (α(s), β(s), γ(s))
\end{align*}

Si la matriz tiene determinante nulo, la única posibilidad para tener una solución $C^1$ es que el rango de la matriz $\Dif Φ$ sea $1$ (ver \fref{sec:CondTransversalidadInvalida} para más detalles), en cuyo caso tendremos que hacer lo mismo que antes para encontrar la solución.

Si tenemos la mala suerte de que nos sale que toda la curva dato es característica, no hay solución y el problema está mal planteado.

\section{Ecuaciones de segundo orden - Series de Fourier}

La primera solución de este tema es para el problema de la ecuación de onda
\[  \begin{cases}
	u_{tt} - c^2 u_{xx} = 0  & x  \in (-L,L), t>0\\
	u(-L,t) = u(L,t) &\forall t > 0\\
	u_x(-L,t) = u_x(L,t) & \forall t > 0 \\
	u(x,0) = f(x) \\
	u_t(x,0) = g(x)
	\end{cases}
\]
que se resuelve con la fórmula de D'Alambert, que tiene la expresión \[ u(x,t) = \frac{1}{2} \left[ f(x+ct) + f(x-ct) \right] + \frac{1}{2} \int^{x+ct}_{x-ct} g(s) \dif s \]

Hay otra forma, que es la resolución con separación de variables y Fourier. En estos problemas, buscaremos una solución $u(x,t) = X(x) T(t)$, y haciendo las cuentas nos saldrá algo como \[ \frac{T'}{T} = \frac{X''}{X} = λ ∈ ℝ \], igual con derivadas de segundo orden también en $T$.

Un ejemplo para el problema que poníamos arriba. Sustituimos $u = X(x) T(t)$ en $u_{tt} - u_{xx} = 0$ y tenemos que \begin{gather*}
X(x)T''(t) -  X''(x) T(t) = 0 \\
\frac{T''}{T} = \frac{X''}{X} = λ ∈ ℝ
\end{gather*}, donde el cociente tiene que ser un número fijo $λ ∈ ℝ$ ya que ambas funciones son iguales a pesar de depender de distintas variables.

La EDO para $X$ es $X'' = λX$, y las condiciones de contorno $X(-L) = X(L)$ y $X'(-L) = X'(L)$ se sacan de las condiciones del problema original. Damos valores a λ ($λ = 0$, $λ = μ^2 > 0$ y $λ = -μ^2 < 0$) y vemos que sólo en el último caso tenemos una solución no trivial: $X(x) = a \cos μx + b \sin μx$, con $μ = \frac{kπ}{L}$, siendo $2L$ la longitud del intervalo. Así, la solución general de $X$ será la suma de las infinitas soluciones \[ X_k(x) = a_k \cos \left( \frac{k\pi}{L} x \right) + b_k \sin \left( \frac{k\pi}{L} x \right) \]

Si siguiésemos haciendo cuentas\footnote{Esto es un resumen, leer los apuntes para más detalles.} el problema que tenemos al final es que nos va a quedar la $X$ (y a veces también $T$) como un desarrollo en serie de senos o cosenos con unos coeficientes, que para sacarlos tendremos que cuadrarlos con los datos iniciales. Esto nos lleva a tener que expresar $f$ y $g$ como series de senos y cosenos, lo que se llama la serie de Fourier. Dada $f$ periódica en $(-L, L)$, entonces su serie de Fourier es
\begin{align*}
f(x) &= \frac{a_0}{2} + \sum_{k = 1}^∞ \left[a_k \cos \left(\frac{kπ}{L}x\right) + b_k \sin \left(\frac{kπ}{L}x\right) \right] \\
a_k &= \frac{1}{L} \int_{-L}^L f(x) \cos \left(\frac{kπ}{L}x\right) \dif x \\
b_k &= \frac{1}{L} \int_{-L}^L f(x) \sin \left(\frac{kπ}{L}x\right) \dif x
\end{align*}

Si queremos que la serie salga sólo en senos o sólo en cosenos hay que jugar con las extensiones para que se anulen los $b_k$ o los $a_k$ respectivamente.

Hay otra posibilidad para expresar los coeficientes, que es dando una base del \nlref{def:EspacioL2} y usar su producto escalar \eqref{eq:ProdEscalarComplejo}, respectivamente: \(
\left\{ \frac{1}{\sqrt{2L}} e^{i \left( \frac{k \pi}{L} x \right)} \right\}_{k = 0, ±1, ±2, …}
\qquad\qquad
\pesc{f,g}_{L^2} = \int^L_{-L} f(x) \conj{g(x)} \dif x
\label{eq:Resumen:L2}
\)

Así, la expresión de $f$ queda como si fuese un vector normal: las coordenadas $α_k$ son el producto escalar de $f$ por cada elemento de la base, y para llegar a la función multiplicamos esas coordenadas $α_k$ por los elementos de la base:
\begin{gather*}
f(x) = \sum_{k=-\infty}^\infty \alpha_k e^{i \left( \frac{k \pi}{L} x \right)} = \sum_{k = -\infty}^\infty \pesc{ f, \frac{1}{\sqrt{2L}} e^{i \left( \frac{k \pi}{L} x \right)} } · \frac{e^{i \left( \frac{k \pi}{L} x \right)}}{\sqrt{2L}} \\
α_k = \pesc{ f, \frac{1}{\sqrt{2L}} e^{i \left( \frac{k \pi}{L} x \right)} } = \frac{1}{\sqrt{2L}} \int_{-L}^L  f(x)  e^{-i \left( \frac{k \pi}{L} x \right)} \dif x
\end{gather*}

Algo más fácil de recordar que la expresión en serie de senos y cosenos.

\subsection{Convergencia de la serie de Fourier}

Si fuésemos físicos, ingenieros o cualquier otra clase de pervertido que se contenta con unas cuentas hechas a ojo, el tema terminaría aquí. A nosotros nos interesará además saber si esa serie infinita converge a lo que queremos y cómo lo hace.

Recordamos las posibles convergencias de una serie de funciones: la \nlref{def:ConvergenciaPuntual}, que dice que cada punto converge pero que no conserva la integral; la \nlref{def:ConvergenciaUniforme}, que dice $∀ε > 0$ las $f_n$ están en una banda de anchura ε alrededor de $f$ para $n ≥ n_0$, y que preserva continuidad e integrabilidad en acotados aunque no la derivabilidad (\fref{thm:ConvergenciaUniforme}); y por último la \nlref{def:ConvergenciaL2} que nos dice que la integral de $f_n - f$ tiende a cero.

Las relaciones entre las convergencias son fáciles: las dos únicas implicaciones son que la convergencia uniforme implica puntual; y si el dominio es acotado también implica convergencia $L^2$ (\fref{fig:DiagramaConvergencias}).

La convergencia $L^2$ tiene nombre del \nlref{def:EspacioL2}, funciones cuadrado integrables. Es un espacio vectorial, con norma \eqref{eq:NormaL2} y producto escalar \eqref{eq:ProdEscalarL2}. Además, en él las funciones continuas son densas: toda función $L^2$ se puede aproximar por funciones continuas.

\subsubsection{Convergencia para $f ∈ C^2$}

Con esto se puede empezar a estudiar la convergencia de la serie de Fourier. Si partimos de $f ∈ C^2$, podemos usar el \nlref{prop:criterio_Weierstrass} (resumen: dadas cotas $M_k ≥ \abs{Φ_k(x)}$, si la suma de los $M_k$ es finita entonces la serie $\sum Φ_k$ converge uniformemente) para decir que la serie de Fourier converge uniformemente (\fref{prop:ConvergenciaUniformeFourier}), aunque no sabemos a qué.

\subsubsection{Convergencia para $f ∈ L^2$}

El siguiente paso es estudiar lo que ocurre si $f ∈ L^2$. Como decíamos antes, $L^2$ es un espacio vectorial con base y producto escalar \eqref{eq:Resumen:L2}. Ahí podemos demostrar la desigualdad de Bessel \eqref{eq:desigualdad_bessel}, que nos dice que \[ \sum_{k=-\infty}^\infty |\pesc{f, \Phi_k}|^2 \leq \|f\|_2^2 \] para $\set{Φ_k}$ base ortonormal de $L^2$ y nos da el \nref{lem:RiemannLebesgue} que nos dice que $\abs{\pesc{f, Φ_k}} \convs[][k] 0$.

La desigualdad de Bessel nos da una cota superior para las sumas parciales de la serie de Fourier, que además están acotadas. Todo eso nos dice que las sumas parciales de Fourier convergen en $L^2$, aunque seguimos sin saber a qué.

La identificación el límite requiere un lío interesante (ver \fref{sec:IdentificacionLimite}), que lleva a la definición del núcleo de Poisson que finalmente nos dará convergencia uniforme con un límite identificado en el \fref{thm:ConvUniformePoisson}.

Además, los teoremas de aproximación de Weierstrass \ref{thm:AproxWeierstrass1} y \ref{thm:AproxWeierstrass2} nos dan finalmente la convergencia en $L^2$ en el \fref{thm:ConvergenciaL2}, que nos dice que los polinomios trigonométricos que mejor aproximan una función son la serie de Fourier. Además, tendremos la identidad de Parseval (la desigualdad de Bessel con igualdad: $\sum_{k= -\infty}^\infty  |\pesc{f, \Phi_k}|^2 = \|f\|^2_2$) y la convergencia de la serie de Fourier a la función en $L^2$ (\fref{thm:ConvergenciaL2Limite}).

\subsubsection{Convergencia puntual}

Buscaremos un resultado aproximado: no hace falta uniforme pero al menos puntual (la convergencia $L^2$ ni siquiera es puntual). Para ello, se define el núcleo de Dirichlet \eqref{eq:nucleoDirichlet}, con la propiedad interesante de que la suma parcial $S_N f$ es la convolución $f * D_N = \int f(s) D_N (x - s) \dif s$.

Ese núcleo permite demostrar convergencia puntual de la serie para funciones acotadas y derivables en el punto (\fref{thm:Dirichlet1}) y convergencia al punto medio ($\lim_{n \to \infty} S_nf(x) = \frac{1}{2} \{f(x^+)+f(x^-)\}$) en funciones con discontinuidades amables (\fref{thm:Dirichlet2})

Otro resultado de convergencia puntual es que para una \nlref{def:FuncContinuaHolder} (existen $K ∈ ℝ$, $α ∈ [0,1]$ tales que $\abs{f(t) -f(s)} < K \abs{t-s}^α$ para todos $t,s$ en el dominio) hay convergencia puntual de la serie de Fourier en todo punto (\fref{thm:ConvPuntualHolder}).

\subsection{Aplicación a la resolución de EDPs}

Cosas intereasantes para la resolución de EDPs: si la derivada de la serie de Fourier converge, entonces la función es derivable y su derivada es precisamente la derivada de la serie (\fref{thm:DerivadaFourier}).

\section{Comportamiento cualitativo}

En esta sección pasamos a estudiar el comportamiento cualitativo de varias ecuaciones. Empezamos por la ecuación de onda.

\subsection{Ondas}

Partiremos del problema
\[ \begin{cases}
	u_{tt} - c^2 u_{xx} = 0, x \in (0,L) t > 0 \\
	u(0,t) = u(L,t) = 0, t > 0\\
	u(x,0) = f(x) \\
	u_t(x,0) = g(x)
\end{cases}\] con posiblemente otras condiciones de contorno.

\subsubsection{Conservación de la energía}

En ondas, la expresión de la energía \eqref{eq:Onda:Energia} $E(t) = \frac{1}{2} \int_0^L (u_t)^2 + (u_x)^2 \dif x$ nos da unicidad de las soluciones: si hay dos soluciones, entonces $u = u_1 - u_2$ resuelve la misma ecuación con funciones iniciales $f \equiv g \equiv 0$. La energía es constante y $E(0) = 0$, luego sale que $u = 0$ y ambas soluciones son la misma. La ecuación de la energía también da dependencia continua de los datos.

\subsubsection{Ondas en dominios no acotados}

Para la resolución de problemas en dominios no acotados, se usa la fórmula de D'Alambert  \eqref{eq:Onda:DAlembert}: \[ u(x,t) = \frac{1}{2} \left(f(x + ct) + f(x-ct)\right) + \frac{1}{2c} \int_{x-ct}^{x+ct} g(s) \dif s \]

La fórmula de D'Alambert nos da lo que se llama el dominio de dependencia de solución y de influencia de los datos (\fref{fig:ReprGraficaEcConda}). Además, la conservación de energía nos da también unicidad y dependencia continua de los datos en dominios no acotados.

Esto tambén se puede aplicar a casos en los que tenemos una ``barra'', por ejemplo la definición para $x > 0$. En estos casos se podrá usar la regla del paralelogramo de \eqref{eq:Onda:Paralelogramo} y \fref{fig:AplicacionParalelogramo} para sacar la solución en zonas desconocidas despejando a partir de valores conocidos.

\subsubsection{Caso no homogéneo}

En el problema no homogéneo (\fref{sec:Onda:NoHomogeneo}) podemos separar en dos sistemas  \[ \begin{cases}
	v_{tt} - c^2 v_{xx} =0  & t ∈ ℝ, \, x ∈ ℝ \\
	v(x,0) = f(x) \\
	v_t(x,0) = g(x) \end{cases} \qquad \begin{cases}
	w_{tt} - c^2 w_{xx} = F(x,t) & t ∈ ℝ, \, x ∈ ℝ \\
	w(x,0) = 0 \\
	w_t(x,0) = 0) \end{cases}
\] donde podemos resolver la parte homogénea por D'Alambert y la no homogénea por Duhamel, viendo que \[ w(x,t) = \frac{1}{2c} \int_0^t \int_{x-c(t-τ)}^{x + c(t - τ)} F(s, τ) \dif s \dif τ \]

\subsection{Ecuación de Laplace}

En esta sección se estudia lo que sale del laplaciano \eqref{eq:Laplaciano}, el operador \[ Δu = \sum_{i=1}^N u_{x_i x_i} = \tr (\Dif^2 u) = \dv (\grad u) \]

Estudiamos las funciones armónicas ($-Δ u= 0$), subarmónicas ($-Δu ≤ 0$) y superarmónicas ($- Δu ≥ 0$). Las funciones armónicas tienen propiedades bastante razonables (\fref{prop:PropsFuncArmonicas}: suma, traslación, escalado, transformación lineal son armónicas). Además, nos traen el principio del máximo y de la media.

\subsubsection{Principio del máximo débil}

El \nref{prop:PrincipioMaximoDebil} nos dice que si $Ω$ es dominio acotado y $u$ es subarmónica, $u ∈ C^2 (Ω) ∩ C(\adh{Ω})$ entonces $\max_{\adh{Ω}} u = \max_{∂Ω} u$. Es decir, que el máximo en el cierre es igual al máximo en la frontera, o que si $u$ alcanza un máximo en el interior entonces alcanza ese mismo máximo en la frontera.

Las principales consecuencias son las siguientes:
\begin{itemize}[itemsep=0pt]
\item \textbf{Comparación} ¿?
\item \textbf{Unicidad}
\item \textbf{Estimación a priori} ¿¿¿¿¿????
\end{itemize}

\subsubsection{Propiedad de la media}

Algo.
