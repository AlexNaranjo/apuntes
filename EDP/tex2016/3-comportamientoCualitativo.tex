\chapter{Comportamiento cualitativo}

% clase 15/3/2016

En este capítulo vamos a ver:

\begin{itemize}

	\item ondas (hiperbólica)
	\item calor (parabólica)
	\item laplace (elípticas)

\end{itemize}


\section{Ondas}

	Hemos visto ya el problema en dimensión 1: la cuerda vibrante.

	Hemos visto también el método de separación de variables que solo nos valdrá en un dominio acotado. Pero ¿qué podemos hacer en dominios no acotados? y ¿Hay más soluciones?.

	Después vimos al fórmula D'Alambert \ref{eq:DALEMBERT}, que si que nos sirve para dominios no acotados pero es difícil de interpretar en dominios acotados. Veamos un ejemplo de esta afirmación:

	\begin{example}

		\[\begin{cases}
			u_{tt} - u_{xx} = 0, x \in (0,L) t > 0 \\
			u(0,t) = u(L,t) = 0, t > 0\\
			u(x,0) = f(x) \\
			u_t(x,0) = 0
		\end{cases}\]

		Y tenemos la fórmula a la que llegamos:

		\[ u(x,t) = \frac{1}{2} \{f(x+t)+f(x-t)\}  \]

		Resulta que tenemos una fórmula muy útil pero ocurre que en un dominio acotado no podemos calcular $f$ cerca del borde al realizar $x+t$ o $x-t$. Tenemos que encontrar una manera de extender $f$ de manera que esté de acuerdo con el contorno. No es lo mismo que una cuerda esté sujeta y la onda se refleje de una manera de vuelta en la cuerda a que no esté sujeta.

		\begin{center}
			\begin{tikzpicture}
			\draw[-] (-1,0) -- (3,0);
			\draw[-] (0,-0.5) -- (0,2);


			\draw[dashed] (1.5,0) node [below] {$L$} -- (1.5,1.5);

			\draw[thick, blue] plot[smooth, tension=.9] coordinates{(0.7,0) (0.8,0.2) (1,0.4) (1.2,0.7) (1.5,0.9)};

			\end{tikzpicture}
		\end{center}

	\end{example}


	\subsection{Unicidad}

		\[\begin{cases}
		   u_{tt}-u_{xx} = 0, x \in (0,L), t >0\\
		   \text{contorno} \begin{cases}
		   	\text{dirichlet}\\
		   	\text{neumann}\\
		   	\text{periódicas}
		   \end{cases}\\
		   \begin{cases}
		   u(x,0) = 0 \\
		   u_t (x,0) = 0
		   \end{cases}
		\end{cases}
		 \]


		 $u_1, u_{2}$ son soluciones. $u = u_1 - u_2$ es solución + contorno + datos $\equiv$ 0. Queremos probar que $u\equiv0$


		 \[ \int_{0}^{L} u_t (u_{tt}- u_{xx}) dx = 0 \]

		 \[ \int_0^L \underbrace{u_t u_{tt}}_{\frac{1}{2}(u^2_t)_t} - u_t u_{xx} dx  \]

		 Operamos por partes:

		 \[ \int_0^L  \underbrace{u_t}_{u} \underbrace{u_{xx}}_{dv}  dx = \left. u_t u_x \right|_{x=0}^{L} - \int_0^L \underbrace{u_x u_{tx}}_{\frac{1}{2}(u_x^2)_t} dx \]

		 Y volviendo tenemos que:

		 \[  \int^{L}_{0} \frac{1}{2} (u_t^2)_t dx  \]

		 (FALTA)


		 \textbf{Conclusión}

		 \[ E(t) = \frac{1}{2} \int_{0}^L u_t^2(x,t) + u_x^2(x,t) dx \]

		 \[ E'(t) = 0 \forall t \quad \text{Conservación de energía total}\]


		 \[ E(t) = E(0) = \frac{1}{2} \int_0^L u_t^2 (x,0) + u_x^2 (x,0) dx = \frac{1}{2} \int_0^L g^2(x)+(f'(x))^2 dx \eqreason{$f=g=0$} = 0 \]

		 \[ E=0 \Rightarrow u_t = u_x = 0 \Rightarrow u \equiv \text{cte. y } u(x,0) = 0 \Rightarrow u \equiv 0\]

		 Por lo tanto hemos demostrado que las dos soluciones son iguales y por lo tanto solo hay una solución.

		 Tenemos también:

		 \[ E(t) =  \frac{1}{2}\int_0^L (u_t)^2 + (u_x)^2 dx \rightarrow E'(t) = u_t (L,t) u_x(L,t) - u_t(0,t) u_x(0,t) \]

		 Veamos que pasa dependiendo del tipo de contorno que tengamos:

		 \begin{itemize}
		 	\item Contorno Dirichlet
		 		\[ u(0,t) = u(L,t) = 0, \forall t \Rightarrow u_t (0,t) = u_t(L,t) = 0 \forall t \Rightarrow E'(t) = 0 \]

		 	\item Controno Neumann
		 		\[ u_x(0,t) = u_x(L,t) = 0, \forall t \Rightarrow  E'(t) = 0 \]

		 	\item Periódicas
		 		\[ u(0,t) = u(L,t), \forall t \Rightarrow u_t (0,t) = u_t(L,t) \forall t \quad \text{y} \quad u_x(0,t) = u_x(L,t), \forall t \Rightarrow E'(t) = 0 \]

		 	\item mixtas

		 \end{itemize}

		 Este resultado es la \concept{conservación de la energía}:

		 \[E(t) = \text{cte.} = E(0) = \frac{1}{2} \int_0^L (g(x))^2 + (f'(x))^2 dx  \]

		 \textbf{Consecuencias:}

		 	Veamos el caso Dirichlet, pero esto es válido para todos:

		 	\[ \begin{cases}
		 		u_{tt} - u_{xx} = F(x,t)\\
		 		u(0,t) = \alpha(t), u(L,t) = \beta{t} \\
		 		u(x,0) = f(x) \\
		 		u_t(x,0) = g(x)
		 	\end{cases}\]
		 	\[u_1,u_2 \in C^2 \text{ soluciones}\]

		 	En este problema no tenemos conservación, debido a $\alpha$, $\beta$ y $F(x,t)$ pero podemos transformarlo en un problema de $W$ en donde si tengamos esta propiedad:

		 	\[\begin{cases}
		 		W = u_1 - u_2 \\
		 		W_{tt} - W_{xx} = 0 \\
		 		W(0,t) = 0 = W(L,t) \\
		 		W(x,0) = 0 \\
		 		W_t(x,0)
		 	\end{cases}\]

		 	Por conservación de la energía tenemos:

		 	\[ \int_0^L (W_t)^2 + (W_x)^2 dx = 0 \Rightarrow W_t = W_x = 0 \Rightarrow W \equiv \text{cte.} \eqreason[\Rightarrow]{$W|_{t=0} = 0$} W(x,t) = 0 \forall x \forall t \Rightarrow u_1 = u_2 \]

		 	Por lo que hemos obtenido unicidad de las soluciones del problema inicial. Pero aunque sepamos que la solución es única, no sabemos si existe tal solución. Volvemos al problema inicial:

		 	\[ \begin{cases}
		 		u_{tt} - u_{xx} = F(x,t)\\
		 		u(0,t) = \alpha(t), u(L,t) = \beta{t} \\
		 		u(x,0) = f(x) \\
		 		u_t(x,0) = g(x)
		 	\end{cases}\]


		 	\[ C(x,t) \text{ tq. } C(0,t) = \alpha(t), C(L,t) = \beta{t}\]

		 	Por ejemplo: $C(x,t) = \alpha(t) + \frac{x}{L} (\beta(t)-\alpha(t))$

		 	Consideramos $v(x,t) = u(x,t) - c(x,t)$ (de modo que $v(0,t) = v(L,t) = 0$)

		 	\[\begin{cases}
		 		v_{tt} - v_{xx} = u_{tt} - c_{tt} - u_{xx} + c_{xx} = F - c_{tt} + c_{xx} = \gor{F} \\
		 		v(0,t) = v(L,t) = 0 \\
		 		v(x,0) = u(x,0) - c(x,0) = f(x) - c(x,0) = \gor{f}(x) \\
		 		u_t (x,0) = … = \gor{g}(x)
		 	\end{cases}
		 	\]

		 	Reescribimos nuestro sistema inicial con $v$ y obtenemos:

		 	\[ \begin{cases}
		 		v_{tt} - v_{xx} = \gor{F}(x,t)\\
		 		v(0,t) = 0, v(L,t) = 0 \\
		 		v(x,0) = \gor{f}(x) \\
		 		v_t(x,0) = \gor{g}(x)
		 	\end{cases}\]

		 	Tomando $v = H + W$ podemos separar nuestro problema en dos:

		 	\[ \begin{cases}
		 			H(x,t) \rightarrow
		 			\begin{cases}
				 		H_{tt} - H_{xx} = 0\\
				 		H(0,t) = H(L,t) = 0 \\
				 		H(x,0) = \gor{f}(x) \\
				 		H_t(x,0) = \gor{g}(x)
			 		\end{cases}\\
			 		W(x,t) \rightarrow
			 		\begin{cases}
				 		W_{tt} - W_{xx} = \gor{F}(x,t)\\
				 		W(0,t) = 0 = W(L,t) \\
				 		W(x,0) = 0 \\
				 		W_t(x,0) = 0
				 	\end{cases}
				 \end{cases}
			\]

			El sistema de la $H$ ya lo sabemos resolver por separación de variables, Fourier...

			El problema de la $W$ lo resolvemos aplicando Duhamel (método del impulso). Fijamos $s>0$:

			\[\begin{cases}
				\left.
				\begin{array}{l}
					\Phi_{tt} - \Phi_{xx} = 0 \\
					\Phi(0,t) = \Phi(L,t) = 0 \\
					\Phi(x,0) = 0 \\
				\end{array}
				\right| (FALTA) \\
				\Phi_t(x,0) = \gor{F}(x,s)
			\end{cases}\]


			\textbf{Notación:}


			\[ \Phi = \Phi(x,t,s)\]

			Respuesta al impulso $\gor{F}(x,s)$ que actúa en $t=0$.

			\[ \Phi(x,t-s,s)\]

			Respuesta al impulso $\gor{F} (x,s)$ que actúa en $t-s = 0$ ($t=s$)

			Aplicando Duhamel:

			\[ W(x,t) = \int_0^t \Phi(x,t-s,s) ds \]

			\[ G(x,t,z) = \int_0^z \Phi(x,t-s,s) ds\]
			\[ G_t = \int_0^z \Phi_t (x,t-s,s) ds \]
			\[ G_z = \Phi(x,t-z,z) \]
			\[ W(x,t) = G(x,t,t)\]
			\[ W_t = G_t + G_z z_t = G_t(x,t,t) + G_z (x,t,t) \cdot 1\]

			\[ W_t(x,t) = \Phi(x,0,t)+ \int_0^t \Phi_t (x,t-s,s) ds\]
			\[ W_{tt} = \Phi_{s} (x,0,t) + \Phi_t(x,0,t) + \int_0^t \Phi_{tt} (x,t-s,s) ds  \]
			\[ W_{xx} = \int_0^L \phi_{xx} (x,t-s,s) ds \]
			\[ W_{tt} - W_{xx} = \Phi_{s}(x,0,t) + \Phi_t(x,0,t) + \int_0^L \phi_{tt} - \Phi_{xx} dx \]

			\[ W(0,t) = \int_0^t \underbrace{\Phi(0,t-s,s)}_{\equiv 0} ds = 0  \]
			\[ W(L,t) = \int_0^t \Phi(L,t-s,s) ds = 0 \]

			\[ W(x,0) = \int_0^0 \Phi(x,0-s,s) ds = 0 \]
			\[ W_t (x,0) = 0\]

			Nos hemos dejado el lado derecho de la ecuación, que es:

			\[\Phi_s (x,0,t) + \underbrace{\Phi_t(x,0,t)}_{\gor{F}(x,t)} \]

			Finalmente:

			\[ \Phi(x,0,s) = 0, \forall s \Rightarrow \Phi_s (x,0,s) = 0 \forall s  \]
			\[ \Phi_t(x,0,s) = \gor{F}(x,s), \forall s \]

			Hemos descompuesto el problema en cuatro problemas más pequeños. Y hemos resuelto cada uno, completándolo con este último. La suma de ellos será una solución del primero, y además sabemos que es única por el resultado obtenido antes.

			Ahora veamos que pasa con otras condiciones de contorno:

			\[ \begin{cases}
		 		u_{tt} - u_{xx} = F(x,t)\\
		 		u(0,t) = \alpha(t), u(L,t) = \beta(t) \\
		 		u(x,0) = f(x) \\
		 		u_t(x,0) = g(x)
		 	\end{cases}\]

		 	Lo cual cambiamos al problema:

		 	\[ \begin{cases}
		 		v_{tt} - v_{xx} = F(x,t)\\
		 		v(0,t) = \alpha(t), v(L,t) = \beta(t) \\
		 		v(x,0) = \gor{f}(x) \\
		 		v_t(x,0) = \gor{g}(x)
		 	\end{cases}\]

		 	Y llegamos igual que antes hasta el problema de $W$ ($W = u-v$):

		 	\[ \begin{cases}
		 		W_{tt} - W_{xx} = 0\\
		 		W(0,t) = 0 = W(L,t) \\
		 		W(x,0) = f - \gor{f} \\
		 		W_t(x,0) = g-  \gor{g}
		 	\end{cases}\]

		 	Por conservación de la energía:

		 	\[ \frac{1}{2} \int_0^L (W_t)^2 + (W_x)^2 dx = \frac{1}{2} \int_0^L (g-\gor{g})^2 + ((f-\gor{f})')^2 dx \]

		 	Hemos demostrado también dependencia contínua de datos. Lo que quiere decir que datos pequeños nos llevan a energías pequeñas.

		 	Y por lo tanto acabamos diciendo que el problema de las ondas es un problema bien propuesto: tiene existencia, unicidad y dependencia contínua.






