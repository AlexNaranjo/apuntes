% -*- root: ../EDP2016.tex -*-
\chapter{Comportamiento cualitativo}

% clase 15/3/2016

En este capítulo vamos a ver:

\begin{itemize}

	\item ondas (hiperbólica)
	\item calor (parabólica)
	\item laplace (elípticas)

\end{itemize}


\section{Ondas}

	Hemos visto ya el problema en dimensión 1: la cuerda vibrante.

	Hemos visto también el método de separación de variables que solo nos valdrá en un dominio acotado. Pero ¿qué podemos hacer en dominios no acotados? y ¿Hay más soluciones?.

	Después vimos al fórmula D'Alambert \ref{eq:DALEMBERT}, que si que nos sirve para dominios no acotados pero es difícil de interpretar en dominios acotados. Veamos un ejemplo de esta afirmación:

	\begin{example}

		\[\begin{cases}
			u_{tt} - u_{xx} = 0, x \in (0,L) t > 0 \\
			u(0,t) = u(L,t) = 0, t > 0\\
			u(x,0) = f(x) \\
			u_t(x,0) = 0
		\end{cases}\]

		Y tenemos la fórmula a la que llegamos:

		\[ u(x,t) = \frac{1}{2} \{f(x+t)+f(x-t)\}  \]

		Resulta que tenemos una fórmula muy útil pero ocurre que en un dominio acotado no podemos calcular $f$ cerca del borde al realizar $x+t$ o $x-t$. Tenemos que encontrar una manera de extender $f$ de manera que esté de acuerdo con el contorno. No es lo mismo que una cuerda esté sujeta y la onda se refleje de una manera de vuelta en la cuerda a que no esté sujeta.

		\begin{center}
			\begin{tikzpicture}
			\draw[-] (-1,0) -- (3,0);
			\draw[-] (0,-0.5) -- (0,2);


			\draw[dashed] (1.5,0) node [below] {$L$} -- (1.5,1.5);

			\draw[thick, blue] plot[smooth, tension=.9] coordinates{(0.7,0) (0.8,0.2) (1,0.4) (1.2,0.7) (1.5,0.9)};

			\end{tikzpicture}
		\end{center}

	\end{example}


	\subsection{Unicidad}

		\[\begin{cases}
		   u_{tt}-u_{xx} = 0, x \in (0,L), t >0\\
		   \text{contorno} \begin{cases}
		   	\text{dirichlet}\\
		   	\text{neumann}\\
		   	\text{periódicas}
		   \end{cases}\\
		   \begin{cases}
		   u(x,0) = 0 \\
		   u_t (x,t) = 0
		   \end{cases}
		\end{cases}
		 \]


		 Supongamos que $u_1, u_{2}$ son soluciones. Luego $u = u_1 - u_2$ es solución + contorno + datos. Queremos probar que $u\equiv0$.

		 Como la ecuación es homogénea, la podemos multiplicar por $u_t$ e integrarla:

		 \[ \int_{0}^{L} u_t (u_{tt}- u_{xx}) dx = 0 \]

		 \( \int_0^L \underbrace{u_t u_{tt}}_{\frac{1}{2}(u^2_t)_t} - u_t u_{xx} dx  \label{eq:03_15_unicidad} \)

		 Miremos el segundo término de la integral. Operamos por partes:

		 \[ \int_0^L  \underbrace{u_t}_{u} \underbrace{u_{xx} dx}_{dv} = \underbrace{\left. u_t u_x \right|_{x=0}^{L}}_{= 0} - \int_0^L \underbrace{u_x u_{tx}}_{\frac{1}{2}(u_x^2)_t} dx \]

		 Y sustituyendo en \ref{eq:03_15_unicidad} tenemos que:

		 \[  \int^{L}_{0} \frac{1}{2} (u_t^2)_t dx - 0 - \left(- \int_0^L \frac{1}{2}(u_x^2)_t dx \right) = \frac{1}{2} \int^{L}_{0}  (u_t^2)_t + (u_x^2)_t dx = \frac{1}{2} \left( \int^{L}_{0} u_t^2 + u_x^2 dx \right)_t\]

		 \textbf{Conclusión}

		 \[ E(t) = \frac{1}{2} \int_{0}^L u_t^2(x,t) + u_x^2(x,t) dx \]

		 \[ E'(t) = 0 \quad \forall t \quad \text{\bf Conservación de la energía total} \]


		 \[ E(t) = E(0) = \frac{1}{2} \int_0^L u_t^2 (x,0) + u_x^2 (x,0) dx = \frac{1}{2} \int_0^L g^2(x)+(f'(x))^2 dx \eqreason{$f=g=0$} = 0 \]

		 \[ E=0 \Rightarrow u_t = u_x = 0 \Rightarrow u \equiv \text{cte. y } u(x,0) = 0 \Rightarrow u \equiv 0\]

		 Luego hemos demostrado que las dos soluciones son iguales y por lo tanto solo hay una solución.

		 Tenemos también:

		 \[ E(t) =  \frac{1}{2}\int_0^L (u_t)^2 + (u_x)^2 dx \rightarrow E'(t) = u_t (L,t) u_x(L,t) - u_t(0,t) u_x(0,t) \]

		 Veamos que pasa dependiendo del tipo de contorno que tengamos:

		 \begin{itemize}
		 	\item Contorno Dirichlet
		 		\[ u(0,t) = u(L,t) = 0, \forall t \Rightarrow u_t (0,t) = u_t(L,t) = 0 \forall t \Rightarrow E'(t) = 0 \]

		 	\item Contorno Neumann
		 		\[ u_x(0,t) = u_x(L,t) = 0, \forall t \Rightarrow  E'(t) = 0 \]

		 	\item Periódicas
		 		\[ u(0,t) = u(L,t), \forall t \Rightarrow u_t (0,t) = u_t(L,t) \forall t \quad \text{y} \quad u_x(0,t) = u_x(L,t), \forall t \Rightarrow E'(t) = 0 \]

		 	\item mixtas

		 \end{itemize}

		 Este resultado es la \concept{conservación de la energía}:

		 \[E(t) = \text{cte.} = E(0) = \frac{1}{2} \int_0^L (g(x))^2 + (f'(x))^2 dx  \]

		 \textbf{Consecuencias:}

		 	Veamos el caso Dirichlet, pero esto es válido para todos:

		 	\[ \begin{cases}
		 		u_{tt} - u_{xx} = F(x,t)\\
		 		u(0,t) = \alpha(t), u(L,t) = \beta{t} \\
		 		u(x,0) = f(x) \\
		 		u_t(x,0) = g(x)
		 	\end{cases}\]
		 	\[u_1,u_2 \in C^2 \text{ soluciones}\]

		 	En este problema no tenemos conservación, debido a $\alpha$, $\beta$ y $F(x,t)$ pero podemos transformarlo en un problema de $W$ en donde si tengamos esta propiedad:

		 	\[\begin{cases}
		 		W = u_1 - u_2 \\
		 		W_{tt} - W_{xx} = 0 \\
		 		W(0,t) = 0 = W(L,t) \\
		 		W(x,0) = 0 \\
		 		W_t(x,0)
		 	\end{cases}\]

		 	Por conservación de la energía tenemos:

		 	\[ \int_0^L (W_t)^2 + (W_x)^2 dx = 0 \Rightarrow W_t = W_x = 0 \Rightarrow W \equiv \text{cte.} \eqreason[\Rightarrow]{$W|_{t=0} = 0$} W(x,t) = 0 \forall x \forall t \Rightarrow u_1 = u_2 \]

		 	Por lo que hemos obtenido unicidad de las soluciones del problema inicial. Pero aunque sepamos que la solución es única, no sabemos si existe tal solución. Volvemos al problema inicial:

		 	\[ \begin{cases}
		 		u_{tt} - u_{xx} = F(x,t)\\
		 		u(0,t) = \alpha(t), u(L,t) = \beta{t} \\
		 		u(x,0) = f(x) \\
		 		u_t(x,0) = g(x)
		 	\end{cases}\]


		 	\[ C(x,t) \text{ tq. } C(0,t) = \alpha(t), C(L,t) = \beta{t}\]

		 	Por ejemplo: $C(x,t) = \alpha(t) + \frac{x}{L} (\beta(t)-\alpha(t))$

		 	Consideramos $v(x,t) = u(x,t) - c(x,t)$ (de modo que $v(0,t) = v(L,t) = 0$)

		 	\[\begin{cases}
		 		v_{tt} - v_{xx} = u_{tt} - c_{tt} - u_{xx} + c_{xx} = F - c_{tt} + c_{xx} = \gor{F} \\
		 		v(0,t) = v(L,t) = 0 \\
		 		v(x,0) = u(x,0) - c(x,0) = f(x) - c(x,0) = \gor{f}(x) \\
		 		u_t (x,0) = … = \gor{g}(x)
		 	\end{cases}
		 	\]

		 	Reescribimos nuestro sistema inicial con $v$ y obtenemos:

		 	\[ \begin{cases}
		 		v_{tt} - v_{xx} = \gor{F}(x,t)\\
		 		v(0,t) = 0, v(L,t) = 0 \\
		 		v(x,0) = \gor{f}(x) \\
		 		v_t(x,0) = \gor{g}(x)
		 	\end{cases}\]

		 	Tomando $v = H + W$ podemos separar nuestro problema en dos:

		 	\[ \begin{cases}
		 			H(x,t) \rightarrow
		 			\begin{cases}
				 		H_{tt} - H_{xx} = 0\\
				 		H(0,t) = H(L,t) = 0 \\
				 		H(x,0) = \gor{f}(x) \\
				 		H_t(x,0) = \gor{g}(x)
			 		\end{cases}\\
			 		W(x,t) \rightarrow
			 		\begin{cases}
				 		W_{tt} - W_{xx} = \gor{F}(x,t)\\
				 		W(0,t) = 0 = W(L,t) \\
				 		W(x,0) = 0 \\
				 		W_t(x,0) = 0
				 	\end{cases}
				 \end{cases}
			\]

			El sistema de la $H$ ya lo sabemos resolver por separación de variables, Fourier...

			El problema de la $W$ lo resolvemos aplicando Duhamel (método del impulso). Fijamos $s>0$:

			\[\begin{cases}
				\left.
				\begin{array}{l}
					\Phi_{tt} - \Phi_{xx} = 0 \\
					\Phi(0,t) = \Phi(L,t) = 0 \\
					\Phi(x,0) = 0 \\
				\end{array}
				\right| (FALTA) \\
				\Phi_t(x,0) = \gor{F}(x,s)
			\end{cases}\]


			\textbf{Notación:}


			\[ \Phi = \Phi(x,t,s)\]

			Respuesta al impulso $\gor{F}(x,s)$ que actúa en $t=0$.

			\[ \Phi(x,t-s,s)\]

			Respuesta al impulso $\gor{F} (x,s)$ que actúa en $t-s = 0$ ($t=s$)

			Aplicando Duhamel:

			\[ W(x,t) = \int_0^t \Phi(x,t-s,s) ds \]

			\[ G(x,t,z) = \int_0^z \Phi(x,t-s,s) ds\]
			\[ G_t = \int_0^z \Phi_t (x,t-s,s) ds \]
			\[ G_z = \Phi(x,t-z,z) \]
			\[ W(x,t) = G(x,t,t)\]
			\[ W_t = G_t + G_z z_t = G_t(x,t,t) + G_z (x,t,t) \cdot 1\]

			\[ W_t(x,t) = \Phi(x,0,t)+ \int_0^t \Phi_t (x,t-s,s) ds\]
			\[ W_{tt} = \Phi_{s} (x,0,t) + \Phi_t(x,0,t) + \int_0^t \Phi_{tt} (x,t-s,s) ds  \]
			\[ W_{xx} = \int_0^L \phi_{xx} (x,t-s,s) ds \]
			\[ W_{tt} - W_{xx} = \Phi_{s}(x,0,t) + \Phi_t(x,0,t) + \int_0^L \phi_{tt} - \Phi_{xx} dx \]

			\[ W(0,t) = \int_0^t \underbrace{\Phi(0,t-s,s)}_{\equiv 0} ds = 0  \]
			\[ W(L,t) = \int_0^t \Phi(L,t-s,s) ds = 0 \]

			\[ W(x,0) = \int_0^0 \Phi(x,0-s,s) ds = 0 \]
			\[ W_t (x,0) = 0\]

			Nos hemos dejado el lado derecho de la ecuación, que es:

			\[\Phi_s (x,0,t) + \underbrace{\Phi_t(x,0,t)}_{\gor{F}(x,t)} \]

			Finalmente:

			\[ \Phi(x,0,s) = 0, \forall s \Rightarrow \Phi_s (x,0,s) = 0 \forall s  \]
			\[ \Phi_t(x,0,s) = \gor{F}(x,s), \forall s \]

			Hemos descompuesto el problema en cuatro problemas más pequeños. Y hemos resuelto cada uno, completándolo con este último. La suma de ellos será una solución del primero, y además sabemos que es única por el resultado obtenido antes.

			Ahora veamos que pasa con otras condiciones de contorno:

			\[ \begin{cases}
		 		u_{tt} - u_{xx} = F(x,t)\\
		 		u(0,t) = \alpha(t), u(L,t) = \beta(t) \\
		 		u(x,0) = f(x) \\
		 		u_t(x,0) = g(x)
		 	\end{cases}\]

		 	Lo cual cambiamos al problema:

		 	\[ \begin{cases}
		 		v_{tt} - v_{xx} = F(x,t)\\
		 		v(0,t) = \alpha(t), v(L,t) = \beta(t) \\
		 		v(x,0) = \gor{f}(x) \\
		 		v_t(x,0) = \gor{g}(x)
		 	\end{cases}\]

		 	Y llegamos igual que antes hasta el problema de $W$ ($W = u-v$):

		 	\[ \begin{cases}
		 		W_{tt} - W_{xx} = 0\\
		 		W(0,t) = 0 = W(L,t) \\
		 		W(x,0) = f - \gor{f} \\
		 		W_t(x,0) = g-  \gor{g}
		 	\end{cases}\]

		 	Por conservación de la energía:

		 	\[ \frac{1}{2} \int_0^L (W_t)^2 + (W_x)^2 dx = \frac{1}{2} \int_0^L (g-\gor{g})^2 + ((f-\gor{f})')^2 dx \]

		 	Hemos demostrado también dependencia continua de datos. Lo que quiere decir que datos pequeños nos llevan a energías pequeñas.

		 	Y por lo tanto acabamos diciendo que el problema de las ondas es un problema bien propuesto: tiene existencia, unicidad y dependencia contínua.


		 	% Clase gjulianm 29/3

		\subsection{Deducción de la ecuación de onda}

		\begin{figure}[hbtp]
		\centering
		\inputtikz{TensionCuerda}
		\caption{Esquema para la demostración, dejando sólo un trozo de cuerda y sustituyendo por la tensión.}
		\label{fig:TensionCuerda}
		\end{figure}

		La idea para la deducción de la ecuación de onda es la siguiente: quitar un cacho de cuerda y sustituirlo por una tensión $T(x,t)$, que representa el efecto del trozo de cuerda a la derecha del punto $x$ en el instante $t$.

		Las componentes horizontales están en equilibrio, luego tienen que anularse: \( T(x+h) \cos θ(x+h) - T(x) \cos θ(x) = 0 \label{eq:Onda:EquilibrioHorizontal} \)

		Las componentes verticales tienen que seguir la Ley de Newton: \( T(x + h) \sin θ(x+h) - T(x) \sin θ(x) = m · u_{tt} \label{eq:Onda:LeyNewton} \), donde $m$ es la masa que se calcula a partir de la densidad ρ (constante) y de la longitud de la curva: \[ m = ρ \int_{x}^{x+h} \sqrt{1+u_x^2} \dif s\]

		Para hacer la deducción de la ecuación, dividiremos entre $h$ en \eqref{eq:Onda:LeyNewton}, y haciendo tender $h \to 0$ tenemos la derivada: \[ \left(T(x) \sin θ(x)\right)_x = ρ\sqrt{1+u_x^2(x,t)} u_{tt} \]

		Haciendo el truco de escribir $\sin θ = \cos θ · \tan θ$, tenemos que $\tan θ = u_x$ y nos queda lo siguiente:  \[ \left(T(x) \cos θ(x) u_x \right)_x = ρ\sqrt{1+u_x^2(x,t)} u_{tt} \]

		Ahora derivamos y vemos qué ocurre: \[ \left(T(x) \cos θ(x) u_x \right)_x = \left(T(x) \cos θ(x)\right)_x u_x + T(x) \cos θ(x) u_{xx} \]

		Podemos simplificar viendo que \eqref{eq:Onda:EquilibrioHorizontal} nos decía que $T(x+h) \cos θ(x+h) - T(x) \cos θ(x) = 0$, luego $(T(x) \cos θ(x))_x = 0$. Además, por no sé qué tenemos que $T(x) \cos θ(x)$ es constante. % TODO

		Sólo nos falta quitarnos la raíz esa fea. Para ello hacemos la simplificación de tener oscilaciones pequeñas, de tal forma que $u_x$ es pequeño, $u_x^2$ es todavía más pequeño y $1 + u_x^2 \approx 1$. Así, nos queda la ecuación que ya conocemos: \[ u_{tt} - c^2 u_{xx} = 0\]

		\subsection{Ecuación de ondas en $ℝ$}

		Partimos de nuestra ecuación homogénea: \[ \begin{cases}
		u_{tt} - c^2 u_{xx} = 0 \\
		u(x,0) = f(x) \\
		u_t(x, 0) = g(x)
		\end{cases}\]

		La cuestión es que aquí no podemos aplicar las series de Fourier, porque no tenemos una longitud acotada: estamos estudiando la ecuación en todo $ℝ$. Lo que haremos será ver si aplicando la Fórmula de D'Alembert nos sale algo. Para eso, hacemos el siguiente cambio de variables: \begin{align*}
		ξ &= x + ct & ξ_x = 1,\,& ξ_t = c \\
		η &= x - ct & η_x = 1,\,& η_t = - c \\
		\end{align*}

		Haciendo las cuentas tenemos lo siguiente: \begin{align*}
		u_x &= u_ξ + u_η \\
		u_{xx} &= u_{ξξ} + 2 u_{ηξ} + u_{ηη} \\
		u_t &= c u_ξ - c u_η \\
		u_{tt} &= c^2 u_{ξξ} + 2c^2 u_{ηξ} + c^2 u_{ηη}
		\end{align*}

		La conclusión de todo esto es que la ecuación nos queda así: \[ 0 = u_{tt} - c^2 u_{xx} = -4c^2 u_{ηξ} \], de tal forma que la ecuación final nos queda muy simple: \[ u_{ηξ} = 0\]

		De ahí podemos sacar fácilmente que $u_η = α(η)$, una función que sólo depende de $η$. Por tanto, podemos sacar la fórmula para $u$: \[ u = \int α(η) \dif η + B(ξ) = A(η) + B(ξ) = A(x-ct) + B(x+ct)\]

		Si $u(x,0) = f(x)$, entonces $f(x) = A(x) + B(x)$, y si $u_t(x,0) = g(x)$ entonces $g(x) = -cA'(x) + cB'(x)$. El sistema resultante es el siguiente: \[ \begin{cases}
		A' + B' = f \\
		-A' + B' = \dfrac{1}{c} g
		\end{cases} \]

		Resolviéndolo, llegaremos de nuevo a la fórmula de D'Alembert: \( u(x,t) = \frac{1}{2} \left(f(x + ct) + f(x-ct)\right) + \frac{1}{2c} \int_{x-ct}^{x+ct} g(s) \dif s \label{eq:Onda:DAlembert} \)

		A partir de aquí podemos encontrar algunas consecuencias y estudiar la función de onda. Lo primero es ver que la única solución del problema en todo $ℝ$ viene dada por esta fórmula: no hemos introducido ni quitado nuevas soluciones.

		\begin{figure}[hbtp]
		\centering
		\inputtikz{ReprGraficaEcOnda}
		\caption{Representación gráfica y dominio de dependencia de la solución y de influencia de los datos.}
		\label{fig:ReprGraficaEcOnda}
		\end{figure}

		Lo siguiente es estudiar lo que se ve en la \fref{fig:ReprGraficaEcOnda}. $u(x,t)$ sólo dependerá de los valores de $f$ en los puntos $a,b$ y del valor de $g$ en el intervalo $(a,b)$. La zona sombreada en la figura será el dominio de influencia del intervalo $[a,b]$: es sólo ahí donde influyen los datos de $f,g$ de esos puntos. Por ejemplo, si en $t = 0$ $f = g \equiv 0$ fuera de $[a,b]$, entonces en $t = T$ la solución será cero fuera de $[a - cT, b + cT]$. En otras palabras, hay una velocidad de propagación finita de los datos.

		La aplicación de esto es la demostración de la conservación de la energía. Partíamos de una ecuación \[ E(t) = \frac{1}{2} \int_{-∞}^∞ u_x^2 \dif x + \frac{c^2}{2} \int_{-∞}^∞ u_t^2 \dif x\], y derivando y haciendo cuentas nos salía que \[ E'(t) = \int_{-∞}^∞ u_t(u_{tt} - c^2 u_{xx}) \dif x = 0\], luego la energía se conservaba.

		Ahora bien, cuando hacíamos eso en dominios acotados, lo que necesitamos era la hipótesis de los valores de contorno para poder demostrar que salía lo que tenía que salir. Aquí lo que necesitaremos simplemente es que $f(x)$ y $g(x)$ sean funciones de soporte compacto (son $0$ fuera de un intervalo $[-M, M]$), de tal forma que para tiempos finitos se cumple que en el instante $t$, $u(x,t)$ es $0$ fuera del intervalo $[-M - ct, M + ct]$ y por tanto los términos de borde en la integración por partes para sacar $E'(t)$ se anulan.

		Como nota, no podríamos pedir sólo que $u_x$ y $u_t$ fuesen integrables $L^2$, más que nada porque eso no implica que se vayan a $0$ en $∞$.

		La conservación de la energía nos dará unicidad de solución, también para el problema no homogéneo; y dependencia continua de los datos. Estos resultados habría que compararlos con el teorema en dominios acotados, que es donde lo demostramos en su día.

		Y luego una prueba de algo con un paralelogramo del que no me he llegado a enterar demasiado bien:

		\begin{figure}[hbtp]
		\centering
		\inputtikz{Paralelogramo}
		\caption{Paralelogramo.}
		\label{fig:Paralelogramo}
		\end{figure}

		Una aplicación de esto es el de problemas con reflexiones: tenemos una cuerda en el intervalo $[0, ∞)$ y nuestro sistema \[ \begin{cases}
		u_{tt} - c^2u_{xx} = 0 & x > 0,\, t ∈ ℝ \\
		u(0,t) = 0 & ∀ t ∈ ℝ \\
		u(x,0) = f(x) & x > 0 \\
		u_t(x,0) = g(x) & x > 0
		\end{cases}\]

		El primer método implicará una extensión de los datos a funciones $\tilde{f}, \tilde{g}$ a las que ya veremos qué valor dar cuando $x ≤ 0$, y sacaremos una solución $\tilde{u}$ por la fórmula de D'Alembert \eqref{eq:Onda:DAlembert}. En particular, querremos que $\tilde{u}(0,t) = 0\,∀t ∈ ℝ$. Sustituyendo en la fórmula, \[ 0 = \tilde{u}(0,t) = \frac{1}{2} \left(f(ct) + f(ct)\right) + \frac{1}{2c} \int_{-ct}^{ct} g(s) \dif s \]

		Para que eso sea 0, necesitamos que $\tilde{f}$ y $\tilde{g}$ sean impares. Así, nuestras extensiones serán \[
		\tilde{f}(x) = \begin{cases}
		f(x) & x > 0 \\
		-f(-x) & x ≤ 0
		\end{cases} \qquad
		\tilde{g}(x) = \begin{cases}
		g(x) & x > 0 \\
		-g(-x) & x ≤ 0
		\end{cases}
		\]

		De ahí sacaríamos la solución $\tilde{u}(x,t)$ y nuestra solución final sería $u(x,t) = \tilde{u}(x,t)$ restringida a $x > 0$.

		Sin embargo, tenemos un segundo método de resolución a través de una interpretación geométrica: gracias al método del paralelogramo podremos sacar el valor de la solución despejando (ver \fref{fig:AplicacionParalelogramo}).

		\begin{figure}[hbtp]
		\inputtikz{AplicacionParalelogramo}
		\caption{Para poder obtener los valores cuando sólo sabemos parte de la solución (en este caso, el sombreado azul), podemos usar la regla del paralelogramo, colocando tres puntos en zonas donde sabemos el valor de la solución (uno en el dato inicial, otros dos en la solución calculada) y despejamos para el cuarto punto (en rojo).}
		\label{fig:AplicacionParalelogramo}
		\end{figure}

		Vamos a ver un ejemplo de este método, con la ecuación de siempre y datos iniciales $g \equiv 0$ y \[ f(x) = \begin{cases} 1 & x ∈ [1,2] \\ 0 x ∈ [0,1) ∪ (2,∞) \end{cases} \]

		Sacando con la regla del paralelogramo los valores tendríamos algo como lo de la \fref{fig:OndaReflexion}, donde la onda rebotaría por la izquierda pero invertida. Además, en la zona donde la onda se anula, la energía se quedaría en la derivada $u_t$, no se nos pierde.

		\begin{figure}[hbtp]
		\centering
		\inputtikz{OndaReflexion}
		\caption{Propagación de la onda con un borde en la izquierda. En las zonas sin sombreado, la onda tiene amplitud nula.}
		\label{fig:OndaReflexion}
		\end{figure}


		Como ejercicios, habría que ver qué ocurre con las ondas en tiempos $t = 2, \frac{3}{4}, \frac{1}{3}$ con $c = 1$. También es interesante ver qué es lo que ocurre cuando $f \equiv 0$ y $g = \ind_{[1,2]}$: el dibujo cambiará de forma esencial.

		El último ejercicio sería ver qué ocurriría con dos bordes. Veríamos los rebotes y las zonas donde las dos ondas que se propagan se juntan.

		\subsection{Caso no homogéneo}

		Nos vata saber qué es lo que ocurre cuando tenemos una fuerza externa, es decir, cuando nuestro sistema es \[ \begin{cases}
		u_{tt} - c^2 u_{xx} = F(x,t) & t ∈ ℝ, \, x ∈ ℝ \\
		u(x,0) = f(x) \\
		u_t(x,0) 0 g(x) \end{cases}\]

		Recuperando lo que habíamos visto en casos anteriores, lo que haremos será separar en dos sistemas \[ \begin{cases}
		v_{tt} - c^2 v_{xx} =0  & t ∈ ℝ, \, x ∈ ℝ \\
		v(x,0) = f(x) \\
		v_t(x,0) = g(x) \end{cases} \qquad \begin{cases}
		w_{tt} - c^2 w_{xx} = F(x,t) & t ∈ ℝ, \, x ∈ ℝ \\
		w(x,0) = 0 \\
		w_t(x,0) = 0) \end{cases}\]

		Resolviendo ambos sistemas, la solución final será la suma de soluciones. Para el primero usaríamos la fórmula de D'Alembert, y para el segundo usaríamos el método de Duhamel. Fijamos $t = τ > 0$, y entonces resolvemos el problema \[ \begin{cases}
		Φ_{tt} - c^2 Φ_{xx} = 0 \\
		Φ(x,0) = 0 \\
		Φ_t(x,0) = F(x,τ) \end{cases}\]

		Así, el impulso $F(x,τ)$ en $t= 0$ sale como \[ Φ(x,t) = \frac{1}{2c} \int_{x-ct}^{x+ct} F(s,τ) \dif τ\], y para $t = τ$ tendríamos \[ Φ(x, t-τ) = \frac{1}{2c} \int_{x-c(t-τ)}^{x + c(t - τ)} F(s, τ) \dif s\]

		\begin{figure}[hbtp]
		\centering
		\inputtikz{IntegralOndaNoseque}
		\caption{La integral que hacemos es la del triángulo, con $f$ aportando en los puntos iniciales (verde), $g$ aportando en el intervalo naranja y $F$ en la zona sombreada.}
		\label{fig:Onda:Noseque}
		\end{figure}

		De esta forma, la solución para un punto dado es el impulso a lo largo del tiempo, esto es, \[ w(x,t) = \frac{1}{2c} \int_0^t Φ(x,t-τ) \dif τ = \frac{1}{2c} \int_0^t \int_{x-c(t-τ)}^{x + c(t - τ)} F(s, τ) \dif s \]

		Esta integral es en el fondo lo mismo que integrar en el triángulo de la \fref{fig:Onda:Noseque} con las aportaciones que se comentan. Además, la conservación de la energía nos daría la unicidad de la solución para este problema.

		Y ahora, dentro del mundo de cosas que no sé qué son (creo que un ejercicio, llevo mucho tiempo copiando), hay un ejercicio en el que una condición adicional es que $u_x(0,t) = 0$. Habría que hacer las extensiones, calculando $\tilde{u}$ y derivando para sacar las condiciones sobre las extensiones. Ahora bien, también se puede pasar a una condición de contorno tipo Dirichlet buscando la solución $v = u_x$, que nos quedaría el sistema de otra ecuación de onda \[ \begin{cases} v_{tt} - c^2v_{xx} = 0 \\ v(0,t) = 0 \\ v(x,0) = f'(x) \\ v_t(x,0) = g'(x) \end{cases}\] con condiciones de contorno tipo Dirichlet, que podemos resolver por la identidad del paralelogramo de antes.

		BOOOOOM.

		\subsection{Energía}

			Vamos a repasar dos de los problemas de la hoja 3:

			\begin{problem}

				\[\begin{cases}
					u_{tt} - c^2u_{xx} + u_t = 0, x \in (a,b)\\
					u(a,t) = u_x (b,t) = 0\\
					u(x,0) = f(x) \\
					u_t(x,0) = g(x)
				\end{cases}\]

				¿Tenemos unicidad?

				\solution

				Suponemos $u_1,u_2$ soluciones:

				\[ \begin{cases} W = u_1 - u2 \\
				W_{tt} - c^2 W_{xx} + W_t = 0 \\
				W(a,t) = W_x (b,t) = 0
				W(x,0) = 0\\
				W_t (x,0) = 0
				\end{cases}\]

				Método de la energía:

				\[ 0 = (W_{tt} - c^2 W_{xx} + W_t) W_t dx\]

				Si integramos seguimos teniendo 0:

				\[ 0 = \int_a^b (W_{tt} - c^2 W_{xx} + W_t) W_t dx\]
				\[ = \int_a^b W_{tt} W_t dx - c^2 \int^b_a W_{xx} W_t dx + \int_{a}^b W^2_t dx  \]

				Tenemos que $W_{tt}W_t \equiv \frac{1}{2} (W_t)^2 $. Además, integrando por partes:

				\[ \int_a^b \underbrace{W_xx}_{dv} \underbrace{W_t}_{u} dx = W_x W_t |_a^b - \int_a^b \underbrace{W_x W_{tx}}_{(\frac{W_x^2}{2})_t} dx \]

				(FALTAN UN PAR DE PUNTOS EN ASTERISCOS) % TODO

				Conclusión:

				\[ E'(t) + \int_a^b W_t^2 dx = 0\]
				\[ E'(t) = -\int_a^b W_t^2 dx \leq 0 \]

				\[ \left. \begin{array}{r}
					E \text{ decrece} \\
					E \geq 0 \\
					E(0) \eqreason{en w} 0 \end{array} \right\} \Rightarrow E \equiv 0 \Rightarrow W \equiv 0 \]

			\end{problem}

			\begin{problem}

				\[\begin{cases}
					u_{tt} - c^2 u_{xx} + hu = F(x,t), x \in \real \\
					u, u_x, u_t \convs[x][±\infty] 0, \forall t\\
					\int_{-\infty}^\infty u_t^2 + c^2 u_x^2 + hu^2 dx < \infty (\forall t) \\
					u(x,0) = f(x) \\
					u_t(x,0) = g(x)
				\end{cases}\]

				¿Hay unicidad?

				\solution

				Suponiendo $u1,u_2$ soluciones:

				\[\begin{cases}
					W = u_1 - u_2 \\
					W_{tt} - c^2 W_{xx} + h W = 0
				\end{cases}\]

				Energía:
				\[ 0 = \int_{-\infty}^\infty (w_{tt} - c^2 W_{xx} + h W)W_t dx = … \]

				Integrando por partes en $\int\limits_{-\infty}^\infty W_{xx} W_{t}$, los términos de borde se anulan como en el caso anterior:

				\[ … = \int_{-\infty}^{\infty}  (\frac{W_t^2}{2})_t dx + c^2 \int_{-\infty}^\infty (\frac{W_x^2}{2})_t dx + h \int_{-\infty}^\infty (\frac{W^2}{2})_t dx \]

				Con lo que llegamos (usando $hWW_t = h (\frac{W^2}{2})_t$):

				\[ 0 = (\infty_{-\infty}^\infty  \frac{W_t^2}{2} + c^2 \frac{W_x^2}{2} + h \frac{W^2}{2} dx)_t\]

				(FALTA FINAL)


			\end{problem}


			Entremos un poco más en detalle con la energía y sus reflexiones en base a lo descrito anteriormente. Tomemos:

			\[\begin{cases}
				u_{tt} - u_{xx} = 0, x > 0 \\
				u(0,t) = 0 \forall t \\
				u(x,0) = f(x) \\
				u_t(x,0) = 0
			\end{cases}\]

			(FALTA TERMINAR EL DIBUJO)

			\begin{center}
			\begin{tikzpicture}
				\draw[->] (-2.1, 0) -- (4.2,0) node[right] {$x$};
				\draw[->] (0, -0.1) -- (0,3.2) node[above] {$t$};

				\node[vnlin, label = {below:$1$}] at (1,0) {};
				\node[vnlin, label = {below:$2$}] at (2,0) {};
				\node[vnlin, label = {below:$-1$}] at (-1,0) {};
				\node[vnlin, label = {below:$-2$}] at (-2,0) {};

				\draw[gray] (1,0) -- (4,3) ;
				\draw[gray] (2,0) -- (4,2) ;
				\draw[gray] (2,0) -- (0,2) -- (1,3);
				\draw[gray] (1,0) -- (0,1) -- (2,3);
				\draw[gray] (-1,0) -- (0,1);
				\draw[gray] (-2,0) -- (0,2);
				\draw[gray,dashed] (-2,0) -- (-3,1);
				\draw[gray,dashed] (-1,0) -- (-2,1);

				\fill[pattern = north east lines, pattern color = red] (1, -0.05) rectangle (2, 0.05);
				\fill[pattern = north east lines, pattern color = blue] (-2, -0.05) rectangle (-1, 0.05);
			\end{tikzpicture}
			\end{center}

			\[ u_x = \frac{1}{2} f'(x+t) (x+t)_x + \frac{1}{2} \gor{f}'(x-t)(x-t)_x = \frac{1}{2} \{f' + \gor{f}'\} \eqreason{$t = \frac{3}{2}$} 0 \]

			\[ u_t = \frac{1}{2} f'(x+t) (x+t)_t + \frac{1}{2} \gor{f}' (x-t)(x-t)_t = \frac{1}{2} \{ f'(x+t) - \gor{f} (x-t) \} \eqreason{$t = \frac{3}{2}$} f'  \]

			% No he copiado la explicación especial que dió sobre funciones no derivables TODO

			\obs Esta definición que viene ahora no entra y viene acompañada de un ejemplo que no se ha incluido aquí.

			\begin{defn}[Derivada débil]

				G es la derivada débil de $F$ si y solo si:

				\[ \int_{-\infty}^\infty G\Phi dx = -\int_{-\infty}^\infty F \Phi' dx, \forall \Phi \in C^1, \text{ sop. compacto} \]

				\begin{itemize}
					\item $F$ derivable (clásico) $\Rightarrow F'$ es su derivada débil.
					\item A veces la derivada débil existe aunque $F$ no sea derivable en el sentido clásico.

				\end{itemize}
			\end{defn}

			%Vamos entonces a ver que diferencias habría (en el caso de una cuerda de guitarra, entre darle un golpe a la)

			\[\begin{cases}
				u_{†t} -u_{xx} =0, x \in \real \\
				u(x,0) = f \\
				u_t(x,0) = g
			\end{cases}\]

			\[ u(x,t) = \frac{1}{2} \{ f(x+t) + f(x-t) \} + \frac{1}{2} \int_{x-t}^{x+t} g(s) ds \]

			Contemplamos varios casos:

			\begin{itemize}
				\item $f \neq 0, g = 0$
					(DIBUJO)

				\item $f=0, g \neq 0$
					(DIBUJO)

			\end{itemize}


	\section{Laplaciano}

	Si nos fijamos en la ecuación del calor, para dimensión espacial $N$ nos queda que \[ u_t - (u_{x_1x_1} + u_{x_2x_2} + \dotsb + u_{x_N x_N})\]

	Para la ecuación de ondas, tenemos algo parecido:  \[ u_{tt} - (u_{x_1x_1} + u_{x_2x_2} + \dotsb + u_{x_N x_N})\]

	Ese último paréntesis es una operación en si misma, que llamaremos el \concept{Laplaciano}: \( Δu = \sum_{i=1}^N u_{x_i x_i} = \tr (\Dif^2 u) = \dv (\grad u) \label{eq:Laplaciano}\)

	Es especialmente interesante verlo como la divergencia del gradiente, para después poder aplicar en un futuro el teorema de Gauss para poder integrar.

	Una forma de ver la ecuación de ondas o del calor es como soluciones estacionarias de algo. Por ejemplo, si nos fijamos en el típico experimento de hacer vibrar una membrana con el sonido, estaríamos ante una ecuación \[ \begin{cases} u_{tt} - u_{xx} - u_{yy} \\ \restr{u}{∂Q} = 0 \\ u(x,0) = f(x) \\ u_t(x,0) = g(x) \end{cases} \] con $Q = (0,1) × (0,1)$.

	Podríamos aplicar entonces el método de separación de variables, buscando dos funciones cuyo producto sea la solución: \[ u(x,y,t) = Φ(x,y) T(t) \] de tal forma que las soluciones serían de la forma \[ \frac{T''}{T} = \frac{ΔΦ}{Φ} = λ \]

	La resolución en $Φ$ nos daría un problema de autovalores: \[ \begin{cases} ΔΦ = λ Φ \\ \restr{Φ}{∂Q} = 0 \end{cases} \], lo cual es un problema de autovalores. El resultado sería una sucesión de autovalores $λ_k$ y autofunciones $Φ_k$.

	Volviendo al caso de la membrana, si ponemos arena en la membrana se quedará con una forma, más concretamente en los nodos de la membrana: los puntos en los que no vibra.

	Nosotros no veremos el problema de resolver el laplaciano para dominios arbitrarios porque necesitamos mucho análisis funcional y en este curso no da tiempo. Nos vamos a dedicar a algo que no tengo claro qué es.

	\subsection{Funciones armónicas, subarmónicas y superarmónicas}

	No sé a qué viene esto.

	\begin{defn}[Función\IS armónica] Si $-Δu = 0$ en Ω, entonces $u$ es armónica en $Ω$. \end{defn}
	\begin{defn}[Función\IS subarmónica] Si $-Δu ≤ 0$ en Ω, entonces $u$ es subarmónica en $Ω$. \end{defn}

	\begin{defn}[Función\IS superarmónica] Si $-Δu ≥ 0$ en Ω, entonces $u$ es superarmónica en $Ω$. \end{defn}

	En dimensión $1$, las funciones armónicas son lineales, las subarmónicas convexas y las superarmónicas convexas.

	En dimensiones superiores, las cosas se complican un poco más. Por ejemplo, una función lineal sigue siendo armónica, pero $u(x,y) = x^2 - y^2$ también es armónica.

	Un ejemplo de función subarmónica es un paraboloide en dimensión $N$: \[ u(x_1, \dotsc, x_N) = \norm{\vx}^2 = x_1^2 + \dotsb + x_N^2 \] es subarmónica ya que $-Δu = -2N$.

	En dimensión $2$, la función \[ F(x,y) = \log (x^2 + y^2)\] tiene laplaciano $-ΔF = 0$ en $ℝ^2 \setminus \set{(0,0)}$, por lo que es armónica.

	En dimensión $N > 2$, la función \[ G(\vx) = \frac{1}{\norm{\vx}^{N-2}}\] también es armónica: $-ΔG = 0$ en $ℝ^N \setminus \set{\vec{0}}$.

	Estos dos ejemplos son llamadas \concept{Soluciones\IS fundamentales}, y ya veremos más tarde por qué se llaman así.

	Las funciones armónicas tienen unas propiedades básicas.

	\begin{prop} \label{prop:PropsFuncArmonicas} Sean $u,v$ funciones armónicas en $Ω$.
	\begin{enumerate}[itemsep = 0pt]
	\item La suma $u +v$ es armónica.
	\item El producto $uv$ en general no es armónico.
	\item La traslación $u(x-x_0)$ es armónica en $\set{x \tq x - x_0 ∈ Ω}$.
	\item El escalado $u(kx)$ es armónica en $\set{x \tq kx ∈ Ω}$.
	\item Dada una transformación lineal $\appl{A}{ℝ^N}{ℝ^N}$ con $\norm{Ax} = \norm{x}\;∀xx ∈ ℝ^N$ (isometría), entonces $u(Ax)$ es armónica en $\set{x \tq Ax ∈ Ω}$.
	\item \concept{Transformada\IS de Kelvin} La función \[ K(\vx) = \frac{1}{\norm{\vx}^{N-2}} u\left(\frac{\vx}{\norm{\vx}^2}\right) \] es armónica en $\set{ \vx \tq \frac{\vx}{\norm{\vx}^2} ∈ Ω}$.
	\end{enumerate}
	\end{prop}

	La transformada de Kelvin es especialmente importante, ya que lleva de puntos de dentro de una bola a puntos fuera, y permite estudiar lo que pasa en el centro de la bola (en el $0$) viendo lo que ocurre en el infinito. La cuenta es un poco infernal, y nosotros lo haremos sólo en un caso.

	\begin{proof}[Transformada de Kelvin - $u$ radial]
	Suponemos que $u$ es radial, esto es, que $u$ sólo depende de la norma del vector $\vx$. Diremos entonces que $u \equiv u(r)$ con $r = \norm{\vx}$.

	Vamos a estudiar la ``armonicidad'' de la función. Primero calculamos la derivada $r_i$: \begin{align*}
	r^2 &= x_1^2 + \dotsb + x_N^2 \\
	2rr_{x_i} &= 2x_i \\
	r_{x_i} &= \frac{x_i}{r}
	\end{align*}

	Con esto ya podemos calcular la segunda derivada de $u$: \begin{align*}
	u_{x_i} &= u_{r} r_{x_i} = u_{r} \frac{x_i}{r} \\
	u_{x_ix_i} &= \dotsb = \left(\frac{ur}{r}\right)_r \frac{x_i^2}{r} + \frac{u_r}{r}
	\end{align*}

	Así, el laplaciano queda que es igual a \[ Δu = \sum_{i=1}^N = \left(\frac{ur}{r}\right)_r · r + N\frac{u_r}{r} = \dotsb = u_{rr} + \frac{N-1}{r} u_r \]

	El ejercicio a cargo del lector consistiría en ver que dada la transformada de Kelvin \[ H(r) = \frac{1}{r^{N-2}} u\left(\frac{1}{r}\right)\] entonces \[ H_{rr} + \frac{N-1}{r} H_r = 0\]

	\end{proof}

	¿Cómo se escribiría la transformada de Laplace si dependiese también del ángulo? En ese caso, cada punto $\vx ∈ ℝ^N$ se escribiría como $\vx = r · ω$, con $r ∈ [0, ∞)$ y $ω ∈ \crc[N-1]$, de tal forma que el laplaciano sería \[ Δu(\vx) = u_{rr} + \frac{N-1}{r} u_r + \frac{1}{r^2} Δ_{\crc[N-1]} u \] , donde $Δ_{\crc[N-1]}$ sería el \concept{Operador\IS de Laplace-Beltrami} que es la restricción del laplaciano a la esfera. Esta sería la conexión de esto con la geometría y no vamos a ver nada más de esto.

	\subsection{Propiedades fundamentales: Principios del máximo y de la media}

	El principio del máximo nos dice lo siguiente: si tenemos una cuerda en equilibrio (una función armónica), no tenemos ni máximos ni mínimos fuera de la frontera. Si, por ejemplo, tenemos ondas en una membrana, entonces no está en equilibrio: está vibrando.

	Podemos mirarlo desde otro lado, tomando una función superarmónica con $-Δu = F ≥ 0$. Es decir, que hay un equilibrio en presencia de una fuerza externa $F$. Esta fuerza será positiva si actúa hacia arriba, y negativa si lo hace hacia abajo. El principio del máximo nos dirá que, respectivamente, sólo será posible una onda hacia arriba $u > 0$ o hacia abajo $u < 0$. (dibujo aquí.)

	\begin{prop}[Principio\IS del máximo débil] Sea $Ω$ un dominio\footnote{Conjunto abierto y conexo} acotado en $ℝ^N$, y sea $u$ una función subarmónica en Ω y $u ∈ C^2(Ω) ∩ C(\adh{Ω})$. Entonces el máximo en el cierre es igual al máximo en la frontera: \[ \max_{\adh{Ω}} u = \max_{∂Ω} u \]
	\end{prop}

	El principio del máximo débil sí que nos permite casos en los que el máximo se alcanza en la frontera y también en el interior. Con artillería más fuerte veremos que estos casos, en la práctica, no son posibles; aunque para ello tendremos que esperar un poco.

	\begin{proof}

	\proofpart{$-Δu < 0$ en Ω (desigualdad estricta)}

	Haremos la demostración por reducción al absurdo: supongamos que $\max_{\adh{Ω}}$ se alcanza en $x_0 ∈ \mop{int} Ω$. En ese punto de máximo tiene que pasar dos cosas: que el gradiente se anula ($\grad u = \vec{0}$) y que el hessiano $\Dif^2 u$ tiene que ser semidefinida negativa. Esto es, los autovalores de $\Dif^2 u$ tienen que ser menores o iguales que $0$.

	En ese caso, la traza ha de ser menor o igual que cero, pero $\tr \Dif^2 u(x_0) = Δu(x_0) ≥ 0$, contradicción porque habíamos dicho que $-Δ u < 0$.

	\proofpart{$-Δu ≤ 0$ en Ω (desigualdad no estricta)}

	En este caso miraremos una función modificada $v_ε \approx u$, tal que $-Δv_ε < 0$. Tomaremos entonces \[ v_ε(\vx) = u(\vx) + ε\norm{\vx}^2\], de manera que \[ -Δv_ε (\vx) = -Δu(\vx) - ε Δ(\norm{\vx}^2) = -Δu(\vx)  - 2εN < 0 \]

	Aplicamos entonces el caso anterior a $v_ε$, y entonces hacemos tender $ε \to 0$. Esto queda como ejercicio para el lector. En el paso al límite, tendremos que aceptar la desigualdad no estricta, lo que nos da la debilidad de este principio.
	\end{proof}

	Aunque este principio lo hemos hecho sólo para funciones subarmónicas, para superarmónicas es sólo cambiar el signo. Para funciones armónicas, como son sub y super armónicas, tendremos que el mínimo y máximo se alcanzan en la frontera.

	Este principio será muy potente: nos permitirá demostrar unicidad, estimaciones, y más cosas que no he llegado a copiar.




