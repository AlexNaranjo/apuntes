\chapter{Ideas generales}


\textbf{Matemáticas de la transmisión de información}

\section{Modelo básico}

	emisor --(info)-> receptor

	la flecha es el canal. Ejemplos de canales son:

	\begin{itemize}
		\item cable
		\item aire/espacio
		\item papel
		\item disco duro / USB / CD
	\end{itemize}

	El canal con el que vamos a trabajar tiene 3 defectos:

	\begin{enumerate}
		\item poco seguro
		\item poco fiable
		\item con poca capacidad
	\end{enumerate}


	\subsection{Poca seguridad}
		Existe un "malo", es decir, que no es el emisor o receptor y puede interferir con la información (leerla y alterarla).

		Se soluciona con criptografía.

		Emisor -> [cifrador]---canal--->[descifrador] -> Receptor

	\subsection{Poca fiabilidad}
		El mensaje puede sufrir alteraciones involuntarias, tales como erratas (papel), polvo o ralladuras (disco duro o CD), rayo cósmico, ruido, etc...

		Se soluciona con códigos detectores/correctores de errores.

		Emisor -> [codificador]---canal--->[descodificador] -> Receptor

	\subsection{Poca capacidad}

		Puede faltar capacidad en varios sentidos. Puede faltar tiempo (como en una clase si el profesor repite mucho), capacidad (como en un disco duro) o energía (como cuando un satélite tiene una energía limitada para transferir información).

		Se soluciona con algoritmos de compresión (no es parte del temario).

		Emisor -> [compresor]---canal--->[descompresor] -> Receptor

	\vspace{1cm}


	Si queremos resolver los tres problemas a la vez es obvio que debemos combinarlos todos, pero en que orden?

	Colocaremos en último lugar (cerca del canal) el proceso de codificación y descodificación, ya que depende mucho del canal y puede verse anulado con un posterior proceso de cifrado o compresión.

	Respecto a los otros dos, podemos debatir sobre su orden. Pero hay que tener en cuenta, por lo menos en el caso de cadenas de texto, que conviene comprimir primero, ya que algunos ata1ues sobre el cifrado se pueden realizar si los mensajes cifrados se repiten y/o tienen patrones reconocibles. Estas posibilidades se reducen bastante al comprimir primero.






