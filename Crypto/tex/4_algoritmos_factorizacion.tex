\chapter{Algoritmos de factorización y tests de primalidad}

La base de la seguridad del sistema RSA es el hecho de que es fácil generar números primos grandes (por eso podemos usar el algoritmo) pero factorizar números grandes en factores primos es muy difícil.

A día de hoy, los ordenadores cuánticos serían capaces de factorizar primos a gran velocidad. En el momento en que estos ordenadores sean bastante grandes (con suficiente memoria) para operar con grandes números, el proceso de factorización será muy rápido, con lo que el algoritmo RSA dejará de ser seguro.

No obstante, vamos a seguir estudiando los procesos de factorización y primalidad desde el punto de vista de los ordenadores y los medios actuales.

Si quisiésemos factorizar un número, empleamos el algoritmo que usábamos en el colegio en el que vamos probando con todos los números primos. Es decir, empleamos la \textbf{criba de Eratóstenes}.

Para hacer esto, necesitamos tener una lista de todos los primos hasta $\sqrt{n}$ con los que ir probando.

\begin{theorem}[Teorema del número primo]
El número de primos menores que un número $x$ es:
\[π(x) = O\left( \frac{x}{\log(x)}\right)\]
\end{theorem}

Una vez tenemos la lista de primos existentes menores que $\sqrt{n}$ (cosa que en una situación real no se tiene pues $n$ es enorme), tendríamos que hacer un total $π(x)$ divisiones y, puesto que cada divisón nos supone $O(\log^3(n))$ bit-operaciones, tenemos que el total del algoritmo nos supone:
\[O(\log^3(n))\cdot O\left(\frac{\sqrt{n}}{\log(\sqrt{n})}\right) = O\left( e^{\frac{1}{2}\log(n)}\log^2(n)\right) \text{ operaciones }\]

\begin{example}
Supongamos que queremos factorizar un número del orden de $10^{75}$ empleando el ordenador mas rápido del punto.

El ordenador mas rápido del punto es capaz de realizar 33,86 PETAFLOPS, es decir, realizar un total de $33.86 \cdot 10^{15}$ operaciones por segundo.

Puesto que un año tiene $3,1537 \cdot 10^{7}$ segundos, tenemos que en un año podemos realizar, \textbf{con el ordenador más potente del mundo} un total de $10^{24}$ operaciones.

Suponiendo que cada división es una operación (que no es así), necesitaremos $10^{50}$ años para poder factorizar el número $10^{75}$.
\end{example}

\section{Test de primalidad}

Si queremos comprobar si un número es primo, no podemos depender de la factorización del mismo, puesto que entonces no podríamos comprobar la primalidad de números grandes. Para hacerlo nos apoyamos en el siguiente teorema.

\begin{theorem}[Teorema de Wilson (1770)]
\[n \text{ es primo } \iff (n-1)! = -1 \mod n\]
\end{theorem}
\begin{proof}
\begin{itemize}
\item $\Longleftarrow$

Partiendo de que $a \to b \iff \neg b \to \neg a$, vamos a demostrar que si $n$ no es primo, entonces no es cierto que $(n-1)!=-1 \mod n$.

Supongamos que $n$ no es primo. Entonces tenemos diferentes casos:
\begin{enumerate}
\item $n=n_1n_2$ con $1<n_1<n_2<n$

En este caso tenemos
\[(n-1)!=1\cdot 2 ... \cdot n_1 \cdot ... n_2 \cdot ... = 0 \mod n\]

\item $n=p^2$ siendo $p$ primo impar

En este caso tenemos
\[(n-1)! = 1 \cdot 2 ... \cdot p ... \cdot 2p ... \cdot (n-1) = 0 \mod n\]

\item $n=4=2^2$

En esta ocasión
\[3! = 6 = 2 \mod 4\]
\end{enumerate}

\item $\implies$

Sea $n$ primo, dado un elemento $a\in \ent_n$ tenemos:
\[a=a^{-1} \mod n \iff a^2 = 1 \mod n \iff a^2-1 = m \mod n\]

y la ecuación $a^2-1=0\mod n$ tiene, como mucho, dos raíces en un cuerpo.

En este caso, estas raíces serían $a=1$, $a=(n-1)$.

Por tanto, puesto que estamos en $\ent_p$, todo elemento tiene su inverso por lo que al escribir $(n-1)!$ estamos multiplicando todos los elementos del cuerpo, por lo que cada factor estará multiplicado por su inverso y se cancelará, excepto $1$ y $(n-1)$ que son sus propios inversos.

Así nos queda:
\[(n-1)! \mod n = 1 \cdot (n-1) \mod n = n-1 \mod n = -1 \mod n\]
\end{itemize}
\end{proof}

Este algoritmo, aunque parece ``muy bonito'' no vale para nada, puesto que no factoriza (por tanto aporta menos información que la criba de Eratóstenes) y requiere calcular factoriales, lo que es muy costoso.

Sin embargo, en 2002 se desarrolló un algoritmo (AKS) capaz de comprobar si un número es primo en tiempo $O\left( \log^{12}(\log\log(n)^2\right)= \tilde{O}(\log^7(n))$.

Años más tarde, en 2006, se desarrolló un algoritmo capaz de hacer esto mismo en tiempo $\tilde{O}\left( \log^6(n)\right)$

Sin embargo, con el tamaño de números que trabajamos, el algoritmo AKS no es rentable, puesto que las constantes asociadas a la $O$ son demasiado elevadas.

\subsection{Tests probabilísticos}

Estos tests se basan en la afirmación filosófica de E. GJorel, que dice: ``Un suceso cuya probabilidad es de $10^{-50}$ o bien es imposible o nunca será observado''

Así estos test buscarán números que serán primos con probabilidad $1-10^{-50}$. Para esto necesitamos observar alguna propiedad de los números primos que sea fácilmente comprobable, para lo que emplearemos el pequeño teorema de Fermat.

\begin{theorem}[Pequeño teorema de Fermat]
Existen dos versiones de este teorema:
\begin{enumerate}
\item
\[p \text{ primo } \implies \forall a \in \ent_p \ a^p=a \mod p\]

\item
\[p \text{ primo } \implies \forall a \in \ent_p \ a^{p-1}=1 \mod p\]
\end{enumerate}
\label{thm:FermatsLittleTh}
\end{theorem}

\newpage
\begin{defn}[Número pseudoprimo en base $a$]
Decimos que un número $p$ es pseudoprimo en base $a$ si se cumple que
\[a^{p-1}=1 \mod p \]
\end{defn}

Una vez visto esto podemos ver el lema en que se basan los test probabilísticos.
\begin{lemma}
Si existe $b_0$ coprimo con $n$ tal que $b_0^{n-1}\neq 1 \mod n$ (es decir, tenemos un número que no satisface el pequeño teorema de Fermat, por lo que $n$ no sería primo), entonces $b^{n-1} \neq 1 \mod n$ para, \textbf{al menos la mitad} de los $b$ posibles en $[1,n-1]$ coprimos con $n$.
\end{lemma}

Con este lemma, si $n$ no es primo, entonces $b^{n-1} \neq 1 \mod n$ con probabilidad mayor que $\frac{1}{2}$. Por tanto, si tomo al azar 200 valores $b$ y todos pasan el test (todos satisfacen el teorema de Fermat), la probabilidad de que el número sea compuesto es menor que $\left( \frac{1}{2}\right)^{200} \approx \left(\frac{1}{10}\right)^{60}$

Veamos la demostración del lemma
\begin{proof}
Consideramos el conjunto de valores $\{a_1,...1_r\} \subset \left(\ent_n\right)^*$ tales que
\[a_i^{n-1} = 1 \mod n\]

Entonces $n$ no es pseudoprimo en la bases $\{b_0a_1,...,b_0a_r\}$ ya que
\[(b_0a_i)^{n-1} = b_0^{n-1}a_i^{n-1} = b^{n-1} \neq 1 \mod n\]

Es decir, \textbf{si existe una base en la que no es pesudoprimo} entonces por cada base en la que sea pseudo primo tenemos otra en la que no lo es con lo que queda claro que al menos en la mitad de los casos no será pesudoprimo.
\end{proof}

Vamos a ver ahora cómo podríamos construir un test probabilístico de primalidad de manera incremental.

Empezamos considerando una versión \textbf{beta} del test.
\begin{enumerate}
\item Considero un número al azar $b \in [2,n-1]\cap \ent$
\item Compruebo si $b$ es coprimo con $n$ que, como ya vimos, es una tarea \textbf{fácil}
\begin{itemize}
\item Si $b$ no es coprimo con $n$ entonces $n$ no es primo, obviamente
\item Si $b$ es coprimo con $n$ calculo $b^{n-1} \mod n$ que también vimos que era una tarea \textbf{fácil}
\end{itemize}
\item Compruebo si $b^{n-1}$ es coprimo con $n$.
\begin{itemize}
\item Si $b^{n-1}\neq 1 \mod n$ entonces $n$ no es primo.
\item Si $b^{n-1} = 1 \mod n$ entonces $n$ podría ser primo. Volvemos al paso 1.
\end{itemize}
\end{enumerate}

Tras repetir el procedimiento 200 veces, si resulta que $n$ es pseudoprimo en las 200 bases, concluimos que $n$ es, \textbf{probabilísticamente primo}

Sin embargo, esta versión tiene un problema: \textbf{¿Qué ocurre si $n$ es pseudoprimo en todas las bases?}.

\subsection{Números de Carmichael}

\begin{defn}[Número de Carmichael]
Un \textbf{número de Carmichael} es un número compuesto tal que $b^{n-1}=1 \mod n$ para todo $b$ coprimo con $n$.
\end{defn}

Lo que nos interesa es comprobar si existen números de Carmichael (evidentemente existen, pues de lo contrario no hablaríamos de ellos), ver cuántos son y si existe alguna forma útil de caracterizarlos.

\begin{theorem}[Teorema de Korset (1899)]
Sea $n \in \ent$ son equivalentes:
\begin{enumerate}
\item \[b^n=b \mod n \ \forall b \in \ent_n\]
\item $n$ es un número de Carmichael, es decir:
\[b^{n-1} = 1 \mod n \ \forall b \in \ent_n, \ t.q. \ (b,n)=1 \]
\item n es libre de cuadrados, es decir:
\[a^2|n \implies a = \pm 1 \text{ si p es primo } \neg p^2 | n\]
O lo que es lo mismo, si $p$ es primo y $p$ divide a $n$ entonces $p-1$ divide a $n-1$
\end{enumerate}
\end{theorem}

\begin{proof}
\begin{itemize}
\item $1 \implies 2$

Sabemos que $(b,n)=1 \iff b \in\left(\ent_n\right)^*$. Por tanto, podemos multiplicar a ambos lados de la ecuación por $b^{-1}$ obteniendo:
\[b^n = b \mod n \implies b^{n-1}=1 \mod b\]

\item $3 \implies 1$

Sabiendo que $n$ es libre de cuadrados tenemos que:
\[b^n = b \mod n \iff b^n = b \mod p \ \forall p|n\]

\begin{proof}
Para cualquier primo $p$ que divida a $n$ tenemos
\[b^n = b \mod p \iff 0^n=0 \mod p \]

Por otro lado, puesto que estamos considerando que $n$ es libre de cuadrados, sabemos que $p-1$ divide a $n-1$ por lo que podemos escribir:
\[(b,p)=1 \implies b^n=bb^{n-1} = bb^{(p-1)k}=b\left(b^{p-1}\right)^k=b\mod p\]
\end{proof}

\item $2 \implies 3$

\begin{itemize}
\item[a)] Supongamos que $n$ no es libre de cuadrados, es decir, supongamos que existe un primo $p$ tal que $n=p^rn'$ siendo $r\geq2$ y $(n',p)=1$.

En estas condiciones tenemos
\[\left| \left(\ent_n\right)^*\right| = \varphi(p^r)=p^{r-1}(p-1) \implies p|\varphi(p^r)\]
donde la última implicación es válida puesto que $r\geq2$

Sabemos, por el teorema de Cauchy, que
\[\exists a \in \left(\ent_{p^r}\right)^* \tq ord(a)=p \text{ es decir } \left\{\begin{array}{l}a\neq 1 \mod p^r \\ a^p=1 \mod p^r \end{array}\right.\]

Por el teorema chino del resto, sabemos que existe un elemento $b \in \ent_n$ tal que:
\[\left\{\begin{array}{l}b=a\mod p^r \\ b=1 \mod n' \end{array}\right.\]

Por hipótesis tenemos que $b^{n-1}=1 \mod n$, por tanto:
\[b^{n-1} = 1 \mod p^r \implies a^{n-1} = 1 \mod p^r \implies ord(a) =p|n-1\]

Pero entonces tenemos que $p$ divide a $n$ y también a $n-1$ con lo que tenemos que $p$ divide a $1$, que es una contradicción.

\item[b)]
Tras llegar a la contradicción del apartado anterior sabemos que $n=p_1...p_r$ con $p_i \neq p_j$. Sea $p | n$ tenemos que $n=pn'$ con $(n',p)=1$.

Sabemos entonces que $(\ent_p)^*=<a>$ es cíclico.

Por el teorema chino del resto sabemos que
\[\exists b \in (\ent_n)^* \tq \left\{\begin{array}{l}b=a\mod p \\ b=1 \mod n' \end{array}\right.\]

Si $n-1 = q(p-1)+r$ con $0 < r < p-1$ tenemos que $b^{n-1}=1 \mod n$ lo que implica que
\[b^{n-1}=1 \mod p \implies a^{n-1} = 1 \mod p \implies (a^{p-1})^qa^r = 1 \mod p \implies\]
\[\implies a^r = 1 \mod p \implies ord(a) = (p-1)|r<p-1 \implies r=0\implies p-1 | n-1\]
\end{itemize}
\end{itemize}
\end{proof}

Korset, con su teorema, dió un criterio para detectar números que fuesen primos en todas las bases. Sin embargo, nunca encontró un número que cumpliese estas propiedades y no fué hasta el año 1910 cuando Carmichael observó que 561 es un número de Carmichael.

Otros ejemplos de números de Carmichael son 1105, 252601, 62745 y 9746347772161.

\begin{prop}
Todo número de Carmichael tiene, al menos, 3 factores primos distintos.
\end{prop}
\begin{proof}
Supongamos que tenemos un número de Carmichael $n=p\cdot q$, con $p<q$, que sólo tiene dos factores primos.

Entonces, por el teorema de Korset, sabemos que
\[q-1 | n-1\]
pero sabemos que
\[n-1 =pq-1=p(q-1)+p-1\]
lo que implicaría que $q-1|p-1$ pero esto es imposible puesto que $p<q$.
\end{proof}

Si los números de Carmichael fuesen finitos podríamos hacer una lista con todos ellos de forma que al tratar de encontrar nuevos números primos podamos descartar todos los de la lista.

No obstante, el \textbf{resultado de Alford-Granville-Pomeanle} obtenido en 1994 demuestra que
\[|\{c \tq \text{ c es de Carmichael y } c < x\}| > x^{2/7}\]
de donde se extiende que cuando $x$ tiende a infinito, tenemos infinitos números de Carmichael.

Con esto llegamos al gran problema del algoritmo probabilístico de factorización que hemos definido. El test considera como primos a los números de Carmichael (que no lo son) y no tenemos forma de comprobar si el número obtenido es o no de Carmichael, salvo aplicando el teorema de Korset, pero esto nos requiere factorizar por lo que es inviable.

\begin{example}
\textbf{Vamos a comprobar si 21 es un número primo}.

Empezamos a aplicar nuestro algoritmo. Al azar tomamos $b=13$, que es coprimo con 21 y el test nos dice que:
\[13^{20} \mod 21 = (13^2)^{10} \mod 21 =1^{10} \mod 21 =1 \mod 21\]
con lo que hemos pasado el test.
\end{example}

Este ejemplo nos sirve para ilustrar una idea importante, \textcolor{blue}{que apareció en los parciales del año pasado\footnote{Estos parciales se encuentran resueltos al final de estos apuntes}}.

\begin{mdframed}
Si $a^2 = 1 \mod n$ sabemos que $n$ no es primo.
\end{mdframed}

Si $n$ fuese primo tendríamos que $\ent_{n}$ sería un cuerpo y la ecuación $x^2-1 = 0 $ tendría exactamente dos soluciones en $\ent_{n}$ que son $x=\pm 1 = 1, n-1$.

Y es en esta idea en lo que se basará nuestro nuevo test de primalidad.

\subsection{Test de primalidad de Miller-Marsin (1974)}
\begin{defn}[Pseudoprimo fuerte en base $b$]
Sea $n \in \ent$ impar con $n>9$ y $n-1=r2^s$ siendo $r$ impar y $s \geq 1$, diremos que $n$ es \textbf{pseudoprimo fuerte} en base $b$ si $(n,b)=1$ y o bien:
\[b^r=1 \mod n\]
o bien para algún $0 \leq i \leq s-1$ se cumple
\[b^{r2^i}=-1\mod n\]
\label{defn:MillerMarsinTest}
\end{defn}

La idea de esta definición es que si $n$ es primo, tenemos:
\[(b^r)^{2^s} \mod n=b^{n-1} = 1 \mod n\]
y por tanto
\[(b^r)^{2^{s-1}} \mod n = \pm 1 \mod n\]
puesto que tiene que ser un número que, en el cuerpo $\ent_n$ tenga cuadrado igual a 1, por lo que sólo podría ser $\pm 1$

Si $(b^r)^{2^{s-1}}=-1$ satisface la condición inicial por lo que podemos decir que $n$ es pseudoprimo fuerte en base $b$. Si fuese $1$, repetimos el procedimiento.

Es decir, si $n$ es primo, entonces será pseudoprimo fuerte en toda base $b$ tal que $(b,n)=1$.
Pero esto no es más que una forma de caracterizar primos pero nosotros necesitamos justo lo contrario, a fin de poder determinar cuándo un número es primo o no.

\begin{theorem}
Si $n$ no es primo entonces no es pseudoprimo fuerte para al menos $\frac{1}{4}$ de los $b\in [1,n]$. Es decir, ``no hay números de Carmichael fuertes"
\end{theorem}

A partir de este teorema podemos desarrollar el \concept{Test de Miller-Marsin} de primalidad:
\begin{enumerate}
\item Elijo $b\in[1,n]$ de forma aleatoria y calculo $(b,n)$. Si el máximo común divisor es distinto de $1$, entonces hemos factorizado $n$ y por tanto $n$ no sería primo.
\item Si $(b,n)=1$ miro si $n$ es peudoprimo fuerte en base $b$. Si no lo es entonces $n$ no es primo
\item Si $n$ es pseudoprimo fuerte en base $b$ vuelvo al paso 1.
\end{enumerate}

Si repetimos el test $x$ veces, la probabilidad de que pasemos el tets todas esas veces es de $\left(\frac{3}{4}\right)^x$. Es decir, la probabilidad de decir que un número $n$ es primo sin serlo es de $\left(\frac{3}{4}\right)^x$.

\begin{example}
	veamos si el 21 pasa o no el \textbf{Test de Miller-Robin}:

	\[n - 1 = 20 = r · 2^s = 5 · 2^2\]
	Saco al azar $b=13$ y miramos si $b^r = 1 \mod n$:
	\[13^5 = 13 \mod 21\]
	Viendo que no se cumple probamos con $b^{r·2^i}=-1 \mod n$ y $0≤i≤s-1$. Para $i=1$ no se cumple:
	\[(13^5)^2 = 1 \mod 21\]
	Y para $i=0$ hemos visto que tampoco se cumple, de modo que 21 no es pseudoprimo fuerte, y por tanto tampoco es primo.\\[3em]
\end{example}


Podemos ver que este algoritmo nos requiere calcular potencias de la forma $b^r \mod n$, que tiene un coste de $O(\log^4n)$. Tras esto tendremos que elevar al cuadrado estos valores de forma recurrente, necesitando $O(\log n)$ operactiones.

Esta última potencia deberá calcularse unas $s=O(\log n)$ veces por lo que en total necesitaremos:
\[d\cdot O(\log^5 n)\text{ operaciones}\]
donde $d$ es lo bastante grande para que $\left(\frac{3}{4}\right)^d < \frac{1}{10^{50}}$

Al margen de este algoritmo existe otra idea: si la \textbf{hipótesis de Riemann generalizada} es cierta tenemos que
\[n \text{ no es primo } \implies \exists b \in [1,n] \text{ t.q.} n \text{ no es pesudo primo fuerte en base } b\]

Entonces puedo probar todas las bases $b=2,3,4,...,2\log^2 n$.

No obstante este procedimiento sería muy largo y estaría apoyándose en una hipótesis aún sin demostrar.

\begin{example}\textbf{Algoritmo determinista}

Sabemos que el menor pseudoprimo fuerte en bases $2$ y $3$ es el número 1373663 = 850x1657

Por tanto, si tenemos un número menor que un millón podemos comprobar si es primo estudiando si es pesudoprimo fuerte en bases $2$ y $3$. De esta forma tendríamos una forma determinista y no probabilística de ver si un número es primo o no.

Basándonos en esta idea podemos mejorar el algoritmo con los siguientes datos:
\begin{enumerate}
\item El menor número ṕseudoprimo en las bases $2,3$ y $5$ es el número 2536001 = 2251x11231.

\item El único número pseudoprimo en las bases $2,3,5,7$ menor que $25 \cdot 10^9$ es 3215031751 = 151z751x28351

\item El menor número pesudoprimo en base $2,3,5,7,11,13,17,19$ y $23$ es del orden de $3 \cdot 10^{18}$

\end{enumerate}
\end{example}

\begin{theorem}[Jim Sinclaire (2011)]
Empleando la base de 7 elementos:
\[2,325,9375,28178,450775,9780504,1795265022\]
podemos decidr con certeza si un número $n<2^{64}\approx 1.89\cdot 10^{19}$ es primo
\end{theorem}

\section{Algoritmos de Factorización}
El mejor algoritmo decimal (que no usa propiedades del número) para factorizar números $n>2^{100}$ es la \concept{criva en cuerpos} de números (\textbf{General Number Field Sieve}), que emplea un tiempo $O\left(e^{64/9 (\log n)^{1/3}(\log\log n)^{2/3}} \right)$.

En general, a la hora de factorizar un número los pasos a seguir son:
\begin{enumerate}
\item Dividr entre los primos pequeños.

Recordemos que la división no es un problema puesto que es una operación fácil. El problema surgía cuando teníamos que dividir entre diferentes números hasta $\sqrt{n}$.

El concepto de primo pequeño es un poco ambigüo, puesto que depende de la capacidad de la máquina pero suelen considerarse los primos hasta $5\cdot 10^{5}$ (hay 41538)

\item Método ρ de Polland o $ρ-1$ de Polland, que nos permite factorizar hasta números de 15-20 cifras.

\item Método de Curvas elípticas o curvas cuadráticas, que funciona bien para números de hasta 25 cifras.

\item Criba de cuerpos de números $(NFS)$, que resulta especialmente útil si $n=b^a\pm 1$ con $b$ un número pequeño como $b=(2,3)$ y $a$ un número grande.

\begin{example}
\[F_9 =2^{2^9}+1 = p_7\cdot p_{49} \cdot p_{99}\]
donde $p_i$ representa un número de $i$ cifras.
\end{example}

\item $GNFS$
\end{enumerate}

\obs Si un número tiene propiedades especiales puede haber métodos específicos más rápidos para factorizar.

\begin{prop}
Si un número primo $p$ divide a un número de la forma $b^n-1$ entonces $p|b^d-1$ siendo $d$ un divisor de $n$ o bien $p=1 \mod n$.

En caso de darse la segunda opción, si $p \neq 2$ y $n$ es impar entonces $p=1 \mod 2n$
\label{prop:pbn-1}
\end{prop}
\begin{proof}
Sabemos que
\[p|b^n-1 \iff b^n =1 \mod p\]

Ahora debemos observar que si $b \in (\ent_p)^*$ y $d=ord(b)$ entonces
\[b^d=1 \mod p \implies p|b^d-1 \text{ y } d|n\]

Además, si $d=n$ entonces $n||(\ent_p)^*|=p-1$, puesto que el orden de cualquier elemento divide al orden del grupo. Por tanto nos queda que $p=1\mod n$

Si $p\neq 2$, entonces $p-1$ es par. Si $n$ es impar entonces
\[n|p-1 \implies 2n|p-1 \implies p=1 \mod 2n\]

donde la primera implicación se deduce del hecho de que si $n$ divide a un par y $n$ es impar, sabemos que el $2$ también es un divisor y por tanto $2n$ es divisor.

\end{proof}

\begin{example}
Supongamos que queremos factorizar $2^{11}-1=2047$.

Si tenemos un primo $p$ que
\[p|2^{11}-1\implies \left\{ \begin{array}{l}p|2^1-1 \implies \text{ imposible}\\ p =1 \mod 22 \end{array}\right.\]

Si $2047$ no es primo entonces tiene factores menores que $\sqrt{2045}\approx 45$. Puesto que es impar, también sabemos que $p=1 \mod 22$. Combinando ambas informaciones tenemos que:
\[p \in \{23 \}\]
\[2047 = 23 \times 89\]
\end{example}

\begin{example}
Tratamos de factorizar $2^{13}-1=8191$. Supongamos que tenemos el factor primo $p$. Entonces:

\[p|2^{13}-1\implies \left\{ \begin{array}{l}p|2^1-1 \implies \text{ imposible}\\ p =1 \mod 13 \end{array}\right.\]

Por ser $13$ un número impar tenemos que $p = 1 \mod 26$. Teniendo en cuenta que debe haber un factor primo menor que $\sqrt{8191}\approx 90$, tenemos que:
\[p \in \{27,53,79\}\]

En este caso procedemos a comprobarlo y vemos que ningún primo en el intervalo es un factor, por lo que podemos concluir que $8191$ es primo.
\end{example}

\begin{example}
Vamos a tratar de factorizar $3^{12}-1 = 531440$.

Para todo factor primo tenemos que:
\[p|3^{12}-1 \implies \left\{ \begin{array}{l}p|(3^1-1) \implies p|2 \implies p=2\\
p|3^2-1 \implies p|8 \implies \text{no aporta nada nuevo}\\
p|3^3-1 \implies p |26 \implies  p=13\\
p|3^4-1 \implies p |(3^2-1)(3^2+1) \implies p|3^2+1 \implies p|10 \implies p=5\\
p|3^6-1 \implies p |(3^3-1)(3^3+1) \implies p|3^3+1 \implies p|28 \implies p=7\\
p =1 \mod 12 \end{array}\right.\]

Donde en todos los apartados hemos desarrollado sólo aquellos datos que aportaban información nueva.

Vamos a empezar a probar con las posibilidades que hemos obtenido hasta ahora. Así obtenemos que:
\[3^{12}-1 = 2^4\times 13 \times 5 \times 7 \times 73\]

\end{example}

\begin{example}
Vamos a factorizar ahora el número $2^{35}-1=34359738367$

Para todo factor primo tenemos:
\[p|2^{35}-1 \implies \left\{ \begin{array}{l}p|(2^1-1)\implies \text{ imposible}\\
p|3^5-1 \implies p|242 \implies p=2 \Or p=11\\
p|3^7-1 \implies p |2186 \implies  p|1093\\
p =1 \mod 70 \end{array}\right.\]

Con los datos que tenemos podemos empezar a probar con lo que obtenemos:
\[34359738367 = 71 \times 483939977\]
\end{example}

\subsection{Idea detrás de los métodos modernos de criva}

Recordemos que la idea de Fermat era considerar que siendo $n$ un número impar (si no es impar sabemos que dos es factor y volvemos a esta situación) podemos escribir:
\[n = x^2-y^2 = (x-y)(x+y)\]

Una vez sabemos esto podemos emplear un algoritmo iterativo sobre la función:
\[Q(x) = x^2 - n\]

para aplicar el algoritmo empezamos probando con $x_0 = \lceil\sqrt{n}\rceil$, es decir, el techo de $\sqrt{n}$. En cada iteración comprobamos si $Q(x_i)$ es un cuadrado perfecto y, en caso de no serlo, pasamos a $x_{i+1}=x_i+1$.

Este algoritmo siempre nos permite factorizar el número $n$ cualquiera. No obstante, es posible que el tiempo necesario para hacerlo sea demasiado grande.

\subsubsection{Idea de Kraichik~1922}
La idea es que no necesitamos que $x^2-y^2 =n$ sino que basta con tener:
\[x^2 = y^2 \mod n \equiv yx = \pm y \mod n\]
puesto que
\[x^2 = y^2 \mod n \iff n|(x^2-y^2) \equiv n|(x-y)(x+y)\]

Si tenemos que $n=p\cdot q$ tenemos diferentes opciones:
\begin{enumerate}
\item
\[n|(x-y) \implies x=y \mod n\]

Este caso no tiene sentido puesto que entonces $n=0$

\item
\[n|(x+y) \implies x=-y \mod n\]

Esta situación tampoco tiene sentido ya que, igual que en el caso anterior, tendríamos $n=0$

\item
\[p|(x+y), q|(x-y)\]
\end{enumerate}

Así, lo que hacemos es calcular el máximo común divisor de $x-y$ y de $n$ para lo que tenemos diferentes opciones:
\[m.c.d.(x-y,n) = \left\{ \begin{array}{ll}
             n &   \text{Estamos en el caso 1} \\
             1 &   \text{Estamos en el caso 2}\\
             p,q & \text{Estamos en el caso 3 y por tanto hemos factorizado}
             \end{array}\right.\]


Vamos a ver ahora un algoritmo de factorización que se apoya en esta idea:
\begin{mdframed}
\begin{enumerate}
\item Fijamos un conjunto pequeño de primos (pequeños), que denotamos como
\[\algb{P} = \{p_1,...,p_N\}\]

\item Calculamos $x_0=\lceil\sqrt{n}\rceil$ y comenzamos a iterar con la función $Q(x)=x^2-n$.

\item Calculamos $Q(x_0)$ y factorizamos en base $\algb{P}$, es decir, escribimos:
\[x_1 = Q(x_0) = \prod_i p_i^{a_{i0}}\]

\item Calculamos $Q(x_1)$ y factorizamos en base $\algb{P}$, es decir, escribimos:
\[Q(x_1) = \prod_i p_i^{a_{i1}}\]

\item Seguimos factorizando los $Q(x_j=x_0+j)$ (aunque podemos toparnos con alguno que no factorice en $\algb{P}$)
\end{enumerate}

Al final tendremos todas las factorizaciones posibles en $\algb{P}$ para algunos $x_0,...,x_r$.

Estos números que factorizan en $\algb{P}$ se denominan \textbf{$\algb{P}$-suaves}

\obs Una gran ventaja hasta aquí es que estas factorizaciones pueden hacerse en paralelo, teniendo muchos ordenadores trabajando de forma simultánea.

Una vez tenemos esto, multiplicamos todos estos números en módulo $n$ obteniendo:
\[\prod_{j=0}^rQ(x_j)=\prod_{j=0}^rx_j^2 \mod n\]
puesto que el término $n$ de cada $Q(x_i)$ nos desaparece al hacer el módulo.

Queremos que el número obtenido a la derecha de la igualdad anterior sea un cuadrado perfecto ya que entonces
\[\prod_j\prod_ip_i^{a_{ij}} = y^2 \implies x^2 = y^2 \mod n\]

Pero, para que esto sea posible, necesitamos que el exponente de cada $p_i$ sea un número par, es decir, necesitamos que
\[\sum_ja_{ij} \text{ sea par para } i={1,…,N}\]

Pues:
\[\prod_j\prod_ip_i^{a_{ij}} = \prod_i p_i^{\sum_ja_{ij}}\]

En lugar de mirar al producto de los $Q(x_i)$ que se factorizan podemos mirar el producto de todos los $Q(x_i)$ que hemos calculado, elevándolos a 1 o 0 según corresponda.
\[\prod_{j=1}^r Q(x_j)^{δ_j} (\text{con } δ_j = 1,0)  = \prod_ip_i^{\sum_jδ_ja_{ij}} \mod n\]

Finalmente, tenemos que resolver el siguiente sistema de $n$ incógnitas:
\[\left\{ \begin{array}{l}
\sum_j^ra_{ij}δ_j = 0 \mod 2 \\
\text{ para } i=1,...,N
\end{array}\right.\]
\end{mdframed}

\begin{example}
Vamos a factorizar el número $n=2041$ tomando como conjunto de primos:
\[\algb{P} = \{ 2,3,5,7\}\]

Procedemos a aplicar el algoritmo con lo que obtenemos:
\[\begin{array}{l}
\lceil\sqrt{2047}\rceil = 46=x_0 \\
x_1 = Q(46)=168=2^3\cdot 3 \cdot 7 \\
x_2 = Q(47) = 263 \text{ no es factorizable en }\algb{P}\\
x_3 = Q(48) = 360 = 2^3\cdot 3^2 \cdot 5 \\
x_4 = Q(49) = 459 \text{ no es factorizable en }\algb{P}\\
x_5 = Q(50) = 560 = 2^4\cdot 5 \cdot 7
\end{array}\]

Ahora multiplicamos módulo $n$ los números que factorizan en $\algb{P}$, con lo que obtenemos:
\[(\underbrace{46\cdot 47 \cdot 49 \cdot 50}_{x})^2 = (\underbrace{2^5\cdot 3^2 \cdot 5^2 \cdot 7}_y)^2\]

Lo siguiente es mirar $mcd(x-y, n)=mcd(46\cdot 47 \cdot 49 \cdot 50 - 2^5\cdot 3^2 \cdot 5^2 \cdot 7, 2014)$, y en caso de que sea $2041$ o $1$ tendríamos $x=\pm y \mod n$ y no se satisfaría el criterio de Kraichick (no nos valen $x$ e $y$ para decir que $n=2041=x^2-y^2$).
\end{example}
