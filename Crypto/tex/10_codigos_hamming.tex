\chapter{Códigos de Hamming}

\section{Códigos de Hamming binarios}
Trabajando con $q=2$ y $d=3$ vamos a trabajar con códigos Ham(r,2). Fijando $r$ vamos a buscar códigos con el mayor $n$ posible.

Para garantizar que la distancia mínima sea mayor o igual que 3, simplemente necesitamos:
\[\forall i \ H_i \neq 0 \ \ \ \forall i,j \ H_i \neq H_j\]
Es decir, basta con tomar cualquier conjunto $\{H_1,...,H_n\}\in F_2^r-\{0\}$ con $n$ vectores distintos.

Si queremos que el $n$ sea el mayor posible, tendremos un total de $n=2^r-1$.

Una vez tenemos $n$, sabiendo que la matriz controladora de paridad, $H$, tendrá $r$ filas y $n$ columnas (y sabiendo que la distancia del código es 3) podemos concluir que nos encontramos ante un $(n=q^r-1,q^{n-r},3)$-código $q$-ario.

\begin{example}
Si tenemos un código $Ham(4,2)$ con distancia mínima $d=3$, estamos ante un $(15,2^{11},3)$-código binario.

\end{example}

En general, si tenemos un código Ham(r,q) con distancia mínima $d=3$ queremos tener $q=p^s$ sobre $F_q$. Pedimos que $q$ sea potencia de un primo para poder trabajar con álgebra de primero puesto que de no ser $F_q$ un cuerpo el concepto de linealmente independiente quedaría bastante más complejo.

Para tener distancia mínima $d \geq 3$ necesitamos que ninguna columna de $H$ sea nula ni múltiplo de otra.

En el cuerpo $F_q^r-\{0\}$ tenemos un total de $q^r-1$ posibles columnas \footnote{Ya estamos descontando la columna nula}. Sin embargo, por cada columna $H_i$ que fijamos, nos estamos eliminando $q-1$ opciones, ya que el resto de columnas no podrán ser múltiplo de la primera.

Por tanto, tenemos como máximo $n=\frac{q^r-1}{q-1}$.

Lo más fácil es exigir que la coordenada más alta sea distinta de 0, por ejemplo 1.

\begin{example}
Si queremos encontrar el código Ham(2,3) óptimo con distancia mínima $d=3$ tenemos $n=\frac{3^2-1}{3-1}=4$. En concreto tenemos
\[H=\left( \begin{array}{cccc}
0 & 1 & 1 & 1 \\
1 & 0 & 1 & 2
\end{array}\right)\]
\end{example}

\begin{prop}
Sean $q,r \geq 2$ enteros (podríamos tomar también $r=1$ pero se trataría de códigos muy triviales), se cumple que:
\[n = \frac{q^r-1}{q-1}, \ M=q^{n-r} \implies M(1+n(q-1))=q^n\]
\end{prop}
\begin{proof}
\[M(1+n(q-1))=q^n \iff 1+n(q-1) = q^r \iff n(q-1)=q^r-n \iff n=\frac{q^r-1}{q-1}\]
\end{proof}

\begin{corol}
Si $q$ es potencia de un primo, entonces el código $Ham(r,q)$ es perfecto con $d=3$
\end{corol}

\obs Que los parámetros satisfagan la ecuación de Hamming no garantiza que el código en cuestión exista.

\begin{example}
Podemos ver que no existe un $(7,6^5,3)$-código 6-ario.

Sin embargo, los parámetros de este código son ``de Hamming'' pues satisfacen
\[\frac{6^2-1}{6-1}=7=n=r+k \text{ siendo } r=2 \ k=5 \text{ y } d=3\]

\end{example}

Si se buscan soluciones a las ecuaciones de Hamming, que recordemos son de la forma:
\[M \times \sum_{i=0}^t {n \choose i} (q-1)^i=q^n\]
con $n,t \leq 1000$ y $q \leq 100$ salen los códigos de Hamming que hemos visto hasta ahora (cuando $t=1$), los triviales y 3 soluciones más que son:

\begin{enumerate}
\item $(23,2^{12},7)$-código binario, conocido como \concept{Código de Golay binario}
\item $(90,2^{78},5)$-código binario.
\item $(11,3^6,5)$-código ternario, conocido como \concept{Código de Golay ternario}
\end{enumerate}

En concreto, para los casos 1 y 3 si que existe un código (el de Golay) con esas características mas no así con el segundo caso.

\begin{theorem}
No existe un $(90,2^{78},5)$-código binario.
\end{theorem}

\begin{theorem}
Si $q$ es potencia de un primo, entonces todo código perfecto no trivial tiene los parámetros de un Hamming o un Golay
\end{theorem}

Con esto tenemos que, definiéndolos sobre cuerpos, existen códigos binarios para corregir 1 o 3 errores (Hamming y Golay binario respectivamente) y códigos ternarios para corregir dos errores (Golay ternario).

Veamos cómo trabaja un algoritmo de corrección para un \textbf{código perfecto} con $d=2t+1$.

\begin{enumerate}
\item Recibo $y\in F_q^n$
\item Busco $x_0 \in \algb{C}$ tal que $d(x_0,y)=\min\{d(x,y)\tq x \in \algb{C}\}$
\item Leo $x_0$
\end{enumerate}

Podemos observar que este código nunca pita sino que siempre lee algo. Esto funciona porque estamos trabajando con un código perfecto que se caracteriza por dividir el espacio \textbf{total} en el que se mueven las palabras en bolas disjuntas.

Puesto que todo el espacio está divido, siempre encontraremos algún $x_0$ y, puesto que las bolas son disjuntas, no tendremos conflictos con el $x_0$ que será único.

Veamos ahora una versión de este algoritmo que utiliza los síndromes. Supongamos que tenemos un código $Ham(r,q)$ con $d=3 \implies t=1$. Los pasos a seguir son:

\begin{enumerate}
\item Recibo $y$
\item Calculo $S=S(y)$
\item Si $S=0$ leo $y$
\item Si $S\neq 0$ busco $i,λ$ tales que $S=λH_i$ y leo $y-λH_i$.
\end{enumerate}

Este algoritmo se apoya en la siguiente propiedad:
\[\forall y \in F_q^n\setminus \{0\} \ \ \exists ! i, \ \exists ! λ\in F_q\setminus\{0\} \text{ t.q } y=λH_i\]

\obs Sea $S=(s_1,...,s_r)$ con $s_1=s_2=...=s_e=0$ y $s_{e+1} \neq 0$ entonces
\[S = s_{e+1}\cdot \left( \begin{array}{c} 0 \\ \vdots \\ 0 \\ 1 \\ \frac{s_{e+2}}{s_{e+1}} \\ \vdots \\ \frac{s_{r}}{e_{e+1}}\end{array}\right)=s_{e+1}\cdot H_i\]

\begin{example}
Supongamos que estamos empleando un código $Ham(3,3)$ y que recibimos el mensaje $y=0112200000000$, ¿Qué leemos?.

Lo primero que debemos hacer es construir la matriz controladora de paridad:
\[H = \left(\begin{array}{ccccccccccccc}
0 & 0 & 0 & 0 & 1 & 1 & 1 & 1 & 1 & 1 & 1 & 1 & 1\\
0 & 1 & 1 & 1 & 0 & 0 & 0 & 1 & 1 & 1 & 2 & 2 & 2\\
1 & 0 & 1 & 2 & 0 & 1 & 2 & 0 & 1 & 2 & 0 & 1 & 2
\end{array}\right)\]

Ahora debemos calcular el síndrome de la palabra recibida:
\[S(y) = \left(\begin{array}{c} 2 \\ 1 \\ 2\end{array} \right) = 2 \cdot  \left(\begin{array}{c} 1 \\ 2 \\ 1\end{array} \right) = 2H_{12}\]

Por tanto leeremos $y-2H_{12}=0112200000010$.
\end{example}

\begin{example}
Vamos a construir el código $Ham(2,4)$.

Lo primero que tenemos que hacer es construir un cuerpo de 4 elementos que, recordando algo de Galois, será $\{0,1,α,α+1\}$ siendo $α^2=α+1$.

En el caso del código que queremos construir tenemos que $r=2$, $n=\frac{4^r-1}{4-1}=5$. Por tanto la matriz $H$ tendrá dimensión $2 \times 5$. Vamos a construirla:
\[H = \left(\begin{array}{ccccc}
0 & 1 & 1 & 1 & 1\\
1 & 0 & 1 & α & α+1
\end{array} \right)\]

Supongamos ahora que leemos el mensaje $y=(0,0,1,α,0)$. Para corregirlo calculamos su síndrome obteniendo:
\[S(y) =  \left(\begin{array}{c} α+1 \\ α\end{array} \right)=(α+1) \left(\begin{array}{c} 1 \\ α+1\end{array} \right)\]

Por tano leemos $y-(α+1)(1,α+1) = (0,0,1,α,α+1)$
\end{example}

\begin{prop}
Sean $q,r \geq 2$ enteros y $q$ una potencia de un primo, entonces
\[A_q\left(\frac{q^r-1}{q-1},3\right) = q^{n-r}\]
\end{prop}

\begin{proof}
Sabemos de antemano que
\[A_q\left(\frac{q^r-1}{q-1},3\right) \leq q^{n-r}\]
por la cota de Hamming.

Por tanto tenemos que centrarnos en mostrar que se siempre se alcanza la igualdad.

Para ello basta con considerar un código $Ham(r,q)$ que, como ya hemos visto, es un código perfecto y alcanza la cota.

Por tanto queda claro que se da la igualdad.
\end{proof}

Ya tenemos que la cota de Hamming se alcanza pero ¿ocurre lo mismo con la cota de Singleton?. Vamos a verlo.

Recordemos que la cota de Singleton establece:
\[A_q(n,d) \leq q^{n-d+1}\]
Por tanto, lo que tenemos que hacer es ver si existe o no un código lineal con $r=d-1$. Los códigos que cumplan esta relación se denominan \concept{códigos Maximum Distance Separable}.

Un ejemplo de códigos de este tipo son los códigos BCH.

\section{Códigos BCH (Bose Ray-Chaudhury y Hocquenghem)}

Este tipo de códigos son empleados en los CDs para evitar que rallajos, que dañan una serie de bits consecutivos, no dejen el CD inutilizable.

Se dice que un CD es 75\% música y 25\% matemáticas.

Veamos cómo construir un código BCH.

Lo primero que debemos hacer es recordar que para generar el código nos basta con ser capaces de escribir la matriz controladora de paridad y esto se hacía atendiendo a la distancia mínima que queremos tener para el código.

En este caso estamos buscando un código con $r=d-1 \implies d=r+1$. Por tanto necesitamos encontrar $n$ columnas $H_i \in F_q^r$ tales que $r$ de ellas sean independientes.

Puesto que tenemos $r$ vectores de dimensión $r$, podemos comprobar si son o no independientes estudiando su determinante.

Por tanto, necesitamos encontrar vectores de tamaño $r$: $H_1,...,H_n$ tales que
\[\text{det}(H_{i_1}...H_{i_r}) \neq 0 \text{ para cualquier combinación de r vectores}\]

\begin{defn}[Determinante de Vandermonde]
Se trata de un determinante famoso que ya hemos visto en algebra lineal y que tiene la forma:
\[\left| \begin{array}{cccc}
1 & 1 & \cdots & 1 \\
x_1 & x_2 & \cdots & x_r \\
x_1^2 & x_2^2 & \cdots & x_r^2 \\
\vdots & \vdots & \ddots & \vdots \\
x_1^{r-1} & x_2^{r-1} & \cdots & x_r^{r-1} \\
\end{array}\right| = \prod_{1 \leq j < i \leq r}(x_i-x_j)\]
\end{defn}

Si trabajamos sobre un dominio tenemos:
\[\prod_{1 \leq j < i \leq r}(x_i-x_j) \neq 0 \iff x_i-x_j \neq 0 \ \forall i,j \iff x_i\neq x_j  \ \forall i,j\]
Por tanto, si en $F_q$ existen $n$ valores distintos podemos construir fácilmente una matriz de la forma del determinante de Vandermonde, lo que nos garantiza que tomando $r$ columnas cualesquiera de esta matriz, tendremos un determinante no nulo.

Es decir, nos quedaría la matriz:
\[H=\left( \begin{array}{cccc}
1 & 1 & \cdots & 1 \\
x_1 & x_2 & \cdots & x_n \\
x_1^2 & x_2^2 & \cdots & x_n^2 \\
\vdots & \vdots & \ddots & \vdots \\
x_1^{r-1} & x_2^{r-1} & \cdots & x_n^{r-1} \\
\end{array}\right)\]
que define un [$n,n-r,r+1$]-código.

\begin{example}
Veamos cuánto vale $A_{11}(7,5)$.

Para calcular este valor empezamos estableciendo la cota de Singleton, que nos garantiza:
\[A_{11}(7,5) \leq 11^3\]

Ahora tenemos que ver si se da la igualdad. Para que se de la igualdad debemos ser capaces de construir una matriz controladora de paridad compuesta por $4$ filas y $7$ columnas.

Recordando lo que hicimos anteriormente debemos encontrar $7$ elementos distintos en $F_{11}$ y ya lo tendremos. Esta tarea es bien sencilla y, como resultado, no permite obtener la matriz:
\[H=\left( \begin{array}{ccccccc}
1 & 1 & 1 & 1 & 1 & 1 & 1 \\
1 & 2 & 3 & 4 & 5 & 6 & 7 \\
1 & 2^2 & 3^2 & 4^2 & 5^2 & 6^2 & 7^2 \\
1 & 2^3 & 3^3 & 4^3 & 5^3 & 6^3 & 7^3
\end{array}\right)\]
\end{example}

\begin{theorem}
Si $q$ es potencia de un primo y $d \leq n \leq q$, se tiene que:
\[A_q(n,d)=q^{n-d+1}\]
\end{theorem}

\begin{theorem}
Si $q$ es potencia de un primo y $d \leq n \leq q+1$, se tiene que:
\[A_q(n,d)=q^{n-d+1}\]
\end{theorem}

\obs Ya no hay $q+1$ elementos distintos en $F_q$.

\begin{proof}
Con lo visto anteriormente es sencillo construir las primeras $q$ columnas de la matriz $H$, pero necesitamos $q+1$.

Basta con añadir una columna de 0s con un 1 al final. De esta forma, al calcular el determinante por adjuntos obtendremos el determinante de VanderMonde con lo que seguimos garantizando que $q+1$ columnas serán independientes.
\end{proof}

Los códigos BCH con $d \leq n \leq q$ se pueden decodificar usando técnicas inspiradas por Ramalujan. Veamos un ejemplo.

\begin{example}
Consideramos un código con matriz controladora:
\[H=\left( \begin{array}{ccccc}
1& 1 & 1 & \cdots & 1 \\
0 & 1 & 2 & \cdots & 10 \\
0 & 1^2 & 2^2 & \cdots & 10^2 \\
0 & 1^3 & 2^3 & \cdots & 10^3 \\
\end{array}\right)\]
de la que sabemos $n=11$, $r=4$, $k=7$ y $d=5$.

Este código puede corregir hasta 2 errores. Vamos a corregir el error usando el síndrome. El algoritmo sería:

\begin{enumerate}
\item Recibimos $y=y_1,...,y_{10} \in F_{11}^{10}$
\item Calculamos su síndrome obteniendo:
\[S(y) = \left( \begin{array}{c} s_1 \\ s_2 \\ s_3 \\ s_4\end{array}\right)\]
\item Si el síndrome es $0$, leemos $y$ y \textbf{fin del algoritmo}
\item Supongamos que se ha producido un error doble. En ese caso tenemos que encontrar los valores $a,b,i,j$ que satisface:
\[\left( \begin{array}{c} s_1 \\ s_2 \\ s_3 \\ s_4\end{array}\right) = a\left( \begin{array}{c} 1 \\ i \\ i^2 \\ i^3\end{array}\right) + b\left( \begin{array}{c} 1 \\ j \\ j^2 \\ j^3\end{array}\right)\]

Resolver este sistema no es tarea fácil, pues no es lineal. Si a cada ecuación le restamos $i$ veces la de debajo, podemos llegar al sistema:
\[\begin{array}{l}
b(i-j) = is_1-s_2\\
bj(i-j)=is_2-s_3\\
bj^2(i-j)=is_3-s_4
\end{array}\]

Podemos ver que el lado izquierdo de la segunda ecuación al cuadrado es igual al producto de los lados iquierdos de las otras dos ecuaciones.

Gracias a esta relación podemos escribir:
\[(s_2^2-s_1s_3)i^2+(s_1s_4-s_2s_3)i+s_3^2-s_2s_4=0\]
que se denomina \concept{polinomio localizador de errores}, cuyas raices son $i$ y $j$.

\item En el paso anterior suponemos que se han producido dos errores pero, ¿Qué ocurre si sólo hay un error?

En este caso, aplicando el mismo procedimiento del apartado anterior, obtenemos que el poliniomio localizador de errores es el polinomio nulo.

\end{enumerate}
\end{example}