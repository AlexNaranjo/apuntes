\chapter{Otros sistemas de clave pública y más aplicaciones}
\section{Criptosistema de ElGamal}
ElGamal fue un criptógrafo egipcio que además de ser conocido por la invención del criptosistema que lleva su nombre, colaboró con el navegador web Netscape.

El proceso de \underline{generación de la clave} sigue los siguientes pasos:
\begin{itemize}
	\item Se elige un grupo cíclico $G=<g>$ que será público y que deseablemente cumplirá que $|G| >>> 0$.
	\item Cada usuario (por ejemplo A) elige al azar su exponente secreto $d_A:\ 1<d<|G|$.
	\item La clave pública de cada usuario es calculada por él mismo como: $e_A=g^{d_A}$.
\end{itemize}

Para el \underline{proceso de cifrado} en el que B envía un mensaje $m∈G$ a A se llevan a cabo los siguientes pasos:
\begin{itemize}
	\item B elige al azar $k:\ 1 < k < |G|$.
	\item B envía a A $(g^k,me_A^k)=(c_1,c_2)∈G×G$.
\end{itemize}

Para el \underline{descifrado} que A tiene que hacer del mensaje recibido por B, realizará lo siguiente:
\begin{itemize}
	\item Puesto que $e_A^k = (g^{d_A})^k = (g^k)^{d_A}$, A puede obtener el mensaje original haciendo $c_2(c_1^{d_A})^{-1}=m$.
\end{itemize}

Observamos que para ``separar'' $e_A^k$ de $c_2$ es necesario conocer $d_A$.
