\Problem[1]
Se dispone de un equipo con un MTTF de 500 horas, que se encuentra bajo un
contrato de mantenimiento que garantiza su reparación en un tiempo medio de
10 horas.
\begin{enumerate}
  \item El proveedor nos ofrece sustituirlo por un equipo de mayor calidad, con un MTTF de 1000 horas. Calcular la mejora de disponibilidad que supone este cambio.
  \item Tras un tiempo trabajando con el nuevo equipo, el proveedor nos ofrece de nuevo un nuevo equipo de mucha mayor calidad, con un MTTF de 2000 horas. Calcular la mejora de disponibilidad que supondría el nuevo cambio respecto al segundo equipo.
\end{enumerate}

%%%%%%%%%%%%%%%%%%%%%%%%%%%%%%%%%%%%%%%%%%%%%%%%%%%%%%%%%%%%%%%%%%%%%
\TheSolution



%%%%%%%%%%%%%%%%%%%%%%%%%%%%%%%%%%%%%%%%%%%%%%%%%%%%%%%%%%%%%%%%%%%%%
\Problem[2]
Se desea prestar un servicio con una indisponibilidad menor de 10 horas por
año.
  \begin{enumerate}
    	\item Calcular la disponibilidad deseada del sistema.
    	\item Se dispone de un servicio de mantenimiento que es capaz de sustituir un equipo averiado en 24 h.\\
Calcular el MTTF que debe tener el equipo para poder satisfacer el requermiento de disponibilidad establecido.
    	\item Si sólo se dispone de equipos con un MTTF de 2000 horas, calcular el número de equipos que se deberían colocar en paralelo para garantizar el requisito.
  \end{enumerate}

%%%%%%%%%%%%%%%%%%%%%%%%%%%%%%%%%%%%%%%%%%%%%%%%%%%%%%%%%%%%%%%%%%%%%
\TheSolution



%%%%%%%%%%%%%%%%%%%%%%%%%%%%%%%%%%%%%%%%%%%%%%%%%%%%%%%%%%%%%%%%%%%%%
\Problem[3]
El servidor de envío de mensajes ("busca") descrito en el problema número 7 del Tema 2 (\ref{tema2:prob7}) se compone de dos dispositivos: uno de recepción de mensajes y otro de difusión de los mismos.
\begin{itemize}
\item El MTTF del dispositivo de recepción de mensajes es de 1000 horas, y
\item el MTTF del dispositivo de difusión es de 1500 horas.
\end{itemize}
Ambos los atiende el mismo servicio técnico, que en ambos casos nos asegura un MTTR de 5 horas.\\

Calcular la disponibilidad total del sistema.

%%%%%%%%%%%%%%%%%%%%%%%%%%%%%%%%%%%%%%%%%%%%%%%%%%%%%%%%%%%%%%%%%%%%%
\TheSolution



%%%%%%%%%%%%%%%%%%%%%%%%%%%%%%%%%%%%%%%%%%%%%%%%%%%%%%%%%%%%%%%%%%%%%
\Problem[4]
Los datos de disponibilidad de cada uno de los componentes de la red de la entidad financiera descrita en el problema número 8 del Tema 2 (\ref{tema2:prob8}) son los siguientes :
\begin{itemize}
	\item MTTF del multiplexor: 2000 h.
	\item MTTF del servidor: 2000 h.
	\item MTTF de cada terminal remoto: 1000 h.
	\item MTTF de la línea de comunicaciones del terminal remoto: 500 horas.
	\item MTTR de todos los elementos: 5 horas.
\end{itemize}

Suponiendo la existenciade dos únicos terminales remotos, realizar el
diagrama de bloques de disponibilidad y calcular la disponibilidad global del
sistema.

%%%%%%%%%%%%%%%%%%%%%%%%%%%%%%%%%%%%%%%%%%%%%%%%%%%%%%%%%%%%%%%%%%%%%
\TheSolution



%%%%%%%%%%%%%%%%%%%%%%%%%%%%%%%%%%%%%%%%%%%%%%%%%%%%%%%%%%%%%%%%%%%%%
\Problem[5]
Un servidor de fecha y hora de una red se compone de dos elementos, que deben estar operativos para
atender el servicio: Un ordenador, que satisface las peticiones, y un receptor
de señales horarias externo, que permite obtener la hora exacta. Si ambos equipos
tienen un MTTF de 1500 horas, calcular el MTTR mínimo que debemos considerar,
supuesto igual en ambos, si se desea una disponibilidad total del servicio mayor del 99\%.

%%%%%%%%%%%%%%%%%%%%%%%%%%%%%%%%%%%%%%%%%%%%%%%%%%%%%%%%%%%%%%%%%%%%%
\TheSolution



%%%%%%%%%%%%%%%%%%%%%%%%%%%%%%%%%%%%%%%%%%%%%%%%%%%%%%%%%%%%%%%%%%%%%
\Problem[6]
Un  servicio que debe tener una disponibilidad mínima de 0.99
se debe proporcionar con un sistema cuyo tiempo medio hasta el fallo es de
 2000 horas.
\begin{enumerate}
	\item Calcular el tiempo medio para reparar el equipo que es necesario para satisfacer la disponibilidad solicitada.
	\item Al comenzar con la explotación del servicio se descubre que el
programa que lo implementa tiene fallos que hacen que en ocasiones se
bloquee, dejando de atender peticiones. Para resolverlo es necesario
detener el proceso y volverlo a arrancar. Se ha estimado que en media se
 produce un fallo cada dos días. El fallo se tarda en descubrir un
promedio de 5 mn. y la reiniciación del proceso requiere, también en
promedio, 7 mn. Calcular la disponibilidad del sistema considerando este
 efecto.
	\item Suponiendo imposible modificar el programa o cambiar el equipo,
proponer una alternativa que permita, en las nuevas circunstancias,
satisfacer el requerimiento inicial de disponibilidad.
\end{enumerate}

%%%%%%%%%%%%%%%%%%%%%%%%%%%%%%%%%%%%%%%%%%%%%%%%%%%%%%%%%%%%%%%%%%%%%
\TheSolution



%%%%%%%%%%%%%%%%%%%%%%%%%%%%%%%%%%%%%%%%%%%%%%%%%%%%%%%%%%%%%%%%%%%%%
\Problem[7]
Una petición que se procesa en un servidor web tiene que pasar por
cuatro clases de elementos distintos para su resolución completa. Estos
elementos son:
  \begin{itemize}
    	\item Un distribuidor de carga, que reparte las peticiones a los servidores web.
    	\item Un servidor web, que entrega al cliente páginas estáticas e
imágenes. Existen en el sistema tres servidores web, de igual
funcionalidad, pudiendo cualquiera de ellos atender las peticiones.
    	\item Un servidor de aplicaciones, que ejecuta programas bajo petición
 de los servidores web. El sistema posee dos de estos servidores, de
igual funcionalidad. Los servidores web pueden enviar indistintamente
sus peticiones a cualquiera de ellos.
    	\item Un servidor de base de datos, al cual acceden los programas que
se ejecutan en los servidores de aplicaciones para recuperar los datos
que necesitan para realizar las peticiones.
  \end{itemize}


  Es necesario, por tanto, que esté disponible al menos un elemento
de cada una de las clases citadas anteriormente para que el sistema
completo funcione.
  \begin{enumerate}
    	\item Dibujar el diagrama de disponibilidad del sistema total definido.
    	\item Suponiendo que todos los ordenadores que implementan los
servidores tienen la misma disponibilidad, A, y que cada servidor se
implementa en un ordenador distinto, calcular la expresión de la
disponibilidad total del sistema, AT en función de A.
    	\item Calcular el valor numérico de AT cuando el tiempo medio entre
fallos es de 2000 horas y el tiempo medio entre reparaciones, 100 horas.
  \end{enumerate}

%%%%%%%%%%%%%%%%%%%%%%%%%%%%%%%%%%%%%%%%%%%%%%%%%%%%%%%%%%%%%%%%%%%%%
\TheSolution



%%%%%%%%%%%%%%%%%%%%%%%%%%%%%%%%%%%%%%%%%%%%%%%%%%%%%%%%%%%%%%%%%%%%%
\Problem[8]
Cada uno de los elementos que componen el servidor de transacciones descrito en el problema número 15 del Tema 2 (\ref{tema2:prob15}) se ejecuta en un ordenador independiente que tiene un MTTF de 4000
horas. Calcular el MTTR necesario para garantizar una disponibilidad del sistema total del 99.9\%.

%%%%%%%%%%%%%%%%%%%%%%%%%%%%%%%%%%%%%%%%%%%%%%%%%%%%%%%%%%%%%%%%%%%%%
\TheSolution



%%%%%%%%%%%%%%%%%%%%%%%%%%%%%%%%%%%%%%%%%%%%%%%%%%%%%%%%%%%%%%%%%%%%%
\Problem[9]
Comparar desde el punto de vista de disponibilidad las dos soluciones propuestas en el problema número 18 del Tema 2 (\ref{tema2:prob18})
 para resolver un cuello de botella en acceso a disco. Suponer que todos
 los discos empleados tienen los mismos MTTF y MTTR, y justifique qué
alternativa es más fiable calculando y comparando las disponibilidades
en ambos casos.

%%%%%%%%%%%%%%%%%%%%%%%%%%%%%%%%%%%%%%%%%%%%%%%%%%%%%%%%%%%%%%%%%%%%%
\TheSolution


%%%%%%%%%%%%%%%%%%%%%%%%%%%%%%%%%%%%%%%%%%%%%%%%%%%%%%%%%%%%%%%%%%%%%
\Problem[10]
Los servidores de directorios DSA1 y DSA2 descritos en el problema número 20 del Tema 2 (\ref{tema2:prob20})
 se encuentran conectados por una línea de comunicaciones que tiene una
disponibilidad de 0.99. La disponibilidad de DSA2 es de 0.98. Si el MTTF
 de DSA1 es de 2000 horas, ¿cuál será el MTTR que deberemos garantizar
para él, para conseguir que la disponibilidad del sistema total sea
igual a 0.95? ¿Cuál sería, en este caso, la disponibilidad de DSA1
considerado aislado del resto del sistema?

%%%%%%%%%%%%%%%%%%%%%%%%%%%%%%%%%%%%%%%%%%%%%%%%%%%%%%%%%%%%%%%%%%%%%
\TheSolution


%%%%%%%%%%%%%%%%%%%%%%%%%%%%%%%%%%%%%%%%%%%%%%%%%%%%%%%%%%%%%%%%%%%%%
\Problem[11]
Se desea definir la arquitectura de un servidor de aplicaciones
 siguiendo el modelo estándar visto en la asignatura, compuesto por
cuatro elementos distintos en la cadena de procesamiento:
\begin{itemize}
	\item Capa de distribución: Balanceador de carga del tráfico de usuarios.
La alta disponibilidad de estos elementos se garantizará mediante el
protocolo VRRP.
	\item Capa de presentación: Servidores web. La capa anterior garantiza una alta disponibilidad activo-activo en esta capa.
	\item Capa de aplicación: Servidores de aplicaciones. El web plug-in
insertado en los servidores web garantiza la alta disponibilidad
activo-activo en esta capa.
	\item Capa de datos: Servidores de bases de datos. Se elige que la
alta disponibilidad de esta capa la proporcione un cluster
activo-pasivo.

\end{itemize}

Los ordenadores con los que se trabaja tienen un MTTF de 4000 horas, y su MTTR es de 48 horas.
\begin{enumerate}
	\item Calcular la disponibilidad de la capa de presentación si se colocan en ella dos servidores web.
	\item Adicionalmente a los posibles problemas de hardware reflejados
en el MTTF genérico de los servidores empleados, se han detectado fallos
 en el software de los servidores de aplicaciones, que hacen que en
promedio exista un fallo por semana que hace necesario rearrancar el
servidor. El tiempo de rearranque del servidor es de 30 mn. Calcular la
disponibilidad de la capa de aplicación si se colocan en ella dos
servidores de aplicaciones.
	\item En el caso de la base de datos, la recuperación tras un fallo
no se realiza a través de las reparaciones habituales, sino que se
dispone de un cluster activo-pasivo. El tiempo de conmutación al
servidor pasivo se compone de los siguientes tiempos:
\begin{itemize}
	\item Tiempo de detección del fallo en el servidor principal: 30s.
	\item Tiempo de arranque del servidor pasivo: 30 mn.
	\item Tiempo de recuperación de la base de datos tras un cierre anormal, en el caso peor: 2 horas.

\end{itemize}

Calcular la disponibilidad de la capa de datos en dicho caso peor.
	\item En la capa de distribución se conocen los siguientes datos:
\begin{itemize}
	\item Advertisement interval (tiempo entre la transmisión de paquetes hello por el balanceador activo): 1s.
	\item Master down interval (tiempo tras el cual si no se reciben paquetes hello se considera el master caído): 3s.
	\item Skew adicional para producir la conmutación tras la expiración del master down interval: 0.

\end{itemize}

Calcular, en el caso peor, la disponibilidad del conjunto de dos balanceadores de carga trabajando en esta configuración.
	\item Calcular la disponibilidad total de la instalación.
\end{enumerate}

%%%%%%%%%%%%%%%%%%%%%%%%%%%%%%%%%%%%%%%%%%%%%%%%%%%%%%%%%%%%%%%%%%%%%
\TheSolution
