\documentclass{apuntes}[nochap]
\usepackage[colorlinks,urlcolor=blue]{hyperref}
\usepackage{xifthen}

\newcounter{problem}
\newcounter{solution}
\renewcommand{\theenumi}{\alph{enumi}}

\newcommand\Problem[1][]{%
  \color{blue}
  \stepcounter{problem}%
  \ifthenelse{\isempty{#1}}{\textbf{Problema \theproblem.}}{\textbf{Problema #1.}}

  \setcounter{solution}{0}%
}

\newcommand\TheSolution{%
  \color{black}
  \textbf{Solución:}\\%
}

\newcommand\ASolution{%
  \stepcounter{solution}%
  \textbf{Solución \thesolution:}\\%
}


\setlength{\arrayrulewidth}{1mm}
\setlength{\tabcolsep}{18pt}
\renewcommand{\arraystretch}{1.5}


\author{Pedro Valero Mejía}
\title{Sistemas Informáticos II}
\parindent 0in
\parskip 1em
\makeindex
\begin{document}

\maketitle
\newpage
\tableofcontents

%%%%%%%%%%%%%%%%%%%%%%%%%%%%%%%%%%%%%%%%%%%%%%%%%%%%%%%%%%%%%%%%%%%%%%%%%% PROBLEMA 1

\chapter{Middleware}
\section{Teoría}
Para este tema es importante tener claros varios conceptos que vamos a ir definiendo y explicando poco a poco.

\begin{defn}[Middleware]
Conjunto de aplicaciones encargadas de enlazar los componentes de un sistema distribuido.
\end{defn}

Este conjunto de aplicaciones está dividido en el protocolo específico del servicio con el que estamos trabajando (ODBC, HTTP, SMTP...), el protocolo de transporte (TCP/IP) y una capa intermedia llamada \concept{Network Operating System (NOS)}.

\begin{defn}[Network Operating System]
Es una extensión del sistema operativo que proporciona transparencia al cliente, para que éste realice las llamadas como si fueran locales.

\end{defn}
Algunas de las maneras de proporcionar transparencia del NOS son 


\section{Problemas Tema 1}
\chapter{Colas}
\section{Teoría}
\section{Problemas Tema 2}
% -*- root: ../SI2.tex -*-

\chapter{Tema 3}
\section{Teoría}
\section{Problemas Tema 3}


\end{document}