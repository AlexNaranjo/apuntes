\appendix
\section{Ejercicios}
\subsection{Tema 1 - Estadística descriptiva}

\begin{problem}[2] Demostrar que \[ \sum_{i=1}^n \left(x_i-\avg{x}\right)^2 = \min_{a\in \real} \sum_{i=1}^n(x_i-a)^2 \]

\solution

Definimos una función \[ g(a) = \sum_{i=1}^n(x_i-a)^2 \], buscamos su derivada \[ g'(a) = -2 \sum_{i=1}^n(x_i-a) \] e igualamos a cero:

\begin{gather*}
-2 \sum_{i=1}^n(x_i-a) = 0 \\
\sum_{i=1}^n x_i - \sum_{i=1}^n a = 0 \\
n \avg{x} = n a \\
\avg{x} = a 
\end{gather*}

Esto quiere decir que la media muestral es el valor que minimiza la distancia con cada uno de los datos de la muestra.
\end{problem}

\begin{problem}[5]Determina si es verdadero o falso:

\ppart Si añadimos 7 a todos los datos de un conjunto, el primer cuartil aumenta en 7 unidades y el rango intercuartílico no cambia.

\ppart Si todos los datos de un conjunto se multiplican por -2, la desviación típica se dobla.
\solution 

\spart Añadir siete a todos los datos es una traslación, así que la distribución de los datos no cambia.

\spart Teniendo en cuenta que si multiplicamos todos los datos del conjunto por $-2$ la media también se multiplica por $-2$, y sustituyendo en la fórmula de la varianza:

\[ \sigma' = \sqrt{\frac{1}{n} \sum_{i=1}n (-2x_i)^2 - (-2\avg{x})^2} = \sqrt{\frac{1}{n} \sum_{i=1}4\left(n x_i^2 - \avg{x}^2\right)} = \sqrt{4\sigma^2} = 2\sigma \]

Por lo tanto, la desviación típica sí se dobla.

\spart Usando los cálculos del apartado anterior vemos que la varianza se multiplica por cuatro.

\spart Efectivamente: cambiar el signo haría una reflexión de los datos sobre el eje Y y la asimetría estaría orientada hacia el lado contrario. 

\end{problem}