\subsection{Tema 1 - Estadística descriptiva}

\begin{problem}[2] Demostrar que \[ \sum_{i=1}^n \left(x_i-\avg{x}\right)^2 = \min_{a\in \real} \sum_{i=1}^n(x_i-a)^2 \]

\solution

Definimos una función \[ g(a) = \sum_{i=1}^n(x_i-a)^2 \], buscamos su derivada \[ g'(a) = -2 \sum_{i=1}^n(x_i-a) \] e igualamos a cero:

\begin{gather*}
-2 \sum_{i=1}^n(x_i-a) = 0 \\
\sum_{i=1}^n x_i - \sum_{i=1}^n a = 0 \\
n \avg{x} = n a \\
\avg{x} = a 
\end{gather*}

Esto quiere decir que la media muestral es el valor que minimiza la distancia con cada uno de los datos de la muestra.
\end{problem}

\begin{problem}[5]Determina si es verdadero o falso:

\ppart Si añadimos 7 a todos los datos de un conjunto, el primer cuartil aumenta en 7 unidades y el rango intercuartílico no cambia.

\ppart Si todos los datos de un conjunto se multiplican por -2, la desviación típica se dobla.
\solution 

\spart Añadir siete a todos los datos es una traslación, así que la distribución de los datos no cambia.

\spart Teniendo en cuenta que si multiplicamos todos los datos del conjunto por $-2$ la media también se multiplica por $-2$, y sustituyendo en la fórmula de la varianza:

\[ \sigma' = \sqrt{\frac{1}{n} \sum_{i=1}n (-2x_i)^2 - (-2\avg{x})^2} = \sqrt{\frac{1}{n} \sum_{i=1}4\left(n x_i^2 - \avg{x}^2\right)} = \sqrt{4\sigma^2} = 2\sigma \]

Por lo tanto, la desviación típica sí se dobla.

\spart Usando los cálculos del apartado anterior vemos que la varianza se multiplica por cuatro.

\spart Efectivamente: cambiar el signo haría una reflexión de los datos sobre el eje Y y la asimetría estaría orientada hacia el lado contrario. 

\end{problem}

\subsection{Tema 2 - Muestreo aleatorio}

\begin{problem}[1] Se desea estimar el momento de orden 4, $\alpha_3 = \esp{X^3}$ en una v.a. $X$ con distirbución exponencial de parámetro 2, es decir, la función de distribución de $X$ es $F(t) = \prob{X ≤ t} = 1 - e^{-2t}$ para $t≥0$. Definir un estimador natural para $\alpha_3$ y calcular su error cuadrático medio.

\solution

Usando el criterio de \textit{plugin}, podríamos definir el estimador \[ \hat{\alpha}_3 = \int_\real x^3\,d\fd_n(x) \]. 

Calculamos ahora el error cuadrático medio:

\begin{gather*}
ECM(\hat{\alpha}_3) = \esp{\hat{\alpha}_3 - \alpha_3}^2 = \esp{(\hat{\alpha}_3 - \esp{\hat{\alpha}_3} + \esp{\hat{\alpha}_3} - \alpha_3) ^2} = \\
= \esp{(\hat{\alpha}_3 - \esp{\hat{\alpha_3}})^2 +  (\esp{\hat{\alpha_3}}- \alpha_3)^2 + 2(\hat{\alpha}_3 - \esp{\hat{\alpha_3}}) (\esp{\hat{\alpha_3}}- \alpha_3)} = \\
= \underbrace{\esp{(\hat{\alpha_3} - \esp{\hat{\alpha_3}})^2}}_{(a)}+ \underbrace{\left(\esp{\hat{\alpha_3}} - \alpha_3\right)^2}_{(b)} + \underbrace{2 \cdot \esp{ (\esp{\hat{\alpha_3}}- \alpha_3)^2 + 2(\hat{\alpha}_3 - \esp{\hat{\alpha_3}})}}_{(c)} 
\end{gather*}

Aquí ya hay cosas raras. (c) es cero por alguna razón, luego hay que calcular la varianza y el sesgo.

\[ \text{sesgo}(\hat{\alpha}_3) = \esp{\hat{\alpha}_3} - \alpha_3 = \alpha_3 - \alpha_3 = 0 \]

\[ \var{\hat{\alpha}_3} = \dotsc \] 

Queda para mañana. 

\end{problem}

\begin{problem}[2] Supongamos que la muestra tiene tamaño $n=50$ y que la distribución de las $X_i$ es una $N(4,1)$. 

\ppart Obtener, utilizando la desigualdad de Chebichev, una cota superior para la probabilidad $\prob{\abs{\avg{X} - 4} > 0.3}$.
\ppart Calcula exactamente $\prob{\abs{\avg{X} - 4} > 0.3}$ utilizando la distribución de $X_i$. 

\solution
\spart

Como la media es cuatro, la desigualdad de Checbichev nos da una cota de 

\[ \frac{\var{\avg{x}}}{0.3^2} = \frac{\var{X}}{n \cdot 0.3^2} \simeq 0.22 \]

\spart

Normalizamos

\[ Z = \frac{\avg{X} - 4}{\frac{1}{\sqrt{50}}} ~ N(0,1) \]

y calculamos.

\[ \prob{\abs{\avg{X} - 4} > 0.3} = \prob{\abs{Z} > \frac{0.3}{\frac{1}{\sqrt{50}}}} = 2 \cdot \prob{Z > 2.12} = 0.038 \]

\end{problem}

\begin{problem}[4] Denotemos por 

\[ C_n = \int_\real \left(\fd_n(t) - F(t)\right)^2 \, dF(t) \]

la llamada discrepancia de Cramer-Von Mises entre $\fd_n$ y $F$. ¿Converge a cero casi seguro esta discrepancia?

Calcular la distribución asintótica de la sucesión $D_n = \sqrt{n}\left(\fd_n(t) - F(t)\right)$ para un valor fijo $t\in\real$.

\solution

\[ C_n = \int_\real \left(\fd_n(t) - F(t)\right)^2 \, dF(t) = \int_\real \left(\fd_n(t) - F(t)\right)^2 f(t) \, dt \]

Como por el teorema de Glivenko-Cantelli (\ref{thmGlivenko}) tenemos que 

\[ \fd_n(t) - F(t) ≤ \sup_t \abs{\fd_n(t) - F(t)} = \md{\fd_n - F}_\infty \]

entonces 

\[ \int_\real \left(\fd_n(t) - F(t)\right)^2 f(t) \, dt ≤  \md{\fd_n - F}_\infty^2 \int_\real f(t) \,dt = \md{\fd_n - F}_\infty^2 \]

Igualmente por Glivenko-Cantelli, 

\[ \md{\fd_n - F}_\infty^2 \convcs 0  \qed \]

\spart

Para calcular la distirbución asintótica de \[ D_n = \sqrt{n}\left(\fd_n(t) - F(t)\right) \] usamos el Teorema Central del Límite (\ref{thmCentral}). Necesitamos algo que se asemeje a una media muestral, y de hecho

\[ \fd_n(t) = \frac{1}{n} \sum_{i=1}^n \ind_{(-\infty, t]} (X_i) = \frac{1}{n} \sum_{i=1}^n Y_i = \avg{Y} \]

Por otra parte, $Y = \ind_{(-\infty, t]}(X)$ y por lo tanto \[ \esp{Y} = \esp{\ind_{(-\infty, t]}(X)} = \prob{X ≤ t} = F(t) \]

Ya podemos aplicar el TCL, pero nos falta saber cuál es la desviación típica de $Y$. Como es una distribución de Bernoulli 

\[ \mathbb{V}(Y) = p(1-p) = F(t)(1-F(t)) \]

y por lo tanto 

\[ D_n \convdist N\left(0, \sqrt{F(t)(1-F(t))}\right) \]

\end{problem}