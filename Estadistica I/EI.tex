\documentclass{apuntes}
\author{Guillermo Julián Moreno}
\date{13/14 C1}
\title{Estadística I}

\begin{document}

\pagestyle{plain}
\maketitle

\tableofcontents
\newpage

\section{Estadística descriptiva de datos univariantes}

La estadística descriptiva es el conjunto de técnicas para resumir la información proporcionada por una gran masa de datos. El primer objetivo natural es resumir la información que proporcionan esos datos.

\subsection{Estadísticos de tendencia central}
 
\begin{defn}[Media]

\[ \avg{x} = \frac{\sum_{i=1}^n x_i}{n} \]

Es la medida de tendencia central más utilizada. Es bastante sensible a los valores atípicos (\textit{outliers}), observaciones anormalmente grandes que aparecen en el conjunto de datos por errores de transcripción o medición.

\end{defn}

\begin{defn}[Mediana]
Es el valor que divide a los datos en dos mitades, de tal forma que la mitad son menores y la otra mitad mayores que la mediana. 

La mediana se calcula de la siguiente forma: dado un conjunto de datos $\{x_1,\dotsc, x_n\}$, la mediana es $x_{\frac{n+1}{2}}$ si $n$ es impar y  el promedio entre $x_{\frac{n}{2}}$ y $x_{\frac{n}{2} + 1}$.
\end{defn} 

\subsection{Estadísticos de dispersión}

\begin{defn}[Varianza]
\[ \sigma^2 = \frac{1}{n} \sum_{i=1}n \left(x_i - \avg{x}\right)^2 = \frac{1}{n} \sum_{i=1}n x_i^2 - \avg{x}^2 \]
\end{defn}

\begin{defn}[Desviación típica]
\[\sigma = \sqrt{\sigma^2} \]

La desviación típica es la raíz de la varianza.
\end{defn}

\begin{defn}[Cuantil]
Para $p\in (0, 1)$ se llama cuantil $p$ o $q_p$ al valor que deja el $100p \%$ de los datos a la izquierda.
\end{defn}

\begin{defn}[Cuartil]
Los cuartiles son los tres datos que dejan a la izquierda el 25, 50 y 75 por ciento de los datos respectivamente. Es decir:

\begin{itemize}
\item $Q_1 = q_{0.25}$
\item $Q_2 = q_{0.5}$. El cuartil dos es la mediana.
\item $Q_3 = q_{0.75}$
\end{itemize}
\end{defn}

Hay varios métodos para el cálculo de cuantiles. Para hacerlo a mano, podemos usar el siguiente método.

Si el dato en la posición $p(n+1)$ no es un número entero, entones se interpola entre las observaciones ordenadas que están en la posición $\floor{p(n+1)}$ y $\floor{p(n+1)} + 1$ de la siguiente forma: sea $j$ la parte entera de $p(n+1)$ y $m$ la parte decimal. Entonces, \[ q_p = (1-m)x_j + m x_{j+1} \]


\begin{defn}[Coeficiente\IS de asimetría]
\index{Skewness}
El tercer momento con respecto a la media se define como \[ \frac{1}{n}\sum_{i=1}^n\left(x_i-\avg{x}\right)^3 \] que, en su versión adimensional dividimos por $\sigma^3$.
\end{defn}

Al ser una función cúbica, los valores que se alejen mucho de la media tendrán un valor muy alto en valor absoluto (positivo o negativo según se aleje por la derecha o izquierda, respectivamente). Si la distribución de datos es muy asimétrica, los valores más altos no se cancelan con los valores altos del otro lado (porque no hay) y saldrá un valor más alejado de cero.\footnote{Está explicado como el p. culo, ya.}

\subsection{Representación gráfica de datos}

\begin{defn}[Box-plot]
El diagrama de caja o \textit{box-plot}  (imagen \ref{imgCaja}) nos permite visualizar las medidas de dispersión respecto a la mediana. Hay que añadir una nueva medida, el \textbf{rango intercuartílico}\index{Rango!intercuartílico}, la diferencia entre el primer y el tercer cuartil: \[RI = Q_3 - Q_1 \]

\easyimg{DiagramaCaja.png}{Diagrama de caja}{imgCaja}
\end{defn}

\begin{defn}[Histograma]
El histograma se trata de una aproximación discreta a la función de densidad continua $f(t)$ de la variable que estamos midiendo. Es un diagrama de frecuencias que \textit{mantiene la forma} de esa función de densidad. 

Definimos una serie, las marcas de intervalos $a^n_1, \dotsc, a^n_n$, donde $n$ es el número de intervalos y la longitud de cada intervalo  es $h_n = a^n_{j+1} - a^n_j$. Sea el conjunto $\{x_i\}_{i=0,\dotsc,m}$ los datos de nuestra muestra. Entonces, el estimador, la función $\hat{f}_n$, se define de la siguiente forma:

\[ \hat{f}^n(t) = \frac{\card{i \tq x_i \in \left( a_j^n, a_{j+1}^n \right]}}{n h_n} = \frac{\sum_{i=1}^m \ind_{(a_j^n, a_{j+1}^n]} (x_i)}{n h_n} \]

Recordemos que \[ \ind_A (n) = \begin{cases} 1 & n \in A \\ 0 & n \notin A\end{cases}\]

A grandes rasgos, lo que hace en una función es definir un número de intervalos fijos de ancho $h_n$. Al evaluar $\hat{f}^n(t)$ buscamos en qué intervalo cae $t$ y contamos cuántas de nuestras mediciones caen también en ese intervalo.

\easyimg{DensidadAHistograma.png}{El histograma es una aproximación de la función de densidad real en base a la muestra que hemos obtenido.}{lblDensidad}

\end{defn}

\subsubsection{Estimadores núcleo o kernel}

\begin{defn}[Método de ventana móvil][Ventana móvil]
El método de ventana móvil nos da una estimación de la función de densidad en un punto $t$ midiendo los $x_i$ que están en el intervalo de radio $h_n$ centrado en $t$. Matemáticamente:

\[ \hat{f}_n(t) = \frac{1}{n2h_n}\sum_{i=1}^n \ind_{[t-h_n, t+h_n]}(x_i) = \frac{1}{n2h_n}\sum_{i=1}^n \ind_{[-1,1]}\left(\frac{t-x_i}{h_n}\right) \]
\end{defn}

Podemos reemplazar la función $\frac{1}{2}\ind_{[-1, 1]}$ por otra, llamada la función de densidad $K$, kernel o núcleo:

\begin{defn}[Estimador núcleo]
Dada una función de densidad $K$ simétrica, no necesariamente positiva, definimos el estimador kernel como:

\[ \hat{f}_n(t) = \frac{1}{n}\sum_{i=1}^n K_h (t - x_i)  = \frac{1}{nh_n} \sum_{i=1}^n K\left(\frac{t-x_i}{h_n}\right) \]

con $K_h(x) = \frac{1}{h}K(\frac{x}{h})$.
\end{defn}

La elección del núcleo $K$ no afecta especialmente a lo bien aproximada que esté la función de densidad. Sin embargo, sí que influye la selección de la ventana $h_n$ (figura \ref{lblSuavizado}), también llamada \textit{bandwith} en inglés.  Si escogemos una ventana muy pequeña, damos demasiado peso a los datos de nuestra muestra. Si elegimos una ventana muy grande, nuestra muestra pierde importancia y podemos perder información importante.

La elección del $h_n$ más habitual es el que minimiza la distancia $L^2$ entre $\hat{f}$ y $f$, es decir, el parámetro que minimice $\displaystyle\int\left(\hat{f}_h-f\right)^2$. Sin embargo, hay un problema: no sabemos qué es $f$. Hay trucos que imagino que veremos más tarde.

\easyimgw{Suavizado.png}{Los efectos que causa elegir una ventana más grande o más pequeña en el estimador}{lblSuavizado}{1}

Las funciones kernel más usadas son la uniforme, $\frac{1}{2}\ind_{[-1, 1]}$, la gaussiana $\frac{1}{\sqrt{2 \pi}}e^{-\frac{t^2}{2}}$ y la de Epanechnikov, que matemáticamente es la que mejor aproxima $f$.

El estimador kernel $\hat{f}_n(t)$ es la función de densidad de una medida de probabilidad que es la convolución \footnote{Ya aprenderemos en al algún momento de nuestra vida qué narices es una convolución} de dos medidas de probabilidad: una, $K_h(x)$ (el kernel reescalado) y otra que da probabilidad $\frac{1}{n}$ a cada punto de la muestra $\{x_i\}$ (distribución o medida empírica).

\paragraph{Generación de datos del estimador kernel} Supongamos que $K$ es el núcelo gaussiano. Podemos generar datos artificiales de la densidad así:

\[ x_i^0 = x_i^* + h_n Z_i,\; i=1,\dotsc, k \]

donde $x_i^*$ es una observación elegida al azar entre los datos originales y $Z_i$ una observación aleatoria con probabilidad $N(0,1)$. Es decir, lo que hacemos es añadir un dato aleatorio de la muestra y sumamos una pequeña perturbación aleatoria.

\section{Estadística descriptiva de datos bivariantes}

En esta sección estudiaremos dos variables $(X, Y)$ para explorar la relación entre ambas y tratar de inferir si existe una relación funcional para predecir los valores de una variable en función de los de la otra.

\subsection{Representación gráfica}

\begin{defn}[Diagrama\IS de dispersión]
El diagrama de dispersión representa cada variable en función de la otra para que podamos ver la posible relación entre ambas. Ver figura \ref{lblDispersion}.

\easyimg{Dispersion.png}{Diagrama de dispersión}{lblDispersion}
\end{defn} 

\subsection{Regresión}

\begin{defn}[Recta de regresión]

La recta de regresión de $y$ sobre $x$ es la recta de forma $\hat{y} = \hat{a} + \hat{b}x$ que más se aproxima a los datos, minimizando los cuadrados de la distancia: \[ (\hat{a},\hat{b}) =\argmin_{a, b} \sum_{i=1}^n\left(y_i - a - bx_i)\right)^2 \]
\end{defn}

La recta de regresión se calcula obteniendo primero $\hat{b}$:

\[ \hat{b} = \frac{\sigma_{x,y}}{\sigma^2_x} \]

donde \[ \sigma_{x,y} = \frac{1}{n} \left( \sum_{i=1}^n x_i y_i\right)  - \avg{x}\avg{y} \] y después, sabiendo que la recta pasa por el punto $(\avg{x}, \avg{y})$, obtenemos $\hat{a}$ \[ \hat{a} = \avg{y} - \hat{b}\avg{x} \]

El valor $b$ se denomina \textbf{coeficiente de regresión lineal}\index{Regresión lineal!coeficiente de} o parámetro de la regresión. Cada valor $e_i= y_i - \hat{y}_i$ se denomina \textbf{residuo}\index{Residuo}. Hay que notar que

\begin{gather*}
 \sum_{i=1}^n e_i = \sum_{i=1}^n \left(y_i - \hat{a} -\hat{b}x_i \right)= \sum_{i=1}^n\left( y_i - (\avg{y} - \hat{b}\avg{x}) - \hat{b}x_i \right) = \\
 = \sum_{i=1}^n  \left(y_i - \hat{b}x_i\right) - n\avg{y}  + n\hat{b}\avg{x} = n\avg{y} - n \hat{b}\avg{x}- n\avg{y} + n\hat{b}\avg{x} = 0 \end{gather*}

Esta ecuación ($\sum_{i=1}^n e_i = 0$) junto con \[ \sum_{i=1}^n x_i e_1 = 0 \] son las dos restricciones entre los residuos que nos dan la recta.

\begin{defn}[Varianza\IS residual]
La varianza residual $s_R^2$ o $\hat{\sigma}_e^2$ mide, aproximadamente el \textit{error cuadrático} cometido en la aproximación dada por la recta de regresión:

\[ s_R^2 = \hat{\sigma}_e^2 = \frac{1}{n}\sum_{i=1}^n e_i^2 \]
\end{defn}

\begin{defn}[Coeficiente\IS de correlación lineal]
\index{Coeficiente!de Pearson}
El coeficiente de correlación lineal o coeficiente de Pearson

\[ r = \frac{\hat{\sigma}_{x,y}}{\hat{\sigma}_x \hat{\sigma}_y} \] que cumple las siguientes condiciones:

\begin{gather*}
0 ≤ r^2 ≤ 1 \\
\hat{\sigma}_e^2 = \hat{\sigma}_y^2(1-r^2) \\
r = \hat{b}\frac{\hat{\sigma}_x}{\hat{\sigma}_y} 
\end{gather*}

nos indica el grado de ajuste lineal entre las dos variables. Un valor absoluto más cercano a 1 indica una correlación más fuerte. Un valor absoluto cercano a cero indica una correlación débil. El signo, positivo o negativo, indica si la correlación es creciente o decreciente.
\end{defn}

\appendix
\section{Ejercicios}
\subsection{Tema 1 - Estadística descriptiva}

\begin{problem}[2] Demostrar que \[ \sum_{i=1}^n \left(x_i-\avg{x}\right)^2 = \min_{a\in \real} \sum_{i=1}^n(x_i-a)^2 \]

\solution

Definimos una función \[ g(a) = \sum_{i=1}^n(x_i-a)^2 \], buscamos su derivada \[ g'(a) = -2 \sum_{i=1}^n(x_i-a) \] e igualamos a cero:

\begin{gather*}
-2 \sum_{i=1}^n(x_i-a) = 0 \\
\sum_{i=1}^n x_i - \sum_{i=1}^n a = 0 \\
n \avg{x} = n a \\
\avg{x} = a 
\end{gather*}

Esto quiere decir que la media muestral es el valor que minimiza la distancia con cada uno de los datos de la muestra.
\end{problem}

\begin{problem}[5]Determina si es verdadero o falso:

\ppart Si añadimos 7 a todos los datos de un conjunto, el primer cuartil aumenta en 7 unidades y el rango intercuartílico no cambia.

\ppart Si todos los datos de un conjunto se multiplican por -2, la desviación típica se dobla.
\solution 

\spart Añadir siete a todos los datos es una traslación, así que la distribución de los datos no cambia.

\spart Teniendo en cuenta que si multiplicamos todos los datos del conjunto por $-2$ la media también se multiplica por $-2$, y sustituyendo en la fórmula de la varianza:

\[ \sigma' = \sqrt{\frac{1}{n} \sum_{i=1}n (-2x_i)^2 - (-2\avg{x})^2} = \sqrt{\frac{1}{n} \sum_{i=1}4\left(n x_i^2 - \avg{x}^2\right)} = \sqrt{4\sigma^2} = 2\sigma \]

Por lo tanto, la desviación típica sí se dobla.

\spart Usando los cálculos del apartado anterior vemos que la varianza se multiplica por cuatro.

\spart Efectivamente: cambiar el signo haría una reflexión de los datos sobre el eje Y y la asimetría estaría orientada hacia el lado contrario. 

\end{problem}


\newpage
\printindex
\end{document}