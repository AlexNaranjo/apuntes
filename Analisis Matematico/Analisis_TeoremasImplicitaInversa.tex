\section{Teoremas de la función implícita y la función inversa}

En el mundo lineal tenemos podemos resolver sistemas de dos formas. Con $\appl{F}{\real^N}{\real^N}$ y $L(\gx) = A\gx$, siendo $A$ una matriz $N\x N$; queremos resolver el sistema $A\gx = \gy$ sabiendo que $A\gor{0} = \gor{0}$.

Este sistema tiene solución si y sólo si  $\det(A) \neq 0$: es la condición para que exista $A^{-1}$.

Existe también la posibilidad de tener una función $\appl{F}{\real^{N+M}}{\real^N}$ con $L(\gx) = A\gx$, $A$ matriz $N\x (N+M)$.

Para resolver el sistema $A\gx = \gy$ parametrizamos $M$ variables.

En el mundo \textbf{no lineal}, consideramos una función $\appl{F}{\real^N}{\real^N}$. Entonces tenemos un sistema de ecuaciones

\[ \left.\begin{matrix}
F(x_1,\dotsc,x_N) = y_1\\
F(x_1,\dotsc,x_N) = y_2\\
\vdots\\
F(x_1,\cdots,x_N) = y_N          
        \end{matrix}
\right\} F(\gx)=\gy \]

Vamos a intentar resolver este problema utilizando Taylor para aproximar al orden lineal, pero tenemos que pagar un precio: para que taylor funcione tenemos que trabajar cerca del punto. Esto significa que \textbf{el resultado va a ser local}.


\begin{theorem}[Teorema\IS de la aplicación contractiva] Sea 
\begin{itemize}
\item $\appl{F}{\real^N}{\real^N}$ o
\item $\appl{F}{C}{C}, C\in \real^N$, cerrado, o
\item $\appl{F}{K}{K}, K\in \real^N$, compacto.
\end{itemize}

Supongamos que existe $\alpha\in(0,1)$ tal que

\[ \md{F(x)-F(y)}\leq \alpha\md{x-y} \forall x,y \in \left\{\begin{matrix}
                                                           \real^N\\
                                                           C\\
                                                           K
                                                          \end{matrix}\right. 
                                              \]
                                              
  Entonces                                          
                                           
\[  \exists ! x_0 \in\left\{\begin{matrix}         \real^N\\
                                                           C\\
                                                           K
                                                          \end{matrix}\right. 
                                                        \tlq F(x_0) = x_0 \text{ (Punto fijo)} \]
\label{thmAC}
\end{theorem}


\begin{proof} Primero llevamos los casos de $C$ y $\real^n$ a un conjunto compacto $K$. Partimos de 

\begin{equation}
\md{F(\gx) - F(\ga)} \leq \alpha\md{\gx-\ga} \label{eqAC_hip}
\end{equation} 

y veamos qué ocurre para un vector general $\gx$: \[ \md{F(\gx)} = \md{(F(\gx)-F(\ga)+F(\ga)} \leq \md{F(\gx)-F(\ga)} + \md{F(\ga)} \] 

Aplicando (\ref{eqAC_hip}) tenemos que 
\[\md{F(\gx)} ≤ \alpha \md{\gx-\ga} + \md{F(\ga)} \]
  
  Si tomamos $\ga=0$ (en el caso  $0 \notin C $ solo haría falta una pequeña traslación), y suponemos $ \md{x} < R$, tenemos entonces que \[ \md{F(\gx)} \leq \alpha R + \md{F(\gor{0})} < R\]
  Es decir, $F$ toma un compacto y lo lleva en sí mismo: $\appl{F}{B_R(0)}{B_R(0)}$. Podemos seguir la demostración ahora suponiendo que estamos trabajando siempre sobre un compacto.
  
  El siguiente paso es llevar a cabo un \textbf{proceso iterativo}. Tenemos \[ \appl{F}{K}{K} \] con  $K\subset \real^N$ conjunto compacto. Definimos entonces la sucesión de $\{x_n\}_{n \in \nat} \subset K$, construido de forma iterativa con $x_n = F(x_{n-1})$. Vamos a demostrar que esa sucesión es de Cauchy, lo que implicaría que es convergente.
  
 Para ello, dado $\epsilon > 0$ hay que hallar $n_0$ tal que si $n,m>n_0$ entonces $\md{x_n-x_m}<\epsilon$. Pongamos, para facilitar la demostración, que $n>m$. 
 
 Entonces, sumamos y restamos a ese módulo cada uno de los $x_i$ entre $n$ y $m$: 
 
 \begin{equation}
  \md{x_n - x_m} = \md{x_n \pm x_{n-1} \pm \dotsb \pm x_{m+1} - x_m} \leq \sum_{i=m}^n \md{x_i - x_{i-1}} \label{eqAC_sum}
 \end{equation}
 
Operamos ahora con cada uno de esos sumandos. Por ejemplo, con $i = n$, vemos que 

\begin{gather*}
\md{x_n - x_{n-1}} = \md{F(x_{n-1}) - F(x_{n-2})} \leq \alpha \md{x_{n-1} - x_{n-2}} = \\
= \alpha \md{F(x_{n-2}) - F(x_{n-3})} ≤ \alpha ^2 \md{x_{n-2}-x_{n-3}} 
\end{gather*}

Si seguimos operando, llegaremos a que $ \md{x_n - x_{n-1}} \leq \alpha^{n-2} \md{x_2-x_1}$. Generalizando, tenemos que

\[ \md{x_i - x_{i-1}} ≤ \alpha^{i-2} \md{x_2-x_1} \]

Aplicando esta fórmula en (\ref{eqAC_sum})
 
\[ \md{x_n-x_m} \leq \left(\alpha^{n-2} + \alpha^{n-3} + \dots + \alpha^{m-1}\right) \md{x_2 - x_1}\]

Esa suma de $\alpha$'s es la suma de una sucesión geométrica de razón $\alpha$. Por lo tanto, la podemos simplificar como \[\sum_{k=m-1}^{n-2} \alpha^k = \alpha^{m-1}\frac{1-\alpha^{n-m}}{1-\alpha} ≤ \frac{\alpha^{m-1}}{1-\alpha} \], y la ecuación nos queda de la forma \[ \frac{\alpha^{m-1}}{1-\alpha}  \md{x_2-x_1} \]. Dado que $\dfrac{\alpha^{m-1}}{1-\alpha}  \convs[][n_0] 0$, tendremos que tomando un $n_0$ suficientemente grande se cumple que 

\[ \frac{\alpha^{m-1}}{1-\alpha}  \md{x_2-x_1} < \varepsilon \]

para un $\epsilon$ arbitrariamente pequeño. Con esto \textbf{demostramos que la sucesión de $x_n$ es de Cauchy} y por lo tanto es convergente a un cierto límite $x_0$.

Tal y como habíamos construido la sucesión, tenemos que 

\begin{equation} \label{eqAC_suc}x_n= F(x_{n-1}) \end{equation}

$x_n$ converge a $x_0$ cuando $n\to\infty$. De la misma forma, como $x_{n-1}$ también converge a $x_0$, está claro que $F(x_{n-1})$ convergerá a $F(x_0)$. Sustituyendo estos dos resultados en (\ref{eqAC_suc}), tenemos que 

\[ x_0 = F(x_0) \]

Hemos demostrado por lo tanto que el límite de esa sucesión que hemos construido \textbf{es un punto fijo} de la función. 

Nos queda \textbf{demostrar ahora que ese punto es único}, y lo haremos por reducción al absurdo:

 Supongamos que existen dos puntos fijos:
 
 \begin{gather*}
 a = F(a)\\
 b= F(b)
\end{gather*}
                     
 Entonces tendríamos que \[ \md{a-b} = \md{F(a)-F(b)}\leq \alpha \md{a-b} \] pero como $\alpha$ es menor estricto que 1, entonces tendríamos que \[ \md{a-b} < \md{a-b} \], lo que es una contradicción.
\end{proof}

El teorema de la aplicación contractiva nos sirve, por ejemplo, para comprobar si hay una solución de una ecuación diferencial ordinaria (EDO).

$$\left.\begin{matrix}y'(x) = f(x,y(x))\\
        y(x_0) = y_0
       \end{matrix}\right\} \leftrightarrow y(x) = y_0 + \int_{x_0}^x f(s,y(s)) ds$$

Podemos definir:

\begin{align*}
y_1(x) &= y_0 + \int_{x_0}^{x} f(s,y_0)ds\\
y_2(x) &= y_0 + \int_{x_0}^x f(s,y_1(s))ds \equiv T(y_1)\\
&\dots\\
y_n &= T(y_{n-1}) =  y_0 + \int_{x_0}^x f(s,y_{n-1}(s))ds\\
¿T(y) &= y?
\end{align*}
Aquí es donde entraría la diferencia entre trabajar en $\real^N$ y un espacio de funciones.

Ejercicio propuesto: Aplicar este argumento a iterativo

$\left. \begin{matrix} y' = y\\
         y(0) = 1 \equiv y_0
        \end{matrix}\right\}$

        
\subsection{Teorema de la función inversa}
En Cálculo I teníamos el siguiente teorema:
\begin{theorem} Sea $\appl{f}{\real}{\real}\; f\in C^1$ y $f'(a) \neq 0$. Entonces $f$ es invertible en un entorno de $f(a)$. 

La inversa es diferenciable en ese entorno, y además $(f^{-1})'(f(a)) = \frac{1}{f'(a)}$
\end{theorem}

El teorema nos daba un resultado \textbf{local} que asegura que existe la inversa, que es diferenciable y además nos daba su fórmula. 

Buscamos ahora lo que ocurre en dimensión $N$. Tomamos $\appl{f}{\real^N}{\real^N}$ y supongamos que en algún abierto $\exists F^{-1}$ y $F,F^{-1} \in  C^1$.

Entonces está claro que

\[ (F\circ \F)(y) = y \implies DF(\F(y))D\F(y) = Id \]
\[ (\F\circ F)(y) = y \implies D\F(F(y))DF(y) = Id \]

Con $\appl{F}{\Omega\subset \real^N}{\real^N}$ queremos probar que existe una aplicación $\appl{G}{V}{U}$ que verifique las siguientes condiciones

\begin{gather*}
G\circ F(\gx) = \gx, \forall\gx\in U \subset \Omega \\
F\circ G(\gy) = \gy, \forall \gy \in V \\
G \text{ diferenciable}
\end{gather*}

\begin{theorem} [Teorema\IS de la función inversa]
\label{TFInv}
Sea $\appl{F}{\Omega\subset\real^N}{\real^N} \text{ con } F\in C^1(\Omega)$.

Supongamos $DF(\ga)$ invertible, $\ga \in \Omega$.

Entonces existe un abierto $V \tlq F(\ga)\in V$, un abierto $U \tlq \ga \in U$ y una inversa local $\appl{G}{V}{U}$.\\
Además, $G$ es diferenciable en $U$ y $DG(y) = \left[DF(\F(y))\right]^{-1}, \forall y \in U$.
\end{theorem}

\begin{proof} Haremos la demostración en varios pasos.

 \paragraph{1: Simplificar la notación}  Queremos invertir $F(x) = y, \gor{y} \sim F(\ga), \gx \sim \gor{a}$.
 
 Llamamos:
 \begin{itemize}
  \item $y = F(\ga) + \gor{z}; \gor{z} \sim \gor{0}$
  \item $y = \ga + \gor{s}; \gor{s} \sim \gor{0}$
\item  $F(\ga+\gor{s}) = F(\ga)+\gor{z}$.
  \end{itemize}
 Definimos además \[ \tilde{F}(\gor{s}) \equiv F(\ga + \gor{s}) - F(\ga) = \gor{z} \] de tal forma que  \[ \tilde{F}(\gor{0}) = \gor{0} \]
 
 Es decir, hacemos una traslación para suponer que $F(\gor{0}) = \gor{0}$.
 
 Por las hipótesis del teorema, sabemos que $\det DF(\gor{0}) \neq 0$
 
 Es claro que resolver para $F$ es exactamente lo mismo que resolver para $CF(\gx) = \gor{y}$, donde C es una matriz invertible.
 
 Tomamos $\tilde{F} = [DF(\gor{0})]^{-1}F$, de tal forma que ganamos la siguiente igualdad

 \[ D\tilde{F}(\gor{0}) = Id \]
 
 \paragraph{2: Formulación como punto fijo.}

 Partimos de $F(\gor{0})=\gor{0}$ y $ DF(\gor{0}) = Id$.
 
 Definimos $f(\gx) = \gx - F(\gx)+\gy$
 
 Entonces resolver $F(\gx) = \gor{y}$ es lo mismo que encontrar un punto fijo $f(\gx) = \gx$. Por lo tanto, ahora nuestro objetivo es probar que $f$ es contractiva para así poder aplicar el teorema de la aplicación contractiva (\ref{thmAC}).
 
 \paragraph{3: Estimar $f$} Empezamos con
 
 \[ Df(\gx) = Id - DF(\gx) \rightarrow Df(\gor{0}) = Id - DF(0) = Id - Id = \begin{pmatrix}  \bigzero \end{pmatrix}  \]
   
 La primera estimación que podemos hacer es sobre el determinante. Si $DF(\gor{0}) = Id$, entonces $\det(DF(\gor{0})) = 1$. Como $F$ es continua, entonces 
 
 \[ \exists \varepsilon_0 >0 \tlq \md{\gx} \leq \varepsilon_0 \implies  \det(DF(\gor{0}))>0 \]
 
  Es decir, en un entorno del $\gor{0}$, el determinante sigue siendo positivo.
  
  
 Ahora estimamos información sobre el diferencial de $f$. Como $F\in C^1$, entonces también se cumple que $f \in C^1$. Como $\appl{f}{\real^N}{\real^N}$, entonces
  
  \[ Df(\gor{0}) = \begin{pmatrix}
                  Df_1(\gor{0}) \rightarrow\\
                  \vdots\\
                  Df_N(\gor{0}) \rightarrow
                 \end{pmatrix} = \begin{pmatrix}  \bigzero \end{pmatrix} \]
                 
 Por tanto $\md{Df_i(\gor{0})} = 0$.
 
 Por continuidad $\exists \varepsilon_i > 0 \tlq \md{x} \leq \varepsilon_i, \implies \md{Df_i(\gx)} <\frac{1}{2N}$. Fijamos un $\varepsilon = \min \{\varepsilon_0,\varepsilon_1,\dotsc, \varepsilon_N\} ,\, i=0,\dotsc,N$.
 
 Entonces, si $\md{x} \leq \varepsilon$ tenemos  que
\begin{gather}
\det(DF(\gx))>0 \label{eqFinv_DetF} \\
\md{Df_i(\gx)}  = \md{\nabla f_i(\gx)} < \dfrac{1}{2N}, i=1,2,...,N \label{eqFinv_Detfi}
\end{gather}
                                            
 \begin{remark} $\varepsilon$ NO depende de $\gy$. \end{remark}

 
  Tomaremos así $\gx\in \gor{B_\varepsilon(\gor{0})}$, donde $\gor{A}$ es el cierre de $A$ (\ref{dfnCierre}).
  
  \paragraph{4: Demostrar que $f$ lleva un cerrado en sí mismo}
  
Es decir, hay que demostrar que \[ \appl{f}{ \gor{B_\varepsilon(\gor{0})}}{ \gor{B_\varepsilon(\gor{0})}} \].
  
  Teniendo $\md{\gor{s}}\leq \varepsilon$ queremos probar $\md{f(\gor{s})}\leq \varepsilon$. Operamos:
  
  \begin{equation}
  f(\gor{s}) = \md{f(\gor{s}) - f(0) + f(0)} \leq \md{f(\gor{s})-f(0)} + \underbrace{\md{f(0)}}_{f(0) = \gor{0} + F(\gor{0}) + \gor{y} = y}
  \label{eqFinv_Des1}
  \end{equation}
Por otra parte  
  
  \[ \md{f(s)-f(0)}^2 = \sum_{i=1}^N (f_i(\gor{s})-f_i(0))^2 \]
  
  Aplicando el teorema del valor medio (\ref{thmTVM}) con un punto $0<\tilde{s}<s$, tenemos que 
  
  \[ f_i(s) - f_i(0) = Df_i(\tilde{s}) \cdot (\gor{s} - \gor{0}) \]
  
  Entonces
  
 \[ \sum_{i=1}^N (f_i(\gor{s})-f_i(0))^2 =\sum_{i=1}^N \left(Df_i(\tilde{s})\cdot\gor{s}\right)^2 = 
 	\sum_{i=1}^N(\pesc{\nabla f_i(\tilde{s}),\gor{s}})^2 
 	\leq \sum_{i=1}^N \md{\nabla f_i(\tilde{s})}^2\md{\gor{s}}^2 \]
 	
 	y usando (\ref{eqFinv_Detfi}) nos queda que 
 	
 	\[ \sum_{i=1}^N (f_i(\gor{s})-f_i(0))^2 ≤ N \frac{1}{4N^2} \md{\gor{s}} \]
  y por lo tanto 
  \[ \md{f(\gor{s}) - f(\gor{0})}^2 \leq \frac{1}{4N} \md{\gor{s}}^2 \]
  
  Usando este resultado recuperamos (\ref{eqFinv_Des1})
 \begin{align*}
  \md{f(s)} &\leq \frac{1}{2\sqrt{N}} \md{\gor{s}} + \md{\gy} \\
  &\leq \frac{1}{2\sqrt{N}}\varepsilon + \md{\gy} \\
  & \leq \frac{\varepsilon}{2} + \md{y} \\
  &< \varepsilon \dimplies \md{y} < \frac{\varepsilon}{2}
  \end{align*}
  
y por lo tanto hemos demostrado que $f$ lleva un cerrado en sí mismo.
 
  Esta última acotación (forzar que $\md{y}< \frac{\epsilon}{2}$) es por la que es un teorema local.
  
  \paragraph{5: $f$ es contractiva en $B_\varepsilon(\gor{0})$} 
  
  Tomamos $\gor{r},\gor{s} \in B_\varepsilon(\gor{0})$
  
  $$\md{f(\gor{r}) - f(\gor{s})} = \left(\sum \left( f_i(\gor{r}) - f_i(\gor{s})\right) ^2 \right)^{\frac{1}{2}}$$
  Aplicando el teorema del valor medio
  $$\left(\sum_{i=1}^N \pesc{Df_i(z_i), (\gor{r}-\gor{s})^2}\right)^{\frac{1}{2}}, z_i\in B_\varepsilon(0) $$
  La misma cuenta de antes:
  $$\md{f(\gor{r}) - f(\gor{s})} \leq \frac{1}{2\sqrt{N}}\md{\gor{r}-\gor{s}}$$
  
  Acabamos de encontrar el $\alpha \in (0, 1)$ que aparecía en el teorema de la aplicación contractiva: \[ \alpha = \frac{1}{2\sqrt{N}} \]
  
  Ahora ya podemos aplicar el teorema de la aplicación contractiva y ya tenemos el punto fijo. Por lo tanto, existen dos vectores $\gx, \gy$ tal que $F(\gx) = \gy$ y por lo tanto existirá también una aplicación $G$ con $x= G(y)$. Vamos a demostrar que
  
  \[ \appl{G}{B_{\frac{\varepsilon}{2}}(0)}{\overline{{B_{\frac{\varepsilon}{2}}(0)}}} \]
  
  Sea $y \in B_{\frac{\varepsilon}{2}}$. Entonces

\[ \md{G(y)} = \md{\gx} = \md{f(\gx)} = \md{f(\gx) \pm f(\gor{0})} \leq \md{f(\gx) - f(\gor{0})} + \md{f(\gor{0})} \] 

Como $f$ es contractiva

\[ \md{f(\gx) - f(\gor{0})} + \md{f(\gor{0})} \leq \frac{1}{2\sqrt{N}} \md{\gx} + \md{\gy} \]

y por lo tanto 
\[ \md{G(y)} \leq \frac{1}{2\sqrt{N}}\varepsilon + \frac{\varepsilon}{2} < \varepsilon \]
  
  Si $\md{G(y)} < \varepsilon$ entonces  $G(y)\in B_{\varepsilon} \implies \appl{G}{B_\varepsilon}{B_\varepsilon}$.
  
  Por lo tanto, podemos concluir que $G(y) = x \dimplies F(x) = y$ con $y\in B_{\frac{\varepsilon}{2}}(\gor{0})\;, x \in B_{\varepsilon}(\gor{0})$
  
  \paragraph{6: Continuidad de $G$}
  
  
  Vamos a ver que pasa con diferencias del tipo $\md{s -G(s)}$. Sea $G(s) = t$ y $s = F(t)$.
  \begin{gather*}
  f(t) = t - F(t) + y\\
  s - G(s) = -G(s) + F(G(s)) = F(t)-t = y-f(t) = y - f(G(s))
  \end{gather*}
 
Consideramos ahora \[ \md{G(u)-G(v)} = \md{G(u)-G(v) -u +u-v+v} \leq \md{u-v} + \md{G(u)-u + v-G(v)} \]

Aplicando el resultado anterior   
 
  \begin{gather*}
\md{u-v} + f(G(u)) - y - [y -f(G(v))] =\\
= \md{u-v} + \md{f(G(u)) - f(G(v))} \leq \\
\leq \frac{1}{2\sqrt{N}} \md{G(u)-G(v)} \leq \frac{1}{2}  \md{G(u)-G(v)}\\
\md{G(u)-G(v)}\leq \md{u-v} + \frac{1}{2}\md{G(u)-G(v)}\\
\md{G(u)-G(v)} \leq 2 \md{u-v}
  \end{gather*}
  
  Por lo tanto $G$ es una función continua uniforme.
  
  \begin{remark}
  En este caso tenemos $\md{G(u)-G(v)} < C\md{u-v} \leftarrow $ \textbf{Espacio de funciones Lipschitz}
  
  Si en cambio $\md{G(u)-G(v)} < C\md{u-v}^\alpha \leftarrow $ \textbf{Espacio de funciones Hölder} ($\alpha<1$).
  \end{remark}
  
  \paragraph{Paso 7: G diferenciable}  Sea $\gy \in B_{\frac{\varepsilon}{2}}(0)$
  
  Aplicamos la definición de diferenciabilidad
  
  \[ \lim_{h \rightarrow \gor{0}}\frac{\md{G(y+h) - G(y) \left[DF(G(y))\right]^{-1} \gor{h}}}{\md{\gor{h}}} = 0 \]
  
  Vamos a intentar trabajar con las $\gx's$ que es donde sabemos todo y no con $\gy's$ que no tenemos ni idea de nada.
  
  \emph{Notación:}
  \begin{align*}
G(\gy) &= \gx & \gor{y} &= F(\gx)\\
G(\gy + \gor{h}) &- G(\gy) = \xi & \gy + \gor{h} &= G(\gx + \gor{\xi})\\
&\downarrow&\ &\downarrow\\
G(\gy + \gor{h}&) = \gx + \xi & \gor{h} &= F(\gx + \gor{\xi}) - F(\gx)
\end{align*}
$$\gor{h} \rightarrow 0 \dimplies \gor{\xi} \rightarrow \gor{0}$$
  Sustituimos con esta notación en la definición:
  
 
  $$\lim_{h \rightarrow \gor{0}}\frac{\md{G(y+h) - G(y) \left[DF(G(y))\right]^{-1} \gor{h}}}{\md{\gor{h}}} = \\$$
  $$...\\$$
  $$\lim_{\xi\rightarrow \gor{0}} \underbrace{\frac{\md{\xi}}{\md{F(\gx + \gor{\xi} - F(\gx)}}}_{A} %&
  \cdot \underbrace{\frac{\gor{\xi} - \left[DF(G(y))\right]^{-1} (F(\gx +\gor{\xi}) - F(\gx)}{\md{\gor{\xi}}}}_{B}\\   $$
  
  Vamos a acotar B aplicando:
  $$\xi = \underbrace{\left[DF(G(y))\right]^{-1}DF(x)}_{=Id} \cdot \xi$$

  $$B = \frac{\md{\left[DF(G(y))\right]^{-1} \left[ -\{F(x+\xi)-F(x) - DF(x)\xi\}\right]}}{\md{\xi}}$$
  
  $$B\leq C \frac{\md{F(x+\xi) - F(x) - DF(x)\xi}}{\md{\xi}} \convs[\text{F dif.}][\xi][0] C\cdot 0 \rightarrow 0$$
  
  Donde $C$ es la norma de la matriz. 

 
  Ahora vamos a probar que $A$ está acotado, probando que $A>0$, lo que acota el inverso:
  
  $$\frac{1}{A} = \frac{\md{F(x+\xi)-F(x)}}{\md{\xi}} = \frac{\md{F(x+\xi) - F(x) - DF(x)\xi + DF(x)\xi}}{\md{\xi}}$$
  $$\geq \frac{\md{DF(x)\xi}}{\md{\xi}} - \underbrace{\frac{\md{F(x+\xi) - F(x) - DF(x)\xi}}{\md{\xi}}}_{\rightarrow 0}$$
  
 
  $$\md{\frac{DF(\gx)\xi}{\md{\xi}}} = \md{DF(\gx)\times\gor{v}}, \md{\gor{v}} = 1$$
  $$\text{Definimos } M(\gor{v}) = \md{DF(\gx)\times\gor{v}}, \text{ para } \gor{v} \in S^1$$
  
  $S^1$ es la esfera de radio 1, un conjunto compacto.
  
  
  Aplicamos: $M$ continua, definida en un conjunto compacto $\implies$ $M$ alcanza su máximo y su mínimo (\ref{thmCompactoMax}). En concreto, si su mínimo es $\delta$ tenemos:
  
  $$ M(v)\geq \delta > 0 \implies \frac{1}{A}\geq \delta > 0 \implies A\leq C $$ 
  
  
  $$\left.\begin{matrix}A \leq C\\B \rightarrow \gor{0}\end{matrix}\right\} \implies \lim_{h \rightarrow \gor{0}}\frac{\md{G(y+h) - G(y) \left[DF(G(y))\right]^{-1} \gor{h}}}{\md{\gor{h}}} = 0$$
  
  y por lo tanto $G$ es diferenciable, y la expresión de su diferencial es \[ DG(y) = \left[DF(\F(y))\right]^{-1} \]
\end{proof}

\paragraph{Ejemplo del teorema de la función inversa.}
Sea $F(x,y) = (x^2-5y^2,4xy)$.

Tomamos $(x_0,y_0)$.

¿$F$ invertible en un entorno de $F(x_0,y_0)$?

Calculamos Df:

$$Df = \begin{pmatrix}
        2x&-10y\\
        4y & 4x 
       \end{pmatrix}
$$

$$\det(DF) = 8x^2 + 40y^2 \neq \text{ si } (x,y) \neq (0,0)$$

Cuando el determinante sea 0, significa que no puedo aplicar el teorema, por tanto, no sé si la función es invertible o no. Aplicamos los fundamentos. ¿Es inyectiva la función?

No es inyectiva en ningún entorno del (0,0).

El hecho de que el teorema sea local nos puede llevar a confusiones. Por ejemplo, sea \[ F (r,\sigma) = (rcos(\sigma),r\sen(\sigma)), r \in [0,1], \sigma \in [0,4\pi]\]

Entonces hay dos soluciones para la inversa:

\[ F^{-1} (0,\frac{1}{2}) = \left\{\begin{matrix}2\pi+\frac{\pi}{2}\\\frac{pi}{2}\end{matrix}\right. \]

Lo que hay que hacer es partir de un punto del conjunto de salida. El teorema dice que escogido un punto del conjunto de salida, existe un entorno en el conjunto de llegada en el que se puede definir la función inversa. \textbf{Hay que acotar también el conjunto de llegada.}

Otro ejemplo es considerar  $g(x) = f(x) + \varepsilon x$ siendo $f(s) =\left\{\begin{matrix}x^2sin\left(\frac{1}{x}\right)& \text{ si } x\neq0\\0 &\text{si } x=0\end{matrix}\right.$
Esta función $g \notin C^1$. $g$ es derivable ($g'(0) = \varepsilon>0$) pero la derivada no es continua. Se deja como ejercicio para el lector la comprobación.      

\subsection{Teorema de la función implícita}

Tomemos el caso particular de una superficie en $\real^3$. Puede venir dada de dos formas:
\begin{itemize}
 \item Conjunto de nivel: $F(x,y,z) = 0$
 \item Gráfica: $z=f(x,y) \rightarrow F(x,y,z) = f(x,y)-z$
\end{itemize}

¿Existe alguna forma de expresar $F(x,y,z)$ de la forma $z=f(x,y)$? Pensando en el ejemplo de la esfera: $F(x,y,z) = x^2+y^2+z^2+1$, se puede expresar de dos formas:

\[ z = \pm \sqrt{1 - x^2 - y^2} \]

¿Cuál es la condición que necesitamos para poder despejar $z$? Supongamos que sabemos despejar $z=f(x,y)$, entonces tenemos: $F(x,y,f(x,y)) = 0$. Derivando implíctamente

\[ \underbrace{\frac{\partial}{\partial x} [F(x,y,f(x,y)]}_{\dpa{F}{x} + \dpa{F}{z}\cdot\dpa{f}{x}}= \underbrace{\frac{\partial}{\partial y} [F(x,y,f(x,y)]}_{\dpa{F}{y} + \dpa{F}{z}\cdot\dpa{f}{y}} = 0 \]

Si $f$ es diferenciable tenemos:
$$\dpa{f}{x}(x,y) = - \frac{\dpa{F}{x}(x,y,f(x,y))}{\dpa{F}{z}(x,y,f(x,y))}$$
$$\dpa{f}{y}(x,y) = - \frac{\dpa{F}{y}(x,y,f(x,y))}{\dpa{F}{z}(x,y,f(x,y))}$$
Necesitamos entonces que $\displaystyle\dpa{F}{z}\left(x,y,f(x,y)\right) \neq 0$. Veamos cómo extrapolar esto de forma general con tres variables.

\begin{theorem}[Teorema\IS de la función implícita (en $\real^3$)] 

\label{TFImp}

Sea $\appl{F}{\Omega\subset\real^3}{\real}, F\in C^1$, con

\[ F(a,b,c) = 0 \] y \[ \dpa{F}{z}(a,b,c)\neq 0 \]

Entonces existe una función $\appl{f}{\omega}{\real}$ con $(a,b)\in \omega \implies  f(a,b) = c$ de tal forma que

$F(x,y,f(x,y)) = 0, \forall(x,y)\in \omega$

\end{theorem}

\begin{proof} Vamos a realizar el siguiente proceso:

Definimos $H(x,y,z) = (x,y,F(x,y,z))$. Esta función aplana la superficie del conjunto de nivel, porque $(x,y,z) \in S \implies F(x,y,z)=0 \implies H(x,y,z) = (x,y,0)$

Esta función nos aplana la superficie del conjunto de nivel, pero lo que estamos buscando es desde un espacio bidimensional (conjunto de partida de $f$) llegar a la superficie de nivel, que es una superficie de $\real^3$, el espacio de partida de $F$. Entonces vamos a buscar $H^{-1}$.

El problema de esta función es que $\appl{H}{\real^3}{\real^3}$, pero toda la información que necesitamos es la tercera componente de $H^{-1}$.

Según el teorema de la función inversa, existen abiertos $U, V$ con $(a,b,c) \in U$ y  $V\in (a,b,0)$; y una única inversa local 

\[ \appl{\inv{H}}{V}{U} \]

De tal forma que 

\[ \inv{H}(u,v,w) = (x,y,z) \dimplies (u,v,w) = H(x,y,z) = (x,y,F(x,y,z)) \]

Es decir \[ \inv{H} (u,v,w) = (u,v,g(u,v,w)) \]

donde $g$ es la función única que depende de $(u,v,	w)$, y no es más que la tercera componente de la inversa que hemos construido. Demostremos que existe: 
 \begin{gather*}
 H\circ H^{-1} = Id \\
 H(H^{-1}(u,v,w)) = H(u,v,g(u,v,w)) = (u,v,w)
 \end{gather*}
 
 En particular si $w=0$:
 $$(u,v,0) = H(u,v,g(u,v,0)) = (u,v,F(u,v,(g(u,v,0)))$$
 Conclusión: Hemos encontrado una única $g$, tal que $F(u,v,g(u,v,0)) = 0$.
 
  Notación: $g(u,v,0) = f(u,v)$
  
  $F(u,v,f(u,v)) = 0, (u,v) \in \text{Proyección sobre el plano horizontal de la superficie}$
 
\end{proof}

\begin{theorem}[Teorema\IS de la función implícita (caso general)]

Dadas $n$ ecuaciones, $n+m$ incógnitas, consideramos 

$$\appl{F}{\Omega\subset\underbrace{\real^M}_{x,y} \times \underbrace{\real^N}_{z}}{\real^N}$$

Con la notación $F = (F_1,...,F_N)$

Los elementos de $\real^M\times\real^N$ los llamamos $(x_1,x_2,...,x_m,y_1,y_2,...,y_n) = (\gor{x},\gy)$.

Supongamos $F\in C^1$. Sea $\ga \in \real^M, \gor{b} \in \real^N \tlq (\gor{a},\gor{b})\in \Omega , F(a,b)=0$

Supongamos $D_yF(\ga,\gb)$ no singular, $\det(D_yF)\neq 0$ siendo:
$$D_yF = \begin{pmatrix}
          \dpa{F_1}{y_1}&...&\dpa{F_n}{y_1}\\
          \vdots&\ddots&\vdots\\
          \dpa{F_n}{y_n}&\cdots&\dpa{F_n}{y_n}
         \end{pmatrix}$$
         
\paragraph{Entonces:} Existen abiertos $\omega \subset \real^M, \Theta \in\real^n$, con $\ga \in \omega, \gb \in \Theta$ y una única función: \[ \appl{g}{\omega\subset\real^M}{\Theta \subset\real^N}, g\in C^1(\omega) \]

Tal que:
\begin{itemize}
 \item $g(\ga) = \gb$
 \item $F(\gx,g(\gx)) = \gor{0}, \forall \gx \in \omega$
 \item $\displaystyle Dg(\gx) = - \left[D_yF(\gx,g(\gx))\right]^{-1} \cdot D_xF(\gx,g(\gx))$
\end{itemize}
\end{theorem}

\begin{proof}
 
 Definimos: $H(\gx,\gy) = (\gx,F(\gx,\gy))$. Esta función $\appl{H}{\real^{m+n}}{\real^{n+m}}, H\in C^1$. Además $H(\ga,\gb) = (\ga,F(\ga,\gb)) = (\ga,\gor{0})$
 
 $$DH_{(a,b)} = \left(
                 %\underbrace{1&0&0&...&0}_{m}&\underbrace{0&...&0}_{n}\\
                 %\underbrace{0&1&0&...&0}_{}&\underbrace{0&...&0}_{}\\
                 \begin{array}{c|c}
                 I_{m\times x} &  0_{m\times x}\\
                 \hline
                 D_x F_{n\times m} &  D_y F_{n\times n}
                 \end{array}
                 \right)
$$

$\det(DH(\ga,\gb)) = \det(D_y(\ga,\gb)) \neq 0$. Aplicando el teorema de la función inversa podemos invertir $H$ en un entorno de $H(\ga,\gb)$.Además, por el teorema también sabemos la unicidad y que $H \in C^1$.

Problema: identificar $g$ dentro de $H^{-1}$.

$$\underbrace{H(\gx,\gy)}_{\equiv (u,v)} = (\gx,F(\gx,gy))$$
$$\implies H^{-1}(u,v) = (x,y) = (u,?)$$implícita
Notación: $H^{-1}(u,v) = (u,\tilde{g}(u,v))$
 
Estamos interesados en la restricción $H^{-1} (u,0) = (u,\tilde{g}(u,0))$. Podemos comprobar que $\tilde{g}(u)\in\real^N, u\in \real^m$.

Tenemos
\[H^{-1} (u,0) = (u,g(u)) \dimplies (u,0) = H(u,g(u)) = ( u,F(u,g(u)))\]


Ya tenemos los 2 primeros apartados, vamos a por la fórmula:

\[F(\gx,g(\gx)) = \gor{0}\]
\[D_x[F(\gx,g(\gx))] = (\gor{0})\]
\[D_x[F(\gx,g(\gx))] = \begin{pmatrix}
                        \dpa{F_1(\gx,g(\gx))}{x_1}&\cdots&\dpa{F_1(\gx,g(\gx))}{x_n}\\
                        \vdots&\ddots&\vdots\\
                        \dpa{F_n(\gx,g(\gx))}{x_n}&\cdots &\dpa{F_n(\gx,g(\gx))}{x_n}
                       \end{pmatrix}
\]
Donde: $$\dpa{F_1(\gx,g(\gx))}{x_1} = \dpa{F_1}{x_1} + \sum_{k=1}^{n} \dpa{F_1}{y_k}\cdot \dpa{y_k}{x_1} \sim (D_y F_1 \rightarrow) \cdot \begin{pmatrix}
\dpa{g}{x_1}\\
\vdots\\
\dpa{g}{x_n}
\end{pmatrix}
$$

Aplicando esto obtenemos:

\[ D_x[F(\gx,g(\gx))] = (D_xF) + (D_yF)(Dg) = D_xF(D_x(\gx,g(\gx))) + D_yF(\gx,g(\gx))\cdot Dg(\gx)\]

Despejando obtenemos la fórmula que nos decía el teorema.

 
\end{proof}

\subsection{Ejemplos:}
\paragraph{Ejemplo: Hoja 3, problema 16}

$$z^3lg(xy) + 2x^2 + 2y^2 +z^2 + 8xz - z + 8 =0$$

Demostrar que la ecuación anterior define DOS funciones $z = f_1(x,y), z = f_2(x,y)$ en un entorno de $(x,y)=(1,1)$.

\paragraph{Solución: }

Esto no contradice al teorema (que tiene unicidad) porque tenemos que anclar los puntos con los que vamos a trabajar en los que $F(x,y,z) = 0$. Asíque vamso a ver $F(1,1,z) = \underbrace{z^3lg(xy)}_{0} + 2 + 2 + z^2 + 8z -z +8 = 0 \implies z^2+7z+12 = 0 \implies $ 2 soluciones.

Tenemos que ver qué pasa con $\displaystyle \dpa{F}{z}(1,1,z_i)$.

\[\dpa{F}{z} = 3z^2lg(xy) + 2z -1\]
\[\dpa{F}{z}(1,1,z_i) = 2z_i + 7 \neq 0\]

Ahora estamos en condiciones de aplicar el teorema.

También nos pide hallar el desarrollo de Taylor de orden 1 para $f_1$.

\[f_1(x,y) = f_1(1,1) + \underbrace{\dpa{f_1}{x}(1,1)(x-1) + \dpa{f_1}{y}(1,1) (y-1)}_{\pesc{\nabla f_1(1,1),(x-1,y-1)}} + err\]
¿Cómo calcular las derivadas? Recurrimos al teorema y sustituyendo $z=f_1(x,y)$
\[(f_1(x,y))^3 lg(xy) + 2x^2 + 2y^2 + (f_1(x,y))^2 + 8x(f_1(x,y))-(f_1(x,y))+8 = 0\]
Derivada con respecto a $x$:
\[ 0 = (f_1(x,y))^2\dpa{f_1}{x}(x,y)\cdot lg(xy) + (f_1(x,y))^3 \frac{1}{x}+4x + 2(f_1(x,y))\dpa{f_1}{x}(x,y) + 8x\dpa{f_1}{x}(x,y) - \dpa{f_1}{x}(x,y)\]
\[...\]
Vamos a evaluar en $(1,1)$ lo que tenemos donde $f_1(x,y) = z_1$ y $f_2(x,y) = z_2$ y despejamos $\dpa{f_1}{x}$.

\paragraph{Ejemplo: Hoja 3, ejercicio 14}

Demostrar que existe una 'unica f... con $f(0,0) = 0$.
$$e^{f(x,y)} = (1+xe^{f(x,y)})(1+ye^{f(x,y)})$$
¿Podemos despejar $z = f(x,y)$?. (Es lo mismo que demostrar que existe una función $$e^{z} = (1+xe^z)(1+ye^z)$$

Definimos $F(x,y,z) = e^z - (1+xe^z)(1+ye^z) = 0$

Comprobamos que $F(0,0,0) = 1-1 = 0$.

Hipótesis del teorema: $\appl{F}{\real^3}{\real}$ con $F(0,0,0) = 0, F \in C^{\infty}$
Nos falta comprobar \[\dpa{F}{z}(0,0,0) \neq 0\]. (Este paso en el caso general es el determinante de $D_zF(0)$)

Entonces podemos aplicar el teorema 
\[\implies \exists ! f \tlq F(x,y,f(x,y)) = 0\]

\paragraph{Ejemplo: Hoja 3, ejercicio 18}

Estudiar si es posible despejar $u(x,y,z), v(x,y,z)$ en las ecuaciones:

\[\left\{\begin{matrix} xy^2+xzu+yv^2 &= 3\\ xyu^3+2xv-u^2v^2 &= 2\end{matrix}\right.\]
En un entorno de $(x,y,z) = (1,1,1)$

Vamos a tener que definir una 
\[\appl{F}{\real^5}{\real^2}\]
\[(x,y,z,u,v) \rightarrow F(x,y,z,u,v)= (xy^2+xzu+yv^2-3,xyu^3+2xv-u^2v^2-2)\]
Podemos comprobar fácilmente que $F\in C^{\infty}$ y $F(1,1,1,1,1) = ... = (0,0)$.

Para poder despejar u,v tenemos que evaluar $D_{(u,v)}F$ en $(1,1,1,1,1)$.

\[D_{(u,v)} = \begin{pmatrix} \dpa{F_1}{u}&\dpa{F_1}{v}\\ \dpa{F_2}{u} &\dpa{F_2}{v}\end{pmatrix} 
= \begin{pmatrix} xz & 2yv\\3xyu^2-2uv^2 & 2x-2u^2v \end{pmatrix} = ... = \begin{pmatrix} 1&2\\ 3&0 \end{pmatrix}\]

Tenemos $\det D_{(u,v)} = -6 \neq 0$. Entonces estamos en las hipótesis para utilizar el teorema y garantizar que en un entorno del punto $(1,1,1)$ blablabla.
COMPLETAR

Vamos a calcular (porque lo pide el enunciado) \[\dpa{u}{x},\dpa{v}{x},\dpa{v}{z}\].

Como el teorema garantiza que existe, derivamos implícitamente:

Vamos a derivar implícitamente respecto a $x$ el sistema:
\[\left\{\begin{matrix} y^2+zu+xzu_x + y 2v v_x = 0 \\ yu^3+xy3u^3u_x  + 2v + 2xv_x - 2uu_xv^2-2u^2vv_x = 0 \end{matrix}\right.\]
Donde $u_x = \dpa{u}{x}$.

\emph{Sabemos:} si $(x,y,z) = (1,1,1) \implies (u,v) = (1,1)$

Sustiuyendo:
\[\left\{\begin{matrix}1+1+u_x+2v_x &= 0 \\ 1+3u_x+2+2v_x-2u_x-2v_x &= 0\end{matrix}\right.\]
\[\left\{\begin{matrix}u_x(1,1,1) + 2v_x(1,1,1) &= -2\\ u_x(1,1,1) &= -3 \end{matrix}\right.\]

Faltaría  calcular $\dpa{v}{z}$

\paragraph{Ejercicio propuesto: Calcular $\dpa{u}{x^2}$}

\paragraph{Ejercicio} Demostrar T.F.Inversa a partir del T.F Implícita
Tenemos:
\begin{gather}
 \appl{F}{\real^N}{\real^N}\\
 F\in C^1\\
 F(\ga) = \gb\\
 \det DF(\ga) \neq 0
\end{gather}
¿Podemos despejar $F(\gx) = \gy$ para $\gx$ en un entorno de $\ga$ $\gy$ en un entorno de $\gb$?

$F(\gx) = \gy$ es lo mismo que $H(\gx,\gy) = 0$ con $H(\gx,\gy) = F(\gx)-\gy$.

$\appl{H}{\real^N\times \real^N}{\real^N}. H\in C^1$ y $H(\ga,\gb) = 0$. Tenemos el punto de partida en el que anclar el teorema. Queremos hallar $\gx$ como función de $\gy$ en la ecuación $H(\gx,\gy) = 0$.

Necesitamos para aplicar el teorema: $\det D_x H(\gx,\gb) \neq 0$.

\obs $D_x H  = D_x F \neq 0$ (por (4))

\paragraph{Conclusión:} $\exists f(\gy) \tlq H(f(\gy),\gy) = \gor{0} \equiv F(f(\gy)) - \gy = \gor{0} \equiv F(f(\gy)) = \gy$.

Ahora tenemos que ver que la composición en el otro sentido también nos da la identidad.

\[f(F(\underbrace{f(\gy)}_{v}) = f(\gy) \implies f(F(v)) = v\]
