\section{Teoría de la integración}

\subsection{Repaso}

\subsubsection{Dimensión 1}

En dimensión 1, la teoría de integración la basábamos en la integral de Riemann. Tomábamos una partición de $[a,b]$ $\mathcal{P} = a = t_0 < t_1<...<t_k = b$ y a cada subintervalo de esa partición $[t_i,t_{i+1}] \rightarrow \begin{array}{cc} 
M_i &= sup\{f(x)\tq x\in []\\
m_i &= inf \{f(x)\tq x\in []\end{array}$ 

Definíamos \[ \alpha = \mathbb{U}(f,\mathcal{P}) = \sum_{i=0}^{k-1} M_i(t_{i+1}-t_i)\]
\[ \beta = \mathbb{L}(f,\mathcal{P}) = \sum_{i=0}^{k-1} \sum_{i=0}^{k-1} m_i(t_{i+1}-t_i)\]

\begin{defn}[Función\IS integrable] Se dice que $f$ es integrable en $[a,b]$ si y sólo si 

\[ \mathbb{U}(f,\mathcal{P}) = \beta \] 

La notación es \[ \int_a^b f(x)dx = \alpha = \beta \]
\end{defn}

\begin{theorem}
Si $f$ continua en $[a,b]$ entonces  $f$ es integrable en $[a,b]$.
\end{theorem}

El recíproco es \textbf{falso}. Se ve fácilmente si $f$ es una función escalonada, por ejemplo.

\begin{theorem} Sea $f$ integrable en $[a,b]$.

Tomamos $\mathcal{P}_h$ una partición de $[a,b] \tlq \max_{i} \abs{t_{i+1}-t_i} = h$ , es decir, $h$ mide el trozo más grande de la partición.

\[ \lim_{h\rightarrow 0} \sum_{i=1}^{k-1} f(s_i)(t_{i+1}-t_i) = \int_a^bf(x)dx \] para \textbf{cualquier} elección $s_i\in[t_i,t_{i+1}]$	

\end{theorem} 

\begin{theorem}[Cambio\IS de variable (dimensión 1)]
Sea $g$ un difeomorfismo, y suponemos que es creciente (la cuenta es análoga si fuese decreceiente); de tal forma que transforma el intervalo $[a,b]$ en el intervalo ($[g(a),g(b)] = [A,B]$).

\todo{Dibujar ese intervalo}

Entonces

\[ \int_A^B f(x)\,dx = \int_{\inv{g}(A)}^{\inv{g}{B}} f(g(t)) g'(t)\,dt \]

Es decir, tenemos cambio de los límites, de las variables y también de la medida.
\end{theorem}

\subsubsection{Dimensión N > 1}

En dimensión mayor que uno teníamos una serie de generalizaciones, y empezábamos con el teorema de Fubini que nos permite el cambio en el orden de integración:

\begin{theorem}[Teorema\IS de Fubini] Sean $A$ y $B$ dos conjuntos con ciertas propiedades \todo{Propiedades} que no sé cuáles son. Entonces

\[ \int_A \int_B f(x,y)\,dx\,dy = \int_B \int_A f(x,y)\,dy\,dx \]

En la práctica, tendremos que cambiar los límites de integración para que la integral exprese el mismo área.
\end{theorem}

Por ejemplo, si tenemos la integral \todo{Dibujito} \[ I_1 = \int_0^2\int_{y^2}^{2y} f(x,y)\,dx\,dx \] y queremos invertirla cambiamos los límites de integración: \[ I_1 = \int_0^4 \int_{\frac{x}{2}}^{\sqrt{x}} f(x,y)\, dy\,dx \]

Otro ejemplo: \todo{¿A que no sabes qué voy a pedir aquí?} estudiemos \[ I_2 = \int_a^bf(x)\,dx = \int_a^b\int_0^{f(x)}dx\,dy \]. Si probamos a cambiar el orden de integración, tenemos que

\[ I_2 = \int_0^M\int_{A(y)}dx\,dy \]

Donde $M$ es el máximo de $f$ y $A(y) = \{ x\in [a,b] \tq f(x) ≥ y \}$. La longitud de ese conjunto $A$ se denomina la medida, y entonces podemos expresar

\[ I_2 = \int_0^\infty \abs{\{x \tq f(x) > y\}}\,dy \] ya que la medida del conjunto será $0$ cuando $y > M$.

Curiosamente, hemos pasado de una integral de Riemann a otra con una expresión distinta, la llamada \textbf{integral de Lebesgue}\index{Integral\IS de Lebesgue}. Esto pertence al campo de la \textbf{teoría de la medida}, y permite estudiar conjuntos extraños y más monstruos y engendros varios.

\begin{theorem}[Cambio\IS de variable (dimensiones superiores)]
Dados unos conjuntos $D,\,D^\ast$ con $\appl{\Phi}{D}{D^\ast}$ un difeomorfismo, y $\gx \in D;\; \gy \in D^\ast$. Entonces

\[ \int_{D^\ast} f(\gy)\,d\gy = \int_D f(\Phi(\gx)) \abs{\det \dpa{\gy}{\gx}}\,d\gx \]
\end{theorem}

\subsubsection{Integración en curvas}

\begin{defn}[Curva\IS $C^1$]  Se denomina curva en $\real^n$ a una aplicación \begin{align*}
\appl{\gamma}{[a,b]\subset \real&}{\real^n} \\
t&\to \gamma(t) = (x_1(t),\dotsc,x_n(t))
\end{align*}

y con $\gamma\in C^1$.

También exigiremos que la curva sea un \textbf{camino regular}\index{Camino! regular}, es decir que

\[ \gamma'(t) \neq \gor{0} \;\forall t \]

y entonces obligamos a que tenga tangente en todo punto.
\end{defn}

Para calcular la longitud aplicamos las ideas básicas del cálculo integral: \textit{troceamos} el intervalo $[a,b]$, y aproximamos cada uno de esos trozos por un segmento. Calculamos la suma de la longitud de esos segmentos, hacemos tender la anchura de los \textit{trozos} y si converge, la longitud se puede medir.

\begin{theorem}[Longitud\IS de una curva] Dada una partición $\mathcal{P} = \{ a= t_0 < t_1 < \dotsb < t_k = b\}$ definimos \[ \abs{\mathcal{P}} = \max_i \abs{t_{i+1} - t_i} \] y la longitud $L$ de la curva.

\[ L(\sigma) = \lim_{\abs{\mathcal{P}}\to 0} \underbrace{\sum_{i=0}^{k-1} \md{\sigma(t_{i+1}) - \sigma(t_i)}}_{=S(\sigma,\mathcal{P})} \]

SI $\sigma\in C^1$, entonces $L(\sigma)$ existe y además

\[ L(\sigma) = \int_a^b \md{\sigma'(t)}\,dt \]
\end{theorem}

Este teorema responde a una idea con respecto al cambio de variable: si tomamos $\Gamma$ como la curva que queremos integrar, entonces se puede expresar

\[ L(\sigma) = \int_\Gamma 1\,d\sigma = \int_a^b \md{\sigma'(t)}\,dt  \]

donde $\md{\sigma'(t)}$ es el cambio en la medida correspondiente.

\begin{proof} Por pura pereza y no escribir más \footnote{A mí también me parece bien} suponemos \[ \sigma(t) = (x(t),y(t)) \]. Tenemos que

\begin{gather*}
 S(\sigma,\mathcal{P}) = \sum_{i=0}^{k-1} \md{\sigma(t_{i+1}) - \sigma(t_i)} = \\
 = \sum_{i=0}^{k-1}\sqrt{\left(x(t_{i+1}-x(t_i)\right)^2 + \left(y(t_{i+1}-y(t_i)\right)^2}
 \end{gather*}
 
 Por el Teorema del valor medio (\ref{thmTVM1var}) tenemos que
 
 \begin{gather*}
 \left(x(t_{i+1}-x(t_i)\right)^2  = x'(s_i^1)^2(t_{i+1}-t_i)^2 \\
 \left(y(t_{i+1}-y(t_i)\right)^2  = y'(s_i^2)^2(t_{i+1}-t_i)^2
 \end{gather*}
 
 Entonces
 
 \[  S(\sigma,\mathcal{P})  = \sum_{i=0}^{k-1} \sqrt{x'(s_i^1)^2 + y'(s_i^2)^2} (t_{i+1}-t_i) \]
 
 No podemos simplificar porque no es seguro que $s_i^1 = s_i^2$. Si fueran el mismo punto, serían sumas de Riemann y habríamos terminado.
 
<<<<<<< HEAD
 Como tenemos una función continua ($\sigma'$) y estamos trabajando en un intervalo cerrado y acotado ($[a,b]$) podremos reducir la expresión a $\sqrt{x'(t_i)^2 + y'(t_i)^2}$ y que gracias a la continuidad uniforme la diferencia con $\sqrt{x'(s_i^1)^2 + y'(s_i^2)^2}$ se puede hacer todo lo pequeña que queramos.
  
\end{proof}

\subsubsection{Parametrización por longitud de arco}

\[ \appl{\sigma}{[a,b]}{\real^n} \]

con $\sigma$ curva $C^1$ y regular. Entonces

\[ L(s) = \int_a^s \md{\sigma'(t)}\,dt \]

y por el Teorema Fundamental del Cálculo

\[ L'(s) = \md{\sigma'(s)} \]

Al imponer que la curva sea regular, entonces $\sigma'(s)\neq 0$ y por lo tanto existe la inversa $\inv{L}$. Si tenemos entonces un $\tau ∈ [0,L(b)]$, entonces $S=\inv{L}(\tau) ∈ [a,b]$.\wtf

Definimos

\begin{gather*}
\sigma^\ast = \sigma \circ \inv{L} \\
\sigma^\ast = \sigma(\inv{L}(\tau)) \\
(\sigma^\ast)'(\tau) = \sigma'(\inv{L}(tau)) (\inv{L}(\tau))' = \sigma'(\inv{L}(\tau)) \frac{1}{L'(\inv{L}(\tau))} = \sigma'(s) \frac{1}{L'(s)} = \frac{\sigma'(s)}{\md{\sigma'(s)}}
\end{gather*}

Es decir, hemos conseguido una parametrización con velocidad constante $\md{(\sigma^\ast)'} = 1$ y por lo tanto

\[ L(\tau) = \int_0^\tau \md{(\sigma^\ast)'}\,ds = \int_0^s 1\,ds = s \]

\subsection{Elemento de área}

Supongamos que estamos en $\real^3$. Podemos hablar de la longitud de una variedad de dimensión 1, del área de una de dimensión 2 o del volumen de una de dimensión 3. Ahora bien, ¿qué ocurre cuando pasamos a dimensiones superiores? La denominación será la siguiente

\begin{itemize}
\item \textbf{1-variedad} Longitud
\item \textbf{k-variedad} $1<k<N$ Área
\item \textbf{N-variedad} Volumen
\end{itemize}

Para calcular esas \textit{cosas} empezaremos partiendo del área de un parelelepípedo.

Definiremos el paralelepípedo como, dados $k$ vectores independientes $\{\gv_i\} \subset \real^N$, 

\[ P_k = \sum_{i=1}^k \lambda_i\gv_i\;\lambda_i \in [0,1] \]
=======
 Como tenemos una función continua ($\sigma'$) y estamos trabajando en un intervalo cerrado y acotado ($[a,b]$) podremos reducir la expresión a $\sqrt{x'(t_i)^2 + y'(t_i)^2}$ y que gracias a la continuidad uniforme la diferencia con $\sqrt{x'(s_i^1)^2 + y'(s_i^2)^2}$ se puede hacer todo lo pequeña que queramos, llegando a hacer coincidir en el límite $s_i = t_i$, por lo que podríamos escribir:
 
\[  S(\sigma,\mathcal{P})  = \sum_{i=0}^{k-1} \sqrt{x'(t_i)^2 + y'(t_i)^2} (t_{i+1}-t_i) \]
  o también:
\[  S(\sigma,\mathcal{P})  = \sum_{i=0}^{k-1} \sqrt{x'(t_{i+1})^2 + y'(t_{i+1})^2} (t_{i+1}-t_i) \]


La forma matemática de escribir este argumento:

Notación: $G(s,t) = \sqrt{(x'(s))^2 + (y'(s))^2}$

Llamamos:
\[(1) = \sum G(s_i^1,s_i^2)(t_{i+1}-t_i)\]
\[(1) = \sum G(t_i^1,t_i^2)(t_{i+1}-t_i)\]

Vamos a estudiar $\abs{(1)-(2)}$

\[0\leq \abs{(1)-(2)} \leq ... \leq \sum_i \abs{G(s_i^1,s_i^2)- G(t_i^1,t_i^2)}(t_{i+1}-t_i)\]

$(s_1^1, s_1^2 \in [t_i,t_{i+1}]$ por tanto:
\[\md{s_i^1,s_i^2) - (t_i,t_i)} = ... \leq \sqrt{2}\md{\mathcal{P}}\]

Aplicamos $G$ continua en $[a,b]\times [a,b]$ (compacto) $\implies G$ uniformemente continua.

...

\todo{Ha borrado la pizarra...}
\end{proof}

\obs \textbf{Falso} en general si la curva es sólo continua

\paragraph{Ejemplo:}

\[\appl{\sigma}{[0,1]}{\real^2}\]

\todo{A dibujar!}

Descripción gráfica: entre $\frac{1}{2^k}$ y $\frac{1}{2^{k+1}}$ y formo un triángulo isósceles con esos 2 puntos y de altura hasta la bisectriz.

Sea el triangulo K:

\todo{Dibujo}

\[Longitud =L_k = ... ??? ... = \frac{1}{2k+1} \{\sqrt{\frac{1}{4k^2}+1} + \sqrt{\frac{1}{4k^2+4} +1} \} \ge \frac{1}{2k+1}\]
\[\text{Longitud total } = \sum L_k \ge \sum_k \frac{1}{2k+1} = \infty\]

Este mostruito no tiene longitud. ¿Y si en vez de rectas tomamos sinusoides? Esa función si debería ser $C^1$ pero la longitud es $\infty$. Entonces la curva no puede ser $C^1$ (sería un contrajemplo del teorema)

\begin{defn}[Curva \IS rectificable]
Si el límite $\lim_{\abs{\mathcal{P}}\rightarrow 0} S(\sigma,\mathcal{P}) $ existe $\implies \sigma$ es \textbf{rectificable}
\end{defn}

>>>>>>> 74c985d5892b51ebe4d909f156f55cda1b032d25
