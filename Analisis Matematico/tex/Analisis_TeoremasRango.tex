\section{Teoremas del Rango}

¿Dónde está escondida la indentidad en un sistema de ecuaciones (lineales o no)?

\begin{theorem}[Teorema\IS del Rango (1)]
Sea \[\appl{F}{\Omega\subset \real^M\times\real^N}{\real^N}\]
\[F\in C^1, (\ga,\gb) \in \Omega \text{ con } \ga \in \real^M, \gb \in \real^N\]
Supongamos que $DF(\ga,\gb)$ tiene rango $n$ (máximo posible).

Entonces: $\exists$ un abierto $\omega$ con $(\ga,\gb) \in \omega$ y una función \[ \appl{G}{\omega\subset\real^M\times\real^N}{\real^M\times\real^N}, G\in C^1 (\omega)\] con inversa local diferenciable, tal que \[F\circ G(\gor{u},\gor{v}) = \gor{v}, \forall(\gor{u},\gor{v})\in \omega\]

\end{theorem}

\begin{proof}
 Reordenamos las variables de manera que \[DF(\ga,\gb) = \begin{array}{c|c} A &B \end{array}\]
 \[A \in M_{m\times m}, B = D_yF(\ga,\gb), \det B \neq 0\]
 
 Definimos \[H(\gx,\gy) = (\gx,F(\gx,\gy))\]
 Calculamos:
 \[DH = \left(\begin{array}{c|c}
         I & \\
         \underbrace{D_xF}_{m\times n} & \underbrace{D_yF}_{n\times n}
        \end{array}\right)\]
        
        
  Estudiamos el determinante: $\det DH = \det D_yF \neq 0$ debido a la reordenación de las variables que hemos hecho al principio. Además, $F\in C^1 \implies H \in C^1$. Estamos en condiciones de aplicar el teorema de la función inversa. $\exists$ una inversa LOCAL $H^{-1}$

  \[H(x,y) = (u,v) \dimplies (x,y) = H^{-1}(u,v) \equiv (\underbrace{G_1(u,v)}_{\real^M},\underbrace{G_2(u,v)}_{\real^N})\]
  
  \begin{gather*}
H(H^{-1}(u,v)) = (u,v)\\
H(G_1(u,v),G_2(u,v))\\
= (G_1(u,v),F(G_1(u,v),G_2(u,v)))
  \end{gather*}
 Tomando $G(u,v) = (G_1(u,v),G_2(u,v))$ obtenemos que $F\circ G(u,v) = v$, para $(u,v)$ en un entorno del punto $(\ga,\gb)$
 
\end{proof}

\begin{theorem}[Teorema\IS del Rango (2)] \label{TR2}
 \[\appl{F}{\real^N}{\real^N\times\real^M}\]
 \[F \in C^1, \ga \in \Omega\]
 
 Rango de $DF(\ga) = n$ (máximo).
 
 Entonces, $\exists \omega $ abierto en $\real^N\times\real^M$, con $F(\ga) \in \omega$.
 
 Y una función $ G\in C^1(\omega), \appl{G}{\real^M\times\real^N}{\real^M\times\real^N}$
 
  $G$ tiene inversa local diferenciable tal que $(G\circ F) (\underbrace{\gx}_{\real^N}) = (\underbrace{\gx}_{\real^N},\underbrace{\gor{0}}_{\real^M})$. 
\end{theorem}

\begin{proof}
 \[DF(\ga) = \begin{pmatrix}
              \dpa{F_1}{x_1} &\dots &\dpa{F_1}{x_n}\\
              \vdots&\ddots&\vdots\\
              \dpa{F_n+m}{x_1} &\dots & \dpa{F_n+m}{x_n}
             \end{pmatrix}
\]
Reordemamos las $F_j$ de la siguiente manera:
\[DF(\ga) = \overbrace{\left(\begin{array}{c}
		  \det\neq 0\\
		  \hline
                  \text{Resto}
                  \end{array}\right)}^{n \text{ columnas }}\begin{array}{c} \} \text{ n filas}\\ \} \text{ m filas}\end{array}\]

Definimos: 
\begin{align*}
H: \real^N\times\real^M &\longrightarrow \real^N\times\real^M\\
(\gx,\gy) &\longrightarrow \underbrace{F(\gx)}_{\real^N} + (\underbrace{\gor{0}}_{\real^N}, \underbrace{\gy}_{\real^M}) = (F(\gx),\gy)
\end{align*}

\[F(\gx) = H(\gx,\gy) = (F_1(\gx),F_2(\gx),...,F_n(\gx), F_{n+1}(\gx)+\gy_1,...,F_{n+m}(\gx)+\gy_m)\]
\[DH = \left(\overbrace{D_xF}^{\text{ N columnas}} \; \begin{array}{|c}0  \\ \hline Id \end{array}  \right) \begin{array}{c} \}\text{ n filas} \\  \}\text{ m filas} \end{array} \implies \det DH \neq 0
\]

Estamos en condiciones de aplicar el teorema de la función inversa $ \implies \exists$ una inversa local $\invers{H}$ tal que
\[\invers{H} \circ H(u,v) = (u,v), u\in \real^N,v\in\real^M\]
En particular:
\[\left. \begin{array}{c} \invers{H} \circ H (\gor{u},\gor{0}) = (\gor{u},\gor{0})\\
H(\gor{u},\gor{0}) = (F(\gor{u}) + (\gor{0}, \gor{0})) \end{array} \right\} \implies \invers{H} \circ H(\gor{u}+(\gor{0},\gor{0})) = \invers{H}(F(\gor{u})) = (\gor{u},\gor{0})\]
REVISAR: A partir del $\implies$ soy yo, no azorero.
\end{proof}

\begin{theorem}[Teorema\IS del rango (resultado general)]
\[\appl{F}{\real^N}{\real^K}, F\in C^1\]
Con rango $DF = p < \min\{n,k\}$

Existe $\Gamma (u_1(\gx),...,u_n(\gx)) = (\underbrace{u_1,...,u_p,0,...,0}_{k})$, siendo $\Gamma = \Phi \circ F \circ \invers{\Phi}$.
 
 
 Siendo $U$ un cambio de variable.
\end{theorem}

\paragraph{Hoja 3: Problema 21}

\[\left\{\begin{array}{cc}
   F_1\equiv x^2+z^2+2xz-2x-2z+1&=0\\
   F_2\equiv x^2+4y^2+4z^2+4xy+4xz+8xy&=0
  \end{array}\right.\]
  ¿Es depejable en función de z en un entorno de $(x,y) = (0,-1), z=1$?
  
Solución:

1) Ver que el punto safisface las ecuaciones.

2) Ese sistema es como definir:
\[\appl{F}{\real^3}{\real^2}, F\in C^1, F(0,-1,1) = (0,0)\]

3) Calculamos el determinante de $DF(0,-1,1)$ y vemos que da $0$. 

Conclusión: no podemos aplicar el teorema, pero pensando un poco vemos que 

\[F_1 = (x+z)^2 -2(x+z) + 1 = (x+z-1)^2\]
\[F_1 = 0 \dimplies x=1-z\]
\[F_2(x,y,z) = F_2(1-z,y,z) = (1-z)^2 + 4y^2 + 4z^2 + 4(1-z)y + 4(1-z)z+8yz\]
\[ =... = a(z)y^2+b(z)y+c(z) \implies y=\frac{-b\pm \sqrt{...}}{2a}\]
¿Cual de las 2 soluciones escoger? Sabemos (por el enunciado) que si $z=1$ entonces $y=-1$. Escogeremos la solución a la que dandole el valor $z=1$ nos de $-1$.

\paragraph{Hoja 3: Problema 22}

\[\begin{array}{cc}
   x^3+z^3y^3+z &= 0\\
   cos(xyz)+sen(z)-1 = 0
  \end{array}\]
 ¿Es despejable en función de $z$ en un entorno de $(x,y) = (0,0), z=0$.
 
 Solución:
 
 1) El punto es solución del sistema.
 
 2) Definimos \[\appl{F}{\real^3}{\real^2}, F\in C^1, F(0,0,0)=(0,0)\]
 
 3) Calculamos el determinante de $DF(0,0,0)$ y vemos que da $0$.
 
 Conclusión: no podemos aplicar el teorema\\
 Supongamos que si se puede despejar. Entonces tendríamos algo de la forma:

 \begin{gather*}
 \left\{\begin{array}{cc}
   [x(z)]^3+z^3[y(z)]^3+z &= 0\\
   cos(x(z)y(z)z)+sen(z)-1 &= 0
  \end{array}\right\}\rightarrow\\
  \,\\
   \left\{ \begin{array}{cc}
               3[x(z)]^3x'(z) + 3z^2[y(z)]^2 + z^32y(z)y'(z) + 1 &=0\\
               -sen(x(z)y(z)z) \cdot\{...\} + cos(z) &= 0
              \end{array}\right\}
              \equiv\\\,\\
              \left\{
              \begin{array}{cc}
               0+1&=0\\
               0+1&=0
              \end{array}\right.
 \end{gather*}
  Esto demuestra que no pueden existir las derivadas.