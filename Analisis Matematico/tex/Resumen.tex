\begin{theorem} [Teorema\IS de la función inversa][] (ver \ref{thmInv})
Sea $\appl{F}{\Omega\subset\real^N}{\real^N} \text{ con } F\in C^1(\Omega)$.

Supongamos $DF(\ga)$ invertible, $\ga \in \Omega$. Entonces existe un abierto $V \tlq F(\ga)\in V$, un abierto $U \tlq \ga \in U$ y una inversa local $\appl{G}{V}{U}$. Además, $G$ es diferenciable en $U$ y 

\[ DG(y) = \left[DF(\F(y))\right]^{-1}, \forall y \in U \]
\end{theorem}

\begin{theorem}[Teorema\IS de la función implícita (caso general)][] (ver \ref{thmFImp}) Dadas $n$ ecuaciones, $n+m$ incógnitas, consideramos 

$$\appl{F}{\Omega\subset\underbrace{\real^M}_{x,y} \times \underbrace{\real^N}_{z}}{\real^N}$$

Supongamos $F\in C^1$. Sea $\ga \in \real^M, \gor{b} \in \real^N \tlq (\gor{a},\gor{b})\in \Omega , F(a,b)=0$. Supongamos $D_yF(\ga,\gb)$ no singular, $\det(D_yF)\neq 0$. Entonces existen abiertos $\omega \subset \real^M, \Theta \in\real^n$, con $\ga \in \omega, \gb \in \Theta$ y una única función: \[ \appl{g}{\omega\subset\real^M}{\Theta \subset\real^N}, g\in C^1(\omega) \]

que cumple

\[ g(\ga) = \gb;\; F(\gx,g(\gx)) = \gor{0}\; \forall \gx \in \omega \]
\[ Dg(\gx) = - \left[D_yF(\gx,g(\gx))\right]^{-1} \cdot D_xF(\gx,g(\gx)) \]
\end{theorem}

 \begin{defn}[Subvariedad\IS diferenciable][](ver \ref{defSubDif}) $M \subset \real^{M+N}$ es una subvariedad diferenciable si y sólo si para todo punto $\ga$ existe un entorno abierto $U \subset \real^{N+K}$ con $\ga \in U$ de tal forma que $U \cap M = \{\gx\in U\tq F(\gx) = \gor{0}\}$ \textbf{para alguna función} $F\in C^1(U)$, $\appl{F}{U\subset \real^{N+K}}{\real^K}$ cuya diferencial $DF$ tenga rango máximo (en este caso, rango $k$).
  \end{defn}
  
  \begin{defn}[Homeomorfismo][] (ver \ref{defHomeomorfismo}) $\appl{\Phi}{\Omega\subset\real^N}{\real^{N+K}}$, con $\Omega$ abierto, es un homeomorfismo sobre su imagen si y sólo si la restricción $\appl{\Phi}{\Omega}{\Phi(\Omega)}$ es continua y tiene una inversa continua.
  \end{defn}
  
\begin{defn}[Parametrización\IS local][] (ver \ref{defParametrizacion})
Diremos que $\appl{\Psi}{\omega \subset \real^N}{\real^{N+K}}, \Psi \in C^1(\omega)$, es una parametrización local si y sólo si el rango de su diferencial es máximo y es un homemorfismo sobre su imagen.
\end{defn}

\begin{defn}[Difeomorfismo][]
$\Psi$ es un difeomorfismo si y sólo si $\Psi\in C^1,\exists \Psi^{-1}, \tlq \Psi^{-1}\in C^1$
\end{defn}

\begin{theorem} (ver \ref{thmSubvariedades}) Son equivalentes: $M$ es una subvariedad, $M$ tiene parametrizaciones locales y $M$ se puede expresar como un sistema de ecuaciones.
\end{theorem}

\paragraph{Espacios tangentes} (ver \ref{thmCaractTangente1} y \ref{CaractSubv_2}) Si $M$ es una subvariedad, para obtener el espacio tangente en $\ga$
\begin{itemize}
\item $M$ definida como $\{F = 0\}$,  $T_{\ga} M = \ker(DF(\ga))$.
\item $M$ dada por una parametrización $\Phi$ tal que $\Phi(\gor{u_0}) = \ga$, $T_{\ga} M = \img D\Phi(\gor{u_0})$.
\end{itemize}

\begin{theorem}[Teorema\IS de los multiplicadores de Lagrange][] (ver \ref{thmMultLagrange}) Para hallar los extremos locales de una función $F$ restringida a una subvariedad $M$ dada como $M = \{ G = \gor{0} \}$, resolvemos el sistema
\[ 
\begin{cases}
\grad F(\gx) &= λ \grad G(\gx) \\
G(\gx) &= 0
\end{cases}\]
\end{theorem}

\begin{theorem}[Cambio\IS de variable][]
Dados unos conjuntos $D,\,D^\ast$ con $\appl{\Phi}{D}{D^\ast}$ un difeomorfismo, y $\gx \in D;\; \gy \in D^\ast$. Entonces

\[ \int_{D^\ast} f(\gy)\,d\gy = \int_D f(\Phi(\gx)) \abs{\det \dpa{\gy}{\gx}}\,d\gx \]
\end{theorem}

\begin{defn}[Integral sobre una subvariedad][] Dada una subvariedad $M$ a través de una parametrización $Φ$
\[ \int_{Φ(D)} f \id A = \int_D f(Φ(s)) \sqrt{\det\left(\pesc{Φ_{s_i},Φ_{s_j}}\right)} \id s \]
\end{defn}

\paragraph{Integración de campos}

\[ \int_{\Gamma} \vf \id{σ} = \int_a^b \pesc{\vf(σ(t)),σ'(t)}\id t \]
\[ \int_{Φ(D)} \vf \id S = \iint\limits_D\pesc{\vf(Φ(u,v)),T_u×T_v}\id{u,v} \]

\begin{theorem}[Teorema\IS de Green][]
Sea $\Gamma$ una curva simple en $\real^2$. Llamamos $D$ al interior de $\Gamma$. Sea $(P,Q)$ el campo con $P,Q\in C^1(D)$. Entonces

\[ \int_{\Gamma^+} (P,Q)\,d\sigma = \iint_D \frac{\partial Q}{\partial x}-\frac{\partial P}{\partial y}\,dx\,dy\]
\end{theorem}

\begin{theorem}[Teorema\IS de Stokes][]
Dada $\Gamma$ una curva cerrada simple en $\real^3$ que es el borde de una superficie $S$, tenemos que 

\[ \int_{\Gamma^+}\vec{F}d\sigma =\iint\limits_{S^+}\rot \vec{F}dS \]

\end{theorem}

\begin{theorem}[Teorema\IS de Gauss][]
Sea $S$ una superficie cerrada que encierra una región $\Omega$, y $\vec{F}\in C^1(\Omega)$.

\[ \int\int_{S^+} = \iiint\limits_\Omega \dv \vec{F}\, dx\, dy\, dz \]
\end{theorem}

\paragraph{Operaciones con formas diferenciales}

\begin{gather*}
\dif f = \sum_i^N \dpa{f}{x_i} \df x_i \\
\omega\y\beta = \sum_{I,J} F_IG_J \df{x_I,x_J} (\text{k+s-forma}) \\
\dif ω = = \sum_I \df{F_I,x_I} \\
(\pb ω)(\gor{s})[\sample[\gor{v}][k]] =  \omega(T(s)) [DT(s)\gv] = \sum f_i (T(s))  \dif T_i [\gv] 
\end{gather*}

\begin{itemize}
\item La diferencial exterior de un campo interpretado como 1-forma es la 2-forma asociada a la \textbf{divergencia}.
\item La diferencial exterior de un campo interpretado como 2-forma es la 3-forma asociada al \textbf{rotacional}.
\item El producto exterior de 2 campos interpretados como 2-formas nos da el campo asociado al \textbf{producto vectorial}.
\end{itemize} 

\begin{theorem}[Teorema\IS de Stokes][]
Sea $M$ una subvariedad compacta, orientable, con frontera relativa $\partial  M$. Entonces

\[\int_{\partial  M^+}\omega = \int_M \dif \omega \]

\end{theorem}