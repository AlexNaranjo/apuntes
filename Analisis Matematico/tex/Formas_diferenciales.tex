\section{Lenguaje de las formas diferenciales}

\paragraph{0-formas}
\index{Formas\IS 0}

Son funciones escalares definidas en un abierto de $\real^n$
\[\appl{f}{\Omega\subset\real^N}{\real}\]

Operaciones habituales:
\begin{itemize}
\item Suma: sí
\item Producto: sí
\item Composiciones: no (porque no cuadran las dimensiones)
\end{itemize}

\paragraph{1-formas}
\index{Formas\IS 1}

Sea $\mathcal{C} = \{e_1,e_2,...,e_n\}$ la base canónica en $\real^N$.

Sea $L$ una aplicación lineal
\[\appl{L}{\real^N}{\real}\]

Que recordamos que cumplen:
\[ L(\gx+\gy) = L(\gx)+L(\gy); L(\lambda\gx) = \lambda L(\gx)\]

Definimos $\gy\in\real^N \leadsto \gy = \displaystyle\sum_1^n y_i e_i$, con lo que \[L(\gy) = \sum y_i L(e_i)\]

Entonces \[\left.\begin{array}{cc}
v_i = L(e_i)\\
y_i = P_i(\gy)
\end{array}\right\} \rightarrow L(\gy) = \sum_i v_iP_i(y)\]

Siendo $P_i$ las proyecciones, una base del espacio dual.

\textbf{Notación:}

$P_i \equiv dx_i$.

$dx_i[\gy] \equiv P_i(\gy) = y_i$

Entonces, dado un $\gv$ podemos construir 
\[L \equiv \sum_i^N v_idx_i\]

\[L[\gy] = \sum_i^N v_idx_i[\gy] = \sum_i^N v_iy_i\]

\begin{defn}[1-forma]
\[\omega(\gx)= \sum_1^N F_i(\gx) dx_i\]

\begin{itemize}
\item Se evalúa en $\gx\in\real$
\item Actúa sobre $\gy\in\real^N$ 
\end{itemize}

Es decir, \[\omega(\gx)[\gy] = \left(\sum F_i(\gx)dx_i\right)[\gy] = \sum F_i(\gx)dx_i[\gy] = \sum F_i(\gx)y_i\]
\end{defn}

Indicaremos con paréntesis el punto en el que estamos evaluando, y con corchetes el punto en el que estamso actuando.

\textbf{Operaciones:}
\begin{itemize}
\item Sumar: sí (lo razonable)
\item Multiplicar: por una función escalar sí está definida.
\end{itemize}


\paragraph{Ejemplo:}

Supongamos $f$ una función escalar (una 0-forma).

\[\grad f(\gx) = \left( \dpa{f}{x_i}(\gx)\right)\, i=1,...,N\]

Nos podemos construir una 1-forma desde el gradiente

\[\dpa{f}{x_i}(\gx)dx_i \]

A esta 1-forma en particular la llamaremos $df(\gx)$.

¿Utilidad? Ya la veremos, pero es una forma de escribir el producto escalar.
\[\pesc{\grad f(\gx),\gy} = df(\gx)[\gy]\]


\paragraph{2-formas}
\index{Formas\IS 2}

Punto de partida: Aplicaciones \textbf{bilineales alternadas}

\[\appl{\Phi}{\real^N\x\real^N}{\real}\]

Que cumplen \begin{itemize}
\item $\Phi([\gu,\gv]) = - \Phi([\gv,\gu]) \implies \Phi(\gu,\gu)=0$
\item $ \Phi([\gu+\gv,\gw]) = \Phi ([\gu,\gw]) + \Phi([u,w])$
\item$\Phi([\lambda \gu,\gv]) = \lambda \Phi([\gu,\gv])$
\end{itemize}

Consecuencias:

\begin{itemize}
\item $\Phi(\gor{r}, \gor{s}+\gor{t}) = \Phi(\gor{r}+\gor{s}) + \Phi(\gor{r}+\gor{t})$
\item $\Phi(\gu,\mu\gv) = \mu\Phi(\gu,\gv)$
\end{itemize}


\paragraph{Ejemplo} en $\real^3$ para facilitar las cuentas.

\[\Phi(\gu,\gv) = \Phi(u_1e_1+u_2e_2+u_3e_3,v_1e_1+v_2e_2+v_3e_3)\]
Aplicando las propiedades anteriores obtenemos:

\begin{gather*}
\overbrace{u_1v_1\Phi(e_1,e_1)}^{\equiv 0} + u_1v_2\Phi(e_1,e_2) + u_1v_3+\Phi(e_1,e_3)+\\
u_2v_1+\Phi(e_2,e_1)+u_2v_2+\Phi(e_2,e_2)+u_2v_3+\Phi(e_2,e_3)+\\
u_3v_1\Phi(e_3,e_1)+u_3v_2+\Phi(e_3,e_2)+u_3v_3+\Phi(e_3,e_3) = \\
\underbrace{(u_1v_2-u_2v_1)}_{\left|\begin{matrix}
u_1&u_2\\v_1&v_2
\end{matrix}\right|}\overbrace{\Phi(e_1,e_2)}^{C_1}+(u_1v_3-u_3v_1)\Phi(e_1,e_3)+(u_2v_3-u_3v_2)\Phi(e_2,e_3)
\end{gather*}

Hemos demostrado que \[\Phi(\gu,\gv) = C_1B_{12}(\gu,\gv) + C_2B_{13}(\gu,\gv) + C_3B_{23}(\gu,\gv)\]

\subparagraph{Notación:} $B_ij = dx_i\y dx_j$

\[dx\y dx_j [\gu,\gv] = \det \begin{pmatrix}
u_i&u_j\\v_i&v_j
\end{pmatrix} = \det \begin{pmatrix}
dx_i[\gu]&dx_j[\gu]\\dx_[\gv]&dx_j[\gv]
\end{pmatrix}\]

\begin{defn}[2-forma]
\[\beta = \sum_{i,j=1}^N F_i(\gx) dx_i\y dx_j\]
\begin{itemize}
\item Se evalúan en puntos $x\in\real^N$
\item Actúan sobre pares de vecotres $[\gu,\gv]\in\real^N\x\real^N$.
\end{itemize}

Es decir:

\[\beta(\gx)[\gu,\gv] = \sum F_{ij}(\gx) dx_i\y dx_j[\gu,\gv] = \sum F_{ij} \det \begin{pmatrix}
u_i&v_i\\u_j&v_j
\end{pmatrix}\]

\emph{Ojo} El cambio del orden (en el determiante)es aposta por la segunda propiedad de las 2 formas
\end{defn}


\subsubsection{K-Forma}

Vamos a dar una definición general de una k-forma.

Elementos básicos:
\[dx_{i_1} \y dx_{i_2}\y...\y dx_{i_k}[\gu^1,\gu^2,...,\gu^k] = \det\begin{pmatrix}
u_{i_1}^1 & ... & u_{i_k}^1\\
\vdots & \ddots & \vdots\\
u_{i_1}^k & ... & u_{i_k}^k
\end{pmatrix}\]

\paragraph{K-forma}

\[
\sum_{i_1,...,i_k=1}^N F_{i_1,...,i_k}(\gx)dx_{i_1} \y ... \y dx_{i_k}
\]

\begin{itemize}
\item Se evalúan en puntos $\gx\in\real^N$
\item Actúa sobre grupos de $K$ vectores.
\end{itemize}

\obs $i_j = i_s \implies dx_{i_j}\y dx_{i_s} = 0$

Esto nos dice que en $\real^N$, teniendo $K$-formas (con $K<N$) tenemos $\comb{N}{K}$ combinaciones distintas.

\obs Si $K>N$ y $\omega$ es una $k-forma \implies \omega \equiv 0$


\paragraph{Ejemplo:} En $\real^3$.

\begin{itemize}
\item 0-forma $\leadsto f(x,y,z) = 0$
\item 1-forma $\leadsto f_1(x,y,z)dx + f_2(x,y,z)dy + f_3(x,y,z)dz$
\item 2-forma $\leadsto g_1(x,y,z)dydz + g_2(x,y,z)dzdx + g_3(x,y,z)dxdy$
\item 3-formas $\leadsto h(x,y,z)dxdydz$
\end{itemize}

\index{Orden cíclico}
\emph{Ojo} Al cambio en la 2-forma, que es $dzdx$. Esto es para seguir el \textbf{orden cíclico} (por temas de la orientación). Esto es $x\to y \to z \to x$


\obs Las funciones escalares las podemos interpretar como 0-formas y como 3-formas.

Los campos vectoriales los podemos interpretar como 1-formas y también como 2-formas.

\paragraph{Notación}
Para escribir un conjunto de subíndices $\{i_1,i_2,...,i_k\} \equiv I$

También acortaremos  $dx_{i_1} \y dx_{i_2} \y ... \y dx_{i_k} \equiv dx_I$. 

La definición quedaría $\displaystyle \sum_I F_I(\gx)dx_I$

\subsection{Operaciones}
Siempre se puede multiplicar por 0-formas y sumar (formas del mismo orden). Estas operaciones son triviales porque son operaciones internas.

Vamos a definir las operaciones externas:

\begin{defn}[Producto \IS exterior (de k-formas)]
Sea \[\omega = \sum_I F_I dx_I (k-forma\in\real^N)\]
\[\beta = \sum_J G_j dx_J (s-forma\in\real^N)\]

\[\omega\y\beta = \sum_{I,J} F_IG_J dx_I\y dx_J (k+s-forma)\]
\end{defn}

\obs Si $K+S>N \implies \omega\y\beta=0$


\paragraph{Ejemplo} $\real^3$

Sea \[\omega = f_1(x,y,z)dx + f_2(x,y,z)dy + f_3(x,y,z) dz\]
\[\beta= g_1(x,y,z)dx + g_2 (x,y,z) dy + f_3 (x,y,z) dz\]

Vamos a calcular $\omega\y\beta$

\[
\omega\y\beta  = f_1g_1dx\y dx + f_1g_2dx\y dy + f_1g_3dx\y dz + f_2g_1dy\y dx + f_2g_2dy\y dy + f_2g_3dy\y dz+ f_3g_1dz\y dx+f_3g_2dz\y dy+f_3g_3dz\y dz
\]
Tachamos los que sean 0 ($dx\y dx = 0$) y tenemos cuidado con el orden cíclico.

\[
(f_2g_3-f_3g_2)dy\y dz + (f_3g_1-f_1g_3)dz\y dx + (f_1g_2 - f_2g_1) dx \y dy
\]

Partiendo de 2 campos vectoriales que eran 1-formas hemos llegado a una 2-forma. 

\obs Acabamos de llegar al producto vectorial de $\real^3$ 

\[\overrightarrow{F}\x\overrightarrow{G} = 
\left((f_2g_3-f_3g_2),(f_3g_1-f_1g_3),(f_1g_2 - f_2g_1)\right)\]

\begin{defn}[Diferencial \IS exterior (de k-formas)]

\[d(\sum_I F_i(\gx)dx_I) = \sum_I \underbrace{dF_I}_{1-forma} \overbrace{\y}^{Prod.ext} \underbrace{dx_I}_{k-forma}\]
\end{defn}

\paragraph{Ejemplo} en $\real^3$

Sean $G = (g_1,g_2,g_3)$

$\omega = $

Vamos a calcular $d\omega$

\[d\omega dg_1\y dy\y dz + dg_2 \y dz\y dz + dg_3 \y dx \y dy = COMPLETAR = \dpa{g_1}{y} + \dpa{g_2}{y}+\dpa{g_3}{z}dx\y dy\y dz\] Que es la divergencia!