\section{Lenguaje de las formas diferenciales}

\paragraph{0-formas}
\index{Formas\IS 0}

Son funciones escalares definidas en un abierto de $\real^n$
\[\appl{f}{\Omega\subset\real^N}{\real}\]

Operaciones habituales:
\begin{itemize}
\item Suma: sí
\item Producto: sí
\item Composiciones: no (porque no cuadran las dimensiones)
\end{itemize}

\paragraph{1-formas}
\index{Formas\IS 1}

Sea $\mathcal{C} = \{e_1,e_2,...,e_n\}$ la base canónica en $\real^N$.

Sea $L$ una aplicación lineal
\[\appl{L}{\real^N}{\real}, L(\gx+\gy) = L(\gx)+L(\gy); L(\lambda\gx) = \lambda L(\gx)\]

Definimos $\gy\in\real^N \leadsto \gy = \displaystyle\sum_1^n y_i e_i$, con lo que \[L(\gy) = \sum y_i L(e_i)\]

Entonces \[\left.\begin{array}{cc}
v_i = L(e_i)\\
y_i = P_i(\gy)
\end{array}\right\} \rightarrow L(\gy) = \sum_i v_iP_i(y)\]

Siendo $P_i$ las proyecciones, una base del espacio dual.

\textbf{Notación:}

$P_i \equiv dx_i$.

$dx_i[\gy] \equiv P_i(\gy) = y_i$

Entonces, dado un $\gv$ podemos construir 
\[L \equiv \sum_i^N v_idx_i\]

\[L[\gy] = \sum_i^N v_idx_i[\gy] = \sum_i^N v_iy_i\]

\begin{defn}[1-forma]
\[\omega(\gx)= \sum_1^N F_i(\gx) dx_i\]

\begin{itemize}
\item Se evalúa en $\gx\in\real$
\item Actúa sobre $\gy\in\real^N$ 
\end{itemize}

Es decir, \[\omega(\gx)[\gy] = \left(\sum F_i(\gx)dx_i\right)[\gy] = \sum F_i(\gx)dx_i[\gy] = \sum F_i(\gx)y_i\]
\end{defn}

Indicaremos con paréntesis el punto en el que estamos evaluando, y con corchetes el punto en el que estamso actuando.

\textbf{Operaciones:}
\begin{itemize}
\item Sumar: sí (lo razonable)
\item Multiplicar: por una función escalar sí está definida.
\end{itemize}


\paragraph{Ejemplo:}

Supongamos $f$ una función escalar (una 0-forma).

\[\grad f(\gx) = \left( \dpa{f}{x_i}(\gx)\right)\, i=1,...,N\]

Nos podemos construir una 1-forma desde el gradiente

\[\dpa{f}{x_i}(\gx)dx_i \]

A esta 1-forma en particular la llamaremos $df(\gx)$.

¿Utilidad? Ya la veremos, pero es una forma de escribir el producto escalar.
\[\pesc{\grad f(\gx),\gy} = df(\gx)[\gy]\]


\paragraph{2-formas}
\index{Formas\IS 2}

Punto de partida: Aplicaciones \textbf{bilineales alternadas}

\[\appl{\Phi}{\real^N\x\real^N}{\real}\]

Que cumplen \begin{itemize}
\item $\Phi([\gu,\gv]) = - \Phi([\gv,\gu])$
\item $ \Phi([\gu+\gv,\gw]) = \Phi ([\gu,\gw]) + \Phi([u,w])$
\item$\Phi([\lambda \gu,\gv]) = \lambda \Phi([\gu,\gv])$
\end{itemize}

