\section{Ejercicios de repaso}
\subsection{Repaso de integración}
\subsubsection{Cambio de variables}

D, región acotada en el 1 cuadrante limitada por:

\[\begin{array}{cc}
xy&=1\\xy&=2\\\frac{y}{x} &=1\\\frac{y}{x}&=4
\end{array}\]

Cuya represencación gráfica es:

\todo{Dibujo}

\textbf{Demostrar:}\[\iint_D f(xy)\id{x,y} = \log 2 \int_1^2f(u)\id{u}\]

Para ello definimos un cambio de variable de la siguiente manera:

\begin{gather*}
xy=u\\
\frac{y}{x}=v
\end{gather*}

Vemos que se simplifica porque $u\in[1,2],v\in[1,4]$. La región en este caso queda:

\todo{Dibujo}

Aplicando este cambio de variable a la integral tenemos:

\[\iint_D f(xy)\id{x,y} = \int_1^2\int_1^4 f(u) \det \left(\dpa{(x,y)}{(u,v)}\right)\id{v,u}\]

Necesitamos despejar $x$ e $y$ en función de $u,v$. En este caso tenemos: 
\begin{gather*}
x=+\sqrt{\frac{u}{v}}\\ y= + \sqrt{uv}
\end{gather*}

¿Para qué? Para poder calcular el determinante de la matriz jacobiana, que en este caso $\displaystyle\det = \frac{1}{2v}$

\[\int_1^2\int_1^4 f(u)\cdot  \frac{1}{2v}\id{v,u} = \frac{1}{2} \int_1^2 f(u) \left.\log v\right|_{v=1}^4 \id{u} = \log 2 \int_1^2 f(u) \id u \]

\textbf{Hemos recordado los 3 pasos al cambiar de variables:}
\begin{itemize}
\item Cambio en la variable
\item Cambio en los límites de integración
\item Cambio en la medida (determiante del Jacobiano)
\end{itemize}


\paragraph{Coordenadas cilíndricas y esféricas}

Sea $\Omega$ la región contenida en el $1^{er}$ octante limitada por:

\begin{gather*}
z=x^2+y^2\\
1=x^2+y^2
\end{gather*}

Queremos escribir la integral en cartesianas, en cilíndricas y en esféricas

\[Vol = \iiint_{\Omega} \id{x,y,z}\]


\paragraph{Cartesianas}
1) Pintamos la proyección en el plano YZ para ver entre qué valores se mueve la $x$. Para ello tenemos que distinguir 2 regiones:

\todo{Dibujo}

\[Vol = \iint_{1} \int_{PAR}^{CIL} \id{x,y,z} + \iint_2 \int_0^{CIL} \id{x,y,z}\]

En ambos, la $y$ se mueve entre 0,1 y la z entre la parábola y 1


\[Vol = \int_{PAR}^1\int_0^1 \int_{PAR}^{CIL} \id{x,y,z} + \int_{PAR}^1\int_0^1 \int_0^{CIL} \id{x,y,z}\]


Pero este camino es un poco complicado. Es más natural integrar $\id{z,y,x}$

\[Vol = \iint_C \int_0^{PAR} \id{z,y,x} = \int_0^1 \int_0^{\sqrt{1-x^2}} \int_0^{x^2+y^2}\id{z,y,x}\]

\paragraph{Cilíndricas}

\begin{gather*}
x=r\,cos\,\theta\\
y=r\,sen\,\theta\\
z=z
\end{gather*}
En este caso, lo natural es escribir la integral en el orden $\id{z,\theta,r}$.

\[Vol = \int_0^1\int_0^{\frac{\pi}{2}} \int_0^{r^2} r \id{z,\theta,r}\]

La $r$ aparece porque es el cambio en la medida.

\paragraph{Esféricas}

\begin{gather*}
x = \rho cos\,\theta\,sen\,\phi\\
y = \rho sen\,\theta\,sen\,\phi\\
z = \rho cos\,\phi
\end{gather*}

\[Vol = \int_{\frac{\pi}{4}}^{\frac{\pi}{2}} \int_0^{\frac{\pi}{2}} \int_{PAR}^{CIL}\rho\id{\rho,\theta,\phi}\]

El $\rho$ aparece por el cambio en la medida.

Sustituyendo en las ecuaciones de las superficies tenemos:
\begin{gather*}
PAR \,: \, \rho = \frac{cos\phi}{sen^2\phi}\\
CIL \,:\, \rho = \frac{1}{sen\phi}
\end{gather*}

\textbf{Entonces}

\[Vol = \int_{\displaystyle\frac{\pi}{4}}^{\displaystyle\frac{\pi}{2}} \int_0^{\displaystyle\frac{\pi}{2}} \int_{\frac{cos\phi}{sen^2\phi}}^{\frac{1}{sen\phi}}\rho\id{\rho,\theta,\phi}\]


\begin{example}
en $\real^2$.

Sea \[F(x,y) = \left(\frac{y}{x^2+y^2}, \frac{-x}{x^2+y^2}\right)\]

\textbf{1)}
Queremos calcular la integral sobre $C_1$, una fircunferencia centrada en el origen.
\[C_1 = \{(cos\,\theta,sen\,\theta),\theta\in[0,2\pi]\}\]

Entonces: $\sigma'(\theta) = (-sen\,\theta,cps(\theta)$

Con la derivada vemos que la orientación es la negativa.

\[\int_{C_1^+} \vf d\sigma = -\int_0^{2\pi} \pesc{\vf\circ\sigma,\sigma'}d\theta\]

\textbf{2)}
¿Lo mismo en una circunferencia no centrada en el origen?

Sale $0$, por el 3)

\textbf{3)} 
¿Lo mismo en un monstruito?

Podemos aplicar el teorema de \textbf{Green} si no podemos parametrizar la curva.

\[\int_{C_3^+} \vf d\sigma_3 = \iint_{D_3} \dpa{Q}{x} - \dpa{P}{y} \id{x,y}\]

Teniendo cuidado con la orientación. En el monstruito propuesto en clase, seguíamos la orientación negativa.

\[\int_{C_3^+} \vf d\sigma_3 =  - \iint_{D_3} \dpa{Q}{x} - \dpa{P}{y} \id{x,y} = 0\]


\obs
¿Y porqué en el primer ejemplo no salía 0? Lo que pasa es que $\vf$ no es $C^1$ en el origen y no podemos aplicar \textbf{Green}.

Si $(0,0)\in D \implies $ no podemos aplicar Green con este campo.

¿Cómo puede ser que la integral del campo en la circunferencia centrada en el origen de $2\pi$ y centrada en un punto para no contener al origen de $0$?  \textbf{Interesante...} el origen es una singularidad muy especial. ¿Y si la curva pasa por el origen? ¿Y si es un ángulo y no una circunferencia? Si quieres saber la respuesta a estas preguntas sigue estudiando matemáticas toda tu vida.

\end{example}

Falta la clase del 18/Diciembre

\begin{example}
\[\vf = (xy,y^2+e^{xz^2},sen(xy))\]
Tenemos una región acotada por 
\begin{gather*}
x=0\\
z=0\\
y=0\\
y+z=2\\
z=1-x^2
\end{gather*}

Queremos calcular:

\[\iint_{S\equiv\partial \Omega} \vf \id{S}\]

\todo{Dibujo}

\paragraph{Solución}

Tenemos un par de posibilidades. Una de ellas es dividir en 5 superficies y hacer las 5 integrales de superficie o aplicar le teorema de \textbf{Gauss} (que es la que vamos a hacer porque es más fácil)

Para aplicar \textbf{Gauss} necesitamos la divergencia:

\[\dv\vf = ... = 3y\]
\[
\iint_{S^+} \vf \id{S} = \iiint_{\Omega} \dv\vf \id{x,y,z}
\]

Vamos a razonar el orden de los límites de integración.

Si cogemos $\id{z,y,x}$, las $x,y$ quedan fáciles (el rectángulo [0,2]x[0,1]) y dividir la integral de $z$ en 2 trozos.

\[
\iiint_{\Omega} \dv\vf \id{x,y,z} = \iint_{1} \int_0^{1-x^2} 3y\id{z,y,x} + \iint_{2} \int_0^{1-x^2} 3y\id{z,y,x}
\]

Para calcular la curva que divide los trozos 1 y 2:

\[\left\{\begin{array}{cc}
y+z&=2\\
z&=1-x^2
\end{array} \right\} y+1-x^2=2\]

Con esto ya podemos completar la integral:
\[
\iiint_{\Omega} \dv\vf \id{x,y,z} = \iint_{1} \int_0^{1-x^2} 3y\id{z,y,x} + \iint_{2} \int_0^{1-x^2} 3y\id{z,y,x} = \int_0^1\int_0^{1+x^2}\int_0^{1-x^2}3y\id{z,y,x}+\int_0^1\int_{1+x^2}^{2}\int_0^{2-y}3y\id{z,y,x}
\]

¿Existe alguna manera de plantear la integral sin tener que dividirla en 2 trozos? \textbf{Sí} $\id{y,z,x}$. Integrando primero la $y$.

\[
\iiint_{\Omega} \dv\vf\id{y,z,x} = \int_0^1\int_0^{1-x^2}\int_0^{2-z}3y\id{y,z,x}
\]

\end{example}

