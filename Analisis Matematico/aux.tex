\documentclass{apuntes}
\usepackage{tikz}
\title{Análisis Matemático}
\author{Víctor de Juan Sanz}
\date{13/14 C1}
\usetikzlibrary{arrows,calc,shapes}

\begin{document}
%\section{Teoría de la integración}

\subsection{Repaso}

\subsubsection{Dimensión 1}

En dimensión 1, la teoría de integración la basábamos en la integral de Riemann. Tomábamos una partición de $[a,b]$ $\mathcal{P} = a = t_0 < t_1<...<t_k = b$ y a cada subintervalo de esa partición $[t_i,t_{i+1}] \rightarrow \begin{array}{cc} 
M_i &= sup\{f(x)\tq x\in []\\
m_i &= inf \{f(x)\tq x\in []\end{array}$ 

Definíamos \[ \alpha = \mathbb{U}(f,\mathcal{P}) = \sum_{i=0}^{k-1} M_i(t_{i+1}-t_i)\]
\[ \beta = \mathbb{L}(f,\mathcal{P}) = \sum_{i=0}^{k-1} \sum_{i=0}^{k-1} m_i(t_{i+1}-t_i)\]

\begin{defn}[Función\IS integrable] Se dice que $f$ es integrable en $[a,b]$ si y sólo si 

\[ \mathbb{U}(f,\mathcal{P}) = \beta \] 

La notación es \[ \int_a^b f(x)dx = \alpha = \beta \]
\end{defn}

\begin{theorem}
Si $f$ continua en $[a,b]$ entonces  $f$ es integrable en $[a,b]$.
\end{theorem}

El recíproco es \textbf{falso}. Se ve fácilmente si $f$ es una función escalonada, por ejemplo.

\begin{theorem} Sea $f$ integrable en $[a,b]$.

Tomamos $\mathcal{P}_h$ una partición de $[a,b] \tlq \max_{i} \abs{t_{i+1}-t_i} = h$ , es decir, $h$ mide el trozo más grande de la partición.

\[ \lim_{h\rightarrow 0} \sum_{i=1}^{k-1} f(s_i)(t_{i+1}-t_i) = \int_a^bf(x)dx \] para \textbf{cualquier} elección $s_i\in[t_i,t_{i+1}]$	

\end{theorem} 

\begin{theorem}[Cambio\IS de variable (dimensión 1)]
Sea $g$ un difeomorfismo, y suponemos que es creciente (la cuenta es análoga si fuese decreceiente); de tal forma que transforma el intervalo $[a,b]$ en el intervalo ($[g(a),g(b)] = [A,B]$).

\todo{Dibujar ese intervalo}

Entonces

\[ \int_A^B f(x)\,dx = \int_{\inv{g}(A)}^{\inv{g}{B}} f(g(t)) g'(t)\,dt \]

Es decir, tenemos cambio de los límites, de las variables y también de la medida.
\end{theorem}

\subsubsection{Dimensión N > 1}

En dimensión mayor que uno teníamos una serie de generalizaciones, y empezábamos con el teorema de Fubini que nos permite el cambio en el orden de integración:

\begin{theorem}[Teorema\IS de Fubini] Sean $A$ y $B$ dos conjuntos con ciertas propiedades \todo{Propiedades} que no sé cuáles son. Entonces

\[ \int_A \int_B f(x,y)\,dx\,dy = \int_B \int_A f(x,y)\,dy\,dx \]

En la práctica, tendremos que cambiar los límites de integración para que la integral exprese el mismo área.
\end{theorem}

Por ejemplo, si tenemos la integral \todo{Dibujito} \[ I_1 = \int_0^2\int_{y^2}^{2y} f(x,y)\,dx\,dx \] y queremos invertirla cambiamos los límites de integración: \[ I_1 = \int_0^4 \int_{\frac{x}{2}}^{\sqrt{x}} f(x,y)\, dy\,dx \]

Otro ejemplo: \todo{¿A que no sabes qué voy a pedir aquí?} estudiemos \[ I_2 = \int_a^bf(x)\,dx = \int_a^b\int_0^{f(x)}dx\,dy \]. Si probamos a cambiar el orden de integración, tenemos que

\[ I_2 = \int_0^M\int_{A(y)}dx\,dy \]

Donde $M$ es el máximo de $f$ y $A(y) = \{ x\in [a,b] \tq f(x) ≥ y \}$. La longitud de ese conjunto $A$ se denomina la medida, y entonces podemos expresar

\[ I_2 = \int_0^\infty \abs{\{x \tq f(x) > y\}}\,dy \] ya que la medida del conjunto será $0$ cuando $y > M$.

Curiosamente, hemos pasado de una integral de Riemann a otra con una expresión distinta, la llamada \textbf{integral de Lebesgue}\index{Integral\IS de Lebesgue}. Esto pertence al campo de la \textbf{teoría de la medida}, y permite estudiar conjuntos extraños y más monstruos y engendros varios.

\begin{theorem}[Cambio\IS de variable (dimensiones superiores)]
Dados unos conjuntos $D,\,D^\ast$ con $\appl{\Phi}{D}{D^\ast}$ un difeomorfismo, y $\gx \in D;\; \gy \in D^\ast$. Entonces

\[ \int_{D^\ast} f(\gy)\,d\gy = \int_D f(\Phi(\gx)) \abs{\det \dpa{\gy}{\gx}}\,d\gx \]
\end{theorem}

\subsubsection{Integración en curvas}

\begin{defn}[Curva\IS $C^1$]  Se denomina curva en $\real^n$ a una aplicación \begin{align*}
\appl{\gamma}{[a,b]\subset \real&}{\real^n} \\
t&\to \gamma(t) = (x_1(t),\dotsc,x_n(t))
\end{align*}

y con $\gamma\in C^1$.

También exigiremos que la curva sea un \textbf{camino regular}\index{Camino! regular}, es decir que

\[ \gamma'(t) \neq \gor{0} \;\forall t \]

y entonces obligamos a que tenga tangente en todo punto.
\end{defn}

Para calcular la longitud aplicamos las ideas básicas del cálculo integral: \textit{troceamos} el intervalo $[a,b]$, y aproximamos cada uno de esos trozos por un segmento. Calculamos la suma de la longitud de esos segmentos, hacemos tender la anchura de los \textit{trozos} y si converge, la longitud se puede medir.

\begin{theorem}[Longitud\IS de una curva] Dada una partición $\mathcal{P} = \{ a= t_0 < t_1 < \dotsb < t_k = b\}$ definimos \[ \abs{\mathcal{P}} = \max_i \abs{t_{i+1} - t_i} \] y la longitud $L$ de la curva.

\[ L(\sigma) = \lim_{\abs{\mathcal{P}}\to 0} \underbrace{\sum_{i=0}^{k-1} \md{\sigma(t_{i+1}) - \sigma(t_i)}}_{=S(\sigma,\mathcal{P})} \]

SI $\sigma\in C^1$, entonces $L(\sigma)$ existe y además

\[ L(\sigma) = \int_a^b \md{\sigma'(t)}\,dt \]
\end{theorem}

Este teorema responde a una idea con respecto al cambio de variable: si tomamos $\Gamma$ como la curva que queremos integrar, entonces se puede expresar

\[ L(\sigma) = \int_\Gamma 1\,d\sigma = \int_a^b \md{\sigma'(t)}\,dt  \]

donde $\md{\sigma'(t)}$ es el cambio en la medida correspondiente.

\begin{proof} Por pura pereza y no escribir más \footnote{A mí también me parece bien} suponemos \[ \sigma(t) = (x(t),y(t)) \]. Tenemos que

\begin{gather*}
 S(\sigma,\mathcal{P}) = \sum_{i=0}^{k-1} \md{\sigma(t_{i+1}) - \sigma(t_i)} = \\
 = \sum_{i=0}^{k-1}\sqrt{\left(x(t_{i+1}-x(t_i)\right)^2 + \left(y(t_{i+1}-y(t_i)\right)^2}
 \end{gather*}
 
 Por el Teorema del valor medio (\ref{thmTVM1var}) tenemos que
 
 \begin{gather*}
 \left(x(t_{i+1}-x(t_i)\right)^2  = x'(s_i^1)^2(t_{i+1}-t_i)^2 \\
 \left(y(t_{i+1}-y(t_i)\right)^2  = y'(s_i^2)^2(t_{i+1}-t_i)^2
 \end{gather*}
 
 Entonces
 
 \[  S(\sigma,\mathcal{P})  = \sum_{i=0}^{k-1} \sqrt{x'(s_i^1)^2 + y'(s_i^2)^2} (t_{i+1}-t_i) \]
 
 No podemos simplificar porque no es seguro que $s_i^1 = s_i^2$. Si fueran el mismo punto, serían sumas de Riemann y habríamos terminado.
 
 Como tenemos una función continua ($\sigma'$) y estamos trabajando en un intervalo cerrado y acotado ($[a,b]$) podremos reducir la expresión a $\sqrt{x'(t_i)^2 + y'(t_i)^2}$ y que gracias a la continuidad uniforme la diferencia con $\sqrt{x'(s_i^1)^2 + y'(s_i^2)^2}$ se puede hacer todo lo pequeña que queramos.
  
\end{proof}

\subsubsection{Parametrización por longitud de arco}

\[ \appl{\sigma}{[a,b]}{\real^n} \]

con $\sigma$ curva $C^1$ y regular. Entonces

\[ L(s) = \int_a^s \md{\sigma'(t)}\,dt \]

y por el Teorema Fundamental del Cálculo

\[ L'(s) = \md{\sigma'(s)} \]

Al imponer que la curva sea regular, entonces $\sigma'(s)\neq 0$ y por lo tanto existe la inversa $\inv{L}$. Si tenemos entonces un $\tau ∈ [0,L(b)]$, entonces $S=\inv{L}(\tau) ∈ [a,b]$.\wtf

Definimos

\begin{gather*}
\sigma^\ast = \sigma \circ \inv{L} \\
\sigma^\ast = \sigma(\inv{L}(\tau)) \\
(\sigma^\ast)'(\tau) = \sigma'(\inv{L}(tau)) (\inv{L}(\tau))' = \sigma'(\inv{L}(\tau)) \frac{1}{L'(\inv{L}(\tau))} = \sigma'(s) \frac{1}{L'(s)} = \frac{\sigma'(s)}{\md{\sigma'(s)}}
\end{gather*}

Es decir, hemos conseguido una parametrización con velocidad constante $\md{(\sigma^\ast)'} = 1$ y por lo tanto

\[ L(\tau) = \int_0^\tau \md{(\sigma^\ast)'}\,ds = \int_0^s 1\,ds = s \]

\subsection{Elemento de área}

Supongamos que estamos en $\real^3$. Podemos hablar de la longitud de una variedad de dimensión 1, del área de una de dimensión 2 o del volumen de una de dimensión 3. Ahora bien, ¿qué ocurre cuando pasamos a dimensiones superiores? La denominación será la siguiente

\begin{itemize}
\item \textbf{1-variedad} Longitud
\item \textbf{k-variedad} $1<k<N$ Área
\item \textbf{N-variedad} Volumen
\end{itemize}

Para calcular esas \textit{cosas} empezaremos partiendo del área de un parelelepípedo.

Definiremos el paralelepípedo como, dados $k$ vectores independientes $\{\gv_i\} \subset \real^N$, 

\[ P_k = \sum_{i=1}^k \lambda_i\gv_i\;\lambda_i \in [0,1] \]
%\section{Ejercicios}
\subsection{Hoja 1}

\begin{problem}[?]
\solution
$\overline{A} = { x \in \real^N; \forall V_x \tq V_x \cup A \neq \text{\O}}$, siendo $V_x$ un entorno abierto de x.
	$\overline{A} = A \cap $ 
	
\begin{theorem}
$A \subset \real^N$ es cerrado $\dimplies \acum{A}\subset A$ 
\end{theorem}

 
\end{problem}
\begin{proof}
\begin{gather*}
A \text{es cerrado} \implies A^c \text{ es abierto} \implies\\
\forall x \in A^c, \exists \varepsilon > 0 \tq B(x,\varepsilon) \subset A^c \implies\\
A \cap B(x,\varepsilon) = \text{\O} \implies x \nexists \acum(A) 
\end{gather*}
\end{proof}
Falta la recíproca.


\begin{problem}[3] 
\solution
a)

$\displaystyle\bigcup_{k=1}^{\infty} \left[-1,\frac{1}{k}\right)$

Es cerrado, porque $=[-1,0]$
Demostración: (hay que demostrar las inclusiones $\subseteq$ y $\supseteq$)

b)
No es ni cerrado ni abierto.
\obs $\real$ es el cierre de $\mathbb{Q}$.

c)
 
\end{problem}

\newpage
\subsection{Hoja 2}

\begin{problem}[?]
\solution
$f(x,y) = \left\{\begin{matrix}
                \displaystyle \frac{x^3}{x^2+7y^2} & \text{ si } (x,y) \neq (0,0)\\
                 0 & \text{ si } (x,y) = (0,0)
                \end{matrix}\right.$
Continuidad:

$$0\leq \left| \frac{x^3}{x^2+7y^2} \right| \leq \left| \frac{x^3}{x^2} \right| \leq |x| \rightarrow 0 \implies$$ Continua en 0.

Derivadas parciales en $(0,0)$

$$\mylim{h}{0} {\frac{f(h,0) - f(0,0)}{h}} = \mylim{h}{0} {\displaystyle\frac{{h^{3}}/{h^{2}}}{h}} = 1$$
$$\mylim{h}{0} {\frac{f(0,h) - f(0,0)}{h}} = \mylim{h}{0} {\frac{0}{0+7h^2}} = 0$$
\end{problem}
 
\subsection{5}

\paragraph{b)}
$C_{D_4}(b)$. Basta con comprobar la conmutación con $a^j$ y con $a^jb$ siendo $j = 0,1,2,3$, ya que con eso podemos ver la conmutación con todos los elementos. Se puede demostrar la conmutatividad multiplicando a derecha e iezquierda por $b$ y $b^{-1}$ y si nos queda $=1$, es conmutativo.

$$\left\{\begin{matrix}b(a^j)b^{-1} = a^{-j}, a^j \in C_{D_4}(b) \dimplies a^2j = 1\\
b(a^jb)b^{-1} = a^{-j} = a^{-j}b, a^jb \in C_{D_4}(b) \dimplies a^2j = 1\end{matrix}\right.$$

\subsection{9}
\paragraph{a)}
\paragraph{b)}

$$f(x,y) = \int_a^xy g(s)ds$$
Aplicando el teorema fundamental del cálculo $\left(f \text{ continua } \implies\displaystyle\int_a^b f(x)dx = F(b)-F(a)\right)$
$$\dpa{f}{x} = g(xy)\underbrace{\dpa{xy}{x}}_{=y} - \underbrace{g(a)\dpa{a}{x}}_{=0}  = g(xy)y$$
$$\dpa{f}{y} = \cdots  = g(xy)x$$

\subsection{Ejercicio de examen:}
$\appl{g}{\real}{\real}$ continua, con $g(1) = 4$.

Sea $f(x,y,z)=\displaystyle \int_0^{x^2ye^z} g(t)dt$.

Demostrar que $f$ es diferenciable y calcular $\nabla f(1,1,0)$.
\newpage
\section{Hoja 3}
\begin{problem}[3]
\paragraph{a)} Probar que si la derivada de $\appl{f}{\real}{\real}$ existe y no se anula entonces $f$ es inyectiva.
\paragraph{b)} Probar que $\appl{f}{\real^2}{\real^2}$ dada por $f(x,y) =( e^xcos(y) + 2e^xsen(y),-e^xcos(y))$ cumple que el determinante de su Jacobiano es siempre positivo pero sin embargo no tiene $f$ no es inyectiva.

\solution

a)

% \begin{itemize}
%  \item $f$ derivable $\implies f$ continua.
%  \item Si $f'(x) \neq 0, \forall x \in \real \implies f $ monótona (estrictamente creciente o decreciente). 
%  \end{itemize}
Me parece demasiado intuitivo...

b)
$$J = \begin{pmatrix}
       e^xcos(y)+2e^xsen(y) & e^xcos(y)+2e^xsen(y) \\
       -e^xcos(y) & e^xsen(y)
      \end{pmatrix}
$$

Calculamos $$det(J) = (e^xcos(y)+2e^xsen(y))+e^xsen(y) + e^xcos(y)sen(y)(-e^xsen(y)+2e^xcos(y)) = $$
$$ = ... = 2e^x > 0 \forall x \in \real$$

Aunque el jacobiano sea siempre positivo, $f$ no es inyectiva porque si tomamos $f(0,0) = (1,-1) = f(0,2\pi)$.
\end{problem}
\begin{problem}[inventado]
\label{inventado}
Sea $F(x,y) = (x^2-y^2,2xy)$. Encontrar los puntos en los que la siguiente aplicación es localmente inversible de clase $C^1$.
\solution
\begin{itemize}
 \item 1) $F \in C^1$ por ser $F_1,F_2$ polinomios.
 \item 2)$det(J)>0 \forall (x,y)\in \real^2$. 
 
 En este caso: $$det\begin{pmatrix}
                  2x&-2y\\
                  2x&2y
                 \end{pmatrix} = 4x^2 + 4y^2 = 0 \dimplies (x,y) = (0,0)$$           
 \item 3) Por el teorema de la funcion inversa, existe una inversa local de $F,C^1$ en todo entorno de $(x,y) \in \real^2$ con $(x,y)\neq (0,0)$. 
 
 Está la posibilidad de que exista la función inversa, pero no podemos deducir nada del teorema. Para verlo, recurrimos a la definición de inyectividad, y en este caso, no es inyectiva porque es una función par.
 \end{itemize}
 \end{problem}
 \begin{problem}[5]



 a)
 $f\in C^1(\real), f'\neq0$.
 No tiene sentido...
 $$\left\{\begin{matrix} u(x,y) =f(x)\\v(x,y) = -y + f(x)\end{matrix}\right.$$
 Probar que tiene inversa global.
 
 b) Si $f (0) = 0$ y $f' (0) = 1$, hallar las derivadas parciales de dicha inversa en el origen.

\solution

 Mismos pasos que en el ejercicio anterior:
 \begin{itemize}
 \item 1) $F \in C^1$ por ser $F_1,F_2$ , porque $f\in C^1$.
 \item 2)$det(J)>0 \forall (x,y)\in \real^2$.
 
 $$det(J) = det\begin{pmatrix} f'(x)&0\\f(x)+xf'(x)&-1\end{pmatrix} = -f'(x) \neq 0\text{ por hipótesis}$$
 \end{itemize}
 \paragraph{b)}
 Como nos piden calcular las derivadas parciales de la función inversa. (La inversa de la matriz jacobiana, es la jacobiana de la matriz inversa)
 $$J(0,0) = \begin{pmatrix}f'(0) & 0 \\f(0) & -1\end{pmatrix}$$
 Lo que buscamos en la matriz inversa, que en este caso es ella misma.
 
 El teorema solo nos demuestra la existencia de la inversa local (contraejemplo:(\ref{inventado})). Hay que ver la inyectividad para hablar de inversa global.
 
\begin{gather*}
F(x,y) = (u(x,y),v(x,y))\\
\text{Condición: }F(x_1,y_1) = F(x_2,y_2) \implies x_1=x_2, y_1=y_2\\
u(x_1,y_1) = u(x_2,y_2) \implies f(x_1) = f(x_2)\\
f' \text{ no se anula } \implies \text{f es inyectiva} \implies x_1=x_2\\
v(x_1,y_1) = v(x_2,y_2) -y_1 + x_1f(x_1) = -y_2 + x_2f(x_2) \underbrace{\implies}_{x_1=x_2}\\
y_1=y_2
\end{gather*}
Hemos demostrado que $F$ es inyectiva y por lo tanto admite inversa global.
 
\end{problem}
\begin{problem}[6]
Estudiar si se puede despejar $(x,y,z)$ en términos de $(u,v,w)$ 
$$F(x,y,z) = \left\{\begin{matrix}u = 2x+2x^2y+2x^2z+2xy^2+2xyz\\v=x+y+2xy+2x^2\\w=4x+y+z+3y^2+3z^2+6yz\end{matrix}\right.$$
\solution
\begin{itemize}
 \item $u,v,w \in C^1$ por se suma de polinomios. 
 \item \begin{gather*}
\dpa{u}{x} = ... \implies \dpa{u}{x}(0,0) = 2\\
\dpa{u}{y} = ... \implies \dpa{u}{y}(0,0) = 0\\
\dpa{u}{z} = ... \implies \dpa{u}{z}(0,0) = 0\\
\dpa{v}{x} = ... \implies \dpa{v}{x}(0,0) = 1\\
\dpa{v}{y} = ... \implies \dpa{v}{y}(0,0) = 0\\
\dpa{v}{z} = ... \implies \dpa{v}{z}(0,0) = 0\\
\dpa{w}{x} = ... \implies \dpa{w}{x}(0,0) = 4\\
\dpa{w}{y} = ... \implies \dpa{x}{y}(0,0) = 1\\
\dpa{w}{z} = ... \implies \dpa{w}{z}(0,0) = 1
       \end{gather*}
       
   $det(J) =\begin{pmatrix}
             2&0&0\\
             1&1&0\\
             4&1&1
            \end{pmatrix}
 = 2 \neq 0 \implies \exists $ inversa local de clase $C^1$ en un entonrno de cualquier punto, en concreto en un entorno del origen.
\end{itemize}
\end{problem}

\begin{problem}[8]
\solution
\paragraph{a)}

$$det(J) = det\begin{pmatrix}
       cos(\varphi)&-rsen(\varphi)&0\\
       sen(\varphi)&rcos(\varphi)&0\\
       0&0&1
      \end{pmatrix} = rcos^2(\varphi) + rsen^2(\varphi) = r$$
      
      Por tanto, por el teorema de la función inversa, existe una inversa de clase $C^1, \forall (r,h,\varphi) \dimplies r\neq 0$.
\end{problem}
 \begin{problem}[9]
 \solution
 \paragraph{b: Calcular la inversa en (2,-2$\sqrt{3}$)}
 
 Resolver: $$\left\{\begin{matrix} 2 = rcos(\varphi)\\-2\sqrt{3} = rsen(\varphi)\end{matrix}\right.$$
 
 Hay que hallar la inversa de: $$\begin{pmatrix}
                                  \frac{1}{2}&2\sqrt{3}\\
                                  \frac{-\sqrt{3}}{2}&2
                                 \end{pmatrix}$$

    \end{problem}                             
  \begin{problem}[13]
  \solution
  
  $$\det(J) = \det \begin{pmatrix}
              \dpa{f_1}{x}&\dpa{f_1}{y}\\
              \dpa{f_2}{x}&\dpa{f_2}{y}
             \end{pmatrix} = 
             \det \begin{pmatrix}
              \dpa{f_1}{x}&-\dpa{f_2}{x}\\
              \dpa{f_2}{x}& \dpa{f_1}{x}
             \end{pmatrix}
      = \left(\dpa{f_1}{x}\right)^2 + \left(\dpa{f_2}{x}\right)^2 \implies \left(\dpa{f_1}{x},\dpa{f_2}{x}\right)$$
Esto es aplicando la primera ecuación de Cauchy-Riemman. Obteniendo una condición

Aplicando la otra condición en el jacobiano llegamos a $\displaystyle\left(\dpa{f_1}{y},\dpa{f_2}{y}\right)\neq (0,0)$
\paragraph{c)}

Queremos ver que $g(x,y) = (f_1(x,y)^2-f_2(x,y)^2,2f_1(x,y)f_2(x,y))$ cumple las ecuaciones de Cauchy-Riemman. Facilito.
\end{problem}


Hoy

\begin{problem}[15.b]
\[f(x,y,z) = 2x^2+y^2+z^2-xy\]
\[\mathbb{K} = \{ (x,y,z)\in \real^3 : \frac{x^2}{2} + \frac{y^2}{4} + \frac{z^2}{8} \leq 1\}\]
\solution
En el interior sale el punto crítico $(0,0,0)$ y en la forntera tendremos el sistema:

\[\left\{\begin{array}{cc}
4x-y&=\lambda x\\
2y-x &=\lambda \frac{y}{2}\\
2z &= \lambda\frac{z}{4}\\
\frac{x^2}{2}+\frac{y^2}{4} + \frac{z^2}{8} &=1
\end{array}\right\}\]
\end{problem}


\begin{problem}[18]
Fórmula distancia de un punto al plano $Ax+By+Cz+D=0$.

\solution
\[g(x,y,z) = Ax+By+Cz+D \implies \nabla g = (A,B,C)\]

Definimos 
\[f(x,y,z) = d^2 = (x-a)^2 + (y-b)^2 + (z-c)^2 \implies \nabla f = (2(x-a),2(y-b),2(z-c))\]

Aplicamos el teorema de Lagrange:

\[\left\{\begin{array}{cc}
\grad f(x,y,z) &= \lambda \grad g\\
g(x,y,z)&=0
\end{array}\right\}\]

Llegando a la solución según las 3 primeras ecuaciones:
\begin{gather*}
x=\frac{\lambda A}{2} +a\\
y = \frac{\lambda B}{2} +b\\
z = \frac{\lambda C}{2} +c
\end{gather*}
Reemplazamos $x,y,z$ en la ecuación del plano (cuarta ecuación) obteniendo:

\[\lambda = \frac{-2(D+Cc+Bb+Aa)}{A^2+B^2+C^2}\]

Sustituimos este $\lambda$:

\begin{gather*}
x=a-\frac{(D+Cc+Bb+Aa)}{A^2+B^2+C^2} A \\
y = b- \frac{(D+Cc+Bb+Aa)}{A^2+B^2+C^2} B\\
z = c - \frac{(D+Cc+Bb+Aa)}{A^2+B^2+C^2} C
\end{gather*}

Y ahora reemplazamos en $d^2$:

\begin{gather*}
d^2 = \left(\frac{(D+Cc+Bb+Aa)}{A^2+B^2+C^2} A\right)^2 + \left(\frac{(D+Cc+Bb+Aa)}{A^2+B^2+C^2} B\right)^2 + \left(\frac{(D+Cc+Bb+Aa)}{A^2+B^2+C^2} C\right)^2\\
d^2 = \left(\frac{(D+Cc+Bb+Aa)}{A^2+B^2+C^2}\right)^2 \cdot (A^2+B^2+C^2) \\
d^2 = \frac{(D+Cc+Bb+Aa)^2}{A^2+B^2+C^2} \implies d=\frac{\abs{D+Cc+Bb+Aa}}{sqrt{A^2+B^2+C^2}}
\end{gather*}
Que es justamente la fórmula de la distancia de un plano a un punto.

\end{problem}


\begin{problem}[19]
patata
\solution

\[\left\{
\begin{array}{cc}
4x^3&=\lambda\\
4y^3&=\lambda\\
4z^3&=\lambda\\
x+y+z=5
\end{array}\right\}\]
\end{problem}
\end{document}