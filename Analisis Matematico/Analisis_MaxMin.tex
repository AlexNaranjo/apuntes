\section{Máximos y mínimos condicionados}

\begin{defn}[Extremo \IS condicional]
$\Omega\subset\real^n$, abierto, $\appl{F}{\Omega}{\real}$ y M subvariedad de $\real^n, M\subset \Omega$

\textbf{Definición:}

\begin{itemize}
\item $F$ alcanza en $\ga\in M$ un extremo local condicionado a $M \dimplies$ existe un abierto \[U\in\real^n\, con \, \ga\in U \tlq \left\{\begin{matrix} F(\gx) \leq F(\ga), \forall \gx \in U\cap M \\
F(\gx) \geq F(\ga), \forall \gx \in u\cap M \end{matrix}\right.\]
\end{itemize}

\end{defn}


\begin{theorem}[Teorema\IS de los multiplicadores de Lagrange] \label{thmMultLagrange}
$\appl{F}{\Omega\subset\rnk}{\real}, F\in C^1(\Omega)$.

M, variedad n-dimensional en $\rnk$. $\ga\in M$ extremo local de $F$ condicionado a M.

Supongamos que en un entorno de $\ga$:

$M = \{G = \gor{0}\} y \, rg(DG)=max$

\textbf{Entonces:} Existen $\lambda_1,...,\lambda_k \in \real$ tales que
\[\nabla F(\ga) = \sum \lambda_i \nabla G_i (\ga)\]
\end{theorem}

\begin{proof}
$\ga \in M$. Consideremos $T_{\ga}M = \ker [DG(\ga)]$

\[DG(\ga) = \begin{pmatrix}
\nabla G_1(\ga) &\longrightarrow \\
\nabla G_2(\ga) &\longrightarrow \\
\vdots\\
\nabla G_k(\ga) &\longrightarrow 
\end{pmatrix}\]

Sea $\gv \in T_{\ga M}$. Entonces $DG(\ga)\cdot\gv = \gor{0}\in\real^K$

Este producto es como \[\begin{pmatrix}
\pesc{\nabla G_1(\ga),\gv}\\
\pesc{\nabla G_2(\ga),\gv}\\
\vdots\\
\pesc{\nabla G_k(\ga),\gv}
\end{pmatrix}\]

Podemos concluir que $\nabla G_j(\ga)\perp T_{\ga} M \forall j=1,...,k$, es decir $\nabla G_j(\ga) \in (T_{\ga}M)^\perp,\forall j=1,...,k$

La condición de rango máximo nos asegura que 
\[\{\nabla G_i, i = 1,...,n\} \text{ es una base de } (T_{\ga} M)^\perp\]

Ahora solo falta probar que  $\gv \in T_{\ga} M \implies \nabla F(\ga)\perp\gv$, es decir $\nabla F(\ga) \in (T_{\ga} M)^\perp$.

\[\gv \in T_{\ga} M \implies \exists \appl{\alpha}{\real}{M} \alpha \in C^1, \alpha(0) = \ga, \alpha'(0) = \gv\]

Definimos $g(t) = F(\alpha(t))$

$F$ extremo en $\ga \implies g$ extremo en $t=0$.

\[ 0 = g'(0) = DF(\alpha(0))\cdot \alpha'(0) = \pesc{\nabla F(\alpha(0)),gv} = \pesc{\nabla F(\ga),gv}\]

Conclusión $ 0 = \pesc{\nabla F(\ga),gv} \implies \nabla F(\ga) \in (T_{\ga}M)^\perp$

\end{proof}


\paragraph{Ejemplos:}

\subparagraph{1)} $F(x,y) = x^2-y^2$. Sea $M = \{x^2+y^2 = 1\}$.

%Revisar. Falta pintar la circunferencia
%\easyimg{imgs/MCNSM.png}{Mapa de conjuntos de nivel de la silla de montar}{lblMCNSM}

%Si calculamos el máximo y el mínimo condicionados a la circunferencia de $R=1$ vemos que $\nabla F$ es paralelo al radio de la circunferencia, como consecuencia de la tangencia entre el conjunto de nivel y la subvariedad $M$. 

\begin{itemize}
\item $M$ es cerrado y acotado, es decir $M$ es compacto, $G^{-1}({0}) \equiv M, G(x,y) = x^2+y^2-1$
\item F continua
\end{itemize}

 $\implies$ se alcanza máximo y mínimo.


\textbf{Vamos con los multiplicadores.}

$\nabla F = (2x,-2y), \nabla G(2x,2y)$

El sistema queda:

\[\left\{\begin{array}{cc}
2x &= \lambda_1 2x  \\
2y &= -\lambda_2 2y\\
x^2+y^2 &= 1
\end{array}\right\}\]

La ecuación 1 nos da 2 casos:
\[\left\{\begin{array}{cc}
caso\, 1 & \lambda = 1\\
caso\, 2 & \lambda = 0
\end{array}\right\}\]

En el caso $\lambda = 1$

\[\left\{\begin{array}{cc}2x&=2x\\2y&=-2y\\x^2+y^2&=1\end{array}\right\} \implies x=\pm 1\]

Puntos críticos: $(1,0),(-1,0)$ con $F(\pm 1,0) = 1 \implies MAX$

En el caso $\lambda = 0$.

\[ \left\{\begin{array}{cc}
x &= 0\\
-2y &= \lambda 2y \\
y^2 &= 1
\end{array}\right\}\dimplies
\left\{ \begin{array}{cc}
y &= \pm 1\\
\lambda &= -1
\end{array}\right\}\]

Puntos críticos: $(0,1),(0,-1)$, con $F(0,\pm 1) = -1 \implies MIN$

\paragraph{Ejemplo 2)}

Hallar los puntos críticos más cercanos al origen entre los contenidos en la curva
\[\Gamma = \left\{ \begin{array}{cc} x^2+xy+y^2-z^2-1&=0\\x^2+y^2-1&=0\end{array}\right.\]

En este caso: $M = \Gamma$ y tomamos $F(x,y,z) = x^2+y^2+z^2$ y $G(x,y,z) = (x^2+xy+y^2-z^2-1,x^2+y^2-1)$

\subparagraph{1) Compacidad}

$\Gamma$ es un conjunto cerrado porque las 2 ecuaciones definen un cerrado, y la intersección de 2 cerrados es un cerrado. Además está acotado porque e2 acota $x$ e $y$, y la e1 acota la z teniendo acotadas $x$ e $y$.

\subparagraph{2) Continuidad}

$F$ es continua.


Vamos con los multiplicadores:

\[
\left\{
\begin{array}{cl}
2x &= \lambda_1(2x+2y)+ \lambda_2 2x\\
2y &= \lambda_1(x+2y) + \lambda_2 2y\\
2z &= \lambda_1(-2z)\\
x^2+xy+y^2-z^2& = 1\\
x^2+y^2 &= 1
\end{array}\right\}\]
La tercera ecuación distingue 2 casos:
\[
\rightarrow
\left\{\begin{array}{cl}
Caso\, 1 &z=0\\
Caso\, 2 &\lambda_1 = -1\\
\end{array}\right.\]

\textbf{Caso 1:}

\[\left\{
\begin{array}{cl}
2x &= \lambda_1(2x+2y) + \lambda_2 2x\\
2y &= \lambda_1(x+2y) + \lambda_2 2y\\
x^2+xy+y^2& = 1\\
x^2+y^2 &= 1
\end{array}\right\}\]

De las últimas 2 ecuaciones vemos que $xy=0$. Distinguimos casos:

\[\left\{\begin{array}{ccc}
Caso\, 1.1& x=0 &\rightarrow y = \pm 1\\
Caso\, 1.2& y=0 &\rightarrow x = \pm 1
\end{array}\right\}\]

Tenemos 4 puntos críticos $P_1 = \{(1,0,0),(0,1,0),(-1,0,0),(0,-1,0)\}$ con $F(x,y,z) = 1, \forall (x,y,z)\in P_1$-

\textbf{Caso 2:}


\[\left\{\begin{array}{cc}
2x&= -(2x+y) + \lambda_2 2x\\
2y&= -(x+2y) + \lambda_1 2y\\
x^2+xy+y^2 -z^2 &= 1\\
x^2+y^2&=1
\end{array}\right\}\]

Las 2 ultimas ecuaciones nos dan que $xy = z^2 \neq 0 \implies x,y\neq 0$.

Podría darse el caso de que nos saliera la misma solución que en el caso 1. No preocuparse, es posible.

\[\left\{\begin{array}{cc}
2xy &= \left[-(2x+y)+\lambda_2 2x\right]y\\
2xy &= \left[-(x+2y)+\lambda_2 2y\right]x
\end{array}\right\}\]

Echate las cuentas lector vago y llegarás a:

\[P_2 = \left\{
\left(\frac{1}{\sqrt{2}},\frac{1}{\sqrt{2}},\frac{1}{\sqrt{2}}\right),
\left(\frac{-1}{\sqrt{2}},\frac{-1}{\sqrt{2}},\frac{1}{\sqrt{2}}\right),
\left(\frac{1}{\sqrt{2}},\frac{1}{\sqrt{2}},\frac{-1}{\sqrt{2}}\right),
\left(\frac{-1}{\sqrt{2}},\frac{-1}{\sqrt{2}},\frac{-1}{\sqrt{2}}\right)
\right\}\] con $F(x,y,z) = \frac{3}{2}, \forall (x,y,z)\in P_2$

\subsection{Clasificación de puntos críticos condicionados}

Sea $M$ una subvariedad de manera que $M=\{G = \gor{0}\}$ definida por un sistema de $k$ ecuaciones $G=(G_1,\dotsc,G_k)$ con $\rango DG = k$. Sea $P\in M$ un punto crítico de una función $F$ restringida a $M$. El teorema de los multiplicadores de Lagrange (\ref{thmMultLagrange}) nos dice que 

\[ \grad F(P) - \lambda_1 DG_1(P) - \dotsb -\lambda_kDG_k(P) = \gor{0} \]

donde los $\lambda_j$ son los multiplicadores. Definimos una función auxiliar 

\begin{equation} H(\gx) = F(\gx) - \lambda_1 G_1(\gx) - \dotsb - \lambda_k G_k(\gx)\label{eqClasAux} \end{equation}

con dos propiedades interesantes:

\begin{gather*}
\gx\in M\implies H(\gx) = F(\gx) \\
\grad H(P) = \gor{0} 
\end{gather*}

Para clasificar el punto crítico vamos a coger una curva dentro de la subvariedad y a componerla con $H$. Pasaremos a tener una función de una variable y ahí veremos en qué condiciones tenemos un máximo o un mínimo.

Tomamos una curva $\alpha\in C^1$ tal que 
\begin{gather*}
\alpha(0) = P \\
\alpha(t) \in M\; \forall t \\
\alpha'(0) = \gv
\end{gather*}

Estudiamos $g(t) = H(\alpha(t)) = F(\alpha(t)$. Entonces

\[ g'(t) = \pesc{\grad H(\alpha(t)),\alpha'(t)} \]

y entonces \[ g'(0) =\pesc{\grad H(P),\gv} = 0\]

Como $F$ tiene un punto crítico condicionado en $P$, entonces $g$ tiene un punto crítico en $t = 0$. Determinar el carácter del punto crítico condicionado es lo mismo que determinar el carácter del punto crítico en $g$. Por lo tanto, tenemos que explorar la segunda derivada $g''(0)$:

\begin{gather*}
g'(t) = \pesc{\grad H(\alpha(t)),\alpha'(t)}  = \sum_{i=1}^n\dpa{H}{x_i}(\alpha(t)) \cdot \alpha_i'(t) \\
\frac{d^2g}{dt^2}(t) =  \sum_{i=1}^n\deriv{}{t} \left(\dpa{H}{x_i}(\alpha(t)) \cdot \alpha_i'(t)\right)
\end{gather*}

Calculamos esa segunda derivada dentro del sumatorio

\[ \deriv{}{t} \left(\dpa{H}{x_i}(\alpha(t)) \cdot \alpha_i'(t)\right) = \underbrace{\dpa{H}{x_i}(\alpha(t))\alpha_i''(t)}_{=0\text{ si } t=0} + \alpha_i'(t) \left[ \sum_{j=1}^n\frac{∂^2H}{∂x_ix_j}(\alpha(t))\alpha_j'(t)\right] \]

Conclusión:

\[ \frac{d^2g}{dt^2}(t) = \sum_{i=1}^n\left[\dpa{H}{x_i}(\alpha(t))\alpha_i''(t) + \alpha_i'(t) \left(\sum_{j=1}^n\frac{∂^2H}{∂x_ix_j}(\alpha(t))\alpha_j'(t)\right)\right] \]

Evaluando en $t=0$

\begin{gather*}
 \frac{d^2g}{dt^2}(0) = \sum_{i=1}^n\left[\underbrace{\dpa{H}{x_i}(P)}_{=0}\alpha_i''(0) + \left(\sum_{j=1}^n\frac{∂^2H}{∂x_ix_j}(P)v_iv_j\right)\right]  = \\
 = \sum_{i=1}^n\sum_{j=1}^n\frac{∂^2H}{∂x_ix_j}(P)v_iv_j = \gv \cdot D^2H(P) \cdot \trans{\gv}
 \end{gather*}
 
 Si estuviésemos viendo la clasificación de un punto crítico en general, miraríamos si la matriz hessiana es definida positiva, negativa o qué. Sin embargo, aquí estamos viendo un punto crítico restringido. Los vectores $\gv$ están siempre en el plano tangente a la variedad, así que sólo tenemos que ver si la matriz es definida positiva/negativa para esos vectores. Por lo tanto, la clasificación queda como
 
 \begin{theorem}[Clasificación\IS de puntos críticos restringidos] Dado un punto crítico restringido $P$ , su carácter depende del signo de $\gv \cdot D^2H(P) \cdot \trans{\gv}\; \forall\gv\in T_P M$ con la función $H$ definida en (\ref{eqClasAux}):
 
 \begin{itemize}
 \index{Máximo/mínimo!local (restringido)}
 \item Signo mayor que cero: \textbf{mínimo local}
 \item Signo menor que cero: \textbf{máximo local}
 \end{itemize}
  \end{theorem}
\subsubsection{Ejemplos} Calculamos los puntos críticos de la función $F$ restringidos a la variedad $G = 0$, con

\begin{gather*}
F(x,y) = (x+1)^2+y^2 \\
G(x,y) = y^2-x^3 = 0 
\end{gather*}

Con la restricción de $G$, tenemos que $x^3=y^2≥0$ y entonces $x≥0$. Entonces, esto significa que $F(x,y) ≥ 1$. Como $F(0,0)=1$, está claro que, sin operar, en $(0,0)$ hay un mínimo.

\[
\begin{matrix}
\grad F= (2(x+1), 2y) \\
\grad G = (-3x^2, 2y)
\end{matrix}
\left\lbrace
\begin{matrix}
2(x+1)=\lambda(-3x^2)\\
2y=\lambda 2y\\
y^2-x^3 = 0
\end{matrix}
\right.
\]

En la segunda ecuación, tenemos dos posibilidades para resolverlas:

\paragraph{Caso 1: $y=0$}

Tendríamos que \[\left.\begin{matrix}2(x+1)=-3\lambda x^2 \\ -x^3 = 0\end{matrix}\right\rbrace \], lo que es imposible.

\paragraph{Caso 2: $\lambda = 1$}

La ecuación queda como \[\left.\begin{matrix}2(x+1)=-3x^2 \\ y^2=x^3 \end{matrix}\right\rbrace \] pero  la ecuación $3x^2+2x+2=0$, imposible también.

Esto no quiere decir que el teorema no funcione, sino que tenemos que tener cuidado con las hipótesis: $\grad G$ tiene rango 0 en $(0,0)$  y por lo tanto \textbf{no podemos aplicar el teorema}.
\subparagraph{Ejemplo 2}
Dada una función \[ F(x,y,z))=xyz \] y nos piden calcular el mínimo en un conjunto \[ K=\{(x,y,z)\in\real^3\tq x+y+z=1,x>0,y>0,z>0\} \]. La restricción no es una variedad, sino un trozo de una variedad, un triángulo en el primer octante. Usamos Lagrange y estudiaremos sólo los puntos críticos que caen dentro del plano. También tendremos que ver qué ocurre en los bordes. Nos hemos encontrado un "monstruo" nuevo, una variedad con un borde.Habrá que llevar a un consenso de la definición porque en el plano todos los puntos son puntos frontera, y hay unos \textit{más fuera que otros.} 

Habrá que resolver el sistema \[ \begin{matrix}\grad F=\lambda\grad G \\G=0\end{matrix} \]. Después seleccionamos puntos con todas las coordenadas positivas y entonces compararemos con el borde ($F=0$).

 
 \subparagraph{Ejemplo 3}
 Hallar los extremos absolutos de $F(x,y) = 2x+y^2$ en 
 \[\mathbb{K} = \left\{(x,y)\in\real^2 \tq \begin{array}{cc} x^2+y^2 &\leq 2\\ x\leq y^2\end{array}\right\}\]
 
¿Tiene sentido plantearse que esta función alcanze algún máximo o mínimo? \todo{Demostrar que este conjunto es compacto}
 
 \todo{Dibujito}
 
 Hay que estudiar:
 \begin{itemize}
 \item interior $\grad F = (2,2y) \neq (0,0) \implies$ imposible.
 \item Frontera, distinguiendo:
 \begin{itemize}
 \item Circunferencia
 	$G_1(x,y) = x^2+y^2-2$.
 	
 	\[\left.\begin{array}{cc}
 	\grad F = \lambda \grad G_1\\
 	G_1 = 0\end{array} \right\}\implies \left\{\begin{array}{c}
 	2 = 2\lambda x\\
 	2y = \lambda 2y\\
 	x^2+y^2=2\end{array} \right\} \implies \{(\sqrt{2},0),(-\sqrt{2},0),(1,1),(1,-1)\} \]
 	De estos 4 puntos, sólo nos interesa en $(-\sqrt{2},0)$ que es el único que pertenece al interior
 \item Parábola
 
 $G_2(x,y) = x-y^2$
 
 \[\left.\begin{array}{cc}
 2 &= \lambda\\
 2y &= \lambda (-2y)\\
 x-y^2 &=0\end{array}\right\} x=y=0\]
 \item Puntos aislados
 \end{itemize}
 \end{itemize}

\textbf{Conclusión} Evaluamos $F$ en los 4 puntos:

\begin{itemize}
\item $F(-\sqrt{2},0)$
\item $F(1,1)$
\item $F(-1,1)$
\item $F(0,0)$
\end{itemize}

\subparagraph{Ejemplo 4}

\begin{problem}[?]

$a>0,b>0, ab(a+b) = 1$.

Consideramos sólidos que tienen como base el triángulo de vértices $(0,0),(a,0),(b,0)$ y tienen como sección al cortar por un plano $x=cte \leadsto$ lo que sale son triángulos isósceles de altura 4.

Objetivo: Maximizar el volumen esos sólidos para cualquier $(a,b)$ que cumplan la propiedad.
\solution

\todo{Fotito}

%es un triángulo isósceles sobre el plano y=0 unido a una recta que une (a,0,0) y (4,0,0)

Para calcular el área del mostruito vamos a aplicar el \textit{Principio de cavalier} \todo{(link a wiki)}

\[Vol(a,b) = \int_0^a A(x)dx\]

La base viene dada por la recta que pasa por $(a,0,0) y (b,0,0) \leadsto y=b-\frac{b}{a}x$

\[A(x) = \displaystyle \frac{\left(b-\frac{b}{a}x\right)4}{2} = 2b - \frac{2b}{a}x\]

Sustituyendo:


\[Vol(a,b) = \int_0^a 2b-\frac{2b}{a}xdx = ... = ab\]

El problema queda traducido a: Máximo de $F(a,b) = ab$ con la restricción $ab(a+b)=1$.

Para evitar bloqueos mentales y quitarnos problemas con notación traducimos a:

\[Vol(x,y) = \int_0^x 2y-\frac{2y}{x}xdx = ... = xy\]

El problema queda traducido a: Máximo de $F(x,y) = xy$ con la restricción $xy(x+y)=1$.
\todo{birujito}

Problema!! La restricción no es compacta, porque no está acotada. Entonces truncamos la gráfica
\todo{grafica truncada}

Ya estamos en condiciones de aplicar el teorema

\[\left.\begin{array}{cc} y&= \lambda (2xy + y^2)\\
x&= \lambda(x^2+2xy)\\
xy\cdot(x+y) &= 1\end{array}\right\}\implies
\left\{
\begin{array}{cc}
1&= \lambda(2x+y)\\
1&= \lambda(x+2y)\\
xy\cdot(x+y) &= 1
\end{array}\right\} 
\]

Solución: $x=y = \frac{1}{\sqrt[3]{2}}$

Como es un conjunto compacto se tiene que alcanzar el máximo y el mínimo. ¿Podemos saber a priori si este punto crítico es máximo o mínimo? Sí!. Tiene que ser el máximo, porque el mínimo se alcanza en los extremos \footnote{en la función sin truncar es $x  = \infty$}.






\end{problem}