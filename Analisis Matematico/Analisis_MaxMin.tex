\section{Máximos y mínimos condicionados}

\begin{defn}[Extremo \IS condicional]
$\Omega\subset\real^n$, abierto, $\appl{F}{\Omega}{\real}$ y M subvariedad de $\real^n, M\subset \Omega$

\textbf{Definición:}

\begin{itemize}
\item $F$ alcanza en $\ga\in M$ un extremo local condicionado a $M \dimplies$ existe un abierto \[U\in\real^n\, con \, \ga\in U \tlq \left\{\begin{matrix} F(\gx) \leq F(\ga), \forall \gx \in U\cap M \\
F(\gx) \geq F(\ga), \forall \gx \in u\cap M \end{matrix}\right.\]
\end{itemize}

\end{defn}

\paragraph{Ejemplos:}

\subparagraph{1)} $F(x,y) = x^2-y^2$. Sea $M = \{x^2+y^2 = 1\}$.

%Revisar. Falta pintar la circunferencia
\easyimg{imgs/MCNSM.png}{Mapa de conjuntos de nivel de la silla de montar}{lblMCNSM}

Si calculamos el máximo y el mínimo condicionados a la circunferencia de $R=1$ vemos que $\nabla F$ es paralelo al radio de la circunferencia, como consecuencia de la tangencia entre el conjunto de nivel y la subvariedad $M$. 
