

\chapter{Ejercicios}
Ultimos días de clase:
\subsection{Repaso de integración}
\subsubsection{Cambio de variables}

D, región acotada en el 1 cuadrante limitada por:

\[\begin{array}{cc}
xy&=1\\xy&=2\\\frac{y}{x} &=1\\\frac{y}{x}&=4
\end{array}\]

Cuya represencación gráfica es:

\todo{Dibujo}

\textbf{Demostrar:}\[\iint_D f(xy)\id{x,y} = lg(2)\int_1^2f(u)\id{u}\]

Para ello definimos un cambio de variable de la siguiente manera:

\begin{gather*}
xy=u\\
\frac{y}{x}=v
\end{gather*}

Vemos que se simplifica porque $u\in[1,2],v\in[1,4]$. La región en este caso queda:

\todo{Dibujo}

Aplicando este cambio de variable a la integral tenemos:

\[\iint_D f(xy)\id{x,y} = \int_1^2\int_1^4 f(u) \det \left(\dpa{(x,y)}{(u,v)}\right)\id{v,u}\]

Necesitamos despejar $x$ e $y$ en función de $u,v$. En este caso tenemos: 
\begin{gather*}
x=+\sqrt{\frac{u}{v}}\\ y= + \sqrt{uv}
\end{gather*}

¿Para qué? PAra poder calcular el determinante de la matriz jacobiana, que en este caso $\displaystyle\det = \frac{1}{2v}$

\[\int_1^2\int_1^4 f(u)\cdot  \frac{1}{2v}\id{v,u} = \frac{1}{2} \int_1^2 f(u) \left.lg(v)\right|_{v=1}^4 \id{u}\]

\textbf{Hemos recordado los 3 pasos al cambiar de variables:}
\begin{itemize}
\item Cambio en la variable
\item Cambio en los límites de integración
\item Cambio en la medida (determiante del Jacobiano)
\end{itemize}


\paragraph{Coordenadas cilíndricas y esféricas}

Sea $\Omega$ la región contenida en el $1^{er}$ octante limitada por:

\begin{gather*}
z=x^2+y^2\\
1=x^2+y^2
\end{gather*}

Queremos escribir la integral en cartesianas, en cilíndricas y en esféricas

\[Vol = \iiint_{\Omega} \id{x,y,z}\]


\paragraph{Cartesianas}
1) Pintamos la proyección en el plano YZ para ver entre qué valores se mueve la $x$. Para ello tenemos que distinguir 2 regiones:

\todo{Dibujo}

\[Vol = \iint_{1} \int_{PAR}^{CIL} \id{x,y,z} + \iint_2 \int_0^{CIL} \id{x,y,z}\]

En ambos, la $y$ se mueve entre 0,1 y la z entre la parábola y 1


\[Vol = \int_{PAR}^1\int_0^1 \int_{PAR}^{CIL} \id{x,y,z} + \int_{PAR}^1\int_0^1 \int_0^{CIL} \id{x,y,z}\]


Pero este camino es un poco complicado. Es más natural integrar $\id{z,y,x}$

\[Vol = \iint_C \int_0^{PAR} \id{z,y,x} = \int_0^1 \int_0^{\sqrt{1-x^2}} \int_0^{x^2+y^2}\id{z,y,x}\]

\paragraph{Cilíndricas}

\begin{gather*}
x=r\,cos\,\theta\\
y=r\,sen\,\theta\\
z=z
\end{gather*}
En este caso, lo natural es escribir la integral en el orden $\id{z,\theta,r}$.

\[Vol = \int_0^1\int_0^{\frac{\pi}{2}} \int_0^{r^2} r \id{z,\theta,r}\]

La $r$ aparece porque es el cambio en la medida.

\paragraph{Esféricas}

\begin{gather*}
x = \rho cos\,\theta\,sen\,\phi\\
y = \rho sen\,\theta\,sen\,\phi\\
z = \rho cos\,\phi
\end{gather*}

\[Vol = \int_{\frac{\pi}{4}}^{\frac{\pi}{2}} \int_0^{\frac{\pi}{2}} \int_{PAR}^{CIL}\rho\id{\rho,\theta,\phi}\]

El $\rho$ aparece por el cambio en la medida.

Sustituyendo en las ecuaciones de las superficies tenemos:
\begin{gather*}
PAR \,: \, \rho = \frac{cos\phi}{sen^2\phi}\\
CIL \,:\, \rho = \frac{1}{sen\phi}
\end{gather*}

\textbf{Entonces}

\[Vol = \int_{\displaystyle\frac{\pi}{4}}^{\displaystyle\frac{\pi}{2}} \int_0^{\displaystyle\frac{\pi}{2}} \int_{\frac{cos\phi}{sen^2\phi}}^{\frac{1}{sen\phi}}\rho\id{\rho,\theta,\phi}\]


\begin{example}
en $\real^2$.

Sea \[F(x,y) = \left(\frac{y}{x^2+y^2}, \frac{-x}{x^2+y^2}\right)\]

\textbf{1)}
Queremos calcular la integral sobre $C_1$, una fircunferencia centrada en el origen.
\[C_1 = \{(cos\,\theta,sen\,\theta),\theta\in[0,2\pi]\}\]

Entonces: $\sigma'(\theta) = (-sen\,\theta,cps(\theta)$

Con la derivada vemos que la orientación es la negativa.

\[\int_{C_1^+} \vf d\sigma = -\int_0^{2\pi} \pesc{\vf\circ\sigma,\sigma'}d\theta\]

\textbf{2)}
¿Lo mismo en una circunferencia no centrada en el origen?

Sale $0$, por el 3)

\textbf{3)} 
¿Lo mismo en un monstruito?

Podemos aplicar el teorema de \textbf{Green} si no podemos parametrizar la curva.

\[\int_{C_3^+} \vf d\sigma_3 = \iint_{D_3} \dpa{Q}{x} - \dpa{P}{y} \id{x,y}\]

Teniendo cuidado con la orientación. En el monstruito propuesto en clase, seguíamos la orientación negativa.

\[\int_{C_3^+} \vf d\sigma_3 =  - \iint_{D_3} \dpa{Q}{x} - \dpa{P}{y} \id{x,y} = 0\]


\obs
¿Y porqué en el primer ejemplo no salía 0? Lo que pasa es que $\vf$ no es $C^1$ en el origen y no podemos aplicar \textbf{Green}.

Si $(0,0)\in D \implies $ no podemos aplicar Green con este campo.

¿Cómo puede ser que la integral del campo en la circunferencia centrada en el origen de $2\pi$ y centrada en un punto para no contener al origen de $0$?  \textbf{Interesante...} el origen es una singularidad muy especial. ¿Y si la curva pasa por el origen? ¿Y si es un ángulo y no una circunferencia? Si quieres saber la respuesta a estas preguntas sigue estudiando matemáticas toda tu vida.

\end{example}

Falta la clase del 18/Diciembre

\begin{example}
\[\vf = (xy,y^2+e^{xz^2},sen(xy))\]
Tenemos una región acotada por 
\begin{gather*}
x=0\\
z=0\\
y=0\\
y+z=2\\
z=1-x^2
\end{gather*}

Queremos calcular:

\[\iint_{S\equiv\partial \Omega} \vf \id{S}\]

\todo{Dibujo}

\paragraph{Solución}

Tenemos un par de posibilidades. Una de ellas es dividir en 5 superficies y hacer las 5 integrales de superficie o aplicar le teorema de \textbf{Gauss} (que es la que vamos a hacer porque es más fácil)

Para aplicar \textbf{Gauss} necesitamos la divergencia:

\[\div\vf = ... = 3y\]
\[
\iint_{S^+} \vf \id{S} = \iiint_{\Omega} \div\vf \id{x,y,z}
\]

Vamos a razonar el orden de los límites de integración.

Si cogemos $\id{z,y,x}$, las $x,y$ quedan fáciles (el rectángulo [0,2]x[0,1]) y dividir la integral de $z$ en 2 trozos.

\[
\iiint_{\Omega} \div\vf \id{x,y,z} = \iint_{1} \int_0^{1-x^2} 3y\id{z,y,x} + \iint_{2} \int_0^{1-x^2} 3y\id{z,y,x}
\]

Para calcular la curva que divide los trozos 1 y 2:

\[\left\{\begin{array}{cc}
y+z&=2\\
z&=1-x^2
\end{array} \right\} y+1-x^2=2\]

Con esto ya podemos completar la integral:
\[
\iiint_{\Omega} \div\vf \id{x,y,z} = \iint_{1} \int_0^{1-x^2} 3y\id{z,y,x} + \iint_{2} \int_0^{1-x^2} 3y\id{z,y,x} = \int_0^1\int_0^{1+x^2}\int_0^{1-x^2}3y\id{z,y,x}+\int_0^1\int_{1+x^2}^{2}\int_0^{2-y}3y\id{z,y,x}
\]

¿Existe alguna manera de plantear la integral sin tener que dividirla en 2 trozos? \textbf{Sí} $\id{y,z,x}$. Integrando primero la $y$.

\[
\iiint_{\Omega} \div\vf\id{y,z,x} = \int_0^1\int_0^{1-x^2}\int_0^{2-z}3y\id{y,z,x}
\]

\end{example}

\section{Ejercicios}
\subsection{Hoja 1}

\begin{problem}[?]
\solution
$\overline{A} = { x \in \real^N; \forall V_x \tq V_x \cup A \neq \text{\O}}$, siendo $V_x$ un entorno abierto de x.
	$\overline{A} = A \cap $ 
	
\begin{theorem}
$A \subset \real^N$ es cerrado $\dimplies \acum{A}\subset A$ 
\end{theorem}

 
\end{problem}
\begin{proof}
\begin{gather*}
A \text{es cerrado} \implies A^c \text{ es abierto} \implies\\
\forall x \in A^c, \exists \varepsilon > 0 \tq B(x,\varepsilon) \subset A^c \implies\\
A \cap B(x,\varepsilon) = \text{\O} \implies x \nexists \acum(A) 
\end{gather*}
\end{proof}
Falta la recíproca.


\begin{problem}[3] 
\solution
a)

$\displaystyle\bigcup_{k=1}^{\infty} \left[-1,\frac{1}{k}\right)$

Es cerrado, porque $=[-1,0]$
Demostración: (hay que demostrar las inclusiones $\subseteq$ y $\supseteq$)

b)
No es ni cerrado ni abierto.
\obs $\real$ es el cierre de $\mathbb{Q}$.

c)
 
\end{problem}

\newpage
\subsection{Hoja 2}

\begin{problem}[?]
\solution
$f(x,y) = \left\{\begin{matrix}
                \displaystyle \frac{x^3}{x^2+7y^2} & \text{ si } (x,y) \neq (0,0)\\
                 0 & \text{ si } (x,y) = (0,0)
                \end{matrix}\right.$
Continuidad:

$$0\leq \left| \frac{x^3}{x^2+7y^2} \right| \leq \left| \frac{x^3}{x^2} \right| \leq |x| \rightarrow 0 \implies$$ Continua en 0.

Derivadas parciales en $(0,0)$

$$\mylim{h}{0} {\frac{f(h,0) - f(0,0)}{h}} = \mylim{h}{0} {\displaystyle\frac{{h^{3}}/{h^{2}}}{h}} = 1$$
$$\mylim{h}{0} {\frac{f(0,h) - f(0,0)}{h}} = \mylim{h}{0} {\frac{0}{0+7h^2}} = 0$$
\end{problem}
 
\subsection{5}

\paragraph{b)}
$C_{D_4}(b)$. Basta con comprobar la conmutación con $a^j$ y con $a^jb$ siendo $j = 0,1,2,3$, ya que con eso podemos ver la conmutación con todos los elementos. Se puede demostrar la conmutatividad multiplicando a derecha e iezquierda por $b$ y $b^{-1}$ y si nos queda $=1$, es conmutativo.

$$\left\{\begin{matrix}b(a^j)b^{-1} = a^{-j}, a^j \in C_{D_4}(b) \dimplies a^2j = 1\\
b(a^jb)b^{-1} = a^{-j} = a^{-j}b, a^jb \in C_{D_4}(b) \dimplies a^2j = 1\end{matrix}\right.$$

\subsection{9}
\paragraph{a)}
\paragraph{b)}

$$f(x,y) = \int_a^xy g(s)ds$$
Aplicando el teorema fundamental del cálculo $\left(f \text{ continua } \implies\displaystyle\int_a^b f(x)dx = F(b)-F(a)\right)$
$$\dpa{f}{x} = g(xy)\underbrace{\dpa{xy}{x}}_{=y} - \underbrace{g(a)\dpa{a}{x}}_{=0}  = g(xy)y$$
$$\dpa{f}{y} = \cdots  = g(xy)x$$

\subsection{Ejercicio de examen:}
$\appl{g}{\real}{\real}$ continua, con $g(1) = 4$.

Sea $f(x,y,z)=\displaystyle \int_0^{x^2ye^z} g(t)dt$.

Demostrar que $f$ es diferenciable y calcular $\nabla f(1,1,0)$.
\newpage
\section{Hoja 3}
\begin{problem}[3]
\paragraph{a)} Probar que si la derivada de $\appl{f}{\real}{\real}$ existe y no se anula entonces $f$ es inyectiva.
\paragraph{b)} Probar que $\appl{f}{\real^2}{\real^2}$ dada por $f(x,y) =( e^xcos(y) + 2e^xsen(y),-e^xcos(y))$ cumple que el determinante de su Jacobiano es siempre positivo pero sin embargo no tiene $f$ no es inyectiva.

\solution

a)

% \begin{itemize}
%  \item $f$ derivable $\implies f$ continua.
%  \item Si $f'(x) \neq 0, \forall x \in \real \implies f $ monótona (estrictamente creciente o decreciente). 
%  \end{itemize}
Me parece demasiado intuitivo...

b)
$$J = \begin{pmatrix}
       e^xcos(y)+2e^xsen(y) & e^xcos(y)+2e^xsen(y) \\
       -e^xcos(y) & e^xsen(y)
      \end{pmatrix}
$$

Calculamos $$det(J) = (e^xcos(y)+2e^xsen(y))+e^xsen(y) + e^xcos(y)sen(y)(-e^xsen(y)+2e^xcos(y)) = $$
$$ = ... = 2e^x > 0 \forall x \in \real$$

Aunque el jacobiano sea siempre positivo, $f$ no es inyectiva porque si tomamos $f(0,0) = (1,-1) = f(0,2\pi)$.
\end{problem}
\begin{problem}[inventado]
\label{inventado}
Sea $F(x,y) = (x^2-y^2,2xy)$. Encontrar los puntos en los que la siguiente aplicación es localmente inversible de clase $C^1$.
\solution
\begin{itemize}
 \item 1) $F \in C^1$ por ser $F_1,F_2$ polinomios.
 \item 2)$det(J)>0 \forall (x,y)\in \real^2$. 
 
 En este caso: $$det\begin{pmatrix}
                  2x&-2y\\
                  2x&2y
                 \end{pmatrix} = 4x^2 + 4y^2 = 0 \dimplies (x,y) = (0,0)$$           
 \item 3) Por el teorema de la funcion inversa, existe una inversa local de $F,C^1$ en todo entorno de $(x,y) \in \real^2$ con $(x,y)\neq (0,0)$. 
 
 Está la posibilidad de que exista la función inversa, pero no podemos deducir nada del teorema. Para verlo, recurrimos a la definición de inyectividad, y en este caso, no es inyectiva porque es una función par.
 \end{itemize}
 \end{problem}
 \begin{problem}[5]



 a)
 $f\in C^1(\real), f'\neq0$.
 No tiene sentido...
 $$\left\{\begin{matrix} u(x,y) =f(x)\\v(x,y) = -y + f(x)\end{matrix}\right.$$
 Probar que tiene inversa global.
 
 b) Si $f (0) = 0$ y $f' (0) = 1$, hallar las derivadas parciales de dicha inversa en el origen.

\solution

 Mismos pasos que en el ejercicio anterior:
 \begin{itemize}
 \item 1) $F \in C^1$ por ser $F_1,F_2$ , porque $f\in C^1$.
 \item 2)$det(J)>0 \forall (x,y)\in \real^2$.
 
 $$det(J) = det\begin{pmatrix} f'(x)&0\\f(x)+xf'(x)&-1\end{pmatrix} = -f'(x) \neq 0\text{ por hipótesis}$$
 \end{itemize}
 \paragraph{b)}
 Como nos piden calcular las derivadas parciales de la función inversa. (La inversa de la matriz jacobiana, es la jacobiana de la matriz inversa)
 $$J(0,0) = \begin{pmatrix}f'(0) & 0 \\f(0) & -1\end{pmatrix}$$
 Lo que buscamos en la matriz inversa, que en este caso es ella misma.
 
 El teorema solo nos demuestra la existencia de la inversa local (contraejemplo:(\ref{inventado})). Hay que ver la inyectividad para hablar de inversa global.
 
\begin{gather*}
F(x,y) = (u(x,y),v(x,y))\\
\text{Condición: }F(x_1,y_1) = F(x_2,y_2) \implies x_1=x_2, y_1=y_2\\
u(x_1,y_1) = u(x_2,y_2) \implies f(x_1) = f(x_2)\\
f' \text{ no se anula } \implies \text{f es inyectiva} \implies x_1=x_2\\
v(x_1,y_1) = v(x_2,y_2) -y_1 + x_1f(x_1) = -y_2 + x_2f(x_2) \underbrace{\implies}_{x_1=x_2}\\
y_1=y_2
\end{gather*}
Hemos demostrado que $F$ es inyectiva y por lo tanto admite inversa global.
 
\end{problem}
\begin{problem}[6]
Estudiar si se puede despejar $(x,y,z)$ en términos de $(u,v,w)$ 
$$F(x,y,z) = \left\{\begin{matrix}u = 2x+2x^2y+2x^2z+2xy^2+2xyz\\v=x+y+2xy+2x^2\\w=4x+y+z+3y^2+3z^2+6yz\end{matrix}\right.$$
\solution
\begin{itemize}
 \item $u,v,w \in C^1$ por se suma de polinomios. 
 \item \begin{gather*}
\dpa{u}{x} = ... \implies \dpa{u}{x}(0,0) = 2\\
\dpa{u}{y} = ... \implies \dpa{u}{y}(0,0) = 0\\
\dpa{u}{z} = ... \implies \dpa{u}{z}(0,0) = 0\\
\dpa{v}{x} = ... \implies \dpa{v}{x}(0,0) = 1\\
\dpa{v}{y} = ... \implies \dpa{v}{y}(0,0) = 0\\
\dpa{v}{z} = ... \implies \dpa{v}{z}(0,0) = 0\\
\dpa{w}{x} = ... \implies \dpa{w}{x}(0,0) = 4\\
\dpa{w}{y} = ... \implies \dpa{x}{y}(0,0) = 1\\
\dpa{w}{z} = ... \implies \dpa{w}{z}(0,0) = 1
       \end{gather*}
       
   $det(J) =\begin{pmatrix}
             2&0&0\\
             1&1&0\\
             4&1&1
            \end{pmatrix}
 = 2 \neq 0 \implies \exists $ inversa local de clase $C^1$ en un entonrno de cualquier punto, en concreto en un entorno del origen.
\end{itemize}
\end{problem}

\begin{problem}[8]
\solution
\paragraph{a)}

$$det(J) = det\begin{pmatrix}
       cos(\varphi)&-rsen(\varphi)&0\\
       sen(\varphi)&rcos(\varphi)&0\\
       0&0&1
      \end{pmatrix} = rcos^2(\varphi) + rsen^2(\varphi) = r$$
      
      Por tanto, por el teorema de la función inversa, existe una inversa de clase $C^1, \forall (r,h,\varphi) \dimplies r\neq 0$.
\end{problem}
 \begin{problem}[9]
 \solution
 \paragraph{b: Calcular la inversa en (2,-2$\sqrt{3}$)}
 
 Resolver: $$\left\{\begin{matrix} 2 = rcos(\varphi)\\-2\sqrt{3} = rsen(\varphi)\end{matrix}\right.$$
 
 Hay que hallar la inversa de: $$\begin{pmatrix}
                                  \frac{1}{2}&2\sqrt{3}\\
                                  \frac{-\sqrt{3}}{2}&2
                                 \end{pmatrix}$$

    \end{problem}                             
  \begin{problem}[13]
  \solution
  
  $$\det(J) = \det \begin{pmatrix}
              \dpa{f_1}{x}&\dpa{f_1}{y}\\
              \dpa{f_2}{x}&\dpa{f_2}{y}
             \end{pmatrix} = 
             \det \begin{pmatrix}
              \dpa{f_1}{x}&-\dpa{f_2}{x}\\
              \dpa{f_2}{x}& \dpa{f_1}{x}
             \end{pmatrix}
      = \left(\dpa{f_1}{x}\right)^2 + \left(\dpa{f_2}{x}\right)^2 \implies \left(\dpa{f_1}{x},\dpa{f_2}{x}\right)$$
Esto es aplicando la primera ecuación de Cauchy-Riemman. Obteniendo una condición

Aplicando la otra condición en el jacobiano llegamos a $\displaystyle\left(\dpa{f_1}{y},\dpa{f_2}{y}\right)\neq (0,0)$
\paragraph{c)}

Queremos ver que $g(x,y) = (f_1(x,y)^2-f_2(x,y)^2,2f_1(x,y)f_2(x,y))$ cumple las ecuaciones de Cauchy-Riemman. Facilito.
\end{problem}

\section{Exámenes}

\subsection{Parcial 1}

\begin{problem}[1]

$\appl{d}{\real^2}{\real}$ homógenea\footnote{$(f(tx,ty) = t^mf(x,y)$} con $m>0$. F acotada superiormente sobre la circunferencia unidad.

\ppart Probar que $F$ es continua en $(0,0)$

\ppart Si $m=0$ ¿F continua en (0,0)?

\solution

\spart
\subparagraph{1)}
Vamos a ver que $F(0,0) = 0$.

$F(0,0) = F(t0,t0) = t^mF(0,0) \implies F(0,0) = 0$

\subparagraph{2)}
Aplicando la caracterización por sucesiones de la continuidad: $\{X_n\} \rightarrow 0 \implies F(X_n) \rightarrow 0$ si $F$ continua.

Sea $Z_n = \{(X_n,Y_n)\}, n\in \mathbb{N}$ una sucesión tal que $Z_n \rightarrow (0,0)$.

\begin{gather*}\abs{F_n} = \abs{F\left(\frac{Z_n}{\md{Z_n}}\md{Z_n}\right)} =
\md{X_n}^n \abs{F\left(\frac{Z_n}{\md{Z_n}}\right)} \implies \\
\exists M>0 \tlq \md{Z_n}^n \abs{F\left(\frac{Z_n}{\md{Z_n}}\right)} \leq \md{Z_n}^n M \rightarrow 0 = F(0,0)
\end{gather*}

\spart
Sí, $m=0 \implies F$ constante. Se puede demostrar pasando a coordenadas polares $x=rcos(t),y=rsen(t)$

\end{problem}

%\begin{problem}[2]
%\solution
%\end{problem}
\subsection{Enero 2013}
\begin{problem}[1]

Sea $f\in C^2(\mathbb{R})$ tal que $f(2) = 0$, $f'(2)=1$. Consideramos la ecuación \\$$F(x,y,z) = x^2+y^2+z^2-f(2+Cz)=0$$

\ppart Probar que existen un abierto $U\subset \mathbb{R}^2$ con $(0,0)\in U$, y un abierto $V\subset \mathbb{R}$ con $0\in V$, y una función $C^1$ \[ \phi: U \rightarrow V \]

tal que $F(x, y, \phi (x, y)) = 0$ para todo $(x,y) \in U$.

\ppart Hallar las derivadas parciales de la función encontrada en el apartado anterior y demostrar que 

\[ y\frac{\partial  \phi (x,y)}{\partial  x} - x\frac{\partial  \phi (x,y)}{\partial  y} = 0 \]

\ppart Estudiar si la función $\phi$ alcanza un extremo relativo en $(0,0)$, y determinar su tipo (en función de la constante C).

\solution

\spart 
\[ \frac{\partial  F}{\partial  z}(0,0,0) = 2z - f'(2+Cz)C (0,0,0) =-f'(2)C = -C \neq 0 \]
con lo cual si $c\neq 0$, por el teorema de la función implícita, existen entornos $U\subset \mathbb{R}^2$ y $V\subset{R}$ con $(0,0)\in U, 0\in V$ y una función $\appl{\phi}{U}{V}$ de clase $C^1$ con $\phi (0,0) = 0$ tal que $F(x,y,\phi (x,y)) = 0 \forall (x,y)\in U$

Falta comprobar el resto de hipótesis del teorema de la función implícita.

\spart 
Derivamos implícitamente respecto de $x$:
\begin{gather*}
2\phi (x,y) \frac{\partial  \phi (x,y)}{\partial  x} - f'(2+C\phi (x,y))C\frac{\partial  \phi (x,y)}{\partial  x}=-2x \\
\hdots \\
\frac{\partial  \phi (x,y)}{\partial  x} = \frac{-2x}{2\phi (x,y) - Cf'(2+C\phi (x,y)}
\end{gather*}

 si $2\phi (x,y) \neq Cf'(2+C\phi (x,y))$.

Procediendo del mismo modo: 
\[ \frac{\partial  \phi (x,y)}{\partial  y} = \frac{-2y}{2\phi (x,y) - Cf'(2+C\phi (x,y)} \] 

si $2\phi (x,y) \neq Cf'(2+C\phi (x,y))$.

Multiplicando a ambos lados la derivada parcial con respecto a $x$ por $y$ y la derivada parcial con respecto a $y$ por $x$ llegamos a que \[ \frac{\partial  \phi (x,y)}{\partial  x} = x\frac{\partial  \phi (x,y)}{\partial  y} \], que era lo que queríamos demostrar.

\spart 

\begin{gather*}
x^2 + y^2 + (\phi (x,y))^2 -f(2+C\phi (x,y)) = 0 \\
\frac{\partial  \phi}{\partial  x}(0,0) = \frac{0}{C} = 0 ||
\frac{\partial  \phi}{\partial  y}(0,0) = \frac{0}{C} = 0
\end{gather*}
Y por lo tanto $\phi$ tiene en $(0,0)$ un extremo relativo.

Procedemos a calcular su tipo:

\begin{gather*}
\frac{\partial  F}{\partial  x}: 2x-2\phi (x,y)\frac{\partial  \phi}{\partial  x}-f'(2+C\phi (x, y))C\frac{\partial  \phi}{\partial  x} = 0 \\
\frac{\partial  ^2 F}{\partial  x^2}(0,0): 2-2[(\frac{\partial  \phi (x,y)}{\partial  x})^2] - f''(2+C\phi (x,y))C^2(\frac{\partial  \phi}{\partial  x})^2
\end{gather*}

Lo siento chicos... no me ha dado tiempo a copiar, es mi primer día copiando cálculo. Pero el resultado es que el extremo es:

\begin{itemize}
\item Mínimo si $C>0$
\item Máximo si $C<0$
\item Nada si $C=0$ porque $2/C$ no estaría definido. 
\end{itemize}

No se hacer matrices, pero si os apetece completar el hessiano es $(2/C, 0)$ la primera fila, y $(0, 2/C)$ la segunda fila. Por tanto el determinante da $4/C^2 > 0$ y hay que analizar el término en la fila uno, columna uno de la matriz.
\end{problem}

\subsection{Junio 2013}
\begin{problem}[1]
 Sea $\appl{f}{\mathbb{R}^3}{\mathbb{R}}$. Se dice que f es homogénea de grado m si\\
$$f(\lambda x, \lambda y, \lambda z) = \lambda ^m f(x,y,z)$$ para todo $\lambda > 0$.

\ppart  Si f es diferenciable y homogénea de grado $m>0$ demostrar que:
\[ \pesc{\nabla f(x,y,z), (x,y,z)} = mf(x,y,z) \]

\ppart Supongamos $f$ continua sobre la esfera ${x^2+y^2+z^2=1}$. Supongamos que $f$ es homogénea de grado $m = 0$, y que $f(0,0,0) = 0$.

\begin{itemize}
\item Estudiar la continuidad de $f$ en el origen.
\item Estudiar la continuidad de $f$ en el resto de puntos del espacio.
\end{itemize}

\solution

\spart \[ x\frac{\partial  f}{\partial  x}(x,y,z) + y\frac{\partial  f}{\partial  y}(x,y,z) + z\frac{\partial  f}{\partial  z}\]
derivamos respecto de $\lambda$

$$\frac{\partial  f}{\partial  \lambda}(\lambda x, \lambda y, \lambda z) = \underbrace{\frac{\partial }{\partial  \lambda}(\lambda^mf(x,y,z))}_{m\lambda^{m-1}f(x,y,z)}$$

$$\frac{\partial  f}{\partial  \lambda x}\frac{\partial  \lambda x}{\partial  \lambda} + \frac{\partial  f}{\partial  \lambda y}\frac{\partial  \lambda y}{\partial  \lambda} + \frac{\partial  f}{\partial  \lambda z}\frac{\partial  \lambda z}{\partial  \lambda} = \frac{\partial  f}{\partial  \lambda x}x + \frac{\partial  f}{\partial  \lambda y}y + \frac{\partial  f}{\partial  \lambda z}z = m\lambda ^{m-1} f(x,y,z) $$

$$<\nabla f(\lambda x,\lambda y,\lambda z), (x,y,z) > = m\lambda^{m-1} f(x,y,z)\ \forall \lambda > 0$$

Si tomamos $\lambda = 1$ entonces $$<\nabla f(x,y,z), (x,y,z) > = mf(x,y,z)$$ que era lo que queríamos demostrar.\footnote{Este es el teorema de euler de las funciones homogéneas, y sí, te piden demostrarlo en un examen, extraordinrio}


\spart 
Considero\footnote{Esto es una idea feliz.} $\lambda = \frac{1}{\norm{(x,y,z)}}$\\
\[ f(x,y,z) = f(\lambda x, \lambda y, \lambda z) \]
Si f es continua en $x_1$ lo va a ser en $x_2$ con $x_1 = \lambda x_2$ por se homogénea.\footnote{Es como tomar un punto $x_2$ en una esfera más pequeña de la esfera de radio uno, y si unimos ese punto con el $x_1$ que esta en la esfera de radio uno, por ser homogénea va a ser continua en la recta que los une.}

Formalmente:

\begin{gather*}
\forall \epsilon > 0, \exists δ  > 0 \tq  \norm{x-y} < δ  \implies \norm{f(x) - f(y)} < \epsilon \\
\norm{x - 0} < δ  \implies \abs{f(x,y,z) - f(0,0,0)} < \epsilon \\
\abs{f(x,y,z)} < \epsilon\ \forall \epsilon > 0 \\
 f = 0 \\
 \frac{x}{\norm{x}} - \frac{y}{\norm{y}} < \norm{x-y} 
 \end{gather*}
 
\end{problem}

