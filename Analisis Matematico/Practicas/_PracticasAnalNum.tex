\section{Hojas de ejercicios}

\subsection{Hoja 1}

\begin{problem}[1] Denotamos por $\md{x}$ la norma euclídea asociada al producto escalar en $ℝ^N$:

\[ \pesc{x,y} = \sum_{i=1}^Nx_iy_i \]

Probar las dos identidades siguientes y dar una interpretación geométrica:

\ppart $2\md{x}^2 + 2\md{y}^2 = \md{x+y}^2 + \md{x-y}^2$.

\ppart $4\pesc{x,y} = \md{x+y}^2 - \md{x-y}^2$.

\solution

\spart Sabiendo que $\md{x}^2 = \pesc{x,x}$ y usando las propiedades del producto escalar (distributiva, multiplicación por constante, conmutativa), tenemos que 

\begin{gather*}
\md{x+y}^2 = \pesc{x,x} + 2\pesc{x,y} + \pesc{y,y} \\ 
\md{x-y}^2 = \pesc{x,x} - 2\pesc{x,y} + \pesc{y,y} 
\end{gather*}

Sumando estos dos términos tenemos el resultado de la parte izquierda de la ecuación.

\spart Trivial usando los dos cálculos anteriores.

\end{problem}

\begin{problem}[3] 
\solution
\spart

$\displaystyle\bigcup_{k=1}^{\infty} \left[-1,\frac{1}{k}\right)$

Es cerrado, porque $=[-1,0]$
Demostración: (hay que demostrar las inclusiones $\subseteq$ y $\supseteq$)

\spart
No es ni cerrado ni abierto.
\obs $\real$ es el cierre de $\mathbb{Q}$.

\spart
 
\end{problem}


\begin{problem}[7] Demostrar que $A⊆ℝ^N$ es compacto (\ref{defCompacto}) si y sólo si cualquier subconjunto infinito de $A$ tiene algún punto de acumulación (\ref{defPtoAcc}) que pertenece a $A$.

\solution

Empezamos demostrando la implicación a la derecha (si $A⊆ℝ^N$ es compacto cualquier subconjunto infinito de $A$ tiene algún punto de acumulación que pertenece a $A$). Supongamos que, para un $B⊆A$, $B$ no tiene ningún punto de acumulación que pertenezca a $A$. Hay dos posibilidades: o bien sus puntos de acumulación no pertenecen a $A$ o bien no tiene puntos de acumulación.

En el primer caso, dado que $B⊆A$, si $p$ es un punto de acumulación de $B$ también lo es de $A$. Entonces podríamos encontrar una sucesión $\{x_n\} ⊆ A$ con $\lim x_n = p$, pero como $p\notin A$ $A$ no estaría cerrado (ver teorema \ref{thmCerradoSucesiones}) y por lo tanto no puede ser compacto.

En el segundo caso, tenemos que $B$ es un conjunto de puntos aislados. Es decir, que ninguna sucesión $\{x_n\} ⊆ B$ puede ser convergente y por lo tanto, según las definiciones de conjunto compacto (\ref{thmCerradoAcotado}, propiedad \ref{propSucesion}) no es compacto.

Por lo tanto, si suponemos que algún subconjunto infinito de $A$ no tiene puntos de acumulación en $A$, $A$ no puede ser compacto. 

Supongamos ahora, para demostrar la implicación a la izquierda por reducción al absurdo, que $A$ no es compacto pero que todo subconjunto infinito $B⊆A$ tiene algún punto de acumulación en $A$. Consideramos el subconjunto infinito dado por una sucesión $\{x_n\} ⊆ A$ infinita. Según la premisa, existe un punto de acumulación en este subconjunto, y por lo tanto podemos encontrar una subsucesión $\{x_{n_j}\}⊆\{x_n\}$ que converja a ese punto de acumulación. Se cumple por tanto la propiedad \ref{propSucesion} del teorema (\ref{thmCerradoAcotado}), contradicción.


\end{problem}

\subsection{Hoja 2}

\begin{problem}[?]
\solution
$f(x,y) = \left\{\begin{matrix}
                \displaystyle \frac{x^3}{x^2+7y^2} & \text{ si } (x,y) \neq (0,0)\\
                 0 & \text{ si } (x,y) = (0,0)
                \end{matrix}\right.$
Continuidad:

$$0\leq \left| \frac{x^3}{x^2+7y^2} \right| \leq \left| \frac{x^3}{x^2} \right| \leq |x| \rightarrow 0 \implies$$ Continua en 0.

Derivadas parciales en $(0,0)$
\begin{gather*}
\lim_{h \to 0} {\frac{f(h,0) - f(0,0)}{h}} = \mylim{h}{0} {\displaystyle\frac{{h^{3}}/{h^{2}}}{h}} = 1\\
\lim_{h\to 0} {\frac{f(0,h) - f(0,0)}{h}} = \mylim{h}{0} {\frac{0}{0+7h^2}} = 0
\end{gather*}
\end{problem}
 
\begin{problem}[5]
a saber
\solution

\paragraph{b)}
$C_{D_4}(b)$. Basta con comprobar la conmutación con $a^j$ y con $a^jb$ siendo $j = 0,1,2,3$, ya que con eso podemos ver la conmutación con todos los elementos. Se puede demostrar la conmutatividad multiplicando a derecha e iezquierda por $b$ y $b^{-1}$ y si nos queda $=1$, es conmutativo.

$$\left\{\begin{matrix}b(a^j)b^{-1} = a^{-j}, a^j \in C_{D_4}(b) \dimplies a^2j = 1\\
b(a^jb)b^{-1} = a^{-j} = a^{-j}b, a^jb \in C_{D_4}(b) \dimplies a^2j = 1\end{matrix}\right.$$
\end{problem}


\begin{problem}[9] Se supone $\appl{g}{ℝ}{ℝ}$ función continua. Hallar $Df(x,y)$ en los casos siguientes:

\ppart $\displaystyle f(x,y) = \int_a^{x+y}g(s)\id s $

\ppart $\displaystyle f(x,y) = \int_a^{xy}g(s)\id s$.

\solution

\spart Sea $φ(s)=\int g(s)\id s$. Entonces

\[ \dpa{f}{x} = \dpa{}{x}\left(φ(x+y) - φ(a)\right) = \dpa{φ(x+y)}{x} = g(x+y) \]

Análogamente, $\dpa{f}{y} = g(x+y)$. Finalmente

\[  Df(x,y) = \begin{pmatrix}g(x+y) & g(x+y)\end{pmatrix} \]

\spart Usando la misma notación del apartado anterior:

\[ \dpa{f}{x} = \dpa{}{x}\left(φ(xy)-φ(a)\right) = \dpa{φ(xy)}{x} = y · g(xy) \]

y análogamente, $\dpa{f}{y} = x·g(xy)$, de tal forma que 


\[  Df(x,y) = \begin{pmatrix} y·g(xy) & y·g(xy) \end{pmatrix} \]

\end{problem}


\begin{problem}[14] Considérese la función 

\[ f(x,y) = x^4 + y^4 + 6x^2y^2 + 8 x^3 \]

\ppart Estudiando el comportamiento de $h(x)=f(x,x)$, probar que $f$ no alcanza ni un máximo ni un mínimo relativo en el origen. 

\ppart Hallar los extremos relativos de $f$ indicando su carácter

\solution

\spart Calculamos $h$

\[ h(x) = x^4 + x^4 + 6x^2x^2 + 8 x^3 = 8x^3(x + 1) \]

y vemos que vale $0$ en el origen, que es positiva cuando $x>0$ y que es negativa cuando $-1<x<0$, por lo tanto no alcanza ni un máximo ni un mínimo en el origen.

\spart Hallamos el vector gradiente:

\[ \grad f = \left(3x^3 + 12xy^2 + 24x^2,\, 3y^3 + 12x^2y\right) = \left(x(3x^2 + 12y^2 + 24x),\, y(3y^2+12x^2)\right) \]

Igualamos a cero

\[ \left.\begin{matrix}
x(3x^2 + 12y^2 + 24x) = 0 \\
y(3y^2+12x^2) = 0  
\end{matrix}\right. \]

Este sistema tiene la solución $(0,0)$, que ya hemos visto antes que no es ni un máximo ni un mínimo, y además las soluciones al sistema 

\[ \left.\begin{matrix} 3x^2 + 12y^2 + 24x = 0 \\
3y^2 +12x^2 = 0
\end{matrix}\right\} 
\left.\begin{matrix} x^2 + 4y^2 + 8x = 0 &  \\
y^2 +4x^2 = 0 & \implies y^2 = -4x^2 
\end{matrix}\right\}  \]

que no tiene solución por ser la segunda ecuación imposible de cumplir.
\end{problem}

\subsection{Hoja 3}

\begin{problem}[0]
\label{ej3_0}
Sea $F(x,y) = (x^2-y^2,2xy)$. Encontrar los puntos en los que la siguiente aplicación es localmente inversible de clase $C^1$.
\solution
Primero,  $F \in C^1$ por ser $F_1,F_2$ polinomios. Además, el determinante del jacobiano es positivo $\forall (x,y)\in \real^2$. 
 
 En este caso: $$\det\begin{pmatrix}
                  2x&-2y\\
                  2x&2y
                 \end{pmatrix} = 4x^2 + 4y^2 = 0 \dimplies (x,y) = (0,0)$$           
                 
  Por el teorema de la función inversa, existe una inversa local de $F,C^1$ en todo entorno de $(x,y) \in \real^2$ con $(x,y)\neq (0,0)$. 
 
 Está la posibilidad de que exista la función inversa, pero no podemos deducir nada del teorema. Para verlo, recurrimos a la definición de inyectividad, y en este caso, no es inyectiva porque es una función par.

 \end{problem}

\begin{problem}[3]

\ppart Probar que si la derivada de $\appl{f}{\real}{\real}$ existe y no se anula entonces $f$ es inyectiva.

\ppart Probar que $\appl{f}{\real^2}{\real^2}$ dada por $f(x,y) =( e^x\cos(y) + 2e^x\sen(y),-e^x\cos(y))$ cumple que el determinante de su Jacobiano es siempre positivo pero sin embargo no tiene $f$ no es inyectiva.

\solution

\ppart Trivial.

\ppart  \[ DF = \begin{pmatrix}
       e^x\cos(y)+2e^x\sen(y) & e^x\cos(y)+2e^x\sen(y) \\
       -e^x\cos(y) & e^x\sen(y)
      \end{pmatrix} \]

Su determinante vale $2e^x > 0 \forall x \in \real$. Sin embargo, aunque el jacobiano sea siempre positivo, $f$ no es inyectiva porque si tomamos $f(0,0) = (1,-1) = f(0,2\pi)$.
\end{problem}


\begin{problem}[5] Si $f∈C^(ℝ)$ y $f'$ no se anula, demostrar que la función $F$ dada por 

\[ \begin{cases}
u(x,y) &= f(x) \\
v(x,y) &= -y + xf(x)
\end{cases} \]

tiene una inversa global. Si $f(0)=0$ y $f'(0)=1$, hallar las derivadas parciales de dicha inversa en el origen.
\solution

Calculamos el diferencial de $F$ y su determinante

\[ DF(x,y) = \begin{pmatrix}
f'(x) & 0 \\
xf'(x) + f(x) & -1 
\end{pmatrix};\; \det DF(x,y) = -f'(x) \]

Según el teorema de la función inversa (\ref{thmInv}), $F$ tiene inversa en todo punto ya que $DF$ es invertible ($f'$ no se anula). Sin embargo, sólo nos demuestra la existencia de la inversa local en todo punto, que no es lo mismo\footnote{Podemos ver un ejemplo de esto en \ref{ej3_0}.}. Hay que ver la inyectividad para hablar de inversa global.
 
Para demostrar que $F$ es inyectiva, tenemos que demostrar 

\[ F(x_1,y_1) = F(x_2,y_2) \implies x_1=x_2, y_1=y_2 \]

Viendo la construcción de la función, tenemos que 
\[ u(x_1,y_1) = u(x_2,y_2) \implies f(x_1) = f(x_2) \]

Como $f'$ no se anula, entonces $f$ es inyectiva y por lo tanto $x_1=x_2$. Sólo falta demostrar ahora para $v$:

\[ v(x_1,y_1) = v(x_2,y_2) \implies  -y_1 + x_1f(x_1) = -y_2 + x_2f(x_2) \]

Dado que antes hemos visto que $x_1=x_2$, entonces despejamos de la ecuación anterior y nos queda que $y_1=y_2$. Hemos demostrado que $F$ es inyectiva y por lo tanto admite inversa global.

Hallamos ahora las derivadas parciales en el origen. Para ello usamos la fórmula 

\[ DG(y) = [DF(\inv{F}(y))]^{-1} \]

Hallamos cuánto vale $\inv{F}(0,0)$. Si $u(x,y) = 0$, entonces $0 = f(x)$ y según el enunciado eso implica que $x=0$. Aplicando este resultado a la segunda ecuación nos queda que $y=0$. Por lo tanto

\[ DG(0,0) = [DF(0,0)]^{-1} = \begin{pmatrix}
1 & 0 \\
0 & -1
\end{pmatrix}^{-1} = 
\begin{pmatrix}
1 & 0 \\
0 & -1
\end{pmatrix} \]

\end{problem}

\begin{problem}[6]
Estudiar si se puede despejar $(x,y,z)$ en términos de $(u,v,w)$ 
\[ F(x,y,z) = \left\{\begin{matrix}
u = 2x+2x^2y+2x^2z+2xy^2+2xyz \\
v = x+y+2xy+2x^2 \\
w = 4x+y+z+3y^2+3z^2+6yz
\end{matrix}\right. \]

\solution
Tenemos que aplicar el teorema de la función inversa (\ref{thmInv}). Por un lado, tenemos que $u,v,w \in C^1$ por ser suma de polinomios. Por otro, calculamos el determinante del diferencial:
  
  \[ \det DF =\left|\begin{pmatrix}
             2&0&0\\
             1&1&0\\
             4&1&1
            \end{pmatrix}\right|
 = 2 \]
 
 que es distinto de 0. Por lo tanto existe inversa local de clase $C^1$ en un entorno de cualquier punto, en concreto en un entorno del origen.
\end{problem}

\begin{problem}[8]
Estudiar alrededor de qué puntos tienen inversa diferenciable los siguientes cambios de coordenada

\ppart \[ \begin{cases} 
x(r,φ,h) &= r\cos φ \\
y(r,φ,h) &= r\sin φ \\
z(r,φ,h) &= h 
\end{cases} \]

\ppart \[ \begin{cases} 
x(r,θ,φ) &= r\cos θ \sin φ \\
y(r,θ,φ) &= r\cos θ \sin φ \\
z(r,θ,φ) &= r\cos φ 
\end{cases} \]
\solution

\spart
Hallamos el determinante del jacobiano:

\[\det\begin{pmatrix}
\cos φ & -r\sen φ & 0\\
\sen φ & r\cos φ & 0\\
0 & 0 & 1
\end{pmatrix} = r\cos^2 φ + r\sen^2 φ = r \]
      
Por tanto, por el teorema de la función inversa, existe una inversa de clase $C^1,\, \forall (r,h,φ)$ si y sólo si $r\neq 0$.
\end{problem}

\begin{problem}[9]
Calcular la matriz de la diferencial de la función inversa del cambio a polares

\[ \begin{cases}
x(r,θ) &= r\cos θ \\
y(r,θ) &= r\sin θ
\end{cases} \]

alrededor del punto $(2, -2\sqrt{3})$.

\solution

Aplicando el teorema de la función inversa, calculamos la matriz diferencial

\[  DF(r,θ) = \begin{pmatrix}
\cos θ & - r \sin θ \\
\sin θ & r \cos θ 
\end{pmatrix} \]

Buscamos ahora $r,θ$ tales que $F(r,θ) = (2, -2\sqrt{3})$. Es decir, $r=\frac{4}{\sqrt{3}},\,θ = -\frac{\pi }{6}$. Entonces

\[ DF(\frac{4}{\sqrt{3}}, -\frac{\pi }{6} = \dotsb \]

y después hallaríamos su inversa.

\end{problem}    
                         
\begin{problem}[13]

Sea $\appl{f}{ℝ^2}{ℝ^2},\, f=(f_1, f_2)∈C^1(ℝ^2)$, satisfaciendo las ecuaciones de Cauchy-Riemann:

\[ \dpa{f_1}{x}=\dpa{f_2}{y},\quad \dpa{f_1}{y}=-\dpa{f_2}{x} \]

\ppart Demostrar que existe $\inv{f}$ diferenciable en algún abierto conteniendo a $(x_0, y_0)$ si y sólo si $Df(x_0, y_0)$ no es la aplicación lineal idénticamente nula.

\ppart Demostrar que si existe la inversa local del apartado anterior, entonces también satisface las ecuaciones de Cauchy-Riemann.

\ppart Suponiendo $f(0,0)≠(0,0)$, probar que $g$ dada por

\[ g(x,y = (f_1(x,y)^2-f_2(x,y)^2, 2f_1(x,y)f_2(x,y) \]

satisface las ecuaciones de Cauchy-Riemann y que $Df(0) = \vec{0}$ si y sólo si $Dg(0) = \vec{0}$.

\ppart Encontrar tres funciones no constantes que satisfagan las ecuaciones de Cauchy-Riemann

\solution
  
\spart \[ \det(J) = \det \begin{pmatrix}
              \dpa{f_1}{x}&\dpa{f_1}{y}\\
              \dpa{f_2}{x}&\dpa{f_2}{y}
             \end{pmatrix} = 
             \det \begin{pmatrix}
              \dpa{f_1}{x}&-\dpa{f_2}{x}\\
              \dpa{f_2}{x}& \dpa{f_1}{x}
             \end{pmatrix}
      = \left(\dpa{f_1}{x}\right)^2 + \left(\dpa{f_2}{x}\right)^2 \implies \left(\dpa{f_1}{x},\dpa{f_2}{x}\right) \]
      
Esto es aplicando la primera ecuación de Cauchy-Riemman. Obteniendo una condición

Aplicando la otra condición en el jacobiano llegamos a $\displaystyle\left(\dpa{f_1}{y},\dpa{f_2}{y}\right)\neq (0,0)$
\ppart

Queremos ver que $g(x,y) = (f_1(x,y)^2-f_2(x,y)^2,2f_1(x,y)f_2(x,y))$ cumple las ecuaciones de Cauchy-Riemman. Facilito.
\end{problem}

 
\begin{problem}[15]
\[sen(xz)+sen(yz)+sen(xy) = 0\]
a) Admite una única solución $z=f(x,y) C^1$ en un entorno del punto $(\pi,0)$ con $f(\pi,0) = 1$

b) Polinomio de Taylor de orden 1

\solution
\paragraph{a)}
Sea $F(x,y,z) = sen(xz)+sen(yz)+sen(xy)$.
\begin{itemize}
\item $F \in C^{\infty}$ por ser todo senos.
\item $F(\pi,0,1) = 0$
\item $z = f(x,y)$
\[\dpa{F}{z} = cos(yz)y + cos(xz)x\]
\[\dpa{F}{z}(\pi,0,1) = -\pi \neq 0\]
Por el teorema de la función implícita, $\exists f \in C^1(\real^2) \tlq z = f(x,y)$ en un entorno de $(\pi,0,1)$.
\end{itemize}
\paragraph{b)}
\[T_1f(\pi,0) =  f(\pi,0) + \dpa{f}{x}(\pi,0)(x-\pi) + \dpa{f}{y}(\pi,0)(x-\pi)\]

Calculamos la derivada respecto de $x$:
\[\dpa{f}{x} : cos(xy)y\dpa{z}{x}(\pi,0) + cos(xz)\left(z+\dpa{z}{x}(\pi,0)\right) + cos(xy)y = 0\]
Evaluando en $(\pi,0,1)$
\[= -\left(1+\dpa{z}{x}(\pi,0)\right) = 0 \equiv \dpa{z}{x}(\pi,0) = -1\]

Hay algún error, porque
$ \dpa{z}{x}(\pi,0) = \frac{-1}{\pi}$

Ahora vamos a calcular la derivada respecto de $y$

\[\dpa{z}{y}(\pi,0): cos(yz)\left(z+y\dpa{z}{y}\right) + cos(xz)x\dpa{z}{y} + cos(xy)x = 0\]
Evaluando en $(\pi,0,1)$
\[1-\pi\dpa{z}{y}(\pi,0) + \pi = 0\]
\[\implies \dpa{z}{y}(\pi,0)= \frac{1}{\pi} + 1\]

Ya tenemos todos los valores para calcular el polinomio de newton.
\end{problem}
 
  \begin{problem}[16]
 \solution
 Visto en teoría
 \end{problem}
 
 \begin{problem}[18]
 \solution
 Visto en teoría
 \end{problem}
\begin{problem}[19]
Dado el sistema:

\[\left\{\begin{array}{cccc}
	 f_1 &\equiv& x^2 -ycos(\mu v) + z^2 &= 1\\
	f_2 &\equiv &x^2+y^2-sen(\mu)cos(v)+2z^2 &= 4\\
	f_3 &\equiv &xy-sen(u)cos(v) + z &= 1 \end{array} \right.\]
Demostrar que se puede definir $(x,y,z)$ en función de $u,v$ en un entorno del punto $(1,1,1\frac{\pi}{2},0) $.

\paragraph{b)} Hallar también $DF(\frac{\pi}{2},0) $

\solution
	
Definimos $F = (f_1,f_2,f_3), \appl{F}{\real^5}{\real^3}$

\begin{itemize}
\item $F \in C^1$
\item $F(1,1,1,\frac{\pi}{2},0)$ satisface la ecuación
\item Vamos a estudiar el determinante:

\[\det J(1,1,1,\frac{\pi}{2},0) = 
	\begin{pmatrix}
		\dpa{f_1}{x} &\dots &\dpa{f_1}{z}\\
		\vdots & \ddots &\vdots\\
		\dpa{f_3}{x} & \dots & \dpa{f_3}{z}
	\end{pmatrix} =
	\begin{pmatrix}
		2&-1&2\\
		2&2&4\\
		1&1&1
	\end{pmatrix} = -6 \neq 0\]	
\end{itemize}

Nos piden también $DF(\frac{\pi}{2},0)$. Derivamos implícitamente $f_i$ respecto de u:

\begin{itemize}
\item $f_1$
\[2x\dpa{x}{u} - \dpa{y}{u}cos(uv)+ysen(uv)v + 2z\dpa{z}{u} =0\]
Evaluamos en el punto $(1,1,1,\frac{\pi}{2})$.
\[2\dpa{x}{u}(\frac{\pi}{2},0) - \dpa{y}{u}(\frac{\pi}{2},0) + 2 \dpa{z}{u}(\frac{\pi}{2},0) = 0\]
 \item $f_2$
 \[2x\dpa{x}{u} + 2y\dpa{y}{u}-cos(uv)v+4z\dpa{z}{u} = 0\]
 Evaluamos en el punto:
 \[2\dpa{x}{u} (\frac{\pi}{2},0)  + 2\dpa{y}{u}(\frac{\pi}{2},0)  - 4 \dpa{z}{u}(\frac{\pi}{2},0) = 0 \]
 \item $f_3$
 ...
 Evaluamos en el punto
 \[\dpa{x}{u}(\frac{\pi}{2},0) +\dpa{y}{u}(\frac{\pi}{2},0)  + \dpa{z}{u}(\frac{\pi}{2},0) = 0 \]
\end{itemize}

Ahora tenemos que resolver el sistema:
(Ejercicio para el lector). 

Queda $D_uF(\frac{\pi}{2},0)  = (0,0,0)$.

Repetimos el mismo proceso respecto de v (Ejercicio también para ti)

Obtenemos $D_vF(\frac{\pi}{2},0)  = (0,0,0)$

\textbf{Conclusión:} $DF(\frac{\pi}{2}) = 0_{3\times 2}$
\end{problem}

 \begin{problem}[20]
 \solution
 Visto en teoría
 \end{problem}
 
  \begin{problem}[21]
 \solution
 Visto en teoría
 \end{problem}
 
 
\begin{problem}[24]
\[xy = ln\left(\frac{x}{y}\right)\]
\paragraph{a)} Admite una única solución $y=f(x)$ diferenciable en un intervalo que contiene a $\sqrt{e}$ y $f(\sqrt{e}) = \frac{1}{\sqrt{e}}$
\paragraph{b)} $f$ tiene un máximo local en el punto $\sqrt{e}$.
\solution
Aplicaremos el teorema de la función implícita a $g(x,y)=xy-\left(ln(x)-ln(y)\right)$.
\paragraph{a)}
Para ello:
\begin{itemize}
\item Es $C^1$
\item $g(\sqrt{e},\frac{1}{\sqrt{e}}) = 0$.
\item
\[\dpa{}{y}(g(x,y)) (\sqrt{e},\frac{1}{\sqrt{e}}) = 2\sqrt{3}\neq 0\]
\end{itemize}

\paragraph{b)}
$f'(\sqrt{e}) = 0, f''(\sqrt{e})<0$ (Ejercicio para ti)
\end{problem}


\subsection{Hoja 4}
\begin{problem} [1]
¿La unión de subvariedades diferenciables es una subvariedad diferenciable? ¿Y la intersección?
\solution
\begin{itemize}
\item Unión: NO. Tomando 2 rectas, la unión no es una subvariedad diferenciable.
\item Intersección: Tampoco: el conjunto vacío no es una subvariedad diferenciable.
\end{itemize}
\end{problem}

\begin{problem}[3]
Sean

\ppart
\[f(x) = \begin{cases}
x \sen \frac{1}{x} & x\neq 0\\
0 & x=0 
\end{cases}\]

\ppart
\[g(x) = \begin{cases}
x^2 \sen\frac{1}{x} & x\neq 0\\
0 & x=0 
\end{cases} \]

\ppart
\[h(x) =  \begin{cases}
x^4 \sen\frac{1}{x} & x\neq 0\\
0 & x=0 
\end{cases} \]

¿Son rectificables? (es decir, si tienen longitud)
\solution

\spart $f(x)$ no es rectificable. (visto en clase, en el ejemplo de curva no rectificable)

\spart 
\begin{gather*}
g'(0) = \lim_{h\to 0} \frac{g(h)-g(0)}{h} = \lim_{h\to 0} h \sen\frac{1}{h} = 0 \\
g'(x) = 2x\sen\frac{1}{x} + x^2\cdot\cos\frac{1}{x}\cdot\frac{-1}{x^2}
\end{gather*}
Nos encontramos con que $\nexists \lim_{x\to 0} g'(x)$. 

Tenemos que no es continua y no podemos aplicar la definición (\ref{defCurvaRectf}). ¿Entonces?

\textbf{Idea:} Sabemos calcular la longitud hasta un punto $\varepsilon$ más o menos cercano al $0$. Vamos a proceder a calcular la longitud entre $\varepsilon$ y $1$ y tomar $\lim_{\varepsilon \to 0}$

Llamamos a $\sigma(x) = (x,g(x))$, con $\sigma \in C^1[\varepsilon,1]$.

Por el teorema tenemos que \[L_{\varepsilon} = \int_{\varepsilon}^1 \md{\sigma'(x)}dx = \int_{\varepsilon}^1 \underbrace{\sqrt{1+2xsen\frac{1}{x} - cos\frac{1}{x}}}_{\leq \sqrt{10}}dx \leq \sqrt{10}(1-\varepsilon) \leq \sqrt{10}\]

Si $\varepsilon = \frac{1}{n}, L_n \leq \sqrt{10}, L_n \nearrow \implies \exists \lim_{n\to\inf} L_n$.

\spart Esta curva es $C^1$, por lo que es rectificable.

\end{problem}


\begin{problem}[15.b]
\[f(x,y,z) = 2x^2+y^2+z^2-xy\]
\[\mathbb{K} = \{ (x,y,z)\in \real^3 : \frac{x^2}{2} + \frac{y^2}{4} + \frac{z^2}{8} \leq 1\}\]
\solution
En el interior sale el punto crítico $(0,0,0)$ y en la forntera tendremos el sistema:

\[\left\{\begin{array}{cc}
4x-y&=\lambda x\\
2y-x &=\lambda \frac{y}{2}\\
2z &= \lambda\frac{z}{4}\\
\frac{x^2}{2}+\frac{y^2}{4} + \frac{z^2}{8} &=1
\end{array}\right\}\]
\end{problem}


\begin{problem}[18] Utilizar los multiplicadores de Lagrange para hallar una fórmula para la distancia de un punto al plano $Ax+By+Cz+D=0$.

\solution
\[g(x,y,z) = Ax+By+Cz+D \implies \nabla g = (A,B,C)\]

Definimos 
\[f(x,y,z) = d^2 = (x-a)^2 + (y-b)^2 + (z-c)^2 \implies \nabla f = (2(x-a),2(y-b),2(z-c))\]

Aplicamos el teorema de Lagrange:

\[\left\{\begin{array}{ccc}
\grad f(x,y,z) & = & \lambda \grad g\\
g(x,y,z) &= & 0
\end{array}\right\}\]

Llegando a la solución según las 3 primeras ecuaciones:
\begin{gather*}
x=\frac{\lambda A}{2} +a\\
y = \frac{\lambda B}{2} +b\\
z = \frac{\lambda C}{2} +c
\end{gather*}
Reemplazamos $x,y,z$ en la ecuación del plano (cuarta ecuación) obteniendo:

\[\lambda = \frac{-2(D+Cc+Bb+Aa)}{A^2+B^2+C^2}\]

Sustituimos este $\lambda$:

\begin{gather*}
x=a-\frac{(D+Cc+Bb+Aa)}{A^2+B^2+C^2} A \\
y = b- \frac{(D+Cc+Bb+Aa)}{A^2+B^2+C^2} B\\
z = c - \frac{(D+Cc+Bb+Aa)}{A^2+B^2+C^2} C
\end{gather*}

Y ahora reemplazamos en $d^2$:

\begin{gather*}
d^2 = \left(\frac{(D+Cc+Bb+Aa)}{A^2+B^2+C^2} A\right)^2 + \left(\frac{(D+Cc+Bb+Aa)}{A^2+B^2+C^2} B\right)^2 + \\
\qquad \left(\frac{(D+Cc+Bb+Aa)}{A^2+B^2+C^2} C\right)^2 = \\
 = \left(\frac{(D+Cc+Bb+Aa)}{A^2+B^2+C^2}\right)^2 \cdot (A^2+B^2+C^2) =\frac{(D+Cc+Bb+Aa)^2}{A^2+B^2+C^2} \\
  d=\frac{\abs{D+Cc+Bb+Aa}}{\sqrt{A^2+B^2+C^2}}
\end{gather*}
Que es justamente la fórmula de la distancia de un plano a un punto.

\end{problem}


\begin{problem}[19]
Determinar los máximos y mínimos de la función

\[ f(x,y,z) = x^4 + y^4 + z^4 \]

sobre la variedad

\[ Π = \{ (x,y,z)∈ℝ^3 \tq x+ y + z = 5 \} \]
\solution

Aplicamos el teorema de Lagrange

\[\left\{
\begin{array}{cc}
4x^3&=\lambda\\
4y^3&=\lambda\\
4z^3&=\lambda\\
x+y+z&=5
\end{array}\right\}\]

y resolvemos:

\begin{gather*}
x = y = z = \sqrt[3]{\frac{λ}{4}} \\
 3\sqrt[3]{\frac{λ}{4}} = 5 \\
 λ = \frac{5^3\cdot 4}{3^3}
 \end{gather*}
 
\end{problem}

\subsection{Hoja 5}

\begin{problem}[2]
a) \[f(x,y,z) = x+y+z;\; \sigma(t) = (\sen t,\cos t,t),\, 0\leq t\leq \pi\]
Hallar $\displaystyle \int_{\sigma}f$

\solution

$\md{\sigma'(t)} = \sqrt{2}$

Vamos con la integral:

\begin{gather*}
\int_0^{\pi} f(\sigma(t))\md{\sigma'(t)} \id t = \sqrt{2}\int_0^{\pi} \sin t + \cos t \id t = \\
= \sqrt{2}\eval{\left(-\cos t+\sen t+\frac{t^2}{2}\right)}_{0}^{\pi} = \frac{\sqrt{2}}{\pi^2+4}
\end{gather*}

\end{problem}

\begin{problem}[3]
\ppart Dibujar y calcular la longitud del arco de cicloide descrito por

\[\left\{\begin{array}{cc}
x&=R(t-\sin t)\\
y&=R(1-\cos t)
\end{array}\right\}  0\leq2\pi\]

\ppart Hallar longitud de la cardioide, dada por $r = 1+\cos\theta,\,0<\theta<\pi$

\ppart Hallar el área de la región limitada por la cardioide anterior cuando $0\leq\theta\leq2\pi$

\solution

\spart

\begin{gather*}
x'(t) = R(1-\cos t)\\
y'(t) = R\sin t
\end{gather*}

Vamos a calcular la longitud:

\[l = \int_0^{2\pi}\sqrt{R^2(1-\cos t)^2 + R^2sen^2\,t}dt = ... = R\sqrt{2}\int_0^{2\pi} \sqrt{1-\cos t}dt\]

Esta integral se resuelve multiplicando numerador y denominador por el conjugado, es decir:

\begin{gather*}
R\sqrt{2}\int_0^{2\pi} \sqrt{1-\cos t}\cdot\frac{\sqrt{1+\cos t}}{\sqrt{1+\cos t}}\id t = \\ 
 = R\sqrt{2}\int_0^{2\pi} \frac{\abs{\sin t}}{\sqrt{1+\cos t}} \id t = \dotsb = \\
 = R\sqrt{2} \left(\eval{-2\sqrt{1+\cos t}}_0^{\pi} + \eval{2\sqrt{1+\cos t}}_{\pi}^{2\pi}\right) = \dotsb = 8R
\end{gather*}

\spart Leo es demasiado rápido para mí...

\spart 
El área (razonando un poco y con otro poco de fe) sabemos que
 \[A = \frac{1}{2} \int_{\theta_2}^{\theta_1}f(\theta)^2d\theta = 
\frac{1}{2} \int_{0}^{2\pi}(1+\cos \theta)^2d\theta = ... = \pi + \frac{1}{2}\int_0^{2\pi}\frac{1+cos(2\theta)}{2} = ... \]
\end{problem}


\begin{problem}[4]

$\displaystyle\int \overrightarrow{F}\cdot \overrightarrow{s}$
\ppart $\alpha$ no se como

\ppart $\alpha_2$ no se como

\solution
\spart
La integral que resolver es:

\[
\int_{\alpha} F = \int_{0}^{1} () \cdot (1,1)dt + \int_{1}^{2} (t^2+(2-t)^2,t^2-(2-t)^2)\cdot(1,-1)dt
\]
Y da como resultado:
\[
\frac{2}{3} + 2\left.\left(4t-2t^2+\frac{t^3}{3}\right|_1^2\right)
\]

\spart
La parametrización es $\sigma(t) = (t,t^2), t\in[-2,1]$.
La integral quedaría $\displaystyle \int_{-2}^{1} F(\sigma(t))\cdot \sigma'(t)dt$

\spart
Aquí tenemos \[\begin{array}{cc}
x(t)= a\,cos(t)\\
y(t)= b\, sen(t)
\end{array}, t\in[0,2\pi]
\]

La integral quedaría $\displaystyle \int_{-2}^{1} F(\sigma(t))\cdot \sigma'(t)dt$

\end{problem}

\begin{problem}[5]
\[
\left\{\begin{array}{cc}x&=sen(t)\\y&=cos(t)\\z&=cosh(t)\end{array}\right\}, 0\leq t \leq log(7)
\]
\solution
Entonces tenemos la integral:
\[
l = -\int_0^{log(7)} \sqrt{sen^t+cos^2t+senh^2t} dt = ... = \left.senh(t)\right|_0^{log(7)}
\]
\end{problem}

\begin{problem}[6]
El campo es: \[F(x,y) = (x^2-y,x+xy+y^2)\]

\solution

La parametrización es: $\alpha(t) = (cos(t),(sen(t))$

Y la integral sería:

\[
\int_0^{\pi} F(\alpha(t))\cdot\alpha'(t) dt = ... = 2\pi
\]

Aplicando el \textbf{teorema de Green:}

\[
\int\int_D \left(\dpa{Q}{x} - \dpa{P}{y}\right) dxdy = \int\int_D  (2+y)dxdy = \int_0^{2\pi} \int_0^1 r(2+rsen(\theta))drd\theta = ... = 2\pi
\]

Se calcula mucho mejor en polares

\end{problem}

\begin{problem}[7]
noseadnsjfdnjsk
\solution
El campo no es $C^1$ en $(0,0)$.
\end{problem}

\begin{problem}[8]
\[\int_{\{Gamma} ... \]
\solution
Aplicando el teorema de Green:

\[\int \int_D 1 - 1 dx dy = 0\]
\end{problem}

\begin{problem}[Generico]
Encontrar el area de un lazo de la rosa de 4 pétalos, con $sen(2\theta)$
\solution
La curva que lo parametriza es \[\sigma(\theta) = \begin{array}{cc}
r(\theta))cos(\theta)\\
r(\theta))cos(\theta)
\end{array}
\]

Definimos el campo $F(x,y) = (0,x)$ y aplicamos el teorema de Green:

\[
\int_{\Omega}dxdy = \int_{\sigma} Fds = \int_0^{\frac{\pi}{2}} F(\sigma(\theta)) \sigma'(\theta)d\theta = \int_0^{\frac{\pi}{2}} sen(2\theta)cos(\theta) \cdot (2cos(2\theta))sen(\theta) + sen(2\theta)cos(\theta)d\theta
\]

\end{problem}

\begin{problem}[Generico 2]
Calcular el trabajo realizado por el campo $F(x,y,z) = (x^2,yz,y)$ para transportar una partícula a lo largo de la trayectoria $\sigma(t) = (e^t,t,t^2), 0\leq t \leq 1$.
\solution
Calcular la integral 
\[
\int_{\sigma} F = \int_0^1 F(\sigma(t))\sigma'(t)dt
\]
\end{problem}

\begin{problem}[El que cae mañana]
Consideremos el conjunto de todas las pirámides de base rectangular, inscritas en la esfera de radio 1, con la base paralela al plano z=0,y vértice en algún vértice.

\ppart Justificar que a igual base, la pirámide de mayor volumen debe tener el vértice situado en $(0,0,1)$.

\ppart Hallar la longitudo de las aristas que parten del vértice, en la pirámide de la clase anterior que tiene volumen distinto.

\solution
\spart
La base va a estar inscrita en una circunferencia.

Llamemos al plano en el que se encuentra la base $z=a$ y sea $V_3$ el vértice de la altura de la pirámide. Entonces tenemos que $h = \abs{V_3 + a}$.

Con lo cual, el volumen es $= \frac{1}{3} l_1\cdot l_2 \cdot \abs{v_3+a}$ (siendo $l_i$ los lados de la base). Si calculamos máximos y mínimos aquí no vamos a encontrar.

Sabemos también que $v_3 = -\sqrt{1-v_1^2-v_2^2}$
Con lo que definimos 
\[
\tilde{V} (V_1,V_2) = \frac{1}{3} l_1 \cdot l_2 \abs{\sqrt{1-v_1^2-v_2^2} + a}
\]

Derivamos:
\[
\dpa{\tilde{V}}{V_1} = \frac{1}{3}l_1l_2\frac{v_1}{\sqrt{-v_1^2-v_2^2}}
\]
\[
\dpa{\tilde{V}}{V_2} = \frac{1}{3}l_1l_2\frac{v_2}{\sqrt{-v_1^2-v_2^2}}
\]

$(V_1,V_2) = (0,0)$ y se comprueba que es un máximo con la segunda derivada. Como $v_1=v_2=0 \implies v_3 = -1$

Una posibilidad es razonar que a igualdad de base, el volumen será máximo con la altura máxima.

\spart

$V(x,y,z)$ volumen de la pirámide. 

$g(x,y,z) = x^2+y^2+z^2-1$

Definimos $\pi_1 = (x,y,z),\pi_2 = (-x,y,z),\pi_3 = (x,-y,z),\pi_4 = (-x,-y,z), V=(0,0,1)$

\[d(\pi_1,\pi_2) = l_1 = \sqrt{(2x)^2} = \abs{2x}\]
\[d(\pi_1,\pi_3) = l_2 = \sqrt{(2y)^2} = \abs{2y}\]

Volumen $ = \frac{1}{3}\abs{2x}\abs{2y}(1+z)$

Aquí se aplican multiplicadores de Lagrange, quedando el sistema:

\[
\left\{
\begin{array}{cc}
\displaystyle
\frac{2y(z+1)}{3} &= 2\lambda x\\
\frac{2}{3}x(z+1) &= 2\lambda y\\
\frac{4}{3}xy&=2\lambda z\\
x^2+y^2+z^2=1
\end{array}\right\}
\]
\end{problem}

\begin{problem}[10]

adsf
\solution
Es simple algebra lineal al parecer.
\end{problem}

\begin{problem}[11]
Sea \[s=\{(x,y,z)\in\real^3: x^2+y^=1, 0<z<1\}\]
Y 2 parametrizaciones:

\[\Phi(s,t)  = (\cos s,\sin s,t) s\in(0,2\pi),t\in(0,1)\]
\[\Psi(u,v)=(u,\sqrt{1-u^2},v),u,v,\in(0,1)\]
¿Inducen la misma orientación?
\solution
Vamos a construir una $\appl{g}{\real^2}{\real^2}$ tal que $\Phi = \Psi \circ g$

De tal forma que

\[\begin{array}{c}
\Phi(s,t) = \Psi(u,v)\circ g(s,t)\\
g(s,t) = (g_1(s,t),g_2(s,t))
\end{array}\]
Donde \[\left.\begin{array}{c}
g_1 (s,t) = \cos s\\
g_2(s,t) = t
\end{array}\right\}\implies g(s,t)= (\cos s,t)\]

Comprobamos que $\Psi(g(s,t)) = (\cos s,\underbrace{\sqrt{1-cos^2\,t}}_{\equiv \sin s},t) = \Phi(s,t)$

Las 2 parametrizaciones inducirán la misma orientación $\dimplies \det Dg > 0$. (por un resultado visto en teoría)

\[Dg = \begin{pmatrix}
-sen(s)&0\\0&1
\end{pmatrix}\]
Tenemos $\det Dg = -sen(s)$

No inducen la misma orientación ya que este determinante no es siempre positivo.

\end{problem}

\subsection{Hoja 6}

\begin{problem}[1]
Sea $\omega$ una 2-forma en $\real^3$.

\ppart
\[\omega = dy\y dz - 2 dz\y dx + 3 dx\y dy\]
Sean $\gu = (1,-1,0)$ y $\gv = (2,1,1)$.

\ppart
\[\omega = 4dx\y dz + 2dx\y dy - 3 dy\y dz + 8 dz\y dx + zdx\y dy\]

¿$\omega(\gx)[\gu,\gv]$?


\solution
\spart
Si $\omega$ es una 2-forma:

\[\omega \equiv \sum F_{iN}(x)dx_i\y dx_j\]
Entonces $\omega(\gx)[\gu,\gv] = \sum_{ij} f_{ij}(x)\det \begin{pmatrix}
u_1&v_i\\u_j&v_j
\end{pmatrix}$

En nuestro caso:

\begin{gather*}
\omega(\gx)[(1,-1,0),(2,1,1)] = 1 \left|\begin{array}{cc}
u_2&v_2\\u_3&v_3
\end{array}\right| +
2 \left|\begin{array}{cc}
u_3&v_3\\u_1&v_1
\end{array}\right|+
3 \left|\begin{array}{cc}
u_1&v_1\\u_2&v_2
\end{array}\right|\\
1 \left|\begin{array}{cc}
-1&1\\0&1
\end{array}\right|
+
2 \left|\begin{array}{cc}
0&1\\1&2
\end{array}\right|
+
3 \left|\begin{array}{cc}
1&2\\-1&1
\end{array}\right| = 10
\end{gather*}

 \spart Lo primero es redefinirla y juntar los términos:
 
 \[\omega = -4dx\y dz + (z+2)dx\y dy - 3 dy\y dz\]
 
 Y ahora ya nos ponemos a calcular con el mismo sistema que antes.
 
 Llegamos a \[\omega(\gx)[\gu,\gv] = ... = z-5\]
\end{problem}

\begin{problem}[2]

\[\omega = cx - zdy\]
\[v= (x^2+y^2+z^2)dx\y dz + (xyz)dy\y dz\]

\ppart $d\omega$
\ppart $\omega \y dw$
\ppart $dv$
\ppart $\omega \y v$

\solution

\spart $dw = d(dx-zdy) = \underbrace{d(dx)}_{\equiv 0} - \underbrace{d(zdy)}_{dz\y dy + z(dy\y dy)} = -dy\y dz$ 

\spart 

\spart 

\spart 


\end{problem}

\begin{problem}[13]
\[S=\{(x,y,z)\in\real^3: z=1-\frac{x^2+y^2}{2},x,y\in[0,1]\}\]

\solution

Buscamos calcular $\int_S xydxdy$. $xy$ Es la densidad superficial (dato del enunciado). Estamos integrando la densidad sobre la superficie.

Definimos: $\Phi(x,y) = \left(x,y,1-\frac{x^2+y^2}{2}\right)$

Calculamos $\md{T_x\x T_y}$ (el cambio en la medida), con
\[T_x = (1,0,-x);T_y=(0,1,-y)\]

\[T_x\x T_y = (x,y,1)\]

\[\int_S xy \md{T_x\x T_y} dxdy = \int_S xy\sqrt{1+x^2+y^2}dxdy = \int_0^1\int_0^1 xy\sqrt{1+x^2+y^2} dxdy = ... = 1.40\]

\end{problem}

\begin{problem}[14]
Calcular el flujo de $\overrightarrow{F}(x,y,z) = (x,y,z)$ a través del cilindro $x^2+y^2=a^2, 0\leq z \leq b$ incluyendo las bases con la normal exterior.

\solution

\textbf{Solución corta y buena}
\[\int \int_S Fds = \int \int \int_{\Omega}div F\]

Utilizando coordenadas cilíndricas:

\[\int_0^b \int_0^a\int_0^{2\pi} 3 p d\alpha dp dz = 3\pi a^2b\]
\textbf{Sin aplicar Gauss}

Vamos a separar el cálculo en la integral sobre las 2 tapas y la superficie de revolución.

El campo en la tapa de abajo es 0 (el campo con z=0 es paralelo a la base)

Nos queda: \[\int \pesc{F(\sigma(\alpha,z)),dA)}\]

Como cambia el ángulo y la altura definimos:
Siendo $\sigma(\alpha,z) =(acos\alpha,asen\alpha,z)$.

\begin{gather*}
\sigma_{\alpha} = (-asen\alpha,acos\alpha,0)\\
\sigma_{z} = (0,0,1)
\end{gather*}

Y ahora calculamos $dA$, que en $\real^3$ es el producto vectorial.

$dA = (acos\alpha,asen\alpha,0)$

Ya podemos integrar sobre la superficie lateral

\[\int_0^{2\pi} \int_0^b a^2dzd\alpha = 2\pi a^2b\]


Ahora procedemos al cálculo del flujo que atraviesa la capa de arriba.

En este caso, como cambia el ángulo y el radio tenemos:


Siendo $\sigma(\alpha,z) =(pcos\alpha,psen\alpha,b)$, y por lo tanto: $F(\sigma(\alpha,p)) = (pcos\alpha,psen\alpha,b)$

\begin{gather*}
\sigma_{\alpha} = (-psen\alpha,pcos\alpha,0)\\
\sigma_{z} = (cos\alpha,sen\alpha,0)
\end{gather*}

Si calculamos $\sigma_a\x\sigma_p = (0,0,-p)$

Problema, porque este vector apunta hacia el interior del cilindro y nos dicen respecto de la normal interior. Por ello le vamos a cambiar de signo.

\[\int_0^{2\pi} \int_0^a b\cdot p \,dpd\alpha=\pi a^2b\]

Tenemos flujo = $2\pi a^2b+\pi a^2b$
\end{problem}


\begin{problem}[9]	
$\omega = xzdy - ydx$

$v = x^3dz+dx$


\[\Phi(s,t) = (cos(s),sen(t),(t))\]

Nos piden calcular 
\ppart
$\Phi^{\ast}(\omega)$
\ppart
$\Phi^{\ast}(d\omega)$
\solution
Recordamos:
\[\pb{\sum f_Idx_I}(s)[\gv] = \sum f_I(T(s))dt_I(\gv)\]
\[\Phi^{\ast}(dx)=d\Phi_1 = -sen(s)ds\]
\spart
\[\Phi^{\ast}(\omega) = \Phi^{\ast} (xzdy - ydx) - \Phi(y)dx = cos^2(s)tds +sen^2(s)ds\]
\spart

Hay 2 posibles caminos, utilizar que el pull-back conmuta con la diferencial, teniendo que calcular la diferencial exterior del pull-back calculado anteriormente y sino:
\[dw = zdx\y dy + x\df{z,y} - \df{y,x} = (z+1)\df{x,y} + x\df{z,y} = (1) = (z+1)\df{x,y} - x\df{y,z}\]
(1) para mantener el orden cíclico.
Ahora habría que calcular el pull-back de esto.

Vamos con el otro camino:

\[d(\Phi^{\ast}\omega) (cos^2(s))dt\y ds\]
\end{problem}

\begin{problem}[17]
Sea $F = (x,y,z)$, con $\begin{array}{cc}
z = x^2+y^2\\
0\leq x,y\\
0\leq z\leq 9
\end{array}$

\solution
Hacemos un dibujo de lo que es, que es una bola de bomberman de las moradas invertida, solo lo perteneciente al primer cuadrante.
Vamos a calcular también el flujo que pasa por las tapas (la de arriba y los planos verticales que cortan al paraboloide) ¿Que no debería ser así? Yo también lo creo... 


En este caso, aplicamos el teorema de la divergencia porque vamos a calcular el flujo de un campo a través de una superficie en $\real^3$.

La divergencia $div \overrightarrow{F} = 1+1+1$

Necesitamos definir el interior de la superficie para poder aplicar el teorema. Podemos definirlo en cartesianas de la siguiente manera: \[\Omega = \{ (x,y,z): x^2+y^2\leq z\leq 9, x,y\geq 0\}\]

Si lo pasamos a coordenadas cilíncricas tenemos: \[\Omega = \{(\rho,\theta,z): 0\leq z\leq 9, 0\leq \rho \sqrt{z},0\leq\theta\leq\frac{\pi}{2}\}\]

También necesitamos el jacobiano del cambio a cilíndricas que sabemos que es $\rho$.

Aplicando esto nos queda la integral:

\[\int_S Fds =  \int_{\Omega} div F dxdydz = \int_{\Omega} 3\,dxdydz = \int_{\Omega}3\cdot\rho \,d\rho d\theta dz = 3 \int_0^{\frac{\pi}{2}} \int_0^9 \int_0^{\sqrt{z}} \rho\, d\rho d\theta dz =\]
\[
 3\frac{\pi}{2} \int_0^9 \left| \frac{1}{2} \rho^2\right|^{\sqrt{z}}_0 dz
 = 3\frac{\pi}{2} \frac{1}{2}\int_0^9 zdz = \frac{3^5\cdot \pi}{2^3}\]

\end{problem}

\begin{problem}[20]
Transformar la integral de superficie \[\int\int rot\overrightarrow{F}\,dS\]
en una integral de linea utilizando el teorema de Stokes y calcular la integral de línea en los siguientes casos:
\ppart
$\overrightarrow{F} = (y^2,xy,xz)$, siendo $S \equiv x^2+y^2+z^2=1, z\geq 0$, tomando como normal la que tenga la tercera componente no negativa.

\ppart
 \[\overrightarrow{F} = (y-z,yz,-xz)\]
 Siendo $S$ un cubo de lado [0,2]-
\solution

\spart
Tomando como frontera de la superficie $\delta S = \sigma \equiv\{ x^2+y^2=1; z=0\}$.

Parametrizamos con $\sigma(t) = (cos(t),sen(t),0)$

\[
\int_{\sigma} FdS = \int_0^{2\pi}
\pesc{ \overrightarrow{F}(\sigma(t)), \sigma'(t)} dt = \int_0^{2\pi} -sen^3(t) + cos^2(t)sen(t) dt = \int_{0-\pi}^{2\pi-\pi} -sen^3(t) + cos^2(t)sen(t) dt = 
\]
Porque es una función impar

\spart

bgdbgdfabafhlafbhahblaflbh

\end{problem}



\begin{problem}[18]
\[F(x,y,z) = \left(e^{xy},cos
,xy,\cos xz^2\right)\]
\ppart Calcular la divergencia
\ppart Calcular el rotacional
\solution

Pasando porque es fácil.

\end{problem}

\begin{problem}[19]

\ppart Ver si el campo  $ F(x,y,z) = (x+z,-(y+z),x-y)$ es conservativo
\ppart 
\ppart
\ppart 

\solution

\spart Calculamos el rotacional y vemos que efectivamente es 0.

Vamos a calcular el potencial:

\[f(x,y,z) = \int_0^x F_1(t,0,0)dt + \int_0^y F_2(x,t,0)dt + \int_0^z F_3(x,y,t)dt \]
Esto que hemos aplicado es una forma de calcular potenciales de campos conservativos. Nos lo creemos.
\[f(x,y,z) = \int_0^x t+0 \,dt + \int_0^y -(t+0)\,dt + \int_0^z (x-y)dt\]

\spart No es conservativo
\spart No es conservativo
\spart Sí es conservativo

\end{problem}

\begin{problem}[20]
Sea el campo $\vf = (y^2,xy,xz)$

Se pide la integral sobre $S= \{(x,y,z)\in \real^3 | x^2+y^2+z^2=1, z\ge 0\}$
\solution
\[\partial S = \{(x,y)\in\real^2|x^2+y^2=1\}\]

Entonces, queremos calcular:

\[\int_C y^2\,dx + xy\,dy + xz\,dz\]

Parametrizamos en cilíndricas con $z=0$

\[\int_0^{2\pi} -sen^3\,t + \sin tcos^2\,t dt = \int_0^{2\pi} \sin t(cos^2-sen^2)\,dt = \int_0^{2\pi} sen(t)\cdot cos^2(2t)dt = 0\]

Nos

\end{problem}

\begin{problem}[29]
Demostrar
\[\iiint_V f laplaciano g = \iint_S f laplaciano g - \iiint_V \grad f \grad g\]

Sugerencia: utilizar:

\[div(f\grad g) =\grad f \cdot \grad g + f laplaciano g\]
\solution

Sea $f\grad g$ un campo en $\real^3$.

Por el teorema de gauss:

\[\iiint_V dif(f\grad g) = \iint_S f\grad g\].


Aplicando la indicación:
\[\iiint_V \grad f \grad g + f laplaciano g = \iint_s f \grad g\]
Despejando de aquí ya sale.

\paragraph{Demostración de la sugerencia}

Se deja como ejercicio para el lector
\end{problem}

\begin{problem}[24]
Calcular la integral de $\vf$ sobre la esfera:
\[\vf= (x+\cos y - log(1+z^2),y+sen\sqrt{1+x^2+z^2},z)\]

\solution
Aplicamos \textbf{Gauss}:

$\div F = 3$

\[\iiint_S 3\id{x,y,z} = 3\cdot Vol(S) = 3\cdot \frac{4\pi}{3} = 4\pi\]
$Vol(S)$=Volumen de la esfera.
\end{problem}

\begin{problem}[25]
\[F=(y-z,z-x,z-y)\]
\solution
$\div \vf = 0 \implies \iint_S \vf \id{S}=0$.

Esto nos dice que en el hemisferio norte (contando la tapa de abajo) es 0.

Podemos interpretar que nos piden sin contar la tapa, con lo que calculamos el flujo sobre la tapa (que será igual que el flujo sobre la semiesfera)

Para ello parametrizamos en polares:

\begin{gather*}
x=rcos\theta\\
y=tsen\theta\\
z=0
\end{gather*}

Quedando el campo:
\[\vf(r,\theta) = (r\sen\theta,-r\cos\theta,r(\cos\theta-\sen\theta))\]

\textbf{Entonces}

\[\int_0^{2\pi}\int_0^1 r^2(\cos(\theta),-\sen(\theta))\id{r,\theta} = \frac{1}{3} \int_0^{2\pi} \cos\theta - \sen\theta \id{\theta} = 0\]

\textit{Miguel sugiere} que veamos $\vf$ como el rotacional de otro campo (que a ojo supone)
\[\vf=\rot\vg = \left(\frac{x^2+y^2+z^2}{2},\frac{x^2+y^2+z^2}{2}\frac{x^2+y^2+z^2}{2}\right)\]
\end{problem}

\begin{problem}[26]
Integrar $\vf = (x^3,y^3,-abz)$ sobre \[\{S \equiv \frac{x^2}{a^2} + \frac{y^2}{b^2} + \frac{z^2}{c^2} = 1\}\]
\solution
Vamos a aplicar \textbf{Gauss}

$\div\vf = 3(x^2+y^2)-ab$

Parametrizamos $S$ en esféricas:
\begin{gather*}
x=arcos\theta sen\phi\\
y=brsen\theta sen\phi\\
z=crcos\phi
\end{gather*}

$\div\vf = 3r^2(a^2 sen^2\theta sen^2\phi + b^2cos^2\theta sen^2\phi)-ab$

El cambio en la medida en este caso es:

\[J = \begin{matrix}
i&j&k\\
.&.&.\\
.&.&.
\end{matrix}\]

$|J| = r^2sen\phi\cdot abc$
\[\int_0^1\int_0^{\pi}\int_0^{2\pi} (r^2sen\phi\cdot abc) (3r^2sen^2\phi(a^2cos\theta + b^2sen\theta))\id{r,\theta,\phi} - \int_0^1\int_0^{\pi}\int_0^{2\pi} r^2sen\phi\cdot abcab \id{r,\theta,\phi}\]

La resolución se deja como ejercicio para el lector.
\end{problem}

\begin{problem}[27]
Integrar sobre el cubo unidad el campo $\vf = (x^2,y^2,z^2)$.
\ppart Con Gauss
\ppart Sin Gauss
\solution

\spart
\[\div\vf = 2(x+y+z)\]
\[\iint_S F\id{S} = 2\int_0^1\int_0^1\int_0^1 x+y+z\id{x,y,z} = 3\]


\spart
Tendremos que integrar sobre todas las caras del cubo.
\begin{gather*}
z=1;0\leq x\leq 1; 0\leq y\leq 1\\
z=0;0\leq x\leq 1; 0\leq y\leq 1\\
...
\end{gather*}
Y calculando los respectivos vectores normales: 
\begin{gather*}
n_1=-n_2=(0,0,1)\\
...
\end{gather*}

\[\int_S F\cdot n \id{S} = \int_{S_1} F_3ds + ... = 1-0+1-0+1-0=3\]
cqc.

\end{problem}
