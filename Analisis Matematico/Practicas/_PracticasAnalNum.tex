\section{Ejercicios}
\subsection{Hoja 1}

\begin{problem}[?]
\solution
$\overline{A} = { x \in \real^N; \forall V_x \tq V_x \cup A \neq \text{\O}}$, siendo $V_x$ un entorno abierto de x.
	$\overline{A} = A \cap $ 
	
\begin{theorem}
$A \subset \real^N$ es cerrado $\dimplies \acum{A}\subset A$ 
\end{theorem}

 
\end{problem}
\begin{proof}
\begin{gather*}
A \text{es cerrado} \implies A^c \text{ es abierto} \implies\\
\forall x \in A^c, \exists \varepsilon > 0 \tq B(x,\varepsilon) \subset A^c \implies\\
A \cap B(x,\varepsilon) = \text{\O} \implies x \nexists \acum(A) 
\end{gather*}
\end{proof}
Falta la recíproca.


\begin{problem}[3] 
\solution
a)

$\displaystyle\bigcup_{k=1}^{\infty} \left[-1,\frac{1}{k}\right)$

Es cerrado, porque $=[-1,0]$
Demostración: (hay que demostrar las inclusiones $\subseteq$ y $\supseteq$)

b)
No es ni cerrado ni abierto.
\obs $\real$ es el cierre de $\mathbb{Q}$.

c)
 
\end{problem}

\newpage
\subsection{Hoja 2}

\begin{problem}[?]
\solution
$f(x,y) = \left\{\begin{matrix}
                \displaystyle \frac{x^3}{x^2+7y^2} & \text{ si } (x,y) \neq (0,0)\\
                 0 & \text{ si } (x,y) = (0,0)
                \end{matrix}\right.$
Continuidad:

$$0\leq \left| \frac{x^3}{x^2+7y^2} \right| \leq \left| \frac{x^3}{x^2} \right| \leq |x| \rightarrow 0 \implies$$ Continua en 0.

Derivadas parciales en $(0,0)$

$$\mylim{h}{0} {\frac{f(h,0) - f(0,0)}{h}} = \mylim{h}{0} {\displaystyle\frac{{h^{3}}/{h^{2}}}{h}} = 1$$
$$\mylim{h}{0} {\frac{f(0,h) - f(0,0)}{h}} = \mylim{h}{0} {\frac{0}{0+7h^2}} = 0$$
\end{problem}
 
\subsection{5}

\paragraph{b)}
$C_{D_4}(b)$. Basta con comprobar la conmutación con $a^j$ y con $a^jb$ siendo $j = 0,1,2,3$, ya que con eso podemos ver la conmutación con todos los elementos. Se puede demostrar la conmutatividad multiplicando a derecha e iezquierda por $b$ y $b^{-1}$ y si nos queda $=1$, es conmutativo.

$$\left\{\begin{matrix}b(a^j)b^{-1} = a^{-j}, a^j \in C_{D_4}(b) \dimplies a^2j = 1\\
b(a^jb)b^{-1} = a^{-j} = a^{-j}b, a^jb \in C_{D_4}(b) \dimplies a^2j = 1\end{matrix}\right.$$

\begin{problem}[9]
\solution
\paragraph{a)}
\paragraph{b)}

$$f(x,y) = \int_a^xy g(s)ds$$
Aplicando el teorema fundamental del cálculo $\left(f \text{ continua } \implies\displaystyle\int_a^b f(x)dx = F(b)-F(a)\right)$
$$\dpa{f}{x} = g(xy)\underbrace{\dpa{xy}{x}}_{=y} - \underbrace{g(a)\dpa{a}{x}}_{=0}  = g(xy)y$$
$$\dpa{f}{y} = \cdots  = g(xy)x$$
\end{problem}

\begin{problem}[Ejercicio de examen]
$\appl{g}{\real}{\real}$ continua, con $g(1) = 4$.

Sea $f(x,y,z)=\displaystyle \int_0^{x^2ye^z} g(t)dt$.
Demostrar que $f$ es diferenciable y calcular $\nabla f(1,1,0)$.

\solution

Ejercicio para el lector.

\end{problem}
\newpage
\subsection{Hoja 3}
\begin{problem}[3]
\paragraph{a)} Probar que si la derivada de $\appl{f}{\real}{\real}$ existe y no se anula entonces $f$ es inyectiva.
\paragraph{b)} Probar que $\appl{f}{\real^2}{\real^2}$ dada por $f(x,y) =( e^xcos(y) + 2e^xsen(y),-e^xcos(y))$ cumple que el determinante de su Jacobiano es siempre positivo pero sin embargo no tiene $f$ no es inyectiva.

\solution

a)

% \begin{itemize}
%  \item $f$ derivable $\implies f$ continua.
%  \item Si $f'(x) \neq 0, \forall x \in \real \implies f $ monótona (estrictamente creciente o decreciente). 
%  \end{itemize}
Me parece demasiado intuitivo...

b)
$$J = \begin{pmatrix}
       e^xcos(y)+2e^xsen(y) & e^xcos(y)+2e^xsen(y) \\
       -e^xcos(y) & e^xsen(y)
      \end{pmatrix}
$$

Calculamos $$det(J) = (e^xcos(y)+2e^xsen(y))+e^xsen(y) + e^xcos(y)sen(y)(-e^xsen(y)+2e^xcos(y)) = $$
$$ = ... = 2e^x > 0 \forall x \in \real$$

Aunque el jacobiano sea siempre positivo, $f$ no es inyectiva porque si tomamos $f(0,0) = (1,-1) = f(0,2\pi)$.
\end{problem}
\begin{problem}[inventado]
\label{inventado}
Sea $F(x,y) = (x^2-y^2,2xy)$. Encontrar los puntos en los que la siguiente aplicación es localmente inversible de clase $C^1$.
\solution
\begin{itemize}
 \item 1) $F \in C^1$ por ser $F_1,F_2$ polinomios.
 \item 2)$det(J)>0 \forall (x,y)\in \real^2$. 
 
 En este caso: $$det\begin{pmatrix}
                  2x&-2y\\
                  2x&2y
                 \end{pmatrix} = 4x^2 + 4y^2 = 0 \dimplies (x,y) = (0,0)$$           
 \item 3) Por el teorema de la funcion inversa, existe una inversa local de $F,C^1$ en todo entorno de $(x,y) \in \real^2$ con $(x,y)\neq (0,0)$. 
 
 Está la posibilidad de que exista la función inversa, pero no podemos deducir nada del teorema. Para verlo, recurrimos a la definición de inyectividad, y en este caso, no es inyectiva porque es una función par.
 \end{itemize}
 \end{problem}
 \begin{problem}[5]
 a)
 $f\in C^1(\real), f'\neq0$.
 No tiene sentido...
 $$\left\{\begin{matrix} u(x,y) =f(x)\\v(x,y) = -y + f(x)\end{matrix}\right.$$
 Probar que tiene inversa global.
 
 b) Si $f (0) = 0$ y $f' (0) = 1$, hallar las derivadas parciales de dicha inversa en el origen.

\solution

 Mismos pasos que en el ejercicio anterior:
 \begin{itemize}
 \item 1) $F \in C^1$ por ser $F_1,F_2$ , porque $f\in C^1$.
 \item 2)$det(J)>0 \forall (x,y)\in \real^2$.
 
 $$det(J) = det\begin{pmatrix} f'(x)&0\\f(x)+xf'(x)&-1\end{pmatrix} = -f'(x) \neq 0\text{ por hipótesis}$$
 \end{itemize}
 \paragraph{b)}
 Como nos piden calcular las derivadas parciales de la función inversa. (La inversa de la matriz jacobiana, es la jacobiana de la matriz inversa)
 $$J(0,0) = \begin{pmatrix}f'(0) & 0 \\f(0) & -1\end{pmatrix}$$
 Lo que buscamos en la matriz inversa, que en este caso es ella misma.
 
 El teorema solo nos demuestra la existencia de la inversa local (contraejemplo:(\ref{inventado})). Hay que ver la inyectividad para hablar de inversa global.
 
\begin{gather*}
F(x,y) = (u(x,y),v(x,y))\\
\text{Condición: }F(x_1,y_1) = F(x_2,y_2) \implies x_1=x_2, y_1=y_2\\
u(x_1,y_1) = u(x_2,y_2) \implies f(x_1) = f(x_2)\\
f' \text{ no se anula } \implies \text{f es inyectiva} \implies x_1=x_2\\
v(x_1,y_1) = v(x_2,y_2) -y_1 + x_1f(x_1) = -y_2 + x_2f(x_2) \underbrace{\implies}_{x_1=x_2}\\
y_1=y_2
\end{gather*}
Hemos demostrado que $F$ es inyectiva y por lo tanto admite inversa global.
 
\end{problem}
\begin{problem}[6]
Estudiar si se puede despejar $(x,y,z)$ en términos de $(u,v,w)$ 
$$F(x,y,z) = \left\{\begin{matrix}u = 2x+2x^2y+2x^2z+2xy^2+2xyz\\v=x+y+2xy+2x^2\\w=4x+y+z+3y^2+3z^2+6yz\end{matrix}\right.$$
\solution
\begin{itemize}
 \item $u,v,w \in C^1$ por se suma de polinomios. 
 \item \begin{gather*}
\dpa{u}{x} = ... \implies \dpa{u}{x}(0,0) = 2\\
\dpa{u}{y} = ... \implies \dpa{u}{y}(0,0) = 0\\
\dpa{u}{z} = ... \implies \dpa{u}{z}(0,0) = 0\\
\dpa{v}{x} = ... \implies \dpa{v}{x}(0,0) = 1\\
\dpa{v}{y} = ... \implies \dpa{v}{y}(0,0) = 0\\
\dpa{v}{z} = ... \implies \dpa{v}{z}(0,0) = 0\\
\dpa{w}{x} = ... \implies \dpa{w}{x}(0,0) = 4\\
\dpa{w}{y} = ... \implies \dpa{x}{y}(0,0) = 1\\
\dpa{w}{z} = ... \implies \dpa{w}{z}(0,0) = 1
       \end{gather*}
       
   $det(J) =\begin{pmatrix}
             2&0&0\\
             1&1&0\\
             4&1&1
            \end{pmatrix}
 = 2 \neq 0 \implies \exists $ inversa local de clase $C^1$ en un entonrno de cualquier punto, en concreto en un entorno del origen.
\end{itemize}
\end{problem}

\begin{problem}[8]
\solution
\paragraph{a)}

$$det(J) = det\begin{pmatrix}
       cos(\varphi)&-rsen(\varphi)&0\\
       sen(\varphi)&rcos(\varphi)&0\\
       0&0&1
      \end{pmatrix} = rcos^2(\varphi) + rsen^2(\varphi) = r$$
      
      Por tanto, por el teorema de la función inversa, existe una inversa de clase $C^1, \forall (r,h,\varphi) \dimplies r\neq 0$.
\end{problem}
 \begin{problem}[9]
 \solution
 \paragraph{b: Calcular la inversa en (2,-2$\sqrt{3}$)}
 
 Resolver: $$\left\{\begin{matrix} 2 = rcos(\varphi)\\-2\sqrt{3} = rsen(\varphi)\end{matrix}\right.$$
 
 Hay que hallar la inversa de: $$\begin{pmatrix}
                                  \frac{1}{2}&2\sqrt{3}\\
                                  \frac{-\sqrt{3}}{2}&2
                                 \end{pmatrix}$$

    \end{problem}                             
  \begin{problem}[13]
  \solution
  
  $$\det(J) = \det \begin{pmatrix}
              \dpa{f_1}{x}&\dpa{f_1}{y}\\
              \dpa{f_2}{x}&\dpa{f_2}{y}
             \end{pmatrix} = 
             \det \begin{pmatrix}
              \dpa{f_1}{x}&-\dpa{f_2}{x}\\
              \dpa{f_2}{x}& \dpa{f_1}{x}
             \end{pmatrix}
      = \left(\dpa{f_1}{x}\right)^2 + \left(\dpa{f_2}{x}\right)^2 \implies \left(\dpa{f_1}{x},\dpa{f_2}{x}\right)$$
Esto es aplicando la primera ecuación de Cauchy-Riemman. Obteniendo una condición

Aplicando la otra condición en el jacobiano llegamos a $\displaystyle\left(\dpa{f_1}{y},\dpa{f_2}{y}\right)\neq (0,0)$
\paragraph{c)}

Queremos ver que $g(x,y) = (f_1(x,y)^2-f_2(x,y)^2,2f_1(x,y)f_2(x,y))$ cumple las ecuaciones de Cauchy-Riemman. Facilito.
\end{problem}


 \begin{problem}[14]
 \solution
 Visto en teoría
 \end{problem}
 
\begin{problem}[15]
\[sen(xz)+sen(yz)+sen(xy) = 0\]
a) Admite una única solución $z=f(x,y) C^1$ en un entorno del punto $(\pi,0)$ con $f(\pi,0) = 1$

b) Polinomio de Taylor de orden 1

\solution
\paragraph{a)}
Sea $F(x,y,z) = sen(xz)+sen(yz)+sen(xy)$.
\begin{itemize}
\item $F \in C^{\infty}$ por ser todo senos.
\item $F(\pi,0,1) = 0$
\item $z = f(x,y)$
\[\dpa{F}{z} = cos(yz)y + cos(xz)x\]
\[\dpa{F}{z}(\pi,0,1) = -\pi \neq 0\]
Por el teorema de la función implícita, $\exists f \in C^1(\real^2) \tlq z = f(x,y)$ en un entorno de $(\pi,0,1)$.
\end{itemize}
\paragraph{b)}
\[T_1f(\pi,0) =  f(\pi,0) + \dpa{f}{x}(\pi,0)(x-\pi) + \dpa{f}{y}(\pi,0)(x-\pi)\]

Calculamos la derivada respecto de $x$:
\[\dpa{f}{x} : cos(xy)y\dpa{z}{x}(\pi,0) + cos(xz)\left(z+\dpa{z}{x}(\pi,0)\right) + cos(xy)y = 0\]
Evaluando en $(\pi,0,1)$
\[= -\left(1+\dpa{z}{x}(\pi,0)\right) = 0 \equiv \dpa{z}{x}(\pi,0) = -1\]

Hay algún error, porque
$ \dpa{z}{x}(\pi,0) = \frac{-1}{\pi}$

Ahora vamos a calcular la derivada respecto de $y$

\[\dpa{z}{y}(\pi,0): cos(yz)\left(z+y\dpa{z}{y}\right) + cos(xz)x\dpa{z}{y} + cos(xy)x = 0\]
Evaluando en $(\pi,0,1)$
\[1-\pi\dpa{z}{y}(\pi,0) + \pi = 0\]
\[\implies \dpa{z}{y}(\pi,0)= \frac{1}{\pi} + 1\]

Ya tenemos todos los valores para calcular el polinomio de newton.
\end{problem}
 
  \begin{problem}[16]
 \solution
 Visto en teoría
 \end{problem}
 
 \begin{problem}[18]
 \solution
 Visto en teoría
 \end{problem}
\begin{problem}[19]
Dado el sistema:

\[\left\{\begin{array}{cccc}
	 f_1 &\equiv& x^2 -ycos(\mu v) + z^2 &= 1\\
	f_2 &\equiv &x^2+y^2-sen(\mu)cos(v)+2z^2 &= 4\\
	f_3 &\equiv &xy-sen(u)cos(v) + z &= 1 \end{array} \right.\]
Demostrar que se puede definir $(x,y,z)$ en función de $u,v$ en un entorno del punto $(1,1,1\frac{\pi}{2},0) $.

\paragraph{b)} Hallar también $DF(\frac{\pi}{2},0) $

\solution
	
Definimos $F = (f_1,f_2,f_3), \appl{F}{\real^5}{\real^3}$

\begin{itemize}
\item $F \in C^1$
\item $F(1,1,1,\frac{\pi}{2},0)$ satisface la ecuación
\item Vamos a estudiar el determinante:

\[\det J(1,1,1,\frac{\pi}{2},0) = 
	\begin{pmatrix}
		\dpa{f_1}{x} &\dots &\dpa{f_1}{z}\\
		\vdots & \ddots &\vdots\\
		\dpa{f_3}{x} & \dots & \dpa{f_3}{z}
	\end{pmatrix} =
	\begin{pmatrix}
		2&-1&2\\
		2&2&4\\
		1&1&1
	\end{pmatrix} = -6 \neq 0\]	
\end{itemize}

Nos piden también $DF(\frac{\pi}{2},0)$. Derivamos implícitamente $f_i$ respecto de u:

\begin{itemize}
\item $f_1$
\[2x\dpa{x}{u} - \dpa{y}{u}cos(uv)+ysen(uv)v + 2z\dpa{z}{u} =0\]
Evaluamos en el punto $(1,1,1,\frac{\pi}{2})$.
\[2\dpa{x}{u}(\frac{\pi}{2},0) - \dpa{y}{u}(\frac{\pi}{2},0) + 2 \dpa{z}{u}(\frac{\pi}{2},0) = 0\]
 \item $f_2$
 \[2x\dpa{x}{u} + 2y\dpa{y}{u}-cos(uv)v+4z\dpa{z}{u} = 0\]
 Evaluamos en el punto:
 \[2\dpa{x}{u} (\frac{\pi}{2},0)  + 2\dpa{y}{u}(\frac{\pi}{2},0)  - 4 \dpa{z}{u}(\frac{\pi}{2},0) = 0 \]
 \item $f_3$
 ...
 Evaluamos en el punto
 \[\dpa{x}{u}(\frac{\pi}{2},0) +\dpa{y}{u}(\frac{\pi}{2},0)  + \dpa{z}{u}(\frac{\pi}{2},0) = 0 \]
\end{itemize}

Ahora tenemos que resolver el sistema:
(Ejercicio para el lector). 

Queda $D_uF(\frac{\pi}{2},0)  = (0,0,0)$.

Repetimos el mismo proceso respecto de v (Ejercicio también para ti)

Obtenemos $D_vF(\frac{\pi}{2},0)  = (0,0,0)$

\textbf{Conclusión:} $DF(\frac{\pi}{2}) = 0_{3\times 2}$
\end{problem}

 \begin{problem}[20]
 \solution
 Visto en teoría
 \end{problem}
 
  \begin{problem}[21]
 \solution
 Visto en teoría
 \end{problem}
 
 
\begin{problem}[24]
\[xy = ln\left(\frac{x}{y}\right)\]
\paragraph{a)} Admite una única solución $y=f(x)$ diferenciable en un intervalo que contiene a $\sqrt{e}$ y $f(\sqrt{e}) = \frac{1}{\sqrt{e}}$
\paragraph{b)} $f$ tiene un máximo local en el punto $\sqrt{e}$.
\solution
Aplicaremos el teorema de la función implícita a $g(x,y)=xy-\left(ln(x)-ln(y)\right)$.
\paragraph{a)}
Para ello:
\begin{itemize}
\item Es $C^1$
\item $g(\sqrt{e},\frac{1}{\sqrt{e}}) = 0$.
\item
\[\dpa{}{y}(g(x,y)) (\sqrt{e},\frac{1}{\sqrt{e}}) = 2\sqrt{3}\neq 0\]
\end{itemize}

\paragraph{b)}
$f'(\sqrt{e}) = 0, f''(\sqrt{e})<0$ (Ejercicio para ti)
\end{problem}


\subsection{Hoja 4}
\begin{problem} [1]
¿La unión de subvariedades diferenciables es una subvariedad diferenciable? ¿Y la intersección?
\solution
\begin{itemize}
\item Unión: NO. Tomando 2 rectas, la unión no es una subvariedad diferenciable.
\item Intersección: Tampoco: el conjunto vacío no es una subvariedad diferencial.
\end{itemize}
\end{problem}

%\begin{problem}[2]
%\[M=\{ (x,y,z)\in \real^3 \tq \left\{\begin{array}{c}
%xy &= 0\\
%x^2+y^2+z^2 &= 1\\
%z&\neq 0, \pm 1
%\end{array}\right.\]
%\solution

%Dibujamos el conjunto

%\end{problem}