\section{Ejercicios}

\subsection{Hoja 1}

\begin{problem}[?]
asdfa
\solution
$\overline{A} = { x \in \real^N; \forall V_x \tq V_x \cup A \neq \text{\O}}$, siendo $V_x$ un entorno abierto de x.
	$\overline{A} = A \cap $ 
	
$A \subset \real^N$ es cerrado $\dimplies \acum{A}\subset A$ 
 
\end{problem}
\begin{proof}
\begin{gather*}
A \text{es cerrado} \implies A^c \text{ es abierto} \implies\\
\forall x \in A^c, \exists \varepsilon > 0 \tq B(x,\varepsilon) \subset A^c \implies\\
A \cap B(x,\varepsilon) = \text{\O} \implies x \nexists \acum(A) 
\end{gather*}
\end{proof}
Falta la recíproca.


\begin{problem}[3] 
\solution
a)

$\displaystyle\bigcup_{k=1}^{\infty} \left[-1,\frac{1}{k}\right)$

Es cerrado, porque $=[-1,0]$
Demostración: (hay que demostrar las inclusiones $\subseteq$ y $\supseteq$)

b)
No es ni cerrado ni abierto.
\obs $\real$ es el cierre de $\mathbb{Q}$.

c)
 
\end{problem}

\newpage
\subsection{Hoja 2}

\begin{problem}[?]
\solution
$f(x,y) = \left\{\begin{matrix}
                \displaystyle \frac{x^3}{x^2+7y^2} & \text{ si } (x,y) \neq (0,0)\\
                 0 & \text{ si } (x,y) = (0,0)
                \end{matrix}\right.$
Continuidad:

$$0\leq \left| \frac{x^3}{x^2+7y^2} \right| \leq \left| \frac{x^3}{x^2} \right| \leq |x| \rightarrow 0 \implies$$ Continua en 0.

Derivadas parciales en $(0,0)$
\begin{gather*}
\lim_{h \to 0} {\frac{f(h,0) - f(0,0)}{h}} = \mylim{h}{0} {\displaystyle\frac{{h^{3}}/{h^{2}}}{h}} = 1\\
\lim_{h\to 0} {\frac{f(0,h) - f(0,0)}{h}} = \mylim{h}{0} {\frac{0}{0+7h^2}} = 0
\end{gather*}
\end{problem}
 
\begin{problem}[5]
a saber
\solution

\paragraph{b)}
$C_{D_4}(b)$. Basta con comprobar la conmutación con $a^j$ y con $a^jb$ siendo $j = 0,1,2,3$, ya que con eso podemos ver la conmutación con todos los elementos. Se puede demostrar la conmutatividad multiplicando a derecha e iezquierda por $b$ y $b^{-1}$ y si nos queda $=1$, es conmutativo.

$$\left\{\begin{matrix}b(a^j)b^{-1} = a^{-j}, a^j \in C_{D_4}(b) \dimplies a^2j = 1\\
b(a^jb)b^{-1} = a^{-j} = a^{-j}b, a^jb \in C_{D_4}(b) \dimplies a^2j = 1\end{matrix}\right.$$
\end{problem}

\begin{problem}[9]
\solution
\paragraph{a)}
\paragraph{b)}

$$f(x,y) = \int_a^xy g(s)ds$$
Aplicando el teorema fundamental del cálculo \[\left(f \text{ continua } \implies\displaystyle\int_a^b f(x)dx = F(b)-F(a)\right)\] obtenemos:
$$\dpa{f}{x} = g(xy)\underbrace{\dpa{xy}{x}}_{=y} - \underbrace{g(a)\dpa{a}{x}}_{=0}  = g(xy)y$$
$$\dpa{f}{y} = \cdots  = g(xy)x$$
\end{problem}

\begin{problem}[Ejercicio de examen]
$\appl{g}{\real}{\real}$ continua, con $g(1) = 4$.

Sea $f(x,y,z)=\displaystyle \int_0^{x^2ye^z} g(t)dt$.
Demostrar que $f$ es diferenciable y calcular $\nabla f(1,1,0)$.

\solution

Ejercicio para el lector.

\end{problem}
\newpage
\subsection{Hoja 3}
\begin{problem}[3]
\paragraph{a)} Probar que si la derivada de $\appl{f}{\real}{\real}$ existe y no se anula entonces $f$ es inyectiva.
\paragraph{b)} Probar que $\appl{f}{\real^2}{\real^2}$ dada por $f(x,y) =( e^xcos(y) + 2e^xsen(y),-e^xcos(y))$ cumple que el determinante de su Jacobiano es siempre positivo pero sin embargo no tiene $f$ no es inyectiva.

\solution

a)
Me parece demasiado intuitivo...

b)
$$J = \begin{pmatrix}
       e^xcos(y)+2e^xsen(y) & e^xcos(y)+2e^xsen(y) \\
       -e^xcos(y) & e^xsen(y)
      \end{pmatrix}
$$

Calculamos $$det(J) = (e^xcos(y)+2e^xsen(y))+e^xsen(y) + e^xcos(y)sen(y)(-e^xsen(y)+2e^xcos(y)) = $$
$$ = ... = 2e^x > 0 \forall x \in \real$$

Aunque el jacobiano sea siempre positivo, $f$ no es inyectiva porque si tomamos $f(0,0) = (1,-1) = f(0,2\pi)$.
\end{problem}
\begin{problem}[inventado]
\label{inventado}
Sea $F(x,y) = (x^2-y^2,2xy)$. Encontrar los puntos en los que la siguiente aplicación es localmente inversible de clase $C^1$.
\solution
\begin{itemize}
 \item 1) $F \in C^1$ por ser $F_1,F_2$ polinomios.
 \item 2)$det(J)>0 \forall (x,y)\in \real^2$. 
 
 En este caso: $$det\begin{pmatrix}
                  2x&-2y\\
                  2x&2y
                 \end{pmatrix} = 4x^2 + 4y^2 = 0 \dimplies (x,y) = (0,0)$$           
 \item 3) Por el teorema de la funcion inversa, existe una inversa local de $F,C^1$ en todo entorno de $(x,y) \in \real^2$ con $(x,y)\neq (0,0)$. 
 
 Está la posibilidad de que exista la función inversa, pero no podemos deducir nada del teorema. Para verlo, recurrimos a la definición de inyectividad, y en este caso, no es inyectiva porque es una función par.
 \end{itemize}
 \end{problem}
 \begin{problem}[5]
 a)
 $f\in C^1(\real), f'\neq0$.
 No tiene sentido...
 $$\left\{\begin{matrix} u(x,y) =f(x)\\v(x,y) = -y + f(x)\end{matrix}\right.$$
 Probar que tiene inversa global.
 
 b) Si $f (0) = 0$ y $f' (0) = 1$, hallar las derivadas parciales de dicha inversa en el origen.

\solution

 Mismos pasos que en el ejercicio anterior:
 \begin{itemize}
 \item 1) $F \in C^1$ por ser $F_1,F_2$ , porque $f\in C^1$.
 \item 2)$det(J)>0 \forall (x,y)\in \real^2$.
 
 $$det(J) = det\begin{pmatrix} f'(x)&0\\f(x)+xf'(x)&-1\end{pmatrix} = -f'(x) \neq 0\text{ por hipótesis}$$
 \end{itemize}
 \paragraph{b)}
 Como nos piden calcular las derivadas parciales de la función inversa. (La inversa de la matriz jacobiana, es la jacobiana de la matriz inversa)
 $$J(0,0) = \begin{pmatrix}f'(0) & 0 \\f(0) & -1\end{pmatrix}$$
 Lo que buscamos en la matriz inversa, que en este caso es ella misma.
 
 El teorema solo nos demuestra la existencia de la inversa local (contraejemplo:(\ref{inventado})). Hay que ver la inyectividad para hablar de inversa global.
 
\begin{gather*}
F(x,y) = (u(x,y),v(x,y))\\
\text{Condición: }F(x_1,y_1) = F(x_2,y_2) \implies x_1=x_2, y_1=y_2\\
u(x_1,y_1) = u(x_2,y_2) \implies f(x_1) = f(x_2)\\
f' \text{ no se anula } \implies \text{f es inyectiva} \implies x_1=x_2\\
v(x_1,y_1) = v(x_2,y_2) -y_1 + x_1f(x_1) = -y_2 + x_2f(x_2) \underbrace{\implies}_{x_1=x_2}\\
y_1=y_2
\end{gather*}
Hemos demostrado que $F$ es inyectiva y por lo tanto admite inversa global.
 
\end{problem}
\begin{problem}[6]
Estudiar si se puede despejar $(x,y,z)$ en términos de $(u,v,w)$ 
$$F(x,y,z) = \left\{\begin{matrix}u = 2x+2x^2y+2x^2z+2xy^2+2xyz\\v=x+y+2xy+2x^2\\w=4x+y+z+3y^2+3z^2+6yz\end{matrix}\right.$$
\solution
\begin{itemize}
 \item $u,v,w \in C^1$ por se suma de polinomios. 
 \item \begin{gather*}
\dpa{u}{x} = ... \implies \dpa{u}{x}(0,0) = 2\\
\dpa{u}{y} = ... \implies \dpa{u}{y}(0,0) = 0\\
\dpa{u}{z} = ... \implies \dpa{u}{z}(0,0) = 0\\
\dpa{v}{x} = ... \implies \dpa{v}{x}(0,0) = 1\\
\dpa{v}{y} = ... \implies \dpa{v}{y}(0,0) = 0\\
\dpa{v}{z} = ... \implies \dpa{v}{z}(0,0) = 0\\
\dpa{w}{x} = ... \implies \dpa{w}{x}(0,0) = 4\\
\dpa{w}{y} = ... \implies \dpa{x}{y}(0,0) = 1\\
\dpa{w}{z} = ... \implies \dpa{w}{z}(0,0) = 1
       \end{gather*}
       
   $det(J) =\begin{pmatrix}
             2&0&0\\
             1&1&0\\
             4&1&1
            \end{pmatrix}
 = 2 \neq 0 \implies \exists $ inversa local de clase $C^1$ en un entonrno de cualquier punto, en concreto en un entorno del origen.
\end{itemize}
\end{problem}

\begin{problem}[8]
\solution
\paragraph{a)}

$$det(J) = det\begin{pmatrix}
       cos(\varphi)&-rsen(\varphi)&0\\
       sen(\varphi)&rcos(\varphi)&0\\
       0&0&1
      \end{pmatrix} = rcos^2(\varphi) + rsen^2(\varphi) = r$$
      
      Por tanto, por el teorema de la función inversa, existe una inversa de clase $C^1, \forall (r,h,\varphi) \dimplies r\neq 0$.
\end{problem}
 \begin{problem}[9]
 \solution
 \paragraph{b: Calcular la inversa en (2,-2$\sqrt{3}$)}
 
 Resolver: $$\left\{\begin{matrix} 2 = rcos(\varphi)\\-2\sqrt{3} = rsen(\varphi)\end{matrix}\right.$$
 
 Hay que hallar la inversa de: $$\begin{pmatrix}
                                  \frac{1}{2}&2\sqrt{3}\\
                                  \frac{-\sqrt{3}}{2}&2
                                 \end{pmatrix}$$

    \end{problem}                             
  \begin{problem}[13]
  \solution
  
  $$\det(J) = \det \begin{pmatrix}
              \dpa{f_1}{x}&\dpa{f_1}{y}\\
              \dpa{f_2}{x}&\dpa{f_2}{y}
             \end{pmatrix} = 
             \det \begin{pmatrix}
              \dpa{f_1}{x}&-\dpa{f_2}{x}\\
              \dpa{f_2}{x}& \dpa{f_1}{x}
             \end{pmatrix}
      = \left(\dpa{f_1}{x}\right)^2 + \left(\dpa{f_2}{x}\right)^2 \implies \left(\dpa{f_1}{x},\dpa{f_2}{x}\right)$$
Esto es aplicando la primera ecuación de Cauchy-Riemman. Obteniendo una condición

Aplicando la otra condición en el jacobiano llegamos a $\displaystyle\left(\dpa{f_1}{y},\dpa{f_2}{y}\right)\neq (0,0)$
\paragraph{c)}

Queremos ver que $g(x,y) = (f_1(x,y)^2-f_2(x,y)^2,2f_1(x,y)f_2(x,y))$ cumple las ecuaciones de Cauchy-Riemman. Facilito.
\end{problem}


 \begin{problem}[14]
 \solution
 Visto en teoría
 \end{problem}
 
\begin{problem}[15]
\[sen(xz)+sen(yz)+sen(xy) = 0\]
a) Admite una única solución $z=f(x,y) C^1$ en un entorno del punto $(\pi,0)$ con $f(\pi,0) = 1$

b) Polinomio de Taylor de orden 1

\solution
\paragraph{a)}
Sea $F(x,y,z) = sen(xz)+sen(yz)+sen(xy)$.
\begin{itemize}
\item $F \in C^{\infty}$ por ser todo senos.
\item $F(\pi,0,1) = 0$
\item $z = f(x,y)$
\[\dpa{F}{z} = cos(yz)y + cos(xz)x\]
\[\dpa{F}{z}(\pi,0,1) = -\pi \neq 0\]
Por el teorema de la función implícita, $\exists f \in C^1(\real^2) \tlq z = f(x,y)$ en un entorno de $(\pi,0,1)$.
\end{itemize}
\paragraph{b)}
\[T_1f(\pi,0) =  f(\pi,0) + \dpa{f}{x}(\pi,0)(x-\pi) + \dpa{f}{y}(\pi,0)(x-\pi)\]

Calculamos la derivada respecto de $x$:
\[\dpa{f}{x} : cos(xy)y\dpa{z}{x}(\pi,0) + cos(xz)\left(z+\dpa{z}{x}(\pi,0)\right) + cos(xy)y = 0\]
Evaluando en $(\pi,0,1)$
\[= -\left(1+\dpa{z}{x}(\pi,0)\right) = 0 \equiv \dpa{z}{x}(\pi,0) = -1\]

Hay algún error, porque
$ \dpa{z}{x}(\pi,0) = \frac{-1}{\pi}$

Ahora vamos a calcular la derivada respecto de $y$

\[\dpa{z}{y}(\pi,0): cos(yz)\left(z+y\dpa{z}{y}\right) + cos(xz)x\dpa{z}{y} + cos(xy)x = 0\]
Evaluando en $(\pi,0,1)$
\[1-\pi\dpa{z}{y}(\pi,0) + \pi = 0\]
\[\implies \dpa{z}{y}(\pi,0)= \frac{1}{\pi} + 1\]

Ya tenemos todos los valores para calcular el polinomio de newton.
\end{problem}
 
  \begin{problem}[16]
 \solution
 Visto en teoría
 \end{problem}
 
 \begin{problem}[18]
 \solution
 Visto en teoría
 \end{problem}
\begin{problem}[19]
Dado el sistema:

\[\left\{\begin{array}{cccc}
	 f_1 &\equiv& x^2 -ycos(\mu v) + z^2 &= 1\\
	f_2 &\equiv &x^2+y^2-sen(\mu)cos(v)+2z^2 &= 4\\
	f_3 &\equiv &xy-sen(u)cos(v) + z &= 1 \end{array} \right.\]
Demostrar que se puede definir $(x,y,z)$ en función de $u,v$ en un entorno del punto $(1,1,1\frac{\pi}{2},0) $.

\paragraph{b)} Hallar también $DF(\frac{\pi}{2},0) $

\solution
	
Definimos $F = (f_1,f_2,f_3), \appl{F}{\real^5}{\real^3}$

\begin{itemize}
\item $F \in C^1$
\item $F(1,1,1,\frac{\pi}{2},0)$ satisface la ecuación
\item Vamos a estudiar el determinante:

\[\det J(1,1,1,\frac{\pi}{2},0) = 
	\begin{pmatrix}
		\dpa{f_1}{x} &\dots &\dpa{f_1}{z}\\
		\vdots & \ddots &\vdots\\
		\dpa{f_3}{x} & \dots & \dpa{f_3}{z}
	\end{pmatrix} =
	\begin{pmatrix}
		2&-1&2\\
		2&2&4\\
		1&1&1
	\end{pmatrix} = -6 \neq 0\]	
\end{itemize}

Nos piden también $DF(\frac{\pi}{2},0)$. Derivamos implícitamente $f_i$ respecto de u:

\begin{itemize}
\item $f_1$
\[2x\dpa{x}{u} - \dpa{y}{u}cos(uv)+ysen(uv)v + 2z\dpa{z}{u} =0\]
Evaluamos en el punto $(1,1,1,\frac{\pi}{2})$.
\[2\dpa{x}{u}(\frac{\pi}{2},0) - \dpa{y}{u}(\frac{\pi}{2},0) + 2 \dpa{z}{u}(\frac{\pi}{2},0) = 0\]
 \item $f_2$
 \[2x\dpa{x}{u} + 2y\dpa{y}{u}-cos(uv)v+4z\dpa{z}{u} = 0\]
 Evaluamos en el punto:
 \[2\dpa{x}{u} (\frac{\pi}{2},0)  + 2\dpa{y}{u}(\frac{\pi}{2},0)  - 4 \dpa{z}{u}(\frac{\pi}{2},0) = 0 \]
 \item $f_3$
 ...
 Evaluamos en el punto
 \[\dpa{x}{u}(\frac{\pi}{2},0) +\dpa{y}{u}(\frac{\pi}{2},0)  + \dpa{z}{u}(\frac{\pi}{2},0) = 0 \]
\end{itemize}

Ahora tenemos que resolver el sistema:
(Ejercicio para el lector). 

Queda $D_uF(\frac{\pi}{2},0)  = (0,0,0)$.

Repetimos el mismo proceso respecto de v (Ejercicio también para ti)

Obtenemos $D_vF(\frac{\pi}{2},0)  = (0,0,0)$

\textbf{Conclusión:} $DF(\frac{\pi}{2}) = 0_{3\times 2}$
\end{problem}

 \begin{problem}[20]
 \solution
 Visto en teoría
 \end{problem}
 
  \begin{problem}[21]
 \solution
 Visto en teoría
 \end{problem}
 
 
\begin{problem}[24]
\[xy = ln\left(\frac{x}{y}\right)\]
\paragraph{a)} Admite una única solución $y=f(x)$ diferenciable en un intervalo que contiene a $\sqrt{e}$ y $f(\sqrt{e}) = \frac{1}{\sqrt{e}}$
\paragraph{b)} $f$ tiene un máximo local en el punto $\sqrt{e}$.
\solution
Aplicaremos el teorema de la función implícita a $g(x,y)=xy-\left(ln(x)-ln(y)\right)$.
\paragraph{a)}
Para ello:
\begin{itemize}
\item Es $C^1$
\item $g(\sqrt{e},\frac{1}{\sqrt{e}}) = 0$.
\item
\[\dpa{}{y}(g(x,y)) (\sqrt{e},\frac{1}{\sqrt{e}}) = 2\sqrt{3}\neq 0\]
\end{itemize}

\paragraph{b)}
$f'(\sqrt{e}) = 0, f''(\sqrt{e})<0$ (Ejercicio para ti)
\end{problem}


\subsection{Hoja 4}
\begin{problem} [1]
¿La unión de subvariedades diferenciables es una subvariedad diferenciable? ¿Y la intersección?
\solution
\begin{itemize}
\item Unión: NO. Tomando 2 rectas, la unión no es una subvariedad diferenciable.
\item Intersección: Tampoco: el conjunto vacío no es una subvariedad diferencial.
\end{itemize}
\end{problem}

%\begin{problem}[2]
%\[M=\{ (x,y,z)\in \real^3 \tq \left\{\begin{array}{c}
%xy &= 0\\
%x^2+y^2+z^2 &= 1\\
%z&\neq 0, \pm 1
%\end{array}\right.\]
%\solution

%Dibujamos el conjunto

%\end{problem}


\begin{problem}[3]
Sean
\ppart
\[f(x) = \left\{\begin{array}{cc}
x sen\frac{1}{x} & x\neq 0\\
0 & x=0 \end{array}\right.\]

\ppart
\[g(x) = \left\{\begin{array}{cc}
x^2 sen\frac{1}{x} & x\neq 0\\
0 & x=0 \end{array}\right.\]

\ppart
\[h(x) = \left\{\begin{array}{cc}
x^4 sen\frac{1}{x} & x\neq 0\\
0 & x=0 \end{array}\right.\]

¿Son rectificables? (es decir, si tienen longitud)
\solution

\spart $f(x)$ no es rectificable. (visto en clase, en el ejemplo de curva no rectificable)

\spart 
\[g'(0) = \lim_{h\to 0} \frac{g(h)-g(0)}{h} = \lim_{h\to 0} h sen\frac{1}{h} = 0\]
\[g'(x) = 2xsen\frac{1}{x} + x^2\cdot cos\frac{1}{x}\cdot\frac{-1}{x^2}\]

Nos encontramos con que $\nexists \lim_{x\to 0} g'(x)$. 

Tenemos que no es continua y no podemos aplicar el teorema. ¿Entonces?

\textbf{Idea:} Sabemos calcular la longitud hasta un punto $\varepsilon$ más o menos cercano al $0$. Vamos a proceder a calcular la longitud entre $\varepsilon$ y $1$ y tomar $\lim_{\varepsilon \to 0}$

Llamamos a $\sigma(x) = (x,g(x))$, con $\sigma \in C^1[\varepsilon,1]$.

Por el teorema tenemos que \[L_{\varepsilon} = \int_{\varepsilon}^1 \md{\sigma'(x)}dx = \int_{\varepsilon}^1 \underbrace{\sqrt{1+2xsen\frac{1}{x} - cos\frac{1}{x}}}_{\leq \sqrt{10}}dx \leq \sqrt{10}(1-\varepsilon) \leq \sqrt{10}\]

Si $\varepsilon = \frac{1}{n}, L_n \leq \sqrt{10}, L_n \nearrow \implies \exists \lim_{n\to\inf} L_n$.




\spart Esta curva es $C^1$, por lo que $\implies$ rectificable.

Comprobar que $h'(x)$ es continua.

\end{problem}


\begin{problem}[15.b]
\[f(x,y,z) = 2x^2+y^2+z^2-xy\]
\[\mathbb{K} = \{ (x,y,z)\in \real^3 : \frac{x^2}{2} + \frac{y^2}{4} + \frac{z^2}{8} \leq 1\}\]
\solution
En el interior sale el punto crítico $(0,0,0)$ y en la forntera tendremos el sistema:

\[\left\{\begin{array}{cc}
4x-y&=\lambda x\\
2y-x &=\lambda \frac{y}{2}\\
2z &= \lambda\frac{z}{4}\\
\frac{x^2}{2}+\frac{y^2}{4} + \frac{z^2}{8} &=1
\end{array}\right\}\]
\end{problem}


\begin{problem}[18]
Fórmula distancia de un punto al plano $Ax+By+Cz+D=0$.

\solution
\[g(x,y,z) = Ax+By+Cz+D \implies \nabla g = (A,B,C)\]

Definimos 
\[f(x,y,z) = d^2 = (x-a)^2 + (y-b)^2 + (z-c)^2 \implies \nabla f = (2(x-a),2(y-b),2(z-c))\]

Aplicamos el teorema de Lagrange:

\[\left\{\begin{array}{cc}
\grad f(x,y,z) &= \lambda \grad g\\
g(x,y,z)&=0
\end{array}\right\}\]

Llegando a la solución según las 3 primeras ecuaciones:
\begin{gather*}
x=\frac{\lambda A}{2} +a\\
y = \frac{\lambda B}{2} +b\\
z = \frac{\lambda C}{2} +c
\end{gather*}
Reemplazamos $x,y,z$ en la ecuación del plano (cuarta ecuación) obteniendo:

\[\lambda = \frac{-2(D+Cc+Bb+Aa)}{A^2+B^2+C^2}\]

Sustituimos este $\lambda$:

\begin{gather*}
x=a-\frac{(D+Cc+Bb+Aa)}{A^2+B^2+C^2} A \\
y = b- \frac{(D+Cc+Bb+Aa)}{A^2+B^2+C^2} B\\
z = c - \frac{(D+Cc+Bb+Aa)}{A^2+B^2+C^2} C
\end{gather*}

Y ahora reemplazamos en $d^2$:

\begin{gather*}
d^2 = \left(\frac{(D+Cc+Bb+Aa)}{A^2+B^2+C^2} A\right)^2 + \left(\frac{(D+Cc+Bb+Aa)}{A^2+B^2+C^2} B\right)^2 + \left(\frac{(D+Cc+Bb+Aa)}{A^2+B^2+C^2} C\right)^2\\
d^2 = \left(\frac{(D+Cc+Bb+Aa)}{A^2+B^2+C^2}\right)^2 \cdot (A^2+B^2+C^2) \\
d^2 = \frac{(D+Cc+Bb+Aa)^2}{A^2+B^2+C^2} \implies d=\frac{\abs{D+Cc+Bb+Aa}}{sqrt{A^2+B^2+C^2}}
\end{gather*}
Que es justamente la fórmula de la distancia de un plano a un punto.

\end{problem}


\begin{problem}[19]
patata
\solution

\[\left\{
\begin{array}{cc}
4x^3&=\lambda\\
4y^3&=\lambda\\
4z^3&=\lambda\\
x+y+z=5
\end{array}\right\}\]
\end{problem}

\subsection{Hoja 5}

\begin{problem}[2]
a) \[f(x,y,z) = x+y+z; \sigma(t) = (sen\,t,cos\,t,t)\, 0\leq t\leq \pi\]
Hallar: $\displaystyle \int_{\sigma}f$

\solution

$\md{\sigma'(t) = \sqrt{2}}$

Vamos con la integral:

\[\int_0^{\pi} f(\sigma(t))\md{\sigma'(t)} dt = \sqrt{2}\int_0^{\pi} sen\,t+cos\,t\,dt = \sqrt{2}\left.\left(-cos\,t+sen\,t+\frac{t^2}{2}\right)\right|_{0}^{\pi} = \frac{\sqrt{2}}{\pi^2+4}\]

\end{problem}

\begin{problem}[3]
\ppart a) 

\[\left\{\begin{array}{cc}
x&=R(t-sin\,t)\\
y&=R(1-cos\,t)
\end{array}\right\}  0\leq2\pi\]

\ppart b) Hallar longitud de la cardioide, dada por $r = 1+cos\,\theta,0<\theta<\pi$

\ppart c) Hallar el área de la región limitada cuando $0\leq\theta\leq2\pi$

\solution

\spart

\begin{gather*}
x'(t) = R(1-cos\,t)\\
y'(t) = Rsen\,t
\end{gather*}

Vamos a calcular la longitud:

\[l = \int_0^{2\pi}\sqrt{R^2(1-cos\,t)^2 + R^2sen^2\,t}dt = ... = R\sqrt{2}\int_0^{2\pi} \sqrt{1-cos\,t}dt\]

Esta integral se resueve multiplicando numerador y denominador por el conjugado, es decir:

\[R\sqrt{2}\int_0^{2\pi} \sqrt{1-cos\,t}\cdot\frac{\sqrt{1+cos\,t}}{\sqrt{1+cos\,t}}dt\]
\[R\sqrt{2}\int_0^{2\pi} \frac{\abs{sen\,t}}{\sqrt{1+cos\,t}} dt\]

\[...\]

\[ = R\sqrt{2} \left(\left.-2\sqrt{1+cos\,t}\right|_0^{\pi} + \left.2\sqrt{1+cos\,t}\right|_{\pi}^{2\pi}\right) = ... = 8R
\]

\spart Leo es demasiado rápido para mí...

\spart 
El área (razonando un poco y con otro poco de fe) sabemos que
 \[A = \frac{1}{2} \int_{\theta_2}^{\theta_1}f(\theta)^2d\theta = 
\frac{1}{2} \int_{0}^{2\pi}(1+cos\,\theta)^2d\theta = ... = \pi + \frac{1}{2}\int_0^{2\pi}\frac{1+cos(2\theta)}{2} = ... \]
\end{problem}


\begin{problem}[4]

$\displaystyle\int \overrightarrow{F}\cdot \overrightarrow{s}$
\ppart $\alpha$ no se como

\ppart $\alpha_2$ no se como

\solution
\spart
La integral que resolver es:

\[
\int_{\alpha} F = \int_{0}^{1} () \cdot (1,1)dt + \int_{1}^{2} (t^2+(2-t)^2,t^2-(2-t)^2)\cdot(1,-1)dt
\]
Y da como resultado:
\[
\frac{2}{3} + 2\left.\left(4t-2t^2+\frac{t^3}{3}\right|_1^2\right)
\]

\spart
La parametrización es $\sigma(t) = (t,t^2), t\in[-2,1]$.
La integral quedaría $\displaystyle \int_{-2}^{1} F(\sigma(t))\cdot \sigma'(t)dt$

\spart
Aquí tenemos \[\begin{array}{cc}
x(t)= a\,cos(t)\\
y(t)= b\, sen(t)
\end{array}, t\in[0,2\pi]
\]

La integral quedaría $\displaystyle \int_{-2}^{1} F(\sigma(t))\cdot \sigma'(t)dt$

\end{problem}

\begin{problem}[5]
\[
\left\{\begin{array}{cc}x&=sen(t)\\y&=cos(t)\\z&=cosh(t)\end{array}\right\}, 0\leq t \leq log(7)
\]
\solution
Entonces tenemos la integral:
\[
l = -\int_0^{log(7)} \sqrt{sen^t+cos^2t+senh^2t} dt = ... = \left.senh(t)\right|_0^{log(7)}
\]
\end{problem}

\begin{problem}[6]
El campo es: \[F(x,y) = (x^2-y,x+xy+y^2)\]

\solution

La parametrización es: $\alpha(t) = (cos(t),(sen(t))$

Y la integral sería:

\[
\int_0^{\pi} F(\alpha(t))\cdot\alpha'(t) dt = ... = 2\pi
\]

Aplicando el \textbf{teorema de Green:}

\[
\int\int_D \left(\dpa{Q}{x} - \dpa{P}{y}\right) dxdy = \int\int_D  (2+y)dxdy = \int_0^{2\pi} \int_0^1 r(2+rsen(\theta))drd\theta = ... = 2\pi
\]

Se calcula mucho mejor en polares

\end{problem}

\begin{problem}[7]
noseadnsjfdnjsk
\solution
El campo no es $C^1$ en $(0,0)$.
\end{problem}

\begin{problem}[8]
\[\int_{\{Gamma} ... \]
\solution
Aplicando el teorema de Green:

\[\int \int_D 1 - 1 dx dy = 0\]
\end{problem}

\begin{problem}[Generico]
Encontrar el area de un lazo de la rosa de 4 pétalos, con $sen(2\theta)$
\solution
La curva que lo parametriza es \[\sigma(\theta) = \begin{array}{cc}
r(\theta))cos(\theta)\\
r(\theta))cos(\theta)
\end{array}
\]

Definimos el campo $F(x,y) = (0,x)$ y aplicamos el teorema de Green:

\[
\int_{\Omega}dxdy = \int_{\sigma} Fds = \int_0^{\frac{\pi}{2}} F(\sigma(\theta)) \sigma'(\theta)d\theta = \int_0^{\frac{\pi}{2}} sen(2\theta)cos(\theta) \cdot (2cos(2\theta))sen(\theta) + sen(2\theta)cos(\theta)d\theta
\]

\end{problem}

\begin{problem}[Generico 2]
Calcular el trabajo realizado por el campo $F(x,y,z) = (x^2,yz,y)$ para transportar una partícula a lo largo de la trayectoria $\sigma(t) = (e^t,t,t^2), 0\leq t \leq 1$.
\solution
Calcular la integral 
\[
\int_{\sigma} F = \int_0^1 F(\sigma(t))\sigma'(t)dt
\]
\end{problem}

\begin{problem}[El que cae mañana]
Consideremos el conjunto de todas las pirámides de base rectangular, inscritas en la esfera de radio 1, con la base paralela al plano z=0,y vértice en algún vértice.

\ppart Justificar que a igual base, la pirámide de mayor volumen debe tener el vértice situado en $(0,0,1)$.

\ppart Hallar la longitudo de las aristas que parten del vértice, en la pirámide de la clase anterior que tiene volumen distinto.

\solution
\spart
La base va a estar inscrita en una circunferencia.

Llamemos al plano en el que se encuentra la base $z=a$ y sea $V_3$ el vértice de la altura de la pirámide. Entonces tenemos que $h = \abs{V_3 + a}$.

Con lo cual, el volumen es $= \frac{1}{3} l_1\cdot l_2 \cdot \abs{v_3+a}$ (siendo $l_i$ los lados de la base). Si calculamos máximos y mínimos aquí no vamos a encontrar.

Sabemos también que $v_3 = -\sqrt{1-v_1^2-v_2^2}$
Con lo que definimos 
\[
\tilde{V} (V_1,V_2) = \frac{1}{3} l_1 \cdot l_2 \abs{\sqrt{1-v_1^2-v_2^2} + a}
\]

Derivamos:
\[
\dpa{\tilde{V}}{V_1} = \frac{1}{3}l_1l_2\frac{v_1}{\sqrt{-v_1^2-v_2^2}}
\]
\[
\dpa{\tilde{V}}{V_2} = \frac{1}{3}l_1l_2\frac{v_2}{\sqrt{-v_1^2-v_2^2}}
\]

$(V_1,V_2) = (0,0)$ y se comprueba que es un máximo con la segunda derivada. Como $v_1=v_2=0 \implies v_3 = -1$

Una posibilidad es razonar que a igualdad de base, el volumen será máximo con la altura máxima.

\spart

$V(x,y,z)$ volumen de la pirámide. 

$g(x,y,z) = x^2+y^2+z^2-1$

Definimos $\pi_1 = (x,y,z),\pi_2 = (-x,y,z),\pi_3 = (x,-y,z),\pi_4 = (-x,-y,z), V=(0,0,1)$

\[d(\pi_1,\pi_2) = l_1 = \sqrt{(2x)^2} = \abs{2x}\]
\[d(\pi_1,\pi_3) = l_2 = \sqrt{(2y)^2} = \abs{2y}\]

Volumen $ = \frac{1}{3}\abs{2x}\abs{2y}(1+z)$

Aquí se aplican multiplicadores de Lagrange, quedando el sistema:

\[
\left\{
\begin{array}{cc}
\displaystyle
\frac{2y(z+1)}{3} &= 2\lambda x\\
\frac{2}{3}x(z+1) &= 2\lambda y\\
\frac{4}{3}xy&=2\lambda z\\
x^2+y^2+z^2=1
\end{array}\right\}
\]
\end{problem}

\begin{problem}[10]

adsf
\solution
Es simple algebra lineal al parecer.
\end{problem}

\begin{problem}[11]
Sea \[s=\{(x,y,z)\in\real^3: x^2+y^=1, 0<z<1\}\]
Y 2 parametrizaciones:

\[\Phi(s,t)  = (cos\,s,sen\,s,t) s\in(0,2\pi),t\in(0,1)\]
\[\Psi(u,v)=(u,\sqrt{1-u^2},v),u,v,\in(0,1)\]
¿Inducen la misma orientación?
\solution
Vamos a construir una $\appl{g}{\real^2}{\real^2}$ tal que $\Phi = \Psi \circ g$

De tal forma que

\[\begin{array}{c}
\Phi(s,t) = \Psi(u,v)\circ g(s,t)\\
g(s,t) = (g_1(s,t),g_2(s,t))
\end{array}\]
Donde \[\left.\begin{array}{c}
g_1 (s,t) = cos\,s\\
g_2(s,t) = t
\end{array}\right\}\implies g(s,t)= (cos\,s,t)\]

Comprobamos que $\Psi(g(s,t)) = (cos\,s,\underbrace{\sqrt{1-cos^2\,t}}_{\equiv sen\,s},t) = \Phi(s,t)$

Las 2 parametrizaciones inducirán la misma orientación $\dimplies \det Dg > 0$. (por un resultado visto en teoría)

\[Dg = \begin{pmatrix}
-sen(s)&0\\0&1
\end{pmatrix}\]
Tenemos $\det Dg = -sen(s)$

No inducen la misma orientación ya que este determinante no es siempre positivo.

\end{problem}

\section{Hoja 6}

\begin{problem}[1]
Sea $\omega$ una 2-forma en $\real^3$.

\ppart
\[\omega = dy\y dz - 2 dz\y dx + 3 dx\y dy\]
Sean $\gu = (1,-1,0)$ y $\gv = (2,1,1)$.

\ppart
\[\omega = 4dx\y dz + 2dx\y dy - 3 dy\y dz + 8 dz\y dx + zdx\y dy\]

¿$\omega(\gx)[\gu,\gv]$?


\solution
\spart
Si $\omega$ es una 2-forma:

\[\omega \equiv \sum F_{iN}(x)dx_i\y dx_j\]
Entonces $\omega(\gx)[\gu,\gv] = \sum_{ij} f_{ij}(x)\det \begin{pmatrix}
u_1&v_i\\u_j&v_j
\end{pmatrix}$

En nuestro caso:

\begin{gather*}
\omega(\gx)[(1,-1,0),(2,1,1)] = 1 \left|\begin{array}{cc}
u_2&v_2\\u_3&v_3
\end{array}\right| +
2 \left|\begin{array}{cc}
u_3&v_3\\u_1&v_1
\end{array}\right|+
3 \left|\begin{array}{cc}
u_1&v_1\\u_2&v_2
\end{array}\right|\\
1 \left|\begin{array}{cc}
-1&1\\0&1
\end{array}\right|
+
2 \left|\begin{array}{cc}
0&1\\1&2
\end{array}\right|
+
3 \left|\begin{array}{cc}
1&2\\-1&1
\end{array}\right| = 10
\end{gather*}

 \spart Lo primero es redefinirla y juntar los términos:
 
 \[\omega = -4dx\y dz + (z+2)dx\y dy - 3 dy\y dz\]
 
 Y ahora ya nos ponemos a calcular con el mismo sistema que antes.
 
 Llegamos a \[\omega(\gx)[\gu,\gv] = ... = z-5\]
\end{problem}

\begin{problem}[2]

\[\omega = cx - zdy\]
\[v= (x^2+y^2+z^2)dx\y dz + (xyz)dy\y dz\]

\ppart $d\omega$
\ppart $\omega \y dw$
\ppart $dv$
\ppart $\omega \y v$

\solution

\spart $dw = d(dx-zdy) = \underbrace{d(dx)}_{\equiv 0} - \underbrace{d(zdy)}_{dz\y dy + z(dy\y dy)} = -dy\y dz$ 

\spart 

\spart 

\spart 


\end{problem}

\begin{problem}[13]
\[S=\{(x,y,z)\in\real^3: z=1-\frac{x^2+y^2}{2},x,y\in[0,1]\}\]

\solution

Buscamos calcular $\int_S xydxdy$. $xy$ Es la densidad superficial (dato del enunciado). Estamos integrando la densidad sobre la superficie.

Definimos: $\Phi(x,y) = \left(x,y,1-\frac{x^2+y^2}{2}\right)$

Calculamos $\md{T_x\x T_y}$ (el cambio en la medida), con
\[T_x = (1,0,-x);T_y=(0,1,-y)\]

\[T_x\x T_y = (x,y,1)\]

\[\int_S xy \md{T_x\x T_y} dxdy = \int_S xy\sqrt{1+x^2+y^2}dxdy = \int_0^1\int_0^1 xy\sqrt{1+x^2+y^2} dxdy = ... = 1.40\]

\end{problem}

\begin{problem}[14]
Calcular el flujo de $\overrightarrow{F}(x,y,z) = (x,y,z)$ a través del cilindro $x^2+y^2=a^2, 0\leq z \leq b$ incluyendo las bases con la normal exterior.

\solution

\textbf{Solución corta y buena}
\[\int \int_S Fds = \int \int \int_{\Omega}div F\]

Utilizando coordenadas cilíndricas:

\[\int_0^b \int_0^a\int_0^{2\pi} 3 p d\alpha dp dz = 3\pi a^2b\]
\textbf{Sin aplicar Gauss}

Vamos a separar el cálculo en la integral sobre las 2 tapas y la superficie de revolución.

El campo en la tapa de abajo es 0 (el campo con z=0 es paralelo a la base)

Nos queda: \[\int \pesc{F(\sigma(\alpha,z)),dA)}\]

Como cambia el ángulo y la altura definimos:
Siendo $\sigma(\alpha,z) =(acos\alpha,asen\alpha,z)$.

\begin{gather*}
\sigma_{\alpha} = (-asen\alpha,acos\alpha,0)\\
\sigma_{z} = (0,0,1)
\end{gather*}

Y ahora calculamos $dA$, que en $\real^3$ es el producto vectorial.

$dA = (acos\alpha,asen\alpha,0)$

Ya podemos integrar sobre la superficie lateral

\[\int_0^{2\pi} \int_0^b a^2dzd\alpha = 2\pi a^2b\]


Ahora procedemos al cálculo del flujo que atraviesa la capa de arriba.

En este caso, como cambia el ángulo y el radio tenemos:


Siendo $\sigma(\alpha,z) =(pcos\alpha,psen\alpha,b)$, y por lo tanto: $F(\sigma(\alpha,p)) = (pcos\alpha,psen\alpha,b)$

\begin{gather*}
\sigma_{\alpha} = (-psen\alpha,pcos\alpha,0)\\
\sigma_{z} = (cos\alpha,sen\alpha,0)
\end{gather*}

Si calculamos $\sigma_a\x\sigma_p = (0,0,-p)$

Problema, porque este vector apunta hacia el interior del cilindro y nos dicen respecto de la normal interior. Por ello le vamos a cambiar de signo.

\[\int_0^{2\pi} \int_0^a b\cdot p \,dpd\alpha=\pi a^2b\]

Tenemos flujo = $2\pi a^2b+\pi a^2b$
\end{problem}


\begin{problem}[9]	
$\omega = xzdy - ydx$

$v = x^3dz+dx$


\[\Phi(s,t) = (cos(s),sen(t),(t))\]

Nos piden calcular 
\ppart
$\Phi^{\ast}(\omega)$
\ppart
$\Phi^{\ast}(d\omega)$
\solution
Recordamos:
\[\pb{\sum f_Idx_I}(s)[\gv] = \sum f_I(T(s))dt_I(\gv)\]
\[\Phi^{\ast}(dx)=d\Phi_1 = -sen(s)ds\]
\spart
\[\Phi^{\ast}(\omega) = \Phi^{\ast} (xzdy - ydx) - \Phi(y)dx = cos^2(s)tds +sen^2(s)ds\]
\spart

Hay 2 posibles caminos, utilizar que el pull-back conmuta con la diferencial, teniendo que calcular la diferencial exterior del pull-back calculado anteriormente y sino:
\[dw = zdx\y dy + x\df{z,y} - \df{y,x} = (z+1)\df{x,y} + x\df{z,y} = (1) = (z+1)\df{x,y} - x\df{y,z}\]
(1) para mantener el orden cíclico.
Ahora habría que calcular el pull-back de esto.

Vamos con el otro camino:

\[d(\Phi^{\ast}\omega) (cos^2(s))dt\y ds\]
\end{problem}