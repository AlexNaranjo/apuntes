% -*- root: ../GeometriaTopologia.tex -*-

Para evitarnos emborronar mucho los apuntes, ponemos en esta sección los cálculos y cuentas largas que vamos haciendo durante el curso.

\section{Proyección estereográfica de $\crc$}
\label{sec:proyeccion_estereografica_crc}

\subsection{Cálculo de las cartas}
Empecemos calculando la carta $(V_1, \alpha_1)$ con $V_1 = \crc \setminus {(0,1)}$.

Utilizando el dibujo \ref{fig:ProyEstereo} como guía, partimos de la recta $y = ax + b$. Como pasa por el polo norte, tenemos que $b = 1$, y como pasa por el punto $(s,t)$ en $\crc$, tenemos que $t=a \cdot s + 1 \implies a = \frac{t-1}{s}$; luego $y = \frac{t-1}{s} \cdot x + 1$.

Vamos a estudiar dónde se anula, ya que estamos haciendo la proyección sobre $y=0$, luego despejamos la $x$: $x = \frac{-s}{t-1} = \frac{s}{1-t}$. Y con esto ya hemos calculado la primera carta.

Ahora calculamos, de manera similar la segunda carta $(V_2, \alpha_2)$, pero esta vez proyectando desde el polo sur: partimos de la recta $y=a \cdot x+b$ y como pasa por $(s,t) \in \crc$, y en $(0,-1)$, tenemos que $y = \frac{t+1}{s} \cdot x - 1$. Proyectando sobre $y=0$, obtenemos que $x = \frac{s}{t+1}$.

\subsection{Compatibilidad de las cartas}
Veamos que son compatibles, es decir, que
\[ \appl{\alpha_2 \circ \alpha_1^{-1}}{\alpha_2(V_1 \cap V_2)}{\alpha_1(V_1 \cap V_2)} \]
está bien definida y es diferenciable.

En primer lugar, observamos que $V_1 \cap V_2 = (-\infty, 0) \cup (0, \infty)$.
En segundo lugar, para calcular la imagen de $\alpha_2 \circ \alpha_1^{-1}$, vamos a hacer la composición por pasos.

Empezamos por $x \xrightarrow{\alpha_1^{-1}} (s,t)$. Así que tenemos que calcular $(s,t)$ en función de x. Como tenemos que $x = \frac{s}{1-t}$, tomamos cuadrados $x^2 = \frac{s^2}{(1-t)^2}$, y como $(s,t) \in \crc$, tenemos que $s^2+t^2=1$.

Despejando $s^2$ y sustituyendo en $x^2$, obtenemos lo siguiente:
\[ x^2 = \frac{1 - t^2}{(1-t)^2} = \frac{(1-t)(1+t)}{(1-t)^2} = \frac{(1+t)}{(1-t)} \]

Despejamos la $t$:
\begin{gather*}
	x^2(1-t) = 1+t \iff x^2 - 1 = x^2t + t = t(x^2 + 1)\\
	t = \frac{x^2 - 1}{x^2 + 1}
\end{gather*}

Ahora nos falta despejar la s de $x = \frac{s}{1-t} \implies s = x (1-t)$. Sustituyendo la t y operando, obtenemos la expresión de $s$ en función de $x$:
\[ s = x \left(1 - \frac{x^2-1}{x^2+1}\right) = x \cdot \frac{x^2 + 1 - x^2 + 1}{x^2+1} = \frac{2x}{x^2+1} \]
Con lo que obtenemos finalmente la expresión de $(s,t)$ en función de $x$:
\[ (s,t) = \left(\frac{2x}{x^2+1}, \frac{x^2-1}{x^2+1}\right) \]

Y ahora calculamos la imagen de $(s,t)$ por $\alpha_2(s,t) = \frac{s}{1+t}$:
\[ \alpha_2\left(\frac{2x}{x^2+1},\frac{x^2-1}{x^2+1}\right) = \frac{\frac{2x}{x^2+1}}{1+\frac{x^2-1}{x^2+1}} = \dots = \frac{1}{x} \]

Luego
\[ \alpha_2 \circ \alpha_1^{-1}(x,y) = \frac{1}{x} \]
