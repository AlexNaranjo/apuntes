% -*- root: ../GeometriaTopologia.tex.tex -*-
\section{Hoja 1}

\begin{problem}[5] Demostrar que las variedades $\quot{ℝ^2}{ℤ^2}$, $\projp^n$ y $\projcp^n$ son compactas.

\solution

\doneby{Guille}


\end{problem}

\begin{problem} Probar que si una variedad $M$ posee un atlas que consiste de dos cartas cuya intersección es conexa entonces $M$ es orientable. Deducir que $\crc[n]$ es orientable.

\solution

Según la \fref{def:VariedadOrientable}, una variedad es orientable si existe un atlas $A = \set{(U_i,φ_i)}_{i∈I}$ tal que $\abs{\Dif (φ_j○\inv{φ_i})} > 0$ para cualesquiera $j,i ∈ I$. Más generalmente, nos vale con que la diferencial de los difeomorfismos de cambio de carta tengan el mismo signo.

Por ser el cambio de carta un difeomorfismo, su diferencial es continua y además no se anula nunca. La única forma de que cambiase de signo es que la intersección de cartas no fuese conexa (podría saltar de valor entre componentes conexas), pero por hipótesis la intersección es conexa, luego la diferencial mantiene signo y por lo tanto la variedad es orientable.

En $\crc[n]$ se puede dar siempre un atlas con dos cartas con intersección conexa a través de la proyección estereográfica, luego es orientable.

\end{problem}

\begin{problem}[8] Sean $M$ y $N$ dos variedades orientables, y sea $\appl{f}{M}{N}$ una aplicación diferenciable.

\ppart Define el concepto de morfismo que preserva la orientación y pon un ejemplo de uno (y de otro que no lo sea).

\ppart Demostrar que una variedad cociente $\quot{M}{G}$ en la que $M$ es orientable y los elementos de $G$ preservan la orientación es orientable.

\ppart Deducir que los espacios proyectivos de dimensión impar son orientables.

\solution

\spart

La definición de orientación de Geometría Diferencial \citep[Def. IV.7]{ApuntesGeoDif} era bastante cómoda, ya que sólo dependía de la existencia de una $n$-forma de volumen que no se anulase. La cuestión es que dudo bastante que podamos usar eso aquí, así que toca ir a la definición fea, dependiente de las cartas.

\begin{defn}[Aplicación\IS compatible con la orientación] Sean $M$ y $N$ dos variedades orientables, y sea $\appl{f}{M}{N}$ una aplicación diferenciable. Diremos que $f$ es compatible con la orientación (o que preserva la orientación) si y sólo si, para dos cartas cualesquiera $(U_i, φ_i)$ y $(U_j, φ_j)$ de $M$ y $N$ respectivamente, el jacobiano dado por \[ \Dif (φ_i ○ f ○ \inv{φ_j})\] es positivo en la región en la que esté definido.
\end{defn}

Los ejemplos no se me ocurren ahora.

\spart

\end{problem}
