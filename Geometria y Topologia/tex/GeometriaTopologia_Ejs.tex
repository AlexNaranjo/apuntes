% -*- root: ../GeometriaTopologia.tex.tex -*-
\section{Hoja 1}

\begin{problem}[1] Considérese en $S^1$ las 2 cartas siguientes:

\[
U_1 = S^1 \setminus {(1,0)} \; \appl{φ_1^{-1}}{U_1}{(0,2π)} \text{ con } φ_1^{-1}(θ) = (\cos θ,\sin θ)
\]
\[
U_2 = S^1 \setminus {(-1,0)} \; \appl{φ_2^{-1}}{U_2}{(-π,π)} \text{ con } φ_2^{-1}(θ) = (\cos θ,\sin θ)
\]

\ppart Comprobar que estas 2 cartas definen un atlas en $S^1$.

\ppart Comprobar que el atlas anteriror es equivalente a los atlas definidos mediante la proyección estereográfica y la proyección a los ejes respectivamente.

\solution

\doneby{Dejuan}

\spart ¿$\{φ_1,φ_2\}$ es un atlas de $S^1$ si $\img(φ_1) \cup \img(φ_2) = S^1$? En caso de ser así, sólo hay que escribir las cosas con cuidado.

\spart 2 atlas son equivalentes si sus cartas son compatibles. Vamos a tomar $\{\Psi_i\}$ las cartas de las proyección estereográfica. La condición de compatibilidad es si $φ_1 \circ \Psi_1^{-1}$ es diferenciable.

Para más información, este ejercicio está esbozado ejemplificando la \ref{def::atlas_compatibles}

\end{problem}


\begin{problem}[5] Demostrar que son compactas las variedades 

\ppart $\quot{ℝ^2}{ℤ^2}$

\ppart  $\projp^n$

\ppart $\projcp^n$


\solution

\doneby{Guille}
\spart 
La relación de equivalencia que usamos para construir el espacio cociente nos dice que dos puntos $x,y ∈ ℝ^2$ pertenecen a la misma clase si existe un difeomorfismo $g ∈ ℤ^2$ tal que $g(x) = y$.
Es decir, si tenemos un par $(m,n) ∈ ℤ^2$ tal que $(x_1 + m, x_2 + n) = (y_1, y_2)$, entonces $x = (x_1, x_2)$ y $y=(y_1, y_2)$ están relacionados.

Con esta construcción, vemos que nuestro conjunto de clases de equivalencia son los puntos en $[0,1] × [0,1]$, con los bordes ``unidos'', el de arriba con el de abajo y el de la izquierda con el de la derecha. Tomando para $0<a<1$, el punto $(0,a) ~ (1,a)$, tomando $(n,m) = (1,0)$. \foonote{topológicamente estamos ante un toro (ver \fref{fig:ToroEspacioCociente})}

Esta variedad es compacta porque es ¿homeomorfa? al cuadrado $[0,1]×[0,1]\subset ℝ^2$, que es cerrado y acotado.

\spart $\projp^n$ podemos escribirlo como $\quot{\crc[n]}{\gen{J}}$, tomando y siendo $J(x_1,...,x_n) = (-x_1,...,-x_n)$. 

\[\projp^n \simeq \quot{\crc[n]}{\gen{J}} \subset \real^n\]

Tenemos un conjunto cerrado y acotado en $\real^n$, con lo que es compacto.


\spart $\projcp^n = \quot{ℂ^{n+1}\setminus 0}{~}$ donde $\gx ~ \gy \dimplies ∃λ∈ℂ^{n+1} λ\gx = \gy$.

De esta manera, $\projcp^n \simeq \projp^{2n} \simeq \quot{\crc[2n]}{\gen{J}}$.

Tenemos un conjunto cerrado y acotado en $\real^{2n}$, con lo que es compacto.
\end{problem}


\begin{problem}[7]
Probar que si una variedad $M$ posee un atlas que consiste de dos cartas cuya intersección es conexa entonces $M$ es orientable. Deducir que $\crc[n]$ es orientable.

\solution

Según la \fref{def:VariedadOrientable}, una variedad es orientable si existe un atlas $A = \set{(U_i,φ_i)}_{i∈I}$ tal que $\abs{\Dif (φ_j○\inv{φ_i})} > 0$ para cualesquiera $j,i ∈ I$. Más generalmente, nos vale con que la diferencial de los difeomorfismos de cambio de carta tengan el mismo signo.

Por ser el cambio de carta un difeomorfismo, su diferencial es continua y además no se anula nunca. La única forma de que cambiase de signo es que la intersección de cartas no fuese conexa (podría saltar de valor entre componentes conexas), pero por hipótesis la intersección es conexa, luego la diferencial mantiene signo y por lo tanto la variedad es orientable.

En $\crc[n]$ se puede dar siempre un atlas con dos cartas con intersección conexa a través de la proyección estereográfica, luego es orientable.

\end{problem}

\begin{problem}[8] Sean $M$ y $N$ dos variedades orientables, y sea $\appl{f}{M}{N}$ una aplicación diferenciable.

\ppart Define el concepto de morfismo que preserva la orientación y pon un ejemplo de uno (y de otro que no lo sea).

\ppart Demostrar que una variedad cociente $\quot{M}{G}$ en la que $M$ es orientable y los elementos de $G$ preservan la orientación es orientable.

\ppart Deducir que los espacios proyectivos de dimensión impar son orientables.

\solution

\spart

La definición de orientación de Geometría Diferencial \citep[Def. IV.7]{ApuntesGeoDif} era bastante cómoda, ya que sólo dependía de la existencia de una $n$-forma de volumen que no se anulase. La cuestión es que dudo bastante que podamos usar eso aquí, así que toca ir a la definición fea, dependiente de las cartas.

\begin{defn}[Aplicación\IS compatible con la orientación] Sean $M$ y $N$ dos variedades orientables, y sea $\appl{f}{M}{N}$ una aplicación diferenciable. Diremos que $f$ es compatible con la orientación (o que preserva la orientación) si y sólo si, para dos cartas cualesquiera $(U_i, φ_i)$ y $(U_j, φ_j)$ de $M$ y $N$ respectivamente, el jacobiano dado por \[ \Dif (φ_i ○ f ○ \inv{φ_j})\] es positivo en la región en la que esté definido.
\end{defn}

Los ejemplos no se me ocurren ahora.

\spart

\end{problem}
