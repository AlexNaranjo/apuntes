\documentclass[palatino]{apuntes}

\title{Geometría y Topología}
\author{}
\date{15/16 C2}

% Paquetes adicionales
\usepackage{enumitem}
\usepackage{tikztools}
\usepackage{fancysprefs}

\setlist{itemsep=1pt, topsep=5pt}
\bibliographystyle{plainnat}
% --------------------

\begin{document}
\pagestyle{plain}

% http://tex.stackexchange.com/a/14243
\relpenalty=10000
\binoppenalty=10000

\begin{abstract}
Estos son los apuntes del curso de Geometría y Topología, del profesor Gabino González.
\end{abstract}

\maketitle


\tableofcontents
\newpage
% Contenido.

\chapter{Conceptos básicos}

En Geometría, los objetos que estudiamos se llaman ``variedades''. Veremos de distintos tipos (por ejemplo, en Geometría Diferencial \citep{ApuntesGeoDif} veíamos variedades diferenciables), aunque nosotros empezaremos con las topológicas.

\begin{defn}[Variedad\IS topológica] Una variedad topológica es un espacio topológico $M$ con las siguientes propiedades:
\begin{enumerate}
\item $M$ es $T_2$ (esto es, \concept{Hausdorff}: dos puntos distintos tienen entornos disjuntos).
\item $∀p ∈ M$ admite un entorno $U$ y un homeomorfismo $\appl{φ_u}{U}{ℝ^N}$ (o $\bola^N$).

Al par $(U,φ_u)$ se le llama \concept{Carta} para $p$. Si $φ_u(p) = 0$, se dice que la carta está centrada en $p$. A la colección de cartas se le llamará \concept{Atlas}.
\item Si para cualquier par de cartas $(U,φ_u)$ y $(V,φ_v)$ la aplicación $$\appl{φ_v ○ \inv{φ_u}}{φ_u(U∩V)}{φ_v(U∩V)}$$ es difeomorfismo, estamos entonces ante una \concept{Variedad\IS diferenciable}.
\end{enumerate}
\end{defn}

\begin{figure}[thbp]
\centering
\inputtikz{Cartas}
\caption{Un esquema de las cartas de una variedad $M$ y cómo se comportan en la intersección.}
\label{fig:Cartas}
\end{figure}

La dimensión de la variedad está dada por la dimensión de $ℝ$ a la que son homeomorfas las cartas. La cuestión es que no tenemos claro si eso está bien definido. En el caso diferenciable, la condición de difeomorfismo para la intersección de cartas implica que la dimensión de ambas cartas ha de ser la misma. En el caso topológico también está bien definido, aunque es más difícil de demostrar ya que dependemos de que no exista un homeomorfismo entre $ℝ^n$ y $ℝ^m$ con $n ≠ m$, que no es trivial.

Un ejemplo sencillo de variedad es $M = \bola^N$, con $φ$ la identidad. Otro ejemplo es $\crc$ (la circunferencia), que es una variedad de dimensión 1, que no se puede dar con sólo una carta (la circunferencia no es homeomorfa a $ℝ$)\footnote{$\crc$ es compacta y $\real$ no, y como compacidad es propiedad topológica y los homeomorfismos las preservan, no pueden ser homeomorfas.}. Podríamos darla tomando las dos mitades superior e inferior usando senos y cosenos, y también podríamos hacer la proyección estereográfica (\ref{fig:ProyEstereo}) desde los polos norte y sur $(0,1), (0,-1)$ respectivamente sobre la recta real. En este caso, tendríamos las siguientes cartas: \[
\begin{matrix}
	\appl{α_1}{V_1 = \crc\setminus\set{(0,1)}&}{&ℝ} \\
	p=(s,t) &\longmapsto& \frac{s}{1-t}
\end{matrix}
\qquad
\begin{matrix}
	\appl{α_1}{V_2 = \crc\setminus\set{(0,-1)}&}{&ℝ} \\
	p=(s,t) &\longmapsto& \frac{s}{1+t}
\end{matrix}\]

\begin{figure}[hbtp]
\inputtikz{ProyeccionCirc}
\caption{Proyección estereográfica de la circunferencia.}
\label{fig:ProyEstereo}
\end{figure}

Para comprobar si este atlas es diferenciable, tendríamos que mirar qué ocurre con $α_2 ○ \inv{α_1}$. Después de un montón de cuentas\footnote{Ver \fref{sec:proyeccion_estereografica_crc}.}, nos sale que efectivamente lo es ($α_2 ○ \inv{α_1} = \frac{1}{x}$) en el dominio en el que está definido (el cero no es un problema porque no está dentro del dominio).

Trivialmente, podemos definir cuándo dos atlas son compatibles.

\begin{defn}[Atlas\IS compatibles] Se dice que dos atlas $A_1, A_2$ son compatibles si $A_1 ∪ A_2$ es un atlas. Esto es, si y sólo si las cartas de $A_1$ son compatibles con las de $A_2$.
\end{defn}

Por ejemplo, podemos estudiar si los dos atlas que hemos visto para la circunferencia \crc son compatibles (recordamos que uno era el trigonométrico y otro la proyección estereográfica). Esto es equivalente a preguntarnos si $α_j ○ \inv{φ_i}$ son diferenciables. Se puede ver fácilmente que \[ α_1 ○ \inv{φ_1} (θ) = α_1(\cos θ, \sin θ) = \frac{\cos θ}{1 - \sin θ}\] que es diferenciable. Así podríamos hacerlo con el resto de combinaciones, y por lo tanto tenemos que ambos atlas dan la misma variedad.

Como ejercicio, podríamos dar un atlas $A_3$ en \crc con cartas en forma de semicircunferencia y después comprobar que es compatible con los dos atlas de antes. Otro ejercicio más largo sería hacer lo análogo para $\crc[2]$.

Una vez que tenemos ya definido qué es una variedad, el siguiente paso es saber si podemos hacer análisis ahí: si podemos definir aplicaciones diferenciables en ella o si podemos integrar una función. La segunda parte la veremos más adelante con las formas diferenciales, pero la primera la podemos estudiar ahora.

\begin{figure}[hbtp]
\centering
\inputtikz{ApplDiferenciable}
\caption{Esquema de la definición de la aplicación diferenciable entre dos variedades en base a las cartas.}
\label{fig:ApplDiferenciable}
\end{figure}

\begin{defn}[Aplicación\IS diferenciable] Una aplicación continua $\appl{f}{M}{N}$ entre dos variedades $M$ y $N$ es diferenciable si $∀p ∈ M$ con $f(p) = q ∈ N$ existe una carta $(U,φ_U)$ alrededor de $p$ y una carta $(V, φ_V)$ alrededor de $q$ tal que $f(U) ⊂ V$ y $\appl{φ_V○f ○ \inv{φ_U}}{\bola^m}{\bola^n}$ es diferenciable.
\end{defn}

Es importante ver que este concepto de diferenciabilidad no depende de las cartas elegidas para cada punto. Suponiendo que tenemos otras dos cartas $α_U, α_V$ alrededor de $p$ y $q$ tendríamos que \[ α_V ○ f ○ \inv{α_U} = (α_V ○\inv{φ_V}) ○ (φ_V ○ f ○ \inv{φ_U}) ○ (φ_U ○ \inv{α_U})\]

Por compatibilidad de las cartas, $α_V○\inv{φ_V}$ y $φ_U ○ \inv{α_U}$ son diferenciables. Además, ya que hemos dicho que $f$ es diferenciable con las cartas $φ_U, φ_V$ luego $φ_V ○ f ○ \inv{φ_U}$ es diferenciable igualmente. Así, la composición de esas tres funciones es diferenciable.

\begin{example} Vamos a definir una función entre variedades y ver si es diferenciable. No nos complicaremos mucho: \begin{align*}
\appl{f}{\crc&}{\crc ⊂ ℂ} \\
z &\longmapsto \conj{z}
\end{align*}

Haciendo los cálculos, vemos que \[ φ_1 ○ f ○ \inv{φ_1} (θ) = φ_1 ○ f(e^{iθ}) = φ_1(e^{-iθ}) = \begin{cases} -θ \\ -θ + 2θ \end{cases} \] que efectivamente es diferenciable.
\end{example}


\begin{example} Definimos ahora una función a priori menos interesante, la identidad: \begin{align*}
\appl{f}{M&}{N} \\
p &\longmapsto p
\end{align*}

$M$ será $(ℝ,φ)$ y $N = (ℝ,α)$, con $φ$ la identidad y $α(t) = t^3$. Es obvio ver que $f$ es diferenciable (si hacemos la composición nos sale directamente). Ahora bien, si tomamos $\appl{f}{N}{M}$, ¿sigue siendo diferenciable? En este caso, si calculamos vemos que $φ ○ f ○ \inv{α}(t) = \sqrt[3]{t}$ pero esta aplicación no es diferenciable en $0$. Esto es lo mismo que decir que estos dos atlas en $ℝ$ no son compatibles: no definen la misma estructura. Sí serían compatibles si estuviésemos hablando sólo de variedades topológicas, porque sí que estamos ante un homomorfismo.
\end{example}

\chapter{Espacios tangente y cotangente}

\chapter{Cohomología de De Rham}

%% Apéndices (ejercicios, exámenes)
\appendix

\chapter{Cálculos y otras explicaciones}
% -*- root: ../GeometriaTopologia.tex -*-

Para evitarnos emborronar mucho los apuntes, ponemos en esta sección los cálculos y cuentas largas que vamos haciendo durante el curso.

\section{Proyección estereográfica de $\crc$}
\label{sec:proyeccion_estereografica_crc}

\subsection{Cálculo de las cartas}
Empecemos calculando la carta $(V_1, \alpha_1)$ con $V_1 = \crc \setminus {(0,1)}$.

Utilizando el dibujo \ref{fig:ProyEstereo} como guía, partimos de la recta $y = ax + b$. Como pasa por el polo norte, tenemos que $b = 1$, y como pasa por el punto $(s,t)$ en $\crc$, tenemos que $t=a \cdot s + 1 \implies a = \frac{t-1}{s}$; luego $y = \frac{t-1}{s} \cdot x + 1$.

Vamos a estudiar dónde se anula, ya que estamos haciendo la proyección sobre $y=0$, luego despejamos la $x$: $x = \frac{-s}{t-1} = \frac{s}{1-t}$. Y con esto ya hemos calculado la primera carta.

Ahora calculamos, de manera similar la segunda carta $(V_2, \alpha_2)$, pero esta vez proyectando desde el polo sur: partimos de la recta $y=a \cdot x+b$ y como pasa por $(s,t) \in \crc$, y en $(0,-1)$, tenemos que $y = \frac{t+1}{s} \cdot x - 1$. Proyectando sobre $y=0$, obtenemos que $x = \frac{s}{t+1}$.

\subsection{Compatibilidad de las cartas}
Veamos que son compatibles, es decir, que
\[ \appl{\alpha_2 \circ \alpha_1^{-1}}{\alpha_2(V_1 \cap V_2)}{\alpha_1(V_1 \cap V_2)} \]
está bien definida y es diferenciable.

En primer lugar, observamos que $V_1 \cap V_2 = (-\infty, 0) \cup (0, \infty)$.
En segundo lugar, para calcular la imagen de $\alpha_2 \circ \alpha_1^{-1}$, vamos a hacer la composición por pasos.

Empezamos por $x \xrightarrow{\alpha_1^{-1}} (s,t)$. Así que tenemos que calcular $(s,t)$ en función de x. Como tenemos que $x = \frac{s}{1-t}$, tomamos cuadrados $x^2 = \frac{s^2}{(1-t)^2}$, y como $(s,t) \in \crc$, tenemos que $s^2+t^2=1$.

Despejando $s^2$ y sustituyendo en $x^2$, obtenemos lo siguiente:
\[ x^2 = \frac{1 - t^2}{(1-t)^2} = \frac{(1-t)(1+t)}{(1-t)^2} = \frac{(1+t)}{(1-t)} \]

Despejamos la $t$:
\begin{gather*}
	x^2(1-t) = 1+t \iff x^2 - 1 = x^2t + t = t(x^2 + 1)\\
	t = \frac{x^2 - 1}{x^2 + 1}
\end{gather*}

Ahora nos falta despejar la s de $x = \frac{s}{1-t} \implies s = x (1-t)$. Sustituyendo la t y operando, obtenemos la expresión de $s$ en función de $x$:
\[ s = x \left(1 - \frac{x^2-1}{x^2+1}\right) = x \cdot \frac{x^2 + 1 - x^2 + 1}{x^2+1} = \frac{2x}{x^2+1} \]
Con lo que obtenemos finalmente la expresión de $(s,t)$ en función de $x$:
\[ (s,t) = \left(\frac{2x}{x^2+1}, \frac{x^2-1}{x^2+1}\right) \]

Y ahora calculamos la imagen de $(s,t)$ por $\alpha_2(s,t) = \frac{s}{1+t}$:
\[ \alpha_2\left(\frac{2x}{x^2+1},\frac{x^2-1}{x^2+1}\right) = \frac{\frac{2x}{x^2+1}}{1+\frac{x^2-1}{x^2+1}} = \dots = \frac{1}{x} \]

Luego
\[ \alpha_2 \circ \alpha_1^{-1}(x,y) = \frac{1}{x} \]


\chapter{Ejercicios}
% -*- root: ../GeometriaTopologia.tex -*-
\section{Hoja 1}

\begin{problem}[1] Considérese en $S^1$ las 2 cartas siguientes:

\[
U_1 = S^1 \setminus {(1,0)} \; \appl{φ_1^{-1}}{U_1}{(0,2π)} \text{ con } φ_1^{-1}(θ) = (\cos θ,\sin θ)
\]
\[
U_2 = S^1 \setminus {(-1,0)} \; \appl{φ_2^{-1}}{U_2}{(-π,π)} \text{ con } φ_2^{-1}(θ) = (\cos θ,\sin θ)
\]

\ppart Comprobar que estas 2 cartas definen un atlas en $S^1$.

\ppart Comprobar que el atlas anteriror es equivalente a los atlas definidos mediante la proyección estereográfica y la proyección a los ejes respectivamente.

\solution

\doneby{Dejuan}

\spart ¿$\{φ_1,φ_2\}$ es un atlas de $S^1$ si $\img(φ_1) \cup \img(φ_2) = S^1$? En caso de ser así, sólo hay que escribir las cosas con cuidado.

\spart 2 atlas son equivalentes si sus cartas son compatibles. Vamos a tomar $\{\Psi_i\}$ las cartas de las proyección estereográfica. La condición de compatibilidad es si $φ_1 \circ \Psi_1^{-1}$ es diferenciable.

Para más información, este ejercicio está esbozado ejemplificando la \ref{def::atlas_compatibles}

\end{problem}


\begin{problem}[5] Demostrar que son compactas las variedades

\ppart $\quot{ℝ^2}{ℤ^2}$

\ppart  $\projp^n$

\ppart $\projcp^n$


\solution

\doneby{Dejuan}
\spart
La relación de equivalencia que usamos para construir el espacio cociente nos dice que dos puntos $x,y ∈ ℝ^2$ pertenecen a la misma clase si existe un difeomorfismo $g ∈ ℤ^2$ tal que $g(x) = y$.
Es decir, si tenemos un par $(m,n) ∈ ℤ^2$ tal que $(x_1 + m, x_2 + n) = (y_1, y_2)$, entonces $x = (x_1, x_2)$ y $y=(y_1, y_2)$ están relacionados.

Con esta construcción, vemos que nuestro conjunto de clases de equivalencia son los puntos en $[0,1] × [0,1]$, con los bordes ``unidos'', el de arriba con el de abajo y el de la izquierda con el de la derecha. Tomando para $0<a<1$, el punto $(0,a) ~ (1,a)$, tomando $(n,m) = (1,0)$. \footnote{topológicamente estamos ante un toro (ver \fref{fig:ToroEspacioCociente})}

Esta variedad es compacta porque es ¿homeomorfa? al cuadrado $[0,1]×[0,1]\subset ℝ^2$, que es cerrado y acotado.

\spart $\projp^n$ podemos escribirlo como $\quot{\crc[n]}{\gen{J}}$, tomando y siendo $J(x_1,...,x_n) = (-x_1,...,-x_n)$.

\[\projp^n \simeq \quot{\crc[n]}{\gen{J}} \subset \real^n\]

Tenemos un conjunto cerrado y acotado en $\real^n$, con lo que es compacto.


\spart $\projcp^n = \quot{ℂ^{n+1}\setminus 0}{~}$ donde $\gx ~ \gy \dimplies ∃λ∈ℂ^{n+1} λ\gx = \gy$.

De esta manera, $\projcp^n \simeq \projp^{2n} \simeq \quot{\crc[2n]}{\gen{J}}$.

Tenemos un conjunto cerrado y acotado en $\real^{2n}$, con lo que es compacto.
\end{problem}

\begin{problem}[6]
Probar que si una variedad $M$ posee un atlas que consiste de dos cartas cuya intersección es conexa entonces $M$ es orientable. Deducir que $\crc[n]$ es orientable.

\solution

Según la \fref{def:VariedadOrientable}, una variedad es orientable si existe un atlas $A = \set{(U_i,φ_i)}_{i∈I}$ tal que $\abs{\Dif (φ_j○\inv{φ_i})} > 0$ para cualesquiera $j,i ∈ I$. Más generalmente, nos vale con que la diferencial de los difeomorfismos de cambio de carta tengan el mismo signo.

Por ser el cambio de carta un difeomorfismo, su diferencial es continua y además no se anula nunca. La única forma de que cambiase de signo es que la intersección de cartas no fuese conexa (podría saltar de valor entre componentes conexas), pero por hipótesis la intersección es conexa, luego la diferencial mantiene signo y por lo tanto la variedad es orientable.

En $\crc[n]$ se puede dar siempre un atlas con dos cartas con intersección conexa a través de la proyección estereográfica, luego es orientable.

\end{problem}

\begin{problem}[7]
Consideremos en $\real\times(-a,a)$ la acción de $ℤ$ definida por:

\[n\circ (x,λ) = (x+n,(-1)^nλ)\]

Ya hemos visto en clase que el cociente $M = \quot{ℝ×(-a,a)}{ℤ}$ es la banda de Möbius, que no es orientable.

\ppart ¿Qué variedad es el $\gor{M} = \quot{ℝ×(-a,a)}{2ℤ}$?

\ppart Demostrar que $\gor{M}$ es un cubrimiento doble de $M$ (Es decir, encontrar una aplicación diferenciable y suprayectiva $\gor{M}\to M$ tal que cada punto tiene 2 preimágenes).

\solution

\doneby{Dejuan}

Si no estamos convencidos de que el cociente $M$ sea la banda, consultar \ref{Mobius}.

No tengo muy claro si $2ℤ = \{ 2n : n∈ℤ\}$ o $2ℤ = \{0,1\}$. He tomado la primera oción.

\spart $\gor{M} = \quot{ℝ×(-a,a)}{2ℤ}$ es el cociente dado por la relación:

\[
(x,λ) ∈ ℝ×(-a,a) \sim (x',λ') \dimplies \left\{\begin{array}{cc} x' = x+n & n∈2ℤ\\λ' = (-1)^nλ & n∈2ℤ \end{array}\right.
\]

Lo primero de lo que nos damos cuenta es que $(-1)^n = 1, ∀n∈2ℤ$.

Vamos a estudiar ahora la primera componente. Vemos que $[x]∈[0,2)$, ya que $2 \sim 0$.

Con estos 2 detalles, tiene pinta de que $\gor{M}$ es el cuadrado $[-a,a]×[0,2]$ con los bordes identificados en el mismo sentido. Es decir:

\begin{figure}[hbtp]
\centering
\begin{tikzpicture}
\fill[blue!20!white] (0, -0.8) rectangle (2, 0.8);
\draw (-1.4,0) -- (3,0);
\draw (0,-1.4) -- (0,1.4);s

\node[hnlin, label = {left:$a$}] at (0, 0.8) {};
\node[hnlin, label = {left:$-a$}] at (0, -0.8) {};
\node[hnlin, label = {below right:$(2,0)$}] at (2,0) {};

\draw[blue, thick, directed]  (2, 0.8) --  (2, -0.8);
\draw[blue, thick, directed] (0, 0.8) -- (0,-0.8);
\end{tikzpicture}
\end{figure}

Estamos ante un cilindro de altura $2a$ sin tapas.\footnote{Normalmente, los cilindros sin tapa se llaman tubos.}


\spart

Buscamos $g$ tal que $∀(x',λ') ∈\gor{M}, ∃(x_1,λ_1),(x_2,λ_2)∈M$ con $(x_1,λ_1)≠(x_2,λ_2)$ y que se cumpla $g(x_1,λ_1) = g(x_2,λ_2) = (x',λ')$

Tomando la proyección de $\gor{M}$ en $M$, tenemos una aplicación diferenciable (por ser identidades o cambios de signo o desplazamiento) en el que cada punto de $M$ tiene 2 preimágenes:

\[
	\left.
		\begin{array}{c}
			(x,λ) ∈ M \\
			(x',λ) ∈ \gor{M}\\
			(x+1',-λ) ∈ \gor{M}
		\end{array}
	\right\}
	\to π^{-1}(x'+1,(-1)^1(-λ)) = π^{-1}(x',(-1)^0λ) = (x,λ)
\]

\begin{figure}[hbtp]
\centering
\begin{tikzpicture}
\fill[pattern=north east lines, pattern color=blue!70!white] (0, -0.8) rectangle (2, 0.8);
\fill[pattern=north west lines, pattern color=red!50!white]  (0, -0.8) rectangle (1, 0.8);

\draw (-1.4,0) -- (3,0);
\draw (0,-1.4) -- (0,1.4);

\node[hnlin, label = {left:$a$}] at (0, 0.8) {};
\node[hnlin, label = {left:$-a$}] at (0, -0.8) {};
\node[hnlin, label = {below right:$(2,0)$}] at (2,0) {};

\draw[blue, thick, directed]	(2, 0.8) --  (2, -0.8);
\draw[blue!50!red, thick, directed]	(0, 0.8) -- (0,-0.8);
\draw[red,	thick, directed]	(1,-0.8) -- (1,0.8);

\node[label = {left:$3$}] at (1.8, 0.5) {};
\node[label = {left:$2$}] at (1.8, -0.5) {};
\node[label = {left:$1$}] at (0.8, -0.5) {};
\node[label = {left:$4$}] at (0.8, 0.5) {};

\end{tikzpicture}
\caption{En rojo la variedad $M$ y en azul, $\gor{M}$}
\label{ej:1.7}
\end{figure}

Con la aplicación $π$ descrita anteriormente y representada en \fref{ej:1.7}, vemos que se corresponden los cuadrantes $1,3$ de $\gor{M}$ con el cuadrante $1$ de $M$ y los cuadrantes $2,4$ de $\gor{M}$ con el cuadrante $2$ de $M$. De esta manera, todo punto de $M$ tiene 2 preimágenes en $\gor{M}$.


\end{problem}

\begin{problem}[8] Sean $M$ y $N$ dos variedades orientables, y sea $\appl{f}{M}{N}$ una aplicación diferenciable.

\ppart Define el concepto de morfismo que preserva la orientación y pon un ejemplo de uno (y de otro que no lo sea).

\ppart Demostrar que una variedad cociente $\quot{M}{G}$ en la que $M$ es orientable y los elementos de $G$ preservan la orientación es orientable.

\ppart Deducir que los espacios proyectivos de dimensión impar son orientables.

\solution

\spart

La definición de orientación de Geometría Diferencial \citep[Def. IV.7]{ApuntesGeoDif} era bastante cómoda, ya que sólo dependía de la existencia de una $n$-forma de volumen que no se anulase. La cuestión es que dudo bastante que podamos usar eso aquí, así que toca ir a la definición fea, dependiente de las cartas.

\begin{defn}[Aplicación\IS compatible con la orientación] Sean $M$ y $N$ dos variedades orientables, y sea $\appl{f}{M}{N}$ una aplicación diferenciable. Diremos que $f$ es compatible con la orientación (o que preserva la orientación) si y sólo si, para dos cartas cualesquiera $(U_i, φ_i)$ y $(U_j, φ_j)$ de $M$ y $N$ respectivamente, el jacobiano dado por \[ \Dif (φ_i ○ f ○ \inv{φ_j})\] es positivo en la región en la que esté definido.
\end{defn}

Los ejemplos no se me ocurren ahora.

\spart

\end{problem}

\section{Hoja 2}

\begin{problem}[1]

Demostrar que $\crc[3]$ es un grupo de Lie.

\solution

$\crc[3]$ es un grupo de Lie si admite una estructura de grupo. $\vec{u} \in \crc[3] \implies \vec{u} = a + bi + cj + dk, \norm{\vec{u}} = 1$, siendo $i,j,k$ las raíces cuartas de la unidad.

Tenemos que comprobar que

\[\vec{x},\vec{y}\in\crc[3] \implies \vec{x}·\vec{y} \in\crc[3]\]

Para ello comprobamos:

\[
(a+bi+cj+dk)(a'+b'i+c'j+d'k) = ... = 1 \to \norm{\vec{x}·\vec{y}} = 1
\]
\end{problem}

\begin{problem}[2]
¿Qué variedad es $M^2\#\crc[2]$? ¿Y en general $M^n\#\crc[n]$?
\solution

\[M^n\#\crc[n] = M^n\]

Porque al sumar conexamente con esferas, obtenemos variedades homeomorfas.

\end{problem}

\begin{problem}[3]

\ppart Demostrar que un abierto de una variedad orientable es él mismo una variedad orientable.

\ppart Deducir de ahí que $\projp^n$ y la superficie $\mathbb{K}$ de Klein son no
orientables.

\solution

\spart Supongamos que tenemos una variedad $M$ orientable que contiene un abierto $A$ que no es orientable. Si $A$ no es orientable, entonces existen cartas en un atlas tal que $|\det(φ_j\circ φ_i)| ≤ 0$.
Bastaría con incluir estas cartas compatiblemente en el atlas de $M$ para obtener que no es orientable, lo que, por hipótesis, no es cierto.

\spart Si los abiertos de variedades orientables son en sí mismos variedades orientables, ni$\projp^n$ ni $\mathbb{K}$ pueden ser variedades orientables puesto que contienen un abierto que es la cinta de Mobius.
\end{problem}

\begin{problem}[4] Exhibir una triangulación del toro $\torus = \crc[1] × \crc[1] = \quot{ℝ^2}{ℤ^2}$ que pruebe que $χ(\torus) = 0$.

\solution

Consultar la \fref{fig:TriangulacionToroPlano}.

Para ver que $\chi(\mathbb{T}_2) = 0$, contamos en la triangulación:

\[V-A+C = 9 - 27 + 18 = 0\]

\end{problem}

\begin{problem}[5]
Demostrar
\[\chi(M_1\#M_2) = \chi(M_1) + \chi(M_2) - 2\]

Deducir $\chi(M^2)$ para cualquier superficie orientable.

\solution

La unión de la suma conexa superpone 2 circunferencias, que podríamos tomar como homeomorfas a un triángulo, con lo que estamos quitando un triángulo.

La $\chi(M_1\#M_2) = \chi(M_1) + \chi(M_2) - cosas$, donde esas cosas que hay que quitar son 3 vértices repetidos (los 3 del triángulo), 3 aristas y 2 caras (una de cada variedad), es decir:

\[\chi(M_1\#M_2) = \chi(M_1) + \chi(M_2) - (3-3+2) = \chi(M_1) + \chi(M_2) - 2\]

Cualquier superficie orientable $M^2$ es, según el \fref{thm:ClasificacionSuperficies}, una suma conexa de toros. Dado que la característica del toro es $0$, las superficies orientables tendrán característica par negativa.
\end{problem}

\begin{problem}[6] Demostrar que $\chi(\projp^2) = 1$ y deducir $\chi(M^2)$ para cualquier superficie no orientable.

\solution

De nuevo según el \fref{thm:ClasificacionSuperficies}, cualquier superficie compacta no orientable es una suma de planos proyectivos, luego su característica será también negativa.

\end{problem}

\section{Hoja 3}

\begin{problem}[6]

\ppart Comprobar que si $M$ y $N$ son dos variedades la proyección obvia $M × N \mapsto N$ es una submersión.

\ppart Comprobar que la fibración de Hopf es una submersión. ¿Cuáles son sus fibras?

\solution

\spart

Según la \fref{def:Submersion}, simplemente tenemos que ver que la proyección es sobreyectiva (obvio) y que su diferencial es sobreyectiva, esto es, que tiene rango máximo (que también tiene, ya que es la identidad).

\spart

Ver \fref{sec:FibracionHopf}.

\end{problem}

\section{Hoja 4}

\begin{problem}[4] Se considera la forma diferencial $ω = x \dif y - y \dif x ∈ Ω(ℝ^2)$. Demostrar que si $\appl{i}{\crc[1]}{ℝ^2}$ denota la aplicación inclusión entonces $i^*(ω) = \dif θ$. Habrá que ver que $i^*ω (∂θ) = 1$.

\solution

\doneby{Guille}

La aplicación inclusión vendrá dada por \[ i(θ) = (\cos θ, \sin θ) \], y entonces \[ i^*ω = \cos θ · (\cos θ \dif θ) - \sin θ (- \sin θ \dif θ) = \dif θ \]

\end{problem}


\begin{problem} Se consideran las formas diferenciales $ω = \sin x \dif y ∧ \dif z$ y $η = \cos x \dif t$, ambas en $Ω(ℝ^4)$. Calcular:

\ppart $\dif ω,\,\difη,\, (ω∧η),\, \dif(ω∧η), \, \difω ∧ η$

\ppart $\dif \dif ω,\, \dif \dif η, \, \dif \dif (ω ∧ η)$.

\ppart $f^*\dif ω, \dif f^* ω$, donde $f(x,y,z,t) = (y, x^2, z + t, t-z)$

\ppart $f^*(ω∧η),\, f^*ω ∧ f^*η$ donde $f$ es la aplicación del apartado anterior.

\solution

\spart

\begin{gather*}
\dif ω = \cos x \dif x ∧ \dif y ∧ \dif z \\
\dif η = - \sin x \dif x ∧ \dif t \\
ω ∧ η = \sin x \cos x \dif y ∧ \dif z ∧ \dif t \\
\dif(ω∧η) = (\cos^2 x - \sin^2 x) \dif x ∧ \dif y ∧ \dif z ∧ \dif t \\
\dif ω ∧ η = 0 \\
ω ∧ \dif η = 0
\end{gather*}

\spart

Todas cero por ser diferenciales de diferenciales.

\spart

\begin{gather*}
f^* \dif ω = \cos y · (\dif y) ∧ (2x \dif x) ∧ (\dif z + \dif t) \\
f^* ω = \sin y (2x \dif x ) ∧ (\dif z + \dif t) \\
\dif f^* ω = \cos y ·2x \dif y ∧ \dif x ∧ (\dif z + \dif t) \\
f^* \dif η = - \sin y (\dif y) ∧ (\dif t - \dif z) \\
f^* η = \cos y (\dif t - \dif z) \\
\dif f^* η = -\sin y \dif y ∧ (\dif t - \dif z)
\end{gather*}

\spart

\begin{align*}
f^*(ω∧η) &= \sin y \cos y (2x \dif x) ∧ (\dif z + \dif t) ∧ (\dif t - \dif z) = \\
	&=2x \sin y \cos y \dif x ∧ \dif z ∧ \dif t - 2x \sin y \cos y \dif x ∧ \dif t ∧ \dif z = \\
	&= 4x \sin y \cos y \dif x ∧ \dif z ∧ \dif t \\
f^*ω ∧ f^* η &= \left(\sin y (2x \dif x ) ∧ (\dif z + \dif t)\right) ∧ \left(\cos y (\dif t - \dif z)\right ) = \\
	&= 4x \sin y \cos y \dif x ∧ \dif z ∧ \dif t
\end{align*}

\end{problem}

\section{Hoja 5}

\begin{problem}[1]
Sea $\appl{f}{M}{N}$ una aplicación diferenciable. Probar que $\appl{f^\ast}{H^\ast(N)}{H^\ast(M)}$ es un homomorfismo de anillos y que si f es un difeomorfismo este homorfismo es un isomorfismo.
\solution

$f^\ast$ es un homomorfismo de anillos porque $f^\ast(a+ b) = f^\ast(a) + f^\ast(b)$ y $f^\ast(a\wedge b) = f^\ast(a) \wedge f^\ast(b)$ (ambas igualdades por definición de pullback).

Si $f$ fuera difeomorfismo, podemos definir $(f^\ast)^{-1}$ siendo $f^\ast$ biyectiva y continua, es decir, es un isomorfismo.
\end{problem}



\begin{problem}[3]
Demostrar que $\crc[n]$ es un retracto de deformación de $\real^{n+1} \setminus {0}$
\solution
Por la proyección estereográfica
\end{problem}

\begin{problem}[4]
 En $\real^2 \setminus {0}$ se consideran las 1-formas


\[ω1 = xdy − ydx\]

\[ω2 = \frac{xdy − ydx}{x^2+y^2}\]

Investigar si alguna de ellas genera  $H^1(\real^2 \setminus {0})$.

\solution

$H^k$ es el espacio de las k-formas cerradas que no son exactas.
%
Sabemos que tiene dimensión 1 ya que $h^1(\crc[1])=1$.
%
Además, por el ejercicio 3 y \ref{crl:CohomHomotopia}: $H^1(\crc[n]) = H^1(\real^{n+1} \setminus 0) = 1$

Si alguna de estas 1-formas fuera cerrada y no exacta, estaría en $H^1(\real^2\setminus {0})$ y sería generador ya que este e.v. tiene dimensión 1.

Vamos a verlo:

\[\dif ω_1 = 2\dif x\wedge \dif y\]

\[\dif ω_2 = ... = 0\]

$ω_2$ es cerrada.
%
Para ver si es exacta, tomamos $\appl{i}{\theta}{(\cos θ,\sin θ)}$ y hacemos:

\[
i^\ast ω_2 = ... = \dif θ
\]

Entonces tendríamos que $ω$ es exacta, porque tomamos: $θ = i^\ast µ$ para alguna $µ$ que debería existir (espero) y diferenciando:

\[
i^\ast ω_2 = \dif θ = \dif (i^\ast µ) = i^\ast(\dif µ)
\]

Como ninguna de las dos es cerrada y no exacta, no están en $H^1(\real^2\setminus{0})$ luego no pueden generarlo.

\end{problem}


\begin{problem}[5]

Dada una sucesión exacta \[ 0 \to V_0 \xrightarrow{h_0} V_1 \to \dotsb \to V_{n-1} \xrightarrow{h_{n-1}} V_{n} \to 0 \] Demostrar que la suma alternada de dimensiones de los espacios vectoriales es cero, es decir:  \[ \sum_{i=0}^n (-1)^i \dim V_i = 0\]


\solution

Consultar \fref{prop:sumaAlternada0}

\end{problem}

\begin{problem}[6]
Sea M una variedad compacta. Construir una partición de la unidad de M subordinada al cubrimiento por abiertos de M que se definió en clase.
\solution
Consultar \fref{thm:SucesionMayerVietoris}
\end{problem}

\begin{problem}[7]

\ppart Calcular los grupos de cohomología de $\crc[n]$ razonando por inducción a partir de los de $\crc[1]$.
\ppart Deducir que si $n ≠ m$ entonces $ℝ^n$ no es homeomorfo a $ℝ^m$.
\ppart Deducir también el Teorema del punto fijo de Brower.

\solution

\spart
\label{ej:Hoja7:CohomologiaSN}

En \eqref{eq:CohomologiaS1} y \eqref{eq:CohomologiaS2} veíamos la cohomología de $\crc[1]$ y $\crc[2]$ respectivamente, así que la base de inducción ya la tenemos. Vamos ahora a por el resto, y lo haremos aplicando Mayer-Vietoris con los siguientes abiertos:
\begin{align*}
U &= \crc[n] \setminus \set{N} \\
V &= \crc[n] \setminus \set{S}
\end{align*} donde $N, S$ son dos puntos distintos de la circunferencia (no hace falta que sean antipodales. Esto nos da la intersección $U ∩ V \simeq \crc[n-1]$.

Construimos la cadena de Mayer-Vietoris (\fref{fig:MayerVietorisSN}), de donde sacamos rápidamente que $h^1(\crc[n]) = 0$ (tenemos $1 - 2 + 1 + h^1(\crc[n]) = 0$), que $h^k(\crc[n]) = 0$ cuando $1 < k < n$ (la cadena es exacta con sólo $H^k(\crc[n])$) en el medio y, para el último caso, que $h^n(\crc[n]) = 1$, que es efectivamente lo que teníamos que sacar \eqref{eq:CohomologiaSN}.

\begin{figure}[hbtp]
\centering
\tikzexternaldisable
\begin{tikzcd}[row sep = 15pt]
0 \rar
	& \overbracket{H^0(\crc[n])}^{\dim 1} \rar
	& \overbracket{H^0(U) \oplus H^0(V)}^{\dim 2} \rar
	& \overbracket{H^0(U ∩ V)}^{\dim 1} \arrow[snake]{dll}{}
	\\
	& H^1(\crc[n]) \rar
	& \overbracket{H^1(U) \oplus H^1(V)}^{\dim 0} \rar
	& \overbracket{H^1(U ∩ V)}^{\dim 0} \arrow[snake]{dll}{}
	\\
	& H^2(\crc[n]) \rar
	& \overbracket{H^2(U) \oplus H^2(V)}^{\dim 0} \rar
	& \overbracket{H^2(U ∩ V)}^{\dim 0} \arrow[snake]{dll}{}
	\\
	& \phantom{H^1(\crc[n])}
	& \dotsb
	& \phantom{H^1(U ∩ V)} \arrow[snake]{dll}{}
	\\
	& H^{n-1}(\crc[n]) \rar
	& \overbracket{H^{n-1}(U) \oplus H^{n-1}(V)}^{\dim 0} \rar
	& \overbracket{H^{n-1}(U ∩ V)}^{\dim 1} \arrow[snake]{dll}{}
	\\
	& H^{n}(\crc[n]) \rar
	& \overbracket{H^{n}(U) \oplus H^{n}(V)}^{\dim 0} \rar
	& H^{n}(U ∩ V) \rar & 0 \\
\end{tikzcd}
\tikzexternalenable
\caption{Cadena de Mayer-Vietoris para $\crc[n]$, tomando $U = \crc[n] \setminus \set{N}$, $V = \crc[n] \setminus \set{S}$ y $U ∩ V = \crc[n-1]$.}
\label{fig:MayerVietorisSN}
\end{figure}

\spart

Podemos quitarle un punto cualquiera a ambos y ver que salen dos cosas que no son homeomorfas ($ℝ^n \setminus \set{p} \simeq \crc[n] \not\cong \crc[m] \simeq ℝ^m \setminus \set{q}$).

\doneby{Dejuan} $h^n(\real^n) = 1 ≠ h^n(\real^m) \to \real^n\not\cong \real^m$ 

\spart

Vamos a ver qué nos dice el teorema.

\begin{theorem}[Teorema\IS del punto fijo de Brouwer] \label{thm:PuntoFijoBrower} Sea $\appl{f}{\disc}{\disc}$ una aplicación continua. del disco cerrado en $ℝ^n$ en sí mismo. Entonces $f$ tiene un punto fijo.
\end{theorem}

\end{problem}


\begin{problem}[9]

\ppart Calcular los grupos de cohomología de $\real^2$ menos n puntos.
\ppart Y la de $\real^3$ menos un punto y una recta.
\ppart Y la de un toro menos un punto.

\solution
\doneby{Dejuan}

No tengo ni idea de calcular los grupos de cohomologías, asique me contentaré con calcular las dimensiones. 
%
Aunque supongo que la única posibilidad de dimensión 1 es $\real$, ya que $H^n(\real^n) = \real$ con $h^1(\real^n) = 1$.

\spart 
Tomamos $n=2$ para ejemplificar.
%
$U = \real^2\setminus\{p_1\}$ y $V=\real^2\setminus\{p_2\}$ de tal manera que $U\cap V = \real^2\setminus \{p_1,p_2\}$

\[
	\underbrace{h^0(\real^2)}_{1} \to 
	\underbrace{h^0(U)\oplus h^0(V)}_{1+1=2} \to 
	\underbrace{h^0(U\cap V)}_{1} \to 
	\underbrace{h^1(\real^2)}_{0} \to 
	\underbrace{h^1(U)\oplus h^1(V)}_{1+1=2} \to 
	\underbrace{h^1(U\cap V)}_{x} \to 
	\underbrace{h^2 (\real^2)}_{0}
\]
Por la suma alternada, $0-2+x-0=0 \to x=2$.


Tomamos $n=3$:
%
$U = \real^2\setminus\{p_1,p_2\}$ y $V=\real^2\setminus\{p_3\}$ de tal manera que $U\cap V = \real^2\setminus \{p_1,p_2,p_3\}$

\[
	\underbrace{h^0(\real^2)}_{1} \to 
	\underbrace{h^0(U)\oplus h^0(V)}_{1+1=2} \to 
	\underbrace{h^0(U\cap V)}_{1} \to 
	\underbrace{h^1(\real^2)}_{0} \to 
	\underbrace{h^1(U)\oplus h^1(V)}_{2+1=3} \to 
	\underbrace{h^1(U\cap V)}_{x} \to 
	\underbrace{h^2 (\real^2)}_{0}
\]

Por la suma alternada, $0-3+x-0=0 \to x=3$.

Por inducción, podemos probar que $h^1(\real^2 \setminus\{p_1,...,p_n\}) = n$ (tomando $h^0(\real^2 \setminus\{p_1,...,p_n\})$ por número de componentes conexas).

\spart Suponemos que el punto no pertenece a la recta.
%
Como son 2 componentes conexas, podemos trabajar por separado.


\end{problem}


\section{Hoja 6}

\begin{problem} Sean $\appl{π_1, π_2}{M × N}{M,N}$ las dos proyecciones obias. Probar que si $ω$ y η son dos formas cerradas en $M$ y $N$ respectivamente (y por tanto definen clases de cohomología en $M$ y $N$) entonces $π_1^* ω ∧ π_2^* η$ es una forma cerrada de $M × N$ y por tanto define una clase de cohomología en $M × N$.

\solution

\doneby{Guille}

Simplemente usando las propiedades de la diferencial (\fref{prop:PropsDiferencial}) sale: \begin{align*}
\dif (π_1^* ω ∧ π_2^* η)
	&= \dif π_1^* ω ∧ π_2^* η + (-1)^{\deg π_1^* ω} π_1^* ω ∧ \dif π_2^* η = \\
	&= π_1^*(\dif ω ) ∧ π_2^* η + (-1)^{\deg π_1^* ω} π_1^* ω ∧ π_2^* (\dif η) = \\
	&= 0 + 0 = 0
\end{align*}

\end{problem}

\begin{problem} Calcular la cohomología de las siguientes variedades:

\ppart $\torus = \crc[1] × \crc[1]$.
\ppart $\crc[n] × \crc[m]$.
\ppart $\crc[3] × T_g$.
\ppart $\crc[1] × \crc[1] × \crc[1]$.

\solution

\doneby{Guille}

\spart

Ya esta hecho en la \fref{sec:CaracteristicaEulerPoincare} usando el \nref{thm:Kunneth} (salen grupos de dimensión 1, 2, 1 en ese orden).

\spart

Aplicamos de nuevo el \nref{thm:Kunneth}, y basándome en la cohomología de $\crc[n]$ \eqref{eq:CohomologiaSN} me voy a tirar el siguiente triple: \[ h^k(\crc[n] × \crc[m]) = \begin{cases}
1 & k = 0 \\
1 & k = n, m \\
1 & k = n + m \\
0 & \text{otro caso}
\end{cases}\]

Si $n = m$, entonces el caso $k = n = m$ sale de dimensión 2.

\spart

No sé qué es $T_g$.

\spart

Aplicamos Mayer-Vietoris con los siguientes conjuntos: \begin{align*}
U &= \crc × \crc × (\crc \setminus p) \\
V &= \set{0} × \set{0} × \bola_ε(p)
\end{align*} considerando $\bola_ε(p)$ como un cacho de la circunferencia. Luego lo completo

\end{problem}

\begin{problem}[4] Probar que $χ(M × N) = χ(M) χ(N)$. ¿Y desde el punto de vista de triangulaciones?

\solution

\doneby{Guille}

Sabemos por una parte que la \nlref{def:CaracteristicaEulerPoincare} es \[ χ(M) = \sum_{k=0}^m (-1)^k h^k(M) \] y por el \nref{thm:Kunneth} que la dimensión de la variedad producto es \[ h^k(M × N) = \sum_{p + q = k} h^p(M) h^q(N) \] luego \[ χ(M ×N) = \sum_{k = 0}^{m + n} (-1)^k\sum_{ p + q = k} h^p(M) h^q(N) \]

Haciendo el producto por el otro lado: \[ χ(M) χ(N) = \sum_{p = 0}^m (-1)^p h^p(M) · \sum_{q = 0}^n (-1)^q h^q(N) = \sum_{p = 0}^m \sum_{q=0}^n (-1)^{p+q} h^p(M) h^q(N) \] y parece que es lo mismo.

Por triangulaciones ni idea.

\end{problem}

\begin{problem}[7] Sea $M = ℝ^2 \setminus \set{p_1, \dotsc, p_n}$ con $p_k = (a_k, b_k)$.

\ppart Probar que \[ ω_i = \frac{(x - a_i) \dif y - (y - b_i) \dif x}{(x-a_i)^2 + (y-b_i)^2} \] define un elemento no nulo de $H^1(M)$.

\ppart Probar que si los puntos $P_i$ son distintos las clases de cohomología inducidas por $n$ diferenciales anteriores son linealmente independientes.

\ppart  Demostrar que $ω = P(x,y) \dif x + Q(x,y) \dif y ∈ Ω^1(M)$ es cerrada si y sólo si el campo $F = P∂_x + Q ∂_y$ es irrotacional (i.e., satisface Green) y que es exacta si y sólo si $F$ es conservativo.

\ppart Demostrar que $F$ es un campo conservativo si y sólo si \[ \int_{γ_i} F = \int_{γ_i} ω = 0\] donde $γ_i$ es una circunferencia centrada en $p_i$ que no contiene a ningún otro $p_j$.

\solution

\doneby{Guille}

\spart

Primero vemos que $ω_i ∈ H^1(M)$, esto es, que $\dif ω_i = 0$. Eso sale rápido:
\[ \dif ω_i = \frac{\dif x ∧ \dif y - \dif y ∧ \dif x}{(x-a_i)^2 + (y-b_i)^2} = 0\]

Para ver que no es nula, tenemos que ver que no es exacta. Hagámoslo por reducción al absurdo, y supongamos que existe una $0$-forma η tal que $\dif η = ω$. En ese caso podemos aplicar el teorema de Stokes, que nos dice que $\int_{∂Ω} η = \int_Ω \dif η$. En particular, podemos tomar $Ω = \crc$ con la circunferencia centrada en $p_i$, luego $∂\crc = ∅$, así que tenemos que tener $\int_{\crc} ω = 0$ si fuese exacta. Ahora veremos que esa integral en realidad no es nula.

Para integrar, hacemos el cambio a polares, que es perfectamente válido en un entorno de $p_i$: \[ \begin{cases} x = r \cos θ + a_i \\ y = r \sin θ + b_i \end{cases} \implies
\begin{cases}
\dif x = \cos θ \dif r - r \sin θ \dif θ \\
\dif y = \sin θ \dif r + r \cos θ \dif θ \\
\end{cases}\]

Sustituyendo: \[ \tilde{ω}_i =  \frac{r \cos θ (\sin θ \dif r + r \cos θ \dif θ) - r \sin θ (\cos θ \dif r - r \sin θ \dif θ)}{(r \cos θ)^2 + (r \sin θ)^2} = \frac{r^2 \cos^2 θ \dif θ + r^2 \sin^2 θ \dif θ}{r^2} = \dif θ \]

Cuidado que aunque parezca que nos ha salido exacta en realidad no es así, principalmente porque $θ(x,y)$ no es una función que podamos definir (el cambio de coordenadas no es inyectivo en $θ$). De hecho, si hacemos la integral que decíamos antes, lo que nos va a quedar es que \[ \int_{\crc} ω_i = \int_0^{2π} \dif θ = 2π ≠ 0 \] por lo que efectivamente $ω_i$ no es exacta.

\spart

Esto no sé cómo sacarlo.

\spart

Supongo que Gabino se refiere a que $P_y = Q_x$ con lo de ``satisface Green'', lo que efectivamente da que $\dif ω = 0$. Lo de que es conservativo implica que sea el gradiente de una función (esto es, $\Dif f = F$) y efectivamente da que es exacta.

\spart

\end{problem}


\bibliography{../Apuntes}{}
\printindex
\end{document}
