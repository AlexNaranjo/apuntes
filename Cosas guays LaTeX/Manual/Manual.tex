\documentclass[nochap]{apuntes}

\usepackage{longtable}

\author{Guillermo Julián Moreno}
\date{\today}
\title{Manual paquete extendido \LaTeX}


\begin{document}

\pagestyle{plain}
\maketitle

\section{Formato y maquetación}

LaTeX se encarga automáticamente de la maquetación del texto, así que por eso no hay que preocuparse. Sólo hay que acostumbrarse a dejar una línea en blanco entre párrafos para que LaTeX los reconozca como tales. Por ejemplo, si ponemos

\begin{verbatim}
Un párrafo.
Otro párrafo.
\end{verbatim}

acabarían los dos párrafos en el mismo. Tenemos que separarlos así:


\begin{verbatim}
Un párrafo.

Otro párrafo.
\end{verbatim}

Si queremos forzar un salto de línea, podemos usar \verb|\\|, \verb|\newline| o \verb|\pbreak|.

\subsection{Modo matemático}

En modo matemático hay que tener cuidado porque el formato es distinto. Por ejemplo, tenemos que tener cuidado si queremos poner texto y no sólo símbolos, ya sea para explicaciones u operaciones. Veamos que

\begin{center}\verb|$arctg y \cdot f sin x = 1 \implies f es inyectiva$| \end{center}

nos deja algo como esto: \[ arctg y \cdot f  sin x = 1 \implies f es inyectiva \]

Si queremos que todo salga bien, primero tenemos que usar los operadores en las funciones. Por ejemplo, \verb|\sin| en lugar de \texttt{sin} a secas, para que la separación esté bien (si no existe el comando que sea, se puede usar \verb|\mop{cmd}| para que salga bien). Y si queremos poner texto, lo ponemos con \verb|\text{}|. Siguiendo con el ejemplo, tendríamos

\begin{center}
\verb|$\mop{arctg} y \cdot f \sin x = 1 \implies \text{f es inyectiva}$|
\end{center}

\[ \mop{arctg} y \cdot f \sin x = 1 \implies \text{f es inyectiva} \]

Si además queremos modificar el espaciado, podemos usar alguno de los comandos de espaciado (\verb|\ \, \; \quad \qquad|).

Además, hay diferencias entre usar el modo \textit{inline} y las ecuaciones en su párrafo (ya sea con \verb|\[ \]| o con otros entornos matemáticos). Por ejemplo, un sumatorio entre dólares sale así: $\sum_{i=1}^{k}$. En su propio entorno, sale así:

\[ \sum_{i=1}^{k} \]

La solución en estos casos es usar \verb|\displaystyle|. Por ejemplo, \verb|$\displaystyle\sum_{i=1}^{k}$| saca $\displaystyle\sum_{i=1}^{k}$.

\section{Entornos}
\subsection{Teorema y definición}

Los teoremas del paquete heredan de los teoremas de \texttt{amsthm}, pero además añaden un índice automático de teoremas que se puede imprimir en el documento con \verb|\printtheorems|. También se cargan automáticamente en el glosario de términos. El uso normal es el siguiente

\begin{verbatim}
\begin{theorem}[Teorema de las gallinas cluecas]
Teorema.
\end{theorem}
\end{verbatim}

\begin{theorem}[Teorema de las gallinas cluecas]
Teorema.
\end{theorem}

El título se carga automáticamente como una entrada para el índice. Si, por lo que sea, queremos separar los términos (por ejemplo, para que en este caso aparezca primero \textit{Teorema} y luego, debajo, \textit{de las gallinas cluecas}, ved la última página), no podemos usar la exclamación del comando \verb|\index{}| porque aparecería en el título. En su lugar, podemos usar el comando \verb|\IS|, que actúa como una separación para el índice pero que no aparece en el título.

\begin{verbatim}
\begin{theorem}[Teorema\IS de las gallinas cluecas]
Teorema.
\end{theorem}
\end{verbatim}

\begin{theorem}[Teorema\IS de las gallinas cluecas]
Teorema.
\end{theorem}

Si, por lo que sea, queremos especificar nosotros el índice, podemos añadir un segundo argumento.

\begin{verbatim}
\begin{theorem}[Teorema de las gallinas cluecas][Mi entrada]
Teorema.
\end{theorem}
\end{verbatim}

\begin{theorem}[Teorema\IS de las gallinas cluecas][Mi entrada]
Teorema.
\end{theorem}

Por otra parte tenemos el entorno \texttt{defn} para definiciones, que funciona exactamente igual que \texttt{theorem}:

\begin{verbatim}
\begin{defn}[Título de la definición][(opcional) entrada para el índice]
\end{defn}
\end{verbatim}

\begin{defn}[Título de la definición][(opcional) entrada para el índice]
\end{defn}

\subsection{Auxiliares para los teoremas}

Hay definidos varios entornos similares a teoremas. Todos ellos se pueden titular poniendo entre corchetes el título.

\begin{verbatim}
\begin{lemma}[Lema de la patata]
\end{lemma}
\end{verbatim}

\begin{lemma}[Lema de la patata]
\end{lemma}

\begin{verbatim}
\begin{corol}
\end{corol}
\end{verbatim}

\begin{corol}
\end{corol}

\begin{verbatim}
\begin{prop}
\end{prop}
\end{verbatim}

\begin{prop}
\end{prop}

\begin{verbatim}
\begin{axiom}
\end{axiom}
\end{verbatim}

\begin{axiom}
\end{axiom}

\begin{verbatim}
\begin{proof}
Probado queda.

Al final agrega un cuadradito de QED.
\end{proof}
\end{verbatim}

\begin{proof}
Probado queda.

Al final agrega un cuadradito de QED.
\end{proof}

\begin{verbatim}
\begin{op}{Nombre de operación}
x = 3 + 1
\end{op}
\end{verbatim}

\begin{op}{Nombre de operación}
x = 3 + \deriv{f}{x}
\end{op}

El entorno \texttt{op} incluye el modo matemático directamente, y el nombre de operación es obligatorio.

\subsection{Ejercicios}

Hay dos entornos para incluir ejercicios o ejemplos en los archivos. El primero es \texttt{example}, que acepta un argumento opcional como nombre del ejemplo.

\begin{verbatim}
\begin{example}[Titulo]
Un ejemplo sobre cosas matemáticas.

\[ 3 = 3 \]

Fin del ejemplo.
\end{example}
\end{verbatim}


\begin{example}[Titulo]
Un ejemplo sobre cosas matemáticas.

\[ 3 = 3 \]

Fin del ejemplo.
\end{example}

El otro sería \texttt{problem}, que viene con bastantes cosillas. Acepta un argumento opcional, que es el número de ejercicio. Si no se pasa este argumento, la numeración es automática y se genera de la forma \textit{Sección.Ejercicio}, de tal forma que se pueda referenciar luego con etiquetas. El comando \verb|\solution| separa enunciado y solución, y los comandos \verb|\ppart| y \verb|spart| sirven para separar los diferentes apartados (la numeración es automática). Ejemplo:

\begin{verbatim}
\begin{problem}[3]
Sea $x$ una cosa, entonces calcula:

\ppart Calcula la divergencia de $x$.
\ppart Calcula 3.
\ppart Supongamos que $x$ es otra cosa distinta. Entonces dime qué te parece

\[ x^2 + 1 \]

\solution

\spart La divergencia es 0.

\spart

\[ 3 = \int_0^1 δ(x^2) \]

\spart  Muy bien.

\end{problem}
\end{verbatim}

Y el resultado es

\begin{problem}[3]
Sea $x$ una cosa, entonces calcula:

\ppart Calcula la divergencia de $x$.
\ppart Calcula 3.
\ppart Supongamos que $x$ es otra cosa distinta. Entonces dime qué te parece

\[ x^2 + 1 \]

\solution

\spart La divergencia es 0.

\spart

\[ 3 = \int_0^1 δ(x^2) \]

\spart  Muy bien.

\end{problem}

 Además se puede añadir una explicación a los ejercicios con el comando:
\begin{verbatim}
\begin{expla}
Para resolver el ejercicio utilizaremos el teorema de Parramón, formulado por Alberto Parramón, matemático español de gran popularidad...
\end{expla}
\end{verbatim}

 que da como resultado:
\begin{expla}
Para resolver el ejercicio utilizaremos el teorema de Parramón, formulado por Alberto Parramón, matemático español de gran popularidad...
\end{expla}

\section{Imágenes}

Hay dos comandos para poner fácilmente imágenes. El principal es \texttt{easyimgw}

\begin{verbatim}
\easyimgw{Patata.jpg}{Leyenda}{lblEtiqueta}{0.3}
\end{verbatim}

El último argumento es la anchura de la imagen expresada como proporción de la anchura del texto. $0.3$ significa que ocupa un $30\%$ de la anchura del texto, por ejemplo.

\easyimgw{Patata.jpg}{Leyenda}{lblEtiqueta}{0.3}

También está el comando \verb|\easyimg|, el uso es el mismo salvo que sólo necesita tres argumentos: la anchura se omite y se toma el valor por defecto del $80\%$ de anchura del texto.

 \section{Clase apuntes}

 También hay un archivo llamado \texttt{apuntes.cls}, que provee la clase \textit{apuntes}. Básicamente, lo único que hace es cambiar la fuente, ajustar la geometría e incluir el paquete \texttt{exmath}. De esta forma, lo único que hay que hacer para usar todo el paquete y clase es cambiar la clase del documento. Es decir, que la primera línea sea

\begin{verbatim}
\documentclass{apuntes}
\end{verbatim}

 Además, la clase genera el título y la cabecera, sólo tenéis que configurar la fecha, título y autor con los siguientes comandos, que deben ir después de la definición de \texttt{documentclass}.

\begin{verbatim}
\author{Autor}
\date{Fecha}
\title{Título del documento}
\end{verbatim}

\section{Consejos generales de \LaTeX}

\begin{itemize}
\item ¿Símbolos matemáticos demasiado pequeños? Por ejemplo, $\sum_{i=0}^n$. Usa \verb|\displaystyle|, así: \verb|$\displaystyle\sum_{i=0}^n$| que queda mejor: $\displaystyle\sum_{i=0}^n$.
\item Si quieres poner un código LaTeX en el texto, hazlo con \verb|\verb$comandoquesea$| o \verb|\begin{verbatim}código que sea.\end{verbatim}|.
\item El entorno \texttt{wrapfigure} permite crear figuras rodeadas por texto, que queda bien cuando no ocupa  todo el ancho del texto.
\item Usa etiquetas \verb|\label{nombreEtiqueta}| cuando quieras referenciar otras partes del documento con \verb|\ref{nombreEtiqueta}}|. Por ejemplo, si pones una etiqueta debajo de un comando de sección, \verb|\ref{nombreEtiqueta}}| mostrará el número de esa sección. También puedes hacerlo en figuras, tablas y listados de código. Además, tienes el comando \verb|\pageref{nombreEtiqueta}| si lo que quieres mostrar es el número de página donde está la etiqueta.
\end{itemize}

\subsection{Entornos matemáticos}

El paquete \texttt{amsmath} viene con un buen número de entornos matemáticos para organizar ecuaciones. Lo escribiría aquí, pero son muchos así que te vas a la documentación de \texttt{amsmath}\footnote{\url{ftp://ftp.ams.org/pub/tex/doc/amsmath/amsldoc.pdf}}, sección 3, y ahí lo explica todo muy bien. Los más interesantes: \texttt{align} y \texttt{gather}. Las versiones con asterisco (\texttt{align*}, \texttt{gather*}) no te ponen los números en las ecuaciones.

La mayor diferencia entre ambos entornos es que \texttt{align} te permite alinear las ecuaciones como en las tablas. Dos ejemplos:

\begin{verbatim}
\begin{align*}
f(x) + λ &= g(x) + α \\
f'(x) &= g'(x) + \dpa{α}{x} = \\
&= 4g'(x)^2
\end{align*}
\end{verbatim}

Resultado:

\begin{align*}
f(x) + λ &= g(x) + α \\
f'(x) &= g'(x) + \dpa{α}{x} = \\
&= 4g'(x)^2
\end{align*}

\begin{verbatim}
\begin{gather*}
f(x) = g(x) + α \\
k(x) = g'(x) + \dpa{α}{x} \\
\end{gather*}
\end{verbatim}

Resultado:

\begin{gather*}
f(x) = g(x) + α \\
k(x) = g'(x) + \dpa{α}{x} \\
\end{gather*}

\section{Tikz}

Para facilitar los dibujos de Tikz, el paquete \texttt{tikztools.sty} tiene varios comandos predefinidos. Incluye las librerías usuales, algunos estilos interesantes: \textit{nodepoint} para poner puntos sencillos, \textit{vnlin, hnlin} para poner marcas verticales u horizontales, varias clases para marcar con flechas líneas (\textit{directed, dense directed, etc}, todas con un nombre bastante claro) y varios comandos para dibujar familias de curvas que programé en su momento para EDO y ahora mismo no tengo huevos a entender. También está el comando \texttt{tikzangle} para marcar ángulos en LaTeX.

Por otra parte, está el paquete \texttt{fastbuild.sty}. Tikz es lento, y cuando hay un montón de dibujos recompilar un documento puede ser un infierno. Este paquete permite activar una caché de dibujos Tikz, de tal forma que los dibujos se generan en un PDF y después se incluyen, de tal forma que no hay que regenerarlos en cada compilación. Para activar esta caché, hay que incluir el comando \verb|\precompileTikz| en el preámbulo del documento.

Internamente, lo que hace LaTeX cuando ahora se encuentre un dibujo Tikz, es abrir un nuevo proceso que compile el dibujo y después incluirlo en el documento. Por seguridad, la configuración por defecto de LaTeX impide la creación de nuevos procesos, por lo que hay que pasar como argumento \texttt{-shell-escape} al comando de compilación. Toda la caché se guarda en el directorio \textit{tikzgen}, hay que procurar que esté creado porque si no LaTeX se quejará con un error bastante críptico.

\section{Comandos}

\setcounter{LTchunksize}{10}
\renewcommand{\arraystretch}{1.5}% Spread rows out...
\begin{longtable}{|c|c|}
 \hline\verb|\mop{operacion}| & $\displaystyle\mop{operacion}$ \\ \midrule 
\verb|\imgref{referencia}| & $\displaystyle\imgref{referencia}$ \\ \midrule 
\bottomrule \multicolumn{2}{|c|}{\textbf{Comandos de lógica}} \\ \toprule 
\verb|\dimplies| & $\displaystyle\dimplies$ \\ \midrule 
\verb|\nimplies| & $\displaystyle\nimplies$ \\ \midrule 
\verb|\Or| & $\displaystyle\Or$ \\ \midrule 
\verb|\y| & $\displaystyle\y$ \\ \midrule 
\verb|\tq| & $\displaystyle\tq$ \\ \midrule 
\verb|\tlq| & $\displaystyle\tlq$ \\ \midrule 
\verb|\wtf| & $\displaystyle\wtf$ \\ \midrule 
\bottomrule \multicolumn{2}{|c|}{\textbf{Cálculo}} \\ \toprule 
\verb|\deriv{f}{arg1}| & $\displaystyle\deriv{f}{arg1}$ \\ \midrule 
\verb|\dpa{arg0}{arg1}| & $\displaystyle\dpa{arg0}{arg1}$ \\ \midrule 
\verb|\rot| & $\displaystyle\rot$ \\ \midrule 
\verb|\dv| & $\displaystyle\dv$ \\ \midrule 
\verb|\grad| & $\displaystyle\grad$ \\ \midrule 
\verb|\img| & $\displaystyle\img$ \\ \midrule 
\verb|\epsilon| & $\displaystyle\epsilon$ \\ \midrule 
\verb|\bigzero| & $\displaystyle\bigzero$ \\ \midrule 
\verb|\tlps| & $\displaystyle\tlps$ \\ \midrule 
\verb|\invers{arg0}| & $\displaystyle\invers{arg0}$ \\ \midrule 
\verb|\f| & $\displaystyle\f$ \\ \midrule 
\verb|\F| & $\displaystyle\F$ \\ \midrule 
\verb|\Rnxm| & $\displaystyle\Rnxm$ \\ \midrule 
\verb|\acum{arg0}| & $\displaystyle\acum{arg0}$ \\ \midrule 
\verb|\rnk| & $\displaystyle\rnk$ \\ \midrule 
\verb|\rk| & $\displaystyle\rk$ \\ \midrule 
\verb|\ncl{f}| & $\displaystyle\ncl{f}$ \\ \midrule 
\bottomrule \multicolumn{2}{|c|}{\textbf{Análisis matemático}} \\ \toprule 
\verb|\normp{vector}{norma-n}| & $\displaystyle\normp{vector}{norma-n}$ \\ \midrule 
\verb|\liminft{variable}| & $\displaystyle\liminft{variable}$ \\ \midrule 
\verb|\mylim{vector}{tiende-a}{variable}| & $\displaystyle\mylim{vector}{tiende-a}{variable}$ \\ \midrule 
\verb|\df{x,y,z}| & $\displaystyle\df{x,y,z}$ \\ \midrule 
\verb|\dfl{comienzo}{fin}| & $\displaystyle\dfl{comienzo}{fin}$ \\ \midrule 
\verb|\id{x,y,z}| & $\displaystyle\id{x,y,z}$ \\ \midrule 
\verb|\pb[T]| & $\displaystyle\pb[T]$ \\ \midrule 
\bottomrule \multicolumn{2}{|c|}{\textbf{Galois}} \\ \toprule 
\verb|\units{arg0}| & $\displaystyle\units{arg0}$ \\ \midrule 
\bottomrule \multicolumn{2}{|c|}{\textbf{Notación de puntos de $R^N$}} \\ \toprule 
\verb|\gx| & $\displaystyle\gx$ \\ \midrule 
\verb|\gs| & $\displaystyle\gs$ \\ \midrule 
\verb|\gy| & $\displaystyle\gy$ \\ \midrule 
\verb|\gz| & $\displaystyle\gz$ \\ \midrule 
\verb|\ga| & $\displaystyle\ga$ \\ \midrule 
\verb|\gb| & $\displaystyle\gb$ \\ \midrule 
\verb|\gv| & $\displaystyle\gv$ \\ \midrule 
\verb|\gu| & $\displaystyle\gu$ \\ \midrule 
\verb|\gw| & $\displaystyle\gw$ \\ \midrule 
\bottomrule \multicolumn{2}{|c|}{\textbf{Caracteres}} \\ \toprule 
\verb|\qeq| & $\displaystyle\qeq$ \\ \midrule 
\verb|\r{A}| & $\displaystyle\r{A}$ \\ \midrule 
\verb|\intr{A}| & $\displaystyle\intr{A}$ \\ \midrule 
\bottomrule \multicolumn{2}{|c|}{\textbf{Operadores}} \\ \toprule 
\verb|\comb{arg0}{arg1}| & $\displaystyle\comb{arg0}{arg1}$ \\ \midrule 
\verb|\abs{arg0}| & $\displaystyle\abs{arg0}$ \\ \midrule 
\verb|\inv{arg0}| & $\displaystyle\inv{arg0}$ \\ \midrule 
\verb|\conj{arg0}| & $\displaystyle\conj{arg0}$ \\ \midrule 
\verb|\avg{arg0}| & $\displaystyle\avg{arg0}$ \\ \midrule 
\verb|\card{arg0}| & $\displaystyle\card{arg0}$ \\ \midrule 
\verb|\no{arg0}| & $\displaystyle\no{arg0}$ \\ \midrule 
\verb|\gor{arg0}| & $\displaystyle\gor{arg0}$ \\ \midrule 
\verb|\floor{arg0}| & $\displaystyle\floor{arg0}$ \\ \midrule 
\verb|\argmin| & $\displaystyle\argmin$ \\ \midrule 
\verb|\argmax| & $\displaystyle\argmax$ \\ \midrule 
\verb|\sign| & $\displaystyle\sign$ \\ \midrule 
\verb|\eq| & $\displaystyle\eq$ \\ \midrule 
\verb|\rango| & $\displaystyle\rango$ \\ \midrule 
\verb|\eqexpl[=]{arg1}| & $\displaystyle\eqexpl[=]{arg1}$ \\ \midrule 
\verb|\eqreason[=]{arg1}| & $\displaystyle\eqreason[=]{arg1}$ \\ \midrule 
\verb|\eqreasonup[=]{arg1}| & $\displaystyle\eqreasonup[=]{arg1}$ \\ \midrule 
\bottomrule \multicolumn{2}{|c|}{\textbf{Comandos para conjuntos y relaciones}} \\ \toprule 
\verb|\x| & $\displaystyle\x$ \\ \midrule 
\verb|\appl{f}{dominio}{imagen}| & $\displaystyle\appl{f}{dominio}{imagen}$ \\ \midrule 
\verb|\uexists| & $\displaystyle\uexists$ \\ \midrule 
\verb|\sint| & $\displaystyle\sint$ \\ \midrule 
\verb|\stdf| & $\displaystyle\stdf$ \\ \midrule 
\verb|\rel| & $\displaystyle\rel$ \\ \midrule 
\verb|\parts{arg0}| & $\displaystyle\parts{arg0}$ \\ \midrule 
\verb|\ind| & $\displaystyle\ind$ \\ \midrule 
\bottomrule \multicolumn{2}{|c|}{\textbf{Estadística}} \\ \toprule 
\verb|\esp[parametro]{arg1}| & $\displaystyle\esp[parametro]{arg1}$ \\ \midrule 
\verb|\prob[parametro]{arg1}| & $\displaystyle\prob[parametro]{arg1}$ \\ \midrule 
\verb|\var[parametro]{arg1}| & $\displaystyle\var[parametro]{arg1}$ \\ \midrule 
\verb|\fd| & $\displaystyle\fd$ \\ \midrule 
\verb|\convs[texto superior][variable][converge a]| & $\displaystyle\convs[texto superior][variable][converge a]$ \\ \midrule 
\verb|\convdist[variable]| & $\displaystyle\convdist[variable]$ \\ \midrule 
\verb|\convprob[variable]| & $\displaystyle\convprob[variable]$ \\ \midrule 
\verb|\convcs[variable]| & $\displaystyle\convcs[variable]$ \\ \midrule 
\verb|\sample[variable][indice]| & $\displaystyle\sample[variable][indice]$ \\ \midrule 
\verb|\sesgo| & $\displaystyle\sesgo$ \\ \midrule 
\verb|\ECM| & $\displaystyle\ECM$ \\ \midrule 
\verb|\emv| & $\displaystyle\emv$ \\ \midrule 
\verb|\cov{arg0}| & $\displaystyle\cov{arg0}$ \\ \midrule 
\verb|\corr{arg0}| & $\displaystyle\corr{arg0}$ \\ \midrule 
\bottomrule \multicolumn{2}{|c|}{\textbf{Abreviaciones para conjuntos usuales}} \\ \toprule 
\verb|\nat| & $\displaystyle\nat$ \\ \midrule 
\verb|\ent| & $\displaystyle\ent$ \\ \midrule 
\verb|\rac| & $\displaystyle\rac$ \\ \midrule 
\verb|\real| & $\displaystyle\real$ \\ \midrule 
\verb|\cplex| & $\displaystyle\cplex$ \\ \midrule 
\verb|\vec{arg0}| & $\displaystyle\vec{arg0}$ \\ \midrule 
\bottomrule \multicolumn{2}{|c|}{\textbf{Vectores}} \\ \toprule 
\verb|\pesc{arg0}| & $\displaystyle\pesc{arg0}$ \\ \midrule 
\verb|\md{arg0}| & $\displaystyle\md{arg0}$ \\ \midrule 
\bottomrule \multicolumn{2}{|c|}{\textbf{Vectores usuales, para no tener que andar escribiendo}} \\ \toprule 
\verb|\vx| & $\displaystyle\vx$ \\ \midrule 
\verb|\vy| & $\displaystyle\vy$ \\ \midrule 
\verb|\vf| & $\displaystyle\vf$ \\ \midrule 
\verb|\vg| & $\displaystyle\vg$ \\ \midrule 
\verb|\va| & $\displaystyle\va$ \\ \midrule 
\verb|\vb| & $\displaystyle\vb$ \\ \midrule 
\verb|\vu| & $\displaystyle\vu$ \\ \midrule 
\verb|\vn| & $\displaystyle\vn$ \\ \midrule 
\verb|\vv| & $\displaystyle\vv$ \\ \midrule 
\verb|\ve| & $\displaystyle\ve$ \\ \midrule 
\verb|\vr| & $\displaystyle\vr$ \\ \midrule 
\bottomrule \multicolumn{2}{|c|}{\textbf{\numberwithin{equation}{section}}} \\ \toprule 
\verb|\ppart[apartado]| & $\displaystyle\ppart[apartado]$ \\ \midrule 
\verb|\spart[apartado]| & $\displaystyle\spart[apartado]$ \\ \midrule 
\bottomrule \multicolumn{2}{|c|}{\textbf{Complejos}} \\ \toprule 
\verb|\ctrig{angulo}| & $\displaystyle\ctrig{angulo}$ \\ \midrule 
\verb|\ceul{modulo}{angulo}| & $\displaystyle\ceul{modulo}{angulo}$ \\ \midrule 
\verb|\i| & $\displaystyle\i$ \\ \midrule 
\bottomrule \multicolumn{2}{|c|}{\textbf{Matrices}} \\ \toprule 
\verb|\trans{arg0}| & $\displaystyle\trans{arg0}$ \\ \midrule 
\bottomrule \multicolumn{2}{|c|}{\textbf{Probabilidad}} \\ \toprule 
\verb|\bin| & $\displaystyle\bin$ \\ \midrule 
\verb|\geom| & $\displaystyle\geom$ \\ \midrule 
\bottomrule \multicolumn{2}{|c|}{\textbf{Estructuras algebraicas}} \\ \toprule 
\verb|\gen{genlist}| & $\displaystyle\gen{genlist}$ \\ \midrule 
\verb|\ord| & $\displaystyle\ord$ \\ \midrule 
\verb|\gr| & $\displaystyle\gr$ \\ \midrule 
\verb|\kbb| & $\displaystyle\kbb$ \\ \midrule 
\bottomrule \multicolumn{2}{|c|}{\textbf{Geometría de curvas y superficies}} \\ \toprule 
\verb|\mv{vector}| & $\displaystyle\mv{vector}$ \\ \midrule 
\verb|\cv[\alpha]| & $\displaystyle\cv[\alpha]$ \\ \midrule 
\verb|\cvv[\alpha]| & $\displaystyle\cvv[\alpha]$ \\ \midrule 
\verb|\cn[\alpha]| & $\displaystyle\cn[\alpha]$ \\ \midrule 
\verb|\ct[\alpha]| & $\displaystyle\ct[\alpha]$ \\ \midrule 
\verb|\cb[\alpha]| & $\displaystyle\cb[\alpha]$ \\ \midrule 
\verb|\ctr[\alpha]| & $\displaystyle\ctr[\alpha]$ \\ \midrule 
\verb|\wein| & $\displaystyle\wein$ \\ \midrule 
\verb|\lfi| & $\displaystyle\lfi$ \\ \midrule 
\verb|\vV| & $\displaystyle\vV$ \\ \midrule 
\bottomrule \multicolumn{2}{|c|}{\textbf{Topología}} \\ \toprule 
\verb|\dst| & $\displaystyle\dst$ \\ \midrule 
\verb|\bola| & $\displaystyle\bola$ \\ \midrule 
\verb|\sdst| & $\displaystyle\sdst$ \\ \midrule 
\verb|\topl| & $\displaystyle\topl$ \\ \midrule 
\verb|\stopl| & $\displaystyle\stopl$ \\ \midrule 
\verb|\base| & $\displaystyle\base$ \\ \midrule 
\verb|\toplb| & $\displaystyle\toplb$ \\ \midrule 
\verb|\adh{arg0}| & $\displaystyle\adh{arg0}$ \\ \midrule 
\verb|\iffdef| & $\displaystyle\iffdef$ \\ \midrule 
\verb|\bolac| & $\displaystyle\bolac$ \\ \midrule 
\verb|\tops| & $\displaystyle\tops$ \\ \midrule 
\verb|\topu| & $\displaystyle\topu$ \\ \midrule 
\verb|\quot{arg0}{arg1}| & $\displaystyle\quot{arg0}{arg1}$ \\ \midrule 
\verb|\dfm| & $\displaystyle\dfm$ \\ \midrule 
\verb|\restr{arg0}{arg1}| & $\displaystyle\restr{arg0}{arg1}$ \\ \midrule 
\verb|\bbs| & $\displaystyle\bbs$ \\ \midrule 
\verb|\crc| & $\displaystyle\crc$ \\ \midrule 
\verb|\torus| & $\displaystyle\torus$ \\ \midrule 
\verb|\disc| & $\displaystyle\disc$ \\ \midrule 
\bottomrule \multicolumn{2}{|c|}{\textbf{TIM}} \\ \toprule 
\verb|\algb{arg0}| & $\displaystyle\algb{arg0}$ \\ \midrule 
\verb|\algbA| & $\displaystyle\algbA$ \\ \midrule 
\verb|\algbM| & $\displaystyle\algbM$ \\ \midrule 
\verb|\algbC| & $\displaystyle\algbC$ \\ \midrule 
\verb|\algbB| & $\displaystyle\algbB$ \\ \midrule 
\verb|\algbT| & $\displaystyle\algbT$ \\ \midrule 
\verb|\algbP| & $\displaystyle\algbP$ \\ \midrule 
\verb|\salgb| & $\displaystyle\salgb$ \\ \midrule 
\verb|\sfin| & $\displaystyle\sfin$ \\ \midrule 
\bottomrule \multicolumn{2}{|c|}{\textbf{Galois}} \\ \toprule 
\verb|\gal| & $\displaystyle\gal$ \\ \midrule 
\bottomrule \multicolumn{2}{|c|}{\textbf{Geometría}} \\ \toprule 
\verb|\tens| & $\displaystyle\tens$ \\ \midrule 
\verb|\dif| & $\displaystyle\dif$ \\ \midrule 
\verb|\Dif| & $\displaystyle\Dif$ \\ \midrule 
\verb|\projp| & $\displaystyle\projp$ \\ \midrule 
\verb|\lie{arg0}| & $\displaystyle\lie{arg0}$ \\ \midrule 
\verb|\tgs| & $\displaystyle\tgs$ \\ \midrule 
\verb|\tgsd| & $\displaystyle\tgsd$ \\ \midrule 
\verb|\inmr{arg0}{arg1}{arg2}| & $\displaystyle\inmr{arg0}{arg1}{arg2}$ \\ \midrule 
\verb|\halfp| & $\displaystyle\halfp$ \\ \midrule 
\bottomrule \multicolumn{2}{|c|}{\textbf{Variable Compleja}} \\ \toprule 
\verb|\cind| & $\displaystyle\cind$ \\ \midrule 
\verb|\Re| & $\displaystyle\Re$ \\ \midrule 
\verb|\Im| & $\displaystyle\Im$ \\ \midrule 
\verb|\Res| & $\displaystyle\Res$ \\ \midrule 
\bottomrule \multicolumn{2}{|c|}{\textbf{Modelización}} \\ \toprule 
\verb|\fe{arg0}| & $\displaystyle\fe{arg0}$ \\ \midrule 
\bottomrule \multicolumn{2}{|c|}{\textbf{Variable real}} \\ \toprule 
\verb|\lebg| & $\displaystyle\lebg$ \\ \midrule 
\verb|\borel| & $\displaystyle\borel$ \\ \midrule 
\verb|\meds[X]| & $\displaystyle\meds[X]$ \\ \midrule 
\verb|\meas[X][\mu]| & $\displaystyle\meas[X][\mu]$ \\ \midrule 
\verb|\minuszero| & $\displaystyle\minuszero$ \\ \midrule 
\verb|\essup| & $\displaystyle\essup$ \\ \midrule 
\verb|\esinf| & $\displaystyle\esinf$ \\ \midrule 
\verb|\convsmed| & $\displaystyle\convsmed$ \\ \midrule 
\verb|\mpos| & $\displaystyle\mpos$ \\ \midrule 
\verb|\msgn| & $\displaystyle\msgn$ \\ \midrule 
\verb|\mfin| & $\displaystyle\mfin$ \\ \midrule 
\verb|\sop| & $\displaystyle\sop$ \\ \midrule 
\verb|\essop| & $\displaystyle\essop$ \\ \midrule 
\verb|\esplloc[1][\real^N]| & $\displaystyle\esplloc[1][\real^N]$ \\ \midrule 
\verb|\espl[1][\real^N]| & $\displaystyle\espl[1][\real^N]$ \\ \midrule 
\verb|\espLloc[1][\real^N]| & $\displaystyle\espLloc[1][\real^N]$ \\ \midrule 
\verb|\espL[1][\real^N]| & $\displaystyle\espL[1][\real^N]$ \\ \midrule 
\verb|\espLp| & $\displaystyle\espLp$ \\ \midrule 
\verb|\esplp| & $\displaystyle\esplp$ \\ \midrule 
\verb|\pnorm{arg0}| & $\displaystyle\pnorm{arg0}$ \\ \midrule 
\end{longtable}

\newpage
\printindex

\end{document}
