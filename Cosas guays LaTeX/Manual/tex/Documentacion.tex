% -*- root: ../Manual.tex -*-
\chapter{Manual paquetes extendidos}
\label{chap:PaqueteExtendido}

\section{Introducción: ¿qué podemos hacer con los paquetes extendidos?}

Mientras escribimos los apuntes, hay muchas cosas que repetimos, que tecleamos muchas veces o que nos gustaría hacer más rápido o de forma unificada. Al final hemos creado varios paquetes que nos permiten tener un formato unificado y con varios comandos para escribir más rápido.

El paquete principal es el \textit{exmath.sty} (\fref{sec:Exmath}), que contiene una gran cantidad de comandos (la lista está en el apéndice \ref{app:Comandos}) que permiten escribir cosas repeitdas fácilmente. Por ejemplo, podemos escribir una aplicación sólo con un comando (\verb|\appl{f}{A}{B}|) o usar atajos como \verb|\cvv| en lugar de \verb|\vec{k}_{\alpha}| para sacar $\cvv$ También hay algunos entornos que hacen cosas solitos, como por ejemplo los de teoremas o definiciones, que añaden notas al margen y entradas en el índice.

También tenemos dos clases de documentos, \textit{apuntes} (\fref{sec:ApuntesCls}) y \textit{ejercicios} (\fref{sec:EjerciciosCls}), que se pueden poner en el \textit{documentclass} en lugar de \textit{article}, y que dan un formato más o menos estándar al documento y cargan varios paquetes habituales, como las cadenas en español, marcos, colores y, por supuesto, el \textit{exmath.sty}.

Un paquete muy útil es \textit{MathUnicode.sty} (\fref{sec:MathUnicode}), que permite introducir letras griegas y otros caracteres no estándar desde el teclado al documento, de tal forma que podamos escribir α y no \verb|\alpha| cada vez que queramos poner ese caracter (y muchos otros).

Por último, hay algunos paquetes auxiliares, como \textit{fancysprefs} para referencias mejoradas (\fref{sec:fancysprefs}), \textit{fastbuild} para mejorar velocidad de compilación en documentos pesados (\fref{sec:fastbuild}) o \textit{tikztools} con algunos comandos adicionales para Tikz (\ref{sec:Tikztools}).

\section{Estructura de los paquetes extendidos}

\index{install}
\index{Cosas guays LaTeX}
Todos los paquetes se encuentran en el directorio \textit{Cosas guays LaTeX} (sic). Es importante que estén en este directorio, ya que es en el que busca el script de instalación de paquetes (archivo \textit{install} en la raíz del repositorio) para moverlos al sistema.

En ese directorio están los paquetes (archivos \textit{.sty}) que hemos desarrollado nosotros, junto con algunos de terceros que a veces desaparecen en algunas instalaciones de \LaTeX. También están las clases \textit{apuntes.cls, ejercicios.cls} y \textit{custombase.cls}: esta última es la clase básica que comparten \textit{apuntes} y \textit{ejercicios}.

También en \textit{Cosas guays LaTeX} está este manual, en la carpeta \textit{Manual}. Ahí están las fuentes \textit{.tex} de este documento y además el archivo \textit{commandsampler.py} para sacar unos ejemplos a partir de un archivo \textit{.sty}: ver la \fref{sec:ComandosAdicionales} para una descripción algo más detallada.

Por último, dentro del directorio hay varios archivos adicionales:

\begin{itemize}
\item AjustesTeXStudio.txsprofile: un archivo con ajustes para TeXStudio que vienen bastante bien, incluyendo archivos de completado por defecto.
\item \index{cwlcreator.py} cwlcreator.py: un script en Python que genera archivos CWL a partir de un paquete \textit{.sty}. Estos archivos se pueden incluir luego en Sublime o en TeXStudio para tener autocompletado de los paquetes que hemos desarrollado.
\item Archivos \textit{*.cwl}: archivos de autocompletado para Sublime o TeXStudio.
\item ConfiguracionTeclado: Configuración para el teclado con caracteres especiales (ver \fref{sec:MathUnicode}).
\end{itemize}

\section{Documetación de los paquetes}

\subsection{\textit{exmath.sty}: Comandos y entornos adicionales de matemáticas}
\label{sec:Exmath}

\subsubsection{Entornos para teoremas y similares}

Los teoremas del paquete heredan de los teoremas de \texttt{amsthm}, pero además añaden un índice automático de teoremas que se puede imprimir en el documento con \verb|\printtheorems|. También se cargan automáticamente en el glosario de términos. El uso normal es el siguiente

\begin{LTXexample}[pos=r]
\begin{theorem}[Teorema de las gallinas cluecas]
Teorema.
\end{theorem}
\end{LTXexample}

El título se carga automáticamente como una entrada para el índice. Si, por lo que sea, queremos separar los términos (por ejemplo, para que en este caso aparezca primero \textit{Teorema} y luego, debajo, \textit{de las gallinas cluecas}, ved la última página), no podemos usar la exclamación del comando \verb|\index{}| porque aparecería en el título. En su lugar, podemos usar el comando \verb|\IS|, que actúa como una separación para el índice pero que no aparece en el título.

\begin{LTXexample}[pos=r]
\begin{theorem}[Teorema\IS de las gallinas cluecas]
Teorema.
\end{theorem}
\end{LTXexample}

Si, por lo que sea, queremos especificar nosotros el índice, podemos añadir un segundo argumento.

\begin{LTXexample}[pos=r]
\begin{theorem}[Teorema de las gallinas cluecas][Mi entrada]
Teorema.
\end{theorem}
\end{LTXexample}

Hay definidos varios entornos similares a teoremas. Todos ellos se pueden titular poniendo entre corchetes el título.

\begin{LTXexample}[pos=r]
\begin{lemma}[Lema cuarenta y dos]
\end{lemma}

\begin{corol}[Coro]
\end{corol}

\begin{prop}[De las desigualdades]
\end{prop}

\begin{axiom}
\end{axiom}

\begin{proof}
Probado queda.

Al final agrega un cuadradito de QED.
\end{proof}

\begin{op}{Nombre de operación}
x = 3 + 1
\end{op}
\end{LTXexample}

El entorno \texttt{op} incluye el modo matemático directamente, y el nombre de operación es obligatorio.

\subsubsection{Definiciones y conceptos}

Por otra parte tenemos el entorno \texttt{defn} para definiciones, que funciona exactamente igual que \texttt{theorem}:

\begin{LTXexample}[pos=r]
\begin{defn}[Título de la definición][(opcional) entrada para el índice]
\end{defn}
\end{LTXexample}

\subsubsection{Ejercicios}
\label{sec:Ejercicios}

Hay dos entornos para incluir ejercicios o ejemplos en los archivos. El primero es \texttt{example}, que acepta un argumento opcional como nombre del ejemplo.

\begin{LTXexample}[pos=r]
\begin{example}[Titulo]
Un ejemplo sobre cosas matematicas.

\[ 3 = 3 \]

Fin del ejemplo.
\end{example}
\end{LTXexample}

\index{problem}
\index{problemS}
El otro sería \textit{problem}, que viene con bastantes cosillas. Acepta un argumento opcional, que es el número de ejercicio. Si no se pasa este argumento, la numeración es automática y se genera de la forma \textit{Sección.Ejercicio}, de tal forma que se pueda referenciar luego con etiquetas. El comando \verb|\solution| separa enunciado y solución, y los comandos \verb|\ppart| y \verb|spart| sirven para separar los diferentes apartados (la numeración es automática). Además, hay una variante \textit{problemS} que pone ejercicios sin la numeración por sección. Ejemplo:

\begin{LTXexample}[pos=r]
\begin{problem}[3] \label{ej:Ejemplo}
Sea $x$ una cosa, entonces calcula:

\ppart Calcula la divergencia de $x$.
\ppart Calcula 3.
\ppart Supongamos que $x$ es otra cosa distinta. Entonces dime que te parece

\[ x^2 + 1 \]

\solution

\spart La divergencia es 0.

\spart

\[ 3 = \int_0^1 \delta(x^2) \]

\spart Muy bien.

\end{problem}

\begin{problemS}
Siguiendo con el ejercicio \ref{ej:Ejemplo}, dime mas cosas

\solution

\spart Mas cosas del \fref{ej:Ejemplo}.
\end{problemS}
\end{LTXexample}

\subsubsection{Imágenes}

Hay dos comandos para poner fácilmente imágenes. El principal es \texttt{easyimgw}

\begin{LTXexample}[pos=r]
\easyimgw{Patata.jpg}{Leyenda}{lblEtiqueta}{0.3}
\end{LTXexample}

El último argumento es la anchura de la imagen expresada como proporción de la anchura del texto. $0.3$ significa que ocupa un $30\%$ de la anchura del texto, por ejemplo.

También está el comando \verb|\easyimg|, el uso es el mismo salvo que sólo necesita tres argumentos: la anchura se omite y se toma el valor por defecto del $80\%$ de anchura del texto.

\subsubsection{Comandos adicionales}
\label{sec:ComandosAdicionales}

Dado que en cada asignatura hay varias cosas que se repiten, viene muy bien definir los comandos en \textit{exmath.sty} para poder usarlos, y además así poder reutilizarlos en otras asignaturas.

\index{commandsampler.py}
Como son muchos, en esta documentación aparecen en el apéndice \ref{chap:Comandos}, con ejemplos de uso y su salida. La tabla se genera de forma automática con el script \textit{commandsampler.py}, en el directorio \textit{Cosas Guays LaTeX}. La documentación de cómo funciona está en el propio paquete \textit{exmath.sty}, aunque en algún momento debería de poner aquí cómo funciona.

\subsection{\textit{fancysprefs.sty}: Referencias mejoradas}
\label{sec:fancysprefs}

\index{fancyref}
\index{fancysprefs}
\cindex{nref}
\cindex{nrefl}
\textit{fancysprefs} está basado en \textit{fancyref}, que ya hemos descrito en la \fref{sec:fancyref}. Este paquete provee dos comandos adicionales, \verb|\nref{nombreEtiqueta}| y \verb|\nlref{nombreEtiqueta}|, además de traducir las cadenas de \textit{fancyref} al castellano.

El comando \texttt{nref} hace algo parecido a \textit{fref}, pero incluye también el nombre de lo que estemos referenciando. Así, sólo nos hace falta poner \verb|\nref{sec:fancysprefs}| para que aparezca ``\nref{sec:fancysprefs}''. Obviamente, el nombre cambiará automáticamente aquí si lo cambiamos en el entorno correspondiente. La variante \texttt{nlref} hace exactamente lo mismo pero con el título en minúsculas (muestra: \nlref{sec:fancysprefs}), que viene bien cuando estamos referenciado definiciones y no queremos que la primera letra esté en mayúscula.

\subsection{\textit{MathUnicode.sty}: Letras griegas directamente desde el teclado}
\label{sec:MathUnicode}

\index{MathUnicode.sty}
Escribir letras griegas o símbolos en \LaTeX\ puede ser un poco rollo. Este paquete (incluido por defecto en la clase \textit{apuntes} y \textit{ejercicios}) permite hacerlo directamente desde el teclado. Así, en lugar de escribir \verb|\alpha|, por ejemplo, simplemente pulsaríamos \textit{Alt + A} para poner α en el texto y que \LaTeX\ lo interprete sin fallos.

El paquete no soporta sólo letras griegas: también símbolos como $∀, ∃, ∂, ∈, ⊂, ℝ, ℂ, ℕ, ≥, ∉$ y unos cuantos más. Para usarlo bien, probablemente tengas que cambiar la distribución de teclado de tu ordenador y así poder introducir estos caracteres. Para Mac, \href{https://github.com/gjulianm/MathUnicode}{es bastante sencillo}, para Linux \href{https://github.com/VicdeJuan/Apuntes/tree/master/Cosas%20guays%20LaTeX/ConfiguracionTeclado}{hay que tocar alguna cosa más}.

\subsection{\textit{tikztools.sty}: Herramientas y comandos extra para Tikz}
\label{sec:Tikztools}

\index{tikztools.sty}
Para facilitar los dibujos de Tikz (ver \fref{sec:Tikz} para una introducción sobre qué es Tikz), el paquete \textit{tikztools.sty} tiene varios comandos predefinidos. Incluye las librerías usuales, algunos estilos interesantes: \textit{nodepoint} para poner puntos sencillos, \textit{vnlin, hnlin} para poner marcas verticales u horizontales, varias clases para marcar con flechas líneas (\textit{directed, dense directed, etc}, todas con un nombre bastante claro) y varios comandos para dibujar familias de curvas que programé en su momento para EDO y ahora mismo no tengo narices a entender. También está el comando \texttt{tikzangle} para marcar ángulos en LaTeX.

\subsection{\textit{fastbuild.sty}: Recortando los tiempos de compilación}
\label{sec:fastbuild}

\index{fastbuild.sty}
Tikz es lento, y cuando hay un montón de dibujos recompilar un documento puede ser un infierno. Este paquete, \textit{fastbuild.sty}, permite activar una caché de dibujos Tikz, de tal forma que los dibujos se generan en un PDF y después se incluyen, de tal forma que no hay que regenerarlos en cada compilación. Para activar esta caché, hay que incluir el comando \verb|\precompileTikz| en el preámbulo del documento.

\index{-shell-escape}
Internamente, lo que hace LaTeX cuando ahora se encuentre un dibujo Tikz, es abrir un nuevo proceso que compile el dibujo y después incluirlo en el documento. Por seguridad, la configuración por defecto de LaTeX impide la creación de nuevos procesos, por lo que hay que pasar como argumento \texttt{-shell-escape} al comando de compilación. Toda la caché se guarda en el directorio \textit{tikzgen}, hay que procurar que esté creado porque si no LaTeX se quejará con un error bastante críptico.

\subsection{\textit{apuntes.cls}: La base para crear documentos de apuntes}
\label{sec:ApuntesCls}

También hay un archivo llamado \texttt{apuntes.cls}, que provee la clase \textit{apuntes}. Básicamente, lo único que hace es cambiar la fuente, ajustar la geometría e incluir el paquete \texttt{exmath}. De esta forma, lo único que hay que hacer para usar todo el paquete y clase es cambiar la clase del documento. Es decir, que la primera línea sea

\begin{verbatim}
\documentclass{apuntes}
\end{verbatim}

Además, la clase genera la portada y la cabecera, sólo hay que configurar la fecha, título y autor con los siguientes comandos, que deben ir después de la definición de \texttt{documentclass}.

\begin{verbatim}
\author{Autor}
\date{Fecha}
\title{Título del documento}
\end{verbatim}

\subsection{\textit{ejercicios.cls}: documentos simples y entrega de ejercicios}
\label{sec:EjerciciosCls}

Por si hace falta hacer documentos que sean únicamente ejercicios (probablemente para entregarlos), la clase \textit{ejercicios} sirve para eso. Reduce un poco los márgenes, pone una portada más discreta y pequeña, cambia a un tipo de letra algo más compacto y quita los marcos enormes a los problemas. El resto de comandos se puede seguir usando normalmente.

En este caso también puede venir útil usar el entorno \texttt{problemS} en lugar de \texttt{problem} para que los ejercicios no tengan la numeración estilo \textit{sección.ejercicio} (ver la \fref{sec:Ejercicios} para una descripción más detallada).

