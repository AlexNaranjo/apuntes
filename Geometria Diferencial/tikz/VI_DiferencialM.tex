\begin{tikzpicture}[scale=3]
\tikzset{
	showing-border/.style={gray},
	hidden-border/.style={gray, opacity=0.5},
	surface/.style={left color=gray!10,right color=gray!70, opacity = 0.5},
	ref-vec/.style={blue, ->},
	tang-border/.style={transparent},
	tang-space/.style={bottom color=blue!10, top color=blue!70, opacity = 0.3, shading angle = -10},
	perp-marker/.style={blue, fill = blue!80, fill opacity = 0.3},
	perp-marker-next/.style={red!80!black, fill = red!80!black!10, fill opacity=0.3},
	connecting-curve/.style={thick, dashed, green!80!black},
	tang-vector/.style={thick, green!60!black, ->},
	ref-vec-next/.style={ref-vec, red!80!black},
	ref-vec-old/.style={ref-vec, opacity=0.6},
	differential/.style={->, very thick, orange}
}

% Surface drawings

\coordinate (corner-1) at (0.4,0,-0.4);
\coordinate (corner-2) at (4,0,2);
\coordinate (corner-3) at (5,0,-2);
\coordinate (corner-4) at (1.7,0,-3);

\draw[showing-border] (corner-1) to[out =  0, in = 150] (corner-2);
\draw[showing-border] (corner-2) to[out = 45, in = 180] (corner-3);
\draw[showing-border] (corner-3) to[out = 135, in = 60] (corner-1);
\draw[hidden-border] (corner-1) to[out = 40, in = 175] (corner-4);
\draw[hidden-border] (corner-4) to[out = 10, in = 160] (corner-3);

\shade[surface]
  (corner-1)
  	to[out =  0, in = 150] (corner-2)
  	to[out = 45, in = 180] (corner-3)
  	to[out = 135, in = 60] (corner-1);


\shade[surface]
  (corner-1)
  	to[out = 40, in = 175] (corner-4)
  	to[out = 10, in = 160] (corner-3)
  	to[out = 135, in = 60] (corner-1);

\coordinate (p) at (2.3, 1, 0.4);

% Differential representation -- part under the tangent space

\coordinate (p-2) at (2.7, 1.2, -0.2);

\draw[connecting-curve] (p) to[out=45, in=200] (p-2);
\node[nodepoint, gray, label={below:$p'$}] at (p-2) {};

% Tangent space

\coordinate (e-1) at (0.96, -0.19, 0.19);
\coordinate (e-2) at (0,0.28,-0.95);
\coordinate (e-3) at (0.127391, 0.919513, 0.275854);

\pgfmathsetmacro{\tangwperc}{1.3}
\pgfmathsetmacro{\tanghperc}{1.7}

\coordinate (tang-corner-1) at ($(p) + \tangwperc*(e-1) + \tanghperc*(e-2)$);
\coordinate (tang-corner-2) at ($(p) + \tangwperc*(e-1) - \tanghperc*(e-2)$);
\coordinate (tang-corner-3) at ($(p) - \tangwperc*(e-1) - \tanghperc*(e-2)$);
\coordinate (tang-corner-4) at ($(p) - \tangwperc*(e-1) + \tanghperc*(e-2)$);

\draw[tang-border] (tang-corner-1) -- (tang-corner-2) -- (tang-corner-3) -- (tang-corner-4) -- cycle;
\shade[tang-space] (tang-corner-1) -- (tang-corner-2) -- (tang-corner-3) -- (tang-corner-4) -- cycle;

\node[blue, anchor = south east] at (tang-corner-4) {$\mathbb{T}_p X$};

% The point

\node[nodepoint, label={left:$p$}] at (p) {};

% Differential representation - above the tangent space

\draw[tang-vector] (p) -- ($(p) + 0.25*(e-1) + 1*(e-2)$) node[anchor=south] {$\xi$};

% Our reference frame

\draw[ref-vec] (p) -- ($(p) + (e-1)$) node[anchor = west] {$e_1$};
\draw[ref-vec] (p) -- ($(p) + (e-2)$) node[anchor = south] {$e_2$};
\draw[ref-vec] (p) -- ($(p) + (e-3)$) node[anchor = south] {$e_3$};

\draw[perp-marker] (p) -- ($(p) + 0.1*(e-1)$) -- ($(p) + 0.1*(e-1) + 0.1*(e-3)$) -- ($(p) + 0.1*(e-3)$) -- cycle;
\draw[perp-marker] (p) -- ($(p) + 0.1*(e-2)$) -- ($(p) + 0.1*(e-2) + 0.1*(e-3)$) -- ($(p) + 0.1*(e-3)$) -- cycle;

% The new reference frame

\coordinate (ep-1) at (0.885501, -0.193905, 0.352431);
\coordinate (ep-2) at (0.386244, 0.0965609, -0.917329);
\coordinate (ep-3) at (0.169066, 0.954725, 0.100497);

\draw[ref-vec-next] (p-2) -- ($(p-2) + (ep-1)$) node[anchor = north] {$e_1'$};
\draw[ref-vec-next] (p-2) -- ($(p-2) + (ep-2)$) node[anchor = west] {$e_2'$};
\draw[ref-vec-next] (p-2) -- ($(p-2) + (ep-3)$) node[anchor = west] {$e_3'$};

\draw[perp-marker-next] (p-2) -- ($(p-2) + 0.1*(ep-1)$) -- ($(p-2) + 0.1*(ep-1) + 0.1*(ep-3)$) -- ($(p-2) + 0.1*(ep-3)$) -- cycle;
\draw[perp-marker-next] (p-2) -- ($(p-2) + 0.1*(ep-2)$) -- ($(p-2) + 0.1*(ep-2) + 0.1*(ep-3)$) -- ($(p-2) + 0.1*(ep-3)$) -- cycle;

\draw[ref-vec-old] (p-2) -- ($(p-2) + (e-1)$) node[anchor = west] {$e_1$};
\draw[ref-vec-old] (p-2) -- ($(p-2) + (e-2)$) node[anchor = east] {$e_2$};
\draw[ref-vec-old] (p-2) -- ($(p-2) + (e-3)$) node[anchor = north east] {$e_3$};

% The differentials!
\draw[differential] (p) -- (p-2) node[midway, below, sloped] {$\mathrm{d}M(\xi)$};
\draw[differential] ($(p-2) + (e-1)$) -- ($(p-2) + (ep-1)$) node[midway, above, sloped] {$\mathrm{d}e_1(\xi)$};
\draw[differential] ($(p-2) + (e-2)$) -- ($(p-2) + (ep-2)$) node[midway, above, sloped] {$\mathrm{d}e_2(\xi)$};
\draw[differential] ($(p-2) + (e-3)$) -- ($(p-2) + (ep-3)$) node[midway, above, sloped] {$\mathrm{d}e_3(\xi)$};

\end{tikzpicture}
