% -*- root: ../GeometriaDiferencial.tex -*-
\chapter{Variedades diferenciables}

\begin{defn}[Variedad\IS diferenciable] Se dice que $X$ es una variedad diferenciable si

\begin{enumerate}
\item $X$ es un espacio topológico con topología conexa y Hausdorff.
\item $U_i ⊂ X$ es una familia numerable de abiertos con aplicaciones $\appl{Φ_i}{U_i}{ℝ^n}$, homomorfismos sobre sus imágenes. El par $(U_i, Φ_i)$ es una \concept[Carta\IS coordenada]{carta coordenada}.
\item Dadas dos cartas coordenadas $U_i, U_j$ y sus correspondientes homomorfismos $Φ_i, Φ_j$, la función $Φ_j ○ \inv{Φ_i}$ es difeomorfismo en el entorno en el que está definida (en $U_i ∩ U_j$).
\end{enumerate}

Si sólo cumple las dos primeras propiedades, $X$ sería sólo una \concept[Variedad\IS topológica]{variedad\IS topológica}.
\end{defn}

Vamos a ver varios comentarios desordenados sobre este tema. En muchos casos pediremos más condiciones a $X$, como por ejemplo que sea compacta para tener un \concept[Atlas]{atlas} (el conjunto de todas las cartas) finito.

Otro comentario es que no perdemos nada por pedir conexión, ya que si tenemos varias componentes conexas tendremos distintas variedades.

Localmente, todas las variedades son conexas y compactas, al tener una carta para cada punto. La conexión es sencilla, la compacidad sólo hay que ver que si se coge una bola cerrada en la variedad, el homomorfismo $Φ_i$ la llevará a una bola cerrada en $ℝ^n$, que es compacta y por lo tanto la bola es compacta también en la variedad.

Otra cosa a recordar es que una función continua e inyectiva de un espacio topológico compacto a un espacio Hausdorff es necesariamente un homomorfismo con su imagen.

Una función inyectiva y diferenciable no tiene por qué ser un difeomorfismo: por ejemplo, $f(x) = x^3$ es biyectiva y diferenciable pero no es difeomorfismo porque en $0$ la derivada se anula. Es decir, falla porque no se puede aplicar el teorema de la función inversa \eqref{thmInv}.

Viendo la definición de las variedades, localmente no son interesantes porque son equivalentes a abiertos de $ℝ^n$. Lo interesante es ver las propiedades globales. Cómo unas cartas se pegan con otras es lo que da lugar a la geometría global de la diferencial.

Por supuesto, también hay aspectos locales, como por ejemplo la curvatura. Pero aun así, topológicamente localmente es igual a $ℝ^n$.

\section{Ejemplos de variedades diferenciables}

Un ejemplo muy sencillo es un abierto $X$ en $ℝ^n$ o todo $ℝ^n$, donde las cartas son la identidad.

Si $X,Y$ son variedades, su producto cartesiano $X × Y$ es variedad también. Las cartas serán los producto $U_i×U_j$ con los homomorfismos $Φ_i × Ψ_j$, y es fácil ver que las cartas producto cumplen las propiedades.

Un ejemplo típico de una variedad producto es $\crc × ℝ$, el cilindro; o $\crc × \crc$, el toro.

Las esferas $\bbs^n$ están en $ℝ^{n+1}$, dadas por la ecuación $x_0^2 + \dotsb + x_n^2 = 1$. La forma más fácil de producir las cartas para la esfera es coger los casquetes, y sólo habría que comprobar que ese solapamiento en un entorno del ecuador no da problemas.

El ejemplo más interesante de todos estos es, sin embargo, el espacio proyectivo. Por definición, es el conjunto de rectas que contienen al origen. Más formalmente, \[ \projp^2 ≝ \quot{ℝ^3 \setminus \set{0}}{\sim} \] donde $\sim$ es una relación de equivalencia definida de la siguiente forma: dados $e, e' ∈ ℝ^3$, están relacionados $e \sim e'$ si y sólo si $∃λ ≠ 0$ tal que $λe = e'$.

Primero hay que dotar de una topología a este espacio. Buscamos\footnote{Parece que la hemos encontrado pero no sé cuál es.} la topología con más abiertos tal que la aplicación $\appl{π}{\real^3 \setminus \set{0}}{\projp^2}$ es continua.

Esa topología es conexa y Hausdorff, y también compacta.

Las cartas son tres, dadas por $U_i = \set{[x_0, x_1, x_2] \tq x_i ≠ 0}$. Los homomorfismos estarán dados por $\appl{Φ_i}{U_i}{ℝ^2}$ llevando $[x_0, x_1, x_2]$ a $\left[\frac{x_0}{x_i},\frac{x_1}{x_i},\frac{x_2}{x_i}\right]$. Dado que una de las coordenadas va a ser $1$ siempre, podemos quitarla y entonces será equivalente a $ℝ^2$.

Hay que ver el cambio de carta, por ejemplo de $U_0$ a $U_1$. Y funciona bien, sólo hay que resolver algunas ecuaciones.

Podemos ver que es posible definir planos proyectivos de $n$ dimensiones por \[ \projp^n ≝ \quot{ℝ^{n+1}\setminus\set{0}}{\sim}\] con $\sim$ la misma relación de equivalencia de antes.

¿Por qué es interesante el plano proyectivo? Porque las rectas en $L ⊂ \projp^2$ son planos vectoriales en $ℝ^3$. Como los planos vectoriales en $ℝ^3$ siempre se cortan (contienen al menos al origen) en una recta, entonces dos rectas en $\projp^2$ siempre se van a cortar en un punto, luego no hay rectas paralelas.

Es decir, que en la geometría proyectiva nos quitamos varias excepciones feas de rectas paralelas. En momentos más avanzados, cuando se cambian los reales por el cuerpo complejo, se pasa a ver que las curvas siempre se cortan o algo así.

También es importante ver que, al ser compacto, permite llegar a cosas finitas que en el euclídeo se quedan infinitas.

Algo interesante del plano proyectivo es que es equivalente a tapar una banda de Möbius con un disco. Vamos a ver por qué.

El plano proyectivo se obtiene identificando planos antipodales de la esfera, luego nos sobra un hemisferio. Topológicamente, un hemisferio es como un rectángulo, en el que identificamos los lados antipodalmente.\footnote{Dibujo aquí correspondiente.}

En esa misma representación, la banda de Möbius sería un rectángulo en el que identificamos los lados cortos. La banda es lo mismo que recortar el plano proyectivo por los lados largos.

Como el plano proyectivo contiene la banda de Möbius, resulta que tampoco es orientable.

\section{Morfismos entre variedades}

Después de ver qué es una variedad diferenciable, es interesante ver cuáles son las funciones ``diferenciables'' entre ellas y que conservan sus propiedades. Es decir, si tenemos $X,Y$ variedades diferenciables, ¿qué podemos decir de $\appl{F}{X}{Y}$?

\begin{defn}[Función\IS diferenciable] Sea $\appl{F}{X_n}{Y_m}$\footnote{Usamos la notación $X_n$ para indicar que $X$ tiene dimensión $n$.} una aplicación entre variedades diferenciables. Si para toda carta $(U_i, Φ_i)$  en $X$ y para toda carta $(V_j, Ψ_j)$ en $Y$, la composición $Ψ_j ○ F ○ \inv{Φ_i}$ es diferenciable ($C^∞$) entre abiertos de $ℝ^n$ y $ℝ^m$ en todo punto donde está definida.

Esto es lo mismo que decir que $F$ es compatible con las cartas.
\end{defn}

\begin{defn}[Difeomorfismo] Una función diferenciable entre variedades es una función diferenciable, biyectiva y con inversa diferenciable.
\end{defn}

En particular, si $F$ es difeomorfismo la dimensión de las dos variedades tiene que ser la misma.

Dos variedades difeomorfas son esencialmente la misma, a todos los efectos. La única diferencia será el nombre de la variedad y el nombre de los puntos. Así, en el fondo, lo que querremos es tener criterios para clasificar las variedades por difeomorfismos, y ver cuándo dos variedades no pueden ser la misma.

Por ejemplo, una de las formas de ver que dos variedades son diferentes es ver que la cohomología de DeRhan (ver \ref{defCohomologiaDeRhan}) es distinta. De la misma forma, todas las invariantes que definamos (curvatura, por ejemplo) también pueden ayudarnos a distinguir entre variedades, o incluso entre distintas estructuras diferenciables (exóticas) del mismo espacio (ver \ref{secMismoEspacioNoDifeomorfo} para ampliar).

\subsection{Equivalencia entre atlas}

Aquí me falta la definición de equivalencia entre atlas.

Tomemos dos atlas, uno que nos lleva de $ℝ_1 \overset{Id}{\longmapsto} ℝ$ y otro $ℝ_2 \overset{x^3}{\longmapsto} ℝ$. La cuestión es es que estos dos atlas no son equivalentes ($x^3$ no es un difeomorfismo), a pesar de que las variedades $ℝ_1, ℝ_2$ sí son difeomorfas con el difemorfismo $x^3$, tendríamos \[ ℝ \overset{\sqrt[3]{x}}{\longmapsto} ℝ_2 \overset{x^3}{\longmapsto} ℝ_1 \overset{Id}{\longmapsto} ℝ \]

Aquí falta una clase entera.


