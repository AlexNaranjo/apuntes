% -*- root: ../GeometriaDiferencial.tex -*-
\chapter{Integración en variedades: teorema de Stokes}
\label{chap:Integracion}

Ahora lo que vamos a querer va a ser integrar en variedades. La cuestión es que tenemos un problema: las variedades se definen ``a cachos'', usando cartas (ver \fref{def:VariedadDiferenciable}). Luego vamos a tener que adaptar las integrales que ya conocemos para que funcionen bien en las cartas. Necesitaremos entonces las particiones de la unidad: una forma de decir cuánto peso le damos a cada carta en cada punto para que todo sea continuo, bonito y magnífico.

\section{Particiones de la unidad}

\begin{figure}[hbtp]
\centering
\inputtikz{IV_ParticionesUnidad}
\caption{Las ``montañas'' son las particiones $ρ_i$ de la unidad que buscamos, funciones que para cada $U_i$ (el disco debajo de cada montaña) nos dan una ponderación determinada, de tal forma que la suma de todos los $ρ_i$ en un determinado punto es uno. Si estamos en un punto de la variedad que sólo está cubierto por una carta, entonces $ρ_i(x) = 1$. Si hay varias cartas, la unidad se divide dando un peso determinado a la partición de cada carta.}
\label{fig:ParticionesUnidad}
\end{figure}

\begin{defn}[Partición\IS de la unidad] Una partición de la unidad subordinada a un cubrimiento de cartas $U_i$ de una variedad $X = \bigcup U_i$ es una familia de funciones $ρ_i ∈ C^∞(X)$ tales que \begin{itemize}
\item $1 ≥ ρ_i ≥ 0$.
\item $\mop{sop} ρ_i ⊂ U_i$.\footnote{El soporte de $ρ_i$ es la clausura del conjunto de puntos en los que $ρ_i$ no es nula. Ver \fref{def:Soporte}.}
\item $\sum_i ρ_i (x) = 1\quad ∀x∈X$.
\end{itemize}
\label{def:ParticionUnidad}
\end{defn}

Para que esta definición tenga sentido, tiene que ocurrir que en todo punto sólo haya una cantidad finita de $ρ_i$ que no tengan valor nulo. Por facilitarnos las cosas, supondremos que $X$ es variedad compacta y que por lo tanto el cubrimiento $\set{U_i}_{i=1}^N$ es finito.

Usaremos las particiones de la unidad para pasar de información local a información global. Localmente, las variedades sólo son abiertos de $ℝ^n$, pero globalmente no son triviales.

Suponemos de momento que las particiones de la unidad existen en ciertas condiciones, y vamos a ver un ejemplo algo tonto. Queremos demostrar que si $X$ es una variedad compacta, existe en $X$ un campo $D$ que no es idénticamente nulo.

Tomamos un abierto $U_1 ⊂ X$ con coordenadas $x_1, \dotsc, x_n$. Un campo no nulo sería $D = \dpa{}{x_1}$. La cuestión es que este campo es local, sólo está definido en una carta, no en toda la variedad. Queremos extenderlo a $X$ de tal forma que no sea nulo en todo punto de $X$.

Lo que hacemos es tomar un cubrimiento que contenga a $U_1$ y hallamos una partición de la unidad subordinada a este cubrimiento. Tendremos entonces una función $ρ_1 ∈ C^∞(X)$ que es nula fuera de $U_1$ (segunda condición de partición), así que \[ \hat{D} ≝ ρ_1 D\] es un campo continuo definido en todo $X$ y que no es nulo en $U_1$.

\seprule

Para demostrar el teorema de existencia de particiones de la unidad, necesitaremos lemas adicionales.

\begin{lemma} Existe $\appl{φ}{\bola_3(0)}{ℝ}$, con $C^∞$ y  $\bola_3(0) ⊂ ℝ^n$, tal que
\begin{itemize}
\item $φ(p) = 1$ si $p ∈ \bola_1(0)$.
\item $φ(p) ∈ (0,1]$ si $p ∈ \bola_2 (0)$.
\item $φ(p) = 0$ si $p ∈ \bola_3(0) \setminus \bola_2 (0)$.
\end{itemize}
\end{lemma}

Esta función φ es una función ``meseta'' porque parece una meseta: el centro está elevado, luego baja y luego es cero. Un pequeño apunte es que estas funciones no son analíticas, así que si quisiésemos extender esto a varaible compleja lo tendríamos más difícil.

\begin{figure}[hbtp]
\centering
\inputtikz{IV_LemaI}
\caption{Esta es la función ``media meseta'' definida como β en la demostración.}
\label{fig:IV_Lema1}
\end{figure}

\begin{proof}
Sea \[ α(t) ≝ e^{-\frac{1}{(t+1)^2(t+2)^2}} \] con $t ∈ (-2, -1)$ y $α(t)= 0$ en otro caso. Definimos también su integral \[ γ(t) ≝ \int_{-∞}^t α(s) \dif s \] y el valor en todo el intervalo \[ A = \int_{-2}^{-1}α(s) \dif s \]. Por último, haremos una ``normalización'' de la integral: $β(t) ≝ \frac{γ(t)}{A}$.

Esta función $β$ va a tener forma de ``media meseta'' (ver \fref{fig:IV_Lema1}), lo que podemos usar para construir la $φ$ que buscamos. De hecho, definiremos \[ φ(\vx) ≝ β(-\md{\vx}) \] con $\vx ∈ ℝ^n$.
\end{proof}

Con esto vamos a poder demostrar el teorema.

\begin{proof}
Tomamos, para $x∈X$, funciones $\appl{f_x}{\bola_3(0)}{X}$ tales que \[ f_x(\bola_3(0)) ≝ V_x ⊂ U_i \] (lema 2 que no he copiado). Se define ahora $w_x ≝ f_x(\bola_1(0)) ⊂ V_x ⊂ U_i$. Tenemos una colección no numerable de $w_x$, pero por compacidad puedo quedarme con un conjunto finito $\set{w_i}_{i=1}^N$ que recubren toda la variedad.

Definimos entonces las funciones \[ θ_i ≝ \appl{φ○\inv{f_i}}{V_i}{ℝ} \], que serían las ``mesetas'' en los abiertos $U_1$. En ese caso, ya tenemos las $φ_i$ dadas por \[ φ_i ≝ \frac{θ_i}{\sum_{j=1}^N θ_j}\]
\end{proof}

\section{Integración}

Para integrar, lo que haremos será integrar $n$-formas $ω$ en variedades $X$ $n$-dimensionales. Lo primero que queremos hacer es definir una expresión como $\int_X ω$. Lo más fácil es ir caso a caso.

Supondremos en esta sección $X$ compacta para tener soportes compactos y $ω$ como $n$-formas dadas por \[ ω = a_1(x_1, \dotsc, x_n) \dfl{x_1}{x_n} \]

El caso más fácil es cuando el soporte (ver \fref{def:Soporte}) está en una única carta, esto es, $\mop{sop}ω ⊂ U_i$. Entonces no hay ambigüedad para definir la integral. Tendremos una parametrización $\appl{f_i}{D ⊂ X ⊂ ℝ^n}{U_i}$ de la carta, luego definiremos la integral como \( \int_X ω ≝ \int_U f_i^* ω = \int_{ℝ^n} a_1(x_1, \dotsc, x_n) \dfl{x_1}{x_n} \label{eq:IntegralCartas}\)

El caso interesante es cuando está en dos cartas o más. Aquí hay que dar una definición que no dependa\footnote{Y así ha dejado la frase. Entiendo que se refiere a la carta que tomemos}

Aquí, tenemos que el soporte $K = \mop{sop} ω$ está en dos cartas, esto es, $k ⊂ U_i ∩ U_j$. Tenemos las dos parametrizaciones correspondientes $f_i, f_j$. Según lo que hemos escrito en el caso anterior, querríamos decir que \[ \int_X ω ≝ \int_{ℝ^n} f_i^* ω \qeq \int_{ℝ^n} f_j^* ω \]

La cuestión es que tenemos que demostrar que esas dos integrales son iguales. Para ello, vemos que las dos formas $f_i^* ω, f_j^* ω$ están relacionadas por el difeomorfismo de cambio de variable: \[ \left(\inv{f_j} ○ f_i\right)^* \left(f_j^* ω\right) = f_i^* ω\], cosa que ya está vista\footnote{¿Dónde? Vaya usted a saber.}. Sabemos además que $\inv{f_j} ○ f_i$ es un difeomorfismo.

Queremos aplicar ahora la fórmula de cambio de variable que nos dice que, dada ciertas condiciones sobre $\appl{F}{V⊂ℝ^n}{U⊂ℝ^n}$, tenemos que \[ \int_V F^* ω = \int_U ω\], una fórmula que incluye el jacobiano que conocemos de cursos de cálculo. Pero hay un problema ahí: el jacobiano saca el valor absoluto del determinante. Es decir, que en realidad la fórmula es \begin{multline} \int_{ℝ^n} a_1(y_1(x_1, \dotsc, x_n), \dotsc) \abs{\det F^*} \dfl{x_1}{x_n} = \\ = \int_{ℝ^n} a_1(y_1, \dotsc, y_n) \dfl{y_1}{y_n} \label{eq:JacobianoIntegral} \end{multline}

Luego, para que \eqref{eq:IntegralCartas} se cumpla, no puede cambiar el signo del jacobiano. Si tenemos múltiples cartas en un entorno, los jacobianos tienen que tener todos el mismo signo. Así, llegamos a la definición de orientable:

\begin{defn}[Orientable] Se dice que una variedad es orientable si, y sólo si, existe un atlas tal que el jacobiano entre todas sus cartas tiene el mismo signo. Por convención, tomaremos signo positivo.
\end{defn}

Necesitaremos entonces que la variedad sea orientable para poder definir bien la integral en la \fref{eq:IntegralCartas}, ya que en ese caso el determinante en la \fref{eq:JacobianoIntegral} se puede quitar sin problemas. Por concretar notación, dos definiciones.

\begin{defn}[Orientación] \label{def:OrientacionCutre} Una orientación es un atlas orientable.
\end{defn}

\begin{defn}[Orientación\IS compatible] Diremos que dos orientaciones o atlas orientables son compatibles si el jacobiano del cambio entre cualquier carta de cada atlas tiene signo positivo.
\end{defn}

Ahora que ya sabemos cómo integrar en cualquier abierto que esté contenido en una o varias cartas, sólo nos falta combinar todo ello para integrar en una variedad entera, usando las particiones de la unidad de la sección anterior.

\begin{defn}[Integración\IS de $n$-formas] Dada $X$ una variedad $n$-dimensional compacta y orientable y $ω$ una $n$-forma en $X$, se define la integral de $ω$ en $X$ como \[ \int_X ω ≝ \sum_i  \int_{U_i} ρ_i ω\] donde $\set{(U_i ρ_i)}$ es una partición de la unidad (ver \fref{def:ParticionUnidad}). \label{def:IntegracionNForma}
\end{defn}

Nos falta ver, eso sí, que la integral no depende de la elección de la partición de la unidad. Suponemos que existen dos particiones $\set{(U_i, ρ_i)}, \set{(V_j, τ_j)}$. Entonces \[ \int_X ω ≝ \sum_i  \int ρ_i ω = \sum_i \int_X ρ_i \underbrace{\left(\sum_j τ_j\right)}_{ = 1} ω = \sum_j \left(\int_X τ_j \underbrace{\left(\sum_i ρ_i\right)}_{ = 1} ω \right) = \sum_j \int_X τ_j ω \]

Una cosa que hay que notar es que uno no va a calcular muchas integrales con esta definición. El uso de todo esto es teórico, no vamos a hacer muchos ejercicios de esto porque es un lío y no hay quién lo calcule.

Conceptualmente, eso sí, es importante, de forma similar a la integración por partes. El teorema de Stokes es la generalización más importante de la regla de Barrow\footnote{Creo que se refiere al teorema fundamental del cálculo pero no lo estoy mirando.}, que se usa mucho en cálculo; y de la misma forma usaremos el análogo partiendo de Stokes en Geometría y análisis de variedades.

\section{Orientación}
\label{sec:Orientacion}

Acabamos de definir cómo integramos formas en una variedad, pero nos hemos dejado por el camino el definir más formalmente lo que es la orientación. La \fref{def:OrientacionCutre} que hemos dado de orientación en es cutre, por decirlo de forma suave. Lo único que hemos dicho es que queremos que el jacobiano del cambio de coordenadas no cambie de signo en entornos en los que haya varias cartas.

La justificación formal que dábamos para buscar eso es que queremos poder quitar un valor absoluto en la \fref{eq:JacobianoIntegral}. La idea que hay detrás es que queremos que la integral valga lo mismo tanto si empezamos por $U_1$ y seguimos por la carta $U_2$ que si lo hacemos al revés: si la integral cambia según cómo nos dé por integrar no nos va a servir para mucho.

\begin{figure}[hbtp]
\centering
\inputtikz{IV_CambioOrdenVectores}
\caption{$F$ sería un cambio que nos conserva la orientación, ya que para pasar de $e_1$ a $e_2$ seguimos teniendo que girar hacia la izquierda. Si hiciésemos su jacobiano, el determinante sería positivo. Sin embargo, $G$ tiene determinante negativo (estamos cambiando simplemente un vector por otro en la matriz) y efectivamente nos cambia el sentido de giro y por lo tanto la orientación.}
\label{fig:CambioOrdenVectores}
\end{figure}

Si buceamos un poco más en el significado del jacobiano, en el fondo es sólo una matriz de cambio de base entre las coordenadas de una carta y de la otra. Si el signo del determinante es negativo, significa que, además de alargar, acortar y girar los vectores, en algún momento los hemos cambiado de orden. No sería malo si no fuese porque eso nos cambia el sentido de giro (ver \fref{fig:CambioOrdenVectores} para una idea intuitiva).

Así, lo único que estamos haciendo al orientar es dar un sentido de giro, de tal forma que el ``orden'' de los vectores siempre se mantiene y que, por así decirlo, nos permite movernos siempre de la misma manera.

La cuestión es que de momento no estamos orientando más que espacios vectoriales, y nosotros lo que queremos orientar es variedades. Por suerte, no es muy difícil extrapolar lo que estábamos haciendo.

Primero, necesitamos formalizar un poco la definición de orientación en espacios vectoriales para poder seguir.

\begin{defn}[Orientación\IS de espacios vectoriales] Se dice que dos bases de un espacio vectorial $E$, $\base_1 = \set{e_1, \dotsc, e_n}, \base_2 = \set{e_1', \dotsc, e_n'}$, tienen la misma orientación si el determinante de la matriz de cambio de base es positivo.

La orientación define una relación de equivalencia en todas las bases de $E$, y de hecho divide ese espacio de bases en dos clases de equivalencia. \label{def:OrientacionEV}
\end{defn}

Partiendo de esta definición formal podemos darnos un paseo y llegar a las variedades diferenciables. Vamos a fijarnos en $\bigwedge^n (E)$, el espacio de $n$-formas diferenciales. Resulta que esas $n$-formas son como $ω = λ · \dfl{e_1}{e_n}$, y en ellas podemos encontrar la misma definición de orientación de antes (\fref{def:OrientacionEV}): es muy fácil ver que si dos formas $ω_1, ω_2$ tienen el mismo signo entonces dan la misma orientación en el espacio vectorial. Si no la diesen, habría una diferencial cambiada de sitio (recordemos que $\dif e_1 \y \dif e_2 = - \dif e_2 \y \dif e_1$) y eso nos cambiaría el signo.

Entonces, una orientación no es más que encontrar una $n$-forma diferencial $ω$ que no se anula, y que por lo tanto nos da la misma orientación en todo el espacio en el que está definida. Una demostración más formal de esto que acabo de comentar se encuentra en \cite[Sección V.7]{boothby03}, aunque con lo que tenemos ya nos vale para formalizar la idea de orientación en variedades.

\begin{defn}[Orientación\IS en variedades] $X$ es orientable si y sólo si existe una $n$-forma $ω$ en $X$ tal que $ω_x ≠ 0\;∀x∈X$. \label{def:OrientableForma}
\end{defn}

Esta definición de orientación es, desde luego, muchísimo más cómoda. Nos permitirá decir, por ejemplo, cuando un difeomorfismo $\appl{F}{X_1}{X_2}$ entre variedades conserva la orientación: simplemente tiene que cumplirse que, dadas dos formas de orientación $ω_1, ω_2$ para $X_1$ y $X_2$ respectivamente, se cumpla que $F^*ω_2 = λ ω_1$ con $λ > 0$ (en otras palabras, que el signo no cambie al movernos de una variedad a otra).

A la $n$-forma $ω$ se le llama la \concept[Forma\IS de volumen]{forma de volumen}, así que se puede definir el \concept[Volumen]{volumen de una variedad}\footnote{Por supuesto, si tenemos una superficie será el área.} como \[ \mop{Vol}(X) ≝ \int_X ω\]

Hay múltiples elecciones posibles para la forma de volumen, como os podéis imaginar. Cada una de esas formas difieren únicamente en la multiplicación por una función positiva, que intuitivamente podríamos decir que es la ``escala'' con la que miramos el volumen. Por ser más concretos, una forma de volumen podría decirnos que el cubo formado por los vectores de la base tiene volumen $1$, y otra podría decirnos que eso tiene volumen $10$.

Ahora bien, lo que normalmente haremos será no complicarnos la vida y elegir una forma de volumen muy simple, $\dfl{e_1}{e_n}$. Precisamente esta será la $n$-forma que viene dada al elegir una base ortonormal según la métrica Riemanniana en la variedad (veremos esto con más detalle en el capítulo \ref{chap:GeometriaRiemman}).


Lo que no hemos hecho todavía ha sido demostrar que las dos definiciones de orientación que acabamos de dar (\fref{def:OrientacionCutre}, \fref{def:OrientableForma}) son en realidad la misma. Vamos a ello.

\begin{prop} Sea $X$ una variedad $n$-dimensional con un atlas $(U_i, f_i)$. Para $i,j$ con $U_i ∩ U_j ≠ ∅$, el determinante del jacobiano cambio de coordenadas $f_i ○ \inv{f_j}$ es positivo si y sólo si existe una $n$-forma $ω_x$ que no se anula en ningún punto de $X$.

Dicho de otra forma, las definiciones \ref{def:OrientacionCutre} y \ref{def:OrientableForma} son equivalentes. \label{prop:EquivalenciaOrientacion}
\end{prop}

\begin{proof}

Tenemos $X$ una variedad $n$-dimensional orientada con $(U_i, ρ_i)$ una partición de la unidad de $X$. Si queremos que sea orientable según la \fref{def:OrientableForma}, buscamos una $n$-forma $ω$ que no se anula en ningún punto de la variedad.

Una $n$-forma que no se anula en una carta es $ω_i ≝ \dfl{x_1}{x_n}$ donde $x_1, \dotsc, x_n$ son las coordenadas de $U_i$. Así, podemos definir la $n$-forma globalmente como \[ ω ≝ \sum ρ_i ω_i \]

Tenemos que demostrar, eso sí, que $ω_x ≠ 0\; ∀x ∈ X$. El problema lo podríamos tener cuando $x$ está en varias cartas. Supongamos $x ∈ U_1$. Entonces, \[ ω_x = ρ_1 \dfl{x_1}{x_n} + ρ_2 J_2 \dfl{x_1}{x_n} + \dotsb \] donde $ρ_1 ≥ 0$ y $J_2$ es el jacobiano del cambio de carta de $U_1$ a $U_2$, que por ser $X$ orientable es positivo. Tenemos entonces que todos los coeficientes son positivos y por lo tanto $ω_x$ es no nulo.


En el otro sentido, tenemos una variedad $X$ $n$-dimensional y $ω$ una $n$-forma tal que $ω_x ≠ 0\; ∀x∈X$. Para un entorno $U$ de $x∈X$, sea $\appl{Ψ_x}{U}{ℝ^n}$ que es un difeomorfismo con su imagen. Entonces $(\inv{Ψ_x})^* ω = f \dfl{x_1}{x_n}$ con $f$ definida en un abierto de $ℝ^n$ y positiva.

Con el conjunto de todos los $Ψ_x$ podemos sacar un atlas para $X$, y queremos ver que el jacobiano del cambio es positivo siempre. Sean $Ψ_j, Ψ_i$ dos difeomorfismos que se intersecan. Tenemos entonces \begin{align*}
(\inv{Ψ_i})^* ω &= f \dfl{x_1}{x_n} \\
(\inv{Ψ_j})^* ω &= g \dfl{y_1}{y_n}
\end{align*} con $f,g > 0$.

Vamos a escribir entonces la diferencial $\dfl{y_1}{y_n}$ en función de las variables $x_1, \dotsc, x_n$. Nos quedará \[ g(y_1(x_1, \dotsc, x_n), \dotsc) \det \left(\dpd{y_i}{x_j}\right)_{i,j} \dfl{x_1}{x_n} \] que es igual a $f(x_1, \dotsc, x_n)\dfl{x_1}{x_n}$. Como por hipótesis $f,g > 0$ entonces el determinante ese tiene que ser positivo.
\end{proof}

\section{Borde de una variedad}

Una vez que hemos llegado hasta aquí, ya sabemos integrar variedades, aunque no todas. No hemos visto qué ocurre con variedades con borde. Hasta ahora, todas las definiciones de integral se basaban en usar entornos abiertos del punto, y si usamos entornos abiertos no vamos a poder llegar en ningún momento a un borde.

¿Cómo formalizamos el concepto de borde? Usando un poco la imaginación, como siempre. Si para definir las variedades normales usábamos lo más simple que se nos ocurría como base (los abiertos $U_i ∈ ℝ^n$ como espacios de coordenadas), para hacerlas con borde usaremos el espacio más simple que se nos pueda ocurrir que tenga un borde, y este espacio no es más que ``medio espacio'' \[ \halfp^n ≝ \set{(x_1, \dotsc, x_n) ∈ ℝ^n \tq x_1 ≤ 0} \], que tiene como borde \[ ∂\halfp^n ≝ \set{(0, x_2, \dotsc, x_n) ∈ ℝ^n }\]

Por ejemplo, $H^2$ sería el lado izquierdo del plano (con $x$ negativo) y su borde sería el eje $x = 0$.

\begin{figure}[hbtp]
\centering
\inputtikz{IV_VariedadConBorde}
\caption{Hasta ahora hemos definido variedades sin borde usando abiertos de $ℝ^n$, como el caso del parche naranja. Para hacerlas con borde, permitiremos usar abiertos de $\halfp^n$ como espacios de coordenadas como el parche naranja.}
\label{fig:VariedadConBorde}
\end{figure}

Con esta base ahora podemos definir una variedad con borde, cambiando dónde están los espacios de coordenadas para que podamos construir bordes.

\begin{defn}[Variedad\IS con borde] Diremos que una variedad $X$ es una variedad con borde $n$-dimensional si $X = \bigcup U_i$ con las cartas definidas por homeomorfismos \[ \appl{Ψ_i}{U_i}{V_i ⊆ \halfp^n} \]

Como observación, no es necesario que exista algún $V_i$ que contenga una parte del borde de $\halfp^n$. Si ocurre esto, tenemos una variedad con borde vacío.
\end{defn}

Lo único que tenemos que hacer es concretar qué significa exactamente que la función de la carta sea diferenciable en el borde. Sea $V ⊂ \halfp^n$ un abierto y $\appl{f}{V}{ℝ}$ una función. Si $V$ no contiene ninguna parte del borde estamos en un abierto normal de $ℝ^n$, y entonces sabemos decir perfectamente qué quiere decir que $f$ sea diferenciable. Cuando sí tenga partes del borde, diremos que $f$ es diferenciable si se puede extender a una función diferenciable que, restringida a $V$, sea $f$. Esto es, si $V ∩ ∂\halfp^n ≠ ∅$, diremos que $f$ es diferenciable si y sólo si existe un abierto $U ⊂ ℝ^n$ con $V⊂U$ y una función $\appl{\gor{f}}{U}{ℝ}$ tal que $\restr{\gor{f}}{V} = f$.

Una vez que tenemos hecho esto, podemos definir formalmente el borde.

\begin{defn}[Borde\IS de una variedad] Dada una variedad con borde $X$, su borde (denotado por $∂X$) se define como \[ ∂X ≝ \set{x ∈ X \tq ∃Ψ_i \; Ψ_i(x) ∈ ∂\halfp^n} \]
\end{defn}

Tenemos que ver, eso sí, que el borde está bien definido y que no varía si cogemos otra carta.

\begin{prop} El borde de $X$ está bien definido y es independiente de la carta escogida. \end{prop}

\begin{proof} Sea $x ∈ X$ un punto en la intersección de dos cartas $U_i, U_j$. Vamos a demostrar la proposición por reducción al absurdo: supongamos que con $U_i$ está en el borde pero no con $U_j$. Sean $f_i, f_j$ las parametrizaciones respectivas, y sean $x_i = \inv{f_i}(x),\, x_j = \inv{f_j}$.

Como $x_j$ no está en el borde según la carta $U_j$, tenemos un entorno $V ⊂ \halfp^n$ cuya intersección con el borde de $\halfp^n$ es vacía, esto es, $V ∩ ∂\halfp^n = ∅$.

Podemos construir el difeomorfismo $\inv{f_i}○f_j$ por ser un cambio de carta. $V$ tiene que tener una imagen en $U_i$ que sea un entorno de $x_i$. Ese entorno debería tener puntos fuera de $\halfp^n$, una contradicción.
\end{proof}

\begin{prop} $∂X$ es una variedad $(n-1)$-dimensional y orientada. \end{prop}
\begin{proof} Demostrar que es una variedad es fácil porque no sé qué.

Vamos ahora a demostrar la orientación. Supongamos que tenemos dos cartas $U_x, U_y$ con determinante del jacobiano del cambio de carta positivo y con coordenadas $x_1, \dotsc, x_n$ y $y_1, \dotsc, y_n$ respectivamente. En la primera carta, el borde estará dado cuando $x_1 = 0$ y en la segunda que $y_1 = 0$.

Estudiamos el determinante del jacobiano: \[ \left|\begin{matrix}
\dpd{y_1}{x_1} & \dpd{y_1}{x_2} & \dots & \dpd{y_1}{x_n} \\
\dpd{y_2}{x_1} & \dots & \dots & \dpd{y_2}{x_n} \\
\vdots &  \dots & \ddots & \vdots \\
\dpd{y_n}{x_1} & \dpd{y_n}{x_2} & \dots & \dpd{y_n}{x_n} \\
\end{matrix}\right| \]

Como $y_1 = 0$, tenemos que $\pd{y_1}{x_i} = 0$ para $i ≠ 1$. Por lo tanto, sólo nos interesa el menor ese. Y no sé por qué induce una orientación.
\end{proof}

\section{Teorema de Stokes}

\begin{theorem}[Teorema\IS de Stokes] \label{thm:Stokes} Sea $X$ una variedad $n$-dimensional compacta\footnote{Esto no es del todo importante. Es porque hemos demostrado la existencia de particiones de la unidad sólo en variedades compactas. Para variedades no compactas se hace con cubrimientos infinitos pero localmente finitos; ie, basta que sea \href{http://en.wikipedia.org/wiki/Paracompact_space}{paracompacta}.}, orientable y con borde $∂X$; y $ω$ es una $(n-1)$-forma en $X$. Llamamos $i$ a la inmersión de $∂X$ en $X$ ($i:∂X \hookrightarrow X$). Entonces \[ \int_X \dif ω = \int_{∂X} i^* ω \]
\end{theorem}

\begin{proof} Suponemos que tenemos una partición $(U_i, ρ_i)$ de la unidad. Lo primero es que basta ver que la fórmula se cumpla en los abiertos de la partición de la unidad, gracias a la forma en la que se definen las integrales (\fref{def:IntegracionNForma}).

Definimos $ω_j ≝ ρ_j ω$ de tal forma que $\sum ω_j = ω$. Entonces \( \dif ω = \dif \left(\sum_j ρ_jω_j\right) = \sum_j \dif(ρ_jω) = \underbracket{\sum_j \dif ρ_j \y ω}_{(A)} + \sum_j  ρ_j \dif ω  = \sum_j ρ_j \dif ω \label{eq:ProofStokes1} \)

La observación que hay que hacer es que $(A)$ es nulo: podemos sacar factor común y entonces

\[ (A) = \left(\sum_j \dif ρ_j\right) \y ω = \left(\dif\left(\sum_j ρ_j \right) \right) \y ω = \dif 1 \y ω = 0\]

Ahora podemos calcular la integral: \( \int_X \dif ω = \sum_j \int_X ρ_j \dif ω \eqexpl{\ref{eq:ProofStokes1}} \sum_j \int_X ρ_j \dif ω \eqreason{Por Stokes en los abiertos.} \sum_j \int_{∂X} i^*(ρ_j ω) \eqreasonup{Por la definición de la integral.} \int_{∂X} i^* ω \label{eq:ProofStokesFinal} \)

Nos falta probar que Stokes es cierto en los abiertos, esa penúltima igualdad ahí.

Ahora va a hacer algo en un abierto coordenado $\mop{sop}ω ⊂ U_i$ que no corta al borde. Podemos tomar entonces un paralelepípedo que no corta al borde, al que llamamos $Q$. Como el soporte no corta al borde, la integral de $ω$ en el borde es nula.Ahora estoy usando $ω$ como una $n-1$ forma con soporte compacto contenido etcétera. Esta $ω$ no es la ω de antes. Es otra, sería como una $ρ_j ω$. Luego aplicaré esto al $ω$ de antes. Por no complicar la notación\footnote{Claro que sí.}.

Bueno, al lío. Tenemos que $\int_{∂X} i^* ω = 0$, para que Stokes sea cierto en este abierto tiene que ser \( \int_X \dif ω = 0 \label{eq:ProofStokesIntegAbierto} \). Para que las cosas sean más claras escribimos \[ ω = \sum a_i \dif x_1 \y \dotsb \y \dif \hat{x}_i \y \dotsb \dif x_n \], donde la $\hat{x}_i$ es la que no aparece en la carta coordenada\footnote{Juraría que esto tiene algo que ver con lo de los semiplanos.}. La diferencial será \[ \dif ω = \sum_j (-1)^{j-1} \dpd{a_j}{x_j} \dfl{x_1}{x_n} \]

Vamos ahora a calcular la integral de la \fref{eq:ProofStokesIntegAbierto}. Como el soporte de $ω$ está contenido en $Q$ cambiamos $X$ por $Q$ y entonces \[ \int_X \dif ω = \int_q \dif ω = \sum (-1)^{j-1} \int_Q \dpd{a_j}{x_j} \dfl{x_1}{x_n} \]

Integramos ahora primero en la coordenada $j$, usando la regla de Barrow\footnote{Qué manía de no llamarle Teorema Fundamental del Cálculo, que mola más.}. Entonces, usando Fubini nos queda la integral \( \sum (-1)^{j-1} \int \mathcal{N} \dif x_1 \y \dotsb \y \dif \hat{x}_i \y \dotsb \dif x_n \label{eq:ProofStokes2} \)

Por alguna razón que me he perdido, la $\mathcal{N}$ es la resta de dos cosas que son cero y luego te sale 0 al final.

Según De Juan, la razón que me he perdido es que como estamos integrando en $Q$ que es más grande que el soporte, los límites de la integral van a estar evaluados fuera del soporte de $ω$ y por lo tanto van a ser 0. A mí me parece convincente.

Edu ha dicho algo\footnote{EDU: pregunto cómo puedes reordenar los diferenciales sin que cambie el signo si el producto exterior es anticonmutativo. La respuesta del profesor es que, en realidad, cuando integras ignoras los productos exteriores y que seguro que Stokes ya lo tuvo en cuenta cuando estudió todo esto. Ole.} y De Juan me comenta que en la página 60 (56 según el libro) del doCarmo por alguna razón se quitan los diferenciales y se quedan productos normales. Gracias De Juan.

El caso que falta es ver lo que ocurre cuando el soporte sí interseca con el borde. La integral sobre el borde es fácil de calcular.

Calculamos rápidamente $i^*ω$. Estamos restringiendo a $x_1 = 0$, por lo tanto nos quitamos todos los sumandos de $ω$ que tienen $\dif x_1$ y nos queda \[ i^* ω = a_1(0, x_2, \dotsc, x_n) = \dfl{x_2}{x_n} \], por lo que la integral es  \[ \int_{∂X} i^*ω = \int_Q a_1(0, x_2, \dotsc, x_n) \dif x_2 \dotsb \dif x_n \]

El cálculo de la diferencial de $ω$ es el mismo, así que reutilizamos la \fref{eq:ProofStokes2}. Los sumandos con $j≠1$ se nos anulan por la misma razón que antes, así que nos queda sólo el sumando con $j=1$:  \[ \int_X \dif ω = \int_{Q^{n-1}} (a_1(0, x_2, \dotsc, x_n) - \underbrace{a_1(\dotsc)}_{(B) = 0}) \dif x_2\dotsb \dif x_n \], donde $(B)$ se anula porque será la evaluación de la integral en el otro lado de $Q$ que no interseca con el soporte de $ω$, y efectivamente nos queda lo que buscábamos: \[ \int_{∂X} i^*ω = \int_X \dif ω \]

Hemos demostrado entonces lo que nos faltaba de la \fref{eq:ProofStokesFinal} y hemos demostrado por tanto el teorema entero.
\end{proof}

\paragraph{Ejercicios recomendados} De los ejercicios que hay en el libro el profesor nos recomienda unos pocos, que son más fáciles. Del capítulo 4, los ejercicios 7,8,9 de la página 72; y el 16 también que refiere al 12 que ya lo hizo en teoría\footnote{Vete tú a saber dónde.}.

Quizás mencionar que muchos de los ejercicios que piden usar Stokes piden relacionar la integral en una esfera maciza con la integral en el borde: es fácil porque sólo tiene una carta. El que quería enunciar es el \fref{ej:EnunciadoRafa}.

