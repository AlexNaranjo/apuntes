% -*- root: ../GeometriaDiferencial.tex -*-
\chapter{Geometría de superficies. Curvatura de Gauss-Bonnet}
\label{chapGeometriaSuperficies}

Tenemos $X$ un campo en $M$, una variedad de dos dimensiones, compacta, orientable y Riemanniana. Vamos con varias definiciones.


\begin{defn}[Punto\IS singular] Se dice que $p∈M$ es singular si $X_p = 0$. \end{defn}

Si hay un entorno de $p$ en el que $p$ es el único punto singular, entonces es un punto singular aislado. Supondremos, porque la variedad es compacta, que el campo sólo tiene puntos singulares finito. No, cuando lo ha dicho no tenía ningún sentido tampoco.

Resulta que el número de puntos singulares es un invariante interesante que depende del campo y la variedad y depende sólo de la topología y nosqué.

Lo primero que vamos a definir es el índice, que es un número entero que cuenta cuánto de cómo singular es el punto (???), cuenta el número de puntos singulares con una cierta multiplicidad que nos da el índice (este hombre no sabe ni lo que está diciendo).

\begin{defn}[{Í}ndice] Suponemos que $X$ tiene un número finito de puntos singulares aislados. Consideramos entonces en $U\setminus\set{p}$ una referencia dada por \[ \gor{e}_1 ≝ \frac{X}{\md{X}} \] y el vector $\gor{e}_2$ que completa la referencia ortonormal y mantiene la orientación. De esa referencia salen $\gor{ω}_1,\gor{ω}_2, \gor{ω}_{12}$ como las 1-formas duales.

Construimos también una referencia $ω_1, ω_2, ω_{12}$ a partir de coordenadas locales. Así, tenemos otra referencia móvil para estudiar la del gorro, y por lo tanto tendremos la fórmula \[ \gor{ω}_{12} = ω_{12} - τ \] del lema \ref{thmOmegaTau}.

Si tomamos ahora una curva $C$ que encierra el disco con el sentido de giro dado por la orientación. Sabemos entonces que \[ \int_C τ = \int_C \dif φ \] donde $φ$ es el ángulo de la dos referencias que vimos en \ref{eqPhiAngulo}. Esa intergral, por ser una curva cerrada, tiene que ser igual a $2πI$, con $I ∈ ℤ$. A esa $I$ se le llama el \textbf{índice} de $X$ en $p$.
\end{defn}

El índice será 1, por ejemplo, si damos una vuelta alrededor del punto en el mismo sentido de la orientación.

Para poder usar esto hay que tener unos lemas que son todos consecuencia de Stokes. Estos lemas dicen que el índice está bien definido y que no depende de elecciones previas.

\begin{lemma} El índice no depende de la curva $C$ escogida siempre que sea topológicamente una circunferencia y contenga únicamente al punto singular $p$.
\end{lemma}

\begin{proof} Si cogemos $I_1, I_2$ los índices dados por dos curvas $C_1, C_2$, vemos que \[ I_1 - I_2 = \frac{1}{2π} \int_{C_1} τ - \frac{1}{2π} \int_{C_2} τ \eqexpl{Stokes} \frac{1}{2π} \int_Δ \dif τ = 0\] donde $Δ$ es la región encerrada entre $C_2$ y $C_1$.
\end{proof}

\begin{lemma} El índice no depende de la referencia tomada $\set{e_1, e_2}$ que nos da las formas $ω_1, ω_2, ω_{12}$.

És más, sea $S_r=\partial B_r$ el borde de un disco de rado $r$ centrado en $p$ y sea $\{\overline{e_1}, \overline{e_2}\}$ entonces el límite:

\[ \lim_{r\to 0} \frac{1}{2π} \int_{S_r} \gor{ω}_{12} ≝ \gor{I} \]

existe y $\gor{I}=I$
\end{lemma}

\begin{proof}
Sean $S_{r_1}$, $S_{r_2}$ dos círculos concéntricos con $r_2<r_1$, tenemos que:
\[ \int_{S_{r_1}} \gor{ω}_{12} - \int_{S_{r_2}} \gor{ω}_{12} \eqexpl{Stokes} \int_Δ \dif \gor{ω}_{12} \]
donde de nuevo $Δ$ es la región encerrada por las dos curvas $S_{r_1}, S_{r_2}$. Cuando hacemos tender $r_1, r_2 \to 0$, es claro que esa integral se va a cero.

Podemos ver que $\overline{ω}_{12}$ no está definida en $B_{r_2}$; sin embargo, $\dif \overline{ω}_{12}=-Kσ$ está definida en todo punto. Por tanto, cualquier serie:
\[\int_{S_{r_1}} \gor{ω}_{12} ... \int_{S_{r_n}} \gor{ω}_{12} ...\]
cuando $r_n \to 0$ es una serie de Cauchy y, por tanto, converge.

Así llegamos a que el límite
\[ \lim_{r\to 0} \frac{1}{2π} \int_{S_r} \gor{ω}_{12} ≝ \gor{I} \]
existe.

Ahora sólo nos queda probar que $\gor{I}=I$.

Si volvemos a los dos círculos concéntricos iniciales y fijamos $r_1$ haciendo que $r_2$ tienda a 0 tenemos que
\[ \int_{S_{r_1}} \gor{ω}_{12} - 2π\gor{I} =\footnote{Si teníamos la región comprendida entre dos círculos y hacemos que el radio del círculo menor tienda a 0, nos queda una 'bola', es decir, un único círculo} \int_{B_{r_1}} \dif\gor{ω}_{12}=-\int_{B_{r_1}} K \overline{ω_1}\y \overline{ω_2}\]

Por otro lado, puesto que el lema \ref{thmOmegaTau} nos garantiza que $\overline{ω}_{12}=ω_{12}+τ$ tenemos

\[ \int_{S_{r_1}} \gor{ω}_{12} = \int_{S_{r_1}} ω_{12} + \int_{S_{r_1}} τ \eqexpl{Stokes} \int_{B_{r_1}} \dif ω_{12} + \int_{S_{r_1}} τ = - \int_{B_{r_1}} K ω_1 \y ω_2  + 2πI \]

Y combinando las dos últimas identidades, nos queda que $\gor{I}=I$
\end{proof}

\begin{lemma} El índice no depende de la métrica Riemanniana escogida.
\end{lemma}

\begin{proof} Para esta demostración vamos a hacer un truco que es una demostración de la métrica. Definimos dos métricas $\pesc{·,·}_1$, $\pesc{·,·}_2$. Tomamos entonces el segmento que une ambas métricas, cosa que podemos hacer porque las métricas son un espacio vectorial de tensores. Definimos entonces  \[ \pesc{·,·}_t = t \pesc{·,·}_1 + (1-t)\pesc{·,·}_2 \] con $t∈[0,1]$

Es fácil ver que $\pesc{·,·}_t$ es Riemanniana si las dos de antes son Riemannianas. Definimos entonces índices $I_t, I_1, I_2$ con respecto a cada métrica respectivamente.

Por el lema anterior, es fácil ver que $I_t$ va a depender continuamente de $t$. Ahora la conclusión es inmediata: si $I_t$ es continua y valora en los enteros, sólo puede ser constante.
\end{proof}

El teorema de Gauss-Bonet lo que dice es que en una superficie en las mismas condiciones que estábamos viendo antes tenemos una función de curvatura que podemos integrar multiplicada por $σ$ (la 2-forma de área). Pues bien, eso debería dar un invariante de la superficie. Lo que dice Gauss-Bonet es que eso es la suma de los índices de un campo $X$ cualquiera con singularidades aisladas. La fórmula tiene varios aspectos interesantes: el lado izquierdo no depende de $X$ luego no depende del campo.

El sobrino del hombre este tiene dos coronillas y lleva el teorema de Gauss Bonet en la cabeza. Que nosqué de lo de la bola de pelo que no se puede peinar que ya vimos en topología\footnote{Lo de ``vimos'' es un decir.}.

Además, la integral de la curvatura no va a depender de la métrica porque los índices no dependían de la métrica. Además, veremos que eso es un invariante puramente topológico, que eso es la característica $χ(M)$ de Poincaré.


Clase clave de geometría.

\begin{theorem}[Gauss bonnet]

SEa $M^2$ una variedad diferenciable orientable de dos dimensiones y sea $X$ un campo de vectores en $M$, también diferenciable, con singularidades aisladas $p_k$ cuyos índices son $I_k$. Entonces para cualquier métrica Riemanniana se tiene:
\[\frac{1}{2π} \int_M Kσ = \sum_{p_i} I_{p_i}(x)\]
donde $K$ es la curvatura Gaussiana de la métrica y σ es el elemento de área.
\end{theorem}

\begin{proof}

Consideremos en $M^2 \setminus U_i\{p_i\}$ la referencia $\{e_1 =X/|X|, e_2\}$, donde $e_2$ es un campo de vectores unitario ortogonal a $e_1$ en la orientación de $M$.

Denotemos por $B_i$ una bola centrada en $p_i$ que no contiene ningún punto singular más que $p_i$. Por el toerema de Stokes tenemos que:

\[\int_{M-∪B_i} K\bar{ω_1} ∧ \bar{ω_2} \overset{defK}{=} -\int_{M-∪B_i} d\bar{ω}_{12} \overset{St} = \int_{∪dB_i}\bar{ω}_{12} = \sum \int_{dB_i} \bar{ω}_{12} \]

donde $\partial B_i$ tiene la orientación inducida por $B_i$ (el contrario de la orientación de $M\setminus B_i$, de ahí el cambio de signo en la primera igualdad).

Ahora tomamos el límite de la igualdad superior cuando el radio de $B_i$ tiende a 0 obteniendo:
\[\int_M K ω_1 \y ω_2 = 2π \sum_i I_i\]

\end{proof}

\obs La parte de la derecha de la igualdad no depende del campo de vectores $X$ y la parte izquierda no depende de la métrica. Por tanto podemos concluir que $\sum_i I_i$ es la misma para todos los campos de vectores con singularidades aisladas y que $\int_M Kσ$ es la misma para cualquier métrica Riemanniana en $M$.


\begin{defn}[Triangulación]  $M=∪T_i$, siendo $T_i$ una cosa homeomorfa a un triángulo, es decir, con 3 vértices, 3 lados y una cara.

Una triangulación tiene que cumplir la siguiente condición:
\[
T_i ∩ T_j = \begin{cases}
\emptyset\\
\text{lado}\\
\text{vétice}\\
\end{cases}
\text{ según el caso}\]
\end{defn}


\begin{defn}[Característica Euler-Poincaré]
Sea $M$ una variedad diferenciable,

$$\chi(M) := \#V - \#L + \#C ∈ℤ$$

Siendo $V$ el número de vértices de una triangulación, $L$ el número de lados y $C$ el número de caras.


Esta característica es una propiedad topológica de la variedad.
\end{defn}

Lo interesante de esta fórmula es que no depende de la triangulación elegida, sino de la topología.

Vamos a ver un par de ejemplos:

\begin{example}
Sea $M$ un tetraedro. Tenemos entonces $V=6,A=12,C=8$, con lo que la característica es $2$. Con esto, tenemos (si de verdad fuera topológico, que no lo hemos demostrado todavía) tendríamos que la característica de EP de la esfera es 2 (porque son ¿homeomorfos? el tetraedro y la esfera)
\end{example}

\begin{example}

Vamos a ver la triangulación de un toro. Para ello, es pensamos en el toro como un cuadrado con los lados identificados. Construimos en ese cuadrado un triángulo rectángulo en una esquina y tiramos 4 paralelas a esa, de tal forma que la tercera sea la diagonal.

Así, tenemos $V=9$, $L=27$ y $C=18$. Para calcularlo tenemos que tener cuidado con las identificaciones (cada lado del cuadrado se identifica con el paralelo).
\end{example}


\begin{theorem}[Poincaré]
Sea $D$ un campo en $M$ una variedad
\[
\sum I_i(D) = \mathcal{X}(M) = \#V - \#L + \#C
\]

\end{theorem}

Por el teorema de Gauss Bonnet sabemos que la suma de los índices no depende del campo elegido. Eso es lo que nos va a dar la clave para que no dependa de la triangulación.

\begin{proof}
Esta demostración merece la pena mirarla en el doCarmo (Página 101) que hay unos dibujos de campos que se utilizan en la demostración de verdad al final de la página 103.

El esquema es probar que para toda tirangulación se puede conseguir un campo que cumpla la fórmula.

Una vez tenemos ese campo construido (Fig 6.2 del doCarmo - página 104), por el teorema de Gauss Bonnet, sabemos que la suma de los índices no depende del campo elegido.

\end{proof}

\begin{theorem}[Gauss-Bonnet-Poincaré]
Sea $M$ una variedad diferencial de dimensión dos, orientable y compacta, con frontera $\partial M$ y sea $X$ un campo de vectores en $M$ diferenciable y transversal a $\partial M$ (esto es, $X$ no es tangente en ningún punto a $\partial M$).

Supongamos que $p_1,...p_k$, singularidades de $X$, son aisladas y no se encuentran en $\partial M$ y sean $I_1,...,I_k$ sus índices respectivos.

Entonces, para cualquier métrica Riemanniana en $M$
\[\frac{1}{2π} \int_M Kσ = \sum_{p_i} I_{p_i}(x) = \mathcal{X}(M)\]
donde $k_g$ es la curvatura geodéisca de $\partial M$ y  $\dif s$ es el elemento de arco de $\partial M$
\end{theorem}

\begin{proof}
Tomemos una métrica Riemanniana en $M$ y consideremos en ella la referencia ortonormal orientada: $\bar{e}_1=X/|X|,\bar{e}_2$. Tomemos, en un entorno $V \subset M$ de $\partial M$ otra referencia orientada $\{e_1,e_2\}$ tal que, restringida a $\partial M$, $e_1$ es tangente a $\partial M$. Entonces
\[i^* \bar{ω}_{12}=i^*ω_{12}+\dif \psi\]
donde $\appl{i}{\partial M}{M}$ es la inclusión y $\psi$ es el ángulo entre $\bar{e}_1$ y $e_1$ en $\partial M$.

Sea $B_i$ una bola de centro $p_i$ para $i=1,...,k$ tal que no contiene ningún punto singular aparte de $p_i$. Entonces:
\[\int_{M - \cup B_i} K \bar{ω}_{1}\y \bar{ω}_2 = - \int_{M - \cup B_i} \dif \bar{ω}_{12} = \int_{\cup \partial B_i} \bar{ω}_{12}-\int_{\partial M}i^* \bar{ω}_{12}\]

por tanto
\[\int_{M - \cup B_i} K \bar{ω}_{1}\y \bar{ω}_2 + \int_{\partial M}i^* \bar{ω}_{12} = \sum_{i=1}^k \int_{\partial B_i} \bar{ω}_{12}\]

Por otro lado, por la definición de curvatura geodésica, tenemos
\[\int_{\partial M} i^*\bar{ω}_{12} = \int_{\partial M} i^*ω_{12}+\int_{\partial M} \dif \psi = \int_{\partial M}k_g \dif s + \int_{\partial M} \dif \psi\]

Como $\bar{e}_1=X/|X|$ no es tangente en ningún punto a $\partial M$, $\int_{\partial M}k_g \dif s=0$. Así, tomando el límite cuando el radio de $B_i$ tiende a 0 obtenemos:
\[\int_M K σ + \int_{\partial M} k_g \dif s = 2π \sum I_i\]
\end{proof}

Aquí finaliza el curso. Nos hemos dejado el último punto del teorema que es la extensión a dimensiones infinitas de la referencia móvil, que es para lo que servía el álgebra multilineal.

Rafa se comenta que ahora siendo los cursos de 3 horas, hay asignaturas  que se quedan cojas. Nos pone el ejemplo de la teoría de Galois, que siendo sólo de 3 horas se queda corta y coja porque no se ven los teoremas realmente interesantes.

