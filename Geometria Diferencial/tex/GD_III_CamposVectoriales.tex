% -*- root: ../GeometriaDiferencial.tex -*-
\chapter{Campos vectoriales y formas diferenciales: Teoremas de Frobenius y Poincaré}

\section{Campos en variedades diferenciables}

No sé si esto va aquí.

Si tenemos una aplicación $\appl{F}{X}{Y}$, entonces la definición de la diferencial en un punto \[ (DF)_x = \appl{F_{*, x}}{Π_x }{Π_{F(x)} Y} \] es la misma que para abiertos de $ℝ^n$.

Si $F$ es inyectiva y $D$ es un campo en $X$, entonces sí existe un campo $F_*(D)$ definido no en $Y$ sino en $F(X) ⊂ Y$.

Vamos a ver varios resultados en esto.

\begin{defn}[Curva\IS integral] Sea $\appl{γ}{I}{X}$ una curva donde $I = [-a, a]$ es un intervalo simétrico alrededor del origen, y $D$ un campo en $X$. Decimos que γ es una curva integral de $D$ por el punto $x_0$ si se cumple que $γ(0) = x_0$ y además $γ_*\left(\dpa{}{t}\right) = \restr{D}{γ(I)}$ donde $\dpa{}{t}$ es un campo de vectores que a cada punto le asigna el vector tangente a la curva.
\end{defn}

Supongamos por ejemplo que $D = \sum a_i(x_1, \dotsc, x_n) \dpa{}{x_i}$. Entonces, dada $γ(t) = (x_1(t), \dotsc, x_n(t))$, tendríamos que \[ γ_*\left(\dpa{}{t}\right) = \sum x_i'(t) \dpa{}{x_i}\] ya que recordemos que aplicar la función es sólo permutar los símbolos: \[ γ_* \left(\dpa{}{t}\right) f = \dpa{}{t} (f○γ) \]

Al igualar $D$ con la composición esa tendríamos que, para $i = 1, \dotsc, n$, \[ x_i'(t) = a_i(x_1(t), \dotsc, x_n(t))\]. Todo eso junto es un sistema de ecuaciones diferenciales ordinarias autónomo. Es decir, que el campo da lugar a un sistema de ecuaciones locales, que al resolverlo dan cosas.

Hay una definición y un teorema útiles en geometría. Vamos a verlos.

\begin{defn}[Caja] Dado $x_0$ un punto y $D$ un campo en $X$, se dice que una caja para $D$ en $x_0$ es una tripleta formada por un entorno abierto $U_0$ de $x_0$, un real $a > 0$ (o infinito) y una función $Φ$ que cumplen ciertas condiciones.

$Φ$ debe ser una función $\appl{Φ}{U_0 × (-a, a)}{X}$ diferenciable ($C^∞$). Además $∀x∈U_0$ fijo se obtiene una función $γ_x ≝ \appl{Φ(x, t)}{(-a, a)}{X}$ que define una curva integral en $X$. Dicho de otra forma, la función $Φ_0$ ``pega'' todas las curvas.

La última condición es que, $∀t ∈ (-a, a)$ fijo podemos definir una función $\appl{τ_t}{U_0}{X}$ que por definición es simplemente $τ_t ≝ Φ(x,t)$ y que cumple el ser un difeomorfismo de $U_0$ con su imagen.
\end{defn}

\begin{theorem} Para toda $D$ y $x_0 ∈ X$, existe una caja y además es única.

Además, si $t_1, t_2, t_1 + t_2 ∈ (-a, a)$, entonces $τ_{t_1+t_2} = τ_{t_1} ○ τ_{t_2} = τ_{t_2} ○ τ_{t_1}$.
\end{theorem}

Ese último enunciado del teorema nos da una especie de propiedad de grupo para τ. De hecho, a τ se le llama el \concept[Flujo\IS de un campo]{flujo de un campo}.

Por ejemplo, si vemos el campo de velocidades de un fluido estacionario, en el que la velocidad sólo depende del punto $x$ del fluido, la función $τ_t(x)$ nos da una progresión del líquido, cómo se mueven las partículas a lo largo del tiempo por la curva que pasa por $x$.

Geométricamente, un campo lo que hace es definir un flujo cuando se integra: para cada $t$, se obtiene una función que va de cada punto en el punto imagenpor el flujo, y ese movimiento determina las curvas solución y está determinado por ellas, por supuesto.

En este caso, se dice que el campo es el generador infinitesimal del flujo y el flujo se dice que es el flujo del campo flujo flujo flujo.

Hay dos teoremas importantes sobre los campos.

\begin{theorem} Sea $D$ un campo en $X$ y $x_0$ un punto en $X$ donde el campo no se anula ($D_{x_0} ≠ 0$). Entonces existe un sistema $y_1, \dotsc, y_n$ de coordenadas locales en un entorno de $x_0$ en el cual $D = \dpa{}{y_1}$.
\end{theorem}

Este teorema lo que está diciendo es que se puede cambiar de carta de forma que en la nueva carta el campo se escribe de esa forma. Es necesario que el campo no sea nulo, porque si no la derivada esa no se anularía.

Esto es importante porque en Geometría Diferencial a veces los problemas se resuelven de manera intrínseca, sin usar coordenadas locales. Pero cuando hagamos los cálculos en coordenadas locales, elegiremos un sistema mejor adaptado al problema para hacer el cálculo como si fuera en abiertos de $ℝ^n$. Lo que dice el teorema es que nos conviene usar las coordenadas locales esas para tener una expresión lo más simple posible.

La idea de la demostración del teorema es la siguiente. Tenemos el campo que no se anula en un punto, y consideramos una hipersuperficie dada por una función $f(x_1, \dotsc, x_n) = K$ constante. La curva solución del sistema que nos daría las nuevas coordenadas será una curva transversal a la superficie.

Entonces, para cada punto cerca de $x_0$, consideramos la curva solución y la cortamos con la hipersuperficie. Como el campo es transversal, todos los vectores hacen cosas y yo ya estoy muy perdido porque hasta el hobre este creo que se está peridiendo. Estoy empezando a escribir cosas sin sentido. Lo que $D = \dpa{}{y_1}$ significa es que $D(y_i)$ es cero si $i ≠ 1$.

Escribir el campo así es lo mismo que encontrar funciones tales que al derivarlas da cero y luego una función $y_1$ tal que vale 1. En definitiva, buscamos funciones que cumplen el sistema
\begin{align*}
D(y_2) = D(y_3) = \dotsb = D(y_n) &= 0 \\
D(y_1) &= 1
\end{align*}

Encontrar las que valen $0$ no es demasiado difícil.

Por definición, cada nosequé le pasa algo y es constante sobre las curvas solución, que implica ese sistema, que es lo que falta y es cierto porque si yo tengo una función en un entorno $x_0$ en el que estamos $f$ restringida a las curvas solución de $D$ es constante en tonces $D(f) = 0$. Esto es lo que me faltaría ver. Fijaros que $f$ restringida a las curvas solución es constante sobre cada curva, no constante entre diferenteces curvas. ¿Por qué es cierto esto? Por la definición de curva solución.

\textit{Guille: No he entendido puto nada de lo que ha dicho ni de lo que ha escrito. Pablo dice que sí pero no se lo cree ni él. Dudo que siquiera el profesor sepa qué puñetas ha dicho.}

\seprule
