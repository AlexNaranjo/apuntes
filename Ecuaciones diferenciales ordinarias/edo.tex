\documentclass[nochap]{apuntes}

\title{Ecuaciones diferenciales ordinarias}
\author{Por definir}
\date{13 / 14 C2}

\begin{document}

\maketitle

\section{Introducción}

\begin{example}
Nos imaginamos que estamos en el origen de coordenadas. Nos dan una cuerda atada a un carro a en un punto $(0,L)$. Nosotros nos movemos a lo largo del eje vertical y nos piden la trayectoria que describe el carro. Este es el problema de la tractriz \index{Problema! tractriz}.

Esta curva tendrá una ecuación de la forma $(x,y(x))$. 

La velocidad a la que nos movemos por el eje vertical es un dato del problema.

Geométricamente está claro que la cuerda de la que tiramos es parte de la recta tangente. Además, el trozo que une el carro con el punto en el que estamos mantiene la longitud $L$.

El vetor tangente será $(1,y'(x))$, por lo que la tangente $\equiv (x + \lambda,y+\lambday')$.

Otro dato que tenemos es que en el punto $P$ en el que estamos nosotros es de la forma $(0,*)$ y además pertence a la recta tangente. Es decir:

\[\begin{array}
x+\lambda = 0\\
*\ast = y+\lambda y' = y-xy'\end{array}
\]

Entonces podemos plantear el sistema:

\[
L^2 = x^2 + x^2+(y')^2 \dimplies
\frac{L^2}{x^2} = 1+(y')^2
\]

Esto es lo que sería una ecuación diferencial. Hay $x,y$ y alguna de sus derivadas. En realidad tenemos 2 soluciones (la positiva y la negativa). Esto nos da 2 ecuaciones diferenciales ¿Qué solución nos interesa en este caso? Por la geometría del problema, la curva es decreciente por lo que la derivada será negativa.

La ecuación seríä:

\begin{op}
y'=-\sqrt{\frac{L^2 - x^2}{x^2}} 
\end{op}

La solución sería integrar, pero tenemos 1 problema, la constante de integración. ¿Cómo solucionarlo? Teniendo un dato de la posición inicial. Este problema tan natural se llama \emph{problema de valores iniciales de Cauchy}.

En este caso la ecuación es: 

\[y(x) = \int \sqrt{\frac{L^2 - x^2}{x^2}} \].

Tomando $L=1$ sale esta solución: \[y(x) = ln\left(\frac{1+\sqrt[1-x^2]}{x}\right) - \sqrt{1-x^2} \]
\end{example}

\begin{example}
Este caso se conoce como la curva de persecución.

La representación geométrica es una liebre corriendo en línea recta por el eje vertical y el perro que empieza en un punto (x,0).


Es un problema parecido al anterior, con la salvedad de que la longitud no es constante. Entonces necesitamos un dato. Como cada animal corre a una velocidad, las distancias recorridas van en función de la velocidad. Vamos a escribirlo utilizando la misma información que ya sabíamos del ejemplo anterior, la recta tengente contiene a los 2 puntos.

\[
\begin{array}
V_l t = y-y-xy' \implies t=\frac{y-y'x}{v_L}\\
V_p t = \int_x^D \sqrt{1+(y'(s))^2} ds
\end{array}\]

Entoncesº
\[\frac{V_p}{V_l} = ...\]

Derivamos y aplicamos el teorema fundamental del cálculo y obtenemos:

\[
\frac{v_p}{v_l}xy'' = \sqrt{1+(y')^2}
\]

Esta es una ecuación de orden 2 (porque tiene una derivada segunda).

Vamos a intentar resolverla. Llamando $y'=p$. Entonces nos queda una ecuación reducible a una de primer orden.

Con esto obtenemos una ecuación separable de la forma:

\[
\frac{p'(x)}{\sqrt{1+(p(x))^2}} = \frac{V_l}{v_p x}
\]

Ahora integramos a ambos lados.

Después tomamos exponenciales para quitarnos los logaritmos.

Tenemos que tener cuidado con la constante de integración.

Hemos obtenido una ecuación implícita de p. Ahora aplicaríamos el teorema de la función implícita del curso de Análisis Matemático. Una vez depejemos $p$ nos quedaría volver a integrar para hallar $y$.

Nos quedaría la solución en función de 2 constantes, por lo que necesitaríamos 2 datos para resolver este problema.


En este caso los 2 datos que tenemos que deducir de la interpretación geométrica.

\begin{itemize}
\item En el instante inicial, el perro está a distancia $D$. $y(D)=0$
\item En el instante inicial el perro está mirando a la liebre, es decir, la tangente es positiva. $y'(D)=0$.
\end{itemize}

\end{example}

\end{document}