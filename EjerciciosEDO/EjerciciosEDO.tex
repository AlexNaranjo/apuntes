\documentclass[nochap]{apuntes}
\usepackage{anysize} 
\usepackage{dsfont}
\usepackage{amssymb}
\usepackage{textcomp}
\usepackage{plain}

%opening

\title{Ejercicios resueltos de ecuaciones diferenciales}
\author{Alberto Parramón Castillo}
\date{2013 / 2014}

\def\misistema[#1]#2{\def\Parente{#1} \def\Pi{i}%
\def\Pd{d} \def\Pid{id} \ifx \Parente\Pi%
\left\{\begin{array}{l} #2 \end{array}\right.
\fi%
\ifx \Parente\Pd%
\left.\begin{array}{r} #2 \end{array}\right\}
\fi%
\ifx \Parente\Pid\left\{\begin{array}{c} #2 \end{array}\right\}
\fi%
}

\makeatletter
\def\sistema{\@ifnextchar[{\misistema}{\misistema[d]}}
\makeatother


\begin{document}

\pagestyle{plain}
\maketitle
\tableofcontents
\newpage

\section{Hoja 1}

\begin{problem}[1]
\ppart Comprobar que, para cada valor de la constante C, la función
\begin{gather*}
y=e^{x^2}\left(C+\int_0^x e^{-t^2}dt\right)
\end{gather*}
	es una solución de la ecuación diferencial
\begin{gather*}
y'=2xy+1
\end{gather*}

\ppart Sean $y_1(x)$, $y_2(x)$ dos funciones de la familia anterior, correspondientes a dos valores distintos de la constante $C$. Hallar la ecuación diferencial que satisface $u(x)=y_1(x)-y_2(x)$.

\solution

\spart
\begin{expla}
Si $y$ es una solución de la ecuación diferencial $y'$, entonces, si derivamos con respecto a $x$, $y$, deberíamos llegar a algo parecido a $y'$.
\end{expla}
$y'=\underbrace{2x}_{2x}\underbrace{e^{x^2}(C+\int_0^x e^{-t^2}dt)}_y+\underbrace{e^{x^2}e^{-x^2}}_1$
\begin{expla}
Puede que os líe el término $e^{-x^2}$ del final de la fórmula. Sale de derivar con respecto a $x$ la integral. Esto se debe al \textit{Teorema Fundamental del Cálculo}, que aplicado a este caso nos dice que: 
\begin{gather*}
\dpa{}{x}\left(\int_0^x f(x)\right)=\dpa{}{x}[F(x)]_0^x=\dpa{F(x)}{x}-\dpa{F(0)}{x}=f(x)-0=f(x)
\end{gather*}
Sustituyendo en el razonamiento anterior $f(x)$ por $e^{-x^2}$, obtenemos dicho término.
\end{expla}


\spart
Partimos $y_1$ e $y_2$, soluciones de la ecuación diferencial. Y tenemos $u=y_1-y_2$. La ecuación diferencial que satisfaga $u$ deberá ser $u'$, por tanto, vamos a calcularla:

$u'=y_1'-y_2'$

Como $y_1$ e $y_2$, son soluciones, tenemos que:

$y_1'-y_2'=2xy_1+1-2xy_2-1=2x(y_1-y_2)=2xu$

Por tanto, la ecuación diferencial que satisface $u$ es: $u'=2xu$.

\end{problem}
\newpage

\begin{problem}[2]
\ppart Hallar los valores de $m$ para los cuales $y=e^{mx}$ es solución de la ecuación diferencial
\begin{gather*}
2y'''+y''-5y'+2y=0
\end{gather*}

\ppart A partir de combinaciones lineales de las soluciones encontradas en el apartado anterior. hallar una solución que verifique las condiciones
\begin{gather*}
y(0)=0,	y'(0)=1,  y''(0)=-1
\end{gather*}

\solution

\begin{expla}
Aquí irá una breve explicación sobre este ejercicio.
\end{expla}

\spart

Puesto que las soluciones son de la forma: $y=e^{mx}$ tenemos que:\\
\begin{gather*}
y'=me^{mx}\;\;\;\;y''=m^{2}e^{mx}\;\;\;\;y'''=m^{3}e^{mx}
\end{gather*}

Por tanto, sustituyendo en la fórmula dada tenemos que:
\begin{gather*}
2y'''+y''-5y'+2y=0 \implies 2m^{3}e^{mx}+m^{2}e^{mx}-5me^{mx}+2e{mx}=0 \implies \\ \implies e^{mx}(2m^3+m^2-5m+2)
\end{gather*}
Que, resolviendo por Ruffini da como soluciones:
$m=1$\;\;\;\;$m=-2$\;\;\;\;$m=1/2$

\spart
El resultado tiene que ser combinación lineal de las soluciones anteriores, es decir será de la forma:
\begin{gather*}
y=ae^{-2x}+be^x+ce^x/2 
\end{gather*}
Planteamos un sistema de 3 ecuaciones con 3 incógnitas con los datos proporcionados por el enunciado:

 $\sistema[i]{y(0)=a+b+c=0\\y'(0)=-2a+b+c/2=1\\y''(0)=4a+b+c/4=-1}$ $\implies$ $\sistema[i]{a=-1/3\\b=1/3\\c=0}$


\end{problem}
\newpage

\begin{problem}[3]
Comprobar que para cada valor de la constante $C>0$, la identidad
\begin{gather*}
Cx-y \int_0^x\frac{\sen t}{t}dt=0
\end{gather*}
define $y$ como función de $x$. Hallar la ecuación diferencial que satisfaga todas las funciones de la familia. Hallar $y(0)$ e $y'(0)$ para cada una de ellas.

\solution

\begin{expla}
Este ejercicio no le deis mucha importancia, es más de repaso de Analisis.

Basta ver que al despejar la y, te queda la integral en el denominador, y para ver que y es una funcion de x, debes comprobar que esa integral no se hace 0, hay dos opciones:\\
a) Usar L'Hopital, y ver que al derivar no te queda un 0 en el denominador. LLegas $\sin (x)/x$ que es 1.\\
b) Usar el teorema de la función implícita. Es decir, que si la derivada parcial con respecto a y (que es la que queremos despajar) es distinta de 0 en todo x, entonces se puede despejar y en función de x. (esta opción no renta mucho en este ejercicio)
\end{expla}

Solución

\end{problem}
\newpage

\begin{problem}[4]
\ppart Utilizando isoclinas, esbozar las soluciones de $y'=y^2-1$
\ppart Resolver explícitamente la ecuación $y'=y^2-1$, $y(0)=0$ y comparar el resultado con lo obtenido por el método de isoclinasintegrate empleado anteriormente.

\solution



\spart
\begin{expla}
La siguiente figura muestra la pendiente de la curva solución que buscamos en cada punto. Como la pendiente es la derivada de la $y$, entonces, los valores que obtenemos en $y'$ son con los que dibujaríamos la curva solución. Por supuesto, dependiendo del punto en que empiece nuestra curva, está seguirá un camino u otro.

\end{expla}
$y'=0$ en $y=1$ e $y=-1$, por tanto en esas rectas horizontales la pendiente de la curva solución será 0, además esas dos rectas serán a su vez una solución de la ecuación diferencial.

Para valores de $y$ entre $1$ y $-1$, la pendiente de la curva variará entre $-1$ y $0$.

Para valores altos de y, la pendiente aumentará infinitamente. Y para valores negativos de y, la pendiente será más negativa y aumentará su negativo también infinitamente.

\easyimgw{Imagenes/Ej4a.png}{Isoclinas de $y'=y^2-1$}{4a}{0.6}

\newpage
\spart 
\begin{expla}
Nos encontramos ante un problema de Cauchy, en el que partimos de una ecuación diferencial $y'=y^2-1$, y de un dato $y(0)=0$. Y lo que hacemos es integrar la ecuación diferencial, obteniendo al hacerlo una constante, que tenemos que eliminar con el dato
\end{expla}
Antes de integrar, vamos a ver si se puede resolver fácilmente. Si escribimos:
\begin{gather*}
\frac{y'}{y^2-1}=1
\end{gather*}

Vemos que tenemos una ecuación de "variables separadas". Procedemos a integrar:
\begin{gather*}
\int\frac{y'}{y^2-1}dx=\int1dx
\end{gather*}

\begin{expla}
Para resolver la integral cambiamos variables: $y=u$, $y'dx=du$. Y sustituimos:

\begin{gather*}
\int\frac{du/dx}{u^2-1}du/y'=\int\frac{1}{u^2-1}du=\int 1dx
\end{gather*}

Resolvemos con raíces:
\newpage
\begin{gather*}
\frac{1}{u^2-1}=\frac{A}{u+1}+\frac{B}{u-1}
\end{gather*}

Nos queda:
$1=A(u-1)+B(u+1)$

Sustituimos:

$u=1$ $\implies$ $B=1/2$

$u=-1$ $\implies$ $A=-1/2$

Y obtenemos finalmente que:
\begin{gather*}
\int\frac{y'}{y^2-1}dx=\int\frac{1}{u^2-1}du=-\frac{1}{2}\ln|u+1|+\frac{1}{2}\ln|u-1|=-\frac{1}{2}\ln|y+1|+\frac{1}{2}\ln|y-1|
\end{gather*}
\end{expla}

Por tanto tenemos que:
\begin{gather*}
-1/2\ln|y+1|+1/2\ln|y-1|=x+C\\\\
\frac{\ln|y-1|}{\ln|y+1|}=2x+C \implies \abs{\frac{y-1}{y+1}}=e^{2x+c} \implies y-1=ye^{2x+c} +e^{2x+c} \implies \\\\
\implies y(1-e^{2x+c})=e^{2x+c}+1 \implies y=\frac{e^{2x+c}+1}{1-e^{2x+c}} \implies \;\; y=\frac{e^{2x}k+1}{1-e^{2x}k} \;\;\;\forall k∈\real
\end{gather*}

NOTA: suponemos que el valor absoluto es "absorbido" por la constante $c$, y por ello no lo tenemos en cuenta al acabar los cálculos.

Sustituimos ahora el dato $y(0)=0$:
\begin{gather*}
0=\frac{k+1}{1-k} \implies k=-1 
\end{gather*}

Por tanto, la solución obtenida es:
\begin{gather*}
y=\frac{1-e^{2x}}{1+e^{2x}}
\end{gather*}
Que podemos ver que cuadra perfectamente con las isoclinas, entre valores de $y=1$ e $y=-1$, concretamente a la solución que pasa por el origen.
\easyimgw{Imagenes/Ej4b.png}{$y=\frac{1-e^{2x}}{1+e^{2x}}$}{4b}{0.6}


\end{problem}
\newpage

\begin{problem}[5]
Trazando algunas isoclinas, esbozar las soluciones de las siguientes ecuaciones:
\ppart $y'=\sen (y+x)$
\ppart $x'= \sqrt{t^2+x^2}$

\solution

\begin{expla}
Aquí irá una breve explicación sobre el ejercicio
\end{expla}

\spart
$y'=0$ si $x+y=k\pi$. Por tanto: pendiente será 0 en rectas $y=-x+k\pi$

$y'=1$ si $x+y=\pi/2 +2\pi k$. Por tanto: pendiente será 1 en rectas $y=-x-\pi/2+2\pi k$.

$y'=-1$ si $x+y=-\pi/2 +2\pi k$. Por tanto: pendiente será -1 en rectas $y=-x+\pi/2+2\pi k$.
\easyimgw{Imagenes/Ej5a.png}{Isoclinas de $y'=\sen (y+x)$}{5a}{0.6}

\newpage
\spart 
En este caso tenemos una curva $x(t)=x$, podemos ver que los conjuntos de nivel (es decir, los valores que toman $t$ y $x$ para valores fijos de x') son circunferencias de radio x' centradas en el origen:

$x'= \sqrt{t^2+x^2} \implies  (x')^2= t^2+x^2$

Por tanto, las isoclinas quedan:
\easyimgw{Imagenes/Ej5b.png}{Isoclinas de $x'= \sqrt{t^2+x^2}$}{5b}{0.6}

\end{problem}
\newpage

\begin{problem}[6]
\ppart Describir geométricamente la familia de curvas
\begin{gather*}
x^2+y^2+2Cx=0
\end{gather*}
y calcular una ecuación diferencial que la satisfaga.
\ppart Determinar la ecuación diferencial de la familia de curvas
\begin{gather*}
y(x) = x\cos (x+C)
\end{gather*}

\solution

\spart
\begin{expla}
Ejercicio tipo:
Dada una familia de curvas, queremos encontrar su ecuación diferencial. Para ello procedemos de la siguiente forma:
1)Derivamos con respecto a x la función ($F(x,y(x),c)=0$)
2)Eliminamos C utilizando la derivada y la función original.

En estos ejercicios partimos de la ecuación de las curvas solución de una ecuación diferencial (por eso tienen el parámetro constante C). Para llegar a la ecuación diferencial no tenemos más que derivar respecto a x (o respecto a t si tenemos $x(t)=0$ por ejemplo), y eliminar la constante C con ayuda de la ecuación original.
\end{expla}

1)Tenemos: $F(x,y(x),c)=x^2+y^2+2cx=0$

Derivamos con respecto a $x$.
\begin{gather*}
\dpa{F(x,y(x),c)}{x}=2x+2yy'+2c=0
\end{gather*}

2)Despejamos C de la ecuación original
\begin{gather*}
c=\frac{-x^2-y^2}{2x}
\end{gather*}

Y sustituimos, obteniendo la ecuación diferencial:
\begin{gather*}
2x+2yy'-\frac{x^2+y^2}{x}=0
\end{gather*}

Contestando a la primera parte, la de la interpretación geométrica, podemos ver, completando cuadrados que:

$x^2+y^2+2cx=0 \implies x^2+y^2+2cx+c^2=c^2 \implies (x+c)^2 + y^2 = c^2 $

Por tanto, la familia de curvas son circunferencias centradas en el punto (-c,0) de radio c.
\newpage
\spart 
1) Tenemos $F(x,y(x),c)=y-x\cos(x+c)=0$

Derivamos con respecto a x:
$y'-\cos(x+c)+x\sen(x+c)=0$

Y operamos para quitarnos la c:
\begin{gather*}
y'=\cos(x+c)-x\sen(x+c)=x\frac{\cos(x+c)}{x}-x\sen(x+c)=\frac{y}{x}-x\sen(x+c)=\\\\
=\frac{y}{x}-x\sqrt{1-\cos ^2(x+c)}=\frac{y}{x}-x\sqrt{1-\frac{x^2\cos^2(x+c)}{x^2}}=\frac{y}{x}-x\sqrt{1-\frac{y^2}{x^2}}=\\\\
=\frac{y}{x}\pm\sqrt{y^2+x^2}
\end{gather*}
\begin{expla}
El $\pm$ viene de meter la $x$ (que podía ser tanto negativa como positiva) dentro de la raíz.
\end{expla}

\end{problem}
\newpage


\begin{problem}[7]
Dada la ecuación diferencial
\begin{gather*}
y'=(1+x)y+1-3x+x^2
\end{gather*}
calcular los primeros términos del desarrollo de Taylor de la solución que satisface $y(0) = 0$.

\solution

\begin{expla}
Aquí irá una breve explicación sobre el ejercicio
\end{expla}

Solución

\end{problem}

\newpage
\begin{problem}[8]
Esbozar las siguientes familias uniparamétricas de curvas y hallar sus trayectorias ortogonales:
\ppart $xy=C$.
\ppart $y=Ce^x$
\ppart $y=Cx^n$, donde $n$ es un entero positivo. Explicar qué sucede con las trayectorias ortogonales cuando aumentamos el valor del entero $n$.

\solution

\begin{expla}
Ejercicio tipo: A partir de una función $f(x,y)$, hallar las familias de curvas ortogonales a esa función. Pasos a seguir:\\
$1)$ Se deriva implícitamente $f(x,y)$ en función de x.\\
$2)$ Se obtiene la EDO asociado, eliminando la constante C.\\
$3)$ Se sustituye $y'=-1/y'$ (Esto sale de que el producto escalar entre las tangentes a $y$ y a $y$ deben ser 0 en el punto de intersección, ya que serán ortogonales. Viene mejor explicado en los apuntes).\\
$4)$ Se resuelve la EDO obtenida de nuevo, para obtener la familia de curvas ortogonales. El resultado es una familia de curvas ya que queda en función de una constante de integración (D).
\end{expla}
\spart
$1)$ Derivamos implícitamente $xy=C$:
\begin{gather*}
y+xy'=0
\end{gather*}

$2)$ Obtenemos la EDO eliminando C. En este caso ya ha sido eliminado C al derivar.

$3)$ Sustituimos  $y'=-1/y'$:
\begin{gather*}
y-x/y'=0
\end{gather*}

$4)$ Integramos:
\begin{gather*}
y-\frac{x}{y'}=0 \implies y'y=x \implies \int ydy=\int xdx \implies \\ \implies \frac{y^2}{2}=\frac{x^2}{2}+D \implies y=\sqrt{x^2+2D} \implies y=\sqrt{x^2+D}
\end{gather*}

\begin{expla}
NOTA 1: Para resolver la integral, hay dos posibles caminos, el que expliqué en el ejercicio 4, con cambio de variables y tal. O de una manera más directa, el que he hecho en este caso, sustituyendo $y'=dy/dx$, e integrando.

NOTA 2:en el último paso, transformo $2D$ en $D$, se que no son iguales, simplemente es por no poner $D'$ o $D*$, me refiero a que no es más que una constante que no sabemos su valor.
\end{expla}


Vemos en negro $y=2/x$ (escogemos el valor c=2), y en rojo y azul dos soluciones, una con $D=0$ y otra con $D=5$ (Podríamos dar los valores que quisiéramos a D y C, que siempre se cumpliría la ortogonalidad).


\easyimgw{Imagenes/Ej8a.png}{negro:$y=2/x$,color:$y=\sqrt{x^2+D}$}{8a}{0.6}

\spart
$1)$ Derivamos implícitamente $y=Ce^x$:
\begin{gather*}
y'=Ce^x
\end{gather*}

$2)$ Obtenemos la EDO eliminando C.\\
Sustituimos $Ce^x=y$ y nos queda $y'=y$

$3)$ Sustituimos  $y'=-1/y'$:
\begin{gather*}
-1/y'=y
\end{gather*}

$4)$ Integramos:
\begin{gather*}
y'y=-1 \implies \int ydy=\int -dx \implies \\ \implies \frac{y^2}{2}=-x+D \implies y=\sqrt{-2x+D}
\end{gather*}

Sustituyendo (insisto en que es por poner un ejemplo y da igual por qué valor sustituyas) C=2 y D=2.
\easyimgw{Imagenes/Ej8b.png}{negro:$y=2e^x$,color:$y=\sqrt{x^2+2}$}{8b}{0.6}
\newpage

\spart
$1)$ Derivamos implícitamente $y=Cx^n$:
\begin{gather*}
y'=nCx^{n-1}
\end{gather*}

$2)$ Obtenemos la EDO eliminando C.\\
Sustituimos $Cx^n=y$ y nos queda $y'=ny/x$

$3)$ Sustituimos  $y'=-1/y'$:
\begin{gather*}
-1/y'=ny/x
\end{gather*}

$4)$ Integramos:
\begin{gather*}
y'=-\frac{x}{ny} \implies y'ny=-x  \implies \int yndy=\int -xdx \implies \\ \implies \frac{ny^2}{2}=-\frac{x^2}{2}+D \implies y=\sqrt{-\frac{x^2}{n}+D}
\end{gather*}
Sustituyendo la n por 3 en el primer caso (curvas azul oscura y azul clarita), y por 25 en el segundo caso (curvas roja y rosa) , C=2 y D=2.
\easyimgw{Imagenes/ej8c.png}{azul oscura:$y=2x^3$,roja:$y=2x^{25}$}{8c}{0.6}
\end{problem}
Lo que ocurre es que AUMENTA LA EXCENTRICIDAD de la curva ortogonal (elipse) al aumentar el valor de n.


\newpage

\newpage

\begin{problem}[9]
Hallar la familia de curvas ortogonales a la familia de parábolas
\begin{gather*}
y^2-Cx=\frac{C^2}{4}
\end{gather*}

\solution
\begin{expla}
Resolvemos como en el ejercicio anterior. Leer explicación del ejercicio anterior si algo no se entiende.
\end{expla}
$1)$ Derivamos implícitamente $y^2-Cx=\frac{C^2}{4}$, y nos queda:
\begin{gather*}
2yy'-C=0
\end{gather*}

$2)$ Obtenemos la EDO eliminando C.\\
Para ello tenemos, la ecuación de la EDO y de la curva:
 $\sistema[i]{2yy'=C\\y^2-Cx=\frac{C^2}{4}}$
 
Sustituimos la C en la curva y obtenemos:
\begin{gather*}
y^2-2yy'x=\frac{4y^2y'^2}{4} \implies y^2-2yy'x=y^2y'^2 
\end{gather*}


$3)$ Sustituimos  $y'=-1/y'$:
\begin{gather*}
y^2+2y/y'x=y^2/y'^2 \implies y^2y'^2+2yy'x=y^2 \implies y^2-2yy'x=y^2y'^2 
\end{gather*}
Como se puede observar, la EDO que satisfacen ambas curvas es la misma y por tanto esta familia de curvas es autoortogonal a si misma. Por tanto, tienen la misma ecuación.

\easyimgw{Imagenes/Ej9.png}{Valores de C=2 y C=-2}{1-9}{0.6}




\end{problem}
\newpage

\begin{problem}[10]
Hallar la familia de curvas ortogonales a la familia de circunferencias definida por:
\begin{gather*}
x^2+(y-C)^2=C^2
\end{gather*}
Interpretar el resultado geométricamente.

\solution

\begin{expla}
Llega un momento en el que para resolverlo hay que usar el método de resolución 2: por ecuaciones homogéneas.
\end{expla}

Despues de derivar implicitamente, despejar la C y sustituirla en la ecuación de la curva nos queda:

\begin{gather*}
2xy+y'(y^2-x^2)=0
\end{gather*}

Sustituimos $y'=-1/y'$
\begin{gather*}
2xy=\frac{y^2-x^2}{y'} \implies \underbrace{(y^2-x^2)}_{M(x,y)}-\underbrace{(2xy)}_{N(x,y)}y'=0 
\end{gather*}

Se cumple que es homogénea de grado 0, esto es porque si despejas la y', el grado que suman (entre las x e y) es 2 en el numerador y en el denominador: (y'=$\frac{y^2-x^2}{2xy}$) \\
Ahora lo que hay que hacer es hacer el cambio de variables $z(x)=y(x)/x$. Por tanto tenemos:\\
$y(x)=z(x)x$ \;\;\;\; $y'(x)=z'(x)x +z(x)$\\
Sustituimos en la ecuación y tras varias operaciones llegamos a:

\begin{gather*}
-\frac{2zz'}{z^2+1}=1/x \implies -\int \frac{2z}{z^2+1}dz=\int \frac{1}{x}dx \implies -\ln \abs{z^2+1}=\ln \abs{x}+D
\end{gather*}

Sustituyendo $z=y/x$ obtenemos:
\begin{gather*}
LO DEJO COMO EJERCICIO  
\end{gather*}

\end{problem}
\newpage

\begin{problem}[11]
Hallar las trayectorias ortogonales de las siguientes familias uniparamétricas de curvas expresadas en coordenadas polares:
\ppart
$r=C(1+\cos \theta)$
\ppart
$r=2C\sen \theta$

\solution

\begin{expla}
Ejercicio tipo: En los ejercicios con coordenadas polares, lo que hacemos es construir nuestra ecuación en polares:
$\alpha(\theta)=(r(\theta)\cos(\theta),r(\theta)\sin(\theta))$

Y queremos hallar:
$\beta(\theta)=(\tilde{r}(\theta)\cos(\theta),\tilde{r}(\theta)\sin(\theta))$

Como tiene que cumplir la condición de ortogonalidad ($<\alpha(\theta), \beta(\theta)>$) y que en el punto de intersección sean iguales ($r(\theta)=\tilde{r}(\theta)$). Concluimos que para construir la curva ortogonal, tenemos que sustituir:

$r' = -r^2/r'$

Por tanto, los pasos a realizar son los MISMOS que cuando tenemos la curva dada en forma de grafo. Sólo que ahora en los pasos 1),2) y 4), vamos a usar $r(\theta)$ en lugar de $y(x)$, y en el paso 3) sustituiremos $r'=-r^2/r'$. 

\end{expla}

\spart
EN ESTE EJERCICIO ME HE EQUIVOCADO AL DERIVAR Y EN OTRAS COSAS QUE ESTAN SIN CORREGIR, SOY TOLI
$1)$ Derivamos implícitamente $r=C(1+\cos(\theta))$:
\begin{gather*}
r'=-C\sin(\theta)
\end{gather*}

$2)$ Obtenemos la EDO eliminando C.\\
Sustituimos $C=r/(1+\cos(\theta))$ y nos queda $r'=-\frac{r\sin(\theta)}{1+\cos(\theta)}$

$3)$ Sustituimos  $r'=-r^2/r'$:
\begin{gather*}
\frac{r^2}{r'}=\frac{r\sin(\theta)}{1+\cos(\theta)} \implies \frac{r'}{r}=\frac{1+\cos(\theta)}{\sin(\theta)}
\end{gather*}

$4)$ Integramos:
\begin{gather*}
\int \frac{1}{r}dr=\int\frac{1+\cos(\theta)}{\sin(\theta)}d\theta \implies \ln(r)= \int\frac{1}{\sin(\theta)}d\theta+\int\frac{\cos(\theta)}{\sin(\theta)}d\theta
\end{gather*}
La segunda integral es inmediata, la primera es jodida, hay que multiplicar y dividir por $sin(\theta)$ y luego hacer el cambio de variable $sin(\theta)=x$, despues resolver por raíces, y ale, listo.

\spart
$1)$ Derivamos implícitamente $r=2C\sin(\theta)$:
\begin{gather*}
r'=-C\cos(\theta)
\end{gather*}

$2)$ Obtenemos la EDO eliminando C.\\
Sustituimos $C=r/2\sin(\theta)$ y nos queda $r'=-\frac{r\cos(\theta)}{\sin(\theta)}$

$3)$ Sustituimos  $r'=-r^2/r'$:
\begin{gather*}
-\frac{r^2}{r'}=-\frac{r\cos(\theta)}{\sin(\theta)} \implies \frac{r'}{r}=\frac{\cos(\theta)}{\sin(\theta)}
\end{gather*}

$4)$ Integramos:
\begin{gather*}
\int \frac{1}{r}dr=\int\frac{\cos(\theta)}{\sin(\theta)}d\theta \implies \ln(r)=-\ln(\cos(\theta))+c \implies \\ \implies r=\frac{1}{\cos(\theta)}+c
\end{gather*}

\end{problem}
\newpage

\begin{problem}[12]
Hallar las curvas que satisfacen las condiciones geométricas siguientes:
\ppart
El segmento de la tangente limitado por los ejes coordenados tiene como punto medio al punto de tangencia.
\ppart
La proyección sobre el eje $OX$ de la parte de la tangente entre $(x,y)$ y el eje $OX$ tiene longitud 1.
\ppart
EL ángulo entre el radio polar y la tangente es constante.
\ppart
La curva pasa por $(0,0)$ y está contenida en el primer cuadrante, de modo que el área bajo la curva desde $(0,0)$ hasta $(x,y)$ es un tercio del área del rectángulo que tiene a esos puntos como vértices opuestos.
\solution

\begin{expla}
Breve explicación.
\end{expla}

\spart
Aquí va la solución del apartado a).

\spart 
Aquí va la solución del apartado b).

\spart
$\f$ Aquí va la solución del apartado c).

\spart 
Aquí va la solución del apartado d).

\end{problem}
\newpage
\begin{problem}[13]
Para cada una de las ecuaciones diferenciales siguientes, hallar la solución particular que satisface la condición inicial dada.
\ppart $y'=e^{3x-2y}$, con $y(0)=0$.
\ppart $e^{-y}+(1+x^2)y'=0$, con $y(0)=0$.
\ppart $xyy'=(x+1)(y+1)$, con $y(1)=0$.

\solution

\begin{expla}
Para integrar intentamos ponerlo como ecuación de variables separadas, las y a un lado y las x a otro. Después solo hay que integrar y sustituir el dato.
\end{expla}
\spart
\begin{gather*}
y'=e^{3x-2y} \implies y'{e^{2y}}=e^{3x} \implies \int e^{2y}dy=\int e^{3x}dx \implies \\ \implies \frac{e^{2y}}{2}=\frac{e^{3x}}{3} \implies 2y=\ln(2/3)+3x+c \implies y=\frac{\ln(2/3)+3x}{2}+C
\end{gather*}

Sustituimos $y(0)=0$:

\begin{gather*}
0=\frac{\ln(2/3)}{2}+c \implies c=\frac{\ln(2/3)}{2}
\end{gather*}

Solución con dato $y(0)=0$:
\begin{gather*}
y=\ln(2/3)+3x/2
\end{gather*}

\spart
Aquí va la solución del apartado c).

\spart
Aquí va la solución del apartado b).

\end{problem}
\newpage
\begin{problem}[14]
Según la \textit{Ley de enfriamiento de Newton} la tasa de variación de la temperatura en un cuerpo es proporcional a la diferencia de temperatura con el ambiente. Si una barra de hierro a 100\gradc C se enfría a 90\gradc C en 5 segundos cuando se deja a una temperatura ambiente de 20\gradc C, ¿cuánto tardará en estar a 30\gradc C?
\solution
\begin{expla}
De la primera frase tenemos que obtener la ecuación diferencial. \textit{La variación de la temperatura en un cuerpo es proporcional a la diferencia de temperatura con el ambiente}

$\Delta(T)=K(T-A)$
(\textit{Ley de enfriamiento de Newton})

Así definimos $T(t)$ como la temperatura del cuerpo en cada instante de tiempo. $\Delta(T)=T'(t)$ como la variación de esa temperatura, $A$ como la temperatura ambiente (suponemos constante); y $K$ como una constante (que será la constante que crea la familia de curvas, y la cual tendremos que eliminar)
\end{expla}
Primero vamos a obtener $T(t)$, para ello integramos:
\begin{gather*}
\frac{T'}{T-A}=k \implies \int\frac{T'}{T-A}dt=\int kdt  \implies \ln(T-A)=KT+c \;\;\;\forall c∈\real \\\\
\implies T=A+e^{Kt}C \;\;\;\forall C∈\real
\end{gather*}

Sabemos: $T(0)=100$, $T(5)=90$ y $A=20$

$T(0)=20+100=C \implies C=80$

$T(5)=90=20+e^{5K}80 \implies 70=e^{5K}80 \implies \frac{\ln(7/8)}{5}=k$

Por tanto:
$T(t_1)=30=20+e^{\frac{\ln(7/8)}{5}t_1}$

Se despeja $t_1$ y terminado. 

\end{problem}
\newpage
\begin{problem}[15]
En una sala que está a una temperatura de 20\gradc C nos sirven dos tazas de café, a una temperatura de 40\gradc C. Disponemos de leche fría, a una temperatura de 10\gradc C. En una de las tazas echamos una cantidad de leche fría igual a la de café contenido en su interior, y esperamos cinco minutos. En la otra taza esperamos cinco minutos, y después agregamos la leche. Determinar cual de las dos tazas tiene el café con leche más caliente.
\solution
\begin{expla}
Utilizamos las mismas ecuaciones que en el ejercicio anterior: $\Delta(T)=K(T-A)$ y $T=A+e^(Kt)C$
Como la temperatura disminuye, la constante K es negativa. Como tenemos dos ecuaciones, una para cada taza, tendremos también dos constantes C diferentes.
\end{expla}

$T_1(t)=A+C_1e^{kt}$

$T_2(t)=A+C_2e^{kt}$\\



\end{problem}
\newpage
\begin{problem}[16]
Una bola de naftalina tiene inicialmente un radio de 1 cm. Al cabo de un mes su radio se ha reducido a 0,5 cm. Suponiendo que la naftalina se evapora a un ritmo proporcional a la superficie de contacto con el aire, hallar la evolución del radio de la bola en función del tiempo.
\solution
\begin{expla}

\end{expla}
Tomamos como ecuación diferencial: $R'(t)=4\pi k{R(t)}^2$
\begin{gather*}
R'=4\pi kR^2 \implies \frac{R'}{R^2}=4\pi k \implies
\int \frac{1}{R^2}dR=\int 4\pi kdt \implies -\frac{1}{R}=4\pi kt+c
\end{gather*}
Tenemos los datos: $R(0)=1$ y $R(1)=0,5$, Sustituyendo en $-\frac{1}{R(t)}=4\pi kt+c$ obtenemos:\\
$R(0)=1 \implies C=-1$\\
$R(1)=1/2 \implies -2=4\pi k +c \implies -2=4\pi k -1  \implies -1=4\pi k  \implies k=-1/4\pi$

Sustituimos en la ecuación anterior y obtenemos la solución:
\begin{gather*}
-\frac{1}{R}=4\pi kt+c \implies -\frac{1}{R}=4\pi (-\frac{1}{4\pi})t-1 \implies  \frac{1}{R}=4\pi (\frac{1}{4\pi})t+1 \implies \frac{1}{R}=t+1 \implies \\ \implies R=\frac{1}{t+1}
\end{gather*} 




\end{problem}
\newpage
\begin{problem}[17]
Un día comenzó a nevar por la mañana y siguió cayendo la nieve de forma constante todo el día. A las 12 del mediodía una quitanieves comenzó a limpiar una carretera, con velocidad inversamente proporcional al espesor de la nieve depositada. Sabiendo que a las 2 de la tarde había limpiado 2 km., y que a las 4 de la tarde había limpiado 1 km. más, determinar a qué hora comenzó a nevar. 
\solution
\begin{expla}
explicación.
\end{expla}
\end{problem}
\newpage
\begin{problem}[18]
Cuatro hormigas situadas en las esquinas de una mesa cuadrada de lado 1 comienzan a andar simultáneamente a la misma velocidad, cada una en la dirección de su vecina más próxima en la dirección contraria a las agujas del reloj. Tomando coordenadas polares con origen en el centro de la mesa y eje polar a lo largo de una diagonal, hallar la trayectoria de la hormiga que parte del eje polar.
\solution
\begin{expla}
hecho en clase, lo podéis encontrar en los apuntes de Rual.
\end{expla}
\end{problem}
\newpage
\begin{problem}[19]
Una población de bacterias sigue la \textit{Ley de Malthus} (la tasa de variación es proporcional al número de individuos) se duplica al cabo de 24h. ¿Cuánto tardará en triplicarse.
\solution
\begin{expla}
\end{expla}
Ecuación planteada: $P'(t)=kP(t)$
Integramos y nos queda: 
\begin{gather*}
\ln(P)=kt+C \implies P=e^{kt+c}
\end{gather*}
Ahora tenemos dos caminos:\\
1) Sustituimos:\\
$P(0)=n$ $\implies$ $n=e^c \implies c=\ln(C) $
\begin{gather*}
P(1)=2n \implies 2n=e^{k+c} \implies 2n=e^{k+\ln(n)} \implies \ln(2n)=k+\ln(n) \implies k=\ln(2n/n)=\ln(2)
\end{gather*}
Y nos queda como ecuación solucion:
\begin{gather*}
P= e^{t\ln(2)+\ln(n)}
\end{gather*}
Para saber cuando se triplica la población:
\begin{gather*}
3n= e^{t\ln(2)+\ln(n)} \implies t=\frac{\ln(3)}{\ln(2)}
\end{gather*}

2)
\begin{gather*}
P(1)=2P(0) \implies e^{k+c}=2e^c \implies k+c=\ln(2)+c \implies k=\ln(2)
\end{gather*}
Y la c nos da igual, ya que ahora repetimos proceso:
\begin{gather*}
P(t)=3P(0) \implies e^{kt+c}=3e^c \implies \ln(2)t+c=\ln(3)+c \implies t=\frac{\ln(3)}{\ln(2)}
\end{gather*}
\end{problem}
\newpage
\begin{problem}[20]
Supongamos que una población sigue el modelo $p'=bp^2-ap$ con $a,b>0$.
\ppart
Representar en un diagrama de fases, clasificando los puntos críticos del sistema.
\ppart
Demostrar que si $p(t_0)<\frac{a}{b} $, entonces la población tiende a extinguirse.

\solution

\begin{expla}
Aquí irá una breve explicación sobre el ejercicio
\end{expla}

\spart
solución a).

\spart
solución b)

\end{problem}
\newpage
\begin{problem}[21]
dada la ecuación $y'=\cos y$, sin encontrar las soluciones explícitamente, estudiar su comportamiento cualitativo (en particular, clasificar los puntos críticos según su estabilidad).
\solution
\begin{expla}
explicación.
\end{expla}
\end{problem}

\section{Hoja 2}
\begin{problem}[2]

Sea $f$ una función continua y supongamos que todo problema de valor inicial para la ecuación autónoma: \[x'=f(x)\]
tiene solución única, (P.V.I.) para el problema del valor inicial.

\ppart Demostrar que toda solución $x(t)$ no constante es una función estrictamente monótona.


\solution

\spart Sea $x$ una solución que no es constante, existe $t_0 \tlq x'(t_0) = f(x(t_0)) \neq 0$.

Vamos a reducirlo al absurdo: Supongamos que $\exists t_1 \tlq x'(t_1) = f(x(t_1)) = 0$ 

Entonces $u\equiv x(t_1) $ es también solución y el problema $\displaystyle \begin{array}{cc} v'=&f(v)\\v(t_1)=&x(t_1)\end{array}$ tiene 2 soluciones: $x$ ( no constante) y $u$ (constante), luego $x'(t)\neq 0 \forall t$ y $x$ es estríctamente monónota.

$x(t)$ es solución tal que $\lim_{t\to\infty} x(t) = c$, $u=c$ es solución si $f(c)=0$.

Entonces tenemos que:
\[f(c) = f(\lim_{t\to\infty} x(t)) = \lim_{t\to\infty} f(x(t)) = \lim_{t\to\infty}x'(t)=0\]
Quedando probada la contradicción por algo que se me escapa.
\end{problem}

\begin{problem}[3]


Probar que el cambio $z = ax + by + c$ transforma la ecuación $y'= f(ax + by + c)$
en otra de variables separadas. Aplicar este método para resolver
\solution
Al parecer es mazo facil asique pasamos de él (de momento)
\end{problem}

\begin{problem}[4]
Enunciado:
\solution

\[z=\frac{y}{x^k} \implies \left\{\begin{array}{cc}
y=x^kz\\
y'=kx^{k-1}z+x^kz'
\end{array}\right\}\]

Sustituyendo en la ecuación obtenemos (no estoy seguro de haber distinguido bien todos los caracteres):

\begin{gather*}
kx^{k-1}z + x^kz' = \frac{x^kz - x^{2k+1}z^2}{x+x^{k+2}z}\\
x^kz'= \frac{x^kz-x^{2k+1}z^2 - kx^kz-kx^{2k+1}z^2}{x+x^{k+2}z}\\
x^kz'= \frac{(1-k)x^kz-(1+k)x^{2k+1}z^2}{x(1+x^{k+1}z)}
\end{gather*}

Queremos encontrar un valor de $k$ para converrtirlo en una ecuación de variables separadas.

Estudiamos el caso $k=-1$: \[z' = \frac{2z}{x(1+z)}\]

\end{problem}


\begin{problem}[5]
\ppart
\ppart $xy'=\sqrt{x^2+y^2}$
\solution
\spart

\spart Aplicando el truco de las homogéneas:

\[y' = \sqrt{1+\left(\frac{y}{x}\right)^2}\]
En realidad esto está mal, porque en la ecuación del enunciado podemos tener pendientes negativas, pero en la transformada no, por lo que 
\[y' = \pm\sqrt{1+\left(\frac{y}{x}\right)^2}\] es lo correcto.

Como el valor de la pendiente sólo depende de $\frac{y}{x}$ (que en el fondo es la tangente del ángulo $\theta$ que forma la recta que pasa por $(0,0)$ y por $(x,y)$), por lo que el valor de las isoclinas será el mismo para todas las rectas dado un ángulo.

\[\begin{array}{|cc|}
(1,1) \to &y' = \sqrt{2}\\
(1,0) \to &y'=1\\
(0,1) \to &y'=\pm\infty
\end{array}\]

Haciendo el cambio $z=\frac{y}{x}$ tenemos:

\[z+xz'=\pm\sqrt{1+z^2} \implies z'=\frac{-z\pm\sqrt{1+z^2}}{x}\]

Ahora ya podemos integrar a ambos lados:

\[\int \frac{d z}{-z\pm\sqrt{1+z^2}} = \int\frac{dx}{x} = log |x| + C\]

Vamos a resolver la ecuación de la izquierda, utilizando el $cosh$ (otra posibilidad sería la tangente).

\begin{gather*}
\int \frac{\dif z}{-z\pm\sqrt{1+z^2}} \frac{-z\pm\sqrt{1+z^2}}{-z\pm\sqrt{1+z^2}} dz = \int \frac{z\pm\sqrt{1+z^2}}{1 + z^2-z^2}dz\\
\int\sqrt{1+z^2}dz = \int cosh^2(t)dt = \int \frac{2cosh(t) + 1}{2}dt\\
\text{Aplicando identidades trigonométricas hiperbólicas obtenemos}\\
= \frac{1}{4}senh(2t) + \frac{1}{2}t + C = \frac{1}{2}cosht(t)senh(t) + \frac{1}{2}t + C
\end{gather*}
Ahora deshacemos los cambios de variables

\[\frac{1}{2}\sqrt{1+z^2}z + \frac{1}{2}arcsenh(z) + C = \frac{1}{2} \left(\frac{y}{x}\right)^2 \pm \left(\frac{1}{2}\sqrt{1+\left(\frac{y}{x}\right)^2}\frac{y}{x}+\frac{1}{2}arcsenh\frac{y}{x}\right) = log|x|+C\]

Solución en forma implícita:

\[\frac{f(v)}{v^*} \convs[v\to\infty][][] 1\]

\todo{Completar}

\end{problem}

\begin{problem}[7]
\ppart
\ppart 
\ppart

\solution
\spart Para nosotros

\spart
Lo que tenemos en realidad es: \[\frac{dy}{dx} = \frac{x+y+4}{x+y-6}\]

Podemos recurrir a la interpretación geométrica. Si $AB\neq DE$ entonces son 2 rectas que se cortan en algún punto (que podemos imaginarnos que será problemático). Si $AB=DE$ tendremos el "cociente" de 2 rectas paralelas. 

Miramos el caso concreto para entender mejor la  generalización. Haciendo el cambio de variable $z=
x+y \implies z'=y'-1$.

Tomando este cambio tenemos:

\[z'-1 = \frac{z+4}{z-6}\] que es una ecuación autónoma, que ahora mismo no vamos a resolver.

\end{problem}

Vamos a resolver algún ejercicio de ecuaciones exactas.

\begin{problem}[9]
\ppart
\ppart
\ppart
...
\solution

\spart
\spart
\spart
\[\underbrace{(2xy^4+sen\,y)}_{M} + \underbrace{(4x^2y^3 + xcos\,y)}_{N}y' = 0\]

Para comprobar si es exacta comprobamos si $\displaystyle\dpa{M}{y} = \dpa{N}{x}$.

\begin{align*}
\dpa{M}{y} &= 6xy^3 + cos\,y\\
\dpa{N}{x} &= \\
&\dpa{M}{y} = \dpa{N}{x}
\end{align*}

La ecuación es exacta, por lo que existe un potencial $V(x,y)$ tal que las soluciones vienen dadas por los conjuntos de nivel del potencial, es decir, $V(x,y) = c$. Hacemos el cambio \[\partial_yV=N\]\[\partial_xN=M\]

Obtenemos: \[V(x,y) = \int M(x,y) dx = \int (2xy^2+sen\,y)dx x^2y^3+xsen\,y + W(y)\]

\[\begin{array}{cc}
\partial_y V(x,y) &= N(x,y) = 4x^2y^3 + xcos\,y\\
\downarrow &\,\\
4x^2+y^3+xcos\,y + W'(y)
\end{array}\]
De aquí deducimos que $W'(y) = 0 \implies W(y) = C, C\in\real$

Las soluciones son (escogiendo $W\equiv 0$): 
\[V(x,y) = x^2+y^4+xsen\,y = C, C\in\real\]

\spart...
\end{problem}

\begin{problem}[12]
\ppart
\ppart
\ppart
\ppart ...

\solution

\spart

\spart

\spart

\spart \[y+(2x - ye^y)y' = 0\]

No es exacta. 
\todo{Comprobación}

Vamos a aplicarle un factor integrante. Buscamos $\mu=\mu(x,y) \tlq \mu y+ \mu(2x-ye^y)y'=0$ sea exacta.

\[\dpa{}{y}\mu y = \dpa{}{y} \mu y + \mu\]
Por otro lado:
\[\dpa{}{x}(\mu(2x-ye^y)) = \dpa{}{x}\mu(2x-ey^y)+2\mu\]

Nos queda un sistema de ecuaciones en derivadas parciales aparentemente más complicado. El enunciado nos limita las opciones de elección de $\mu$. En nuestro caso, vemos que nos interesa $\mu(x,y) = \mu(y)$ para simplificar el problema.

Nos queda: \[\frac{\partial}{\partial y} \mu = \frac{d}{dy}\mu\]

\begin{align*}
\frac{d\mu}{dy}y+\mu&=2\mu\\
\frac{d\mu}{dy}y&=\mu\\
\frac{d\mu}{\mu} &= \frac{dy}{y}\\
log|\mu| = log|y| + C, C\in\real\\
\mu=ky,k\in\real
\end{align*}

Tomando $\mu(y)=y$, tenemos:

\[\underbrace{y^2}_{M}+\underbrace{(2xy-y^2e^y)}_{N}y'=0\] es exacta. 

\todo{comprobar}

Como es exacta entonces las soluciones vienen dadas por los conjuntos de nivel de un potencial $V(x,y) \tlq \partial_xV=M;\partial_yV=N$.

\[V(x,y) = \int M(x,y)dx = \int y^2 dx = y^2x + W(y)\]
La $W(y)$ la sacamos de la ecuación que no hemos usado:
\[2xy+W'(y) = \deriv{V}{y} = N(x,y) = 2xy-y^2e^y\]

Entonces: $W'(y) = -y^2e^y \implies W(y) =- \int y^2e^ydy$ que se resuelve integrando por partes y da \[W(y) = -y^2e^y + 2ye^y - 2e^y+K,K\in\real\]

Escogiendo $K=0$, las soluciones son: $V(x,y) = y^2x-y^2e^y + 2ye^y - 2e^y = C, C\in\real$.
\end{problem}

\begin{problem}[nose]
$y'-2xy=6xe^{x^2}$
\solution
Buscamos un factor integrante $\mu(x)$, de modo que $\displaystyle\mu( - 6xe^{x^2} - 2xy) + \mu y' = 0 $ sea exacta.

Se plantea:

\[\dpa{\mu}{x} = \deriv{\mu}{y}(...) - \mu 2 x\]
y se resuelve.

\end{problem}

\begin{problem}[17]
\ppart
\ppart
\ppart
\ppart
\solution
\spart
\spart $y'-2xy=6xe^{x^2}$.

Buscamos un factor integrante $\mu = \mu(x)$ de modo que \[\underbrace{\mu (-2xy-6xe^{x^2})}_{M}...\]

\todo{Completar}

\spart $x'+x = 2te^{-t}+t^2$. Si definimos $y = x_1,x_2$ tenemos que $y' + y=0$ una ecuación diferencial más facil.

Vamos a resolverlo con el \textbf{método de variación  de las constantes.} \index{Método\IS variación de las constantes.}

\begin{itemize}
\item[1]Buscamos una solución de la eucación homogenea (la que no tiene término independiente). En este caso es: $x'+x=0$. y vemos que serán exponenciales negativas: $x = ke^{-t}$.

\item[2] Buscamos soluciones de la ecuación general de la forma
$k=k(t)e^{-t}$.

Si $x$ es solución tenemos entonces que $x'+x = k'(t)e^{-t} - k(t)e^{t} + k(t)e^{t} = 2te^{-t}+t^2$.

Entonces tenemos que 
\begin{gather*}
k'(t) = 2t+t^2e^t\\
k(t) = \int 2t + t^2e^tdt = (1) = t^2+t^2e^t-\int 2te^tdt \\
 (1)  = t^2+t^2e^t-2te^t+\int e^tdt =\\ t^2+t^2-2te^t+2e^t+C,C\in\real\\
\end{gather*}
(1) resolviendo por partes.

\paragraph{Solución} Ahora que ya tenemos definido k(t), tenemos:

\[x(t) = k(t)e^{-t} = t^2e^{-t}+t^2-2t+22+Ce^{-t},\,c\in\real\]
\end{itemize}

\end{problem}

\begin{problem}[20]
\ppart
\solution
\spart $xy'' = y' +(y')^3$.

Parece complicada, pero si \textbf{reducimos el orden} \index{Método\IS Reductor de orden} queda de variables separadas.

Si hacemos el cambio: $z=y';\,z'=y''$ tenemos:

\[xz' = z+z^3 \dimplies \int \frac{dz}{z+z^3} = \int\frac{dx}{x}\]

Se deja como ejercicio para el lecto r la resolución (de momento).
\end{problem}

\begin{problem}[22]
\ppart Demostrar que...
\ppart Resolver $\displaystyle y'=\frac{y}{x} + x^3y^2 -x^5$, siendo $y_1=x$ una solución.
\solution
\spart $$y'=p(x) + q(x)t+r(x)y^2$$

Cambio de variable, sea $z = y - y_1$ siendo $y_1$ una solución conocida. 

Sustituyendo tenemos:

\[z'+ y_1' = p(x)+ q(x)(z+y_1)+r(x)(z^2+y_1^2+2zy_1) \dimplies z'+ \underbrace{y_1'}_{A} = \underbrace{p(x) + q(x)y_1+r(x)y_1^2}_{B} + (q(x) + 2y_1)z+r(x)\]
Si nos fijamos $A=B$, porque $y_1$ es solución, asíque podemos simpflificar  quedando la ecuación:$$z'=(q(x)+2y_1)z+r(x)z^2$$ Que es una ecuación de tipo Bernoulli que sí sabemos resolver.

\spart
$$y'=\frac{y}{x} + x^3y^2 -x^5$$

Hacemos el cambio descrito en el apartado anterior, que es el \textbf{método para resolver ecuaciones de Ricatti} \index{Método\IS Ricatti}

Quedando: \[z'+1 = \frac{(z+x)}{x}x^3(z^2+x^2+2xz)-x^5\]

Haciendo cuentas...

\[z'=\left(\frac{1}{x}+2x^4\right)z+x^3z^2\], que es una ecuación de tipo Bernoulli, que resolveremos aplicando el siguiente cambio de variables para conventir la ecuación en una lineal.

\[\begin{array}{cc}
W&=z^{1-2} = \frac{1}{z}\\
z'&=\frac{-W'}{W^2}
\end{array}\]

Nos queda transformada (tras unas cuentas) en

$$-W' = W\left(\frac{1}{x}+2x^4\right)+x^3$$

Que es una ecuación lineal que se deja como ejercicio para el lector (o para Parra o Kasner).

\end{problem}

\section{Hoja 3}

Tenemos el sistema lineal $x'=Ax$, siendo $A$ una matriz \textbf{real} $4\x4$, cuyos autovalores (previamente calculados) son $\lambda_i = \{2,1+2i,1-2i\}$, siedo $\lambda_1 = 2$ un autovalor doble (multiplicidad algebraica = 2 $\ge$ multiplicidad geométrica).

Los autovectores asociados: 

\begin{align*}
\lambda_1=2 & \to \begin{pmatrix}
1\\1\\0\\0
\end{pmatrix}\\
\lambda_2 = 1+2i & \to \begin{pmatrix}
3+i\\2\\i\\1+2i
\end{pmatrix} = u\\
\lambda_3 = 1-2i & \to \gor{u}\\
\end{align*}

Tenemos:

\[A=\underbrace{\begin{pmatrix}
1&0&3+i&3-i\\
1&2&2&2\\
0&0&i&-i\\
0&1&1+2i&1-2i
\end{pmatrix}}_{P} 
\underbrace{\begin{pmatrix}
2 & 1 & 0 & 0\\
0 & 2 & 0 & 0\\
0 & 0 & 1+2i & 0\\
0 & 0 & 0 & 1-2i
\end{pmatrix}}_{J}P^{-1}\]

Buscamos una matriz fundamental:

\[\Phi(t) = e^{tA} = P e^{tJ}P^{-1}\]

Esta matriz fundamental cumple que $\Phi(0) = t$. La matriz $P e^{tJ}$ también es fundamental pero no cumple $\Phi(0) = t$.

Vamos a calcular $\Phi(t)$.

\[\Phi(t) = P\begin{pmatrix}
e^2t&te^{2t}&0&0\\
0&e^2t&0&0\\
0&0&e^{(1+2i)t}&0\\
0&0&0&e^{(1-2i)t}
\end{pmatrix} = \begin{pmatrix}
e^{2t} & te^{2t} & (3+i)e^{(1+2i)t} & (3-i)e^{(1-2i)t}\\
e^{2t} & & 2e^{(1+2i)t} & 2 e^{(1-2i)t}\\
0 & 0 & ie^{(1+2i)t} & -ie^{(1-2i)t}\\
0 & e^{2t} & (1+2i) e^{(1+2i)t} & (1-2i) e^{(1-2i)t}
\end{pmatrix}\]


Cada una de las columnas de esta matriz fundamental es una solución. Cualquier combinación lineal de columnas también será una solución. Como nos interesa una solución \textbf{real} (porque el problema planteado era real, en cuanto a que la matriz A era real y buscamos soluciones reales).

Si nos damos cuenta podemos combinar linealmente la 3ª y la 4ª columna que son conjugadas.

Para ello, primero expresamos en senos y cosenos las exponenciales complejas (para facilitar el cálculo)

\[\begin{pmatrix}
e^{2t} & te^{2t} & e^t\left([3cos(2t)-sen(2t)] + i [3sen(2t) + cos(2t)]\right)& e^t\left([3cos(2t)-sen(2t)] - i [3sen(2t) + cos(2t)]\right) \\
e^{2t} & e^{2t}(i+2)& 2e^t[cos(2t) + isen(2t)] & 2e^t[cos(2t) - isen(2t)]\\
0 & 0 & e^t[-sen(2t) + icos(2t)] & e^t[-sen(2t) - icos(2t)] \\
0 & e^{2t} & e^t\left([(cos(2t) - 2sen(2t)] + i[sen(2t) + 2cos(2t)]\right) & e^t\left([cos(2t) - 2sen(2t)] - i[sen(2t) + 2cos(2t)]\right)
\end{pmatrix}\]

Vamos a poner en la tercera columna la combinación lineal $\frac{u+\gu}{2}$ (que es la parte real) y en la cuarta columna $\frac{u-\gu}{2}$ que es la parte imaginaria, siendo $u$ la tercera columna (y $\gu$ su conjugado), obteniendo:

\[\begin{pmatrix}
e^{2t} & te^{2t} & e^t\left(3cos(2t)-sen(2t)\right)& e^t\left(3sen(2t) + cos(2t)\right) \\
e^{2t} & e^{2t}(i+2)& 2e^t[cos(2t)] & 2e^t[sen(2t)]\\
0 & 0 & -e^t[sen(2t)] & e^t[cos(2t)] \\
0 & e^{2t} & e^t[(cos(2t) - 2sen(2t)] & e^t[sen(2t) + 2cos(2t)]
\end{pmatrix}\]

Cualquier solución se escribe como $\Phi(t)C$, con $C = (c_1,c_2,c_3,c_4)$.

\newpage


\newpage
\printindex
\end{document}
