\documentclass[nochap]{apuntes}
\usepackage{anysize} 
\usepackage{dsfont}
\usepackage{amssymb}
\usepackage{textcomp}
\usepackage{plain}

%opening

\title{Ejercicios resueltos de ecuaciones diferenciales}
\author{Alberto Parramón Castillo}
\date{2013 / 2014}

\def\misistema[#1]#2{\def\Parente{#1} \def\Pi{i}%
\def\Pd{d} \def\Pid{id} \ifx \Parente\Pi%
\left\{\begin{array}{l} #2 \end{array}\right.
\fi%
\ifx \Parente\Pd%
\left.\begin{array}{r} #2 \end{array}\right\}
\fi%
\ifx \Parente\Pid\left\{\begin{array}{c} #2 \end{array}\right\}
\fi%
}

\makeatletter
\def\sistema{\@ifnextchar[{\misistema}{\misistema[d]}}
\makeatother


\begin{document}

\pagestyle{plain}
\maketitle
\tableofcontents
\newpage

\section{Hoja 1}

\begin{problem}[1]
\ppart Comprobar que, para cada valor de la constante C, la función
\begin{gather*}
y=e^{x^2}\left(C+\int_0^x e^{-t^2}dt\right)
\end{gather*}
	es una solución de la ecuación diferencial
\begin{gather*}
y'=2xy+1
\end{gather*}

\ppart Sean $y_1(x)$, $y_2(x)$ dos funciones de la familia anterior, correspondientes a dos valores distintos de la constante $C$. Hallar la ecuación diferencial que satisface $u(x)=y_1(x)-y_2(x)$.

\solution

\spart
\begin{expla}
Si $y$ es una solución de la ecuación diferencial $y'$, entonces, si derivamos con respecto a $x$, $y$, deberíamos llegar a algo parecido a $y'$.
\end{expla}
$y'=\underbrace{2x}_{2x}\underbrace{e^{x^2}(C+\int_0^x e^{-t^2}dt)}_y+\underbrace{e^{x^2}e^{-x^2}}_1$
\begin{expla}
Puede que os líe el término $e^{-x^2}$ del final de la fórmula. Sale de derivar con respecto a $x$ la integral. Esto se debe al \textit{Teorema Fundamental del Cálculo}, que aplicado a este caso nos dice que: 
\begin{gather*}
\dpa{}{x}\left(\int_0^x f(x)\right)=\dpa{}{x}[F(x)]_0^x=\dpa{F(x)}{x}-\dpa{F(0)}{x}=f(x)-0=f(x)
\end{gather*}
Sustituyendo en el razonamiento anterior $f(x)$ por $e^{-x^2}$, obtenemos dicho término.
\end{expla}


\spart
Partimos $y_1$ e $y_2$, soluciones de la ecuación diferencial. Y tenemos $u=y_1-y_2$. La ecuación diferencial que satisfaga $u$ deberá ser $u'$, por tanto, vamos a calcularla:

$u'=y_1'-y_2'$

Como $y_1$ e $y_2$, son soluciones, tenemos que:

$y_1'-y_2'=2xy_1+1-2xy_2-1=2x(y_1-y_2)=2xu$

Por tanto, la ecuación diferencial que satisface $u$ es: $u'=2xu$.

\end{problem}
\newpage

\begin{problem}[2]
\ppart Hallar los valores de $m$ para los cuales $y=e^{mx}$ es solución de la ecuación diferencial
\begin{gather*}
2y'''+y''-5y'+2y=0
\end{gather*}

\ppart A partir de combinaciones lineales de las soluciones encontradas en el apartado anterior. hallar una solución que verifique las condiciones
\begin{gather*}
y(0)=0,	y'(0)=1,  y''(0)=-1
\end{gather*}

\solution

\begin{expla}
Aquí irá una breve explicación sobre este ejercicio.
\end{expla}

\spart

Puesto que las soluciones son de la forma: $y=e^{mx}$ tenemos que:\\
\begin{gather*}
y'=me^{mx}\;\;\;\;y''=m^{2}e^{mx}\;\;\;\;y'''=m^{3}e^{mx}
\end{gather*}

Por tanto, sustituyendo en la fórmula dada tenemos que:
\begin{gather*}
2y'''+y''-5y'+2y=0 \implies 2m^{3}e^{mx}+m^{2}e^{mx}-5me^{mx}+2e{mx}=0 \implies \\ \implies e^{mx}(2m^3+m^2-5m+2)
\end{gather*}
Que, resolviendo por Ruffini da como soluciones:
$m=1$\;\;\;\;$m=-2$\;\;\;\;$m=1/2$

\spart
El resultado tiene que ser combinación lineal de las soluciones anteriores, es decir será de la forma:
\begin{gather*}
y=ae^{-2x}+be^{mx}+ce^{mx/2} 
\end{gather*}
Planteamos un sistema de 3 ecuaciones con 3 incógnitas con los datos proporcionados por el enunciado:

 $\sistema[i]{y(0)=a+b+c=0\\y'(0)=-2a+b+c/2=1\\y''(0)=4a+b+c/4=-1}$ $\implies$ $\sistema[i]{a=-1/3\\b=1/3\\c=0}$


\end{problem}
\newpage

\begin{problem}[3]
Comprobar que para cada valor de la constante $C>0$, la identidad
\begin{gather*}
Cx-y \int_0^x\frac{\sen t}{t}dt=0
\end{gather*}
define $y$ como función de $x$. Hallar la ecuación diferencial que satisfaga todas las funciones de la familia. Hallar $y(0)$ e $y'(0)$ para cada una de ellas.

\solution

\begin{expla}
Aquí irá una breve explicación sobre el ejercicio
\end{expla}

Solución

\end{problem}
\newpage

\begin{problem}[4]
\ppart Utilizando isoclinas, esbozar las soluciones de $y'=y^2-1$
\ppart Resolver explícitamente la ecuación $y'=y^2-1$, $y(0)=0$ y comparar el resultado con lo obtenido por el método de isoclinas empleado anteriormente.

\solution



\spart
\begin{expla}
La siguiente figura muestra la pendiente de la curva solución que buscamos en cada punto. Como la pendiente es la derivada de la $y$, entonces, los valores que obtenemos en $y'$ son con los que dibujaríamos la curva solución. Por supuesto, dependiendo del punto en que empiece nuestra curva, está seguirá un camino u otro.

\end{expla}
$y'=0$ en $y=1$ e $y=-1$, por tanto en esas rectas horizontales la pendiente de la curva solución será 0, además esas dos rectas serán a su vez una solución de la ecuación diferencial.

Para valores de $y$ entre $1$ y $-1$, la pendiente de la curva variará entre $-1$ y $0$.

Para valores altos de y, la pendiente aumentará infinitamente. Y para valores negativos de y, la pendiente será más negativa y aumentará su negativo también infinitamente.

\easyimgw{Imagenes/Ej4a.png}{Isoclinas de $y'=y^2-1$}{4a}{0.6}

\newpage
\spart 
\begin{expla}
Nos encontramos ante un problema de Cauchy, en el que partimos de una ecuación diferencial $y'=y^2-1$, y de un dato $y(0)=0$. Y lo que hacemos es integrar la ecuación diferencial, obteniendo al hacerlo una constante, que tenemos que eliminar con el dato
\end{expla}
Antes de integrar, vamos a ver si se puede resolver fácilmente. Si escribimos:
\begin{gather*}
\frac{y'}{y^2-1}=1
\end{gather*}

Vemos que tenemos una ecuación de "variables separadas". Procedemos a integrar:
\begin{gather*}
\int\frac{y'}{y^2-1}dx=\int1dx
\end{gather*}

\begin{expla}
Para resolver la integral cambiamos variables: $y=u$, $y'dx=du$. Y sustituimos:

\begin{gather*}
\int\frac{du/dx}{u^2-1}du/y'=\int\frac{1}{u^2-1}du=\int 1dx
\end{gather*}

Resolvemos con raíces:
\newpage
\begin{gather*}
\frac{1}{u^2-1}=\frac{A}{u+1}+\frac{B}{u-1}
\end{gather*}

Nos queda:
$1=A(u-1)+B(u+1)$

Sustituimos:

$u=1$ $\implies$ $B=1/2$

$u=-1$ $\implies$ $A=-1/2$

Y obtenemos finalmente que:
\begin{gather*}
\int\frac{y'}{y^2-1}dx=\int\frac{1}{u^2-1}du=-\frac{1}{2}\ln|u+1|+\frac{1}{2}\ln|u-1|=-\frac{1}{2}\ln|y+1|+\frac{1}{2}\ln|y-1|
\end{gather*}
\end{expla}

Por tanto tenemos que:
\begin{gather*}
-1/2\ln|y+1|+1/2\ln|y-1|=x+C\\\\
\frac{\ln|y-1|}{\ln|y+1|}=2x+C \implies \abs{\frac{y-1}{y+1}}=e^{2x+c} \implies y-1=ye^{2x+c} +e^{2x+c} \implies \\\\
\implies y(1-e^{2x+c})=e^{2x+c}+1 \implies y=\frac{e^{2x+c}+1}{1-e^{2x+c}} \implies \;\; y=\frac{e^{2x}k+1}{1-e^{2x}k} \;\;\;\forall k∈\real
\end{gather*}

NOTA: suponemos que el valor absoluto es "absorbido" por la constante $c$, y por ello no lo tenemos en cuenta al acabar los cálculos.

Sustituimos ahora el dato $y(0)=0$:
\begin{gather*}
0=\frac{k+1}{1-k} \implies k=-1 
\end{gather*}

Por tanto, la solución obtenida es:
\begin{gather*}
y=\frac{1-e^{2x}}{1+e^{2x}}
\end{gather*}
Que podemos ver que cuadra perfectamente con las isoclinas, entre valores de $y=1$ e $y=-1$, concretamente a la solución que pasa por el origen.
\easyimgw{Imagenes/Ej4b.png}{$y=\frac{1-e^{2x}}{1+e^{2x}}$}{4b}{0.6}


\end{problem}
\newpage

\begin{problem}[5]
Trazando algunas isoclinas, esbozar las soluciones de las siguientes ecuaciones:
\ppart $y'=\sen (y+x)$
\ppart $x'= \sqrt{t^2+x^2}$

\solution

\begin{expla}
Aquí irá una breve explicación sobre el ejercicio
\end{expla}

\spart
$y'=0$ si $x+y=k\pi$. Por tanto: pendiente será 0 en rectas $y=-x+k\pi$

$y'=1$ si $x+y=\pi/2 +2\pi k$. Por tanto: pendiente será 0 en rectas $y=-x-\pi/2+2\pi k$.

$y'=-1$ si $x+y=-\pi/2 +2\pi k$. Por tanto: pendiente será 0 en rectas $y=-x+\pi/2+2\pi k$.
\easyimgw{Imagenes/Ej5a.png}{Isoclinas de $y'=\sen (y+x)$}{5a}{0.6}

\newpage
\spart 
En este caso tenemos una curva $x(t)=x$, podemos ver que los conjuntos de nivel (es decir, los valores que toman $t$ y $x$ para valores fijos de x') son circunferencias de radio x' centradas en el origen:

$x'= \sqrt{t^2+x^2} \implies  (x')^2= t^2+x^2$

Por tanto, las isoclinas quedan:
\easyimgw{Imagenes/Ej5b.png}{Isoclinas de $x'= \sqrt{t^2+x^2}$}{5b}{0.6}

\end{problem}
\newpage

\begin{problem}[6]
\ppart Describir geométricamente la familia de curvas
\begin{gather*}
x^2+y^2+2Cx=0
\end{gather*}
y calcular una ecuación diferencial que la satisfaga.
\ppart Determinar la ecuación diferencial de la familia de curvas
\begin{gather*}
y(x) = x\cos (x+C)
\end{gather*}

\solution

\spart
\begin{expla}
Ejercicio tipo:
Dada una familia de curvas, queremos encontrar su ecuación diferencial. Para ello procedemos de la siguiente forma:
1)Derivamos con respecto a x la función ($F(x,y(x),c)=0$)
2)Eliminamos C utilizando la derivada y la función original.

En estos ejercicios partimos de la ecuación de las curvas solución de una ecuación diferencial (por eso tienen el parámetro constante C). Para llegar a la ecuación diferencial no tenemos más que derivar respecto a x (o respecto a t si tenemos $x(t)=0$ por ejemplo), y eliminar la constante C con ayuda de la ecuación original.
\end{expla}

1)Tenemos: $F(x,y(x),c)=x^2+y^2+2cx=0$

Derivamos con respecto a $x$.
\begin{gather*}
\dpa{F(x,y(x),c)}{x}=2x+2yy'+2c=0
\end{gather*}

2)Despejamos C de la ecuación original
\begin{gather*}
c=\frac{-x^2-y^2}{2x}
\end{gather*}

Y sustituimos, obteniendo la ecuación diferencial:
\begin{gather*}
2x+2yy'-\frac{x^2+y^2}{x}=0
\end{gather*}

Contestando a la primera parte, la de la interpretación geométrica, podemos ver, completando cuadrados que:

$x^2+y^2+2cx=0 \implies x^2+y^2+2cx+c^2=c^2 \implies (x+c)^2 + y^2 = c^2 $

Por tanto, la familia de curvas son circunferencias centradas en el punto (-c,0) de radio c.
\newpage
\spart 
1) Tenemos $F(x,y(x),c)=y-x\cos(x+c)=0$

Derivamos con respecto a x:
$y'-\cos(x+c)+x\sen(x+c)=0$

Y operamos para quitarnos la c:
\begin{gather*}
y'=\cos(x+c)+x\sen(x+c)=x\frac{\cos(x+c)}{x}+x\sen(x+c)=\frac{y}{x}+x\sen(x+c)=\\\\
=\frac{y}{x}+x\sqrt{1-{cos(x+c)}^2}=\frac{y}{x}+x\sqrt{1-\frac{x^2{cos(x+c)}^2}{x^2}}=\frac{y}{x}+x\sqrt{1-\frac{y^2}{x^2}}=\\\\
=\frac{y}{x}\pm\sqrt{y^2+x^2}
\end{gather*}
\begin{expla}
El $\pm$ viene de meter la $x$ (que podía ser tanto negativa como positiva) dentro de la raíz.
\end{expla}

\end{problem}
\newpage


\begin{problem}[7]
Dada la ecuación diferencial
\begin{gather*}
y'=(1+x)y+1-3x+x^2
\end{gather*}
calcular los primeros términos del desarrollo de Taylor de la solución que satisface $y(0) = 0$.

\solution

\begin{expla}
Aquí irá una breve explicación sobre el ejercicio
\end{expla}

Solución

\end{problem}

\newpage
\begin{problem}[8]
Esbozar las siguientes familias uniparamétricas de curvas y hallar sus trayectorias ortogonales:
\ppart $xy=C$.
\ppart $y=Ce^x$
\ppart $y=Cx^n$, donde $n$ es un entero positivo. Explicar qué sucede con las trayectorias ortogonales cuando aumentamos el valor del entero $n$.

\solution

\begin{expla}
\end{expla}
\spart
Aquí va la solución del apartado a).

\spart
Aquí va la solución del apartado c).

\spart
Aquí va la solución del apartado b).

\end{problem}

\newpage
\begin{problem}[9]
Hallar la familia de curvas ortogonales a la familia de parábolas
\begin{gather*}
y^2-Cx=\frac{C^2}{4}
\end{gather*}

\solution

\begin{expla}
Aquí irá una breve explicación sobre el ejercicio
\end{expla}

Solución

\end{problem}
\newpage

\begin{problem}[10]
Hallar la familia de curvas ortogonales a la familia de circunferencias definida por
\begin{gather*}
x^2+(y-C)^2=C^2
\end{gather*}
Interpretar el resultado geométricamente.

\solution

\begin{expla}
Aquí irá una breve explicación sobre el ejercicio
\end{expla}

Solución

\end{problem}
\newpage

\begin{problem}[11]
Hallar las trayectorias ortogonales de las siguientes familias uniparamétricas de curvas expresadas en coordenadas polares:
\ppart
$r=C(1+\cos \theta)$
\ppart
$r=2C\sen \theta$

\solution

\begin{expla}
Aquí irá una breve explicación sobre el ejercicio
\end{expla}

\spart
solución a).

\spart
solución b)

\end{problem}
\newpage

\begin{problem}[12]
Hallar las curvas que satisfacen las condiciones geométricas siguientes:
\ppart
El segmento de la tangente limitado por los ejes coordenados tiene como punto medio al punto de tangencia.
\ppart
La proyección sobre el eje $OX$ de la parte de la tangente entre $(x,y)$ y el eje $OX$ tiene longitud 1.
\ppart
EL ángulo entre el radio polar y la tangente es constante.
\ppart
La curva pasa por $(0,0)$ y está contenida en el primer cuadrante, de modo que el área bajo la curva desde $(0,0)$ hasta $(x,y)$ es un tercio del área del rectángulo que tiene a esos puntos como vértices opuestos.
\solution

\begin{expla}
Breve explicación.
\end{expla}

\spart
Aquí va la solución del apartado a).

\spart 
Aquí va la solución del apartado b).

\spart
$\f$ Aquí va la solución del apartado c).

\spart 
Aquí va la solución del apartado d).

\end{problem}
\newpage
\begin{problem}[13]
Para cada una de las ecuaciones diferenciales siguientes, hallar la solución particular que satisface la condición inicial dada.
\ppart $y'=e^{3x-2y}$, con $y(0)=0$.
\ppart $e^{-y}+(1+x^2)y'=0$, con $y(0)=0$.
\ppart $xyy'=(x+1)(y+1)$, con $y(1)=0$.

\solution

\begin{expla}
\end{expla}
\spart
Aquí va la solución del apartado a).

\spart
Aquí va la solución del apartado c).

\spart
Aquí va la solución del apartado b).

\end{problem}
\newpage
\begin{problem}[14]
Según la \textit{Ley de enfriamiento de Newton} la tasa de variación de la temperatura en un cuerpo es proporcional a la diferencia de temperatura con el ambiente. Si una barra de hierro a 100\gradc C se enfría a 90\gradc C en 5 segundos cuando se deja a una temperatura ambiente de 20\gradc C, ¿cuánto tardará en estar a 30\gradc C?
\solution
\begin{expla}
De la primera frase tenemos que obtener la ecuación diferencial. \textit{La variación de la temperatura en un cuerpo es proporcional a la diferencia de temperatura con el ambiente}

$\Delta(T)=K(T-A)$
(\textit{Ley de enfriamiento de Newton})

Así definimos $T(t)$ como la temperatura del cuerpo en cada instante de tiempo. $\Delta(T)=T'(t)$ como la variación de esa temperatura, $A$ como la temperatura ambiente (suponemos constante); y $K$ como una constante (que será la constante que crea la familia de curvas, y la cual tendremos que eliminar)
\end{expla}
Primero vamos a obtener $T(t)$, para ello integramos:
\begin{gather*}
\frac{T'}{T-A}=k \implies \int\frac{T'}{T-A}dt=\int kdt  \implies \ln(T-A)=KT+c \;\;\;\forall c∈\real \\\\
\implies T=A+e^{Kt}C \;\;\;\forall C∈\real
\end{gather*}

Sabemos: $T(0)=100$, $T(5)=90$ y $A=20$

$T(0)=20+100=C \implies C=80$

$T(5)=90=20+e^{5K}80 \implies 70=e^{5K}80 \implies \frac{\ln(7/8)}{5}=k$

Por tanto:
$T(t_1)=30=20+e^{\frac{\ln(7/8)}{5}t_1}$

Se despeja $t_1$ y terminado. 

\end{problem}
\newpage
\begin{problem}[15]
En una sala que está a una temperatura de 20\gradc C nos sirven dos tazas de café, a una temperatura de 40\gradc C. Disponemos de leche fría, a una temperatura de 10\gradc C. En una de las tazas echamos una cantidad de leche fría igual a la de café contenido en su interior, y esperamos cinco minutos. En la otra taza esperamos cinco minutos, y después agregamos la leche. Determinar cual de las dos tazas tiene el café con leche más caliente.
\solution
\begin{expla}
Utilizamos las mismas ecuaciones que en el ejercicio anterior: $\Delta(T)=K(T-A)$ y $T=A+e^(Kt)C$
Como la temperatura disminuye, la constante K es negativa. Como tenemos dos ecuaciones, una para cada taza, tendremos también dos constantes C diferentes.
\end{expla}

$T_1(t)=A+C_1e^{kt}$

$T_2(t)=A+C_2e^{kt}$\\



\end{problem}
\newpage
\begin{problem}[16]
Una bola de naftalina tiene inicialmente un radio de 1 cm. Al cabo de un mes su radio se ha reducido a 0,5 cm. Suponiendo que la naftalina se evapora a un ritmo proporcional a la superficie de contacto con el aire, hallar la evolución del radio de la bola en función del tiempo.
\solution
\begin{expla}
explicación.
\end{expla}
\end{problem}
\newpage
\begin{problem}[17]
Un día comenzó a nevar por la mañana y siguió cayendo la nieve de forma constante todo el día. A las 12 del mediodía una quitanieves comenzó a limpiar una carretera, con velocidad inversamente proporcional al espesor de la nieve depositada. Sabiendo que a las 2 de la tarde había limpiado 2 km., y que a las 4 de la tarde había limpiado 1 km. más, determinar a qué hora comenzó a nevar. 
\solution
\begin{expla}
explicación.
\end{expla}
\end{problem}
\newpage
\begin{problem}[18]
Cuatro hormigas situadas en las esquinas de una mesa cuadrada de lado 1 comienzan a andar simultáneamente a la misma velocidad, cada una en la dirección de su vecina más próxima en la dirección contraria a las agujas del reloj. Tomando coordenadas polares con origen en el centro de la mesa y eje polar a lo largo de una diagonal, hallar la trayectoria de la hormiga que parte del eje polar.
\solution
\begin{expla}
explicación.
\end{expla}
\end{problem}
\newpage
\begin{problem}[19]
Una población de bacterias sigue la \textit{Ley de Malthus} (la tasa de variación es proporcional al número de individuos) se duplica al cabo de 24h. ¿Cuánto tardará en triplicarse.
\solution
\begin{expla}
explicación.
\end{expla}
\end{problem}
\newpage
\begin{problem}[20]
Supongamos que una población sigue el modelo $p'=bp^2-ap$ con $a,b>0$.
\ppart
Representar en un diagrama de fases, clasificando los puntos críticos del sistema.
\ppart
Demostrar que si $p(t_0)<\frac{a}{b} $, entonces la población tiende a extinguirse.

\solution

\begin{expla}
Aquí irá una breve explicación sobre el ejercicio
\end{expla}

\spart
solución a).

\spart
solución b)

\end{problem}
\newpage
\begin{problem}[21]
dada la ecuación $y'=\cos y$, sin encontrar las soluciones explícitamente, estudiar su comportamiento cualitativo (en particular, clasificar los puntos críticos según su estabilidad).
\solution
\begin{expla}
explicación.
\end{expla}
\end{problem}






\newpage
\section{Para que yo aprenda latex}

\begin{problem}[Ejemplo]

$\appl{d}{\real^2}{\real}$ homógenea\footnote{$(f(tx,ty) = t^mf(x,y)$} con $m>0$. F acotada superiormente sobre la circunferencia unidad.

\ppart Probar que $F$ es continua en $(0,0)$

\ppart Si $m=0$ ¿F continua en (0,0)?

\solution

\spart
\subparagraph{1)}
Vamos a ver que $F(0,0) = 0$.

$F(0,0) = F(t0,t0) = t^mF(0,0) \implies F(0,0) = 0$

\subparagraph{2)}
Aplicando la caracterización por sucesiones de la continuidad: $\{X_n\} \rightarrow 0 \implies F(X_n) \rightarrow 0$ si $F$ continua.

Sea $Z_n = \{(X_n,Y_n)\}, n\in \mathbb{N}$ una sucesión tal que $Z_n \rightarrow (0,0)$.

\begin{gather*}\abs{F_n} = \abs{F\left(\frac{Z_n}{\md{Z_n}}\md{Z_n}\right)} =
\md{X_n}^n \abs{F\left(\frac{Z_n}{\md{Z_n}}\right)} \implies \\
\exists M>0 \tlq \md{Z_n}^n \abs{F\left(\frac{Z_n}{\md{Z_n}}\right)} \leq \md{Z_n}^n M \rightarrow 0 = F(0,0)
\end{gather*}

\spart
Sí, $m=0 \implies F$ constante. Se puede demostrar pasando a coordenadas polares $x=rcos(t),y=rsen(t)$

\end{problem}

\begin{expla}
Primero demostramos que el neutro es único. \\
Sean $e$, y $e'$ dos elementos neutros de $G$, se cumple que $e\ast e'=e'$, pero tambien se cumple que $e'\ast e=e$. Esto implica que $e'=e$.

Por otro lado, si suponemos la existencia de dos elementos inversos $a',a''\in G$, entonces $e=a\ast a'=a\ast a''$. \\
Si multiplicamos por $a$ en ambos lados de la ecuación tenemos: $a\ast (a\ast a')=(a\ast a')\ast a''$, pudiendo reordenar los paréntesis por la 
propiedad asociativa. Así pues, obtenemos $a'=a''$.
\end{expla}

\begin{defn}[Grupo]
Dado $G\neq \emptyset$  y $\appl{\varphi}{G\ast G}{G}$, diremos que $G$ es un grupo si cumple las siguientes propiedades:
\begin{enumerate}
\item \textbf{Cerrado por la operación}. La combinación por $\ast$  de dos elementos de G da otro elemento del mismo. 
\item $\ast$ es asociativa.
\item \textbf{Existencia del elemento neutro}. $\exists  e \in G \tq \; \forall\, x\in G\; x\ast e=x$.
\item \textbf{Existencia del inverso}. $\forall x \in G \exists x' \in G \tq x\ast x'=x'\ast x=e$.
\end{enumerate}
\end{defn}

\begin{example}
  $ (\ent, +); (\real, +); (\real, \cdot); (\real/x>0, \cdot)$ son grupos, mientras que 
  $(\ent, \cdot); (\real, \cdot)$ no lo son.
  
  Otro ejemplo: A partir de un conjunto $A$ definimos $B(A)$ como el conjunto de todas las biyecciones de A en sí mismo.  Puesto que la composición de dos biyecciones es otra biyección, la composición es una operación definida sobre B(A). Si tomamos como neutro la biyección identidad y como inverso, la función inversa, podemos comprobar que $B(A)$ es un grupo respecto a la composición.
\end{example}

\begin{theorem}[Unicidad\IS del neutro e inverso]
  En todo grupo se cumplen las propiedades de unicidad del elemento neutro y del inverso.
\end{theorem}


Vamos a definir ahora las transformaciones lineales rígidas, que nos resultarán muy interesantes para definir algunos grupos.

\begin{problem}




\solution
\easyimgw{Patata.jpg}{Leyenda}{lblEtiqueta}{0.3}
ver figura \ref{lblEtiqueta}

\end{problem}


\newpage
\printindex
\end{document}
