\documentclass[nochap]{apuntes}
\usepackage{anysize} 
\usepackage{dsfont}
\usepackage{amssymb}
\usepackage{textcomp}
\usepackage{plain}

%opening

\title{Ejercicios resueltos de ecuaciones diferenciales}
\author{Alberto Parramón Castillo}
\date{2013 / 2014}

\def\misistema[#1]#2{\def\Parente{#1} \def\Pi{i}%
\def\Pd{d} \def\Pid{id} \ifx \Parente\Pi%
\left\{\begin{array}{l} #2 \end{array}\right.
\fi%
\ifx \Parente\Pd%
\left.\begin{array}{r} #2 \end{array}\right\}
\fi%
\ifx \Parente\Pid\left\{\begin{array}{c} #2 \end{array}\right\}
\fi%
}

\makeatletter
\def\sistema{\@ifnextchar[{\misistema}{\misistema[d]}}
\makeatother


\begin{document}

\section{Para que yo aprenda latex}

\begin{problem}[Ejemplo]

$\appl{d}{\real^2}{\real}$ homógenea\footnote{$(f(tx,ty) = t^mf(x,y)$} con $m>0$. F acotada superiormente sobre la circunferencia unidad.

\ppart Probar que $F$ es continua en $(0,0)$

\ppart Si $m=0$ ¿F continua en (0,0)?

\solution

\spart
\subparagraph{1)}
Vamos a ver que $F(0,0) = 0$.

$F(0,0) = F(t0,t0) = t^mF(0,0) \implies F(0,0) = 0$

\subparagraph{2)}
Aplicando la caracterización por sucesiones de la continuidad: $\{X_n\} \rightarrow 0 \implies F(X_n) \rightarrow 0$ si $F$ continua.

Sea $Z_n = \{(X_n,Y_n)\}, n\in \mathbb{N}$ una sucesión tal que $Z_n \rightarrow (0,0)$.

\begin{gather*}\abs{F_n} = \abs{F\left(\frac{Z_n}{\md{Z_n}}\md{Z_n}\right)} =
\md{X_n}^n \abs{F\left(\frac{Z_n}{\md{Z_n}}\right)} \implies \\
\exists M>0 \tlq \md{Z_n}^n \abs{F\left(\frac{Z_n}{\md{Z_n}}\right)} \leq \md{Z_n}^n M \rightarrow 0 = F(0,0)
\end{gather*}

\spart
Sí, $m=0 \implies F$ constante. Se puede demostrar pasando a coordenadas polares $x=rcos(t),y=rsen(t)$

\end{problem}

\begin{expla}
Primero demostramos que el neutro es único. \\
Sean $e$, y $e'$ dos elementos neutros de $G$, se cumple que $e\ast e'=e'$, pero tambien se cumple que $e'\ast e=e$. Esto implica que $e'=e$.

Por otro lado, si suponemos la existencia de dos elementos inversos $a',a''\in G$, entonces $e=a\ast a'=a\ast a''$. \\
Si multiplicamos por $a$ en ambos lados de la ecuación tenemos: $a\ast (a\ast a')=(a\ast a')\ast a''$, pudiendo reordenar los paréntesis por la 
propiedad asociativa. Así pues, obtenemos $a'=a''$.
\end{expla}

\begin{defn}[Grupo]
Dado $G\neq \emptyset$  y $\appl{\varphi}{G\ast G}{G}$, diremos que $G$ es un grupo si cumple las siguientes propiedades:
\begin{enumerate}
\item \textbf{Cerrado por la operación}. La combinación por $\ast$  de dos elementos de G da otro elemento del mismo. 
\item $\ast$ es asociativa.
\item \textbf{Existencia del elemento neutro}. $\exists  e \in G \tq \; \forall\, x\in G\; x\ast e=x$.
\item \textbf{Existencia del inverso}. $\forall x \in G \exists x' \in G \tq x\ast x'=x'\ast x=e$.
\end{enumerate}
\end{defn}

\begin{example}
  $ (\ent, +); (\real, +); (\real, \cdot); (\real/x>0, \cdot)$ son grupos, mientras que 
  $(\ent, \cdot); (\real, \cdot)$ no lo son.
  
  Otro ejemplo: A partir de un conjunto $A$ definimos $B(A)$ como el conjunto de todas las biyecciones de A en sí mismo.  Puesto que la composición de dos biyecciones es otra biyección, la composición es una operación definida sobre B(A). Si tomamos como neutro la biyección identidad y como inverso, la función inversa, podemos comprobar que $B(A)$ es un grupo respecto a la composición.
\end{example}

\begin{theorem}[Unicidad\IS del neutro e inverso]
  En todo grupo se cumplen las propiedades de unicidad del elemento neutro y del inverso.
\end{theorem}


Vamos a definir ahora las transformaciones lineales rígidas, que nos resultarán muy interesantes para definir algunos grupos.

\begin{problem}






\solution
\easyimgw{Patata.jpg}{Leyenda}{lblEtiqueta}{0.3}
ver figura \ref{lblEtiqueta}

\end{problem}



\newpage
\printindex
\end{document}
