\section{Computación}

\subsection{Motivación}

\paragraph{La ecuación diofántica:} $x^2+y^2=z^2$ tiene soluciones no triviales en enteros, por ejemplo, $x=3, y=4, z=5$. Consideramos el caso $n>2$, luego tenemos $x^n+y^n = z^n$, que tiene soluciones triviales ($x=y=z=0$). y no tiene soluciones no triviales en enteros (Wiles, 1995).

\paragraph{X problema de Hilbert:} Hallar un algoritmo tal que, dada cualquier ecuación diofántica, nos dice si tiene soluciones en enteros o no. Es decir, dado cualquier polinomio con coeficientes en $\mathbb{Z}$, dependiente de cualquier número de variables, $x_1,\hdots, x_m$, el algoritmo decide si $p(x_1, \hdots, x_m)=0$ tiene soluciones en enteros o no.
\begin{obs}
	Se pide un sólo algoritmo que debe funcionar para todas las ecuaciones diofánticas.
\end{obs}
En 1970, Yuri Matiyasevich, basándose en trabajos previos de Martin Davies, Hilary Putnam y Julia Robinson, demostró que no existe tal algoritmo.
