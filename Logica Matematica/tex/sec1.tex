\section{Lógica proposicional}
Sea $A$ un conjunto no vacío. 
Definimos el lenguage como
$$L \equiv A\cup\{\neg,\implies, \wedge, \vee, \iff, \perp, \top, ), ( \}$$
donde los símbolos representan
\begin{itemize}
	\item $\neg:$ negación.
	\item $\implies:$ implicación.
	\item $\wedge:$ conjunción.
	\item $\vee:$ disyunción.
	\item $\iff:$ equivalencia.
	\item $\perp:$ verdadero.
	\item $\top:$ falso.
	\item $):$ paréntesis derecho.
	\item $(:$ paréntesis izquierdo.
\end{itemize}

\begin{mdframed}
\noindent\textbf{Nota:}\\
$p\implies q$ es una abreviación de $\neg p \vee q$.
\end{mdframed}

\subsection{Sintaxis}
Dado $A$, un conjunto (finito, numerable o no numerable), de variables proposicionales $p, q, r, \hdots$, se definen:

\begin{defn}[Palabra]
	Concatenación de símbolos.
\end{defn}

\begin{defn}[Proposición o fórmula bien formada]
	Palabra que pertenece a la clase más pequeña cerrada bajo las siguientes propiedades:
	\begin{itemize}
		\item Los elementos de $A$: $\perp$ y $\top$ son fórmulas atómicas (no se pueden dividir).
		\item Si $F_1$ y $F_2$ son fórmulas, también lo son
		\begin{itemize}
			\item $(\neg F_1)$
			\item $(F_1\implies F_2)$
			\item $(F_1\iff F_2)$
			\item $(F_1\wedge F_2)$
			\item $(F_1\vee F_2)$
		\end{itemize}
	\end{itemize}
	En notación polaca son fórmulas bien formadas:
	\begin{itemize}
		\item $\neg$ $F_1$
		\item $\implies F_1$ $F_2$
		\item $\iff$ $F_1$ $F_2$
		\item $\wedge$ $F_1$ $F_2$
		\item $\vee$ $F_1$ $F_2$
	\end{itemize}
\end{defn}

\noindent Las fórmulas bien formadas pueden definirse de manera inductiva:
\begin{equation*}
\begin{array}{l l l}
	\textbf{FBF}_\textbf{0} &=& A\cup\{\top, \perp \}\\
	\textbf{FBF}_\textbf{n+1} &=& FBF_n \cup \{(\neg F_1), (F_1\implies F_2),(F_1\iff F_2),(F_1\wedge F_2),(F_1\vee F_2)\}\\ &&\text{con } F_1,F_2\in \textbf{FBF}_\textbf{n}.
\end{array}
\end{equation*}
Las fórmulas bien formadas son $FBF=\cap_{n=0}^\infty FBF_n$

\begin{defn}[Aridad]
	La aridad de un operador es el número de argumentos que admite:
	\begin{itemize}
		\item $\perp$, $\top$ son $0$-arias.
		\item $\neg$ es $1$-aria.
		\item $\wedge$, $\vee$, $\implies$, $\iff$ son $2$-arias (binarias).
	\end{itemize}
\end{defn}

\begin{lemma}{\textbf{Legibilidad única:}}
	Las fórmulas no son ambiguas, sólo pueden leerse o construirse de forma única.
\end{lemma}
Hay que distinguir los siguientes casos:
\begin{itemize}
	\item $F$ es atómica. La legibilidad única es obvia.
	\item $F$ es es de la forma $\neg G$, donde $G$ es una FBF. La legibilidad única es obvia.
	\item $F$ es de la forma $(G_1\ast G_2)$ donde $G_1$ y $G_2$ son FBF y $\ast$ es una conectiva binaria. En este caso habría que ver si existen las FBF $G_3$ y $G_4$ y la conectiva binaria $\ast'$, tal que $(G_1\ast G_2) = (G_3\ast ' G_4)$. La demostración de que esto sólo es posible si $G_1=G_2$, $G_3=G_4$ y $\ast = \ast'$, será un ejercicio de las hojas de problemas. La idea es que $G_1$ no puede ser un segmento inicial de $G_3$.
\end{itemize}

\begin{obs}
	$a\wedge b \wedge c$ no es una fórmula bien formada. Si son fórmulas bien formadas las siguientes. $(a\wedge (b\wedge c))$, $((a\wedge b) \wedge c)$. Pero como fórmulas son distintas, aunque su significado sea el mismo.
\end{obs}

\begin{mdframed}
	\textbf{Notación:} 
	\begin{itemize}
	\vspace{-3mm}
	\item Usamos $\equiv$ para denotar equivalencia en el metalenguaje (en el lenguaje se usa $\iff$).
	
	\item Usamos $p,q,r,s,\hdots$ para denotar fórmulas bien formadas.
	\end{itemize}
\end{mdframed}




\subsection{Axiomas de la lógica proposicional}
A continuación se proporciona una lista de axiomas proposicionales.
\begin{enumerate}
	\item $\top$
	\item $p\implies (p\vee q)$, $p\implies (q\vee p)$
	\item $\neg p \implies (\neg q \implies \neg (p\vee q))$
	\item $(p\wedge q)\implies p$, $(p\wedge q)\implies q$
	\item $p\implies (q\implies (p\wedge q))$
	\item $(p\implies(q\implies r))\implies ((p\implies q)\implies(p\implies r))$
	\item $p\implies(\neg p\implies \perp)$
	\item $(\neg p\implies \perp)\implies p$
\end{enumerate}

Los axiomas 2-8 son esquemas de axiomas. Por ejemplo, si $p\neq r$, $p\implies (p\vee q)$ y $r\implies (r\vee q)$ son axiomas distintos.

La regla de deducción que vamos a utilizar es el \textbf{modus ponens}: Si tenemos $p$ y $p\implies q$, deducimos $q$.

\begin{defn}[Demostración o prueba]
	Una demostración o prueba (formal) es una sucesión (finita) $p_1, p_2, \hdots, p_n$ donde cada $p_i$ o bien es un axioma o se obtiene de dos fórmulas anteriores mediante modus ponens.
\end{defn}

\begin{mdframed}
\textbf{Notación}
\begin{itemize}
	\vspace{-3mm}
	\item Escribimos $\vdash p$ para denotar que existe una demostración de $p$.
	\item Escribimos $\vDash p$ para denotar que $p$ es verdad.
\end{itemize}
\end{mdframed}

\begin{theorem}[Teorema de completitud]
$$\vdash p \iff \vDash p$$
\end{theorem}

\begin{defn}[Teoría]
	$T$ (o $\Sigma$) es una teoría si es un conjunto de FBF$(L)$.
\end{defn}
\begin{example}
	$T$ nos dice que un grafo $G$ puede colorearse con cuatro colores. (Ver ejercicio 4 de la hoja 2).
\end{example}

\noindent \textbf{Recordatorio}: $(G,E)$ es un grafo si 
\begin{equation*}
	\left\{
	\begin{array}{l l}
	G=\{v_\alpha \tq \alpha \in \Lambda \} & \\
	E = \left\{\{v_\alpha, v_\beta \}\tq v_\alpha, v_\beta\in G \right\} &  v_\alpha \neq v_\beta
	\end{array}
	\right.
\end{equation*}

Dado que son pares de elementos distintos, no puede haber una arista que empiece y acabe en el mismo vértice, puesto que en dicho caso tendríamos un conjunto de un sólo elemento.

\begin{defn}[Demostración o prueba (II)]
	Una demostración a partir de los axiomas y una teoría $T$ es una sucesión finita $p_1, \hdots, p_n$, donde cada $p_i$ es o un axioma, o $p_i\in T$ o se obtiene de dos $p_j,p_k$ anteriores mediante modus ponens.
\end{defn}

\begin{mdframed}
\textbf{Notación: } Usamos $T\vdash p$ para denotar que existe una prueba de $p$ usando los axiomas y las fórmulas de $T$.
\end{mdframed}

\subsection{Semántica}
La semántica se refiere a las interpretaciones que se le dan a las FBF. En el lenguaje formal interpretamos $\top$ como ``verdadero'' y $\perp$ como ``falso''. Informalmente utilizaremos $0$ para ``falso'' y $1$ para ``verdadero''.

Vamos a considerar el siguiente lenguaje: tomamos $$L= A\cup \left\{\neg, \wedge, \vee, \top, \perp, ), ( \right\}$$

\begin{defn}[Interpretación]
Una interpretación es una valoración booleana $\sigma$: $$\sigma: FBF(L)\to (\mathbb{Z}_2, +, \cdot) $$
que satisface
\begin{itemize}
	\item $\sigma(\top) = 1$, $\sigma(\perp) = 0$
	\item $\sigma(\neg p) = 1+\sigma(p)$,  siendo $p$ una FBF.
	\item $\sigma (p\vee q) = \sigma(p) \vee \sigma (q) = \max\left\{\sigma(p), \sigma(q)\right\}$
	\item $\sigma (p\wedge q) = \sigma(p) \wedge \sigma (q) = \min\left\{\sigma(p), \sigma(q)\right\}$
\end{itemize}
\end{defn}

\begin{theorem}
	Sea $\sigma: A \to \{0, 1\}$ una función arbitraria. Entonces $\sigma$ se extiende, de modo único, a una valoración booleana $$\hat{\sigma}:FBF(L) \to \mathbb{Z}_2$$
\end{theorem}
\begin{proof}
	Por el lema de legibilidad única.
\end{proof}

A partir de los valores $\sigma(p), \sigma(q),\hdots$ se define $\sigma$ para las $FBF(L)$ tal y como describen las tablas de verdad.
\begin{center}
\begin{tabular}{c|c}
	& $\top$\\
	\hline
	& 1 \\
\end{tabular}\hspace{1cm}
\begin{tabular}{c|c}
	& $\perp$\\
	\hline
	& 0 \\
\end{tabular}\hspace{1cm}
\begin{tabular}{|c|c|}
	\hline
	$p$ & $\neg p$\\
	\hline
	1 & 0 \\
	\hline
	0 & 1\\
	\hline
\end{tabular}

\begin{tabular}{|c|c|c|c|c|c|}
	\hline
	$p$ & $q$ & $p\vee q$ & $p \wedge q$ & $p\implies q$ &  $p\iff q$\\
	\hline
	1 & 1 & 1 & 1 & 1 & 1 \\
	\hline
	1 & 0 & 1 & 0 & 0 & 0 \\
	\hline
	0 & 1 & 1 & 0 & 1 & 0 \\
	\hline
	0 & 0 & 0 & 0 & 1 & 1 \\
	\hline
\end{tabular}
\end{center}

\begin{obs}
	\begin{itemize}
	\item $\top \equiv (p\vee \neg p)$
	\item $\perp \equiv \neg \top $
	\item $(p\implies q) \equiv (\neg p \vee q)$
	\item $(p\iff q) \equiv (p\implies q)\wedge (q \implies p)$
	\end{itemize}
\end{obs}

\begin{defn}[Sistema completo]
	Un sistema de conectivas es completo si las demás conectivas pueden definirse en términos de ellas.
\end{defn}

\begin{example}
	Los siguientes conjuntos son completos.
	$$\{\neg, \vee\}, \{\neg, \wedge\}, \{\neg, \implies\}, \{\neg, \iff\}, \{\perp, \implies\}$$
	
	$\{\vee,\wedge\}$ no es completo. Se puede probar por monotonía.
\end{example}

\begin{defn}[Modelo]
	Sea $T\subset FBF(L)$. Un modelo de $T$ es una valoración booleana $\sigma \tq \forall p\in T, \sigma (p) = 1$.
\end{defn}

\begin{defn}[Consecuencia tautológica]
	Decimos que $p$ es una consecuencia tautológica de $T$, y escribimos $T\vDash p$, si para todo modelo $\sigma$ de $T$, $\sigma(p) = 1$, decimos que $p$ es una \textbf{tautología} y escribimos  $\vDash p$.
	Si $p\in T = \emptyset$, entonces toda $\sigma$ es un modelo de $T$.
\end{defn}

\begin{defn}
	Si $\vDash p \iff q$, es decir, si $p\iff q$ es una tautología, decimos que $p$ y $q$ son lógicamente equivalentes.
\end{defn}

\begin{obs}
	Sea $r$ una $FBF(L)$, sea $p$ una subfórmula de $r$ y se $\sigma$ cualquier valoración booleana. Si $p\iff q$ y reemplazamos cada instancia de $p$ en $r$ por $q$ denotando por $r^\prime$ la fórmula así obtenida, entonces $\sigma(r^\prime) = \sigma(r)$.
\end{obs}

\begin{defn}[Teoría inconsistente]
	Una teoría $T$ es inconsistente si $T\vdash \perp$.
\end{defn}

\begin{theorem}[Teorema de completitud]
	$T$ demuestra $p$ si y sólo si $p$ es consecuencia lógica de $T$.
	$$T\vdash p \iff T\vDash p$$
\end{theorem}

\begin{obs}
	Si $T$ es inconsistente, $T\vdash$ cualquier cosa.
	Por otra parte, ninguna valoración satisface $T$, es decir, ninguna $\sigma$ cumple $\sigma(r) = 1$ $\forall r\in T$. Luego cualquier cosa es una consecuencia lógica de $T$ porque no hay ninguna valoración que satisfaga $T$.
\end{obs}