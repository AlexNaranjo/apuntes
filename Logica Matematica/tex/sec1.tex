% -*- root: ../LogicaMatematica.tex -*-
\section{Lógica proposicional}
Sea $A$ un conjunto no vacío, definimos el lenguaje como
$$L \equiv A\cup\{\neg,\implies, \wedge, \vee, \iff, \perp, \top, ), ( \}$$
donde los símbolos representan
\begin{itemize}
	\item $\neg:$ negación.
	\item $\implies:$ implicación.
	\item $\wedge:$ conjunción.
	\item $\vee:$ disyunción.
	\item $\iff:$ equivalencia.
	\item $\perp:$ verdadero.
	\item $\top:$ falso.
	\item $):$ paréntesis derecho.
	\item $(:$ paréntesis izquierdo.
\end{itemize}

\subsection{Sintaxis}
Dado $A$, un conjunto (finito, numerable o no numerable) de variables proposicionales $p, q, r, \hdots$, se definen:

\begin{defn}[Palabra]
	Concatenación de símbolos.
\end{defn}

\begin{defn}[Proposición o fórmula bien formada]
	Palabra que pertenece a la clase más pequeña cerrada bajo las siguientes propiedades:
	\begin{itemize}
		\item Los elementos de $A$, $\perp$ y $\top$ son fórmulas atómicas (no se pueden dividir).
		\item Si $F_1$ y $F_2$ son fórmulas, también lo son
		\begin{itemize}
			\item $(\neg F_1)$
			\item $(F_1\implies F_2)$
			\item $(F_1\iff F_2)$
			\item $(F_1\wedge F_2)$
			\item $(F_1\vee F_2)$
		\end{itemize}
		\newpage
		\item En notación polaca son fórmulas bien formadas:
		\begin{itemize}
			\item $\neg$ $F_1$
			\item $\implies F_1$ $F_2$
			\item $\iff$ $F_1$ $F_2$
			\item $\wedge$ $F_1$ $F_2$
			\item $\vee$ $F_1$ $F_2$
		\end{itemize}
	\end{itemize}
\end{defn}

\noindent Las fórmulas bien formadas pueden definirse de manera inductiva:
\begin{equation*}
\begin{array}{l l l}
	\textbf{FBF}_\textbf{0} &=& A\cup\{\top, \perp \}\\
	\textbf{FBF}_\textbf{n+1} &=& FBF_n \cup \{(\neg F_1), (F_1\implies F_2),(F_1\iff F_2),(F_1\wedge F_2),(F_1\vee F_2)\}\\ &&\text{con } F_1,F_2\in \textbf{FBF}_\textbf{n}.
\end{array}
\end{equation*}
Las fórmulas bien formadas son $FBF=\cap_{n=0}^\infty FBF_n$

\begin{defn}[Aridad]
	La aridad de un operador es el número de argumentos que admite:
	\begin{itemize}
		\item $\perp$, $\top$ son $0$-arias.
		\item $\neg$ es $1$-aria.
		\item $\wedge$, $\vee$, $\implies$, $\iff$ son $2$-arias (binarias).
	\end{itemize}
\end{defn}

\begin{lemma}{\textbf{Legibilidad única:}}
	Las fórmulas no son ambiguas, sólo pueden leerse o construirse de forma única.
\end{lemma}
Hay que distinguir los siguientes casos:
\begin{itemize}
	\item $F$ es atómica. La legibilidad única es obvia.
	\item $F$ es es de la forma $\neg G$, donde $G$ es una FBF. La legibilidad única es obvia.
	\item $F$ es de la forma $(G_1\ast G_2)$ donde $G_1$ y $G_2$ son FBF y $\ast$ es una conectiva binaria. En este caso habría que ver si existen las FBF $G_3$ y $G_4$ y la conectiva binaria $\ast'$, tal que $(G_1\ast G_2) = (G_3\ast ' G_4)$. La demostración de que esto sólo es posible si $G_1=G_2$, $G_3=G_4$ y $\ast = \ast'$ será un ejercicio de las hojas de problemas. La idea es que $G_1$ no puede ser un segmento inicial de $G_3$.
\end{itemize}

\begin{obs}
	$a\wedge b \wedge c$ no es una fórmula bien formada, ya que no tiene paréntesis exteriores. Sí son fórmulas bien formadas las siguientes. $(a\wedge (b\wedge c))$, $((a\wedge b) \wedge c)$. Pero como fórmulas (sintácticamente) son distintas, aunque su significado sea el mismo.
\end{obs}

\newpage
\begin{mdframed}
	\textbf{Notación:}
	\begin{itemize}
	\vspace{-3mm}
	\item Usamos $\equiv$ para denotar equivalencia en el metalenguaje (en el lenguaje se usa $\iff$).

	\item Usamos $p,q,r,s,\hdots$ para denotar fórmulas bien formadas.
	\end{itemize}
\end{mdframed}

\subsection{Axiomas de la lógica proposicional}
A continuación se proporciona una lista de axiomas proposicionales.
\begin{enumerate}
	\item $\top$
	\item $p\implies (p\vee q)$, $p\implies (q\vee p)$
	\item $\neg p \implies (\neg q \implies \neg (p\vee q))$
	\item $(p\wedge q)\implies p$, $(p\wedge q)\implies q$
	\item $p\implies (q\implies (p\wedge q))$
	\item $(p\implies(q\implies r))\implies ((p\implies q)\implies(p\implies r))$
	\item $p\implies(\neg p\implies \perp)$
	\item $(\neg p\implies \perp)\implies p$
\end{enumerate}

Los axiomas 2-8 son esquemas de axiomas, lo que quiere decir que si $p\neq r$, entonces $p\implies (p\vee q)$ y $r\implies (r\vee q)$ son axiomas distintos.

\paragraph*{Modus ponens}\mbox{}

La regla de deducción que vamos a utilizar es el \textbf{modus ponens}: Si tenemos $p$ y $p\implies q$, deducimos $q$.

\begin{center}
	\begin{tabular}{r}
		$p$\\
		$p\implies q$\\
		\hline
		$q$
	\end{tabular}
\end{center}

\begin{defn}[Demostración o prueba]
	Una demostración o prueba (formal) es una sucesión (finita) $p_1, p_2, \hdots, p_n$ donde cada $p_i$ o bien es un axioma o se obtiene de dos fórmulas anteriores mediante modus ponens.
\end{defn}

\begin{mdframed}
\textbf{Notación}
\begin{itemize}
	\vspace{-3mm}
	\item Escribimos $\vdash p$ para denotar que existe una demostración de $p$.
	\item Escribimos $\vDash p$ para denotar que $p$ es verdad.
\end{itemize}
\end{mdframed}

\begin{defn}[Teoría]
	$T$ (o $\Sigma$) es una teoría si es un conjunto de FBF$(L)$.
\end{defn}
\begin{example}
	$T$ nos dice que un grafo $G$ puede colorearse con cuatro colores. (Ver ejercicio 4 de la hoja 2).
\end{example}

\noindent \textbf{Recordatorio}: $(G,E)$ es un grafo si
\begin{equation*}
	\left\{
	\begin{array}{l l}
	G=\{v_\alpha \tq \alpha \in \Lambda \} & \\
	E = \left\{\{v_\alpha, v_\beta \}\tq v_\alpha, v_\beta\in G \right\} &  v_\alpha \neq v_\beta
	\end{array}
	\right.
\end{equation*}

Dado que son pares de elementos distintos, no puede haber una arista que empiece y acabe en el mismo vértice, puesto que en dicho caso tendríamos un conjunto de un sólo elemento.

\begin{defn}[Demostración o prueba (II)]
	Una demostración a partir de los axiomas y una teoría $T$ es una sucesión finita $p_1, \hdots, p_n$, donde cada $p_i$ es o un axioma, o $p_i\in T$ o se obtiene de dos $p_j,p_k$ anteriores mediante modus ponens.
\end{defn}

\begin{mdframed}
\textbf{Notación: } Usamos $T\vdash p$ para denotar que existe una prueba de $p$ usando los axiomas y las fórmulas de $T$.
\end{mdframed}

\subsection{Semántica}
La semántica se refiere a las interpretaciones que se le dan a las FBF. En el lenguaje formal interpretamos $\top$ como ``verdadero'' y $\perp$ como ``falso''. Informalmente utilizaremos $0$ para ``falso'' y $1$ para ``verdadero''.

Vamos a considerar el siguiente lenguaje: tomamos $$L= A\cup \left\{\neg, \wedge, \vee, \top, \perp, ), ( \right\}$$

\begin{defn}[Interpretación]
Una interpretación es una valoración booleana $\sigma$: $$\sigma: FBF(L)\to (\mathbb{Z}_2, +, \cdot) $$
que satisface
\begin{itemize}
	\item $\sigma(\top) = 1$, $\sigma(\perp) = 0$
	\item $\sigma(\neg p) = 1+\sigma(p)$,  siendo $p$ una FBF.
	\item $\sigma (p\vee q) = \sigma(p) \vee \sigma (q) = \max\left\{\sigma(p), \sigma(q)\right\}$
	\item $\sigma (p\wedge q) = \sigma(p) \wedge \sigma (q) = \min\left\{\sigma(p), \sigma(q)\right\}$
\end{itemize}
\end{defn}

\begin{theorem}
	Sea $\sigma: A \to \{0, 1\}$ una función arbitraria. Entonces $\sigma$ se extiende, de modo único, a una valoración booleana $$\hat{\sigma}:FBF(L) \to \mathbb{Z}_2$$
\end{theorem}
\begin{proof}
	Por el lema de legibilidad única.
\end{proof}

A partir de los valores $\sigma(p), \sigma(q),\hdots$ se define $\sigma$ para las $FBF(L)$ tal y como describen las siguientes tablas de verdad:
\begin{center}
\begin{tabular}{c|c}
	& $\top$\\
	\hline
	& 1 \\
\end{tabular}\hspace{1cm}
\begin{tabular}{c|c}
	& $\perp$\\
	\hline
	& 0 \\
\end{tabular}\hspace{1cm}
\begin{tabular}{|c|c|}
	\hline
	$p$ & $\neg p$\\
	\hline
	1 & 0 \\
	\hline
	0 & 1\\
	\hline
\end{tabular}

\begin{tabular}{|c|c|c|c|c|c|}
	\hline
	$p$ & $q$ & $p\vee q$ & $p \wedge q$ & $p\implies q$ &  $p\iff q$\\
	\hline
	1 & 1 & 1 & 1 & 1 & 1 \\
	\hline
	1 & 0 & 1 & 0 & 0 & 0 \\
	\hline
	0 & 1 & 1 & 0 & 1 & 0 \\
	\hline
	0 & 0 & 0 & 0 & 1 & 1 \\
	\hline
\end{tabular}
\end{center}

\begin{obs}
	\begin{itemize}
	\item $\top \equiv (p\vee \neg p)$
	\item $\perp \equiv \neg \top $
	\item $(p\implies q) \equiv (\neg p \vee q)$
	\item $(p\iff q) \equiv (p\implies q)\wedge (q \implies p)$
	\end{itemize}
\end{obs}

\begin{defn}[Sistema completo]
	Un sistema de conectivas es completo si las demás conectivas pueden definirse en términos de ellas.
\end{defn}

\begin{example}
	Los siguientes conjuntos son completos.
	$$\{\neg, \vee\}, \{\neg, \wedge\}, \{\neg, \implies\}, \{\perp, \implies\}$$

	$\{\vee,\wedge\}$ no es completo. Se puede probar por monotonía.

	$\{\neg, \iff\}$ está propuesto como ejercicio (\ref{ejer::H3P10}) demostrar si es completo o no.
\end{example}

\begin{defn}[Modelo]
	Sea $T\subset FBF(L)$. Un modelo de $T$ es una valoración booleana $\sigma \tq \forall p\in T, \sigma (p) = 1$.
\end{defn}

\begin{defn}[Consecuencia\IS tautológica]
	Decimos que $p$ es una consecuencia tautológica de $T$, y escribimos $T\vDash p$, si para todo modelo $\sigma$ de $T$, $\sigma(p) = 1$. En este contexto, decimos que $p$ es una \textbf{tautología} y escribimos  $\vDash p$.
\end{defn}

\begin{obs}
	Si $T = \emptyset$, entonces toda $\sigma$ es un modelo de $T$.
\end{obs}

\newpage

\begin{defn}[Equivalencia lógica]
	Si $\vDash p \iff q$, es decir, si $p\iff q$ es una tautología, decimos que $p$ y $q$ son lógicamente equivalentes.
\end{defn}

\begin{obs}
	Sea $r$ una $FBF(L)$, sea $p$ una subfórmula de $r$ y sea $\sigma$ cualquier valoración booleana. Si $p\iff q$ y reemplazamos cada instancia de $p$ en $r$ por $q$ denotando por $r^\prime$ la fórmula así obtenida, entonces $\sigma(r^\prime) = \sigma(r)$.
\end{obs}

\begin{defn}[Teoría\IS consistente]
 $T$ es consistente si $T\nvdash \perp$. La consistencia o coherencia puede caracterizarse de forma sencilla:
 $$T\nvdash\perp \iff \exists p\in FBF(L) \tq  T\nvdash p$$
 Equivalentemente, una teoría $T$ es inconsistente si $T\vdash \perp$.
 $$T\vdash\perp \iff T\vdash p\ \forall p\in FBF(L)$$

\end{defn}

\begin{obs}
	Si $T$ es inconsistente, $T\vdash$ cualquier cosa.
	Por otra parte, ninguna valoración satisface $T$, es decir, ninguna $\sigma$ cumple $\sigma(r) = 1$ $\forall r\in T$. Luego cualquier cosa es una consecuencia lógica de $T$ porque no hay ninguna valoración que satisfaga $T$.
\end{obs}

\begin{theorem}[Teorema\IS de la deducción]
	Si $T\cup\{p\}\vdash q$, entonces $$T\vdash p\implies q$$
\end{theorem}
\begin{proof}
	\begin{itemize}
		\item \textbf{Caso 1}: $q\notin T$, $q\neq p$, $q$ no es un axioma lógico.

		Consideramos $p_1,\hdots, p_n=q$ una demostración de $q$. Llevaremos a cabo la demostración por inducción en la longitud de las pruebas (fórmulas).

		\begin{itemize}
			\item Si $n=1$ entonces $q=p$ o $q\in T$ o $q$ es un axioma lógico, lo cual no corresponde a este caso, por tanto suponemos que $n>1$ y que el Teorema de la deducción es cierto para todas las pruebas de longitud menor que $n$.
			\item Si para alguna $j<n$, $p_j = p\implies q$, no hay nada que probar.
			\item Si no, $\exists i,j<n \tq p_i=r$, $p_j=r\implies q$. Por la hipótesis de inducción tenemos $T\vdash (p\implies r)$ y $T\vdash (p\implies (r\implies q))$.
			Utilizamos el axioma
			$$(p\implies (r\implies q)) \implies ((p\implies r)\implies (p\implies q))$$
			y por modus ponens junto con lo anterior obtenemos
			$$(p\implies r)\implies (p\implies q)$$
			Finalmente, por modus ponens aplicado a $T\vdash (p\implies r)$ y a lo anterior obtenemos $T\vdash (p\implies q)$.
		\end{itemize}
	\item \textbf{Caso 2}: $q$ es un axioma lógico, $q\in T$ o $q=p$.

	\begin{itemize}
		\item Si $q$ es un axioma o $q\in T$

		Tenemos que $T\vdash q$ y utilizando el segundo axioma de la lista de axiomas:
		$$r\implies (q \vee r) \equiv r\implies (\neg q \implies r)$$ en el que usamos $q$ en lugar de $r$ y $p$ en lugar de $\neg q$, instanciamos
		$$q\implies(p\implies q)$$

		Usando modus ponens y lo anterior
		$T\vdash (p\implies q)$.

		\item Si $q$ no es axioma ni está en $T$, entonces $q = p$.
		Tenemos $T\cup\{p\}\vdash p$.
		Y utilizando (del ejercicio 1a de la hoja dos) $\vdash p\implies p$.
		entonces $$T\vdash p\implies p \equiv p\implies q$$
	\end{itemize}
	\end{itemize}
\end{proof}

\subsection{Problemas TAUT y SAT}
\begin{mdframed}
	\textbf{Problema de las tautologías (TAUT)}: dada $F\in FBF(L)$ determinar si $F$ es una tautología. Es decir, determinar si para toda interpretación $\sigma$, $\sigma (F)=1$.
\end{mdframed}

El problema es saber cuánto tiempo se tarda en averiguar si $F$ es una tautología en función del tamaño de los datos de los que se dispone.
\begin{example}
	Dada la siguiente FBF y la tabla de verdad correspondiente:
	$$p\implies(q\implies p)$$
	\begin{center}
		\begin{tabular}{|c|c|c|c|}
			\hline
			$p$ & $q$ & $q\implies p$ & $p\implies (q\implies p)$\\
			\hline
			1 & 1 & 1 & 1 \\
			\hline
			1 & 0 & 1 & 1 \\
			\hline
			0 & 1 & 0 & 1 \\
			\hline
			0 & 0 & 1 & 1 \\
			\hline
		\end{tabular}
	\end{center}
	Tenemos que:
		\begin{itemize}
			\item Número $n$ de variables: $2$
			\item Número $k$ de conectivas: $2$
			\item Tiempo (tamaño de los datos) $< O((n+k)2^n)$
		\end{itemize}
\end{example}

\paragraph{Clases de complejidad}
\begin{itemize}
	\item \textbf{P:} clase de problemas para los que existe un algoritmo que los resuelve en tiempo polinómico en función del número de datos de entrada, es decir, $\exists K\ge 0$ (dependiendo el problema pero no del número de datos $n$) tal que si tenemos $n$ datos de entrada, el algoritmo requiere menos de $O(n^K)$ operaciones.
	\item \textbf{NP:} tiempo polinómico no determinista. La solución la proporciona un oráculo, que podemos comprobar que es correcta (o no) en tiempo polinómico. Obviamente P$\subset$NP.
		\item La clase de problemas \textbf{NP-difícil} se define como aquella que contiene a los problemas que son como mínimo tan difíciles como un problema de NP. Alternativamente, se define como la clase de problemas H tal que todo problema en NP puede ser transformado polinomialmente en H.
	\item La clase de problemas \textbf{NP-completo} se define como la intersección entre la clase de problemas NP y la clase de problemas NP-difícil.
	\item No se sabe si P$=$NP. Clay Mathematics Institute ha ofrecido un premio de un millón de dólares estadounidenses para quién desarrolle la primera demostración correcta (el profesor no tiene tanto dinero, el te ofrece una matrícula).
\end{itemize}
Se sabe que \textbf{TAUT} es NP-difícil.

\begin{defn}[Satisfación]
	$\sigma$ satisface a $T$ si $\forall p \in T$, $\sigma(p) = 1$, o equivalentemente, si $\sigma$ es un modelo de $T$.
\end{defn}

\begin{mdframed}
	\textbf{Problema de la satisfación (SAT)}: Si $F\in FBF(L)$, determinar si existe una interpretación $\sigma$ que satisfaga a $F$. \textbf{SAT} es {NP-completo}.
\end{mdframed}

\begin{example}
	Comprobar si $p\implies (p\implies q)$.

	El oráculo dice que
	\vspace{-3mm}
	\begin{itemize}
		\item $\sigma(p) = 0$
		\item $\sigma(q) = 0$
	\end{itemize}
	\begin{center}
		\begin{tabular}{|c|c|c|c|}
			\hline
			$p$ & $q$ & $p\implies$ & $p\implies(p\implies q)$\\
			\hline
			0 & 0 & 1 & 1\\
			\hline
		\end{tabular}
	\end{center}
	\begin{itemize}
		\item \textbf{Comprobación:} tiempo $O(n+k)$.
	\end{itemize}
\end{example}
\subsection{Completitud}
\textbf{Recordatorio:} Podemos reemplazar $p$ por cualquier fórmula equivalente. Por ejemplo, podemos remplazar $p\implies q$ por $\neg p \Or q$, $(p\y q)\y r$ por $p\y (q\y r)$, cualquier contradicción por $\perp$, etc.

\begin{theorem}
	Si existe una prueba de $p$ a partir de los axiomas y modus ponens, entonces, $p$ es una tautología.
	$$\vdash p \implies \vDash p$$
\end{theorem}
\begin{proof}
	Por inducción en la longitud de la demostración.

	Sean $p_1,\hdots, p_n=p$.
	\begin{itemize}
		\item Caso $n=1$: $p_1$ = $p$ es un axioma, luego es una tautología.
		\item Caso $n>1$: $p=p_n$, y para toda $j\subset n$, $p_j$ es una tautología.
		\begin{itemize}
			\item Si $p$ es una axioma, entonces $p$ es una tautología.
			\item Si $p$ no es un axioma, entonces se obtiene de dos fórmulas anteriores mediante modus ponens, digamos $p_i = r$ y $p_k=r\implies p$ con $i,k<n$.

			Sea $\sigma$ una valoración booleana. Hay que probar que $\sigma(p) = 1$. Por la hipótesis inicial, $\sigma(r) = 1 = \sigma(r\implies p) = \sigma(\neg r \Or p) = \sigma(\neg r) \Or \sigma(p) = \max(0,\sigma(p))$.
			Por tanto $\sigma(p) = 1$.
		\end{itemize}
	\end{itemize}
\end{proof}

\begin{theorem}[Teorema\IS de completitud]	Sea $T = \emptyset$.
	\begin{itemize}
		\item \textbf{Validez}:
		$T \vdash p \implies T \vDash p$
		\item \textbf{Adecuación:}
		$T \vDash p \implies T \vdash p$
	\end{itemize}
	En general $$T\vdash p \iff T\vDash p$$
\end{theorem}
\begin{proof}

	\textbf{Validez:} Sea $V$ el conjunto de todas las valoraciones que satisfacen a $T$. Si $V=\emptyset$, entonces $T\vdash \sigma$ trivialmente. Si $V\neq \emptyset$, entonces, argumentamos como en el teorema anterior pero considerando sólo las valoraciones en $V$ en lugar de todas.

	\textbf{Adecuación:} Sea $p$ una tautología sin demostración. Como $\nvdash p$ entonces $T\cup\{\neg p\}$ es consistente (ver teorema \ref{thm:pconsnegp}) lo que nos dice que, además, $T \vdash \neq p$. Pero, puesto que $p$ es una tautología, $\neg p = \perp$ con lo que nos queda $T\vdash \neg p\equiv \perp$, que es una contradicción pues esta suponiendo que la teoría es consistente.

	\textit{Otra forma:} Probamos la formulación equivalente ``$T\nvdash p \implies T\nvDash p$'', es decir, hay un modelo $\sigma$ de $T$ tal que $\sigma(p) = 0$. $T\nvdash p \implies T\cup\{\neg p\}$ es consistente. Por \ref{thm:comp2}, $T\cup\{\neg p\}$ tiene un modelo $\sigma$. Por tanto, $\forall q\in T$, $\sigma(q) = 1$, pero $\sigma(p) = 0$ porque $\sigma(\neg p) = 1$.

\end{proof}

\noindent\textbf{Criterio: }\mbox{}

	\framebox{$T\text{ es consistente } \iff \exists p \tq T\nvdash p$} $\equiv$
	\framebox{$T\text{ es inconsistente } \iff \forall p, T\vdash p$}


\begin{theorem}
	Si $T$ es consistente, entonces, para toda $p$, o $T\vdash p$, o $T\vdash \neg p$, o no prueba ninguna, pero no puede probar ambas.
\end{theorem}
\begin{proof}\mbox{}

	Probamos la formulación equivalente: ``si $T\vdash p$ y $T\vdash \neg p$, entonces $T$ es inconsistente''.

	\textbf{Usando el axioma 7}:
	Por un lado con $p\implies(\neg p \implies \perp)$, $T\vdash p$, y modus ponens, obtenemos $T\vdash \neg p \implies \perp$. Por otro lado, usando $T\vdash \neg p$ y modus ponens, obtenemos $T\vdash\perp$.
\end{proof}

\begin{theorem}
	Si $T \nvdash p$, entonces $T\cup \{\neg p\}$ es consistente. Por tanto $T$ es consistente.
	\label{thm:pconsnegp}
\end{theorem}
\begin{proof}

	Vamos a reescribir lo que queremos probar como:
	\begin{center}``Si $T\cup \{\neg p\}$ es inconsistente, entonces $T\vdash p$''.\end{center}

	Si $T\cup \{\neg p\}$ es inconsistente podemos escribir
	\[T\cup \{\neg p\}\vdash \perp\]
	Por el teorema de la deducción
	\[T\vdash \neg p\implies \perp\]
	Por el axioma 8, que dice:
	\[(\neg p\implies \perp)\implies p\]
	tomando $(\neg p \implies \perp)\implies p$ y aplicando modus ponens, obtenemos $T\vdash p$.
\end{proof}

\begin{defn}[Teoría\IS completa]
	$T$ es completa si es consistente y $\forall p\in FBF(L)$, o $T\vdash p$ o $T\vdash \neg p$.
\end{defn}

\begin{theorem}[Teorema\IS de Lindenbaum]
	Toda teoría consistente puede extenderse a una teoría completa.
	\label{thm:lindenbaum}
\end{theorem}
\begin{proof}
	En esta demostración nos apoyaremos en el Lema de Zorn.

	Sea $T$ una teoría consistente y sea $P$ el conjunto de todas las teorías consistentes que extienden a $T$, definimos un orden parcial en $P$ mediante inclusión:
	\[T_1\le T_2 \iff T_1\subset T_2\]

	Sea $\{T_\alpha\}_{\alpha\in\Lambda}$ una cadena en $P$.

	Si $\cup_{\alpha\in\Lambda} T_\alpha$ es consistente, entonces es una cota superior de la cadena.

	Puesto que todas las $T_\alpha$ son consistentes, la unión será consistente.
	\begin{proof}
	Queremos probar que:
	\[\bigcup_α T_α \vdash \perp \implies \exists α \ T_α \vdash \perp \equiv \forall α \ T_α \nvdash \perp \implies \bigcup_αT_α \nvdash \perp\]

	Realizaremos la prueba por reducción al absurdo.

	Sea $p_1,\hdots,p_n = \perp$ una prueba formal de $\perp$. Entonces, $\exists T_{\alpha_1}\vdash p_1$, $\exists T_{\alpha_2} \supset T_{\alpha_1}$ tal que $T_{\alpha_2}\vdash p_2, \hdots, T_{\alpha_n}\supset \exists T_{\alpha_{n-1}}$ tal que $T_{\alpha_n}\vdash p_n$.

	\obs Al final del ``procedimiento'' que acabamos de definir llegamos a una teoría $T_n$ que contiene a todos los $p_i$ y que, por tanto $T_n \vdash \perp$. Pero estábamos considerando que todas las $T_i$ eran consistentes, por lo que acabamos de llegar a una contradicción.

	Queda claro pues que la unión de teorías consistentes es consistente.
	\end{proof}

	Por el lema de Zorn sabemos que $P=\bigcup_αT_α$ tiene un elemento maximal $M$, que contiene a $T$. Por tanto $M\in P$, lo que implica que $M$ es consistente.

	Además, por ser $M$ maximal tenemos que $\forall p \ M \vdash p \Or M \vdash \neg p$ puesto que de no ser así, tanto $M\cup\{p\}$ como $M\cup\{\neg p\}$ serían consistentes y extenderían $T$ por lo que pertenecerían a $P$. Puesto que $M$ es el elemento maximal de $P$ tenemos que $M\cup\{p\} \subset M \implies p \in M$ y $M\cup\{\neg p\}\subset M \implies \neg p \in M$ con lo que $M$ no sería consistente.

	Queda claro pues que $M$ es la extensión completa de la teoría que estábamos buscando.
\end{proof}

\begin{theorem}[Teorema\IS de completitud II]
	$T$ es consistente $\iff$ $T$ tiene un modelo.
	\label{thm:comp2}
\end{theorem}
\begin{proof}
	\begin{itemize}
		\item $\Longleftarrow$: Por validez, $T\vdash p \implies T\vDash p$. Si $T$ es inconsistente, $T\vdash \perp$, luego $T\vDash\perp$. Como para toda valoración booleana, $\sigma(\perp) = 0$, ninguna $\sigma$ satisface a $T$.
		\item $\Longrightarrow$: Sea $T$ consistente, queremos probar que tiene un modelo. Por el teorema de Lindenbaum (\ref{thm:lindenbaum}), existe una $M\supset T \tq M$ es completa, es decir, $M$ es consistente y $\forall p \in FBF(L)$, o $M\vdash p$ o $M\vdash \neg p$.

		Definimos $\sigma(p) = 1 \iff M\vdash p$, y $\sigma(p) = 0 \iff M\nvdash p$.

		$\sigma$ está definida sobre el conjunto $FBF(L)$, y $\sigma(p) = 1$ $\forall p \in M$, y por tanto $\forall p\in T$.

		Como $\{\neg, \vee\}$ es un sistema completo de conectivas, podemos suponer que $L=A\cup\{\neg\vee\}$. Para ver que $\sigma$ es una valoración booleana, hay que comprobar
		\begin{enumerate}
			\item $\sigma(\neg p) = 1+\sigma(p)$ (mod $2$).

			Si $M \vdash \neg p$, entonces $M\nvdash p$ (por consistencia), luego
			$$1 = \sigma(\neg p) = 1+\sigma(p) = 1 + 0$$

			Si $M \nvdash \neg p$, entonces $M\vdash p$, y
			$$0=\sigma(\neg p) = 1+\sigma(p) \text{ (mod 2)} = 1+1 \text{ (mod 2)}$$

			\item $\sigma(p\vee q) = \sigma(p)\vee\sigma(q)$.
			Si $M\vdash p$ o $M\vdash q$, entonces, por $p\implies (p\vee q)$ o $q\implies (p\vee q)$, que es el axioma 2, y modus ponens, obtenemos
			$$M\vdash p \vee q$$

			Por definición de $\sigma$,
			$$\sigma(p\vee q) = 1 = \max\{\sigma(p), \sigma(q)\} = \sigma(p)\vee\sigma(q)$$

			Si $M\vdash \neg p$ y $M\vdash \neg q$, entonces, usando el axioma 3: $\neg p \implies (\neg q \implies \neg(p\vee q))$ y modus ponens dos veces, obtenemos:
			$$M\vdash \neg(p\vee q)$$

			En este caso, $\sigma(p\vee q) = 0 = \sigma(p) \vee \sigma(q)$.
		\end{enumerate}
	\end{itemize}
\end{proof}

\begin{theorem}[Teorema\IS de compacidad]
	Si todo subconjunto finito de una teoría $T$ tiene un modelo, entonces $T$ tiene un modelo.
\end{theorem}
\begin{proof}
	Como todo subconjunto finito de $T$ tiene un modelo, todo subconjunto finito es consistente, luego $T$ es consistente.

	De otro modo, si $T\vdash \perp$, existe una demostración $p_1, \hdots, p_n = \perp$ usando las premisas en $T$. Pero $T\cap\{p_1,\hdots,p_{n-1}\}\vdash \perp$ y es finito.

	Por \ref{thm:comp2}, $T$ tiene un modelo.

\doneby{Edu}

Supongamos que todo subconjunto finito de una teoría $T$ tiene un modelo. Por el \ref{thm:comp2}, dichos subconjuntos son consistentes. (1)

Supongamos que $T\vdash \perp$. Como las pruebas formales son un conjunto finito de proposiciones $p_1, \hdots, p_n = \perp$ que o bien son axiomas, o bien pertenecen a $T$, o bien se obtienen por MP de dos anteriores, $T \cap \set{p_1, \hdots, p_{n-1} } = S$ es un subconjunto finito de $T$ tal que $S \vdash \perp$, luego S es inconsistente, contradicción con (1).

Luego T es consistente, y por \ref{thm:comp2}, tiene un modelo.
\end{proof}

\begin{obs}
	Si $\sigma$ es un modelo de $T$, también o es de todos sus subconjuntos finitos, luego la otra dirección en el teorema de compacidad es trivial.
\end{obs}

\begin{theorem}
	La lógica proposicional es decidible, es decir, existe un algoritmo que nos permite saber si $\vdash p$ o $\nvdash p$.
\end{theorem}
\begin{proof}
	Escribimos la tabla de verdad de $p$.
	\begin{itemize}
		\item Si $\sigma(p) = 1$ $\forall\sigma$, entonces $\vdash p$
		\item Si $\sigma(p) = 0$ para algún $\sigma$, entonces $\nvdash p$.
	\end{itemize}
\end{proof}
