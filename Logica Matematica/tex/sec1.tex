\section{Lógica proposicional}
Sea $A$ un conjunto no vacío. 
Definimos el lenguage como
$$L \equiv A\cup\{\neg,\implies, \wedge, \vee, \iff, \perp, \top, ), ( \}$$
donde los símbolos representan
\begin{itemize}
	\item $\neg:$ negación.
	\item $\implies:$ implicación.
	\item $\wedge:$ conjunción.
	\item $\vee:$ disyunción.
	\item $\implies:$ equivalencia.
	\item $\perp:$ verdadero.
	\item $\top:$ falso.
	\item $):$ paréntesis derecho.
	\item $(:$ paréntesis izquierdo.
\end{itemize}

\begin{mdframed}
\noindent\textbf{Nota:}\\
$p\implies q$ es una abreviación de $\neg p \vee q$.
\end{mdframed}

\subsection{Fórmulas bien formadas}
Dado $A$, un conjunto (finito, numerable o no numerable), de variables proposicionales $p, q, r, \hdots$, se definen:

\begin{defn}{\textbf{Palabra.}}
	Concatenación de símbolos.
\end{defn}

\begin{defn}{\textbf{Proposición o fórmula bien formada.}}
	Palabra que pertenece a la clase más pequeña cerrada bajo las siguientes operaciones:
	\begin{itemize}
		\item Los elementos de $A$: $\perp$ y $\top$ son fórmulas atómicas (no se pueden dividir).
		\item Si $F_1$ y $F_2$ son fórmulas, también lo son
		\begin{itemize}
			\item $(\neg F_1)$
			\item $(F_1\implies F_2)$
			\item $(F_1\iff F_2)$
			\item $(F_1\wedge F_2)$
			\item $(F_1\vee F_2)$
		\end{itemize}
	\end{itemize}
	En notación polaca son fórmulas bien formadas:
	\begin{itemize}
		\item $\neg$ $F_1$
		\item $\implies F_1$ $F_2$
		\item $\iff$ $F_1$ $F_2$
		\item $\wedge$ $F_1$ $F_2$
		\item $\vee$ $F_1$ $F_2$
	\end{itemize}
\end{defn}

\noindent Las fórmulas bien formadas pueden definirse de manera inductiva:
\begin{equation*}
\begin{array}{l l l}
	\textbf{FBF}_\textbf{0} &=& A\cup\{\top, \perp \}\\
	\textbf{FBF}_\textbf{n+1} &=& FBF_n \cup \{(\neg F_1), (F_1\implies F_2),(F_1\iff F_2),(F_1\wedge F_2),(F_1\vee F_2)\}\\ &&\text{con } F_1,F_2\in \textbf{FBF}_\textbf{n}.
\end{array}
\end{equation*}
Las fórmulas bien formadas son $FBF=\cap_{n=0}^\infty FBF_n$

\begin{defn}{\textbf{Aridad.}}
	La aridad de un operador es el número de argumentos que admite:
	\begin{itemize}
		\item $\perp$, $\top$ son $0$-arias.
		\item $\neg$ es $1$-aria.
		\item $\wedge$, $\vee$, $\implies$, $\iff$ son $2$-arias (binarias).
	\end{itemize}
\end{defn}
