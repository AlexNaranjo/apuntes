\section{Incompletitud}

Los principales enunciados básicos correspondientes a esta sección ya los hemos visto. Lo que haremos será añadir ciertas definiciones y varios resultados/teoremas importantes.

\textbf{Recordatorio de lógica proposicional}
\[\{p,q\}\vdash r \iff \vdash (p\y q) \to r\]
Para demostrar esto empezábamos apoyándonos en el teorema de la deducción:
\[\{p,q\} \vdash r \implies \{p\} \vdash q \to r\]
Aplicando de nuevo el teorema de la deducción obteníamos:
\[\{p\} \vdash q \to r \implies \vdash p \to (q \to r) \equiv (p \y q ) \to r\]

\begin{theorem}
Consideramos el lenguage de la aritmética formal, definido como:
\[L=\{0,S_i,+_1,\cdot_1, <_1\}\]
y la teoría $\underline{N}=\{N_1,...,N_9\}$ correspondiente a los axiomas de Peano (sin incluir inducción) y definida en el libro en la página 91.

en este lenguaje, $L$, y con la teoría $\underline{N}$, toda función recursiva total $\appl{f}{\nat^k}{\nat}$ es representable para una fórmula de primer orden $\varphi_f$.

Lo mismo ocurre con los conjuntos recursivos.
\end{theorem}

\begin{prop}[Codificación]
Las fórmulas pueden codificarse usando números naturales.

Las demostraciones también.

\end{prop}

Para llevar a cabo estas codificaciones debemos definir símbolos del lenguaje $n \in \nat$ y asociar a cada variable $v_i$ el número $n(v_i)=2i$ con lo que empleamos sólamente números pares.

Una vez tenemos esto podemos asignar a las operaciones los números impares como sigue:
\[n(\top) = 1, \ \ n(\perp)=3, \ \ n(\neg) = 5, \ \ n(\Or) = 7, \ \ n(\y)=9, \ \ n(=)=11\]
\[n(\exists)=13, \ \ n(\forall) =15, \ \ n('(')=17, \ \ n(')')=19, \ \ n(0)=21, \ \ n(S_1)=23\]
\[n(+_1)=25, \ \ n(\cdot_1)=27, \ \ n(<_1) = 29\]

El problema que nos deriva de esta codificación es que no es claro cómo descifrar el predicado ``230'' puesto que puede interpretarse como el símbolo asociado al 23 seguido de la variable asociada al 0, o como la variable asociada al 2 seguida de la variable asociada al 15, etc.

Empleamos la factorización única para resolver este problema. Para ello definimos la función primo($K$) que nos devuelve el primero que ocupa la posición $K+1$.

Así el número de Godel de una palabra se calcula como:
\[g(u_0,u_1,...,u_k)=2^{n(u_0)}\times 2^{n(u_1)}\times ... \times \text{primo}(k)^{n(u_k)}\]

De esta forma, al leer el número asociado a una fórmula basta con factorizarlo, lo que nos permite obtener la fórmula original de forma unívoca.

%TODO completar. codificacion de vectores usando el dibujito ese que recorre todo.