% -*- root: ../LogicaMatematica.tex -*-
\section{Hoja 8}
Los 7  primeros problemas en esta hoja proporcionan una ``construcción" moderadamente explicita (Zorn) de modelos
no estándar de los naturales. Consideramos probabilidades $P$ finitamente aditivas, que sólo toman los valores
$0$ y $1$.

\begin{problem}
Sea $P$ una probabilidad finitamente aditiva, definida en un álgebra de subconjuntos de $\mathbb{N}$,
con valores en $\{0,1\}$. Los conjuntos con probabilidad 1 forman un {\em filtro}. Definir {\em filtro} sin utilizar
probabilidades, y probar que las dos definiciones coinciden.

\solution

Un subconjunto no vacío $F$ de un conjunto parcialmente ordenado $(P,≤)$ es un filtro si se dan las siguientes condiciones:

\begin{enumerate}
\item Para cada $x \in F$, $y \in P$, $x ≤ y$ implica que $y$ también está en F.
\item Para cada $x, y \in F$, existe cierto elemento z en F, tal que z ≤ x y z ≤ y. (F es un conjunto filtrado)
\end{enumerate}
Un filtro se dice propio si no es igual a todo el conjunto P completo.

En este caso el conjunto parcialmente ordenado del que partimos sería $\nat$. Si consideramos que la probabilidad será 1 para los conjuntos subconjuntos parcialmente ordenados de $\nat$.

Nuestro filtro estaría formado por todos los subconjuntos de $\
nat$ parcialmente ordenados y este sería nuestro ...

\textcolor{red}{ya no se que más inventarme}
\end{problem}

\begin{problem}
Probar que todo filtro de subconjuntos de un conjunto $S$,
 puede extenderse a un ultrafiltro (a un filtro maximal con respecto a la inclusión) en $S$.
Sugerencia: Zorn.

\solution

\end{problem}

\begin{problem}
Probar que si $U$ es un ultrafiltro en $S$, para todo $A\subset S$, o $A\in U$ o $A^c\in U$.

\solution

\end{problem}

\begin{problem}
Probar que si $U$ es un ultrafiltro en $S$, y  $A_1 \cup \cdots \cup A_n = S$, entonces existe una $j\in \{ 1, \dots, n\}$ tal que
$A_j\in U$.

\solution

\end{problem}

\begin{problem}
Sea  $\prod_{n\in \mathbb{N}} X_n$ un producto cartesiano de conjuntos no vacios, y sea $U$ un ultrafiltro en $\mathbb{N}$.
Para $x,y \in  \prod_{n\in \mathbb{N}}\ X_n$, definimos $x\sim y$ si $x$ e $y$ son iguales casi seguro, es
decir si $\{n\in \mathbb{N}: x_n = y_n\}\in U$. Probar que $\sim$ es una relación de equivalencia.
Al conjunto $\mathcal{U}:=\prod_{n\in \mathbb{N}} X_n/\sim$ se le denomina {\em ultraproducto}.

\solution

Si $\sim$ es una relación de equivalencia, entonces cumple las siguientes propiedades:
\begin{itemize}
	\item $x \sim x$ {\bf Reflexiva}
	\item $x \sim y \implies y \sim x$ {\bf Simétrica}
	\item $x \sim y, y \sim z \implies x \sim z$ {\bf Transitiva}
\end{itemize}

Tenemos que probar que para $x,y \in \prod\limits_{n \in \nat} X_n$, $U$ ultrafiltro en $\nat$
\[ \set{ n \in \nat : x_n = y_n} \in U \xLeftrightarrow{\text{def}} x \sim y \]

Gracias a que la igualdad es una relación de equivalencia, tenemos que si $x \sim y$,
\[\set{ n \in \nat : x_n = y_n} = \set{ n \in \nat : y_n = x_n} \implies y \sim x \]
Debido a que la igualdad es simétrica, ambos conjuntos de índices contendrán los mismos elementos $n$, y si $x\sim y$, entonces dicho conjunto de índices pertenece a $U$.

Por el mismo motivo, si $\set{ n \in \nat : x_n = x_n} \in U$, entonces $x \sim x$; ya que\\
$\forall x \in \prod\limits_{n \in \nat} X_n$, $x_n = x_n \ \forall n$.

Finalmente, tenemos que si $x_n = y_n$, $y_n = z_n \implies x_n = z_n$, es decir,
\[x \sim y, y \sim z \implies x \sim z\]

\end{problem}

\begin{problem}
Tomando $X_n = \mathbb{N}$ para todo $n$ en la construcción anterior, probar que $\mathcal{U}$ es un modelo de los axiomas de Peano. Sugerencia: Buscar en la literatura el Teorema Fundamental de los Ultraproductos, de {\L}o\'s, e invocarlo.

\solution

\end{problem}

\begin{problem}
Probar que si el ultrafiltro $U$ contiene a todos los subconjuntos de $\mathbb{N}$ con complemento finito, entonces
existe una $c\in \mathcal{U}$ tal que para todo $n\in \mathbb{N}, c>n$. Sugerencia: tomar $c_n = n$.

\solution

\end{problem}

\begin{problem}
Dada la MT con alfabeto $\set{B, 1, 2}$, definida por $\{(q_0 \ 1 \ 2 \ q_0), (q_0 \ 2 \ D \ q_0),  (q_0 \ B \ D \ q_0)\}$,
determinar su output cuando el input es $1 \ 1  \ B \ B \ 1 \  B  \ 2  \ B \ $ (el resto de la cinta son $B's$). Describir en general que hace esta MT, y si se detiene el algún momento o no.

\solution
Esta MT recorre la cinta buscando $1s$, cuando encuentra un 1, lo cambia por un 2, y sigue avanzando hacia la derecha. Esta MT no para nunca, ya que en la cinta solo puede haber caracteres en el alfabeto $\set{B, 1, 2}$, si encuentra un 1 lo convierte en un 2, y si se encuentra un 2 ó un B avanza a la derecha, de manera que no parará nunca de desplazarse a la derecha.

Para el {\it input} dado, realizará las siguientes acciones:

\begin{enumerate}
	\item empezamos en el estado $q_0$ y con el cabezal en el 1 de la izquierda. Reescribimos ese 1 por un 2 y pasamos al estado $q_0$. La cinta queda así: $2 \ |\ 1  \ |\ B \ |\ B \ |\ 1 \ |\  B  \ |\ 2  \ |\ B \dots$
	\item como estamos en el estado $q_0$ y leemos un 2, avanzamos a la derecha y pasamos al estado $q_0$. La cinta no cambia.
	\item como estamos en el estado $q_0$ y leemos un 1, lo cambiamos por un 2 y pasamos al estado $q_0$. La cinta queda así: $2 \ |\ 2 \ |\ B \ |\ B \ |\ 1 \ |\  B  \ |\ 2  \ |\ B \dots$
	\item como estamos en el estado $q_0$ y leemos un 2, avanzamos a la derecha y pasamos al estado $q_0$. La cinta no cambia.
	\item como estamos en el estado $q_0$ y leemos B, avanzamos a la derecha y pasamos al estado $q_0$. La cinta no cambia.
	\item como estamos en el estado $q_0$ y leemos B, avanzamos a la derecha y pasamos al estado $q_0$. La cinta no cambia.
	\item como estamos en el estado $q_0$ y leemos 1, lo cambiamos por un 2 y pasamos al estado $q_0$. La cinta queda así: $2 \ |\ 2 \ |\ B \ |\ B \ |\ 2 \ |\  B  \ |\ 2  \ |\ B \dots$
	\item como estamos en el estado $q_0$ y leemos un 2, avanzamos a la derecha y pasamos al estado $q_0$. La cinta no cambia.
	\item como estamos en el estado $q_0$ y leemos B, avanzamos a la derecha y pasamos al estado $q_0$. La cinta no cambia.
	\item como estamos en el estado $q_0$ y leemos un 2, avanzamos a la derecha y pasamos al estado $q_0$. La cinta no cambia.
	\item como estamos en el estado $q_0$ y leemos B, avanzamos a la derecha y pasamos al estado $q_0$. La cinta no cambia.
\end{enumerate}

A partir de ahora, suponiendo una cadena infinita de símbolos B, la MT no parará de avanzar a la derecha.

\end{problem}


