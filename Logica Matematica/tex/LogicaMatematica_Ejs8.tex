% -*- root: ../LogicaMatematica.tex -*-
\section{Hoja 8}
Los 7  primeros problemas en esta hoja proporcionan una ``construcción" moderadamente explicita (Zorn) de modelos
no estándar de los naturales. Consideramos probabilidades $P$ finitamente aditivas, que sólo toman los valores
$0$ y $1$.

\begin{problem}
Sea $P$ una probabilidad finitamente aditiva, definida en un álgebra de subconjuntos de $\mathbb{N}$,
con valores en $\{0,1\}$. Los conjuntos con probabilidad 1 forman un {\em filtro}. Definir {\em filtro} sin utilizar
probabilidades, y probar que las dos definiciones coinciden.

\solution

\begin{defn}[Filtro]
Un subconjunto no vacío $F$ de un conjunto parcialmente ordenado $(R,≤)$ es un filtro si se dan las siguientes condiciones:

\begin{enumerate}
\item Para cada $x \in F$, $y \in R$, $x ≤ y$ implica que $y$ también está en F.
\item Para cada $x, y \in F$, existe cierto elemento $z \in F$, tal que $z ≤ x$ y $z ≤ y$. (F es un conjunto filtrado)
\end{enumerate}
Un filtro se dice propio si no es igual a todo el conjunto P completo.
\end{defn}

En concreto, parece que nos interesa trabajar con filtros de conjuntos, cuya definición es:

\begin{defn}[Filtro de conjuntos]
Un filtro de conjuntos se obtiene tomando el conjunto potencia de un conjunto dado $S$, visto como orden parcial y ordenado por la inclusión de subconjuntos. Con ello tendremos que un filtro $F$ sobre un conjunto $S$ es un conjunto de subconjuntos de $S$ con las siguientes propiedades:

\begin{enumerate}
\item $S$ está en $F$. ($F$ es no vacío)
\item $F$ no contiene al conjunto vacío. ($F$ es propio)
\item Si $A$ y $B$ están en $F$, también su intersección. ($F$ es cerrado bajo intersecciones finitas)
\item Si $A$ está en $F$ y $A$ es un subconjunto de $B$, entonces $B$ está en $F$, para todos los subconjuntos $B$ de $S$. ($F$ es cerrado bajo supercontenencias)
\end{enumerate}
\end{defn}
Si resctamaos la definición de álgebra de conjuntos de TIM tenemos:

\begin{defn}[{Á}lgebra\IS de conjuntos]
Dado un conjunto X, no vacío, y dado un subconjunto de sus partes $\algb{M} \subset \parts{X}$, se dice que $\algb{M}$ es un álgebra de conjuntos si:
\begin{enumerate}
\item $\algb{M}$ es no vacío.
\item $\algb{M}$ es cerrado por la unión finita. Esto es, si tenemos $A_1, A_2, \dotsc, A_n ∈ \algb{M}$, entonces $\bigcup_{i=1}^n A_i ∈ \algb{M}$.
\item $\algb{M}$ es cerrado por complementación. Es decir, si $A ∈ \algbM$ entonces $A^c = X \setminus A ∈ \algbM$.
\end{enumerate}
\end{defn}

Por definición de la probabilidad $P$ que estamos trabajando sabemos que, dado una sucesión finita y disjunta de conjuntos $E_i$ tenemos:
\[P\left( \bigcup_{i=1}^n E_i\right) = \sum_{i=1}^nP(E_i)\]

Es claro que el comienzo de la definición de filtro de conjuntos encaja a la perfección con la definición de filtro que estamos manejando con probabilidades. Nuestro filtro, recordemos, serían los conjuntos con probabilidad 1.

Ahora debemos comprobar que se satisfacen las 4 propiedades de la definición de \textbf{filtro de conjuntos} donde las dos primeras y la última son triviales, si recordamos que la probabilidad de un conjunto nunca puede ser mayor que 1.

\textcolor{red}{Que alguién que estudie ahora probabilidad trate de completar esto}



\end{problem}

\begin{problem}
Probar que todo filtro de subconjuntos de un conjunto $S$,
 puede extenderse a un ultrafiltro (a un filtro maximal con respecto a la inclusión) en $S$.
Sugerencia: Zorn.

\solution

Recordemos que el lemma de zorn dice: ``Todo conjunto parcialmente ordenado no vacío en el que toda cadena (subconjunto totalmente ordenado) tiene una cota superior, contiene al menos un elemento maximal''.

Dado el conjunto $S$ tomamos un filtro cualquiera $F$ y a partir de él definimos $\algb{F}$ como la colección de filtros que contienen a $F$.

Es sencillo ver que $\algb{F}$ no es vacío, pues contiene a $F$ y, además, $\algb{F}$ está ordenador por inclusión.

Consideramos ahora $\algb{D}=(D_i)$ una cadena en $\algb{F}$ y definimos $D=\cup D_i$.

Con esta definición es sencillo ver que $D$ es un filtro, pues satisface las condiciones de la definición:
\begin{proof}
Si tomamos dos elementos $x,y \in D$ entonces existen dos subíndices $i,j$ tales que $x\in D_i$ y $y \in D_j$. Al ser $\algb{D}$ una cadena, para algún $k\in \{i,j\}$ se ha de tener que $x,y \in D_k$. Como $D_k$ es un filtro, $x,y \in D_k \subset D$.

Por otro lado, si $x\in D$ y tenemos un elemento $x \leq y$, entonces para algún $i$ tenemos $x\in D_i$ y al ser este un filtro, $y \in D_i \subset D$. Ahora bien, como para todo $i$ $\emptyset \notin D_i$, se tiene que $\emptyset \notin D$.

\end{proof}

Queda concluido que $D$ es un filtro propio que acota superiormente a la cadena $\algb{D}$ con lo que se satisfacen las hipótesis del lema de zorn.

Como conlusión, existe un elemento maximal de $\algb{F}$, que será un ultrafiltro.
\end{problem}

\begin{problem}
Probar que si $U$ es un ultrafiltro en $S$, para todo $A\subset S$, o $A\in U$ o $A^c\in U$.

\solution

\end{problem}

\begin{problem}
Probar que si $U$ es un ultrafiltro en $S$, y  $A_1 \cup \cdots \cup A_n = S$, entonces existe una $j\in \{ 1, \dots, n\}$ tal que
$A_j\in U$.

\solution

\end{problem}

\begin{problem}
Sea  $\prod_{n\in \mathbb{N}} X_n$ un producto cartesiano de conjuntos no vacios, y sea $U$ un ultrafiltro en $\mathbb{N}$.
Para $x,y \in  \prod_{n\in \mathbb{N}}\ X_n$, definimos $x\sim y$ si $x$ e $y$ son iguales casi seguro, es
decir si $\{n\in \mathbb{N}: x_n = y_n\}\in U$. Probar que $\sim$ es una relación de equivalencia.
Al conjunto $\mathcal{U}:=\prod_{n\in \mathbb{N}} X_n/\sim$ se le denomina {\em ultraproducto}.

\solution

Si $\sim$ es una relación de equivalencia, entonces cumple las siguientes propiedades:
\begin{itemize}
	\item $x \sim x$ {\bf Reflexiva}
	\item $x \sim y \implies y \sim x$ {\bf Simétrica}
	\item $x \sim y, y \sim z \implies x \sim z$ {\bf Transitiva}
\end{itemize}

Tenemos que probar que para $x,y \in \prod\limits_{n \in \nat} X_n$, $U$ ultrafiltro en $\nat$
\[ \set{ n \in \nat : x_n = y_n} \in U \xLeftrightarrow{\text{def}} x \sim y \]

Gracias a que la igualdad es una relación de equivalencia, tenemos que si $x \sim y$,
\[\set{ n \in \nat : x_n = y_n} = \set{ n \in \nat : y_n = x_n} \implies y \sim x \]
Debido a que la igualdad es simétrica, ambos conjuntos de índices contendrán los mismos elementos $n$, y si $x\sim y$, entonces dicho conjunto de índices pertenece a $U$.

Por el mismo motivo, si $\set{ n \in \nat : x_n = x_n} \in U$, entonces $x \sim x$; ya que\\
$\forall x \in \prod\limits_{n \in \nat} X_n$, $x_n = x_n \ \forall n$.

Finalmente, tenemos que si $x_n = y_n$, $y_n = z_n \implies x_n = z_n$, es decir,
\[x \sim y, y \sim z \implies x \sim z\]

\end{problem}

\begin{problem}
Tomando $X_n = \mathbb{N}$ para todo $n$ en la construcción anterior, probar que $\mathcal{U}$ es un modelo de los axiomas de Peano. Sugerencia: Buscar en la literatura el Teorema Fundamental de los Ultraproductos, de {\L}o\'s, e invocarlo.

\solution

\end{problem}

\begin{problem}
Probar que si el ultrafiltro $U$ contiene a todos los subconjuntos de $\mathbb{N}$ con complemento finito, entonces
existe una $c\in \mathcal{U}$ tal que para todo $n\in \mathbb{N}, c>n$. Sugerencia: tomar $c_n = n$.

\solution

\end{problem}

\begin{problem}
Dada la MT con alfabeto $\set{B, 1, 2}$, definida por $\{(q_0 \ 1 \ 2 \ q_0), (q_0 \ 2 \ D \ q_0),  (q_0 \ B \ D \ q_0)\}$,
determinar su output cuando el input es $1 \ 1  \ B \ B \ 1 \  B  \ 2  \ B \ $ (el resto de la cinta son $B's$). Describir en general que hace esta MT, y si se detiene el algún momento o no.

\solution
Esta MT recorre la cinta buscando $1s$, cuando encuentra un 1, lo cambia por un 2, y sigue avanzando hacia la derecha. Esta MT no para nunca, ya que en la cinta solo puede haber caracteres en el alfabeto $\set{B, 1, 2}$, si encuentra un 1 lo convierte en un 2, y si se encuentra un 2 ó un B avanza a la derecha, de manera que no parará nunca de desplazarse a la derecha.

Para el {\it input} dado, realizará las siguientes acciones:

\begin{enumerate}
	\item empezamos en el estado $q_0$ y con el cabezal en el 1 de la izquierda. Reescribimos ese 1 por un 2 y pasamos al estado $q_0$. La cinta queda así: $2 \ |\ 1  \ |\ B \ |\ B \ |\ 1 \ |\  B  \ |\ 2  \ |\ B \dots$
	\item como estamos en el estado $q_0$ y leemos un 2, avanzamos a la derecha y pasamos al estado $q_0$. La cinta no cambia.
	\item como estamos en el estado $q_0$ y leemos un 1, lo cambiamos por un 2 y pasamos al estado $q_0$. La cinta queda así: $2 \ |\ 2 \ |\ B \ |\ B \ |\ 1 \ |\  B  \ |\ 2  \ |\ B \dots$
	\item como estamos en el estado $q_0$ y leemos un 2, avanzamos a la derecha y pasamos al estado $q_0$. La cinta no cambia.
	\item como estamos en el estado $q_0$ y leemos B, avanzamos a la derecha y pasamos al estado $q_0$. La cinta no cambia.
	\item como estamos en el estado $q_0$ y leemos B, avanzamos a la derecha y pasamos al estado $q_0$. La cinta no cambia.
	\item como estamos en el estado $q_0$ y leemos 1, lo cambiamos por un 2 y pasamos al estado $q_0$. La cinta queda así: $2 \ |\ 2 \ |\ B \ |\ B \ |\ 2 \ |\  B  \ |\ 2  \ |\ B \dots$
	\item como estamos en el estado $q_0$ y leemos un 2, avanzamos a la derecha y pasamos al estado $q_0$. La cinta no cambia.
	\item como estamos en el estado $q_0$ y leemos B, avanzamos a la derecha y pasamos al estado $q_0$. La cinta no cambia.
	\item como estamos en el estado $q_0$ y leemos un 2, avanzamos a la derecha y pasamos al estado $q_0$. La cinta no cambia.
	\item como estamos en el estado $q_0$ y leemos B, avanzamos a la derecha y pasamos al estado $q_0$. La cinta no cambia.
\end{enumerate}

A partir de ahora, suponiendo una cadena infinita de símbolos B, la MT no parará de avanzar a la derecha.

\end{problem}


