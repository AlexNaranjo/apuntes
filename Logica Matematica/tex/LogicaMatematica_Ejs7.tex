\section{Hoja 7}

Asumimos genérica e informalmente la no trivialidad. Al escribir fórmulas, se permiten
las abreviaciones razonables.

\begin{problem}
Leer los ejemplos 1-6 p.24 y los ejemplos 1-10  pp. 35-36 del libro. Sea $\sigma$
una sentencia en el lenguaje de los anillos (Ejemplo 6 p.24), verdadera en todos los
cuerpos de caracter\'{\i}stica 0. Decidir razonadamente si la siguiente afirmaci\'on
es verdadera o falsa: existe una $N$ tal que para todo primo $p > N$, $\sigma$ es verdad
en todos los cuerpos de caracter\'{\i}stica $p$.
\solution

% https://en.wikipedia.org/wiki/Compactness_theorem
% https://amathew.wordpress.com/2010/03/29/robinsons-characteristic-p-principle-and-the-ax-grothendieck-theorem/#more-1271
% http://self.gutenberg.org/article/whebn0000152207/compactness%20theorem

Sea $\sigma$ verdadera en todos los cuerpos de característica 0, sea Fl la teoría de los cuerpos, y sea Fl(k) la teoría de los cuerpos con característica k.

Con esta nomenclatura es evidente que Fl(0) $\cup \set{\neg \sigma}$ es insatisfacible.

Por tanto, podemos encontrar un subconjunto S de Fl(0) que contenga al menos Fl y a un conjunto finito de sentencias $\set{n \cdot 1 \neq 0 : n=2,3,\dots, N}$ tal que al añadirle $\set{\neg \sigma}$, sea insatisfacible.

Así cualquier modelo de S será un cuerpo de característica $\geq N$ que no satisface $\set{\neg \sigma}$, es decir, cualquier modelo de la teoría de los cuerpos con característica $\geq N$ satisfacerá $\set{ \sigma}$, luego por el Teorema de Compacidad todos los cuerpos de característica $\geq N$ satisfacerán $\sigma$.
\end{problem}

\begin{problem}
Dar un ejemplo no trivial de teoría completa. Sugerencia: buscarlo en el libro.
\solution

Según el libro, la teoría de los cuerpos algebraicos de característica 0 es completa (p 47).
Dicha teoría se describe en la página 36 de la siguiente forma:

\[ ACF(0) := \text{ ACF } \cup \set{n \cdot 1 \neq 0 \tq n = 2,3,5,7,11, \dots} \]
Con ACF la teoría de los cuerpos algebraicos cuyos modelos son los cuerpos (anillos conmutativos con inverso multiplicativo) en los que todos los polinomios de grado mayor o igual que 2 tienen al menos un 0.
\end{problem}

\begin{problem}
Sea  $L$  el siguiente lenguaje de primer orden  con igualdad: las constantes de $L$ son
$\{\underline{n}: n\in\mathbb{N}\}$, la \'unica funci\'on unaria, $S_1$, las funciones binarias,
$+_1$ y $\cdot_1$, y la relaci\'on binaria $<_1$ (los subrayados y subindices est\'an para
distinguir los s\'{\i}mbolos de sus interpretaciones habituales). A efectos de este problema
consideramos como versi\'on oficial de los axiomas de Peano los N1-N9 p. 95 del libro, junto con el esquema
de axiomas de la p. 111 (inducci\'on), expresados en el lenguaje $L$ de la forma
obvia.

\ppart Sea $\Sigma_1$ la teor\'{\i}a,
 en el lenguaje $L$, que se obtiene al a$\operatorname{\tilde{n}}$adir a los
axiomas de Peano la coleccion $\{ \psi_n: n\in\mathbb{N}\}$, donde
$\psi_n$ es la fbf $\exists x (\underline{n} <_1 x )$.
 Decidir razonadamente si
$\Sigma_1$ es consistente. De modo m\'as espec\'{\i}fico, decidir razonadamente si $\Sigma_1$ tiene un modelo.  A\'un m\'as espec\'{\i}ficamente, decidir si los n\'umeros naturales
de toda la vida, en cuya existencia todos creemos,  son un modelo de $\Sigma_1$.
De existir alg\'un modelo, decidir razonadamente si es \'unico.
Sugerencia: recordar el primer Teorema de Incompletitud de G\"odel, o el segundo.

\ppart Sea $L_c$ el lenguaje de primer orden que se obtiene al a$\operatorname{\tilde{n}}$adir
a $L$ una nueva constante $c$, y sea
$\Sigma_2$ la teor\'{\i}a,
 en el lenguaje $L_c$, que se obtiene al a$\operatorname{\tilde{n}}$adir a los
axiomas de Peano la colección $\{ \phi_n: n\in\mathbb{N}\}$, donde
$\phi_n$ es la fbf $\underline{n} <_1 c$.
Decidir razonadamente si
$\Sigma_2$ es consistente. De modo m\'as espec\'{\i}fico, decidir razonadamente si $\Sigma_2$ tiene un modelo.  Decidir si los n\'umeros naturales
de toda la vida, en cuya existencia todos creemos,  son un modelo de $\Sigma_2$.
De existir alg\'un modelo, decidir razonadamente si es \'unico.
Sugerencia: recordar el primer Teorema de Incompletitud de G\"odel, o el segundo.
\solution
\spart

%Por completitud sabemos que si $Σ_1$ tiene algún modelo entonces será consistente.
%
%Podemos ver que el conjunto de los naturales es un modelo de $Σ_1$ pues satisface todos los axiomas de Peano y todas las $\psi_n$, puesto que es infinito.
%
%No obstante no es el único modelo pues $\nat[x]$, por ejemplo, es también un modelo.
%
%Además es sencillo ver que $\nat$ no es isomorfo a $\nat[x]$ puesto que si asociamos cada natural con un polinomio que lo tenga a él como término independiente vemos claramente que la función no es sobreyectiva.

\textbf{Solución del profesor:}
Dados los axiomas de Peano más $\{\exists x (\underline{n}<_1 x): \underline{n}\in \nat \}$,
tenemos que $\nat$ es un modelo si tomamos $x = \underline{n+1}$. Por incompletitud de Gödel,
$\exists \sigma \tq \Sigma_1 \nvdash \sigma$ y $\Sigma_1 \nvdash \neg \sigma$. Por tanto, $\Sigma_1\cup\{\sigma\}$ y $\Sigma_1\cup \{\neg\sigma\}$ son consistentes, luego tienen modelos distintos no isomorfos.

\spart

%Por el Primer Teorema de incompletitud de Gödel,
%\[ \Sigma \vdash \sigma \iff \mathcal{A} \models \sigma, \ \forall \mathcal{A} \text{ modelo de } \Sigma\]
%
%Aunque el valor de $c$ dependa de la interpretación, $\Sigma$ no puede tener modelos ya que $\nexists c $ tal que $\forall n, n < c$. Porque de ser cierto, los números naturales estarían acotados y por tanto, existiría $m = \max\set{ n : n < c}, m \in \nat$, pero $m < Sm$, luego $Sm > c \lor Sm < c$. En el primer caso, entonces $\nat$ no sería un modelo de $\Sigma_2$, y en el segundo obtendríamos una contradicción con que m es el máximo.

\textbf{Solución del profesor:}
Sea $L_c = \varphi_n: \underline{n} < c$. Para cada subconjunto finito, $\{\varphi_{n_1}, \hdots, \varphi_{n_k}\}$ y sea $N = \max\{n_1, \hdots, n_k\}+1\in \nat$. Interpretando $c$ como $N$ se tiene que $\nat$ es un modelo de Peano$+\{\varphi_{n_1}, \hdots, \varphi_{n_k}\}$, luego $\Sigma_2$ tiene un modelo por completitud, que no es $\nat$ porque no podemos interpretar $c$ como un elemento de $\nat$ que satisfaga todas las $\varphi$.

Sea $^*\nat$ un modelo de $\Sigma_2$. Entonces $c\in ^*\nat \tq c>0, c>1, \hdots$.
$c$ es un número natural infinito.

El esquema de inducción no falla porque no hay una sentencia de primer orden que exprese la noción ``x es finito''.
\end{problem}
