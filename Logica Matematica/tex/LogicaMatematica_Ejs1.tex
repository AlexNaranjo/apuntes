% -*- root: ../LogicaMatematica.tex -*-
\chapter{Problemas resueltos}
\section{Hoja 1}

	
	Asumimos gen\'erica e informalmente la no trivialidad. Por ejemplo,
	cuando hablamos de conjuntos, lenguajes, etc., suponemos que no  son vac\'{\i}os (salvo que 
	expl\'{\i}citamente
	se diga lo contrario),  cuando hablamos de f\'ormulas suponemos que est\'an bien formadas, etcetera.
	En caso de duda, consultar con el intructor. 
	
	\begin{problem}[1]
	Aprenderse el lema de Zorn de memoria. Comentario: el lema de Zorn es equivalente al Axioma de
	Elecci\'on.
	\solution
	
\textit{Todo conjunto parcialmente ordenado no vacío en el que toda cadena (subconjunto totalmente ordenado) tiene una cota superior, contiene al menos un elemento maximal.}
	
	
	\end{problem}
	
	\begin{problem}[2]
	Sea $A$ un conjunto finito y parcialmente ordenado. Denotamos por $R$ el correspondiente
	orden parcial (usamos $R$ porque un orden parcial es una relaci\'on binaria, o con ``aridad" 2). Demostrar que
	$A$ contiene un elemento maximal. Demostrar que $R$ puede extenderse a un orden total o lineal,
	es decir, existe un orden total  $R^\prime$ tal que $R\subset R^\prime$ (orden lineal o total
	significa que para todo par
	$x, y\in A$, o bien $(x,y)\in R^\prime$, \'o $(y,x)\in R^\prime$).
	\solution
	%Vamos a escribir el conjunto $A$ como unión de subconjuntos $B_i$ totalmente ordenados:
	%$$A = \cup_i B_i$$
	%$$B_i = \{x,y \in A: xR'y\}$$
	
	%Al ser $A$ un conjunto finito, $\forall i$ $B_i \neq \emptyset$ es un subconjunto finito, que al estar ordenado tiene cota superior.
	
	%Con esto ya podemos usar el \textbf{Lema de Zorn} y afirmar que $A$ tiene un elemento maximal.

	Vamos a escribir el conjunto A como:
	\[A = \bigcup_i B_i\]
	para ello vamos a construir los conjuntos $B_i$ como sigue:
	\begin{enumerate}
	\item Tomamos un elemento $x \in A$, que usaremos para definir $B_1$
	\item Recorremos uno a uno de $A$ y los insertamos en $B_1$ forzando que $R$ siempre constituya un orden total en $B_1$. Esto es, cada vez que vayamos a insertar un elemento, comprobamos primero que tenga relación con todos los elementos que ya se encuentran en $B_1$
	\item Cuando hemos recorrido todos los elementos de $A$, damos por concluida la construcción de $B_1$
	\item Repetimos el proceso empezando con un elemento distinto de $A$
	\end{enumerate}
	
	Puesto que $A$ es finito, este procedimiento será también finito y nos permitira escribir $A$ como unión de $n$ conjuntos totalmente ordenados, siendo $n$ el cardinal de $A$.
	
	Es sencillo comprobar que no es posible que ningún conjunto quede vacío (como mínimo tiene el elemento que usamos como base para construirlo) y tampoco puede constar de sólo un elemento, pues no puede haber ningun elemento en $A$ que no guarde relación con ningún otro.

	Puesto que los conjuntos $B_i$ son finitos y totalmente ordenados, es claro que tienen un máximo, por lo que podemos aplicar el lema de Zorn, probando así que el conjunto $A$ tiene un elemento maximal.

	Ahora tenemos que ver como extendemos el orden parcial $R$ a un orden total $R'$. 

	Para que sea una extensión debe cumplirse:
	\[xRy \implies xR'y\]
	Para definir el orden total $R'$ podemos apoyarnos en $R$ siempre que sea posible, con lo que garantizamos que se cumpla la implicación. Tras esto, nos quedaría definir $R'$ para aquellos elementos que no estén relacionados en $R$.

	Si tenemos un par de elementos tales que $(x,y)\notin R \& (y,x)\notin R$ entonces, por definición, tenemos que $x,y$ pertenecen a distintos $B_i$. 

	Llegados a este punto podemos dar un orden ``arbitrario'' al conjunto finito de $B_i$ de forma que 
	\[b_i R' b_{i+1} \]

	Así nos queda:
	\[
	xR'y = \left\{ \begin{array}{lcc}
             xRy &   si  & \exists i  \tq x,y \in B_i \\
             \\ i R j &  si  & x \in B_i \ \& y \in B_j 
             \end{array}
   \right.\]
	\end{problem}
	
	\begin{problem}[3]
	El lema de legibilidad \'unica nos dice que las f\'ormulas bien formadas no son ambiguas, pueden
	leerse de un \'unico modo. Reescribir las f\'ormulas que aparecen a continuaci\'on en notaci\'on polaca, usando
	la notaci\'on est\'andar, y las que aparecen en notaci\'on est\'andar, en polaca.
	
	a) $\vee \ \neg \ \to  \  p \  q \ \leftrightarrow \ r \  p$, 
	$\to\ \to \ \wedge \ \to \ p \ q \ \vee q \ r \ \vee p \ r \ \neg \ \vee q \ s$.
	
	b)  $(q \to  (p \to q ))$, $( (p \leftrightarrow q ) \to ((\neg \  q)  \vee  r))$.
	
	Hallar el \'arbol de descomposici\'on de las f\'ormulas en b). 
	\solution
	a)
	$$\neg(p\rightarrow q)\vee (r \leftrightarrow p)$$
	
	$$(((p\rightarrow q)\wedge (q \vee r)) \rightarrow (p \vee r) \rightarrow (\neg(q \vee s))$$
	
	b)
	
	\textbf{Árbol de descomposición}
	
	\begin{center}

		\Tree[.(q$\to$(p$\to$q)) [.q$\to$ [.p ] ] [.(p$\to$q) [.p ] [.q ] ] ]
	\end{center}
	
	
	\textbf{Notación polaca:}
	$$\rightarrow q \rightarrow pq$$
	
	\textbf{Árbol de descomposición}
	
	\begin{center}
		
		\Tree[.$((p\leftrightarrow q)\to((\neg q)\vee r))$ [.$(p\leftrightarrow q)$ [.$p$ ] [.$q$ ] ] [.$((\neg q)\vee r)$ [.$(\neg q)$ [.$q$ ] ] [.$r$ ] ] ]
	\end{center}
	
	
	\textbf{Notación polaca:}
	$$\rightarrow \leftrightarrow pq \vee \neg qr$$
	\end{problem}
	
	\begin{problem}
	 Sea $p$ una f\'ormula (bien formada). Probar que el n\'umero de par\'entesis izquierdos en
	$p$ es igual al  n\'umero de par\'entesis derechos. Sugerencia, usar inducci\'on. 
	\solution
	
	Atendiendo a la notación dada en clase , tenemos un conjunto de elementos atómicos , que llamamos $A$ y que le añadios los elementos $\top$ y $\bot$. La familia de todas las fórmulas bien formadas (FBF) se constituye de la siguiente forma:
\begin{equation*}
	\begin{array}{l l l}
		\textbf{FBF}_\textbf{0} &=& A\cup\{\top, \perp \}\\
		\textbf{FBF}_\textbf{n+1} &=& FBF_n \cup \{(\neg F_1), (F_1\implies F_2),(F_1\iff F_2),(F_1\wedge F_2),(F_1\vee F_2)\}\\ &&\text{con } F_1,F_2\in \textbf{FBF}_\textbf{n}.
	\end{array}
\end{equation*}

Con esta notación vamos a demostrar el ejercicio usando inducción:

\underline{Caso base: $n=0$}

$$\textbf{FBF}_\textbf{0} = \{(a) : a\in A \cup \{\top , \bot\}\}$$
El elemento $(a)$ tiene el mismo número de parétesis derechos que izquierdos.

Suponemos que esto también se cumple para el caso $\textbf{FBF}_\textbf{n}$, vamos a ver que también se cumple para $\textbf{FBF}_\textbf{n+1}$

$$\textbf{FBF}_\textbf{n+1} = FBF_n \cup \{(\neg F_1), (F_1\implies F_2),(F_1\iff F_2),(F_1\wedge F_2),(F_1\vee F_2)\}\\ \text{con } F_1,F_2\in \textbf{FBF}_\textbf{n}.$$

Hemos supuesto que todos los elementos de $\textbf{FBF}_\textbf{n}$ tienen el mismo número de paréntesis derechos e izquierdos, los elementos que tenemos que comprobar que también lo cumplen son:$(\neg F_1), (F_1\implies F_2),(F_1\iff F_2),(F_1\wedge F_2),(F_1\vee F_2)$.

Es fácil ver que esto es cierto ya $ F_1,F_2\in \textbf{FBF}_\textbf{n}$ y se les añade un paréntesis por la derecha y otro por la izquierda.
\end{problem}