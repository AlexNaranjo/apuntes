% -*- root: ../LogicaMatematica.tex -*-
\section{Hoja 4}

Asumimos gen\'erica e informalmente la no trivialidad. Al escribir f\'ormulas, se permiten
las abreviaciones razonables (por ejemplo, las que he usado en el primer problema).

\begin{problem}[1]
Forma normal disyuntiva: una fbf est\'a en forma normal
disyuntiva si se expresa como $p_1 \vee \dots \vee p_n$,
donde $n\ge1$,  cada $p_i$ es de la forma
$q_{i, 1} \wedge \dots \wedge q_{i,n_i}$, con $n_i\ge1$,
y cada $q_{i, j}$ es o bien at\'omica o la negaci\'on de una f\'ormula at\'omica.
Por ejemplo, si $p, q, $ y $r$ son \'atomos, entonces $\neg p$, $p\vee q$,
$p \wedge q \wedge r$, y  $p \vee (q \wedge r)$ est\'an en forma normal disyuntiva,
mientras que  $p \wedge (q \vee r)$ no lo est\'a. Probar que toda fbf
es equivalente a una fbf en forma normal disyuntiva (las formas
normales conjuntivas se definen de modo an\'alogo, intercambiando conjunciones
y disyunciones, y el resultado tambi\'en es cierto para ellas).

\solution


\end{problem}

\begin{problem}
Sea $A=\{p, q, r\}$, sea $L = A \cup \{\neg, \vee, \wedge, ), (\}$,
y sea $\sigma : A\to \{0,1\}$ una funci\'on arbitraria. Denotando tambi\'en
por $\sigma$ su extensi\'on \'unica a una valoraci\'on Booleana definida en $\operatorname{fbf}(L)$, decidir razonadamente
si existe una fbf $F$ tal que $\sigma (F) = 1$. Decidir razonadamente
qu\'e ocurre si la cardinalidad de $A$ es 2015 en vez de 3.
\solution

\end{problem}

\begin{problem}
Sea $\sigma$ una valoraci\'on Booleana, y sea $T :=\{p\in FBF(L) : \sigma(p) = 1\}$.
Demostrar que $T$ es una teor\'{\i}a completa.
\solution

\end{problem}

\begin{problem}
Sea $G$ un grafo infinito, tal que todos sus subgrafos finitos pueden colorearse con cuatro colores.
Decidir razonadamente si $G$ puede colorearse con cuatro colores.
\solution

\end{problem}

\begin{problem}
Sea $X$ un conjunto parcialmente ordenado. Decidir razonadamente si el orden parcial puede
extenderse a un orden lineal, o total.
\solution

\end{problem}

\begin{problem}
Expresar ``$f$ es una funci\'on inyectiva"  en un lenguaje de primer orden.
\solution

\end{problem}

\begin{problem}
Expresar ``$f$ es una funci\'on sobreyectiva"  en un lenguaje de primer orden.
\solution

\end{problem}

\begin{problem}
Una forma de decir que un conjunto $X$ es finito consiste en afirmar que toda funci\'on inyectiva de $X$ en $X$
es sobreyectiva. Usar cuantificaci\'on sobre todas las funciones para probar que la noci\'on ``$X$ es finito"
puede expresarse en una l\'ogica de segundo orden.
\solution

\end{problem}

\begin{problem}
Expresar la definici\'on habitual de  ``$f$ es continua"  en un lenguaje de primer orden (usando epsilones y deltas;
interpretamos que el dominio y el rango de $f$ son los reales, con todas sus funciones habituales, valor absoluto, suma, etc.).
\solution

\end{problem}

\begin{problem}
Expresar la definici\'on habitual de  ``$f$ es uniformemente continua"  en un lenguaje de primer orden (usando epsilones y deltas).


\solution

\end{problem}

\begin{problem}
Buscar los axiomas de la Aritm\'etica de Peano, entender lo que dicen, y entregarlos escritos.

\solution
\end{problem}




