% -*- root: ../LogicaMatematica.tex -*-
\section{Hoja 4}

Asumimos gen\'erica e informalmente la no trivialidad. Al escribir f\'ormulas, se permiten
las abreviaciones razonables (por ejemplo, las que he usado en el primer problema).

\begin{problem}[1]
Forma normal disyuntiva: una fbf est\'a en forma normal
disyuntiva si se expresa como $p_1 \vee \dots \vee p_n$,
donde $n\ge1$,  cada $p_i$ es de la forma
$q_{i, 1} \wedge \dots \wedge q_{i,n_i}$, con $n_i\ge1$,
y cada $q_{i, j}$ es o bien at\'omica o la negaci\'on de una f\'ormula at\'omica.
Por ejemplo, si $p, q, $ y $r$ son \'atomos, entonces $\neg p$, $p\vee q$,
$p \wedge q \wedge r$, y  $p \vee (q \wedge r)$ est\'an en forma normal disyuntiva,
mientras que  $p \wedge (q \vee r)$ no lo est\'a.

Probar que toda fbf
es equivalente a una fbf en forma normal disyuntiva (las formas
normales conjuntivas se definen de modo an\'alogo, intercambiando conjunciones
y disyunciones, y el resultado tambi\'en es cierto para ellas).

\solution

La forma más clara de demostrar que toda FBF puede expresarse en FND es mediante la descripción del algoritmo empleado para llevar a cabo esta conversión.

El algoritmo es el siguiente:
\begin{enumerate}
\item \textbf{Eliminación de la doble implicación}

Siempre que tengamos algo de la forma $(A \leftrightarrow B)$ lo convertimos en la representación equivalente:
\[(A \to B) \y (B \to A)\]

\item \textbf{Eliminación de la implicación}

Siempre que tengamos una FBF de la forma: $(A \to B)$ lo transformamos en la representación equivalente:

\[((\neg A) \Or B)\]

\item \textbf{Desplazar los $\not$ hacia el interior}

Empleamos las leyes de Morgan para transformar FBFs de la forma $\neg(A \Or B)$ y $\neg(A \y B)$ en:
\[((\neg A) \y (\neg B)) \text{ y } ((\neg A) \Or (\neg B)) \text{ respectivamente}\]

\item \textbf{Distribuir ANDs hacia dentro sobre ORs}

Todas las FBFs de la forma $(A \y (B \Or C))$ las convertiremos en su representación equivalente:

\[(A \y B) \Or (A \y C)\]

obteniendo una FBF en FND.

\end{enumerate}

Por tanto, queda claro que dada cualquier FBF, si le aplicamos el algoritmo que acabamos de definir, queda convertida a una FBD equivalente.

\end{problem}

\begin{problem}[2]
Sea $A=\{p, q, r\}$, sea $L = A \cup \{\neg, \vee, \wedge, ), (\}$,
y sea $\sigma : A\to \{0,1\}$ una funci\'on arbitraria. Denotando tambi\'en
por $\sigma$ su extensi\'on \'unica a una valoraci\'on Booleana definida en $\operatorname{fbf}(L)$, decidir razonadamente
si existe una fbf $F$ tal que $\sigma (F) = 1$.

Decidir razonadamente
qu\'e ocurre si la cardinalidad de $A$ es 2015 en vez de 3.
\solution

Por supuesto que existe la FBF definida: $(p \Or ( \neg p ))$

Para comprobarlos basta con ver que:
\[σ(p \Or ( \neg p )) = \max\{σ(p), 1 + σ(p)\} = \left\{ \begin{array}{lcc}
            \max\{0,1\} = 1 &   si  & σ(p) = 0 \\
            \max\{1,0\} = 1 &  si  & σ(p)=1
             \end{array}
   \right.\]

El problema no cambiar al variar la cardinalidad de $A$ de modo que cuando la cardinalidad de $A$ sea 2015 seguimos teniendo la misma FBF.

\textcolor{red}{Deduzco que está mal mi razonamiento pero no se por qué. Debe haber alguna definición que no estoy entendiendo bien pero no se cual.}
\end{problem}

\begin{problem}[3]
Sea $\sigma$ una valoraci\'on Booleana, y sea $T :=\{p\in FBF(L) : \sigma(p) = 1\}$.
Demostrar que $T$ es una teor\'{\i}a completa.
\solution

Una teoría será completa si dada una FBF cualquiera $q$ se tiene que $T \vdash q$ o $T \nvdash q$

Si tenemos que $σ(q)=1$ entonces $q \in T$, por definición de $T$, y por tanto, $T \vdash q$.

Por otro lado, si $σ(q)=0$ es claro que no puede deducirse $q$ a partir de $T$. Vamos a demostrarlo.

Toda FBF que pueda deducirse de $T$ es una tautología (puesto que en clase hemos visto que $T \vdash x \implies T \vDash x$).

Por la definición de tautología vista en clase, sabemos que todo modelo de $T$ tendrá evaluación verdadera sobre la tautología. Es decir:

\[T \vdash x \implies σ(x) = 1 \ \forall x\in FBF(L)\]

con lo que queda claro que si $σ(x)=0 \implies T \nvdash x$
\end{problem}

\begin{problem}[4]
Sea $G$ un grafo infinito, tal que todos sus subgrafos finitos pueden colorearse con cuatro colores.
Decidir razonadamente si $G$ puede colorearse con cuatro colores.
\solution

%La solución a este ejercicio puede encontrarse en el libro de la asignatura
Sea $G=(V,E)$ un grafo infinito, definimos $A=V \times \{1,2,3,4\}$ como un conjunto de átomos, donde cada átomo $(v,i)$ indica que el nodo $v$ tiene color $i$.

Así, para el grafo $G=(V,E)$, poder colorearlo con 4 colores implica que el siguiente conjunto $\sum \subset \text{ Prop} A$ tiene un modelo:
\[\sum = \{(v,1)\Or (v,2) \Or (v,3) \Or (v,4) \ : \ v \in V\} \cup \{\neg((v,i)\y(v,j)) \ : v \in V, \ 1 \leq i < j \leq 4\}\atop \cup \{\neg((v,i)\y (w,i)) \ : \ (v,w) \in E,  1 \leq i \leq 4\}\]

Puesto que todo subgrafo finito de $G=(V,E)$ es coloreable con 4 colores, tenemos que todo subconjunto finito de $\sum$ tiene un modelo y, por compacidad, $\sum$ tiene un modelo.

\end{problem}

\begin{problem}[5]
Sea $X$ un conjunto parcialmente ordenado. Decidir razonadamente si el orden parcial puede
extenderse a un orden lineal, o total.
\solution

%Solución obtenida de: http://math.stackexchange.com/questions/271003/every-partial-order-can-be-extended-to-a-linear-ordering

Sea $(P,R)$ la representación de un conjunto (P) parcialmente ordenado (dada la relación R).

Vamos a denominar como $C$ el conjunto de todos los ordenes parciales $(P,\leq)$ que extienden $R$, es decir, $R \subset \leq$.

Podemos ver que los elementos de $C$ satisfacen el lema de Zorn respecto a la relación $\subset$ puesto que si tenemos una cadena:
\[\leq_1 \subset \leq_2 \subset ... \subset \leq_n\]
entonces el orden parcial $(P,\leq_n)$ será una cota superior.

% ¡Prueba por Zornicación!
Por tanto, podemos aplicar el lema de Zorn y sabemos que hay un elemento maximal en $C$.

Este elemento maximal, será tal que dados dos elementos cualesquiera de $P$, véase $p_1$ y $p_2$, si hay un order parcial $\leq_i$ que los relaciona ($p_1 \leq p_2$, por ejemplo), entonces también están relacionados en el orden maximal.

Puesto que $C$ es el conjunto de todos los ordenes parciales existentes sobre los elementos de $P$, dos elementos cualesquiera de $P$ estarán relacionados por algún order parcial y por tanto estarán relacionados en el orden maximal. Es decir, el orden maximal es un orden total.
\end{problem}

\begin{problem}[6]
Expresar ``$f$ es una funci\'on inyectiva"  en un lenguaje de primer orden.
\solution

Siendo la función $f$ de la forma $\appl{f}{A}{B}$

\[\forall x,y \in A \ f(x)=f(y) \iff x=y \]

\end{problem}

\begin{problem}[7]
Expresar ``$f$ es una funci\'on sobreyectiva"  en un lenguaje de primer orden.
\solution


Siendo la función $f$ de la forma $\appl{f}{A}{B}$

\[\forall y \in B \ \exists x \in A \ f(x)=y\]

\end{problem}

\begin{problem}[8]
Una forma de decir que un conjunto $X$ es finito consiste en afirmar que toda funci\'on inyectiva de $X$ en $X$
es sobreyectiva. Usar cuantificaci\'on sobre todas las funciones para probar que la noci\'on ``$X$ es finito"
puede expresarse en una l\'ogica de segundo orden.
\solution

Podemos expresar la noción ``$X$ es finito'' como:
\[\forall f \in F \ (\forall x_1,x_2 \in X f(x_1)=f(x_2)\implies x_1=x_2) \iff (\forall x_2 \in X \exists x_1 \in X \ f(x_1)=x_2)\]

siendo $F$ el espacio de funciones $\appl{f}{X}{X}$
\end{problem}

\begin{problem}[9]
Expresar la definici\'on habitual de  ``$f$ es continua"  en un lenguaje de primer orden (usando epsilones y deltas;
interpretamos que el dominio y el rango de $f$ son los reales, con todas sus funciones habituales, valor absoluto, suma, etc.).
\solution

Una función $f$ será continua si y sólo si:
\[\forall ε \in \real \  \exists δ \in \real \ \forall x_1,x_2 \in \real \  \norm{x_1,x_2} < δ \implies \norm{f(x_1),f(x_2)} < ε \]

\doneby{Edu}
\[\forall x_0 \forall \epsilon (\epsilon > 0 \y (\exists \delta (\delta > 0 \y (\forall x ((\norm{x,x_0} < \delta) \implies (\norm{f(x),f(x_0)} < \epsilon) ))))) \]

\end{problem}

\begin{problem}[10]
Expresar la definici\'on habitual de  ``$f$ es uniformemente continua"  en un lenguaje de primer orden (usando epsilones y deltas).


\solution

Siendo la función $f$ de la forma $\appl{f}{A}{B}$

\[\forall ε \in \real \ \exists δ \in \real \ (ε>0)\y(δ>0) \forall x_1,x_2 \in A D(x_1,x_2,δ) \implies D(f(x_1),f(x_2),ε)\]

Siendo el predicado $D$:
\[D(a,b,d) = \norm{a-b} < d\]
es decir, tomará el valor 1 cuando la distancia entre los puntos $a$, $b$ sea menor que $d$ y 0 en caso contrario.

\doneby{Edu}
\[\forall \epsilon (\epsilon > 0 \y (\exists \delta (\delta > 0 \y (\forall x_1 \forall x_2 ((\norm{x_1,x_2} < \delta) \implies (\norm{f(x_1),f(x_2)} < \epsilon) ))))) \]

\end{problem}

\begin{problem}[11]
Buscar los axiomas de la Aritm\'etica de Peano, entender lo que dicen, y entregarlos escritos.

\solution

Los 5 Axiomas de Peano son:
\begin{enumerate}
\item Existe un número natural 1. En otros términos,1 está en N, el conjunto de los números naturales.
\item Todo número natural n tiene un sucesor n* (este axioma es usado para definir posteriormente la suma).
\item 1 no es el sucesor de algún número natural.
\item Si hay dos números naturales n y m con el mismo sucesor, entonces n y m son el mismo número natural.
\item Si 1 pertenece a un conjunto K de números naturales, y dado un elemento cualquiera n, el sucesor n* también pertenece al conjunto K, entonces el conjunto K debe ser el conjunto de todos los números naturales.2 Este último axioma es el principio de inducción matemática.
\end{enumerate}
\end{problem}




