\section{Hoja 6}

Asumimos gen\'erica e informalmente la no trivialidad. Al escribir f\'ormulas, se permiten
las abreviaciones razonables.

\begin{problem}
\ppart
Tomando $\exists$ como abreviaci\'on de $\neg \forall \neg$, probar que el axioma de cuantificaci\'on para $\forall$
implica la siguiente vesi\'on para $\exists$:  si $t$ puede sustituir a $x$ en $\varphi$, entonces $\varphi (t|x) \to \exists x \varphi(x)  $.

\ppart Del mismo modo, tomando $\forall$  como abreviaci\'on de $\neg \exists\neg$, probar que el axioma de cuantificaci\'on para $\exists$,
``si $t$ puede sustituir a $x$ en $\varphi$, entonces $\varphi (t|x) \to \exists x \varphi(x)  $", implica el axioma para $\forall$.
\solution

\doneby{Pedro}

\spart

Recordemos que el axioma de cuantificación para $\forall$ nos dice que:
\[\forall x \varphi(x) \to \varphi(t|x)\]

Por definición de la implicación podemos ver que este axioma es equivalente a:
\[\neg \varphi(t|x) \to \neg \forall x \varphi(x)\]

Si tomamos $\exists$ como abreviación de $\neg \forall \neg$ podemos escribir:
\[\neg \varphi(t|x) \to \exists x \neg \varphi(x)\]

Por último, podemos simplemente redefinir la fórmula $\varphi$ para que incluya la negación por si misma con lo que obtenemos el resultado buscado.

\spart

El axioma de cuantificación para $\exists$ es:
\[\varphi(t|x) \to \exists x \varphi(x)\]

Basándonos de nuevo en la definición de la implicación tenemos
\[\neg \exists x \varphi(x) \to \neg \varphi(t|x)\]

Tomando la abreviación del enunciado llegamos a:
\[\forall x \neg \varphi(x) \to \neg \varphi(t|x)\]

Nuevamente podemos redefinir la función $\varphi$ con lo que obtenemos el resultado deseado
\end{problem}

\begin{problem} Probar que la regla de generalizaci\'on para  $\forall$ es equivalente a la siguiente regla de generalizaci\'on para
$\exists$: si $x$ no aparece libre en $\varphi$, de $\psi \to \varphi$ deducimos $(\exists x \psi) \to \varphi$.

\solution
\doneby{Pedro}

Recordamos que la regla de generalización de $\forall$ nos dice que, siendo $x$ una variable libre en $\psi$:
\[\psi \to \varphi \implies \psi \to \forall x \varphi\]

Por definición de la implciación tenemos que la regla anterior es equivalente a:
\[\neg \varphi \to \neg \psi \implies (\neg \forall x \varphi \to \neg \psi)\]

Reemplazando $\neg \forall \neg$ por $\exists$ tenemos:
\[\neg \varphi \to \neg \psi  \implies (\exists x \neg \varphi \to \neg \psi)\]

Renombrando $\neg \psi =a $ y $\neg \varphi =b$ llegamos a:
\[b \to a \implies (\exists x b) \to a \]

\end{problem}

\begin{problem} Usamos $\phi \models \psi$ como abreviaci\'on de  $\{\phi\} \models \psi$.  Probar o refutar:

\ppart $\exists x (\phi \vee \psi) \models (\exists x  \phi )\vee  (\exists x \psi)$.

\ppart  $(\exists x  \phi )\vee  (\exists x \psi) \models  \exists x (\phi \vee \psi) $.

\ppart $\exists x (\phi \wedge \psi) \models (\exists x  \phi )\wedge  (\exists x \psi)$.

\ppart  $(\exists x  \phi ) \wedge  (\exists x \psi) \models  \exists x (\phi  \wedge \psi) $.

\ppart $\exists x (\phi \to \psi) \models (\exists x  \phi )\to  (\exists x \psi)$.

\ppart  $(\exists x  \phi )\to  (\exists x \psi) \models  \exists x (\phi \to \psi) $.

\ppart $\exists x (\phi \to \psi) \models (\forall x  \phi ) \to  (\exists x \psi)$.

\ppart  $(\forall x  \phi ) \to  (\exists x \psi) \models  \exists x (\phi  \to \psi) $.


\ppart  $ \phi  \models  \forall x \phi$.

\ppart $\forall x (\phi \to \psi) \models (\forall x  \phi )\to  (\forall x \psi)$.

\ppart  $(\forall x  \phi )\to  (\forall x \psi) \models  \forall x (\phi \to \psi) $.

\ppart $\exists x (\phi \to \psi) \models (\forall x  \phi ) \to  (\exists x \psi)$.

\ppart  $(\exists x  \phi ) \to  (\forall x \psi) \models  \forall x (\phi  \to \psi) $.

\ppart $\forall x (\phi \to \psi) \models (\exists x  \phi )\to  (\exists x \psi)$.


\solution

\spart

Vamos a demostrar que

\[\underbrace{\exists x (\phi \vee \psi)}_{Σ} \models \underbrace{(\exists x  \phi )\vee  (\exists x \psi)}_{σ}\]

Sea $a$ cualquier modelo de Σ, si tomamos $b\in A$ tenemos que
\[a\models (\phi(b) \Or \psi(b)) \implies a \models (\exists x  \phi )\vee  (\exists x \psi) \implies a \models σ\]

Puesto que para todo modelo $a$ de Σ tenemos $a \models σ$, podemos concluir que
\[Σ \models σ\]

\spart

Vamos a demostrar que:
\[\underbrace{(\exists x  \phi )\vee  (\exists x \psi)}_{Σ} \models  \underbrace{\exists x (\phi \vee \psi)}_{σ}\]
Sea $a$ cualquier modelo de Σ y sean $b,c \in A$ tenemos que:
\[a \models (\phi(b) \Or \psi(c)) \implies a \models \exists x (\phi \Or \psi) \implies a \models σ\]

Puesto que para todo modelo $a$ de Σ tenemos $a \models σ$, podemos concluir que
\[Σ \models σ\]

\spart

Vamos a demostrar que

\[\underbrace{\exists x (\phi \y \psi)}_{Σ} \models \underbrace{(\exists x  \phi )\y  (\exists x \psi)}_{σ}\]

Sea $a$ cualquier modelo de Σ, si tomamos $b\in A$ tenemos que
\[a\models (\phi(b) \y \psi(b)) \implies a \models (\exists x  \phi )\y  (\exists x \psi) \implies a \models σ\]

Puesto que para todo modelo $a$ de Σ tenemos $a \models σ$, podemos concluir que
\[Σ \models σ\]


\spart

En esta ocasión vamos a ver que no es cierta la FBF:
\[\underbrace{(\exists x  \phi )\y  (\exists x \psi)}_{Σ} \models  \underbrace{\exists x (\phi \y \psi)}_{σ}\]

Sea $a$ cualquier modelo de Σ y sean $b,c \in A$ tenemos que:
\[a \models (\phi(b) \y \psi(c)) \]

Si no es posible encontrar dos valores como los indicados que satisfagan $b=c$, no podremos concluir $\exists x (\phi \y \psi)$.

Por ejemplo:
\[\begin{array}{l}
\phi(a) = \text{ a es par}\\
\psi(a) = \text{a es impar}
\end{array} \implies Σ = T \y σ = \perp \]

Con lo que cualquier FBF es modelo de Σ, pues es una tautología, pero σ nunca toma el valor verdadero.

\spart

Vamos a demostrar que
\[\underbrace{\exists x (\phi \to \psi)}_{Σ} \models \underbrace{(\exists x  \phi )\to  (\exists x \psi)}_{σ}\]

Sea $a$ cualquier modelo de Σ y sea $b \in A$ tenemos que:
\[a \models (\phi(b) \to \phi(b)) \implies a \models ((\exists x\phi) \to (\exists x \psi) \implies a \models σ\]

Puesto que para todo modelo $a$ de Σ tenemos $a \models σ$, podemos concluir que
\[Σ \models σ\]

\newpage

\spart
Vamos a demostrar que
\[\underbrace{(\exists x  \phi )\to  (\exists x \psi)}_{Σ} \models  \underbrace{\exists x (\phi \to \psi)}_{σ}\]

Sea $a$ cualquier modelo de Σ, existen $b,c \in A$ tales que:
\[a \models \phi(b) \to \psi(c) \implies a \models \phi(c) \to \psi(c) \implies a\models \exists x (\phi \to \psi) \implies a \models σ\]

Puesto que para todo modelo $a$ de Σ tenemos $a \models σ$, podemos concluir que
\[Σ \models σ\]

\spart

Vamos a demostrar que
\[\underbrace{\exists x (\phi \to \psi)}_{Σ} \models \underbrace{(\forall x  \phi ) \to  (\exists x \psi)}_{σ}\]

Sea $a$ cualquier modelo de Σ, existe $b \in A$ tal que:
\[a \models \phi(b) \to \psi(b) \implies a \models \phi \to \psi(b) \implies a\models  (\forall x \phi) \to (\exists x \psi) \implies a \models σ\]

Puesto que para todo modelo $a$ de Σ tenemos $a \models σ$, podemos concluir que
\[Σ \models σ\]

\textbf{Solución del profesor:}
Si $a\vDash (\exists x(\varphi\implies \psi))$, entonces $$\exists a\in A \tq a\vDash \varphi (\underline{a}|x)\to\psi(\underline{a}|x)$$

Esto es equivalente a que o bien $a\vDash \neg\varphi(\underline{a})$ o bien $a\vDash \psi(\underline{a})$.

Si $a\vDash \psi(\underline{a})$, entonces $a\vDash (\forall x \varphi(x)\to \exists x\psi(x))$

Si $a\vDash \neg \varphi(\underline{a})$, entonces $a\vDash (\neg \forall x\varphi(x))$

Por tanto, $a\vDash (\forall x\varphi(x)\to\exists x \psi(x))$
\spart

Vamos a demostrar que
\[\underbrace{(\forall x  \phi ) \to  (\exists x \psi)}_{Σ} \models \underbrace{ \exists x (\phi  \to \psi)}_{σ} \]

Sea $a$ calquier modelo de $Σ$, tomamos $b\in A$ tal que:
\[a \models \phi \to \psi(b) \implies a \models \phi(b) \to \psi (b) \implies a \models \exists x (\phi \to \psi) \implies a \models σ\]

Puesto que para todo modelo $a$ de Σ tenemos $a \models σ$, podemos concluir que
\[Σ \models σ\]

\spart

Vamos a demostrar que
\[\underbrace{ \phi }_{Σ} \models \underbrace{ \forall x \phi}_{σ}\]

Sea $a$ cualquier modelo de $Σ$ tenemos que
\[a \models \phi \implies a \models \forall x \phi \implies a \models σ\]
Puesto que para todo modelo $a$ de Σ tenemos $a \models σ$, podemos concluir que
\[Σ \models σ\]

\spart

Vamos a demostrar que
\[\underbrace{\forall x (\phi \to \psi)}_{Σ} \models \underbrace{(\forall x  \phi )\to  (\forall x \psi)}_{σ}\]

La única forma de que no sea cierto σ es tener una L-estructura en la que
\[\forall x \phi(x) \ \ \y \exists y \ \neg\psi(y)\]

Sea $a$ cualquier modelo de Σ tenemos
\[a \models \forall x (\phi \to \psi)\]
Por tanto
\[a \models \forall x \phi \implies a \models \forall x \psi \implies a \nvDash \exists y \psi\]
Por tanto no puede ocurrir que σ no sea cierta.

\spart

En esta ocasión vamos a ver que la FBF
\[\underbrace{(\forall x  \phi )\to  (\forall x \psi)}_{Σ} \models \underbrace{ \forall x (\phi \to \psi)}_{σ} \]
es falsa.

Por ejemplo:
\[\begin{array}{l}
\phi(a) = \text{a es par}\\
\psi(a) = \text{a es múltiplo de 4}
\end{array}\]

Tomamos un modelo $a$ de Σ que contenga en $A$ los valores: $\{2,3,4\}$.

Es sencillo ver que una L-estructura como la mencionada es un modelo de Σ pues $\forall x \phi $ tiene valor falso, lo que hace que Σ sea verdadero.

Sin embargo, en estas condiciones es sencillo ver que σ es falso puesto que el caso concreto del 2 cumple que es par pero no es múltiplo de 4, es decir,
\[\phi(2)=\top \y \psi(2) = \perp \implies σ = \perp \implies a \nvDash σ\]

\newpage

\spart

Vamos a demostrar que
\[\underbrace{\exists x (\phi \to \psi)}_{Σ} \models \underbrace{(\forall x  \phi ) \to  (\exists x \psi)}_{σ}\]

Sea $a$ cualquier modelo de Σ y sea $b \in A$ tenemos dos opciones:
\begin{enumerate}
\item $\phi(b) = \top$

\[a \models \phi(b) \to \psi(b) \implies a \models \psi(b) \implies a \models \phi \to \psi(b) \implies a \models σ\]
\item $\phi(b) = \perp$
En este caso $\forall x \phi = \perp$ por lo que nos encontramos con que σ es una tautología y de manera trivial
\[a \models σ\]
\end{enumerate}

Puesto que para todo modelo $a$ de Σ tenemos $a \models σ$, podemos concluir que
\[Σ \models σ\]

\spart

Vamos a demostrar que

\[\underbrace{(\exists x  \phi ) \to  (\forall x \psi)}_{Σ} \models \underbrace{ \forall x (\phi  \to \psi)}_{σ} \]

Sea $a$ cualquier modelo de Σ y sea $b \in A$, tenemos que
\[a \models \phi(b) \to \forall x \psi \]

Aquí tenemos dos opciones:
\begin{enumerate}
\item $\exists b \tq \phi(b)=\top$
\[a \models \phi(b) \to \forall x \psi \implies a \models \forall x \psi \implies a \models \forall x (\phi \to \psi) \implies a \models σ\]
\item $\neg \exists b \tq \phi(b) = \top$

En este caso tendremos que $\phi$ siempre evalúa a falso y por tanto σ siempre es verdadero.

Por tanto
\[a \models σ\]
\end{enumerate}

Puesto que para todo modelo $a$ de Σ tenemos $a \models σ$, podemos concluir que
\[Σ \models σ\]

\spart

Vamos a demostrar que
\[\underbrace{\forall x (\phi \to \psi)}_{Σ} \models \underbrace{(\exists x  \phi )\to  (\exists x \psi)}_{σ}\]

Sea $a$ cualquier modelo de Σ tenemos dos opciones
\begin{enumerate}
\item $\exists b \tq \phi(b)=\top$
\[a \models \phi(b) \to \psi (b) \implies a \models (\exists x \phi) \to (\exists x \psi) \implies a \models σ\]
\item $\neg \exists b \tq \phi(b) = \top$

En este caso tendremos que $\phi$ siempre evalúa a falso y por tanto σ siempre es verdadero.

Por tanto
\[a \models σ\]
\end{enumerate}
\end{problem}

\begin{problem}   Usamos $\phi \vdash \psi$ como abreviaci\'on de  $\{\phi\} \vdash \psi$.  Probar (sin usar completitud):

\ppart $ \phi  \vdash   \forall x \phi$.

\ppart  $\exists x \forall y \phi (x,y)  \vdash   \forall y \exists x \phi (x,y)$.

\ppart  $ \vdash  \exists x \forall y \phi (x,y)  \to \forall y \exists x \phi (x,y)$.

\ppart   $  \forall x \forall y \phi (x,y) \vdash   \forall y \forall x \phi (x,y)$.

\ppart   $  \forall x \forall y \phi (x,y) \vdash   \forall y \phi (y,y)$.

\solution

Abreviaturas: Axioma de Cuantificación (versión 1)(CA1), Axioma de Cuantificación (versión 2) (CA2) Regla de Generalización (versión 1) (G1), Regla de Generalización (versión 2) y  Clausura Universal (UC).


CA1: Si el término t puede sustituir al término x en $\phi$ entonces: $\forall x \phi(x) \to \phi(t|x)$.

CA2: Si el término t puede sustituir al térmio x en $\phi$ entonces: $\phi(t|x) \to \exists x \phi(x)$. (Demostrado en el Ejercicio 1)

G1: Si x no aparece libre en $\phi$ entonces: de $\phi \to \psi$ deducimos $\phi \to \forall x \psi$

G2: Si x no aparece libre en $\phi$ entonces: de $\phi \to \psi$ deducimos $\exists x \phi \to \psi$ (Demostrado en el Ejercicio 2)


\spart
Tenemos $\phi$. No sabemos si tiene variables libres o ligadas, luego no podemos aplicar las reglas de generalización ni los axiomas de cuantificación. Así que sólo se me ocurre la siguiente prueba:

\begin{enumerate}
	\item Tomamos la clausura universal de $\phi$: $\forall x \phi$
\end{enumerate}

\textbf{Solución del profesor:}
\begin{enumerate}
	\item $\varphi$ premisa
	\item Usamos la siguiente tautología:
	$$\varphi\to(\neg \forall x \varphi \to \varphi)$$
	\item Obtenemos $\neg\forall x\varphi\to \varphi$ usando los pasos anteriores y modus ponens.
	\item Como no tenemos variables libres, podemos usar el axioma de generalización:
	$$\neg \forall x \varphi \to \forall x \varphi$$
	\item Tenemos la tautología $(\neg \forall x\varphi \to \forall x\varphi)\to \forall x\varphi$.
	\item Por modus ponens
	$$\forall x \varphi $$
\end{enumerate}

\spart

\begin{enumerate}
\item Por CA1: $\exists x \forall y \phi (x,y) \to \exists x \phi (x,y)$
\item Por UC: $ \exists x \phi (x,y) \to \forall y \exists x \phi (x,y)$

%Obs: No podemos usar G1 por que y aparece libre en $ \exists x \phi (x,y)$

\end{enumerate}

\spart

\begin{enumerate}
	\item Instanciando el axioma $p \to p$:  $\forall y \phi (x,y)  \to \forall y \phi (x,y)$
	\item Por CA2: $(\forall y \phi (x,y)  \to \forall y \phi (x,y))\to (\exists x \forall y \phi (x,y)  \to  \forall y \phi(x,y))$
	\item  Por CA2:  $ \forall y \phi(x,y) \to \forall y \exists x \phi(x,y) $
	\item Por tanto:  $ \exists x \forall y \phi (x,y) \to \forall y \exists x \phi (x,y) $
\end{enumerate}

\textbf{Solución del profesor:}
Por el teorema de la deducción, basta probar
$$\exists x  \forall y \varphi(x,y)\vdash \forall y \exists x \varphi(x,y)$$

Por el apartado \textbf{a}, basta probar
$$\vdash \exists x \varphi(x,y)$$ y aplicar la regla de generalización.

Basta probar que
$\{\exists x \forall y \varphi(x,y), \not\exists x \varphi(x,y)\}$ es inconsistente. En realidad, hay que probar que la teoría
$\{\exists x \forall y\varphi (x,y), \forall y \not \exists x \varphi(x,y) \}$ es inconsistente.
\begin{obs}
	Si $\Sigma\vdash \perp$, entonces $\{\exists x \forall y\varphi (x,y), \forall y \not \exists x \varphi(x,y) \}\vdash \perp$
\end{obs}

Basta ver que $\{\exists x\forall y \varphi, \neg \varphi \}\vdash \perp$.
Esto es
$$\neg\exists x \varphi \equiv \forall x(\neg \varphi)$$
Por cuantificación y porque $x$ puede reemplazar a x:
$$\forall x (\neg\varphi(x,y))\implies \neg \varphi(x,y)$$
Luego deducimos, por modus ponens:
$$\neg \varphi(x,y)$$
Ahora, utilizando cuantificación:
$$\neg \varphi(x,y)\implies \exists y (\neg \varphi(x,y))$$
Luego basta ver que
$$\{\exists x \forall y \varphi, \exists y \neg \varphi(x,y) \}\vdash \perp $$
Ahora podemos utilizar generalización y basta ver que
$$\Sigma = \{\exists x \forall y \varphi , \forall x \exists y \neg \varphi(x,y)\}\vdash \perp$$
Equivalentemente $$\{\exists x  \forall y \varphi, \neg(\exists x \forall y \varphi) \}\vdash \perp$$

\begin{obs}
	Sea $\alpha$ una fórmula, entonces $\{\alpha, \neg \alpha \}\vdash \perp$
	\begin{itemize}
		\item Tomamos como premisa $\alpha$.
		\item $\alpha \implies(\neg \alpha \implies \perp)$ es una tautología.
		\item Con modus ponens $\neg \alpha \implies \perp$.
		\item Usamos la premisa $\neg \alpha$.
		\item Obtenemos, con modus ponens, $\perp$.
	\end{itemize}
\end{obs}

Ahora convertimos la siguiente expresión:
$$\{\exists x  \forall y \varphi, \neg(\exists x \forall y \varphi) \}\vdash \perp$$
El profesor se ha quedado un poco atascado.

\spart

\begin{enumerate}
	\item Por CA1: $ \forall x \forall y \phi (x,y) \to \forall x \phi (x,y)$
	\item Por G1: $ (\forall x \forall y \phi (x,y) \to \forall x \phi (x,y)) \to (\forall x \forall y \phi (x,y) \to \forall y \forall x \phi (x,y))$
\end{enumerate}

\textbf{Solución del profesor:}
\begin{enumerate}
	\item $\forall x\forall y \varphi(x,y)$ premisa.
	\item Utilizamos el axioma de cuantificación ya que $x$ puede sustituir a $x$
	$$\forall x\forall y \varphi(x,y) \to \forall y\varphi (x,y)$$
	\item Utilizando modus ponens
	$$\forall y \varphi(x,y)$$
	\item Dado que $y$ puede sustituir a $y$ utilizamos el axioma de cuantificación de nuevo
	$$\forall y\varphi(x,y) \to \varphi(x,y)$$
	\item Utilizando modus ponens obtenemos $\varphi(x,y)$.
	\item Por el apartado a) de este ejercicio: $\forall x \varphi(x,y)$.
	\item $\forall y \forall x \varphi(x,y)$.
\end{enumerate}

\spart


\begin{enumerate}
	\item Por el apartado d: $ \forall x \forall y \phi (x,y) \to \forall y \forall x \phi (x,y)$
	\item Por CA1: $\forall y \forall x \phi (x,y) \to \forall y (\phi (y|x, y))$ (esto es, $\forall y~\phi (y,y)$)
\end{enumerate}

\end{problem}