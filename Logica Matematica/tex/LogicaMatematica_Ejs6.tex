\section{Hoja 6}

Asumimos gen\'erica e informalmente la no trivialidad. Al escribir f\'ormulas, se permiten
las abreviaciones razonables. 

\begin{problem} 
\ppart
Tomando $\exists$ como abreviaci\'on de $\neg \forall \neg$, probar que el axioma de cuantificaci\'on para $\forall$
implica la siguiente vesi\'on para $\exists$:  si $t$ puede sustituir a $x$ en $\varphi$, entonces $\varphi (t|x) \to \exists x \varphi(x)  $.

\ppart Del mismo modo, tomando $\forall$  como abreviaci\'on de $\neg \exists\neg$, probar que el axioma de cuantificaci\'on para $\exists$,
``si $t$ puede sustituir a $x$ en $\varphi$, entonces $\varphi (t|x) \to \exists x \varphi(x)  $", implica el axioma para $\forall$.
\solution

\doneby{Pedro}

\spart

Recordemos que el axioma de cuantificación para $\forall$ nos dice que:
\[\forall x \varphi(x) \to \varphi(t|x)\]

Por definición de la implicación podemos ver que este axioma es equivalente a:
\[\neg \varphi(t|x) \to \neg \forall x \varphi(x)\]

Si tomamos $\exists$ como abreviación de $\neg \forall \neg$ podemos escribir:
\[\neg \varphi(t|x) \to \exists x \neg \varphi(x)\]

Por último, podemos simplemente redefinir la fórmula $\varphi$ para que incluya la negación por si misma con lo que obtenemos el resultado buscado.

\spart

El axioma de cuantificación para $\exists$ es:
\[\varphi(t|x) \to \exists x \varphi(x)\]

Basándonos de nuevo en la definición de la implicación tenemos
\[\neg \exists x \varphi(x) \to \neg \varphi(t|x)\]

Tomando la abreviación del enunciado llegamos a:
\[\forall x \neg \varphi(x) \to \neg \varphi(t|x)\]

Nuevamente podemos redefinir la función $\varphi$ con lo que obtenemos el resultado deseado
\end{problem}

\begin{problem} Probar que la regla de generalizaci\'on para  $\forall$ es equivalente a la siguiente regla de generalizaci\'on para
$\exists$: si $x$ no aparece libre en $\varphi$, de $\psi \to \varphi$ deducimos $(\exists x \psi) \to \varphi$.

\solution
\doneby{Pedro}

Recordamos que la regla de generalización de $\forall$ nos dice que, siendo $x$ una variable libre en $\psi$:
\[\psi \to \varphi \implies \psi \to \forall x \varphi\]

Por definición de la implciación tenemos que la regla anterior es equivalente a:
\[\neg \varphi \to \neg \psi \implies (\neg \forall x \varphi \to \neg \psi)\]

Reemplazando $\neg \forall \neg$ por $\exists$ tenemos:
\[\neg \varphi \to \neg \psi  \implies (\exists x \neg \varphi \to \neg \psi)\]

Renombrando $\neg \psi =a $ y $\neg \varphi =b$ llegamos a:
\[b \to a \implies (\exists x b) \to a \]

\end{problem}

\begin{problem} Usamos $\phi \models \psi$ como abreviaci\'on de  $\{\phi\} \models \psi$.  Probar o refutar:

\ppart $\exists x (\phi \vee \psi) \models (\exists x  \phi )\vee  (\exists x \psi)$.

\ppart  $(\exists x  \phi )\vee  (\exists x \psi) \models  \exists x (\phi \vee \psi) $.

\ppart $\exists x (\phi \wedge \psi) \models (\exists x  \phi )\wedge  (\exists x \psi)$.

\ppart  $(\exists x  \phi ) \wedge  (\exists x \psi) \models  \exists x (\phi  \wedge \psi) $.

\ppart $\exists x (\phi \to \psi) \models (\exists x  \phi )\to  (\exists x \psi)$.

\ppart  $(\exists x  \phi )\to  (\exists x \psi) \models  \exists x (\phi \to \psi) $.

\ppart $\exists x (\phi \to \psi) \models (\forall x  \phi ) \to  (\exists x \psi)$.

\ppart  $(\forall x  \phi ) \to  (\exists x \psi) \models  \exists x (\phi  \to \psi) $.


\ppart  $ \phi  \models  \forall x \phi$.

\ppart $\forall x (\phi \to \psi) \models (\forall x  \phi )\to  (\forall x \psi)$.

\ppart  $(\forall x  \phi )\to  (\forall x \psi) \models  \forall x (\phi \to \psi) $.

\ppart $\exists x (\phi \to \psi) \models (\forall x  \phi ) \to  (\exists x \psi)$.

\ppart  $(\exists x  \phi ) \to  (\forall x \psi) \models  \forall x (\phi  \to \psi) $.

\ppart $\forall x (\phi \to \psi) \models (\exists x  \phi )\to  (\exists x \psi)$.



\solution
\end{problem}

\begin{problem}  Usamos $\phi \vdash \psi$ como abreviaci\'on de  $\{\phi\} \vdash \psi$.  Probar (sin usar completitud):

\ppart $ \phi  \vdash   \forall x \phi$.

\ppart  $\exists x \forall y \phi (x,y)  \vdash   \forall y \exists x \phi (x,y)$.

\ppart  $ \vdash  \exists x \forall y \phi (x,y)  \to \forall y \exists x \phi (x,y)$.

\ppart   $  \forall x \forall y \phi (x,y) \vdash   \forall y \forall x \phi (x,y)$.

\ppart   $  \forall x \forall y \phi (x,y) \vdash   \forall y \phi (y,y)$.

\solution

\end{problem}