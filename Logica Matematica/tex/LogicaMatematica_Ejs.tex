% -*- root: ../LogicaMatematica.tex -*-
\chapter{Problemas resueltos}
\section{Hoja 1}

	
	Asumimos gen\'erica e informalmente la no trivialidad. Por ejemplo,
	cuando hablamos de conjuntos, lenguajes, etc., suponemos que no  son vac\'{\i}os (salvo que 
	expl\'{\i}citamente
	se diga lo contrario),  cuando hablamos de f\'ormulas suponemos que est\'an bien formadas, etcetera.
	En caso de duda, consultar con el intructor. 
	
	\begin{problem}[1]
	Aprenderse el lema de Zorn de memoria. Comentario: el lema de Zorn es equivalente al Axioma de
	Elecci\'on.
	\solution
	\end{problem}
	
	\begin{problem}[2]
	Sea $A$ un conjunto finito y parcialmente ordenado. Denotamos por $R$ el correspondiente
	orden parcial (usamos $R$ porque un orden parcial es una relaci\'on binaria, o con ``aridad" 2). Demostrar que
	$A$ contiene un elemento maximal. Demostrar que $R$ puede extenderse a un orden total o lineal,
	es decir, existe un orden total  $R^\prime$ tal que $R\subset R^\prime$ (orden lineal o total
	significa que para todo par
	$x, y\in A$, o bien $(x,y)\in R^\prime$, \'o $(y,x)\in R^\prime$).
	\solution
	\end{problem}
	
	\begin{problem}[3]
	El lema de legibilidad \'unica nos dice que las f\'ormulas bien formadas no son ambiguas, pueden
	leerse de un \'unico modo. Reescribir las f\'ormulas que aparecen a continuaci\'on en notaci\'on polaca, usando
	la notaci\'on est\'andar, y las que aparecen en notaci\'on est\'andar, en polaca.
	
	a) $\vee \ \neg \ \to  \  p \  q \ \leftrightarrow \ r \  p$, 
	$\to\ \to \ \wedge \ \to \ p \ q \ \vee q \ r \ \vee p \ r \ \neg \ \vee q \ s$.
	
	b)  $(q \to  (p \to q ))$, $( (p \leftrightarrow q ) \to ((\neg \  q)  \vee  r))$.
	
	Hallar el \'arbol de descomposici\'on de las f\'ormulas en b). 
	\solution
	\end{problem}
	
	\begin{problem}
	 Sea $p$ una f\'ormula (bien formada). Probar que el n\'umero de par\'entesis izquierdos en
	$p$ es igual al  n\'umero de par\'entesis derechos. Sugerencia, usar inducci\'on. 
	\solution
	\end{problem}