% -*- root: ../LogicaMatematica.tex -*-
\section{Hoja 3}

\begin{problem}[1]
Sea $\sigma$ una valoraci\'on Booleana. Expresar $\sigma(p\vee q)$ y
$\sigma(p\wedge q)$ en t\'erminos de
$\sigma(p)$ y $\sigma(q)$,
usando la suma y el producto en $\mathbb{Z}_2$.
\solution

Consideramos el 1 como verdadero y el 0 como falso.

\spart
\[\sigma(p \Or q) = \neg\sigma(\neg p \y \neg q) = 1+(1+\sigma(p))\cdot (1+\sigma(q))\]

Alternativamente:

\[σ(p\Or q) = σ(p) + σ(q) + σ(p)·σ(q)\]

\spart
\[\sigma(p \y q ) = \sigma(p) \cdot \sigma(q)\]
\end{problem}

\begin{problem}[2]
Demostrar que los siguientes conjuntos de conectivas son completos:
$\{\neg, \vee\}$,  $\{\neg, \wedge\}$,  $\{\neg, \to\}$, $\{\to, \perp\}$.
\solution

Para ver que son completos debemos demostrar que podemos expresar el resto de conectivas a partir de las dadas. Vamos a ello:

\begin{itemize}
\item \textbf{$\{\neg, \vee\}$}
\[a \to b \equiv (\neg a) \Or b\]
\begin{center}
\begin{tabular}{|c|c|c|>{\columncolor[rgb]{0.88,1,1}}c|c|c|}
\hline
a & b & a $\to$ b & $\leftrightarrow$ & $\neg$ a & $\Or$ b \\
\hline
0 & 0 & 1 & 1 & 1 & 1 \\
\hline
0 & 1 & 1 & 1 & 1 & 1 \\
\hline
1 & 0 & 0 & 1 & 0 & 0 \\
\hline
1 & 1 & 1 & 1 & 0 & 1 \\
\hline
\end{tabular}
\end{center}
\[a \y b \equiv \neg( \neg a \Or \neg b)\]
\begin{center}
\begin{tabular}{|c|c|c|>{\columncolor[rgb]{0.88,1,1}}c|c|c|c|c|}
\hline
a & b & a $\y$ b & $\leftrightarrow$ & $\neg ($ & $\neg$ a & $\Or$ & $\neg b)$ \\
\hline
0 & 0 & 0 & 1 & 0 & 1 & 1 & 1 \\
\hline
0 & 1 & 0 & 1 & 0 & 1 & 1 & 0 \\
\hline
1 & 0 & 0 & 1 & 0 & 0 & 1 & 1 \\
\hline
1 & 1 & 1 & 1 & 1 & 0 & 0 & 0 \\
\hline
\end{tabular}
\end{center}
\[\top \equiv a \Or (\neg a)\]
\begin{center}
\begin{tabular}{|c|c|>{\columncolor[rgb]{0.88,1,1}}c|c|c|}
\hline
a & $\top$ & $\leftrightarrow$ & a $\Or$ & $(\neg a)$ \\
\hline
0 & 1 & 1 & 1 & 1 \\
\hline
1 & 1 & 1 & 1 & 0 \\
\hline
\end{tabular}
\end{center}
\[\perp \equiv \neg (a \Or (\neg a))\]
\begin{center}
\begin{tabular}{|c|c|>{\columncolor[rgb]{0.88,1,1}}c|c|c|c|}
\hline
a & $\perp$ & $\leftrightarrow$ & $\neg ($ & a $\Or$ & $(\neg a))$ \\
\hline
0 & 0 & 1 & 1 & 1 & 1\\
\hline
1 & 0 & 1 & 1 & 1 & 0\\
\hline
\end{tabular}
\end{center}
\[a \leftrightarrow b \equiv (a \to b) \y (b \to a) \equiv (\neg a \Or b) \y (\neg b \Or a) \equiv \neg \left( \neg (\neg a \Or b) \Or \neg (\neg b \Or a)\right)\]
\begin{center}
\begin{tabular}{|c|c|c|>{\columncolor[rgb]{0.88,1,1}}c|c|c|c|c|c|c|c|c|}
\hline
a & b & $(a \leftrightarrow b)$ & $\leftrightarrow$ & $\neg ($ & $\neg ($ & $\neg a$ & $\Or$ $b )$ & $\Or$ & $\neg ($ & $\neg b$ & $\Or$ $a ))$ \\
\hline
0 & 0 & 1 & 1 & 1 & 0 & 1 & 1 & 0 & 0 & 1 & 1 \\
\hline
0 & 1 & 0 & 1 & 0 & 0 & 1 & 1 & 1 & 1 & 0 & 0 \\
\hline
1 & 0 & 0 & 1 & 0 & 1 & 0 & 0 & 1 & 0 & 1 & 1 \\
\hline
1 & 1 & 1 & 1 & 1 & 0 & 0 & 1 & 0 & 0 & 0 & 1 \\
\hline
\end{tabular}
\end{center}

\item \textbf{$\{\neg, \wedge\}$}

\[a \Or b \equiv \neg( (\neg a) \y (\neg b))\]
\begin{center}
\begin{tabular}{|c|c|c|>{\columncolor[rgb]{0.88,1,1}}c|c|c|c|c|}
\hline
a & b & a $\Or$ b & $\leftrightarrow$ & $\neg( $ & $(\neg a)$ & $\y$ & $(\neg b))$ \\
\hline
0 & 0 & 0 & 1 & 0 & 1 & 1 & 1 \\
\hline
0 & 1 & 1 & 1 & 1 & 1 & 0 & 0 \\
\hline
1 & 0 & 1 & 1 & 1 & 0 & 0 & 1 \\
\hline
1 & 1 & 1 & 1 & 1 & 0 & 0 & 0 \\
\hline
\end{tabular}
\end{center}

\[a \to b \equiv \neg a \Or b \equiv \neg (a \y (\neg b))\]
\begin{center}
\begin{tabular}{|c|c|c|>{\columncolor[rgb]{0.88,1,1}}c|c|c|c|}
\hline
a & b & a $\to$ b & $\leftrightarrow$ & $\neg$ & (a $\y$ & $(\neg b))$ \\
\hline
0 & 0 & 1 & 1 & 1 & 0 & 1 \\
\hline
0 & 1 & 1 & 1 & 1 & 0 & 0 \\
\hline
1 & 0 & 0 & 1 & 0 & 1 & 1 \\
\hline
1 & 1 & 1 & 1 & 1 & 0 & 0 \\
\hline
\end{tabular}
\end{center}

\[\top \equiv \neg (a \y (\neg a))\]
\begin{center}
\begin{tabular}{|c|c|>{\columncolor[rgb]{0.88,1,1}}c|c|c|c|}
\hline
a & $\top$ & $\leftrightarrow$ & $\neg$ & (a $\y$ & $(\neg a))$ \\
\hline
0 & 1 & 1 & 1 & 0 & 1 \\
\hline
1 & 1 & 1 & 1 & 0 & 0 \\
\hline
\end{tabular}
\end{center}

\[\perp \equiv a \y (\neg a)\]
\begin{center}
\begin{tabular}{|c|c|>{\columncolor[rgb]{0.88,1,1}}c|c|c|}
\hline
a & $\perp$ & $\leftrightarrow$ & a $\y$ & $(\neg a)$ \\
\hline
0 & 0 & 1 & 0 & 1 \\
\hline
1 & 0 & 1 & 0 & 0 \\
\hline
\end{tabular}
\end{center}

\[a \leftrightarrow b \equiv (a \y b) \Or (\neg a \y \neg b) \equiv \neg \left((\neg(a \y b)) \y (\neg ((\neg a) \y (\neg b))) \right)\]
\begin{center}
\begin{tabular}{|c|c|c|>{\columncolor[rgb]{0.88,1,1}}c|c|c|c|c|c|c|c|c|}
\hline
a & b & $a \leftrightarrow b$ & $\leftrightarrow$ & $\neg ($ & $(\neg$ & $(a \y b))$ & $\y$ & $(\neg$ & $((\neg a)$ & $\y$ & $(\neg b))))$ \\
\hline
0 & 0 & 1 &  1  & 1 & 1 & 0 & 0 & 0 & 1 & 1 & 1 \\
\hline
0 & 1 & 0 &  1  & 0 & 1 & 0 & 1 & 1 & 1 & 0 & 0 \\
\hline
1 & 0 & 0 &  1  & 0 & 1 & 0 & 1 & 1 & 0 & 0 & 1 \\
\hline
1 & 1 & 1 &  1  & 1 & 0 & 1 & 0 & 1 & 0 & 0 & 0 \\
\hline
\end{tabular}
\end{center}

\item \textbf{$\{\neg, \to\}$}

\[a \Or b \equiv (\neg a) \to b\]
\begin{center}
\begin{tabular}{|c|c|c|>{\columncolor[rgb]{0.88,1,1}}c|c|c|}
\hline
a & b & $a \Or b$ & $\leftrightarrow$ & $(\neg a)$ & $\to b$ \\
\hline
0 & 0 & 0 &  1  & 1 & 0 \\
\hline
0 & 1 & 1 &  1  & 1 & 1 \\
\hline
1 & 0 & 1 &  1  & 0 & 1 \\
\hline
1 & 1 & 1 &  1  & 0 & 1 \\
\hline
\end{tabular}
\end{center}

\[a \y b \equiv \neg ( \neg a \Or \neg b) \equiv \neg ( a \to (\neg b))\]
\begin{center}
\begin{tabular}{|c|c|c|>{\columncolor[rgb]{0.88,1,1}}c|c|c|c|}
\hline
a & b & $a \y b$ & $\leftrightarrow$ & $\neg$ & $(a \to$ & $(\neg b))$ \\
\hline
0 & 0 & 0 &  1  & 0 & 1 & 1 \\
\hline
0 & 1 & 0 &  1  & 0 & 1 & 0 \\
\hline
1 & 0 & 0 &  1  & 0 & 1 & 1 \\
\hline
1 & 1 & 1 &  1  & 1 & 0 & 0 \\
\hline
\end{tabular}
\end{center}

\[\top \equiv a \to a\]
\begin{center}
\begin{tabular}{|c|c|>{\columncolor[rgb]{0.88,1,1}}c|c|}
\hline
a & $\top$ & $\leftrightarrow$ & $a \to a$ \\
\hline
0 & 1 &  1  & 1 \\
\hline
1 & 1 &  1  & 1 \\
\hline
\end{tabular}
\end{center}

\[\perp \equiv \neg (a \to a)\]
\begin{center}
\begin{tabular}{|c|c|>{\columncolor[rgb]{0.88,1,1}}c|c|c|}
\hline
a & $\perp$ & $\leftrightarrow$ & $\neg$ & $(a \to a)$ \\
\hline
0 & 0 &  1  & 0 & 1 \\
\hline
1 & 0 &  1  & 0 & 1 \\
\hline
\end{tabular}
\end{center}

\[a \leftrightarrow b \equiv (a \to b) \y (b \to a) \equiv \neg \left( (a \to b) \to (\neg ( b \to a ))\right)\]
\begin{center}
\begin{tabular}{|c|c|c|>{\columncolor[rgb]{0.88,1,1}}c|c|c|c|c|c|}
\hline
a & b & $a \leftrightarrow b$ & $\leftrightarrow$ & $\neg$ & $((a \to b)$ & $\to$ & $(\neg$ & $(b \to a )))$ \\
\hline
0 & 0 & 1 &  1  & 1 & 1 & 0 & 0 & 1 \\
\hline
0 & 1 & 0 &  1  & 0 & 1 & 1 & 1 & 0 \\
\hline
1 & 0 & 0 &  1  & 0 & 0 & 1 & 0 & 1 \\
\hline
1 & 1 & 1 &  1  & 1 & 1 & 0 & 0 & 1 \\
\hline
\end{tabular}
\end{center}

\item \textbf{$\{\to, \perp\}$}

\[\top \equiv \perp \to \perp\]
\begin{center}
\begin{tabular}{|c|>{\columncolor[rgb]{0.88,1,1}}c|c|c|c|}
\hline
$\top$ & $\leftrightarrow$ & $\perp$ & $\to$ & $\perp$ \\
\hline
1 &  1  & 0 & 1 & 0 \\
\hline
\end{tabular}
\end{center}

\[\neg a \equiv a \to \perp\]
\begin{center}
\begin{tabular}{|c|c|>{\columncolor[rgb]{0.88,1,1}}c|c|c|}
\hline
a & $\neg$ a & $\leftrightarrow$ & $a \to$ & $\perp$ \\
\hline
0 & 1 &  1  & 1 & 0 \\
\hline
1 & 0 &  1  & 0 & 0 \\
\hline
\end{tabular}
\end{center}

\[a \Or b \equiv \neg a \to b \equiv (a \to \perp) \to b\]
\begin{center}
\begin{tabular}{|c|c|c|>{\columncolor[rgb]{0.88,1,1}}c|c|c|c|}
\hline
a & b & $a \Or b$ & $\leftrightarrow$ & $(a \to$ & $\perp)$ & $\to b$ \\
\hline
0 & 0 & 0 &  1  & 1 & 0 & 0 \\
\hline
0 & 1 & 1 &  1  & 1 & 0 & 1 \\
\hline
1 & 0 & 1 &  1  & 0 & 0 & 1 \\
\hline
1 & 1 & 1 &  1  & 0 & 0 & 1 \\
\hline
\end{tabular}
\end{center}

\[a \y b \equiv \neg ( \neg a \Or \neg b) \equiv \neg (a \to \neg b) \equiv (a \to (b \to \perp))\to \perp\]
\begin{center}
\begin{tabular}{|c|c|c|>{\columncolor[rgb]{0.88,1,1}}c|c|c|c|c|}
\hline
a & b & $a \y b$ & $\leftrightarrow$ & $(a \to$ & $(b \to$ & $\perp))$ & $\to \perp$ \\
\hline
0 & 0 & 0 &  1  & 1 & 1 & 0 & 0 \\
\hline
0 & 1 & 0 &  1  & 1 & 0 & 0 & 0 \\
\hline
1 & 0 & 0 &  1  & 1 & 1 & 0 & 0 \\
\hline
1 & 1 & 1 &  1  & 0 & 0 & 0 & 1 \\
\hline
\end{tabular}
\end{center}

\[a \leftrightarrow b \equiv (a \to b ) \y (b \to a ) \equiv ((a \to b )\to ((b \to a ) \to \perp)) \to \perp\]
\begin{center}
\begin{tabular}{|c|c|c|>{\columncolor[rgb]{0.88,1,1}}c|c|c|c|c|c|c|c|}
\hline
a & b & $a \leftrightarrow b$ & $\leftrightarrow$ & $((a \to b )$ & $\to$ & $((b \to a )$ & $\to$ & $\perp))$ & $\to$ & $\perp$ \\
\hline
0 & 0 & 1 &  1  & 1 & 0 & 1 & 0 & 0 & 1 & 0 \\
\hline
0 & 1 & 0 &  1  & 1 & 1 & 0 & 1 & 0 & 0 & 0 \\
\hline
1 & 0 & 0 &  1  & 0 & 1 & 1 & 0 & 0 & 0 & 0 \\
\hline
1 & 1 & 1 &  1  & 1 & 0 & 1 & 0 & 0 & 1 & 0 \\
\hline
\end{tabular}
\end{center}

\end{itemize}

Las equivalencias expresadas a lo largo del ejercicio se han obtenido forzando el cumplimiento de las tablas de verdad.

\end{problem}


\begin{problem}[3]
Demostrar que el  conjunto de conectivas $\{\vee, \wedge\}$ no es completo.
Sugerencia: usar monoton\'{\i}a.
\solution

Sea $C$ el conjunto de FBFs constituidas a partir del lenguaje $\set{a, \y, \Or}$. Este conjunto puede describirse como:
\begin{enumerate}
\item $a \in C$
\item Si $α,β \in C \implies (α \y β) \ , (α \Or β) \in C$
\item $C$ es el menor conjunto de FBFs que satisface las propiedades $1$ y $2$.
\end{enumerate}

Cuando tengamos $a=\top$ tendremos que las fórmulas $(a \Or a)$, $(a \y a)$ son también verdaderas y, por definición del conjunto $C$, cualquier fórmula $ε \in C$ tendrá valor verdadero cuando $a$ sea verdadera, ya que las conectivas $\Or$ y $\y$ siempre evalúan a verdadero cuando a ambos lados tienen valores verdaderos.

Por tanto, es imposible que la negación, $\neg a$, que toma valor $\perp$ cuando $a=\top$, se deduzca de ese conjunto y, por tanto, el conjunto no es completo.

\end{problem}



\begin{problem}[4]
Leer con cuidado el Lema 2.1.2 p. 16, y demostrar la columna derecha de los apartados
4-7 (distributividad, absorci\'on, De Morgan, tercero exclu\'{\i}do).
\solution

El lema simplemente nos da una serie de 8 equivalencias. Procedemos a demostrar las 4 construyendo las tablas de verdad asociadas y comprobando que la doble implicación siempre tiene valor verdadero.
\begin{itemize}
\item \textbf{$(p ∧ (q ∨ r)) ↔ (p ∧ q) ∨ (p ∧ r)$}
\begin{center}
\begin{tabular}{|c|c|c|c|c|>{\columncolor[rgb]{0.88,1,1}}c|c|c|c|c|c|c|c|}
\hline
(p & $∧$ & (q & $∨$ & r)) & $↔$ & (p & $∧$ & q) & $∨$ & (p & $∧$ & r) \\
\hline
0 & 0 & 0 & 0 & 0 & 1 & 0 & 0 & 0 & 0 & 0 & 0 & 0\\
\hline
0 & 0 & 0 & 1 & 1 & 1 & 0 & 0 & 0 & 0 & 0 & 0 & 1\\
\hline
0 & 0 & 1 & 1 & 0 & 1 & 0 & 0 & 1 & 0 & 0 & 0 & 0\\
\hline
0 & 0 & 1 & 1 & 1 & 1 & 0 & 0 & 1 & 0 & 0 & 0 & 1\\
\hline
1 & 0 & 0 & 0 & 0 & 1 & 1 & 0 & 0 & 0 & 1 & 0 & 0\\
\hline
1 & 1 & 0 & 1 & 1 & 1 & 1 & 0 & 0 & 1 & 1 & 1 & 1\\
\hline
1 & 1 & 1 & 1 & 0 & 1 & 1 & 1 & 1 & 1 & 1 & 0 & 0\\
\hline
1 & 1 & 1 & 1 & 1 & 1 & 1 & 1 & 1 & 1 & 1 & 1 & 1\\
\hline
\end{tabular}
\end{center}
\item \textbf{$(p ∧ (p ∨ q)) ↔ p$}

\begin{center}
\begin{tabular}{|c|c|c|c|c|>{\columncolor[rgb]{0.88,1,1}}c|c|}
\hline
(p & $∧$ & (p & $∨$ & q)) & $↔$ & p \\
\hline
 0 & 0 & 0 & 0 & 0 & 1 & 0\\
\hline
 0 & 0 & 0 & 1 & 1 & 1 & 0\\
\hline
 1 & 1 & 1 & 1 & 0 & 1 & 1\\
\hline
 1 & 1 & 1 & 1 & 1 & 1 & 1\\
\hline
\end{tabular}
\end{center}
\item \textbf{$(¬(p ∧ q)) ↔ (¬p ∨ ¬q)$}
\begin{center}
\begin{tabular}{|c|c|c|c|>{\columncolor[rgb]{0.88,1,1}}c|c|c|c|}
\hline
($¬$ & (p & $∧$ & q)) & $↔$ & $¬$p & ∨ & $¬$q)\\
\hline
1 & 0 & 0 & 0 & 1 & 1 & 1 & 1\\
\hline
1 & 0 & 0 & 1 & 1 & 1 & 1 & 0\\
\hline
1 & 1 & 0 & 0 & 1 & 0 & 1 & 1\\
\hline
0 & 1 & 1 & 1 & 1 & 0 & 0 & 0\\
\hline
\end{tabular}
\end{center}
\item \textbf{$(p ∧ ¬p) ↔ ⊥$}
\begin{center}
\begin{tabular}{|c|c|c|>{\columncolor[rgb]{0.88,1,1}}c|c|}
\hline
(p & $∧$ & $¬$p) & $↔$ & $⊥$\\
\hline
0 & 0 & 1 & 1 & 0 \\
\hline
1 & 0 & 0 & 1 & 0 \\
\hline
\end{tabular}
\end{center}
\end{itemize}

\end{problem}



\begin{problem}[5]
Demostrar el Lema 2.1.3 p. 17.
\solution

El lema que queremos demostrar dice:
\begin{lemma}
Sea $\sum \subset Prop(A)$ y sean $p,q \in Prop(A)$, entonces
\begin{enumerate}
\item $\sum \vDash p \y q \iff \sum \vDash p \text{ y } \sum \vDash q$
\item $\sum \vDash p \implies \sum \vDash p \Or q$
\item $\sum \cup \{p\} \vDash q \iff \sum \vDash p \to q$
\item Si $\sum \vDash p$ y $\sum \vDash p\to q$, entonces $\sum \vDash q$ (Modus Ponens)
\end{enumerate}
\end{lemma}

Vamos a demostrarlo
\begin{enumerate}
\item

Por definición decimos que $\sum \vDash p$ si para todo modelo $σ$ de $\sum$ tenemos que $\sigma(p) = 1$.

Si tenemos que para todo modelo $σ$ de $\sum$, $σ(p \y q) = \top$ y sabemos que
\[σ(p \y q ) = σ(p) \y σ(q) = \top\]
queda claro que $σ(p)=\top$ y $σ(q)=\top$.

Evidentemente esto también es cierto en sentido contrario, si tenemos $σ(p)=\top$ y $σ(q)=\top$ es claro que
\[σ(p \y q ) = σ(p) \y σ(q) = \top\]

\item
Puesto que tenemos
\[σ(p\Or q ) = σ(p) \Or σ(q)\]

Es claro que si $σ(p)$ = $\top$ para todo modelo $σ$ de $\sum$, entonces tendremos $σ(p\Or q) = σ(p) \Or σ(q) = \top$ para todo modelo $σ$ de $\sum$.

\item

Tomemos un modelo cualquier σ de $\sum$. En caso de que la interpretación σ sea también un modelo en $\sum \cup \{p\}$, es decir, además de ser modelo de $\sum$ tenemos $σ(p)=\top$, tendremos tambíen $σ(q)=\top$.

Es decir, dado $σ$ tenemos que si $σ(p) = \top$ entonces $σ(q)=\top$ y que no sabremos nada de $σ(q)$ en caso contrario.

Esta situación coincide exactamente con $σ(p \to q) = σ(p) \to σ(q)$.

Por otro lado, si tenemos $σ(p \to q) = σ(p) \to σ(q) = \top$ para todo modelo σ de $\sum$ entonces tenemos que siempre que si sigma es un modelo de $\sum$ y de $p$, tendremos $σ(q)$, es decir: $\sum \cup \{p\} \vDash q$

\item

	Tenemos que para todo modelo σ de $\sum$, $σ(p) = σ(p \to q ) = \top$. Es decir, tenemos que
	\[σ(p)=σ(\neg p \Or q) = \neg σ(p) \Or σ(q)\]

	Sabiendo que $σ(p)=\top$ llegamos a
	\[\top = \perp \Or σ(q) \implies σ(q) = \top\]

\end{enumerate}

\end{problem}



\begin{problem}[6]
Comprobar que los tres axiomas en el ejercico 1 de la hoja 2 son tautolog\'{\i}as.
\solution

Para demostrar que estos axiomas son tautologías debemos comprobar que para toda valoración de las proposiciones se cumplen las FBF. Vamos a verlo
\begin{enumerate}
\item  $ (p\to (q\to p))$

\begin{center}
\begin{tabular}{|c|>{\columncolor[rgb]{0.88,1,1}}c|c|c|c|}
\hline
($p$ & $\to$ & (q & $\to$ & p))\\
\hline
1 & 1 & 1 & 1 & 1\\
 %\hline
1 & 1 & 0 & 1 & 1\\
\hline
0 & 1 & 1 & 0 & 0\\
\hline
0 & 1 & 0 & 1 & 0\\
\hline
\end{tabular}
\end{center}

%$$\neg p \Or (\neg q \Or p) \o$$


\item $( (p\to (q\to r)) \to  ( (p\to q) \to (p\to r)))$



%\begin{align*}
%	(\neg(\neg p\Or\neg q\Or r) \Or ((\neg (\neg p\Or q)\Or (\neg p \Or r)))\\
%	(p\y q\y\neg r) \Or ((p\y\neg q)\Or(\neg p\Or r))\\
%	(p\y q\y \neg r)\Or((p\Or\neg p \Or r)\y(\neg q\Or\neg p \Or r))\\
%	(p\y q\y \neg r)\Or(\top \y(\neg q\Or\neg p\Or r))\\
%	(p\y q\y \neg r)\Or(\neg q\Or \neg p \Or r)\\
%	(p\Or q\Or \neg p \Or r)\y(q\Or\neg q \Or \neg p \Or r)\y(\neg r\Or\neg q \Or \neg p \Or r)\\
%	\top\y \top\y \top\\
%	\top
%\end{align*}
%
%\item $(( (\neg p) \to (\neg q)) \to  ( ((\neg p) \to q) \to p))$
%
%\begin{align*}
%	\neg (p\Or \neg q) \Or (\neg (p\Or q) \Or p)\\
%	(\neg p \y q) \Or ((\neg p \y \neg q)\Or p)\\
%	(\neg p\y q) \Or ((\neg p \Or p)\y(p\Or \neg q))\\
%	(\neg p\y q) \Or (\top\y(p\Or \neg q))\\
%	(\neg p \y q) \Or p \Or \neg q\\
%	(\neg p \Or p) \y (q\Or p) \Or \neg q\\
%	\top \y (q\Or p) \Or \neg q\\
%	q\Or p \Or \neg q\\
%	\top
%\end{align*}
%
\begin{center}
\begin{tabular}{|c|c|c|c|c|>{\columncolor[rgb]{0.88,1,1}}c|c|c|c|c|c|c|c|}
\hline
((p & $\to$ & (q & $\to$ & r)) & $\to$ & ((p & $\to$ & q) & $\to$ & (p & $\to$ & r)))\\
\hline
1 & 1 & 1 & 1 & 1 & 1 & 1 & 1 & 1 & 1 & 1 & 1 & 1\\
\hline
1 & 0 & 1 & 0 & 0 & 1 & 1 & 1 & 1 & 0 & 1 & 0 & 0\\
\hline
1 & 1 & 0 & 1 & 1 & 1 & 1 & 0 & 0 & 1 & 1 & 1 & 1\\
\hline
1 & 1 & 0 & 1 & 0 & 1 & 1 & 0 & 0 & 1 & 1 & 0 & 0\\
\hline
0 & 1 & 1 & 1 & 1 & 1 & 0 & 1 & 1 & 1 & 0 & 1 & 1\\
\hline
0 & 1 & 1 & 0 & 0 & 1 & 0 & 1 & 1 & 1 & 0 & 1 & 0\\
\hline
0 & 1 & 0 & 1 & 1 & 1 & 0 & 1 & 0 & 1 & 0 & 1 & 1\\
\hline
0 & 1 & 0 & 1 & 0 & 1 & 0 & 1 & 0 & 1 & 0 & 1 & 0\\
\hline
\end{tabular}
\end{center}

\item $(( (\neg p) \to (\neg q)) \to  ( ((\neg p) \to q) \to p))$

\begin{center}
\begin{tabular}{|c|c|c|>{\columncolor[rgb]{0.88,1,1}}c|c|c|c|c|c|}
\hline
((($\neg$ p) & $\to$ & ($\neg$ q)) & $\to$ & ((($\neg$ p) & $\to$ & q) & $\to$ & p))\\
\hline
1 & 1 & 1 & 1 & 1 & 0 & 0 & 1 & 0\\
\hline
0 & 1 & 1 & 1 & 0 & 1 & 0 & 1 & 1\\
\hline
1 & 0 & 0 & 1 & 1 & 1 & 1 & 0 & 0\\
\hline
0 & 1 & 0 & 1 & 0 & 1 & 1 & 1 & 1\\
\hline
\end{tabular}
\end{center}

\end{enumerate}

\end{problem}


\begin{problem}[7]
Comprobar que los ocho axiomas en las p\'aginas 20-21 del libro son tautolog\'{\i}as.
\solution
Procedemos como en el ejercicio anterior
\begin{enumerate}
\item \textbf{$\top$}

El símbolo atómico $\top$ es una tautología por definición, pues siempre tiene el valor verdadero.


\item \textbf{$p → (p ∨ q)$}
\begin{center}

\begin{tabular}{|@{ }c@{ }@{ }c | c@{}@{ }c@{ }@{ }c@{ }@{}c@{}@{ }c@{ }@{ }c@{ }@{ }c@{ }@{}c@{}@{}c@{ }|}\hline
p & q & ( & p & $\rightarrow$ & ( & p & $\lor$ & q & ) & )\\
\hline
$1$ & $1$ &  & $1$ & \textcolor{blue}{$1$} &  & $1$ & $1$ & $1$ &  & \\\hline
$1$ & $0$ &  & $1$ & \textcolor{blue}{$1$} &  & $1$ & $1$ & $0$ &  & \\\hline
$0$ & $1$ &  & $0$ & \textcolor{blue}{$1$} &  & $0$ & $1$ & $1$ &  & \\\hline
$0$ & $0$ &  & $0$ & \textcolor{blue}{$1$} &  & $0$ & $0$ & $0$ &  & \\\hline
\end{tabular}

\end{center}

\item \textbf{$¬p → (¬q → ¬(p ∨ q))$}
\begin{center}

\begin{tabular}{|@{ }c@{ }@{ }c | c@{}@{ }c@{ }@{ }c@{ }@{ }c@{ }@{}c@{}@{ }c@{ }@{ }c@{ }@{ }c@{ }@{ }c@{ }@{}c@{}@{ }c@{ }@{ }c@{ }@{ }c@{ }@{}c@{}@{}c@{}@{}c@{ }|}\hline
p & q & ( & ¬ & p & $\rightarrow$ & ( & ¬ & q & $\rightarrow$ & ¬ & ( & p & $\lor$ & q & ) & ) & )\\
\hline
$1$ & $1$ &  & $0$ & $1$ & \textcolor{blue}{$1$} &  & $0$ & $1$ & $1$ & $0$ &  & $1$ & $1$ & $1$ &  &  & \\\hline
$1$ & $0$ &  & $0$ & $1$ & \textcolor{blue}{$1$} &  & $1$ & $0$ & $0$ & $0$ &  & $1$ & $1$ & $0$ &  &  & \\\hline
$0$ & $1$ &  & $1$ & $0$ & \textcolor{blue}{$1$} &  & $0$ & $1$ & $1$ & $0$ &  & $0$ & $1$ & $1$ &  &  & \\\hline
$0$ & $0$ &  & $1$ & $0$ & \textcolor{blue}{$1$} &  & $1$ & $0$ & $1$ & $1$ &  & $0$ & $0$ & $0$ &  &  & \\\hline
\end{tabular}
\end{center}

\item \textbf{$(p ∧ q) → p$}

\begin{center}
\begin{tabular}{|@{ }c@{ }@{ }c | c@{}@{}c@{}@{ }c@{ }@{ }c@{ }@{ }c@{ }@{}c@{}@{ }c@{ }@{ }c@{ }@{}c@{ }|}\hline
p & q & ( & ( & p & $\wedge$ & q & ) & $\rightarrow$ & p & )\\
\hline
$1$ & $1$ &  &  & $1$ & $1$ & $1$ &  & \textcolor{blue}{$1$} & $1$ & \\\hline
$1$ & $0$ &  &  & $1$ & $0$ & $0$ &  & \textcolor{blue}{$1$} & $1$ & \\\hline
$0$ & $1$ &  &  & $0$ & $0$ & $1$ &  & \textcolor{blue}{$1$} & $0$ & \\\hline
$0$ & $0$ &  &  & $0$ & $0$ & $0$ &  & \textcolor{blue}{$1$} & $0$ & \\\hline
\end{tabular}
\end{center}

\item \textbf{$p → (q → (p ∧ q)$}
\begin{center}
\begin{tabular}{|@{ }c@{ }@{ }c | c@{}@{ }c@{ }@{ }c@{ }@{}c@{}@{ }c@{ }@{ }c@{ }@{}c@{}@{ }c@{ }@{ }c@{ }@{ }c@{ }@{}c@{}@{}c@{}@{}c@{ }|}\hline
p & q & ( & p & $\rightarrow$ & ( & q & $\rightarrow$ & ( & p & $\wedge$ & q & ) & ) & )\\
\hline
$1$ & $1$ &  & $1$ & \textcolor{blue}{$1$} &  & $1$ & $1$ &  & $1$ & $1$ & $1$ &  &  & \\\hline
$1$ & $0$ &  & $1$ & \textcolor{blue}{$1$} &  & $0$ & $1$ &  & $1$ & $0$ & $0$ &  &  & \\\hline
$0$ & $1$ &  & $0$ & \textcolor{blue}{$1$} &  & $1$ & $0$ &  & $0$ & $0$ & $1$ &  &  & \\\hline
$0$ & $0$ &  & $0$ & \textcolor{blue}{$1$} &  & $0$ & $1$ &  & $0$ & $0$ & $0$ &  &  & \\\hline
\end{tabular}
\end{center}

\item \textbf{$(p → (q → r)) → ((p → q) → (p → r))$}
\begin{center}
\begin{tabular}{|@{ }c@{ }@{ }c@{ }@{ }c | c@{}@{}c@{}@{ }c@{ }@{ }c@{ }@{}c@{}@{ }c@{ }@{ }c@{ }@{ }c@{ }@{}c@{}@{}c@{}@{ }c@{ }@{}c@{}@{}c@{}@{ }c@{ }@{ }c@{ }@{ }c@{ }@{}c@{}@{ }c@{ }@{}c@{}@{ }c@{ }@{ }c@{ }@{ }c@{ }@{}c@{}@{}c@{}@{}c@{ }|}\hline
p & q & r & ( & ( & p & $\rightarrow$ & ( & q & $\rightarrow$ & r & ) & ) & $\rightarrow$ & ( & ( & p & $\rightarrow$ & q & ) & $\rightarrow$ & ( & p & $\rightarrow$ & r & ) & ) & )\\
\hline
$1$ & $1$ & $1$ &  &  & $1$ & $1$ &  & $1$ & $1$ & $1$ &  &  & \textcolor{blue}{$1$} &  &  & $1$ & $1$ & $1$ &  & $1$ &  & $1$ & $1$ & $1$ &  &  & \\\hline
$1$ & $1$ & $0$ &  &  & $1$ & $0$ &  & $1$ & $0$ & $0$ &  &  & \textcolor{blue}{$1$} &  &  & $1$ & $1$ & $1$ &  & $0$ &  & $1$ & $0$ & $0$ &  &  & \\\hline
$1$ & $0$ & $1$ &  &  & $1$ & $1$ &  & $0$ & $1$ & $1$ &  &  & \textcolor{blue}{$1$} &  &  & $1$ & $0$ & $0$ &  & $1$ &  & $1$ & $1$ & $1$ &  &  & \\\hline
$1$ & $0$ & $0$ &  &  & $1$ & $1$ &  & $0$ & $1$ & $0$ &  &  & \textcolor{blue}{$1$} &  &  & $1$ & $0$ & $0$ &  & $1$ &  & $1$ & $0$ & $0$ &  &  & \\\hline
$0$ & $1$ & $1$ &  &  & $0$ & $1$ &  & $1$ & $1$ & $1$ &  &  & \textcolor{blue}{$1$} &  &  & $0$ & $1$ & $1$ &  & $1$ &  & $0$ & $1$ & $1$ &  &  & \\\hline
$0$ & $1$ & $0$ &  &  & $0$ & $1$ &  & $1$ & $0$ & $0$ &  &  & \textcolor{blue}{$1$} &  &  & $0$ & $1$ & $1$ &  & $1$ &  & $0$ & $1$ & $0$ &  &  & \\\hline
$0$ & $0$ & $1$ &  &  & $0$ & $1$ &  & $0$ & $1$ & $1$ &  &  & \textcolor{blue}{$1$} &  &  & $0$ & $1$ & $0$ &  & $1$ &  & $0$ & $1$ & $1$ &  &  & \\\hline
$0$ & $0$ & $0$ &  &  & $0$ & $1$ &  & $0$ & $1$ & $0$ &  &  & \textcolor{blue}{$1$} &  &  & $0$ & $1$ & $0$ &  & $1$ &  & $0$ & $1$ & $0$ &  &  & \\\hline
\end{tabular}
\end{center}
\item \textbf{$p → (¬p → ⊥)$}
\begin{center}


\begin{tabular}{|@{ }c@{ }@{ }c | c@{}@{ }c@{ }@{ }c@{ }@{}c@{}@{ }c@{ }@{ }c@{ }@{ }c@{ }@{ }c@{ }@{}c@{}@{}c@{ }|}\hline
$\perp$ & p & ( & p & $\rightarrow$ & ( & ¬ & p & $\rightarrow$ & $\perp$ & ) & )\\
\hline
$0$ & $1$ &  & $1$ & \textcolor{blue}{$1$} &  & $0$ & $1$ & $1$ & $0$ &  & \\\hline
$0$ & $0$ &  & $0$ & \textcolor{blue}{$1$} &  & $1$ & $0$ & $0$ & $0$ &  & \\\hline
\end{tabular}

\end{center}
\item \textbf{$(¬p → ⊥) → p$}
\begin{center}

\begin{tabular}{|@{ }c@{ }@{ }c | c@{}@{}c@{}@{ }c@{ }@{ }c@{ }@{ }c@{ }@{ }c@{ }@{}c@{}@{ }c@{ }@{ }c@{ }@{}c@{ }|}\hline
$\perp$ & p & ( & ( & ¬ & p & $\rightarrow$ & $\perp$ & ) & $\rightarrow$ & p & )\\
\hline
$0$ & $1$ &  &  & $0$ & $1$ & $1$ & $0$ &  & \textcolor{blue}{$1$} & $1$ & \\\hline
$0$ & $0$ &  &  & $1$ & $0$ & $0$ & $0$ &  & \textcolor{blue}{$1$} & $0$ & \\\hline
\end{tabular}
\end{center}
\end{enumerate}
\end{problem}



\begin{problem}[8]
Reescribir  los ocho axiomas en las p\'aginas 20-21 del libro
usando s\'olo las conectivas en $\{\neg, \to\}$,  y  comprobar que dichos axiomas
se deducen de los   tres axiomas en el ejercico 1 de la hoja 2.
\solution

%Para empezar, recordemos cuáles eran los tres axiomas que menciona el enunciado:
%\begin{enumerate}
%\item  $ (p\to (q\to p))$
%
%\item $( (p\to (q\to r)) \to  ( (p\to q) \to (p\to r)))$
%
%\item $(( (\neg p) \to (\neg q)) \to  ( ((\neg p) \to q) \to p))$
%\end{enumerate}

Tal y como hemos visto en teoría, por el teorema de completitud, existe una prueba de $p$ a partir de los axiomas y modus ponens si y sólo si $p$ es una tautología.

Por tanto, comprobar que las siguientes fórmulas se deducien de los axiomas, basta con probar que son tautologías, es decir, que su tabla de verdad nos da valor verdadero para todas las interpreaciones posibles.
\begin{enumerate}
\item \textbf{$\top$}

\[\top \equiv p \to p\]

Trivialmente podemos ver que se trata de una tautología.


\item \textbf{$p → (p ∨ q)$}
\[p → (p ∨ q) \equiv p \to ((\neg p) \to q ) \equiv p \to ((\neg q) \to p)\]

La última equivalencia es una instancia del primer axioma. Para verlo basta con sustituir $q$ por $\neg q$ en ese axioma.


\item \textbf{$¬p → (¬q → ¬(p ∨ q))$}
\[¬p → (¬q → ¬(p ∨ q)) \equiv (\neg p) \to ((\neg q ) \to ( \neg (( \neg p) \to q )))\]

\begin{center}
\begin{tabular}{|c|>{\columncolor[rgb]{0.88,1,1}}c|c|c|c|c|c|c|}
\hline
$¬p $& $→ $ & $(¬q $ & $→ $& $¬ $& $(p $& $ ∨ $& $q))$\\
\hline
1 & 1 & 1 & 1 & 1 & 0 & 0 & 0 \\
\hline
1 & 1 & 0 & 1 & 0 & 0 & 1 & 1 \\
\hline
0 & 1 & 1 & 0 & 0 & 1 & 1 & 0 \\
\hline
0 & 1 & 0 & 1 & 0 & 1 & 1 & 1 \\
\hline
\end{tabular}
\end{center}

\item \textbf{$(p ∧ q) → p$}
\[(p ∧ q) → p \equiv (\neg (( \neg p) \Or (\neg q ))) \to p \equiv (\neg (p \to (\neg q))) \to p\]

\begin{center}
\begin{tabular}{|c|c|c|>{\columncolor[rgb]{0.88,1,1}}c|c|}
\hline
$(p $&$ ∧ $&$ q) $&$ → $&$ p$\\
\hline
0 & 0 & 0 & 1 & 0\\
\hline
0 & 0 & 1 & 1 & 0\\
\hline
1 & 0 & 0 & 1 & 1\\
\hline
1 & 1 & 1 & 1 & 1\\
\hline
\end{tabular}
\end{center}

\item \textbf{$p → (q → (p ∧ q))$}
\[p → (q → (p ∧ q)) \equiv p \to ( q \to (\neg ( p \to ( \neg q )))) \]

\begin{center}
\begin{tabular}{|c|>{\columncolor[rgb]{0.88,1,1}}c|c|c|c|c|c|}
\hline
$p $& $→ $ & $(q $ & $→ $& $(p $& $ \y $& $q))$\\
\hline
0 & 1 & 0 & 1 & 0 & 0 & 0 \\
\hline
0 & 1 & 1 & 0 & 0 & 0 & 1 \\
\hline
1 & 1 & 0 & 1 & 1 & 0 & 0 \\
\hline
1 & 1 & 1 & 1 & 1 & 1 & 1 \\
\hline
\end{tabular}
\end{center}

\item \textbf{$(p → (q → r)) → ((p → q) → (p → r))$}

Ya está en el formato deseado y es uno de los axiomas.

\item \textbf{$p → (¬p → ⊥)$}
\[p \to ( \neg p \to (\neg (p \to p)))\]

\begin{center}
\begin{tabular}{|c|>{\columncolor[rgb]{0.88,1,1}}c|c|c|c|}
\hline
$p $& $\to$ & $(\neg p$ & $\to $ & $\perp)$\\
\hline
0 & 1 & 1 & 0 & 0 \\
\hline
1 & 1 & 0 & 1 & 0 \\
\hline
\end{tabular}
\end{center}

\item \textbf{$(¬p → ⊥) → p$}
\[(\neg p \to (\neg (p \to p))) \to p\]

\begin{center}
\begin{tabular}{|c|c|c|>{\columncolor[rgb]{0.88,1,1}}c|c|}
\hline
$(\neg p $& $\to$ & $\perp)$ & $\to $ & $p$\\
\hline
0 & 1 & 0 & 1 & 1 \\
\hline
1 & 0 & 0 & 1 & 0 \\
\hline
\end{tabular}
\end{center}
\end{enumerate}
\end{problem}

\begin{problem}[9]
Obtener $\vdash (p\to p)$ a partir del Lema de la Deducci\'on.
\solution


El Lema de la Deducción nos dice: \[(T \cup \{p\} \vdash q) \implies (T\vdash (p\to q))\]

Tomando $T=\emptyset$ y $q=p$, tenemos $\{p\} \vdash p$ cuya prueba es trivial, ya que $p$ en este caso es un miembro de la teoría.

Por el Lema de la Deducción, concluimos $\emptyset \vdash (p\to p)$, es decir $\vdash (p\to p)$


\end{problem}

\begin{problem}[10]
Decidir razonadamente si el conjunto de conectivas $\{¬,\iff\}$ es completo.
\label{ejer::H3P10}
\solution

No es completo debido a que tanto el $\land$ como el $\lor$ generan siempre un numero impar de 0's y 1's, mientras que $\iff$ siempre genera un numero par de 0's y un número par 1's y el $\neg$ solo intercambia el numero de 0's por el número de 1's. Por tanto, no es posible solo con $\iff$ y con $\neg$ generar un número impar de 0's ó 1's, y en consecuencia, no se puede encontrar una fórmula equivalente al $\land$ o al $\lor$, es decir, $\{\neg,\iff\}$ NO es completo.

\end{problem}
