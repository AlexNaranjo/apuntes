\section{Hoja 3}

\begin{problem}[1]
Sea $\sigma$ una valoraci\'on Booleana. Expresar $\sigma(p\vee q)$ y
$\sigma(p\wedge q)$ en t\'erminos de
$\sigma(p)$ y $\sigma(q)$,
usando la suma y el producto en $\mathbb{Z}_2$.
\solution

Consideramos el 1 como verdadero y el 0 como falso.

\spart
\[\sigma(p \Or q) = \neg\sigma(\neg p \Or \neg q) = 1+(1+\sigma(p))\cdot (1+\sigma(q))\]

\spart
\[\sigma(p \y q ) = \sigma(p) \cdot \sigma(q)\]
\end{problem}

\begin{problem}[2]
Demostrar que los siguientes conjuntos de conectivas son completos:
$\{\neg, \vee\}$,  $\{\neg, \wedge\}$,  $\{\neg, \to\}$, $\{\neg, \leftrightarrow\}$,
$\{\to, \perp\}$.
\solution

Para ver que son completos debemos demostrar que podemos expresar el resto de conectivas a partir de las dadas. Vamos a ello:

\begin{itemize}
\item \textbf{$\{\neg, \vee\}$}
\[a \to b \equiv \neg a \Or b\]
\[a \y b \equiv \neg( \neg a \Or \neg b)\]
\[\top \equiv a \Or \neg a\]
\[\perp \equiv \neg (a \Or \neg a)\]
\[a \leftrightarrow b \equiv (a \to b) \y (b \to a) \equiv (\neg a \Or b) \y (\neg b \Or a) \equiv \neg \left( \neg (\neg a \Or b) \Or \neg (\neg b \Or a)\right)\]

\item \textbf{$\{\neg, \wedge\}$}
\[a \Or b \equiv \neg( \neg a \y \neg b)\]
\[a \to b \equiv \neg a \Or b \equiv \neg (a \y \neg b)\]
\[\top \equiv \neg (a \y \neg a)\]
\[\perp \equiv a \y \neg a\]
\[a \leftrightarrow b \equiv (a \y b) \Or (\neg a \y \neg b) \equiv \neg \left(\neg(a \y b) \y \neg (\neg a \y \neg b) \right)\]

\item \textbf{$\{\neg, \to\}$}
\[a \Or b \equiv \neg a \to b\]
\[a \y b \equiv \neg ( \neg a \Or \neg b) \equiv \neg ( a \to \neg b)\]
\[\top \equiv a \to a\]
\[\perp \equiv \neg (a \to  a)\]
\[a \leftrightarrow b \equiv (a \to b) \y (b \to a) \equiv \neg \left( (a \to b) \to \neg ( b \to a )\right)\]

\item \textbf{$\{\neg, \leftrightarrow\}$}
\[\top \equiv a \leftrightarrow a\]
\[\perp \equiv a \leftrightarrow \neg a\]
\[a \y b \equiv \]
\[a \Or b \equiv\]
\[a \to b \equiv\]

\item \textbf{$\{\to, \perp\}$}
\[\top \equiv \]
\[\leftrightarrow \equiv \]
\[a \y b \equiv \]
\[a \Or b \equiv\]
\[a \to b \equiv\]
\end{itemize}

\end{problem}


\begin{problem}[3]
Demostrar que el  conjunto de conectivas $\{\vee, \wedge\}$ no es completo.
Sugerencia: usar monoton\'{\i}a.
\solution

\end{problem}



\begin{problem}[4]
Leer con cuidado el Lema 2.1.2 p. 16, y demostrar la columna derecha de los apartados
4-7 (distributividad, absorci\'on, De Morgan, tercero exclu\'{\i}do).
\solution

El lema simplemente nos da una serie de 8 equivalencias. Procedemos a demostrar las 4 pedidas:
\begin{itemize}
\item \textbf{$(p ∧ (q ∨ r)) ↔ (p ∧ q) ∨ (p ∧ r)$}
\item \textbf{$(p ∧ (p ∨ q)) ↔ p$}
\item \textbf{$(¬(p ∧ q)) ↔ (¬p ∨ ¬q)$}
\item \textbf{$(p ∧ ¬p) ↔ ⊥$}

\end{itemize}
\end{problem}



\begin{problem}[5]
Demostrar el Lema 2.1.3 p. 17.
\solution
\textcolor{red}{Viene demostrado en el libro. Que alguien lo copie}
\end{problem}



\begin{problem}[6]
Comprobar que los tres axiomas en el ejercico 1 de la hoja 2 son tautolog\'{\i}as.
\solution

Para demostrar que estos axiomas son tautologías debemos comprobar que para todo modelo de la teoría se cumplen las FBF. Vamos a verlo
\begin{enumerate}
\item  $ (p\to (q\to p))$
\item $( (p\to (q\to r)) \to  ( (p\to q) \to (p\to r)))$
\item $(( (\neg p) \to (\neg q)) \to  ( ((\neg p) \to q) \to p))$
\end{enumerate}

\end{problem}


\begin{problem}[7]
Comprobar que los ocho axiomas en las p\'aginas 20-21 del libro son tautolog\'{\i}as.
\solution
Procedemos como en el ejercicio anterior
\begin{enumerate}
\item \textbf{$\top$}
\item \textbf{$p → (p ∨ q); \  p → (q ∨ p)$}

\item \textbf{$¬p → ¬q → ¬(p ∨ q)$}

\item \textbf{$(p ∧ q) → p; \ (p ∧ q) → q$}


\item \textbf{$p → q → (p ∧ q)$}

\item \textbf{$p → (q → r) → (p → q) → (p → r)$}
\item \textbf{$p → (¬p → ⊥)$}
\item \textbf{$(¬p → ⊥) → p$}

\end{enumerate}
\end{problem}



\begin{problem}[8]
Reescribir  los ocho axiomas en las p\'aginas 20-21 del libro
usando s\'olo las conectivas en $\{\neg, \to\}$,  y  comprobar que dichos axiomas
se deducen de los   tres axiomas en el ejercico 1 de la hoja 2.
\solution

\begin{enumerate}
\item \textbf{$\top$}
\item \textbf{$p → (p ∨ q); \  p → (q ∨ p)$}

\item \textbf{$¬p → ¬q → ¬(p ∨ q)$}

\item \textbf{$(p ∧ q) → p; \ (p ∧ q) → q$}


\item \textbf{$p → q → (p ∧ q)$}

\item \textbf{$p → (q → r) → (p → q) → (p → r)$}
\item \textbf{$p → (¬p → ⊥)$}
\item \textbf{$(¬p → ⊥) → p$}

\end{enumerate}
\end{problem}

\begin{problem}[9]
Obtener $\vdash (p\to p)$ a partir del Lema de la Deducci\'on.
\solution

\end{problem}

