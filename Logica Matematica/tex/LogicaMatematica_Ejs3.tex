\section{Hoja 3}

\begin{problem}[1]
Sea $\sigma$ una valoraci\'on Booleana. Expresar $\sigma(p\vee q)$ y
$\sigma(p\wedge q)$ en t\'erminos de
$\sigma(p)$ y $\sigma(q)$,
usando la suma y el producto en $\mathbb{Z}_2$.
\solution

Consideramos el 1 como verdadero y el 0 como falso.

\spart
\[\sigma(p \Or q) = \neg\sigma(\neg p \y \neg q) = 1+(1+\sigma(p))\cdot (1+\sigma(q))\]

\spart
\[\sigma(p \y q ) = \sigma(p) \cdot \sigma(q)\]
\end{problem}

\begin{problem}[2]
Demostrar que los siguientes conjuntos de conectivas son completos:
$\{\neg, \vee\}$,  $\{\neg, \wedge\}$,  $\{\neg, \to\}$, $\{\neg, \leftrightarrow\}$,
$\{\to, \perp\}$.
\solution

Para ver que son completos debemos demostrar que podemos expresar el resto de conectivas a partir de las dadas. Vamos a ello:

\begin{itemize}
\item \textbf{$\{\neg, \vee\}$}
\[a \to b \equiv \neg a \Or b\]
\[a \y b \equiv \neg( \neg a \Or \neg b)\]
\[\top \equiv a \Or \neg a\]
\[\perp \equiv \neg (a \Or \neg a)\]
\[a \leftrightarrow b \equiv (a \to b) \y (b \to a) \equiv (\neg a \Or b) \y (\neg b \Or a) \equiv \neg \left( \neg (\neg a \Or b) \Or \neg (\neg b \Or a)\right)\]

\item \textbf{$\{\neg, \wedge\}$}
\[a \Or b \equiv \neg( \neg a \y \neg b)\]
\[a \to b \equiv \neg a \Or b \equiv \neg (a \y \neg b)\]
\[\top \equiv \neg (a \y \neg a)\]
\[\perp \equiv a \y \neg a\]
\[a \leftrightarrow b \equiv (a \y b) \Or (\neg a \y \neg b) \equiv \neg \left(\neg(a \y b) \y \neg (\neg a \y \neg b) \right)\]

\item \textbf{$\{\neg, \to\}$}
\[a \Or b \equiv \neg a \to b\]
\[a \y b \equiv \neg ( \neg a \Or \neg b) \equiv \neg ( a \to \neg b)\]
\[\top \equiv a \to a\]
\[\perp \equiv \neg (a \to  a)\]
\[a \leftrightarrow b \equiv (a \to b) \y (b \to a) \equiv \neg \left( (a \to b) \to \neg ( b \to a )\right)\]

\item \textbf{$\{\neg, \leftrightarrow\}$}
\[\top \equiv a \leftrightarrow a\]
\[\perp \equiv a \leftrightarrow \neg a\]
\[a \y b \equiv  \]
\[a \Or b \equiv\]
\[a \to b \equiv\]

\item \textbf{$\{\to, \perp\}$}
\[\top \equiv \perp \to \perp\]
\[\neg a \equiv a \to \perp\]
\[a \Or b \equiv \neg a \to b \equiv (a \to \perp) \to b\]
\[a \y b \equiv \neg ( \neg a \Or \neg b) \equiv \neg (a \to \neg b) \equiv (a \to (b \to \perp))\to \perp\]
\[a \leftrightarrow b \equiv (a \to b ) \y (b \to a ) \equiv ((a \to b )\to ((b \to a ) \to \perp)) \to \perp\]
\end{itemize}

\end{problem}


\begin{problem}[3]
Demostrar que el  conjunto de conectivas $\{\vee, \wedge\}$ no es completo.
Sugerencia: usar monoton\'{\i}a.
\solution

\end{problem}



\begin{problem}[4]
Leer con cuidado el Lema 2.1.2 p. 16, y demostrar la columna derecha de los apartados
4-7 (distributividad, absorci\'on, De Morgan, tercero exclu\'{\i}do).
\solution

El lema simplemente nos da una serie de 8 equivalencias. Procedemos a demostrar las 4 construyendo las tablas de verdad asociadas y comprobando que la doble implicación siempre tiene valor verdadero.
\begin{itemize}
\item \textbf{$(p ∧ (q ∨ r)) ↔ (p ∧ q) ∨ (p ∧ r)$}
\begin{center}
\begin{tabular}{|c|c|c|c|c|>{\columncolor[rgb]{0.88,1,1}}c|c|c|c|c|c|c|c|}
\hline
(p & $∧$ & (q & $∨$ & r)) & $↔$ & (p & $∧$ & q) & $∨$ & (p & $∧$ & r) \\
\hline
0 & 0 & 0 & 0 & 0 & 1 & 0 & 0 & 0 & 0 & 0 & 0 & 0\\
\hline
0 & 0 & 0 & 1 & 1 & 1 & 0 & 0 & 0 & 0 & 0 & 0 & 1\\
\hline
0 & 0 & 1 & 1 & 0 & 1 & 0 & 0 & 1 & 0 & 0 & 0 & 0\\
\hline
0 & 0 & 1 & 1 & 1 & 1 & 0 & 0 & 1 & 0 & 0 & 0 & 1\\
\hline
1 & 0 & 0 & 0 & 0 & 1 & 1 & 0 & 0 & 0 & 1 & 0 & 0\\
\hline
1 & 1 & 0 & 1 & 1 & 1 & 1 & 0 & 0 & 1 & 1 & 1 & 1\\
\hline
1 & 1 & 1 & 1 & 0 & 1 & 1 & 1 & 1 & 1 & 1 & 0 & 0\\
\hline
1 & 1 & 1 & 1 & 1 & 1 & 1 & 1 & 1 & 1 & 1 & 1 & 1\\
\hline
\end{tabular}
\end{center}
\item \textbf{$(p ∧ (p ∨ q)) ↔ p$}

\begin{center}
\begin{tabular}{|c|c|c|c|c|>{\columncolor[rgb]{0.88,1,1}}c|c|}
\hline
(p & $∧$ & (p & $∨$ & q)) & $↔$ & p \\
\hline
 0 & 0 & 0 & 0 & 0 & 1 & 0\\
\hline
 0 & 0 & 0 & 1 & 1 & 1 & 0\\
\hline
 1 & 1 & 1 & 1 & 0 & 1 & 1\\
\hline
 1 & 1 & 1 & 1 & 1 & 1 & 1\\
\hline
\end{tabular}
\end{center}
\item \textbf{$(¬(p ∧ q)) ↔ (¬p ∨ ¬q)$}
\begin{center}
\begin{tabular}{|c|c|c|c|>{\columncolor[rgb]{0.88,1,1}}c|c|c|c|}
\hline
($¬$ & (p & $∧$ & q)) & $↔$ & $¬$p & ∨ & $¬$q)\\
\hline
1 & 0 & 0 & 0 & 1 & 1 & 1 & 1\\
\hline
1 & 0 & 0 & 1 & 1 & 1 & 1 & 0\\
\hline
1 & 1 & 0 & 0 & 1 & 0 & 1 & 1\\
\hline
0 & 1 & 1 & 1 & 1 & 0 & 0 & 0\\
\hline
\end{tabular}
\end{center}
\item \textbf{$(p ∧ ¬p) ↔ ⊥$}
\begin{center}
\begin{tabular}{|c|c|c|>{\columncolor[rgb]{0.88,1,1}}c|c|}
\hline
(p & $∧$ & $¬$p) & $↔$ & $⊥$\\
\hline
0 & 0 & 1 & 1 & 0 \\
\hline
1 & 0 & 0 & 1 & 0 \\
\hline
\end{tabular}
\end{center}
\end{itemize}

\end{problem}



\begin{problem}[5]
Demostrar el Lema 2.1.3 p. 17.
\solution
\textcolor{red}{Viene demostrado en el libro. Que alguien lo copie}
\end{problem}



\begin{problem}[6]
Comprobar que los tres axiomas en el ejercico 1 de la hoja 2 son tautolog\'{\i}as.
\solution

Para demostrar que estos axiomas son tautologías debemos comprobar que para toda valoración de las proposiciones se cumplen las FBF. Vamos a verlo
\begin{enumerate}
\item  $ (p\to (q\to p))$

\begin{center}
\begin{tabular}{|c|>{\columncolor[rgb]{0.88,1,1}}c|c|c|c|}
\hline
($p$ & $\to$ & (q & $\to$ & p))\\
\hline
1 & 1 & 1 & 1 & 1\\
 %\hline
1 & 1 & 0 & 1 & 1\\
\hline
0 & 1 & 1 & 0 & 0\\
\hline
0 & 1 & 0 & 1 & 0\\
\hline
\end{tabular}
\end{center}

%$$\neg p \Or (\neg q \Or p) \o$$


\item $( (p\to (q\to r)) \to  ( (p\to q) \to (p\to r)))$



%\begin{align*}
%	(\neg(\neg p\Or\neg q\Or r) \Or ((\neg (\neg p\Or q)\Or (\neg p \Or r)))\\
%	(p\y q\y\neg r) \Or ((p\y\neg q)\Or(\neg p\Or r))\\
%	(p\y q\y \neg r)\Or((p\Or\neg p \Or r)\y(\neg q\Or\neg p \Or r))\\
%	(p\y q\y \neg r)\Or(\top \y(\neg q\Or\neg p\Or r))\\
%	(p\y q\y \neg r)\Or(\neg q\Or \neg p \Or r)\\
%	(p\Or q\Or \neg p \Or r)\y(q\Or\neg q \Or \neg p \Or r)\y(\neg r\Or\neg q \Or \neg p \Or r)\\
%	\top\y \top\y \top\\
%	\top
%\end{align*}
%
%\item $(( (\neg p) \to (\neg q)) \to  ( ((\neg p) \to q) \to p))$
%
%\begin{align*}
%	\neg (p\Or \neg q) \Or (\neg (p\Or q) \Or p)\\
%	(\neg p \y q) \Or ((\neg p \y \neg q)\Or p)\\
%	(\neg p\y q) \Or ((\neg p \Or p)\y(p\Or \neg q))\\
%	(\neg p\y q) \Or (\top\y(p\Or \neg q))\\
%	(\neg p \y q) \Or p \Or \neg q\\
%	(\neg p \Or p) \y (q\Or p) \Or \neg q\\
%	\top \y (q\Or p) \Or \neg q\\
%	q\Or p \Or \neg q\\
%	\top
%\end{align*}
%
\begin{center}
\begin{tabular}{|c|c|c|c|c|>{\columncolor[rgb]{0.88,1,1}}c|c|c|c|c|c|c|c|}
\hline
((p & $\to$ & (q & $\to$ & r)) & $\to$ & ((p & $\to$ & q) & $\to$ & (p & $\to$ & r)))\\
\hline
1 & 1 & 1 & 1 & 1 & 1 & 1 & 1 & 1 & 1 & 1 & 1 & 1\\
\hline
1 & 0 & 1 & 0 & 0 & 1 & 1 & 1 & 1 & 0 & 1 & 0 & 0\\
\hline
1 & 1 & 0 & 1 & 1 & 1 & 1 & 0 & 0 & 1 & 1 & 1 & 1\\
\hline
1 & 1 & 0 & 1 & 0 & 1 & 1 & 0 & 0 & 1 & 1 & 0 & 0\\
\hline
0 & 1 & 1 & 1 & 1 & 1 & 0 & 1 & 1 & 1 & 0 & 1 & 1\\
\hline
0 & 1 & 1 & 0 & 0 & 1 & 0 & 1 & 1 & 1 & 0 & 1 & 0\\
\hline
0 & 1 & 0 & 1 & 1 & 1 & 0 & 1 & 0 & 1 & 0 & 1 & 1\\
\hline
0 & 1 & 0 & 1 & 0 & 1 & 0 & 1 & 0 & 1 & 0 & 1 & 0\\
\hline
\end{tabular}
\end{center}

\item $(( (\neg p) \to (\neg q)) \to  ( ((\neg p) \to q) \to p))$

\begin{center}
\begin{tabular}{|c|c|c|>{\columncolor[rgb]{0.88,1,1}}c|c|c|c|c|c|}
\hline
((($\neg$ p) & $\to$ & ($\neg$ q)) & $\to$ & ((($\neg$ p) & $\to$ & q) & $\to$ & p))\\
\hline
1 & 1 & 1 & 1 & 1 & 0 & 0 & 1 & 0\\
\hline
0 & 1 & 1 & 1 & 0 & 1 & 0 & 1 & 1\\
\hline
1 & 0 & 0 & 1 & 1 & 1 & 1 & 0 & 0\\
\hline
0 & 1 & 0 & 1 & 0 & 1 & 1 & 1 & 1\\
\hline
\end{tabular}
\end{center}

\end{enumerate}

\end{problem}


\begin{problem}[7]
Comprobar que los ocho axiomas en las p\'aginas 20-21 del libro son tautolog\'{\i}as.
\solution
Procedemos como en el ejercicio anterior
\begin{enumerate}
\item \textbf{$\top$}

El símbolo atómico $\top$ es una tautología por definición, pues siempre tiene el valor verdadero.

\item \textbf{$p → (p ∨ q)$}



\item \textbf{$¬p → ¬q → ¬(p ∨ q)$}

\item \textbf{$(p ∧ q) → p$}


\item \textbf{$p → q → (p ∧ q)$}

\item \textbf{$p → (q → r) → (p → q) → (p → r)$}
\item \textbf{$p → (¬p → ⊥)$}
\item \textbf{$(¬p → ⊥) → p$}

\end{enumerate}
\end{problem}



\begin{problem}[8]
Reescribir  los ocho axiomas en las p\'aginas 20-21 del libro
usando s\'olo las conectivas en $\{\neg, \to\}$,  y  comprobar que dichos axiomas
se deducen de los   tres axiomas en el ejercico 1 de la hoja 2.
\solution

\begin{enumerate}
\item \textbf{$\top$}

\[\top \equiv a \to a\]

\item \textbf{$p → (p ∨ q)$}
\[p → (p ∨ q) \equiv p \to ((\neg p) \to q )\]

\item \textbf{$¬p → ¬q → ¬(p ∨ q)$}
\[¬p → ¬q → ¬(p ∨ q) \equiv (\neg p) \to ((\neg q ) \to ( \neg (( \neg p) \to q )))\]

\item \textbf{$(p ∧ q) → p$}
\[(p ∧ q) → p \equiv (\neg (( \neg p) \Or (\neg q ))) \to p\]

\item \textbf{$p → q → (p ∧ q)$}
\[p → q → (p ∧ q) \equiv p ( q \to (\neg ( p \to ( \neg q ))))\]

\item \textbf{$p → (q → r) → (p → q) → (p → r)$}

Ya está en el formato deseado

\item \textbf{$p → (¬p → ⊥)$}
\[p \to ( \neg p \to (\neg (p \to p)))\]

\item \textbf{$(¬p → ⊥) → p$}
\[(\neg p \to (\neg (p \to p))) \to p\]

\end{enumerate}
\end{problem}

\begin{problem}[9]
Obtener $\vdash (p\to p)$ a partir del Lema de la Deducci\'on.
\solution

\end{problem}

