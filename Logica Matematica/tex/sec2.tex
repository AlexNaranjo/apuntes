\section{Lógica de primer orden (o de predicados)}
\subsection{Exposición informal y ejemplos}
Asumimos que todos los conjuntos usados para definir el lenguaje son disjuntos. Usaremos:
\begin{itemize}
	\item \textbf{Símbolos} lógicos y de puntuación:
	$$\{\neg, \vee, ),(, ``," \}$$
	También usaremos los demás símbolos logicos como abreviaciones.
	\item Tendremos un conjunto infinito numerable de \textbf{variables}:
	$$Var = \{v_1, v_2, \hdots\}$$
	Aunque en la práctica usaremos $x,y,z$.
	\item Un conjunto de constantes, que podría ser vacío:
	$$Con = \{C_0, C_1, \hdots\}$$
	\item Funciones:
	$$Fun = \{f_0, f_1, \hdots\}$$
	\item Relaciones o predicados:
	$$Rel = \{R_1, R_2, \hdots\}$$
	\item Cuantificadores:
	$$\{\forall, \exists\}$$
	Podemos considerar $\exists x(P(x))$ como abreviación de $\neg (\forall x(\neg P(x)))$.
\end{itemize} 

No se puede cuantificar sobre constantes, funciones y relaciones en lógica de primer orden. En segundo orden, se permite cuantificar sobre funciones y predicados.

\begin{example}[Expresiones de primer orden] 

Sabemos que $P(x) \implies P(x)$, donde $x$ es libre, es cierto. Por otro lado $\forall x(P(x)\implies P(x))$ tiene el mismo significado, pero $x$ ahora es una variable ligada al cuantificador.
\end{example}

\begin{example}[Expresion de segundo orden] 

La expresión 
\[\forall P \forall x (P(x)\implies P(x))\]
es una expresión de segundo orden al estar cuantificando sobre predicados.
\end{example}

\begin{mdframed}
	\textbf{Principio de inducción matemática en $\mathbb{N}$}:
	$$\forall P (\left[P(0)\y (\forall n(P(n)\implies P(n+1)))\right]\implies \forall n(P(n)))$$
	Es una fórmula de segundo orden.
\end{mdframed}
En una lógica de primer orden, en vez de ``principio'', tenemos un ``esquema de principios'': Un principio de inducción por cada $P$.

\begin{example}[Axiomas de la teoría de grupos: Lenguaje $L_G$]\mbox{}

\begin{itemize}

	\item Axiomas:
	\begin{itemize}
		\vspace{-3mm}
		\item $\varphi_1: \forall x\forall y\forall z ((x\cdot y)\cdot z) = (x\cdot(y\cdot z))$
		\item $\varphi_2: \forall x(x\cdot e = e\cdot x = x)$
		\item $\varphi_3: \forall x\exists y (x\cdot y = e)$
	\end{itemize}

	\item Funciones:
	$$Fun = \{\cdot\}$$
	
	\item Relaciones:
	$$Rel = \{=\}$$
	
	\item Constantes:
	$$Con = \{e\}$$
\end{itemize}

Tenemos la teoría $T_G = \{\varphi_1, \varphi_2, \varphi_3\}$. Un grupo es cualquier modelo de $T_G$. Queremos ver si podemos demostrar, a partir de la misma $\psi: \forall x\forall y(x\cdot y = y\cdot x)$. Es decir, si $T_G\vdash \psi$. Vemos que si $G$ es no abeliano, entonces $G\nvDash \psi$, como conclusión tenemos $T_G\nvdash \psi$.

Esto es un ejemplo de que el teorema de validez también es cierto en lógica de primer orden.
$$T_G\vdash \psi \implies T_G \vDash \psi$$

Equivalentemente
$$T_G\nvDash \psi \implies T_G \nvdash \psi$$

¿Podemos probar entonces que $T_G\vdash \neg \psi$? No, porque si $G_2$ es abeliano, entonces $G_2\nvDash \neg\psi$ y por tanto, por validez, $T_G\nvdash \neg \psi$.

Como conclusión, tenemos que $T_G$ no es una teoría completa.
\end{example}

\begin{defn}[Clausura universal]
	La clausura universal de $\phi$ es $\forall x_1, \forall x_2,\hdots, \forall x_n \phi$, donde $x_1, x_2, \hdots, x_n$ son todas las variables que aparecen libres en $\phi$.
	
	\textbf{Abreviación}: $\forall x\in z$ $(\phi(x))$ es abreviación de $\forall x ((x\in z) \y \phi (x))$.
	
	\textbf{Abreviación}: $\exists ! x(\phi (x))$ abrevia $\exists x(\phi (x) \y (\forall y (\phi \implies y = x)))$.
\end{defn}

\begin{example}
	La clausura universal de $x < 5 + y$ es $$\forall x\forall y (x < 5 + y)$$ que tiene valor de verdad \textbf{falso}.
\end{example}

\subsection{Axiomas de ZFC (Zermelo-Fraenkel con Elección)}
	Todos son conjuntos, hereditarios y bien fundados.
	No hay ``urelementos''\footnote{Se utiliza el prefijo alemán ``ur'' que significa primordial} o átomos.
	\begin{itemize}
		\item \textbf{Axioma 1:} Existe un conjunto vacío:
		$$\exists z (\forall x (x\notin z))$$
		(En el lenguaje de la teoría de conjuntos $L_S$, tenemos las relaciones binarias $\{\in, =\}$. Por otro lado $Fun: \emptyset$, $Cons: \emptyset$).
		\item \textbf{Axioma 2: Extensionalidad}.
		Dos conjuntos con los mismos elementos son iguales.
		$$\forall x\forall y(\forall z (z\in x\iff z\in y)\iff x = y)$$
		\begin{corol}
			El conjunto vacío es único.
		\end{corol}
		
		\item (Esquema de) \textbf{Axioma}(s) \textbf{3: Compresión}. Las FBF permiten definir subconjuntos. Hablar de subconjuntos permite evitar la paradoja de Russel.
		
		Tenemos un axioma por cada $\phi\in FBF(L_S)$. El axioma es la clausura universal
		$$\forall z \exists y \forall x(x\in y\iff((x\in z)\y \phi))$$
		
		Si no tuvieramos $z$, el ``axioma'' quedaría así.
		
		$$\exists y \forall x (x\in y\iff \phi)$$
		
		Si tomamos $\phi: x\notin x \;\; (\equiv\neg (x\in x))$
		$$\exists y \forall x (x\in y\iff x\notin x)$$
		Sabemos que $\exists y$, digamos $y_0$.
		$$\forall x (x\in y_0 \iff x\notin x)$$ 
		Como tenemos el cuantificador $\forall x$, podemos sustituir $x$ por $y_0$, obteniendo
		$$y_0\in y_0 \iff y_0\notin y_0$$
		Que es una contradicción. Esto se soluciona añadiendo $z$ como se ha hecho.
		
		\begin{obs}
			$\{z_1,\hdots, z_n\}$ es un conjunto.
			
			$S(n) = \{0, \hdots, n\}$ es un conjunto.
			
			$y=\{1, \hdots, n\} = \{x\in S(n)\tq x>0\}$ es un conjunto.
			
			La función $f(i) = z_i$ es una función (conjunto de pares ordenados). Por el axioma de reemplazo, el rango $f$ es un conjunto. Donde $f:\{1,\hdots, n\}\to \{z_1, \hdots, z_n\}$.
		\end{obs}
		
		\item \textbf{Axioma 4: Emparejamiento.} Si $x$ e $y$ son conjuntos, el par $\{x, y\}$ es un conjunto. Si $z$ es un conjunto, $\{\{x,y\},z\}$ es un conjunto. Si $x = y$, entonces obtenemos $\{x,x\} = \{x\}$. Formalmente
		$$\forall x\forall y\exists z\forall u(u\in z\iff \left[(u=x)\Or (u=y)\right])$$
		$z$ es $\{x,y\}$, si $x=y$, $z$ es $\{x,x\}=\{x\}$.
	\end{itemize}
	
	\subsubsection{Intermedio con definiciones}
	
	\textbf{Abreviación: } $\emptyset$ es el conjunto vacío (dado por los axiomas $1$ y $2$).
	
	\textbf{Los naturales:} 
	\begin{itemize}
		\item $0:=\emptyset$
		\item $1:=\{\emptyset\}$ (por emparejamiento o partes cuando lo veamos).
		\item $2:=\{\emptyset, \{\emptyset\}\} = \{0,1\}$ (por emparejamiento).
		\item $3:=\{1,2,3\} = \{\emptyset,\{\emptyset\},\{\emptyset, \{\emptyset\}\}\}$ (Por el axioma 4).
	\end{itemize}
	\begin{defn}[Sucesor]
		El sucesor de $n$, $S(n)$ es $$S(n) = \{0,1,2,\hdots, n\}$$
	\end{defn}
	
	\begin{defn}[Suma en los naturales]
		$n+1 := S(n)$. Si hemos definido $n+m$, entonces
		$n+m+1 := S(n+m)$.
	\end{defn}
	
	\begin{defn}[Producto de los naturales]
		En cuanto al producto:
		\begin{itemize}
			\item $n\cdot0 = 0$
			\item $n\cdot1=n$
			\item Si $n\cdot m$ está definido, entonces $n(m+1) = n\cdot (m+1) = n\cdot m + n$.
		\end{itemize}
	\end{defn}
	
	\begin{defn}[Menor]
		$$n<m\iff n\in m$$
	\end{defn}
	
	Usando los axiomas vistos, tenemos que cada n es un conjunto, pero nada nos dice que $\mathbb{N} :=w$ sea un conjunto, o que exista un conjunto infinito.
	
	\begin{itemize}
		\item \textbf{Axioma 5: Uniones.}
		  
		 La unión de conjuntos en un conjunto es un conjunto.
		 
		 $$\forall x \exists y \forall z (z\in y\iff\exists u((u\in x)\y (z\in u)))$$
		 
		 Se usa al definir $$S(n) = \{0, 1,\hdots,n\} = n\cup \{n\}$$
		 
		 Nos permite definir cada $n$ como un conjunto, pero nada nos dice que $\mathbb{N}$ sea un conjunto.		 
		 \begin{example}
			Sea $X=\{1,\{2\},\{2,3\}\}$, entonces
			$$\cup X = \{0, 2, 3\}$$ porque $1=\{\emptyset\}$.
		 \end{example}
		 
		 \item (Esquema de) \textbf{Axioma}(s) \textbf{6: Reemplazo.}
		 Sea $\varphi$ una FBF en la que $y$ no aparece libre. Entonces, la clausura universal de la siguiente fórmula es un axioma:
		 $$\forall a \left[\forall x\in a ((\exists! y (\varphi(x,y))))\implies (\exists z \forall x \in a \exists y \in z (\varphi(x,y)))\right]$$
		 
		 Este axioma nos dice que para definir una función, $Ran f$ debe pertenecer a un conjunto preexistente. En particular, $Ranf$ no es ``cofinal'' en el universo de los conjuntos.
		 
		 \item \textbf{Axioma 7: Fundación.}
		 Nos dice que los conjuntos están bien fundados. Usamos $\emptyset$ para abreviar conjunto vacío.
		 $$\forall x \left[(x\neq \emptyset)\implies (\exists y \in x (\forall z \in x (z\notin y)))\right]$$
		 
		 Este axioma nos dice que si $x$ no es vacío, contiene algún elemento $y$ tal que $x\cap y = \emptyset$. Este axioma implica que para todo conjunto $z$, $z\notin z$. Supongamos que existe un $z$ tal que $z\in z$. Entonces, aplicamos el axioma de fundación al conjunto $\{z\}$, porque $z$ no es vacío (al contener a $z$).  $\{z\}$ es un conjunto por pares y extensión, $\{z,z\} = \{z\}$ o por compresión: $\{y : y\in z \y  y=z\}$.
		 
		 Por el axioma, $\exists y\in\{z\}$ tal que $z\cap y=\emptyset$. Pero $y=z$ luego $z\cap y = z \neq \emptyset$. Luego para todo conjunto, $z\notin z$.
		 
		 \begin{obs}
		 	Si hubiese un $z\in z$, entonces $z\ni z\ni z\ni z\hdots$. Obtendríamos una cadena infinita descendiente de $\in$.
		 	Pude demostrarse que este axioma prohibe todas las cadenas infinitas descendientes y los bucles tales como $z_1\ni z_2\ni \hdots \ni z_n \ni z_1$.
		 \end{obs}
		 
		 \item \textbf{Axioma 8: Conjunto de partes}

		 Si $X$ es un conjunto, el conjunto de partes (el conjunto de todos los subconjuntos de $X$) es un conjunto. Usamos $\mathcal{P}(X)$.
		 
		 \textbf{Par ordenado:} si $x$ e $y$ son conjuntos $(x,y):=\{x, \{x,y\} \}$ podemos definir funciones, relaciones, etc. Hemos construido cada $n$ con $n$ elementos. Sea $X$ un conjunto ya construido (por tanto, finito)
		 $$X,Y \subset \mathbb{N}, f:X\to Y$$
		 
		 $Ran f \subset N = \{0, \hdots, N-1\}$ conjunto ya construido. Podemos tomar $N = \max \{f(x): x \in X\} + 1$.
		 
		 En general $Card(\mathcal{P}(X)) = 2^{Card(X)}$.
		 
		 \textbf{Abreviación: } Usamos $z\subset X \equiv \forall w((w\in z)\implies (w\in X))$
		 $\forall X \exists y \forall z(z\subset X\iff x\in y)$.
		 
		 \textbf{Nota aleatoria:}
		 
		 $\mathbb{R}^1 = \mathbb{R}^{\{\emptyset\}} = \{(\emptyset, x): x\in\mathbb{R}\}$
		 
		 $\mathbb{R}^2 = \mathbb{R}^{\{0,1\}} = \{f:(0,1)\to \mathbb{R}\} = \{(x_0, x_1): x_0, x_1 \in \mathbb{R}\}$
		 
		 $\mathbb{R}^0 = R^\emptyset = \{f:\emptyset\to \mathbb{R}\}$ (es elegir 0 elementos de $\mathbb{R}$, que es lo mismo que rechazarlos todos, lo cual sólo hay una forma de hacerlo, que es nuestra constante).
		 
		 \item \textbf{Axioma 9: Infinitud.}
		 $\exists$ un conjunto que contiene a $\mathbb{N}$.
		 $$\exists x ((0\in x)\y (\forall y(y\in x \implies y\cup \{y\} \in x)))$$
		 
		 A partir de este conjunto, definimos $\mathbb{N} = \omega$ exigiendo que todo elemento distinto del 0 sea sucesor de otro elemento. Sea $X$ un conjunto dado por el axioma de infinitud.
		 $$\mathbb{N}:= \{x\in X:\left[\exists y \in X (x=y\cup \{y\})\y (\forall w\in x (w=0\Or (\exists z(w=z\cup \{z\}1)))) \right]\}$$
		 
		 \textbf{Ejemplo informal:}
		 
		 $\mathbb{N}_0:=\{(0, n): n\in \mathbb{N}\}\equiv \mathbb{N}$
		 
		 $\mathbb{N}_1:=\{(1, n): n\in \mathbb{N}\}\equiv \mathbb{N}$
		 
		 Luego $\mathbb{N}_0\cup\mathbb{N}_1$ con el orden del diccionario es $\{(0,0), (0,1), (0,2), \hdots, (1,0),(1,1), (1,2), \hdots\}$
		 
		 Por inducción: Sea $S\subset\mathbb{N}$. Si $0\in S\y(\forall n(n\in S\implies n+1\in S))$. En conclusión, $\mathbb{N}\subset S. (S\subset \mathbb{N}\implies \mathbb{N}=S)$
		 
		 \item \textbf{Axioma 10: Axioma de Elección.}
		 
		 $$\forall X(\emptyset\notin X \implies \left[ \exists f:X\to\cup X (\forall t\in x(f(t)\in t))\right])$$
		 		 
		 Hay que notar que al cuantificar funciones estamos pasando a lógica de segundo orden.
		 
		 Si $\mathcal{C}=\{A_\alpha: \alpha\in \Lambda\}$ es una colección de conjuntos disjuntos no vacíos, $\exists f:\mathcal{C}\to \cup\mathcal{C}$ tal que $$\forall A_\alpha \in \mathcal{C}, f(A_\alpha)\in A_\alpha$$
		 
		 Luego escrito en primer orden:
		 
		 $\forall y\forall u (u\subset P(y) \y \forall z\forall w\left[(z\in u \y w\in u)\implies (z\neq \emptyset\y (z = w \Or z\cap w = \emptyset ))\right]\implies \exists x \in P(y) (\forall z \in u \exists v (x\cap z = \{v\})))$
		 		 
		 El axioma de elección es equivalente al lema de Zorn, que es equivalente a que todo conjunto tiene un buen orden.
		 
		 Algunas consecuencias:
		 \begin{itemize}
		 	\item Todo conjunto infinito contiene un conjunto numerable.
		 	\item Toda relacion contiene a una función con el mismo dominio.
		 	\item Todo espacio vectorial tiene una base.
		 	\item Existen conjuntos en $[0,1]$ que no son medibles con respecto a la medida de Lebesgue.
		 	\item Si \textbf{AP} (axiomas de Peano) es completa, puede extenderse a una teoría completa.
		 \end{itemize}
		 
		 
	\end{itemize}
	
	\begin{theorem}
		$\mathbb{N}$ es único en el sentido de que si $A$ y $B$ satisfacen los axiomas de Peano, entonces $A=B=\mathbb{N}$. 
	\end{theorem}
	\begin{proof}
		Por hioótesis, $0\in A\cap B$.
		Si $n\in A\cap B$, luego $n+1\in A\cap B$. Por lo tanto $A=B=A\cap B=\mathbb{N}$.
	\end{proof}
	
	Hasta ahora hemos visto que
	\begin{itemize}
		\item Los axiomas de la teoría de grupos: admiten muchos modelos muy distintos.
		\item Axioma de Peano: intentan caracterizar a $\mathbb{N}$. Hay varios conjuntos que satisfacen los axiomas de Peano, estos modelos son todos isomorfos.
	\end{itemize}


\textcolor{red}{--------------------APUNTES GUARROS TOMADOS POR PARRA------------------------}
\textbf{Motivación para la lógica de primer orden:} Permite expresar los axiomas de ZFC que fundamentan la mayor parte de las matemáticas (en particular ZF $\Rightarrow$ Axiomas de Peano (1 orden))
\textbf{Desmotivación para las lógicas de 2º orden:} ¿Por qué no usamos lógicas de 2º orden todo el tiempo? En la práctica lo hacemos, y no nos preocupamos de si los conjuntos son demasiado grandes.

\begin{example}
\textbf{Teorema:} Sea K un espacio topológico compacto y sea $f: K \rightarrow \mathbb{R}$ continua. Entonces f alcanza sus valores máximo y mínimo.

\textbf{Traducción semiformal:} Sea $\algb{C}$= clase de todos los conjuntos compactos, y sea:
$$F_k = \{f: K \rightarrow \mathbb{R} | \text{f es continua}\}$$

Entonces nos queda:
$$\forall K \in \algb{C} \textbf{ }\forall f \in F_k \left( \exists a,b,c \in K \left( \forall x \in K \left( f(a) \leq f(x) \leq f(b) \right) \right) \right)$$

\textbf{OJO:} $\algb{C}$ es el conjunto de todos los conjuntos.

De modo que, una manera de interpretar el teorema anterior en una lógica de primer orden es considerar que tenemos un esquema de teoremas, es decir, tenemos un teorema por cada $(K,f)$ donde $K$ y $f$ pueden describirse en el lenguaje.
\end{example}

No usamos la lógica de 2º orden todo el tiempo porque compacidad, y por tanto completitud, fallan.

\begin{theorem}
Existe una teoría T en un lenguaje de 2º orden, tal que, todo subconjunto finito de T tiene un modelo, pero T no tiene ningún modelo.
\end{theorem}
\begin{proof}
Todos los modelos de $\Psi_{\infty}$ don infinitos, donde $\Psi_{\infty}$ nos dice que existe una función inyectiva que no es sobreyectiva:
$$ \Psi_{\infty}: \exists f \left(\forall x \textbf{ } \forall y \left( \left( f(x)=f(y)\right) \Rightarrow x=y \right) \y \left( \exists z \forall x \left( z \neq f(x) \right) \right) \right) $$
\end{proof}


Todo modelo de $\neg \Psi_{\infty}$ es finito.

Sea $\Psi_n$ la sentencia que nos dice que por lo menos hay n objetos distintos:
$$ \Psi_2: \exists v_1 \exists v_2 \left(v_1 \neq v_2 \right) $$
$$ \Psi_3: \exists v_1, v_2, v_3 \left( \left(v_1 \neq v_2 \right) \y \left(v_2 \neq v_3 \right) \y \left(v_1 \neq v_3 \right) \right) $$
$$\vdots$$
$$ \Psi_n: \exists v_1,..., v_n \left( \y _{1\leq i < j \leq n} \left(v_i \neq v_j \right)  \right) $$

Sea:
$$T = \left\{ \neg \Psi_{\infty}, \Psi_n : n \in \mathbb{N} \setminus \left\{0,1\right\} \right\}$$

Sea: $\epsilon \subset T$ un subconjunto finito, si $\epsilon = \{\neg \Psi_{\infty} \}$, $1=\{0\}$ es un modelo.

Si para algún n, $\Psi_n \in \epsilon$, sea N el natural más grande tal que $\Psi_n \in \epsilon$. Entonces $N=\{0,1,...,N-1\}$ es un modelo de $\epsilon \cup \{\neg \Psi_{\infty} \}$, por tanto, un modelo de $\epsilon$.

Pero T no tiene modelos, porque cualquier modelo debe ser a la vez finito (por $\neg \Psi_{\infty}$) y contener al menos n elementos distintos $\forall n \in \mathbb{N}$.

De modo que..., ¿hacia dónde vamos?.

\begin{theorem}[Teorema de completitud]
Sea T una teoría en el lenguaje de primer orden.
\begin{itemize}
\item \textbf{Versión 1:} $T \vdash \phi \Leftrightarrow T \vDash \phi$
\begin{proof}
$\Rightarrow)$ Por validez.
$\Leftarrow)$ Por Gödel (Henkin)
\end{proof}
\item \textbf{Versión 2:} $T \text{ es consistente } \Leftrightarrow T \text{ tiene un modelo }$
\end{itemize}
\end{theorem}

\textbf{Corolario:} Compacidad: Sea T una teoría en un lenguaje de primer orden. T tiene un modelo $\Leftrightarrow$ todo subconjunto finito de T tiene un modelo.
\textcolor{red}{---------------------------------------------------------------------------------------------}