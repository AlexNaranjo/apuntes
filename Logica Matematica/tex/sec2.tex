\section{Lógica de primer orden (o de predicados)}
\subsection{Exposición informal y ejemplos}
Asumimos que todos los conjuntos usados para definir el lenguaje son disjuntos. Usaremos:
\begin{itemize}
	\item \textbf{Símbolos} lógicos y de puntuación:
	$$\{\neg, \vee, ),(, ``," \}$$
	También usaremos los demás símbolos logicos como abreviaciones.
	\item Tendremos un conjunto infinito numerable de \textbf{variables}:
	$$Var = \{v_1, v_2, \hdots\}$$
	Aunque en la práctica usaremos $x,y,z$.
	\item Un conjunto de constantes, que podría ser vacío:
	$$Con = \{C_0, C_1, \hdots\}$$
	\item Funciones:
	$$Fun = \{f_0, f_1, \hdots\}$$
	\item Relaciones o predicados:
	$$Rel = \{R_1, R_2, \hdots\}$$
	\item Cuatificadotes:
	$$\{\forall, \exists\}$$
	Podemos considerar $\exists x(P(x))$ como abreviación de $\neg (\forall x(\neg P(x)))$.
\end{itemize} 

\begin{example}[Expresiones de primer orden] Sabemos que $P(x) \implies P(x)$ con $x$ es libre es cierto. Por otro lado $\forall x(P(x)\implies P(x))$ tiene el mismo significado, pero $x$ ahora es una variable ligada al cuantificador.
\end{example}

\begin{example}[Expresion de segundo orden] La expresión $\forall P \forall x (P(x)\implies P(x))$ es una expresión de segundo orden al estar cuantificando sobre predicados.
\end{example}

\begin{mdframed}
	\textbf{Principio de inducción matemática en $\mathbb{N}$}:
	$$\forall P (\left[P(0)\y (\forall n(P(n)\implies P(n+1)))\right]\implies \forall n(P(n)))$$
	Es una fórmula de segundo orden.
\end{mdframed}
En una lógica de primer orden, en vez de ``principio'', tenemos un ``esquema de principios'': Un principio de inducción por cada $P$.

\begin{example}[Axiomas de la teoría de grupos: Lengiaje $L_G$]\mbox{}

\begin{itemize}

	\item Axiomas:
	\begin{itemize}
		\vspace{-3mm}
		\item $\varphi_1: \forall x\forall y\forall z ((x\cdot y)\cdot z) = (x\cdot(y\cdot z))$
		\item $\varphi_2: \forall x(x\cdot e = e\cdot x = x)$
		\item $\varphi_3: \forall x\exists y (x\cdot y = e)$
	\end{itemize}

	\item Funciones:
	$$Fun = \{\cdot\}$$
	
	\item Relaciones:
	$$Rel = \{=\}$$
	
	\item Constantes:
	$$Con = \{e\}$$
\end{itemize}

Tenemos la teoría $T_G = \{\varphi_1, \varphi_2, \varphi_3\}$. Un grupo es cualquier modelo de $T_G$. Queremos ver si podemos demostrar, a partir de la misma $\psi: \forall x\forall y(x\cdot y = y\cdot x)$. Es decir, si $T_G\vdash \psi$. Vemos que si $G$ es no abeliano, entonces $G\vDash \psi$, como conclusión tenemos $T_G\vdash \psi$.

Esto es un ejemplo de que el teorema de validez también es cierto en lógica de primer orden.
$$T_G\vdash \psi \implies T_G \vDash \psi$$

Equivalentemente
$$T_G\nvDash \psi \implies T_G \nvdash \psi$$

¿Podemos probar entonces que $T_G\vdash \neg \psi$? No, porque si $G_2$ es abeliano, entonces $G_2\nvDash \neg\psi$ y por tanto, por validez, $T_G\nvdash \neg \psi$.

Como conclusión, tenemos que $T_G$ no es una teoría completa.
\end{example}


