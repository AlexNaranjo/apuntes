\section{Lógica de primer orden (o de predicados)}
\subsection{Exposición informal y ejemplos}
Asumimos que todos los conjuntos usados para definir el lenguaje son disjuntos. Usaremos:
\begin{itemize}
	\item \textbf{Símbolos} lógicos y de puntuación:
	$$\{\neg, \vee, ),(, ``," \}$$
	También usaremos los demás símbolos logicos como abreviaciones.
	\item Tendremos un conjunto infinito numerable de \textbf{variables}:
	$$Var = \{v_1, v_2, \hdots\}$$
	Aunque en la práctica usaremos $x,y,z$.
	\item Un conjunto de constantes, que podría ser vacío:
	$$Con = \{C_0, C_1, \hdots\}$$
	\item Funciones:
	$$Fun = \{f_0, f_1, \hdots\}$$
	\item Relaciones o predicados:
	$$Rel = \{R_1, R_2, \hdots\}$$
	\item Cuantificadores:
	$$\{\forall, \exists\}$$
	Podemos considerar $\exists x(P(x))$ como abreviación de $\neg (\forall x(\neg P(x)))$.
\end{itemize} 

No se puede cuantificar sobre constantes, funciones y relaciones en lógica de primer orden. En segundo orden, se permite cuantificar sobre funciones y predicados.

\begin{example}[Expresiones de primer orden] Sabemos que $P(x) \implies P(x)$ con $x$ es libre es cierto. Por otro lado $\forall x(P(x)\implies P(x))$ tiene el mismo significado, pero $x$ ahora es una variable ligada al cuantificador.
\end{example}

\begin{example}[Expresion de segundo orden] La expresión $\forall P \forall x (P(x)\implies P(x))$ es una expresión de segundo orden al estar cuantificando sobre predicados.
\end{example}

\begin{mdframed}
	\textbf{Principio de inducción matemática en $\mathbb{N}$}:
	$$\forall P (\left[P(0)\y (\forall n(P(n)\implies P(n+1)))\right]\implies \forall n(P(n)))$$
	Es una fórmula de segundo orden.
\end{mdframed}
En una lógica de primer orden, en vez de ``principio'', tenemos un ``esquema de principios'': Un principio de inducción por cada $P$.

\begin{example}[Axiomas de la teoría de grupos: Lengiaje $L_G$]\mbox{}

\begin{itemize}

	\item Axiomas:
	\begin{itemize}
		\vspace{-3mm}
		\item $\varphi_1: \forall x\forall y\forall z ((x\cdot y)\cdot z) = (x\cdot(y\cdot z))$
		\item $\varphi_2: \forall x(x\cdot e = e\cdot x = x)$
		\item $\varphi_3: \forall x\exists y (x\cdot y = e)$
	\end{itemize}

	\item Funciones:
	$$Fun = \{\cdot\}$$
	
	\item Relaciones:
	$$Rel = \{=\}$$
	
	\item Constantes:
	$$Con = \{e\}$$
\end{itemize}

Tenemos la teoría $T_G = \{\varphi_1, \varphi_2, \varphi_3\}$. Un grupo es cualquier modelo de $T_G$. Queremos ver si podemos demostrar, a partir de la misma $\psi: \forall x\forall y(x\cdot y = y\cdot x)$. Es decir, si $T_G\vdash \psi$. Vemos que si $G$ es no abeliano, entonces $G\vDash \psi$, como conclusión tenemos $T_G\vdash \psi$.

Esto es un ejemplo de que el teorema de validez también es cierto en lógica de primer orden.
$$T_G\vdash \psi \implies T_G \vDash \psi$$

Equivalentemente
$$T_G\nvDash \psi \implies T_G \nvdash \psi$$

¿Podemos probar entonces que $T_G\vdash \neg \psi$? No, porque si $G_2$ es abeliano, entonces $G_2\nvDash \neg\psi$ y por tanto, por validez, $T_G\nvdash \neg \psi$.

Como conclusión, tenemos que $T_G$ no es una teoría completa.
\end{example}

\begin{defn}[Clausura universal]
	La clausura universal de $\phi$ es $\forall x_1, \forall x_2,\hdots, \forall x_n \phi$, donde $x_1, x_2, \hdots, x_n$ son todas las variables que aparecen libres en $\phi$.
	
	\textbf{Abreviación}: $\forall x\in z$ $(\phi(x))$ es abreviación de $\forall x ((x\in z) \y \phi (x))$.
	
	\textbf{Abreciación}: $\exists ! x(\phi (x))$ abrevia $\exists x(\phi (x) \y (\forall y (\phi \implies y = x)))$.
\end{defn}

\begin{example}
	La clausura universal de $x < 5 + y$ es $$\forall x\forall y (x < 5 + y)$$ que tiene valor de verdad \textbf{falso}.
\end{example}

\subsection{Axiomas de ZFC (Zermelo-Fraenkel con Elección)}
	Todos son conjuntos, hereditarios y bien fundados.
	No hay ``urelementos''\footnote{Se utiliza el prefijo alemán ``ur'' que significa primordial} o átomos.
	\begin{itemize}
		\item \textbf{Axioma 1:} Existe un conjunto vacío:
		$$\exists x (\forall x (x\notin z))$$
		(En el lenguaje de la teoría de conjuntos $L_S$, tenemos las relaciones binarias $\{\in, =\}$. Por otro lado $Fun: \emptyset$, $Cons: \emptyset$).
		\item \textbf{Axioma 2: Extensionalidad}.
		Dos conjuntos con los mismos elementos son iguales.
		$$\forall x\forall y(\forall z (z\in x\iff z\in y)\iff x = y)$$
		\begin{corol}
			El conjunto vacío es único.
		\end{corol}
		
		\item (Esquema de) \textbf{Axioma}(s) \textbf{3: Compresión}. Las FBF permiten definir subconjuntos. Hablar de subconjuntos permite evitar la paradoja de Russel.
		
		Tenemos un axioma por cada $\phi\in FBF(L_S)$. El axioma es la clausura universal
		$$\forall z \exists y \forall x(x\in y\iff((x\in z)\y \phi))$$
		
		Si no tuvieramos $z$, el ``axioma'' quedaría así.
		
		$$\exists y \forall x (x\in y\iff \phi)$$
		
		Si tomamos $\phi: x\notin x (\neg (x\in x))$
		$$\exists y \forall x (x\in y\iff x\notin x)$$
		Sabemos que $\exists y$, digamos $y_0$.
		$$\forall x (x\in y_0 \iff x\notin x)$$ 
		Como tenemos el cuantificador $\forall x$, podemos sustituir $x$ por $y_0$, obteniendo
		$$y_0\in y_0 \iff y_0\notin y_0$$
		Que es una contradicción. Esto se soluciona añadiendo $z$ como se ha hecho.
		
		\item \textbf{Axioma 4: Emparejamiento.} Si $x$ e $y$ son conjuntos, el par $\{x, y\}$ es un conjunto. Si $z$ es un conjunto, $\{\{x,y\},z\}$ es un conjunto. Si $x = y$, entonces obtenemos $\{x,x\} = \{x\}$. Formalmente
		$$\forall x\forall y\exists z\forall u(u\in z\iff \left[(u=x)\Or (u=y)\right])$$
		$z$ es $\{x,y\}$, si $x=y$, $z$ es $\{x,x\}=\{x\}$.
	\end{itemize}
	
	\subsubsection{Intermedio con definiciones}
	
	\textbf{Abreviación: } $\emptyset$ es el conjunto vacío (dado por los axiomas $1$ y $2$).
	
	\textbf{Los naturales:} 
	\begin{itemize}
		\item $0:=\emptyset$
		\item $1:=\{\emptyset\}$ (por emparejamiento o partes cuando lo veamos).
		\item $2:=\{\emptyset, \{\emptyset\}\} = \{0,1\}$ (por emparejamiento).
		\item $3:=\{1,2,3\} = \{\emptyset,\{\emptyset\},\{\emptyset, \{\emptyset\}\}\}$ (Por el axioma 4).
	\end{itemize}
	\begin{defn}[Sucesor]
		El sucesor de $n$, $S(n)$ es $$S(n) = \{0,1,2,\hdots, n\}$$
	\end{defn}
	
	\begin{defn}[Suma en los naturales]
		$n+1 := S(n)$. Si hemos definido $n+m$, entonces
		$n+m+1 := S(n+m)$.
	\end{defn}
	
	\begin{defn}[Producto de los naturales]
		En cuanto al producto:
		\begin{itemize}
			\item $n\cdot0 = 0$
			\item $n\cdot1=n$
			\item Si $n\cdot m$ está definido, entonces $n(m+1) = n\cdot (m+1) = n\cdot m + n$.
		\end{itemize}
	\end{defn}
	
	\begin{defn}[Menor]
		$$n<m\iff n\in m$$
	\end{defn}
	
	Usando los axiomas vistos, tenemos qe cada n es un conjunto, pero nada nos dice que $\mathbb{N} :=w$ sea un conjunto, o que exista un conjunto infinito.
	
	\begin{itemize}
		\item \textbf{Axioma 5: Uniones.}
		  
		 La unión de conjuntos en un conjunto es un conjunto.
		 
		 $$\forall x \exists y \forall z (z\in y\iff\exists u((u\in x)\y (z\in u)))$$
		 
		 Se usa al definir $$S(n) = \{0, 1,\hdots,n\} = n\cup \{n\}$$
		 
		 Nos permite definir cada $n$ como un conjunto, pero nada nos dice que $\mathbb{N}$ sea un conjunto.		 
		 \begin{example}
			Sea $X=\{1,\{2\},\{2,3\}\}$, entonces
			$$\cup X = \{0, 2, 3\}$$ porque $1=\{\emptyset\}$.
		 \end{example}
		 
		 \item (Esquema de) \textbf{Axioma}(s) \textbf{6: Reemplazo.}
		 Sea $\varphi$ una FBF en la que $y$ no aparece libre. Entonces, la clausura universal de la siguiente fórmula es un axioma:
		 $$\forall a \left[\forall x\in a ((\exists! y (\varphi(x,y))))\implies (\exists z \forall x \in a \exists y \in z (\varphi(x,y)))\right]$$
		 
		 Este axioma nos dice que para definir una función, $Ran f$ debe pertenecer a un conjunto preexistente. En particular, $Ranf$ no es ``cofinal'' en el universo de los conjuntos.
		 
		 \item \textbf{Axioma 7: Fundación.}
		 Nos dice que los conjuntos están bien fundados. Usamos $\emptyset$ para abreviar conjunto vacío.
		 $$\forall x \left[(x\neq \emptyset)\implies (\exists y \in x (\forall z \in x (z\notin y)))\right]$$
		 
		 Este axioma nos dice que si $x$ no es vacío, contiene algún elemento $y$ tal que $x\cap y = \emptyset$. Este axioma implica que para todo conjunto $z$, $z\notin z$. Supongamos que existe un $z$ tal que $z\in z$. Entonces, aplicamos el axioma de fundación al conjunto $\{z\}$, porque $z$ no es vacío (al contener a $z$).  $\{z\}$ es un conjunto por pares y extensión, $\{z,z\} = \{z\}$ o por compresión: $\{y : y\in z \y  y=z\}$.
		 
		 Por el axioma, $\exists y\in\{z\}$ tal que $z\cap y=\emptyset$. Pero $y=z$ luego $z\cap y = z \neq \emptyset$. Luego para todo conjunto, $z\notin z$.
		 
		 \begin{obs}
		 	Si hubiese un $z\in z$, entonces $z\ni z\ni z\ni z\hdots$. Obtendríamos una cadena infinita descendiente de $\in$.
		 	Pude demostrarse que este axioma prohibe todas las cadenas infinitas descendientes y los bucles tales como $z_1\ni z_2\ni \hdots \ni z_n \ni z_1$.
		 \end{obs}
		 
		 
		 \item \textbf{Axioma 8: Conjunto de partes}

		 Si $X$ es un conjunto, el conjunto de partes (el conjunto de todos los subconjuntos de $X$) es un conjunto. Usamos $\mathcal{P}(X)$.
		 
		 \textbf{Par ordenado:} si $x$ e $y$ son conjuntos $(x,y):=\{x, \{x,y\} \}$ podemos definir funciones, relaciones, etc. Hemos construido cada $n$ con $n$ elementos. Sea $X$ un conjunto ya construido (por tanto, finito)
		 $$X,Y \subset \mathbb{N}, f:X\to Y$$
		 
		 $Ran f \subset N = \{0, \hdots, N-1\}$ conjunto ya construido. Podemos tomar $N = \max \{f(x): x \in X\} + 1$.
		 
		 \textbf{Abreviación: } Usamos $z\subset X \equiv \forall w((w\in z)\implies (w\in X))$
		 $\forall X \exists y \forall z(z\subset X\iff x\in y)$.
	\end{itemize}

