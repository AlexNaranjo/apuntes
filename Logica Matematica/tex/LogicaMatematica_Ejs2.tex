\section{Hoja 2}

Asumimos gen\'erica e informalmente la no trivialidad. Por ejemplo,
cuando hablamos de conjuntos, lenguajes, etc., suponemos que no  son vac\'{\i}os (salvo que 
expl\'{\i}citamente
se diga lo contrario),  cuando hablamos de f\'ormulas $p, q, r, \dots$ 
suponemos que est\'an bien formadas, etcetera.
En caso de duda, consultar con el intructor. 

\begin{problem}[1]
Con el lenguaje $L = A \cup \{\neg, \to\}$, $A \ne \emptyset$, la regla de deducci\'on modus ponens,
y los axiomas

1.  $ (p\to (q\to p))$,

2. $( (p\to (q\to r)) \to  ( (p\to q) \to (p\to r)))$, y

3. $(( (\neg p) \to (\neg q)) \to  ( ((\neg p) \to q) \to p))$,

comprobar que 

a) $\vdash (p\to p)$,

b) $\vdash ((( \neg p) \to p)\to p)$,

c)  $\{(p\to q), (q\to r)\} \vdash (p\to r)$.
\solution
\end{problem} 

\begin{problem}[2]
Demostrar la parte no trivial del lema de legibilidad \'unica: si $(p * q)$ es una fbf, donde $*$ es una conectiva
binaria, y $(p * q) = (r *^\prime s)$, entonces $p = r$,  $*  = *^\prime$, y $q = s$.
\solution
\end{problem}


\begin{problem}[3]
Un grafo es coloreable con 4 colores, si es posible asignar a todo v\'ertice uno de los 4 colores de modo
que v\'ertices adyacentes tengan colores distintos. Sea $G$ un grafo con v\'ertices $\{v_\alpha : \alpha \in\Lambda\}$,
y conjunto de aristas $E$ (que podr\'ia ser vacio; las aristas son subconjuntos de $G$ con cardinalidad 2).
Sea $L =  A \cup \{\neg, \vee, \wedge \}$, $A = \{v_\alpha(i) : \alpha \in\Lambda, i = 1, 2, 3, 4\}$. Interpretando 
$v_\alpha(i)$ como ``el v\'ertice $v_\alpha$ tiene color $i$", enunciar los axiomas de la teor\'{\i}a $T$ seg\'un la
cual el grafo $(G, E)$ es coloreable con 4 colores. 
\solution
\end{problem}