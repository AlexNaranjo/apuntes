\section{Hoja 2}

Asumimos gen\'erica e informalmente la no trivialidad. Por ejemplo,
cuando hablamos de conjuntos, lenguajes, etc., suponemos que no  son vac\'{\i}os (salvo que
expl\'{\i}citamente
se diga lo contrario),  cuando hablamos de f\'ormulas $p, q, r, \dots$
suponemos que est\'an bien formadas, etcetera.
En caso de duda, consultar con el intructor.

\begin{problem}[1]
Con el lenguaje $L = A \cup \{\neg, \to\}$, $A \ne \emptyset$, la regla de deducci\'on modus ponens,
y los axiomas

1.  $ (p\to (q\to p))$,

2. $( (p\to (q\to r)) \to  ( (p\to q) \to (p\to r)))$, y

3. $(( (\neg p) \to (\neg q)) \to  ( ((\neg p) \to q) \to p))$,

comprobar que

\ppart $\vdash (p\to p)$,

\ppart $\vdash ((( \neg p) \to p)\to p)$,

\ppart  $\{(p\to q), (q\to r)\} \vdash (p\to r)$.
\solution

\spart
Si sustituimos $q$ por $p$ obtenemos una instancia del primer axioma:
\[(p \to ( p \to p ))\]

Sustituyendo ahora $q$ por $p \to p$ en el primer axioma obtenemos:
\[(p \to (( p \to p) \to p))\]

La sustitución de $q$ por $p \to p$ y de $r$ por $p$ en el segundo axioma da lugar a la instancia:
\[((p \to (( p \to p ) \to p )) \to ((p \to (p \to p)) \to (p \to p)))\]

Aplicando Modus Ponens a las dos últimas fórmulas tenemos:
\[((p \to (p \to p)) \to (p \to p))\]

Y finalmente, aplicando Modus Ponens de nuevo con la primera y última fórmulas obtenidas en este ejercicio llegamos a:
\[p \to p\]
\spart
Por sustituición podemos convertir la fórmula del apartado anterior en
\[ (\neg p \to \neg p)\]

Ahora sustituimos $q$ por $p$ en el axioma 3, obteniendo:
\[ (( (\neg p) \to (\neg p)) \to  ( ((\neg p) \to p) \to p))\]

Aplicando Modus Ponens tenemos:
\[ ( ((\neg p) \to p) \to p)\]

\spart

Sustituyendo en el tercer axioma $\neg p$ por $p$ y $\neg q $ por $q$ tenemos:
\[(( p \to q ) \to ((p \to (\neg q)) \to (\neg p)))\]

Haciendo Modus Ponens con $(p \to q)$ tenemos
\[((p \to (\neg q)) \to (\neg p))\]

Sustituyendo $\neg q$ por $r$ nos queda
\[((p \to r) \to (\neg p))\]
y aplicando Modus Ponens con $p \to r$ llegamos a
\[(\neg p)\]

Si en el axioma 1 sustituimos $p$ por $\neg p$ y aplicamos Modus Ponens con la última regla que hemos deducido, llegamos a:
\[(q \to (\neg p))\]
\end{problem}

\begin{problem}[2]
Demostrar la parte no trivial del lema de legibilidad \'unica: si $(p * q)$ es una fbf, donde $*$ es una conectiva
binaria, y $(p * q) = (r *^\prime s)$, entonces $p = r$,  $*  = *^\prime$, y $q = s$.
\solution
{\color{red}{\textbf{Hay que terminarlo y revisarlo}}}

Sea $(p \ast q) = (r\ast^\prime s)$ donde $p,q,r,s$ son FBF.

Ningún segmento inicial de una FBF puede ser una FBF.

\textbf{Razón: } Dado que toda FBF comienza con $($, acaba necesariamente con $)$ y tiene el mismo número de paréntesis izquierdos que derechos (ver ejercicio 4 de la hoja 1), cualquier segmento inicial de una FBF no es una FBF porque tendría un número mayor de $($ que de $)$.

Por tanto, tenemos que $r$ no puede ser un segmento inicial de $p$ porque $r$ no sería FBF y tenemos que $p$ lo es.

Por otro lado, $p$ no puede ser un segmento inicial de $r$ porque tenemos que $p$ es una FBF, por lo tanto $r$ no sería FBF.

Obtenemos que $p$ y $r$ tienen el mismo número de símbolos, luego $p=r$. A partir de aquí, tenemos $\ast = \ast^\prime$ y utilizando el mismo razonamiento $q=s$.

\end{problem}


\begin{problem}[3]
Un grafo es coloreable con 4 colores, si es posible asignar a todo v\'ertice uno de los 4 colores de modo
que v\'ertices adyacentes tengan colores distintos. Sea $G$ un grafo con v\'ertices $\{v_\alpha : \alpha \in\Lambda\}$,
y conjunto de aristas $E$ (que podr\'ia ser vacio; las aristas son subconjuntos de $G$ con cardinalidad 2).
Sea $L =  A \cup \{\neg, \vee, \wedge \}$, $A = \{v_\alpha(i) : \alpha \in\Lambda, i = 1, 2, 3, 4\}$. Interpretando
$v_\alpha(i)$ como ``el v\'ertice $v_\alpha$ tiene color $i$", enunciar los axiomas de la teor\'{\i}a $T$ seg\'un la
cual el grafo $(G, E)$ es coloreable con 4 colores.
\solution
\end{problem}