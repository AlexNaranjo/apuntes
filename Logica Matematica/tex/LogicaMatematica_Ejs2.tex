\section{Hoja 2}

Asumimos gen\'erica e informalmente la no trivialidad. Por ejemplo,
cuando hablamos de conjuntos, lenguajes, etc., suponemos que no  son vac\'{\i}os (salvo que
expl\'{\i}citamente
se diga lo contrario),  cuando hablamos de f\'ormulas $p, q, r, \dots$
suponemos que est\'an bien formadas, etcetera.
En caso de duda, consultar con el intructor.

\begin{problem}[1]
Con el lenguaje $L = A \cup \{\neg, \to\}$, $A \ne \emptyset$, la regla de deducci\'on modus ponens,
y los axiomas

1.  $ (p\to (q\to p))$,

2. $( (p\to (q\to r)) \to  ( (p\to q) \to (p\to r)))$, y

3. $(( (\neg p) \to (\neg q)) \to  ( ((\neg p) \to q) \to p))$,

comprobar que

\ppart $\vdash (p\to p)$,

\ppart $\vdash ((( \neg p) \to p)\to p)$,

\ppart  $\{(p\to q), (q\to r)\} \vdash (p\to r)$.
\solution

\spart
% http://www.lidiagroup.org/images/descargas/varios/011_rcra.pdf
Si sustituimos $q$ por $p$ obtenemos una instancia del primer axioma:
\[(p \to ( p \to p ))\]

Sustituyendo ahora $q$ por $p \to p$ en el primer axioma obtenemos:
\[(p \to (( p \to p) \to p))\]

La sustitución de $q$ por $p \to p$ y de $r$ por $p$ en el segundo axioma da lugar a la instancia:
\[((p \to (( p \to p ) \to p )) \to ((p \to (p \to p)) \to (p \to p)))\]

Aplicando Modus Ponens a las dos últimas fórmulas tenemos:
\[((p \to (p \to p)) \to (p \to p))\]

Y finalmente, aplicando Modus Ponens de nuevo con la primera y última fórmulas obtenidas en este ejercicio llegamos a:
\[p \to p\]
\spart
Podemos repetir toda la demostración del apartado anterior instanciando todas las fórmulas sustituyendo todos los símbolos por $\neg p$ con lo que llegaríamos a:
\[ (\neg p \to \neg p)\]

Ahora sustituimos $q$ por $p$ en el axioma 3, obteniendo:
\[ (( (\neg p) \to (\neg p)) \to  ( ((\neg p) \to p) \to p))\]

Aplicando Modus Ponens tenemos:
\[ ( ((\neg p) \to p) \to p)\]

\spart
%http://math.stackexchange.com/questions/290725/proof-of-transitivity-in-hilbert-style
Sustituyendo $p$ por $q \to r$ y $q$ por $p$ en el primer axioma tenemos la instanciación:
\[((q \to r) \to (p \to (q \to r)))\]

Aplicando ahora Modus Ponens a esta formula junto con $(q \to r)$ tenemos:
\[(p \to (q \to r))\]

Aplicando ahora Modus Ponens a la fórmula anterior junto con el segundo axioma llegamos a:
\[ ( (p\to q) \to (p\to r))\]

Finalmente, aplicando nuevamente Modus Ponens con $(p \to q)$ llegamos a
\[(p \to r )\]

\textbf{Solución del profesor}

Sea $$T=\left\{(p\to q), (q\to r)\right\}\vdash(p\to r)$$

\begin{enumerate}
	\item Escribimos el axioma 1 con $(q\to r)$ en vez de $p$ y $p$ en vez de $q$:
	$$((q\to t)\to(p\to (q\to r)))$$
	\item Tenemos la premisa en $T$: $(q\to r)$.
	\item Modus ponens utilizando $1$ y $2$:
	$$(p\to(q\to r))$$
	\item $$\left((p\to(q\to r))\to ((p\to q)\to(p\to r))\right)$$
	\item Modus ponens utilizando $3$ y $4$: $$((p\to q) \to (p\to r))$$
	\item Tenemos la premisa en $T$:
	$(p\to q)$.
	\item Modus ponens utilizando $5$ y $6$:
	$$(p\to r)$$
\end{enumerate}

\end{problem}

\begin{problem}[2]
Demostrar la parte no trivial del lema de legibilidad \'unica: si $(p * q)$ es una fbf, donde $*$ es una conectiva
binaria, y $(p * q) = (r *^\prime s)$, entonces $p = r$,  $*  = *^\prime$, y $q = s$.
\solution

Sea $(p \ast q) = (r\ast^\prime s)$ donde $p,q,r,s$ son FBF.

Ningún segmento inicial de una FBF puede ser una FBF.

\textbf{Razón: } Dado que toda FBF comienza con $($, acaba necesariamente con $)$ y tiene el mismo número de paréntesis izquierdos que derechos (ver ejercicio 4 de la hoja 1), cualquier segmento inicial de una FBF no es una FBF porque tendría un número mayor de $($ que de $)$.

Por tanto, tenemos que $r$ no puede ser un segmento inicial de $p$ porque $r$ no sería FBF y tenemos que $p$ lo es.

Por otro lado, $p$ no puede ser un segmento inicial de $r$ porque tenemos que $p$ es una FBF, por lo tanto $r$ no sería FBF.

Obtenemos que $p$ y $r$ tienen el mismo número de símbolos, luego $p=r$. A partir de aquí, tenemos $\ast = \ast^\prime$ y, utilizando el mismo razonamiento, $q=s$.

\end{problem}


\begin{problem}[3]
Un grafo es coloreable con 4 colores, si es posible asignar a todo v\'ertice uno de los 4 colores de modo
que v\'ertices adyacentes tengan colores distintos. Sea $G$ un grafo con v\'ertices $\{v_\alpha : \alpha \in\Lambda\}$,
y conjunto de aristas $E$ (que podr\'ia ser vacio; las aristas son subconjuntos de $G$ con cardinalidad 2).
Sea $L =  A \cup \{\neg, \vee, \wedge \}$, $A = \{v_\alpha(i) : \alpha \in\Lambda, i = 1, 2, 3, 4\}$. Interpretando
$v_\alpha(i)$ como ``el v\'ertice $v_\alpha$ tiene color $i$", enunciar los axiomas de la teor\'{\i}a $T$ seg\'un la
cual el grafo $(G, E)$ es coloreable con 4 colores.
\solution

% http://wikis.uca.es/wikira/index.php/Coloreo_de_mapas

El grafo será coloreable siempre que dos vértices del mismo color no estén conectados. Por tanto los axiomas serían:A\begin{itemize}
\item \textbf{Cada vértice tiene solamente un color}
\[\{(\neg v_α(1) \Or \neg v_α(2)) \y (\neg v_α(1) \Or \neg v_α(3)) \y\]
\[\y (\neg v_α(1) \Or \neg v_α(4)) \y (\neg v_α(2) \Or \neg v_α(3)) \y \]
\[\y(\neg v_α(2) \Or \neg v_α(4)) \y (\neg v_α(3) \Or \neg v_α(4)) \ | \ α \in \Lambda\}\]

\item \textbf{Dos vértices conectados por una arista no tienen el mismo color}
\[\{(\neg v_a(1) \lor \neg v_b(1)) \land (\neg v_a(2) \lor \neg v_b(2)) \land (\neg v_a(3) \lor \neg v_b(3)) \land(\neg v_a(4) \lor \neg v_b(4)) \ | \ E(a,b)\}\]
\end{itemize}

\end{problem}