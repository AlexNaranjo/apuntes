% -*- root: ../AlgebraConmutativa.tex -*-
\chapter{Teorema de los ceros de Hilbert}

Hemos jugado en $\Akn$ y en $\K[x_1,\dotsc,x_n]$. Y sabemos movernos de un sitio a otro:

\begin{align*}
	\Akn & \longmapsto  \K[x_1,...,x_N] \\
	X & \longmapsto  \I(X) \\
	\V(J) & \leftarrow  J \\
\end{align*}

\begin{itemize}
	\item Del lado izquierdo me interesan las v.a.a., si cojo $X\in \Akn$ v.a.a., puedo ir a $\I(X) \in \K[x_1,...,x_n]$ y volver a $\V(\I(X)) \in \Akn$ que es igual a $X$ por ser una v.a.a..

	\item Del lado derecho, me interesan los ideales, porque determinan v.a.a., pero me basta quedarme con los ideales radicales ya que genera la misma v.a.a. un ideal $I$ y su $\mop{Rad}\,(I)$.

	Sin embargo, no ocurre como antes en cuanto a que si cojo $J$ ideal radical en $\K[x_1,...,x_n]$, puedo ir a $\V(J)$ en $\Akn$, y volver a $\I(\V(J)) \supseteq J$. Pero puede pasar que la inclusión sea estricta:
	\begin{example}
		\begin{enumerate}
			\item Ejemplo de inclusión estricta:
			\begin{align*}
				\A^1_{\real} & \longmapsto  \real[x] \\
				\V(\gen{x^2+1}) & \leftarrow  \gen{x^2+1} \\
				\shortparallel & \\
				\emptyset & \longmapsto  \I(\{ \emptyset \})=\real[x] \\
			\end{align*}
			\item Otro ejemplo de inclusión estricta:

			Cogemos en $\real[x,y]$ e ideal  radical $\gen{(x-1)^2+(y-2)^2}$, que es primo (y por tanto radical) porque $p(x,y)=(x-1)^2+(y-2)^2$ es irreducible en $\real[x,y]$.

			Esto es así porque $\real[x,y] \subset \cplex[x,y]$  que es un dominio de factorización única.

			Y en $\cplex[x,y]$ tenemos que $p(x,y)=(x-1)^2+(y-2)^2=((x-1)+i(y-2))((x-1)-i(y-2))$. Por tanto no se puede factorizar de otra forma.

			\begin{align*}
				\A^2_{\real} & \longmapsto  \real[x,y] \\
				\V(J) & \leftarrow  \gen{(x-1)^2+(y-2)^2}=J \\
				\shortparallel & \\
				X=\{1,2\} & \longmapsto  \I(X)=\gen{x-1,y-2}\\
			\end{align*}

			Y obviamente $\gen{(x-1)^2+(y-2)^2} \subsetneqq \gen{x-1,y-2}$
		\end{enumerate}
	\end{example}
\end{itemize}

\textbf{Conclusión:} Ideales radicales \textbf{distintos} determinan variedades algebraicas iguales, esto nos molesta, porque no hay una relación 1 a 1 entre ideales radicales y v.a.a.. No hay un diccionario entre v.a.a. en $\Akn$ e ideales radicales en $\K[x_1,...,x_n]$.

Y aquí empieza la motivación de este tema, en buscar algo que nos permita tener ese diccionario, y eso lo conseguimos gracias a Hilbert!!

\begin{theorem}[Teorema\IS de los ceros de Hilbert]\label{thm:tma_0_hilbert}
	Si $\K$ es algebraicamente cerrado y $J \subset \K[x_1,...,x_n]$ es radical, entonces $J=\I(\V(J))$
\end{theorem}



\textbf{Corolario:}	Si $\K$ es algebraicamente cerrado, hay una correspondencia biyectiva entre variedades algebraicas afines en $\Akn$ e ideales radicales en $\K[x_1,...,x_n]$

Nuestro objetivo en este tema es probar el \nref{thm:tma_0_hilbert}. Para ello, vamos a utilizar algunas proposiciones y teoremas:

\begin{theorem}[Teorema\IS básico] \label{thm:Basico}
	Sea $\K$ un cuerpo algebraicamente cerrado, entonces todo ideal maximal en $\K[x_1,...,x_n]$ es de la forma $\gen{x_1-a_1,...,x_n-a_n}$ con $a_1,...,a_n \in \K$
\end{theorem}

\obs En $\K[x_1,...,x_n]$ todo ideal de la forma $\gen{x_1-a_1,...,x_n-a_n}$ es maximal aunque $\K$ no sea algebraicamente cerrado ya que:
$$ \quot{\K[x_1,...,x_n]}{\gen{x_1-a_1,...,x_n-a_n}} \simeq \K $$

\obs Supongamos $\rac = \K$. Tomamos $\rac[x,y]$, entonces $\gen{x-1,y^2-2}$ es maximal ya que $\quot{\rac[x,y]}{\gen{x-1,y^2-2}} \simeq \rac(\sqrt{2})$  que es un cuerpo.

Pasa lo mismo con $\gen{x-1,y^7+3y^2+3}$ o con $\gen{x^5+5x+5, y^7+3y^2+3}$. \textcolor{red}{wtf son estos ejemplos, luego dice algo así como que salen extensiones finitas de grado el producto}

\begin{theorem}[Teorema\IS débil] \label{thm:Debil}
	Sea $\K$ un cuerpo algebraicamente cerrado y sea $I$ un ideal propio $\subsetneqq \K[x_1,...,x_n]$. Entonces $\V(I) \neq \emptyset$.
\end{theorem}

\obs Si tomamos un cuerpo que no es algebraicamente cerrado esto no tiene porque pasar, por ejemplo en $\real[x]$ el conjunto $\V(\gen{x^2+5})=\emptyset$


Demostramos el teorema débil:
\begin{proof}
	Sea $I \subsetneqq \K[x_1,...,x_n] \implies \exists M$ maximal en $\K[x_1,...,x_n]$ tal que $I \subset M$. Por el teorema básico, $M$ es de la forma $\gen{x_1-a_1,...,x_n-a_n}$ para ciertos $a_1,...,a_n \in \K$ $\implies \V(I) \supset \V(M)=\{a_1,...,a_n\}$ luego $\V(I)\neq \emptyset$.
\end{proof}


Demostramos el teorema de los ceros de Hilbert usando el teorema débil:
\begin{proof}
	Queremos probar que $\I(\V(I))=\sqrt{I}$.

	Siempre tenemos (incluso si $\K$ no es algebraicamente cerrado) que $\sqrt{I} \subset \I(\V(I))$. \textcolor{red}{Ya que $\sqrt{I} \subset I$} Por tanto nos falta probar que $\I(\V(I)) \subset \sqrt{I}$.

	Sea $g \neq 0 \in \I(\V(I))$, tenemos que ver si existe un $N$ tal que $g^N \in I$.

	Sean $f_1(x_1,...,x_n),...,f_s(x_1,...,x_n) \in I$ tal que $I=\gen{f_1,...,f_s}$ (conjunto finito de generadores por ser $\K[x_1,...,x_n]$ noetheriano).

	Vamos a trabajar en $\K[x_1,...,x_n,T]$. Cogemos $J=\gen{f_1(x_1,...,x_n),...,f_s(x_1,...,x_n), g(x_1,...,x_n)T-1}$.

	Ahora miramos los $0's$ de $J$: $\V(J) \subset \A^{n+1}_K$. Queremos ver si $\exists A=\{a_1,..,a_n,a_{n+1} \} \in \V(J)$ es decir, si existe un conjunto $A$ de n+1 puntos que se anulen en $J$, es decir, $f_i(a_1,...,a_n)=0$ $\forall i \in [1,n]$. Por tanto, para que $\textbf{A} \in \V(J)$ necesariamente \textbf{tiene que pertenecer a $\V(I)$.}

	Por tanto nos falta probar que $g(a_1,...,a_n)a_{n+1}-1=0$. Hemos dicho que $g \in \I(\V(I))$ luego $g(a_1,...,a_n)=0$ $\implies g(a_1,...,a_n)T-1\neq 0$ siempre. $\implies \V(J)=\emptyset \underbrace{\implies}_{\text{Tma débil}} J=\gen{1}=$ el total.

	Por tanto existe una combinación lineal que da $1$, es decir,  $\exists q_j(x_1,...,x_n,T), j=[1,s]$ y $q_{s+1}$ tales que:
	$$ q_1(x_1,..,x_n,T)f_1(x_1,...,x_n)+...+q_s(x_1,..,x_n,T)f_s(x_1,...,x_n) + q_{s+1}(x_1,..,x_n,T)(g(x_1,...,x_n)T-1)=1 $$

	Reemplazamos $T$ por $\frac{1}{g(x_1,...,x_n)}$ y nos queda:
	$$ q_1(x_1,..,x_n, \frac{1}{g(x_1,...,x_n)})f_1(x_1,...,x_n)+...+q_s(x_1,..,x_n,\frac{1}{g(x_1,...,x_n)})f_s(x_1,...,x_n)+0=1 $$

	Esta expresión es una combinación de elementos de $q_i \in \K[x_1,...,x_n,T]$

	Tomando $N$ suficientemente grande y multiplicando por $g^N$, cancelamos todos los denominadores, y nos quedan elementos pertenecientes a $\K[x_1,...,x_n]$:

	$$ \underbrace{g^N\left(q_1(x_1,..,x_n, \frac{1}{g})\right)}_{\in \K[x_1,...,x_n]}f_1(x_1,...,x_n)+...+ \underbrace{g^N \left( q_s(x_1,..,x_n,\frac{1}{g})\right)}_{\in \K[x_1,...,x_n]}f_s(x_1,...,x_n) =g^N \in I $$

	Luego acabamos de expresar $g^N$ como combinación lineal de elementos de $I=\gen{f_1,...,f_s}$ por elementos de $\K[x_1,...,x_n]$.
\end{proof}


De ahora en adelante tenemos por objetivo desarrollar la teoría necesaria para probar el teorema básico:

\section{Extensiones enteras de anillos}

Una extensión de anillos es simplemente una inclusión de anillos $A\subset B$.

\begin{defn}[Elemento\IS entero] \label{def:ElementoEntero}
	Dados $A \subset B$, diremos que $b \in B$ es entero sobre $A$ si $\exists p(x)=a_0+a_1x+ \dotsb +a_{n-1}x^{n-1}+x^n \in A[x]$ mónico y no nulo con coeficientes en $A$ tal que $p(b)=0$.
\end{defn}

\begin{example}
	\begin{itemize}
	\item $\ent \subset \real$: $\sqrt{2}$ es entero sobre $\ent$ por ser raíz de $x^2-2 \in \ent[x]$
	\item $\frac{1}{2}$ no es entero sobre $\ent$ ya que $p(x)=2x-1$ no es mónico.
	\end{itemize}
\end{example}


\begin{defn} [Elemento\IS algebraico]
	Dada una extensión de cuerpos $A \subset B$, se dice que $b \in B$ es algebraico sobre $A$ si $\exists p(x)=a_0+a_1x+...+a_nx^n \in A[x]$ (no nulo) tal que $p(b)=0$.
\end{defn}

\obs entero $\implies$ algebraico. Es decir, un elemento puede ser algebraico, pero no entero.

\obs Cuando $A$ es un cuerpo: entero $\Leftrightarrow$ algebraico.

%16 Marzo

\begin{example}
	$\frac{1}{2}$ es algebraico sobre $\ent$ pero no es entero.
\end{example}

\begin{example}
	Sea $A= \ent$. Como hemos visto antes $\sqrt{2}$ es entero sobre $\ent$ ya que $\sqrt{2}$ es raíz de $x^2-2 \in \ent[x]$.

	Sea $\ent(\sqrt{2})$ el anillo más pequeño que contiene a $\ent$ y a $\sqrt{2}$. Es por tanto una $\ent$-álgebra de tipo finito y también es un $\ent$-módulo finitamente generado.
	$$ \ent(\sqrt{2})=\{ a+b\sqrt{2}:a,b \in \ent \} \quad \text {1 y }\sqrt{2} \text{ generan } \ent(\sqrt{2}) \text{ como } \ent\text{-módulo }$$

	Consideramos $\ent \subset \ent(\frac{1}{2})$ que es una $\ent$-álgebra de tipo finito (basta coger $\alpha_1=\frac{1}{2}$). En concreto  $\ent(\frac{1}{2})$ es el anillo que sale de hacer invertible a $\frac{1}{2}$, por tanto, es como localizar en las potencias de 2.

	Pero no es un $\ent$-módulo finitamente generado (ya que si cogemos $m_1=\frac{1}{2}$, no tendremos el elemento $\frac{1}{4}$, y en general el elemento $\frac{1}{2^n}$). Lo demostramos a continuación:

	Supongamos que sí es $\ent$-módulo, es decir, que existen $m_1,...,m_n \in \ent(\frac{1}{2})$ tales que $m_1,...,m_n$ generan $\ent(\frac{1}{2})$ como $\ent$-módulo. Entonces, los $m_1,...,m_n$ se podrían expresar con $a_{ij} \in \ent$ de la siguiente forma:

	\begin{align*}
		& m_1=a_{01}+a_{11}\frac{1}{2}+...+a_{n1}\frac{1}{2^{s_1}} = \frac{b_1}{2^{s_1}} \\
		& \vdots \\
		& m_n=a_{0n}+a_{11}\frac{1}{2}+...+a_{n1}\frac{1}{2^{s_n}} = \frac{b_n}{2^{s_n}} \\
	\end{align*}

	Supongamos sin pérdida de generalidad que  $s_1 = \max\{s_1,...,s_n\}$ entonces si $m_1,...,m_n$ generan $\ent(\frac{1}{2})$ como $\ent$-módulo, entonces existen enteros $c_1,...,c_n \in \ent$ tal que $\frac{1}{2^{s_1+1}}=c_1\frac{b_1}{2^{s_1}}+...+c_n\frac{b_n}{2^{s_n}}$

	Multiplicamos por $2^{s_1}$ y nos queda:
	$$ \frac{1}{2} = \underbrace{c_1 b_1+...+c_nb_n2^{s_1-s_n}}_{\in \ent} \rightarrow \text{ IMPOSIBLE} $$
\end{example}

\textbf{Conclusiones:}
\begin{itemize}
	\item Si a un anillo $A$ le adjunto un entero, se transforma en un $A$-módulo.
	\item Si a un anillo $A$ le adjunto un elemento algebraico, no tiene porque transformarse en $A$-módulo. (sólo si es entero)
\end{itemize}

\begin{prop}
	Sea $A\subset B$ una extensión de anillos y sea $b \in B$. Entonces $b$ es entero sobre $A \Leftrightarrow A[b]$ es un $A$-módulo finitamente generado. \textcolor{blue}{SI! hace unos días ponia $A(b)$ en vez de $A[b]$, me equivoque, lo correcto es $A[b]$}
\end{prop}

\begin{proof}

	\proofpart{$\implies$}

	Supongamos que b es entero sobre $A \implies \exists p(x)$ mónico $= a_0+a_1x+...+a_{n-1}x^{n-1}+x^n \in A[x]$ con coeficientes en $A$ tal que $p(b)=0$.

	Lo que queremos probar es que $1,b,...,b^{n-1}$ generan $A(b)$ como $A$-módulo. Es decir, sea $\alpha \in A(b) \implies \alpha=c_0+c_1b+...+c_rb^r$, lo comprobamos por casos:
	\begin{itemize}
		\item Si $r \leq n-1$ obvio.
		\item Si $r = n \implies$ reemplazamos $b^n=-(a_0+a_1b+...+a_{n-1}b^{n-1})$
		\item Si $r \geq n+1 \implies$ se reemplaza $b^n$ por esa expresión tantas veces como sea necesario. (Al ser polinomio mónico puedo despejar $b^n$)
	\end{itemize}

	Aquí se ve la importancia de que el polinomio sea mónico, que es lo que nos permite despejar $b^N$.

	\proofpart{$\Leftarrow$}

	Suponemos que $A[b]$ es un A-módulo finitamente generado, es decir, $\exists b_1,...,b_t \in A[b]$ tal que $A[b]=\{ a_1b_1+...+a_tb_t: a_i \in A \}$. Tenemos que ver que entonces b es entero sobre $A$.

	Vamos a probar que cada $\alpha \in A[b]$ es entero sobre A. Fijamos $\alpha \in A[b]$ y hacemos lo siguiente.Sabemos que $\alpha\cdot b_i$ vive en $A[b]$ así que,  sea $a_{ij}\in A$:

	\[
	\left. \begin{array}{c}
	\alpha b_1=a_{11}b_1+...+a_{1t}b_t \\
	\alpha b_2=a_{21}b_1+...+a_{2t}b_t \\
	\vdots \\
	\alpha b_n=a_{t1}b_1+...+a_{tt}b_t \\
	\end{array}
	\right\} \in A(b) \implies
	\begin{array}{c}
	0=(a_{11}-\alpha)b_1+...+a_{1t}b_t \\
	0=a_{21}b_1+(a_{22}-\alpha)b_2+...+a_{2t}b_t \\
	\vdots \\
	0=a_{t1}b_1+...+(a_{tt}-\alpha)b_t \\
	\end{array}
	\]

	\[
	\underbrace{
	\left( \begin{array}{cccc}
	a_{11}-\alpha & a_{12} & ... & a_{1t} \\
	a_{21} & a_{22} -\alpha & ... & a_{2t} \\
	\vdots & & & \\
	a_{t1} & a_{t2}& ... & a_{tt} - \alpha \\
	\end{array}
	\right)}_{D}
	\underbrace{
	\left( \begin{array}{c}
	b_1 \\
	\vdots \\
	\vdots \\
	b_t\\
	\end{array}
	\right)}_{b} =
	\left( \begin{array}{c}
	0 \\
	\vdots\\
	\vdots \\
	0\\
	\end{array}
	\right) \implies \det(D)= \pm(\alpha)^t+ a=0 \text{ con } a \in A
	\]

	Es decir, $det(D)$ es mónico con $\alpha$ como variable.

	Pero no sabemos si $det(D)=0$. Si $det(D)=0 \implies$ b es entero sobre A, ya que ya habríamos encontrado un polinomio mónico que es igual a 0 al sustituir $\alpha$.

	Entonces, vamos a ver por qué podemos afirmar, que efectivamente, $det(D)=0$.

	Si consideramos que $\K$ es un cuerpo sería más sencillo ya que  tendría una matriz cuadrada por un vector no nulo (ya que los $b_i$ no pueden ser todos 0) y sería obvio que el det(D) tendría que ser 0. Pero $A$ y $B$ son anillos arbitrarios.

	Multiplicamos por la izquierda por $ (D^T)^{Adj}$.

	$$ (D^T)^{Adj}\cdot D \cdot b = (D^T)^{Adj}\cdot \cls{0} $$

	Pero desde que dimos Álgebra lineal sabemos que:

	$$ (D^T)^{Adj}\cdot D \cdot b = \left( \begin{array}{cccc}
	det(D) & 0 & \dots & 0 \\
	0 & det(D) & \dots & 0 \\
	\vdots &  & &  \\
	0 & \dots & 0 & det(D)\\
	\end{array}
	\right) \left( \begin{array}{c}
	b_1 \\
	\vdots \\
	\vdots \\
	b_t\\
	\end{array}
	\right) =
	\left( \begin{array}{c}
	0 \\
	\vdots\\
	\vdots \\
	0\\
	\end{array}
	\right) \implies
	\begin{array}{c}
	det(D)b_1=0 \\
	\vdots\\
	det(D)b_t=0\\
	\end{array}
$$

Todavía no tenemos asegurado que $det(D)=0$ ya que podríamos estar trabajando en anillos que no son dominios de integridad (y tienen divisores de 0). Por tanto, seguimos:

	 Como $1 \in A[b]$, $\exists c_1,...,c_t \in A$ tales que $1=c_1b_1+...+c_tb_t \implies det(D)=det(D)(c_1b_1+...+c_tb_t)=0$ por lo que acabamos de ver.
\end{proof}

\begin{defn}[Extensión\IS finita]
	Diremos que una extensión $A \subset B$ es finita si $B$ es un $A$-módulo finitamente generado.
\end{defn}

\begin{defn}[Extensión\IS entera] Dados dos anillos $A ⊂ B$, se dice que es una extensión entera sobre $A$ si todo elemento $b ∈ B$ es entero sobre $A$.
\end{defn}

\textbf{Corolario:} Si $A\subset B$ es finita $\implies$ $A \subset B$ es entera.

Pero sin embargo no toda extensión entera es finita.
\begin{example}
	\begin{itemize}
		\item $\rac \subset \rac[\sqrt{2}]$ es finita y entera.
		\item $\rac \subset \rac[\sqrt{2}, \sqrt[3]{2}]$ es finita y entera.
		\item $\rac \subset \rac[\sqrt{2}, \sqrt[3]{2},..., \sqrt[p_1]{2},...,\sqrt[p_n]{2},...]$, con $p_i$ primo (no paro nunca de añadir) es entera pero no finita.
	\end{itemize}
\end{example}

\obs Sea $A \subset B$ una extensión y $b \in B$ es entero sobre $A \implies A \subset A[b]$ finita. (Y por el colorario tenemos $A\subset A[b]$ es entera, es decir que todo elemento en $A[b]$ es entero sobre $A$.)

\begin{defn}[Clausura\IS entera]
	Dada una extensión $A \subset B$ definimos $\cls{A}$ como el conjunto de todos los elementos de $b$ que son enteros sobre A.
\end{defn}

\obs $\cls{A}=\emptyset$?? No, $A \subset \cls{A}$. De hecho $\cls{A}$ es un anillo y es la clausura entera de $A$ en $B$. Si $\alpha, \beta \in B$, son enteros sobre $A \implies \alpha + \beta$ es entero sobre $A$. Y de hecho $A[\alpha, \beta]$ es una extensión finita y todos los elementos de $A[\alpha, \beta]$ son enteros. Por ejemplo $\alpha \cdot \beta$ o $\alpha + \beta$.

% Clase 17/3/16

%\section{Extensiones enteras}

	En cuerpos trabajábamos con las extensiones algebraicas, como por ejemplo $ℚ[i]$. En anillos hemos dado una noción similar, que será la de extensiones enteras.

	%\begin{defn}[Elemento\IS entero] Dado $A$ un anillo y $b ∈ A$, se dice que $a$ es un elemento entero si y sólo si existen $n ≥ 1$ y $\set{a_i}_{i = 0}^{n-1} ⊂ A$ tales que \[ a_0 + a_1 b + a_2 b^2 + \dotsb + b^n = 0\]

	%Esto es, $b$ es entero sobre $A$ si y sólo si es raíz de un polinomio mónico con coeficientes en $A$.
	%\end{defn}

	Así, podemos definir la extensión entera de anillos:

	%\begin{defn}[Extensión\IS entera] Dados dos anillos $A ⊂ B$, se dice que es una extensión entera sobre $A$ si todo elemento $b ∈ B$ es entero sobre $A$.
	%\end{defn}

	Ahora vamos a ver el comportamiento de estas extensiones con anillos cociente y parte multiplicativa:

	\begin{problem}
		Suponemos que $A \subset B$ es entera.
		Si $J \subset B$ es un ideal, entonces $A \subset B \rightarrow \frac{B}{J}$

		\ppart Comprobar que hay una inclusión:

		\( \frac{A}{J \cap A} \densein \frac{B}{J} \label{eq:ceros_problema1}\)

		\ppart Comprobar que la extensión (\ref{eq:ceros_problema1}) es entera.

		\solution
	\end{problem}

	\notacion $\hookrightarrow$ aparte de indicar una aplicación de un conjunto A en otro B, indica que $A \subset B$.

	\begin{problem}
		Suponemos que $A \subset B$ es entera.
		Sea $S \subset A$ una parte multiplicativa.

		\ppart Comprobar que $S \subset B$ es una parte multiplicativa

		\ppart Comprobar que hay una inclusión:

			\(S^{-1} A \subset S^{-1}B \label{eq:ceros_problema2} \)

		\ppart Comprobar que la extensión (\ref{eq:ceros_problema2}) también es entera.

		\solution
	\end{problem}

	Con esto comprobamos extensiones enteras entre dominios de integridad.

	\begin{theorem}
		Sean $A$ y $B$ dominios de integridad con $A \subset B$. Supongamos que $A \subset B$ es entera. Entonces $A$ es un cuerpo $\Leftrightarrow B$ es un cuerpo.
	\end{theorem}

	\begin{proof}

		\proofpart{$\Rightarrow$}

		Supongo que $A$ es un cuerpo. Sea $b \neq 0, b \in B$. ¿$b^{-1} \in B$?

		Como $A \subset B$ es entera entonces $b$ es entero$/A$. Entonces $\exists p(x) \in A[x]$ mónico con $p(b)=0$

		Suponiendo que $p(x)=a_0 + a_1x + … + a_{n-1}x^{n-1} + x^n, a_i \in A$. Entonces $a_0 + a_1 b + … + a_{n-1}b^{n-1} + b^n = 0$.

		¿Podemos suponer $a_0 \neq 0$?

		Si $a_0 = 0 \implies  a_1 b + … + a_{n-1}b^{n-1}+b^{n} = 0$

		\[ \underbrace{b}_{\neq 0} \underbrace{(a_1 + … a_{n-1}b^{n-2} + b^{n-1})}_{=0} = 0\]

		Como la segunda parte es 0 $\implies$ podría haber tomado como polinomio mónico $p'(x) = a_1 + a_2 x + … + a_{n-1}x^{n-2} + x^{n-1}$, ya que $p'(b)=0$.

		Iterando el proceso llegamos a que:

		\begin{itemize}
			\item O bien $b$ es raíz de un polinomio mónico con coeficientes en $A$ de grado 1 $\Rightarrow b \in A$ (porque $b \neq 0$) $\implies$ por ser A cuerpo $b^{-1} \in A$  $\implies b \in B$.

			\item O bien $b$ es raíz de un polinomio mónico con coeficientes en $A$ de grado $\geq 2$ con término constante $\neq 0$.

		\end{itemize}

		Por lo tanto podemos suponer que tenemos $p(x) = a_0 + a_1x + … + a_{n-1}x^{n-1} + x^n$ con $a_0 \neq 0$. Y nos quedaría que:

		\begin{align*}
		b^n + a_{n-1}b^{n-1}+…+a_1b + a_0 &= 0 \\
		b^n + a_{n-1}b^{n-1}+…+a_1b &= -a_0 \\
		b(a_{n-1}b^{n-1}+…+a_1) &= \underbrace{-a_0}_{\neq 0} \in A (\text{cuerpo}) \\
		b \underbrace{(-a_0^{-1})(b^{n-1}+a_{n-1}b^{n-2}…+a_1)}_{\in B} &= 1 \\
		\end{align*}

		Luego $b$ tiene inverso en $B$, entonces $B$ es un cuerpo.

		\proofpart{$\Leftarrow$}

		Suponemos $B$ cuerpo. Sea $a \neq 0,a \in A$ ¿$a^{-1} \in A$? Sabemos que $a^{-1} \in B \Rightarrow a^{-1}$ es entero$/A$.

		Entonces $\exists p(x) \in A[x]$: $p(x) = a_0 + a_1x + … + a_{n-1}x^{n-1} + x^n $, tal que $p(a^{-1})=0$

		\[ 0 = a_0 + a_1(a^{-1}) + … + a_{n-1}(a^{-1})^{n-1} + (a^{-1})^n \]

		Multiplicando por $(a^{n-1})$:

		\[ 0 = a_0 a^{n-1} + a_1 a^{n-1} a^{-1} + … + a_{n-1} a^{n+1} a^{n-1} + a^{-n} a^{n-1} \]

		\[ 0 = \underbrace{a_0 a^{n-1}}_{\in A} + \underbrace{a_1 a^{n-2}}_{\in A} + … + \underbrace{a_{n-1}a^0}_{\in A} + a^{-1} \Rightarrow a^{-1} \in A \]


	\end{proof}

	\begin{example}
		Veamos el caso $n = 3$.

		\[ a_0 + a_1 x + a_2 x^2 + x^3 \]
		\[ a_0 + a_1 a^{-1} + a_2 (a^{-1})^2  + (a^{-1})^3 = 0 \]

		Multiplicando por $a^2$:

		\[ \underbrace{a_0 a^2 + a_1 a + a_2 1}_{\in A} + a^{-1} = 0\]
	\end{example}

	Recordamos el teorema que hemos demostrado:

	\[ A \eqreason[\subset]{entera} B \quad A,B \text{ D.I. }\]
	\[ A \text{ cuerpo} \Leftrightarrow B \text{ cuerpo} \]

	\begin{example}[1]
		Si $A \subset B$ no es entera, no podemos aplicar el teorema y tenemos, por ejemplo:
		\begin{itemize}
			\item $\underbrace{\ent}_{\text{no cuerpo}} \subset \underbrace{\rac}_{\text{ cuerpo}}$
			\item $\underbrace{\rac}_{\text{ cuerpo}} \subset \underbrace{\rac[X]}_{\text{no cuerpo}}$. Recordemos que $\rac[X]$ no es un cuerpo ya que los polinomios de grado mayor o igual que 1 no tienen inverso.
		\end{itemize}
	\end{example}

	\begin{example}[2]
		\[ \rac \eqreason[\subset]{entera} \quot{\rac[x]}{<x^2-1>} = \rac[\cls{x}] \]
		\[ \cls{x}^2 - 1 \in \rac[\cls{x}] \]

		Para ver que sea entera lo único que tenemos que comprobar es que $\cls{x}$ sea entera spbre $\rac$ ya que $\rac \subset \rac[\cls{x}]$, y lo es ya que  $\cls{x}$ se anula en el polinomio mónico $\cls{x}^2 - 1$.

		¿Es $\quot{\rac[x]}{<x^2-1>}$ cuerpo? No porque

		\[ \underbrace{(\gor{x-1})}_{\neq 0} \underbrace{(\gor{x+1})}_{\neq 0}  = \cls{x^2-1} = 0\]

		Es decir $\rac[x]/<x^2-1>$ no es D.I. y por tanto no es un cuerpo.
	\end{example}

	\begin{example}[3]
		Pero si podemos usar el teorema y coger $\rac$ y empezar a añadir elementos algebraicos (recordemos que como $\rac$ es un cuerpo, elementos enteros y algebraicos es lo mismo).

		\[ \rac \eqreason[\subset]{entera} \rac[\sqrt{2}, \underbrace{e^{\frac{2\pi i}{3}}}_{\subset \cplex}] \]
		\[ \Rightarrow \rac[\sqrt{2}, e^{\frac{2\pi i}{3}}] \text{ es un cuerpo.}\]

		Así que, por el teorema, como son los dos dominios de integridad, y $\rac$ es un cuerpo, y la extensión es entera, entonces $\rac[\sqrt{2}, e^{\frac{2\pi i}{3}}]$ es un cuerpo.
	\end{example}

	\notacion Fijarse que se han usado corchetes y no paréntesis, y con corchetes no se admiten denominadores. Me explico mejor, tenemos dos casos, sea $A$ un anillo y $b\notin A$ un elemento cualquiera. \label{notacion: 1}
	\begin{itemize}
		\item $A[b]=\{ \alpha+\beta b: \alpha, \beta \in A \}$ es el anillo más pequeño que contiene a A y a b.
		\item $A(b)=\{ \frac{\alpha_1+\beta_1 b}{\alpha_2+\beta_2 b}: \alpha_1, \alpha_2, \beta_1, \beta_2 \in A \}$ es el cuerpo más pequeño que contiene a A y a b (\textbf{permite la división}).
	\end{itemize}


	\begin{corol}\label{cor:1}
		Sea $K \subset L$ una extensión de cuerpos. Sea $\alpha \in L$. Entonces $\alpha$ es un algebraico$/K \Leftrightarrow K[\alpha]$ es un cuerpo, siendo $K[\alpha]$ el anillo más pequeño que contiene a $k$ y $\alpha$.
	\end{corol}

	\begin{proof}

	\proofpart{$\implies$}

		Supongamos que $\alpha$ es algebraico$/K \Rightarrow$ por estar en un cuerpo $ \alpha$ es entero$/K \Rightarrow K \subset K[\alpha]$ es entera (porque es finita).

		\[ \underbrace{K}_{\text{D.I.}} \eqreason[\subset]{entera} \underbrace{K[\alpha]}_{\text{D.I.}} \subset L\]

		($\K[\alpha]$ es D.I. porque vive dentro de $L$ que es un cuerpo)

		Entonces, por el teorema, $K[\alpha]$ es un cuerpo.

	\proofpart{$\impliedby$}

		Suponemos $K[\alpha]$ es un cuerpo ¿$\alpha$ es algebráica$/K$?

		\begin{align*}
			\text{Sea } \phi: K[x] &\longrightarrow K[\alpha] \text{ homomorfísmo de }K\text{-anillos}\\
			x &\longrightarrow \alpha \\
			r\in K &\longrightarrow r
		\end{align*}

		¿Es $\phi$ sobreyectivo? Sí, porque los elementos de $K[\alpha]$ se construyen con sumas de productos por $\alpha^n \rightarrow $ es claro que lo es.

		Entonces, por el primer teorema de isomorfía \ref{thm:IsomorfiaAnillos1}:

		\[ \frac{K[x]}{\ker \phi} \cong K[\alpha]\]

		$p \in \ker \phi \Leftrightarrow p(\alpha) = 0 $. Supongamos $\ker \phi = \{0\}$ ($\phi$ inyectiva $\Rightarrow \phi$ isomorfismo).

		\[ \rightarrow \underbrace{K[x]}_{\text{no cuerpo}} \cong \underbrace{K[\alpha]}_{\text{cuerpo}}  \]

		Lo cual es una contradicción, luego $\ker \phi \neq \{0\}$, y $\exists p(x) \in K[x]$ tal que $p(\alpha) = 0 \Rightarrow \alpha $ es algebraica$/K$.
	\end{proof}



%Clase 30/03/2016
Un par de conclusiones:
\begin{itemize}
\item Sea  $\K \subset \K[\alpha]$. Si $\alpha$ no es algebraico sobre K, entonces $\K[\alpha]\simeq \K[x]$ (anillo de polinomios). Y $\K[\alpha]$ no será un cuerpo ya que $\K[x]$ tampoco lo es.
\item Si $\alpha$ es algebraico sobre K, entonces $\K[\alpha] \simeq \quot{\K[x]}{\gen{p(x)}}$, que es un cuerpo (con p irreducible).
\end{itemize}

Con esto ya podemos probar el Teorema básico para n=1.
\begin{proof}
	Sea $M$ un ideal maximal en $\K[x]$ (algebraicamente cerrado). Entonces, como $M$ es maximal, $\quot{\K[x]}{M}$ es un cuerpo. Este cuerpo cumple que:
	$$ \K  \subset \K[\cls{x}] = \quot{\K[x]}{M}$$

	Por hipótesis $\K[\cls{x}]$ es un cuerpo. Por tanto, por el corolario anterior (\ref{cor:1}) $\cls{x}$ es un elemento algebraico sobre $\K$.

	Edu nos pregunta que por qué $\K[\cls{x}] = \quot{\K[x]}{M}$, es decir, por qué solo depende de la clase de x, ya que al cocientar por $M$ hay otras cosas que tienen su clase y tal. Y Ana nos pone este ejemplo:

	\begin{example}
		Tomamos $\rac \subset \quot{\rac[x]}{\gen{x^2+1}} = \rac[\cls{x}]$.

		Da igual que $p(x)$ por el que dividas sea irreducible que no.

		Este ejemplo no me ha aclarado nada, así que nos lo creemos.
	\end{example}

	Seguimos: la hipótesis del enunciado es que $\K$ es un cuerpo algebraicamente cerrado, es decir que contiene a todos los elementos que son algebraicos sobre $\K$. Por tanto $\cls{x}=a \in \K \implies x-a \in M \implies \gen{x-a} \subset M$. ($\cls{x}-a = \cls{0}$)

	Pero $\gen{x-a}$ ya es maximal $\implies M = \gen{x-a}$
\end{proof}

%Queremos ver este teorema
%\begin{theorem}
%	Sea $A\subset B$ una extensio de dominios de integridad, suponemos que $A \subset B$ es entera, entinces A es un cuerpo si y solo si B es un cuerpo.
%\end{theorem}

\begin{prop}
	\textbf{Paso 1 para probar el caso general del teorema básico:}
	Sea $A \subset B$ una extensión de dominios de integridad. Y sean $\alpha_1,...,\alpha_s \in B$. Supongamos $A[\alpha_1,...,\alpha_s]$ y que este anillo es algebraico sobre A y es un cuerpo.

	Entonces existe $t \in A, t \neq 0$ tal que $A_{\{t\}}$ (A localizado en t) es un cuerpo. (si me dijeran que la extensión es entera ya tendría que A es un cuerpo, pero solo tenemos que es algebraica)
\end{prop}

\begin{proof}
Por hipótesis cada $\alpha_i$ es algebraico sobre A. Para cada i, $\exists p_i(x) \neq 0 \in A[x]$ tal que $p_i(\alpha_i)=0$.

Para i=1: $p_1(x)=a_{01}+a_{11}x+...+a_{n_1 1}x^{n_1}$ con $a_{i1}\in A$ , entonces $p_1(\alpha_1)=0$

Para i=2: $p_2(x)=a_{02}+a_{12}x+...+a_{n_2 2}x^{n_2}$ con $a_{i2}\in A$ , entonces $p_2(\alpha_2)=0$

Para i=s: $p_s(x)=a_{0s}+a_{1s}x+...+a_{n_s s}x^{n_s}$ con $a_{is}\in A$ , entonces $p_s(\alpha_s)=0$

Me encuentro con que los elementos que acompañan a la x de mayor grado no son invertibles.

Cogemos $t=a_{n_11} a_{n_2 2}....a_{n_s s}\neq 0$. Considero $A \rightarrow A_{\{t\}}$.

\begin{tikzpicture}
\matrix (m) [matrix of math nodes,row sep=4em,column sep=2em,minimum width=2em]
{
	A &  &  A[\alpha_1,...,\alpha_n]\\
	& A_{\{t\}} &  \\};
\path[-stealth]
(m-1-1) edge node [above] {$\subset$} (m-1-3)
(m-1-1) edge node [below] {$\subset$} (m-2-2)
(m-2-2) edge node [right] {$\textcolor{red}{\subset} \text{ y } \exists \text{ por la propiedad universal de la localización (\ref{thm:PropUniversalLoc})} $} (m-1-3);
\end{tikzpicture}

\textcolor{red}{Parra: No veo donde cumplimos la hipótesis de f(s) invertible de la prop universal de la localización} \textcolor{blue}{Parra: quizás sea porque $A[\alpha_1,...,\alpha_n]$ es un cuerpo por hipótesis y todos los elementos en un cuerpo son invertibles :D}

Como $A_{\{t\}} \subset A[\alpha_1,...,\alpha_n]$ es una extensión entera y $A[\alpha_1,...,\alpha_n]$ es un cuerpo, entonces $A_{\{t\}}$ es un cuerpo.
\end{proof}

\begin{example}
	\begin{enumerate}
		\item Cogemos $\ent \subset \rac$ extensión. ¿Es algebraica? Siiii. $\rac$ es un cuerpo. Pero la extensión no es de tipo finito. ¿Puedo aplicar el teorema?  Noooo
		\item
		Recordemos que $\ent_{\gen{3}}$ es hacer invertibles los elementos que no son múltiplos de 3. Más concretamente: es un subanillo de los racionales en el que sólo nos quedamos con aquellos números racionales que admiten una escritura cuyo denominador no es un múltiplo de 3.

		Entonces para conseguir llegar a $\rac[\sqrt{2}]$ solo tendría que añadir 1/3 y $\sqrt{2}$ a $\ent_{\gen{3}}$. Así tendría:

		Cogemos $\pi: \ent_{\gen{3}} \hookrightarrow \rac[\sqrt{2}]=
		\underbrace{\ent_{\gen{3}}\left[\frac{1}{3}, \sqrt{2}\right]}_{\text{cuerpo}}$

		Entonces $\ent_{\gen{3}} \subset	 \ent_{\gen{3}}\left[\frac{1}{3},\sqrt{2}\right]$ es una extensión algebraica finita. He añadido un número finito de elementos algebraicos sobre $\ent_{\gen{3}}$ (1/3 y $\sqrt{2}$) y puedo aplicar entonces la proposición. Y puedo concluir que $\ent_{\gen{3}}$ localizado en el ideal primo $\gen{3}$ es un cuerpo.
	\end{enumerate}
\end{example}

% clase del 31/03/2016
\textbf{Ayudándonos de esta proposición seguimos con la demostración del teorema básico}
\begin{proof}
	Por inducción en r:
	\begin{enumerate}
		\item $r=1$: $\K \subset \K[\alpha_1] \implies \alpha_1$ algebraico sobre $\K$.
		\item Supongamos que el teorema vale para una álgebra de tipo finito sobre un cuerpo generado por $r-1$ elementos.
		\item \textbf{Caso general:}
		$$\K \hookrightarrow \K[\alpha_1] \hookrightarrow \K[\alpha_1,...,\alpha_r]=\K[\alpha_1]\K[\alpha_2,...,\alpha_r]$$

		Pero no sabemos que $\K[\alpha_1]$ sea un cuerpo, pero sí sabemos que es un dominio de integridad, por tanto,se que $\{0\}$ es un ideal primo y puedo localizar en el complementario. Vamos a pasar de $\K[\alpha_1]$ a $\K(\alpha_1)$ que es localizar en $\quot{\K(\alpha_1)}{\{0\}}$, que sí es un cuerpo. Lo único que estamos haciendo es hacer invertibles todos los elementos de $\quot{\K(\alpha_1)}{\{0\}}$ (Recordemos la diferencia entre A[b] y A(b): \ref{notacion: 1})

		¿Pero se puede hacer $\K(\alpha_1) \hookrightarrow \K[\alpha_1]\K[\alpha_2,...,\alpha_r]$?  Sí, usamos la propiedad universal de la localización. Como $\K[\alpha_1,...,\alpha_r]$ es un cuerpo (por hipótesis), todo elemento no nulo de $\K[\alpha_1]$ es invertible en $\K[\alpha_1,...,\alpha_r]$, luego por la propiedad universal de la localización hay un homomorfismo de $\K(\alpha_1)$ en $\K[\alpha_1,...,\alpha_r]$, que además es inyectivo porque $\K(\alpha_1)$ es un cuerpo.

		\begin{tikzcd}
			\K \arrow[hookrightarrow]{r}{}
			& \K [\alpha_1] \arrow[hookrightarrow]{dr} \arrow[hookrightarrow]{rr} & &  \K[\alpha_1][\alpha_2,...,\alpha_n]\\
			& & \K(\alpha_1) \arrow[hookrightarrow]{ur}
		\end{tikzcd}

		Razonamos el diagrama: por hipótesis $\K[\alpha_1][\alpha_2,...,\alpha_n]$ es un cuerpo. Además sabemos que  $\K(\alpha_1)$ también es un cuerpo. Entonces estoy en la hipótesis de inducción, y por tanto $\alpha_2,...,\alpha_r$ son algebraicos sobre $\K(\alpha_1)$.

		Pero yo lo quiero probar que $\alpha_1,..., \alpha_r$ son algebraicos sobre $\K$. Por tanto, aún no hemos acabado.

		La extensión $\K[\alpha_1] \hookrightarrow \K(\alpha_1)$ es algebraica? Sí, en $\K[\alpha_1]$ tengo expresiones polinómicas en $\alpha_1$ y en $\K(\alpha_1)$ lo que tengo son cocientes de expresiones polinómicas en $\alpha_1$;  así que podemos usar el mismo argumento que me dice que $\rac$ es algebraico sobre $\ent$.

		Y la extensión $\K(\alpha_1) \hookrightarrow \K[\alpha_1,...,\alpha_r]$ también es algebraica. Por tanto:

		$$ \K \subset \K[\alpha_1] \stackrel{\text{algebraico tipo finito}}{\hookrightarrow} \K[\alpha_1,...,\alpha_r]$$

		Entonces, por la proposición, $\exists t \in \K[\alpha_1]$ tal que $\K[\alpha_1]_{\{t\}}$ es un cuerpo.

		¿Como es $\K[\alpha_1]$? O $\alpha_1$ es algebraico sobre $\K$ o no. Supongamos que no lo es, si a un cuerpo le añado un elemento que no es algebraico entonces $\K[\alpha_1] \simeq \K[x]$ (es como un anillo de polinomios). Hemos probado que existe un polinomio $p(x)\neq 0 \in \K[x]$ tal que $\K[x]_{\{p(x)\}}$ es un cuerpo.

		\begin{example}
			Sea $\K[x]_{\{x\}}$, si cogemos $\frac{q(x)}{x^n}$ hay montones de polinomios que no son invertibles.
		\end{example}

		Contradicción. Luego necesariamente $\alpha_1$ es algebraico sobre $\K$
	\end{enumerate}
\end{proof}

Entonces hemos probado que:
\begin{enumerate}
	\item $\alpha_1$ es algebraico sobre $\K$, y eso implica que $\K(\alpha_1)=\K[\alpha_1]$.
	\item la hipótesis de inducción es que $\alpha_2,...,\alpha_r$ son algebraicos sonre $\K(\alpha_1)=\K[\alpha_1]$
	\item La extensión $ \K  \hookrightarrow \K[\alpha_1,...,\alpha_r]$ es algebraica.
\end{enumerate}



\textbf{Terminamos la demostración del teorema básico}
\begin{proof}
 $$ \K \subset \underbrace{\quot{\K[x_1,...,x_n]}{m}}_{cuerpo} = \K[\cls{x_1},...,\cls{x_n}]$$

 Esto implica, por el teorema anterior, que $\cls{x_1},...,\cls{x_n}$ son algebraicos sobre $\K$.

 Como $\K$ es algebraicamente cerrado $\implies$ $\cls{x_1},...,\cls{x_n} \in \K$, es decir:
 $$ \quot{\K[x_1,...,x_n]}{m}= \K $$

 Entonces $\exists a_1,...,a_n \in \K$ tal que $\cls{x_i}=a_i$ en $\quot{\K[x_1,...,x_n]}{m}$, $\implies x_i-a_i \in m\subset \K[x_1,...,x_n] \implies \underbrace{\gen{x_1-a_1,...,x_n-a_n}}_{\text{ya es maximal entonces son iguales}} \subset m$
\end{proof}

\textcolor{blue}{\textbf{HASTA AQUÍ EL SEGUNDO PARCIAL}}

%clase del  dia 4 de Abril
Algunas consecuencias del teorema de  los ceros de Hilbert:

$K=\cls{K}, I\subset K[x_1,...,x_n] \implies \I(\V(I))=\sqrt{I}$

\begin{example}
	En $\real[x,y,z]$, ($\real$ no es algebraicamente cerrado) $J=\gen{x^2+y^2z^2}$. Es $X=\V(J)$ irreducible? $\V(J):x^2+y^2z^2=0$, tomamos $x=0, yz=0$.

	$X=\V(J)=\V(\gen{x,y})\cup\V(\gen{x,z})$. Por tanto X no es irreducible.

	Por otro lado, $J$ es un ideal primo en $\real[x,y,z]$ porque está generado por $x^2+y^2z^2$ que es irreducible en $\real[x,y,z]$. Porque su factorización (única) en $\cplex[x,y,z]$ es $x+iyz)(x-iyz)$.

	A lo que tenemos que mirar es al ideal de definición de X, que es $\I(X)$ y que se cumple que $J \subsetneqq \I(X)$.

	Afirmación: $ \I(X)=\gen{yz,x}$, vamos a probarlo:
	La inclusión $\supset$ es clara. Basta observar que $x$ se anula en los puntos de $X$ y que $yz$ también se anula en los puntos de $X$.

	Vamos a probar la otra inclusión $\subset$: Sea $p(x,y,z) \in \I(X)$. Como $\gen{yz,x} \subset \I(x)$, si $\I(X) \neq \gen{yz,x} \implies \cls{\I(X)}$ no es el ideal cero en:

	$$ \quot{\real[x,y,z]}{\gen{x,yz}} \supset \cls{\I(X)} $$

	Supongamos que $p(x,y,z) \in \I(X)$, pero que $p(x,y,z) \notin \gen{x,y,z} \implies \cls{p(x,y,z)} \neq 0 \in \quot{\real[x,y,z]}{\gen{x,yz}} \simeq \quot{\real[y,z]}{\gen{yz}}$.

	Entonces $\cls{p(x,y,z)}=\underbrace{a_0}_{\in \real}+yq(y)+zs(z) \implies$ en $\real[x,y,z]$ podemos suponer que $\underbrace{p(x,y,z)}_{\in \I(X)}=a_0+yq(y)+zs(z)$

	Como $p(0,0,0)=0 \implies a_0=0$, pero además, como $p(x,y,z)$ se anula en X $\implies$ se anula en todos los puntos de la forma $(0,a,0)$ con $a\in \real \implies yq(y)$ se anula en toda la recta real $\implies yq(y)=0$, y como $p(x,y,z)$ se anula en todo punto $(0,0,b)$ con $b\in \real \implies zs(z)=0$.
\end{example}


\begin{example}
	Sea $\K=\cplex$ un cuerpo algebraicamente cerrado, cogemos $\K[x_1,...,x_n]$. Sea $J$ un ideal primo. Afirmación: $X=\V(J)$ es irreducible (seguro que la variedad que determina va a ser irreducible). Esto ocurre gracias al teorema de los 0's de Hilbert, que dice que el ideal de definicón de X, $\underbrace{\I(X)}_{=\I(\V(J))}=\sqrt{J}=J$ que es primo.
\end{example}

\textbf{Conlcusión:} Si $\K$ es algebraicamente cerrado y $J\subset \K[x_1,...,x_n]$ es primo, $\implies \V(J)$ es irreducible.

Otra consecuencia del teorema de los ceros de Hilbert.
Vimos que si $X \subset \Akn$ (vale para cualquier $\K$) es una v.a.a. entonces $X$ se escribe de manera única no redundante como unión de v.a.a irreducibles.
Tenemos $X \subset \Akn$ y su ideal de definición $\I(X) \subset \K[x_1,...,x_n]$. $I(X)$ es radical.

$\I(X)=p_1 \cap ... \cap p_r$. Los $p_1$ son primos minimales con la condicion de contener a $\I(X)$.

$\underbrace{\V(\I(X))}_{=X \text{por ser v.a.a.}}=\V(p_1\cap...\cap p_r)$

Y $X=\V(p_1)\cup...\cup\V(p_r)$.

Supongamos $K=\cls{K}$ y $X=\V(p_1)\cap...\cap\V(p_r)$. Si $K$ es algebraicamente cerrado todos los $\V(p_i)$ son irreducibles. Osea que habré escrito X como unión de v.a.a. irreducibles.

Afirmación: esta es la descomposición de $X$ como unión finita (no reducible) de v.a.a. irreducibles.

\begin{proof}
	\begin{itemize}
		\item si $r=1$ está claro, ya que $\I(X)=p_1 \implies X$ es irreducible.
		\item Supongamos $r \geq 2$. $X=\V(p_1)\cup...\cup \V(p_r)$. Veamos que $\forall i=1,...,r$ tenemos que $X \supsetneqq \V(p_i)$.

		Supongamos que $\exists i \in \{1,...,r\}$ tal que $X=\V(p_i)\implies \I(X)=\I(\V(p_i))=p_i \implies \I(X)$ es primo $\implies p_i \subset p_j$ para $i \neq j$, pero eso contradice el hecho de que $p_1,...,p_r$ son minimales con la condición de contener a $\I(X)$. Por tanto, necesariamente tenemos que $X \supsetneqq \V(p_i)$.

		Vemos ahora que la escritura $ X=\V(p_1)\cup...\cup\V(p_r) $ no es redundante, es decir que no existe $i \neq j$ con $\V(p_i) \subset \V(p_j)$. Supongo que $\exists i \neq j$ tal que $\V(p_i)\subset \V(p_j) \implies \underbrace{\I(\V(p_i))}_{=p_i} \supset \underbrace{\I(\V(p_j))}_{=p_j}$ es decir que $p_i \supset p_j$

		Imposible si las $p_1,...p_r$ eran minimales con la condición de contener a $\I(X)$.
	\end{itemize}



\end{proof}
