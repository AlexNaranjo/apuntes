% -*- root: ../AlgebraConmutativa.tex -*-
\chapter{Teorema de los ceros de Hilbert}

Hemos jugado en $\Akn$ y en $\K[x_1,\dotsc,x_n]$. Y sabemos movernos de un sitio a otro:

\begin{align*}
	\Akn & \longmapsto  \K[x_1,...,x_N] \\
	X & \longmapsto  \I(X) \\
	\V(J) & \leftarrow  J \\
\end{align*}

\begin{itemize}
	\item Si cojo $X\in \Akn$ v.a.a., puedo ir a $\I(X) \in \K[x_1,...,x_n]$ y volver a $\V(\I(X)) \in \Akn$ que es igual a $X$ por ser una v.a.a..

	Tendría el siguiente diagramilla:

	\item Sin embargo, si cojo $J$ ideal radical en $\K[x_1,...,x_n]$, puedo ir a $\V(J)$ en $\Akn$, y volver a $\I(\V(J)) \supseteq J$. Pero puede pasar que la inclusión sea estricta:
	\begin{example}
		\textcolor{red}{Julian majo pon esto mas bonito}
		\begin{enumerate}
			\item Ejemplo de inclusión estricta:
			\begin{align*}
				\A^1_{\real} & \longmapsto  \real[x] \\
				\V(\gen{x^2+1}) & \leftarrow  \gen{x^2+1} \\
				\shortparallel & \\
				\emptyset & \longmapsto  \I(\{ \emptyset \})=\real[x] \\
			\end{align*}
			\item Otro ejemplo de inclusión estricta:

			Cogemos en $\real[x,y]$ e ideal  radical $\gen{(x-1)^2+(y-2)^2}$, que es primo (y por tanto radical) porque $p(x,y)=(x-1)^2+(y-2)^2$ es irreducible en $\real[x,y]$.

			Esto es así porque $\real[x,y] \subset \cplex[x,y]$  que es un dominio de factorización única.

			Y en $\cplex[x,y]$ tenemos que $p(x,y)=(x-1)^2+(y-2)^2=((x-1)+i(y-2))((x-1)-i(y-2))$. Por tanto no se puede factorizar de otra forma.

			\begin{align*}
				\A^2_{\real} & \longmapsto  \real[x,y] \\
				\V(J) & \leftarrow  \gen{(x-1)^2+(y-2)^2}=J \\
				\shortparallel & \\
				X=\{1,2\} & \longmapsto  \I(X)=\gen{x-1,y-2}\\
			\end{align*}

			Y obviamente $\gen{(x-1)^2+(y-2)^2} \subsetneqq \gen{x-1,y-2}$
		\end{enumerate}
	\end{example}
\end{itemize}

\textbf{Conclusión:} No hay un diccionario entre v.a.a. en $\Akn$ e ideales radicales en $\K[x_1,...,x_n]$.

\begin{theorem}[Teorema de los ceros de Hilbert]\label{teor:tma_0_hilbert}
	Si $\K$ es algebraicamente cerrado y $J \subset K[x_1,...,x_n]$ es radical, entonces $J=\I(\V(J))$
\end{theorem}



\textbf{Corolario:}	Si $\K$ es algebraicamente cerrado, hay una correspondencia biyectiva entre variedades algebraicas afines en $\Akn$ e ideales radicales en $\K[x_1,...,x_n]$

Nuestro objetivo en este tema es probar el teorema de los ceros de Hilbert \ref{teor:tma_0_hilbert}. Para ello, vamos a utilizar algunas proposiciones y teoremas:

\begin{theorem}[Teorema básico]
	Sea $\K$ un cuerpo algebraicamente cerrado, entonces todo ideal maximal en $\K[x_1,...,x_n]$ es de la forma $\gen{x_1-a_1,...,x_n-a_n}$ con $a_1,...,a_n \in \K$
\end{theorem}

\obs En $\K[x_1,...,x_n]$ todo ideal de la forma $\gen{x_1-a_1,...,x_n-a_n}$ es maximal aunque $\K$ no sea algebraicamente cerrado ya que:
$$ \quot{\K[x_1,...,x_n]}{\gen{x_1-a_1,...,x_n-a_n}} \simeq \K $$

\obs Supongamos $\rac = \K$. Tomamos $\rac[x,y]$, entonces $\gen{x-1,y^2-2}$ es maximal ya que $\quot{\rac[x,y]}{\gen{x-1,y^2-2}} \simeq \rac(\sqrt{2})$  que es un cuerpo.

Pasa lo mismo con $\gen{x-1,y^7+3y^2+3}$ o con $\gen{x^5+5x+5, y^7+3y^2+3}$.

\begin{theorem}[Teorema débil]
	Sea $\K$ un cuerpo algebraicamente cerrado y sea $I$ un ideal propio $\subsetneqq \K[x_1,...,x_n]$. Entonces $\V(I) \neq \emptyset$.
\end{theorem}

\begin{example}
	Tomamos en $\real[x]$ el conjunto $\V(\gen{x^2+5})$ \textcolor{red}{¿?¿?¿?}
\end{example}

Demostramos el teorema débil:
\begin{proof}
	Sea $I \subsetneqq \K[x_1,...,x_n] \implies \exists M$ maximal en $\K[x_1,...,x_n]$ tal que $I \subset M$. Por el teorema básico, $M$ es de la forma $\gen{x_1-a_1,...,x_n-a_n}$ para ciertos $a_1,...,a_n \in \K$ $\implies \V(I) \supset \V(M)=\{a_1,...,a_n\}$ luego $\V(I)\neq \emptyset$.
\end{proof}

Demostramos el teorema de los ceros de Hilbert usando el teorema débil:
\begin{proof}
	Queremos probar que $\I(\V(I))=\sqrt{I}$.

	Siempre tenemos (incluso si $\K$ no es algebraicamente cerrado) que $\sqrt{I} \subset \I(\V(I))$. Por tanto nos falta probar que $\I(\V(I)) \subset \sqrt{I}$.
\end{proof}

% Clase 17/3/16

\section{Extensiones enteras}

	En cuerpos trabajábamos con las extensiones algebraicas, como por ejemplo $ℚ[i]$. En anillos daremos una noción similar, que será la de extensiones enteras.

	\begin{defn}[Elemento\IS entero] Dado $A$ un anillo y $b ∈ A$, se dice que $a$ es un elemento entero si y sólo si existen $n ≥ 1$ y $\set{a_i}_{i = 0}^{n-1} ⊂ A$ tales que \[ a_0 + a_1 b + a_2 b^2 + \dotsb + b^n = 0\]

	Esto es, $b$ es entero sobre $A$ si y sólo si es raíz de un polinomio mónico con coeficientes en $A$.
	\end{defn}

	Así, podemos definir la extensión entera de anillos:

	\begin{defn}[Extensión\IS entera] Dados dos anillos $A ⊂ B$, se dice que es una extensión entera sobre $A$ si todo elemento $b ∈ B$ es entero sobre $A$.
	\end{defn}

	\begin{problem}
		Suponemos que $A \subset B$ es entera.
		Si $J \subset B$ es un ideal, entonces $A \in B \rightarrow \frac{B}{J}$

		\ppart Comprobar que hay una inclusión:

		\( \frac{A}{J \cap A} \densein \frac{B}{J} \label{eq:ceros_problema1}\)

		\ppart Comprobar que la extensión (\ref{eq:ceros_problema1}) es entera.

		\solution
	\end{problem}

	\begin{problem}
		Suponemos que $A \subset B$ es entera.
		Sea $S \subset A$ una parte multiplicativa.

		\ppart Comprobar que $S \subset B$ es una parte multiplicativa

		\ppart Comprobar que hay una inclusión:

			\(S^{-1} A \subset S^{-1}B \label{eq:ceros_problema2} \)

		\ppart Comprobar que la extensión (\ref{eq:ceros_problema2}) también es entera.

		\solution
	\end{problem}

	Con esto comprobamos extensiones enteras entre dominios de integridad.

	\begin{theorem}
		Sean $A$ y $B$ dominios de integridad con $A \subset B$. Supongamos que $A \subset B$ es entera. Entonces $A$ es un cuerpo $\Leftrightarrow B$ es un cuerpo.
	\end{theorem}

	\begin{proof}

		\proofpart{$\Rightarrow$}

		Supongo que $A$ es un cuerpo. Sea $b \neq 0, b \in B$. ¿$b^{-1} \in B$?

		Como $A \subset B$ es entera entonces $b$ es entero$/A$. Entonces $\exists p(x) \in A[x]$ mónico con $p(b)=0$

		Suponiendo que $p(x)=a_0 + a_1x + … + a_{n-1}x^{n-1} + x^n, a_i \in A$. Entonces $a_0 + a_1 b + … + a_{n-1}b^{n-1} + b^n = 0$.

		¿Podemos suponer $a_0 \neq 0$?

		Si $a_0 = 0  a_1 b + … + a_{n-1}b^{n-1}+b^{n} = 0$

		\[ \underbrace{b}_{\neq 0} \underbrace{(a_1 + … a_{n-1}b^{n-2} + b^{n-1})}_{=0} = 0\]

		Como la segunda parte es 0 $\rightarrow$ $p'(x) = a_1 + a_2 x + … + a_{n-1}x^{n-2} + x^{n-1}$

		Iterando el proceso llegamos a que:

		\begin{itemize}
			\item O bien $b$ es raíz de un polinomio mónico con coeficientes en $A$ de grado 1 $\Rightarrow b \in A$ (porque $b \neq 0$)

			\item O bien $b$ es raíz de un polinomio mónico con coeficientes en $A$ de grado $\geq 2$ con término constante $\neq 0$.

		\end{itemize}

		Podemos suponer que $p(x) = a_0 + a_1x + … + a_{n-1}x^{n-1} + x^n$ con $a_0 \neq 0$.

		\begin{align*}
		b^n + a_{n-1}b^{n-1}+…+a_1b + a_0 &= 0 \\
		b^n + a_{n-1}b^{n-1}+…+a_1b &= -a_0 \\
		b(a_{n-1}b^{n-1}+…+a_1b) &= \underbrace{-a_0}_{\neq 0} \in A (\text{cuerpo}) \\
		b \underbrace{(-a_0^{-1})(b^{n-1}+a_{n-1}b^{n-2}…+a_0)}_{\in B} &= 1 \\
		\end{align*}

		\proofpart{$\Leftarrow$}

		Suponemos $B$ cuerpo. Sea $a \neq 0,a \in A$ ¿$a^{-1} \in A$? Sabemos que $a^{-1} \in B \Rightarrow a^{-1}$ es entero$/A$.

		Entonces $\exists p(x) \in A[x]$: $p(x) = a_0 + a_1x + … + a_{n-1}x^{n-1} + x^n $, tal que $p(a^{-1})=0$

		\[ 0 = a_0 + a_1(a^{-1}) + … + a_{n-1}(a^{-1})^{n-1} + (a^{-1})^n \]

		Multiplicando por $(a^{n-1})$:

		\[ 0 = a_0 a^{n-1} + a_1 a^{n-1} a^{-1} + … + a_{n-1} a^{n+1} a^{n-1} + a^{-n} a^{n-1} \]

		\[ 0 = \underbrace{a_0 a^{n-1}}_{\in A} + \underbrace{a_1 a^{n-2}}_{\in A} + … + \underbrace{a_{n-1}a^0}_{\in A} + a^{-1} \Rightarrow a^{-1} \in A \]


	\end{proof}

	\begin{example}
		Veamos el caso $n = 3$.

		\[ a_0 + a_1 x + a_2 x^2 + x^3 \]
		\[ a_0 + a_1 a^{-1} + a_2 (a^{-1})^2  + (a^{-1})^3 = 0 \]

		Multiplicando por $a^2$:

		\[ \underbrace{a_0 a^2 + a_1 a + a_2 1}_{\in A} + a^{-1} = 0\]
	\end{example}

	Recoramos el teorema que hemos demostrado:

	\[ A \eqreason[\subset]{entera} B \quad A,B \text{ D.I. }\]
	\[ A \text{ cuerpo} \Leftrightarrow B \text{ cuerpo} \]

	\begin{example}[1]
		Si $A \subset B$ no es entera:
		\begin{itemize}
			\item $\mathbb{Z} \in \underbrace{\rac}_{\text{ cuerpo}}$
			\item $\underbrace{\rac}_{\text{ cuerpo}} \subset \rac[X]$
		\end{itemize}
	\end{example}

	\begin{example}[2]
		\[ \rac \eqreason[\subset]{entera} \frac{\rac[x]}{<x^2-1>} = \rac[\gor{x}] \]
		\[ T^2 - 1 \in \rac[T] \]

		¿Es $\rac[x]/<x^2-1>$ cuerpo?

		\[ \underbrace{(\gor{x-1})}_{\neq 0} \underbrace{(\gor{x+1})}_{\neq 0} \]

		No: $\rac[x]/<x^2-1>$ no es D.I.
	\end{example}

	\begin{example}[3]
		\[ \rac \eqreason[\subset]{entera} \rac[\sqrt{2}, \underbrace{e^{\frac{2\pi i}{3}}}_{\subset \cplex}] \]
		\[ \Rightarrow \rac[\sqrt{2}, e^{\frac{2\pi i}{3}}] \text{ es un cuerpo.}\]
	\end{example}

	\begin{corol}
		Sea $K \subset L$ una extensión de cuerpos. Sea $\alpha \in L$. Entonces $\alpha$ es un algebraico$/K \Leftrightarrow K[\alpha]$ es un cuerpo, siendo $K[\alpha]$ el anillo más pequeño que contiene a $k$ y $\alpha$.
	\end{corol}

	\begin{proof}

	\proofpart{$\implies$}

		Supongamos que $\alpha$ es algebraico$/K \Rightarrow \alpha$ es entero$/K \Rightarrow K \subset K[\alpha]$ es entera (porque es finita).

		\[ \underbrace{K}_{\text{D.I.}} \eqreason[\subset]{entera} \underbrace{K[\alpha]}_{\text{D.I.}} \subset L\]

		Entonces, por el teorema, $K[\alpha]$ es un cuerpo.

	\proofpart{$\impliedby$}

		Suponemos $K[\alpha]$ es un cuerpo ¿$\alpha$ es algebráica$/K$?

		\begin{align*}
			\text{Sea } \phi K[x] &\longrightarrow K[\alpha] \text{ homomorfísmo de }K\text{-anillos}\\
			x &\longrightarrow \alpha \\
			r &\longrightarrow r
		\end{align*}

		¿Es $\phi$ sobreyectivo? Sí, porque los elementos de $K[\alpha]$ se construyen con sumas de productos por $\alpha^n \rightarrow $ es claro que lo es.

		Entonces, por el primer teorema de isomorfía \ref{thm:IsomorfiaAnillos1}:

		\[ \frac{K[x]}{\ker \phi} \cong K[x]\]

		$p \in \ker \phi \Leftrightarrow p(\alpha) = 0 $. Supongamos $\ker \phi = \{0\}$ ($\phi$ inyectiva $\Rightarrow \phi$ isomorfismo).

		\[ \rightarrow \underbrace{K[x]}_{\text{no cuerpo}} \cong \underbrace{K[\alpha]}_{\text{cuerpo}}  \]

		Lo cual es una contradicción, luego $\ker \phi \neq \{0\}$, y $\exists p(x) \in K[x]$ tal que $p(\alpha) = 0 \Rightarrow \alpha $ es algebráica$/K$.
	\end{proof}

	%Clase 30/03/2016
	Sea  $K \subset K[\alpha]$. Si $\alpha$ no es algebraico sobre K, entonces $K[\alpha]\simeq \K[x]$. Si $\alpha$ es algebraico sobre K, entonces $\K[\alpha] \simeq \quot{\K[x]}{\gen{p(x)}}$, que es un cuerpo (con p irreducible).

Vamos a probar el Teorema básico para n=1.
\begin{proof}
	Sea $M$ un ideal maximal en $\K[x]$ (algebraicamente cerrado). Entonces, como $M$ es maximal, $\quot{\K[x]}{M}$ es un cuerpo. Este cuerpo cumple que:
	$$ \K  \subset \K[\cls{x}] = \quot{\K[x]}{M} \supset \K $$

	Por hipótesis $\K[\cls{x}]$ es un cuerpo. Por tanto $\cls{x}$ es un elemento algebraico sobre $\K$.

	¿Por que $\subset \K[\cls{x}] = \quot{\K[x]}{M}$?

	\begin{example}
		Tomamos $\rac \subset \quot{\rac[x]}{\gen{x^2+1}} = \rac[x]$.

		Da igual que $p(x)$ sea irreducible que no.
	\end{example}

	Seguimos: la hipótesis del enunciado es que $\K$ es un cuerpo algebraicamente cerrado, es decir que contiene a todos los elementos que son algebraicos sobre $\K$. Por tanto $\cls{x}=a \in \K \implies x-a \in M \implies \gen{x-a} \subset M$. ($\cls{x}-a = \cls{0}$)

	Pero $\gen{x-a}$ ya es maximal $\implies M = \gen{x-a}$
\end{proof}

%Queremos ver este teorema
%\begin{theorem}
%	Sea $A\subset B$ una extensio de dominios de integridad, suponemos que $A \subset B$ es entera, entinces A es un cuerpo si y solo si B es un cuerpo.
%\end{theorem}

\begin{prop}
	\textbf{Paso 1 para probar el caso general del teorema básico:}
	Sea $A \subset B$ una extensión de dominios de integridad. Y sean $\alpha_1,...,\alpha_s \in B$. Supongamos $A[\alpha_1,...,\alpha_s]$ y que este anillo es algebraico sobre A y es un cuerpo.

	Entonces existe $t \in A, t \neq 0$ tal que (extendido de A de $t$) $A_{\{t\}}$ es un cuerpo. (si me dijeran que la extensión es entera ya tendría que A es un cuerpo, pero solo tenemos que es algebraica)
\end{prop}

\begin{proof}
Por hipótesis cada $\alpha_i$ es algebraico sobre A. Para cada i, $\exists p_i(x) \neq 0 \in A[x]$ tal que $p_i(\alpha_i)=0$.

Para i=1: $p_1(x)=a_{01}+a_{11}x+...+a_{n_1 1}x^{n_1}$ con $a_{i1}\in A$ , entonces $p_1{\alpha_1}=0$

Para i=2: $p_2(x)=a_{02}+a_{12}x+...+a_{n_2 2}x^{n_2}$ con $a_{i2}\in A$ , entonces $p_2{\alpha_2}=0$

\vdots

Para i=s: $p_s(x)=a_{0s}+a_{1s}x+...+a_{n_s s}x^{n_s}$ con $a_{is}\in A$ , entonces $p_s{\alpha_s}=0$

Me encuentro con que los elementos que acompañan a la x de mayor grado no son invertibles.

Cogemos $t=a_{n_1 1}a_{n_2 2}....a_{n_s s}\neq 0$. Considero $A \rightarrow A_{\{t\}}$ (A localizado en t).

DIBUJO CUADERNO

Como $A[\alpha_1,...,\alpha_n]$ es un cuerpo, entonces $A_{\{t\}}$ es un cuerpo.

¿ de que nos sirve la propiedad universal de la localizacion?
\end{proof}

\begin{example}
	\begin{enumerate}
		\item Cogemos $\ent \subset \rac$ extensión. ¿Es algebraica? Siiii. $\rac$ es un cuerpo. Pero la extensión no es de tipo finito. ¿Puedo aplicar el teorema?  Noooo
		\item Cogemos $\pi: \ent_{\gen{3}} \hookrightarrow \rac[\sqrt{2}]=
		\underbrace{\ent_{\gen{3}}\left[\frac{1}{3}, \sqrt{2}\right]}_{\text{cuerpo}}$

		$\pi$ es algebraico. He añadido un número finito de elementos (1/3 y $\sqrt{2}$) y puedo aplicar entocnes la proposicion.
	\end{enumerate}
\end{example}

% clase del 31/03/2016
\begin{prop}
	Sea $A \subset B$ una extensión de D.I. Sean $\beta_1...\beta_s \in B$ y supongamos que $Z[\beta_1,....,\beta_s]$ es un cuerpo algebraico sobre A. Entonces existe $t \in A, t \neq 0$ tal que $A_{\{t\}}$ es un cuerpo.
\end{prop}

\begin{proof}
	Por inducción en r:
	\begin{enumerate}
		\item $r=1$: $\K \subset \K[\alpha_1] \implies \alpha_1$ algebraico sobre $\K$.
		\item Supongamos que el teorema sale para una álgebra de tipo finito sobre un cuerpo generado por $r-1$ elementos.
		\item \textbf{Caso general:}
		$$\K \hookrightarrow \K[\alpha_1] \hookrightarrow \K[\alpha_1,...,\alpha_r]=\K[\alpha_1]\K[\alpha_2,...,\alpha_r]$$
		
		No sabemos que $\K[\alpha_1]$ sea un cuerpo. Vamos a pasar de $\K[\alpha_1]$ a $\K(\alpha_1)$ que es localizar en $\quot{\K(\alpha_1)}{\{0\}}$, que sí es un cuerpo.
		
		¿Pero se puede hacer $\K(\alpha_1) \hookrightarrow \K[\alpha_1]\K[\alpha_2,...,\alpha_r]$?  Si, usamos la propiedad universal de la localización. Como $\K[\alpha_1,...,\alpha_r]$ es un cuerpo, todo elemento no nulo de $\K[\alpha_1]$ es invertible en $\K[\alpha_1,...,\alpha_r]$, luego por la propiedad universal de la localización hay un homomorfismo de $\K(\alpha_1)$ en $\K[\alpha_1,...,\alpha_r]$, que además es inyectivo porque $\K(\alpha_1)$ es un cuerpo.
		
		DIBUJO CUADERNO
		
		Por la hipótesis de inducción $\alpha_2,...,\alpha_r$ son algebraicos sobre $\K(\alpha_1)$
		
		Es $\K[\alpha_1]$ es algebraica sobre $\K(\alpha_1)$.  Si por el mismo argumento que me dice que $\rac$ es algebraico sobre $\ent$. Y $\K(\alpha_1)$ es algebraica sobre $\K[\alpha_1,...,\alpha_r]$. Por tanto:
		
		$$ \K \subset \K[\alpha_1] \stackrel{\text{algebraico tpo finito}}{\hookrightarrow} \K[\alpha_1,...,\alpha_r]$$
		
		Entonces, por la proposición, $\exists t \in \K[\alpha_1]$ tal que $\K[\alpha_1]_{\{t\}}$ es un cuerpo.
		
		¿Como es $\K[\alpha_1]$? O $\alpha_1$ es algebraico sobre $\K$ o no. Supongamos que no lo es, si a un cuerpo le añado un elemento que no es algebraico entonces $\K[\alpha_1] \simeq \K[x]$ (es como un anillo de polinomios). Hemos probado que existe un polinomio $p(x)\neq 0 \in \K[x]$ tal que $\K[x]_{\{p(x)\}}$ es un cuerpo.
		
		\begin{example}
			Sea $\K[x]_{\{x\}}$, si cogemos $\frac{q(x)}{x^n}$ hay montones de polinomios que no son invertibles.
		\end{example}
		
		Contradicción. Luego necesariamente $\alpha_1$ es algebraico sobre $\K$
	\end{enumerate}
\end{proof}

\begin{theorem}
	Sea $\K$ un cuepro algebraicamente cerrado y sea $m \subset \K[x_1,...,x_n]$ un ideal maimal. Entonces $m=\gen{genlist}$
\end{theorem}

\begin{proof}
 $$ \K \subset \underbrace{\quot{\K[x_1,...,x_n]}{m}}_{cuerpo} = \K[\cls{x_1},...,\cls{x_n}]$$
 
 Esto implica, por el teorema anterior, que $\cls{x_1},...,\cls{x_n}$ son algebraicos sobre $\K$.
 
 Como $\K$ es algebraicamente cerrado $\implies$ $\cls{x_1},...,\cls{x_n} \in \K$, es decir:
 $$ \quot{\K[x_1,...,x_n]}{m}= \K $$
 
 Entonces $\exists a_1,...,a_n \in \K$ tal que $\cls{x_i}=a_i$ en $\quot{\K[x_1,...,x_n]}{m}$, $\implies x_i-a_i \in m\subset \K[x_1,...,x_n] \implies \underbrace{\gen{x_1-a_1,...,x_n-a_n}}_{\text{ya es maximal etonces son iguales}} \subset m$
\end{proof}



