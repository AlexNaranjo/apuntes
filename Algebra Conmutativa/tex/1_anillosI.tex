% -*- root: ../AlgebraConmutativa.tex -*-
\chapter{Anillos I: nociones básicas}

Durante este curso vamos a trabajar mucho con anillos. Recordemos conceptos básicos vistos en \cite{apuntesEA}.

\begin{defn}[Anillo]
Un anillo $(R, +, * )$ está formado por un conjunto, R, que con la primera operación, que denotaremos +, es un grupo conmutativo; y que la segunda operación, que denotaremos *, es cerrada, asociativa y distributiva \wrt +\footnote{Pensemos que ésta propiedad es como el policía local que distribuye el tráfico}.
\end{defn}

Pero además, le pediremos las siguientes propiedades:

\begin{defn}[Anillo\IS conmutativo]
Se dice que un anillo es conmutativo si * tiene la propiedad conmutativa:
\[ a*(b*c) = (a*b)*c \]
\end{defn}

\begin{defn}[Anillo\IS unitario]
Se dice que un anillo es unitario si tiene un elemento neutro respecto de * (que denotaremos por \one):
\[ x*\one = \one*x = x \]
\end{defn}

\obs Durante el curso trabajaremos con anillos conmutativos con unidad\footnote{Sinónimo de unitario.}, aunque no se mencione de manera explícita.

\begin{example}
\begin{itemize}
	\item Anillo conmutativo: matrices cuadradas.
	\item Anillo sin unidad: los pares en $\ent$.
\end{itemize}
\end{example}

Ejemplos de anillos que utilizaremos:

\begin{example}
\begin{itemize}
	\item $\ent$, $\ent_n$ $n \geq 1$ (casi siempre que pida un ejemplo, estará basado en ellos).
	\item $\rac$, $\real$, $\cplex$, $\field_{p^n}$.
	\item R anillo de polinomios: $R[x] = \set{ \sum_{i=0}^n a_i x^i \tq a_i \in R, n \in \nat}$.
\end{itemize}
\end{example}

Para evitar dudas, comencemos dejando claro que:

\begin{theorem}
Los $\nat$ contienen al 0.
\end{theorem}

\begin{theorem}
Dos expresiones en $R[x]$ son iguales $\iff$ los coeficientes son iguales uno a uno\ie $a_i = b_i \ \forall i$.
\end{theorem}

\begin{defn}[Anillo\IS de polinomios en $n$ variables]
Denotamos por $R[x_1,\dots,x_n]$ a un anillo de polinomios con coeficientes en $R$ en $n$ variables:
\[ R[x_1,\dots,x_n] = \set{ \sum_{i_1,\dots,i_n} a_{i_1,\dots,i_n} x^{i_1}\dots x^{i_n} \tq a_{i_1,\dots,i_n} \in R } \]
\end{defn}

\begin{defn}[Unidad]
Sea $R$ un anillo, se dice que $a\in R$ es {\bf invertible}\ie $a\in \U(R)$, si $\exists b\in R $ \st $a*b = \one$.
\end{defn}

\begin{defn}[Cuerpo]
Sea R un anillo, diremos que es un cuerpo\footnote{\url{https://en.wikipedia.org/wiki/Field_(mathematics)}} si $R\setminus \set{0} = R^*$ es un grupo conmutativo\ie si todo elemento no nulo de $R$ es una unidad.
\end{defn}

\begin{example}
	\begin{itemize}
		\item $\rac, \real, \cplex, \field_{p^n}$ % si, los ejemplos se repiten
		\item $\rac[\sqrt2] = \set{ a + b\sqrt2 \tq a,b \in \rac }$
	\end{itemize}
\end{example}

\begin{prop}
	Si $\zero=\one$, entonces todos los elementos del anillo son 0.
\end{prop}

\begin{defn}[Divisor de cero]
Sea $R$ un anillo. Se dice que $a \in R$, $a\neq\zero$, es un {\bf divisor de cero}, si $\exists b \in R$, $\zero \neq b$, \st $a*b = \zero$.
\end{defn}

\begin{defn}[Nilpotente]
Se dice que $a\in R$ es nilpotente si $\exists n \in \nat$, $n\geq 1$, \st $a^n = \zero$.
\end{defn}

\begin{example}
	\begin{itemize}
		\item En $\ent_6$, $\cls{2} * \cls{3} = \zero$, y como tenemos que $\cls{2} \neq \zero, \cls{3}\neq \zero$, $\cls{2}\neq\cls{3}$, sabemos que $\cls{2}$ y $\cls{3}$ son divisores de cero.
		\item En $\ent_4$, $\cls{2}^2 = \zero \implies \cls{2}$ es nilpotente.
		\item $\ent_6[x]$
	\end{itemize}
\end{example}

\begin{lemma}
	Todo elemento nilpotente es divisor de cero.
\end{lemma}

\begin{proof}
	Sea A grupo y sea $a \in A$, $a\neq\zero$, elemento nilpotente y sea
		\[n = \min\set{n \tq a^n = \zero, n \in \nat}\]
	Luego $a^n = a * a^{n-1} = \zero$. Y tenemos que $\zero \neq b = a^{n-1} \in A$ es el elemento \st $a*b=\zero$ ya que como $a\neq \zero$, $a^n = \zero \implies n\geq 2 \implies a^{n-1} \neq \zero$.
\end{proof}

\begin{lemma}
	No todo divisor de cero es nilpotente.
\end{lemma}

\begin{proof} % los contraejemplos siempre son sencillos
	En $\ent_6$, $\cls{2}$ es divisor de $\zero$ pero no es nilpotente ya que:
	\begin{gather*}
		\cls{2}^0 = \cls{1}, \ \cls{2}^1 = \cls{2}, \ \cls{2}^2 = \cls{4}, \ \cls{2}^3 = \cls{2}, \ \cls{2}^4 = \cls{4}, \ \dots
	\end{gather*}
\end{proof}

\begin{defn}[Dominio (de integridad)]
Diremos que R es un dominio\footnote{\url{https://en.wikipedia.org/wiki/Integral_domain}} si no tiene divisores de cero.
\end{defn}

\begin{example}
	$\ent$
\end{example}

\begin{defn}[Anillo\IS reducido]
Diremos que R es un anillo reducido si no tiene elementos nilpotentes no nulos.
\end{defn}

\begin{example}
	\begin{itemize}
		\item Anillos reducidos que son dominios: $\ent, \ent_2$
		\item Anillo reducido que no es dominio: $\ent_6$
		\item Anillos no reducidos: $\ent_4$, $\ent_8$, $\ent_9$, \dots
	\end{itemize}
\end{example}

\obs Las variedades algebraicas usan anillos reducidos: si el anillo es dominio, la variedad es reducible; si el anillo no es dominio, se dice que la variedad es no reducible.
% Y os preguntaréis, ¿Qué es una variedad algebraica? Pues hasta el tema 3...

\section{Normas de convivencia}
Los siguientes enunciados deben saberse y la demostración queda como ejercicio para el lector: % Edu: ya lo intentaré luego

\begin{itemize}
	\item Si $a \in R$ divisor de cero $\implies$ a \underline{\bf NO} invertible.
	\item Un anillo con unidad tiene al menos una unidad (el $\cls{1}$), y dos si $\cls{1} \neq \cls{-1}$. % nada que demostrar, la verdad
\end{itemize}

\begin{defn}[Subanillo]
Sea $R$ un anillo y sea $S \subseteq R$, $S \neq \emptyset$,diremos que $S$ es subanillo de $R$ si $S$ es un anillo con las operaciones definidas para $R$.
\end{defn}

\begin{prop}
Ningún $\ent_n$ es subanillo de $\ent$, porque $\ent_n \not \subseteq \ent$.

{\bf Razón}: en $\ent$, 4 y 6 son distintos, pero en $\ent_n$ pueden ser iguales.
\end{prop}

\begin{example}
	\begin{enumerate}
		\item $R \subset R[x]$
		\item $\ent \subset \rac \subset \real \subset \cplex$
	\end{enumerate}
\end{example}

% Mención de recordatorio del infierno de paso al cociente que a todo alumno de 3º le chifla

\begin{defn}[Ideal] % no, no se refiere a un modelito
Sea $R$ un anillo y sea $I \subseteq R$. Se dice que $I$ es un ideal si:

	\begin{enumerate}
		\item $I \neq \emptyset$
		\item $(I, +)$ es un \underline{subgrupo}.
		\item \concept{Propiedad de absorción}: $\forall i \in I, \forall r \in R \implies r*i \in I$.
	\end{enumerate}
\end{defn}

\begin{lemma}
	Como $R$ es abeliano con +\footnote{recuerden que siempre hablamos de anillos conmutativos con unidad} entonces $I$ es un \underline{subgrupo normal}.
\end{lemma}

\begin{example}
	\begin{itemize}
		\item Fijado $k \in \ent$, el conjunto $\set{n\cdot k \tq n \in \nat }$ es un ideal de $\ent$.
		\item En $R$, $(R, +)$ es siempre un \concept{Ideal\IS no propio} y $\set{\zero}$ es siempre un ideal.
	\end{itemize}
\end{example}

\begin{defn}[Ideal\IS propio]
Decimos que un ideal $I \subseteq R$ es \underline{propio} si $I \neq R$.
\end{defn}


