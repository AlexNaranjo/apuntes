% -*- root: ../AlgebraConmutativa.tex -*-
\chapter{Anillos I: nociones básicas}

Durante este curso vamos a trabajar mucho con anillos. Recordemos conceptos básicos vistos en Estructuras Algebraicas \cite{apuntesEA}.


\begin{defn}[Grupo]
Un grupo es un conjunto, G, con una operación binaria $'\bullet'$, que compone dos elementos $a.b \in G$ para formar otro elemento notado como $a \bullet b$ o $ab$. Se escribe: $(G,\bullet)$. Para ser considerado grupo debe satisfacer:
\begin{enumerate}
	\item Cerrado: $\forall a,b \in G$, entonces $a \bullet b \in G$
	\item Asociatividad: $\forall a,b,c \in G$, entonces $(a\bullet b) \bullet c = a \bullet (b \bullet c)$
	\item Elemento neutro: $\exists e \in G$ tal que $\forall a \in G$ se cumple que $e\bullet a=a\bullet e = a$. A este elemento neutro se le denomina como elemento identidad o $1_G$.
	\item Elemento inverso: $\forall a \in G$ existe un elemento $b \in G$ tal que $a \bullet b = b \bullet a = e$. Sea $a \in G$, a su elemento inverso se le denomina $a^{-1}$.
\end{enumerate}
\end{defn}

\begin{defn}[Grupo\IS conmutativo]\index{Grupo!abeliano}
Un grupo $(G,\bullet)$ es conmutativo o abeliano si es un grupo y además la operación $\bullet$ es conmutativa. Es decir, $\forall a,b \in G, a\bullet b = b \bullet a$.
\end{defn}

\begin{defn}[Anillo]
Sea $R$ un conjunto no vacío, y sean $'+'$ y $'\cdot'$ dos operaciones binarias en $R$. Se dice que el conjunto $(R, +, \cdot )$ es un anillo si se cumplen las siguientes propiedades:

\begin{enumerate}
	\item $R$ es un grupo conmutativo con la primera operación $'+'$. Es decir, $(R,+)$ es un grupo conmutativo. Si utilizamos como primera operación la suma ($'+'$), al elemento neutro lo denotaremos por \zero.
	\item $R$ es cerrado con respecto a la segunda operación $'\cdot'$. Es decir $\forall a,b \in R$, entonces $a \cdot b \in R$
	\item $R$ es asociativo con respecto a la segunda operación $'\cdot'$. Es decir, $\forall a,b,c \in R$, entonces $(a\cdot b) \cdot c = a \cdot (b \cdot c)$
	\item La segunda operación $'\cdot'$ es distributiva con respecto a la primera $'+'$. Es decir, $a \cdot (b+c) = a\cdot b + a \cdot c$
	\[
	\left\{ \begin{array}{c}
	a \cdot (b+c) = a\cdot b + a \cdot c \\
	(b+c) \cdot a = b\cdot a + c\cdot a \\
	\end{array}
	\right.
	\]
\end{enumerate}
\end{defn}


\begin{defn}[Anillo\IS unitario]
	Se dice que un anillo $(R, +, \cdot)$ es unitario si tiene un elemento neutro respecto de la segunda operación $'\cdot'$. A ese elemento neutro lo denotaremos por \one):
	\[ x\cdot \one = \one \cdot x = x \]
\end{defn}

\begin{defn}[Anillo\IS conmutativo]
Se dice que un anillo $(R, +, \cdot)$ es conmutativo si la segunda operación $'\cdot'$ tiene la propiedad conmutativa:
\[ a\cdot b = b\cdot a \]
\end{defn}


\textbf{Nota:} Durante el curso trabajaremos con anillos conmutativos con unidad (anillos conmutativos unitarios), aunque no se mencione de manera explícita.

\begin{example}
\begin{itemize}
	\item Anillo no conmutativo: matrices cuadradas.
	\item Anillo sin unidad: los pares en $\ent$.
\end{itemize}
\end{example}


\begin{example}
Anillos que utilizaremos:
\begin{itemize}
	\item $\ent$, $\ent_n$ $n \geq 1$ (casi siempre que pida un ejemplo, estará basado en ellos).
	\item $\rac$, $\real$, $\cplex$, $\field_{p^n}$.
	\item R anillo de polinomios: $R[x] = \set{ \sum_{i=0}^n a_i x^i \tq a_i \in R, n \in \nat}$. Dos expresiones en $R[x]$ son iguales $\iff$ los coeficientes son iguales uno a uno\ie $a_i = b_i \ \forall i$.
	\item R anillo de polinomios en n variables. Denotaremos como $R[x_1,\dots,x_n]$ a un anillo de polinomios con coeficientes en $R$ en $n$ variables:
	\[ R[x_1,\dots,x_n] = \set{ \sum_{i_1,\dots,i_n} a_{i_1,\dots,i_n} x^{i_1}\dots x^{i_n} \tq a_{i_1,\dots,i_n} \in R } \]
	\item $\ent_6$ es un anillo finito, mientras que $\ent_6[x]$ es infinito.
\end{itemize}
\end{example}


\textbf{Nota:} Consideraremos que los $\nat$ contienen al 0.

\textbf{Notación:} Para abreviar en muchas ocasiones llamaremos $R$ a $(R,+,\cdot)$.

\begin{defn}[Unidad]\index{Elemento\IS invertible}
Sea $(R,+,\cdot)$. un anillo, se dice que $a\in R$ es invertible (también llamado unidad) ($a\in \U(R)$) si $\exists b\in R $ \st $a \cdot b = \one$. Es decir, tiene elemento inverso con respecto a la segunda operación $'\cdot'$.
\end{defn}

\begin{defn}[Unidades de R: $\U(R)$][Unidades]
Al conjunto de todas las unidades (o elementos invertibles) de un anillo R se le denomina $\U(R)$.
\end{defn}

\begin{defn}[Cuerpo]
Sea $(R,+,\cdot)$ un anillo, diremos que es un cuerpo si $R\setminus \set{0} = R^*$ es un grupo conmutativo con la segunda operación $'\cdot'$ \ie si todo elemento no nulo de $R$ es una unidad ($\U(R) = R^*$), donde $R^* = R \setminus \set{0}$.
\end{defn}

\begin{example} Ejemplos de cuerpos:
	\begin{itemize}
		\item $\rac, \real, \cplex, \field_{p^n}$ ($\ent$ no es un cuerpo) % si, los ejemplos se repiten
		\item $\rac[\sqrt2] = \set{ a + b\sqrt2 \tq a,b \in \rac }$
	\end{itemize}
\end{example}

\begin{prop}
	Si $\zero=\one$, entonces todos los elementos del anillo son 0.
\end{prop}

\begin{defn}[Divisor de cero]
Sea $(R,+,\cdot)$ un anillo. Se dice que $a \in R$, $a\neq\zero$, es un {\bf divisor de cero}, si $\exists b \in R$, $b \neq \zero$, \st $a\cdot b = \zero$.
\end{defn}

\begin{defn}[Nilpotente]
Se dice que $a\in R$ es nilpotente si $\exists n \in \nat$, $n\geq 1$, \st $a^n = \zero$.
\end{defn}

\begin{example}
	\begin{itemize}
		\item En $\ent_6$, $\cls{2} * \cls{3} = \zero$, y como tenemos que $\cls{2} \neq \zero, \cls{3}\neq \zero$, $\cls{2}\neq\cls{3}$, sabemos que $\cls{2}$ y $\cls{3}$ son divisores de cero.
		\item En $\ent_4$, $\cls{2}^2 = \zero \implies \cls{2}$ es nilpotente.
	\end{itemize}
\end{example}


\begin{lemma}
	Todo elemento nilpotente es divisor de cero.
\end{lemma}

\begin{proof}
	Sea A grupo y sea $a \in A$, $a\neq\zero$, elemento nilpotente y sea
		\[n = \min\set{n \tq a^n = \zero, n \in \nat}\]
	Luego $a^n = a * a^{n-1} = \zero$. Y tenemos que $\zero \neq b = a^{n-1} \in A$ es el elemento \st $a*b=\zero$ ya que como $a\neq \zero$, $a^n = \zero \implies n\geq 2 \implies a^{n-1} \neq \zero$.
\end{proof}

\begin{lemma}
	No todo divisor de cero es nilpotente.
\end{lemma}

\begin{proof} % los contraejemplos siempre son sencillos
	En $\ent_6$, $\cls{2}$ es divisor de $\zero$ pero no es nilpotente ya que:
	\begin{gather*}
		\cls{2}^0 = \cls{1}, \ \cls{2}^1 = \cls{2}, \ \cls{2}^2 = \cls{4}, \ \cls{2}^3 = \cls{2}, \ \cls{2}^4 = \cls{4}, \ \dots
	\end{gather*}
\end{proof}

\begin{defn}[Dominio (de integridad)]
Diremos que R es un dominio si no tiene divisores de cero.
\end{defn}

\begin{example}
	$\ent$
\end{example}

\begin{defn}[Anillo\IS reducido]
Diremos que R es un anillo reducido si no tiene elementos nilpotentes no nulos.
\end{defn}

\begin{example}
	\begin{itemize}
		\item Anillos reducidos que son dominios: $\ent, \ent_2$
		\item Anillo reducido que no es dominio: $\ent_6$
		\item Anillos no reducidos: $\ent_4$, $\ent_8$, $\ent_9$, \dots
	\end{itemize}
\end{example}

%\obs Las variedades algebraicas usan anillos reducidos: si el anillo es dominio, la variedad es reducible; si el anillo no es dominio, se dice que la variedad es no reducible.
% Y os preguntaréis, ¿Qué es una variedad algebraica? Pues hasta el tema 3...

\obs Si $a \in R$ divisor de cero, entonces $a$ \underline{\bf NO} es invertible. El recíproco también es cierto. (Las unidades y los divisores de 0 se llevan muy mal)

\obs Un anillo con unidad tiene al menos una unidad (el $\cls{1}$), y dos si $\cls{1} \neq \cls{-1}$. % nada que demostrar, la verdad

\begin{defn}[Subanillo]
Sea $(R,+,\cdot)$ un anillo y sea $S \subseteq R$, $S \neq \emptyset$. Diremos que $S$ es subanillo de $R$ si $S$ es un anillo con las operaciones definidas en $R$. Basta ver que:
\begin{enumerate}
	\item \one $\in S$
	\item $\forall a,b \in S$, entonces $a-b \in S$
	\item $\forall a,b \in S$, entonces $a\cdot b \in S$
\end{enumerate}
\end{defn}



\begin{example} Ejemplos de subanillos.
	\begin{enumerate}
		\item $R \subset R[x]$
		\item $\ent \subset \rac \subset \real \subset \cplex$
		\item Ningún $\ent_n$ es subanillo de $\ent$, porque $\ent_n \not \subseteq \ent$. Esto es así porque ni siquiera $\ent_n$ es un subconjunto de $\ent$ ya que los elementos de $\ent$ no son los mismo que los de $\ent_n$, en el primer caso son números normales y corrientes, y en el segundo son $\{\cls{i},  \forall 0<i<n\}$, y representan el resto de dividir $i$ entre $n$.
		% porque en $\ent$, 4 y 6 son distintos, pero en $\ent_n$ pueden ser iguales
	\end{enumerate}
\end{example}


\begin{defn}[Ideal] % no, no se refiere a un modelito
Sea $(R,+,\cdot)$ un anillo y sea $I \subseteq R$. Se dice que $I$ es un ideal si:

	\begin{enumerate}
		\item $I \neq \emptyset$
		\item $(I, +)$ es un subgrupo de R.
		\item \concept{Propiedad de absorción}: $\forall i \in I, \forall r \in R \implies r\cdot i \in I$.
	\end{enumerate}
\end{defn}

\obs Como $R$ es abeliano con  $'+'$ por definición, entonces si $(I,+)$ es un subgrupo, entonces es un subgrupo normal.
% Revisado. Guille.

\begin{defn}[Ideal\IS propio]
	Decimos que un ideal $I \subseteq R$ es propio si $I \neq R$.
\end{defn}

\begin{example} Ejemplos de ideales:
	\begin{itemize}
		\item Fijado $k \in \ent$, el conjunto $\set{n\cdot k \tq n \in \nat }$ es un ideal de $\ent$.
		\item En $R$, hay siempre dos ideales, R (que es un ideal no propio) y $\set{\zero}$.
	\end{itemize}
\end{example}




