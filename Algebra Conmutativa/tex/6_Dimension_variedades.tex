% -*- root: ../AlgebraConmutativa.tex -*-

\chapter{La dimensión de las variedades algebraicas afines}

Esquema de contenidos del tema:

\begin{itemize}
	\item Definición intuitiva de lo que debería ser la dimensión de una v.a.a.
	\begin{itemize}
		\item Traducción de la noción anterior en términos de lo que vamos a llamar dimensión de Krull de un anillo.
		\item Nos vamos a creer que $\dim_{Krull}\K[x_1,...,x_n]=n$ (dimensión de Krull de un anillo de polinomios con coeficientes en un cuerpo).
	\end{itemize}
	\item Veremos que si tenemos $A \subset B$, una extensión entera de anillos, entonces $ \dim_{Krull} A = \dim_{Krull} B$
	\item Vamos a ver el \textbf{Lema de Normalización de Noether:} Toda v.a.a. X se puede proyectar de manera finita sobre un espacio afín $\Akn$, donde $n=\dim_{Krull}\K[X]$. (siendo $\K[X]$ un anillo de coordenadas) %(Enunciado geométrico)
	\item Y si nos diese tiempo veríamos el teorema que nos dice que $\dim_{Krull}\K[x_1,...,x_n]=n$.
\end{itemize}

\section{Definición intuitiva de lo que debería ser la dimensión de una v.a.a.}

Por ejemplo, si tenemos un espacio vectorial V sobre un cuerpo $\K$, para saber la dimensión de V buscamos una base $B=\{ v_1,...,v_n\}$ que es una serie de vectores linealmente independientes que generan el espacio. 

Cada vez que me dan una base puedo construir una cadena de subespacios vectoriales:
$\{0\} \subsetneqq \gen{v_1} \subsetneqq \gen{v_1,v_2} \subsetneqq ... \subsetneqq \gen{v_1,...,v_n}$. Entonces esta cadena es la más larga dentro de $V$, y la longitud de esta cadena se dice que es n (el primero se considera 0). Por tanto $n= \dim V$

Vamos a usar el mismo razonamiento para calcular la dimensión de $X$ v.a.a.. Vamos a suponer $X$ es irreducible y a considerar dentro de $X$ subvariedades y construir la cadena más larga dentro de $X$. Consideramos dentro de $X$ cadenas estrictamente crecientes de subvariedades irreducibles. Tendríamos:

$$Y_0 \subsetneqq Y_1 \subsetneqq ... \subsetneqq Y_n=X$$

Entonces podríamos definir $\dim X$ como la longitud de la cadena más larga.

Si $X$ no fuese irreducible se separa en variedades irreducibles y se le asigna como dimensión el máximo de las longitudes.

\subsection{Traducción de la noción de dimension en términos del álgebra conmutativa} 

Tenemos $X \subseteq  \Akn$, y tenemos también $\I(X) \subseteq \K[x_1,...,x_n]$

La cadena anterior de ideales irreducibles se traduce en una cadena de ideales  primos (por la proposición \ref{prop:VaaIrreducibleIdPrimo}):

$$\I(Y_0) \supsetneqq \I(Y_1) \supsetneqq ... \supsetneqq \I(Y_n)=\I(X)$$

Pero estas cadenas (que contienen a $\I(X)$) se corresponden biyectivamente con cadenas de ideales primos en $\K[X]=\quot{\K[x_1,...,x_n]}{\I(X)}$ . Así, para definir la dimensión de $X$ miraremos al máximo de las longitudes de las cadenas  de ideales primos en $\K[X]$.

Esto es bastante fuerte ya que vimos que dada una v.a.a. $X$ el anillo de coordenadas es el mismo sea donde sea que viva $X$ (dando igual que la $X$ viva en un espacio afín o en otro). Imaginaos que $X$ es una curva en $A^3$, la misma curva la puedo ver en $A^4$, su anillo de coordenadas es el mismo. \textcolor{red}{Donde hemos visto esto?}

\textcolor{red}{Poner reflexion de anillos de coordenadas unicos para variedades dando igual donde viven}

\begin{defn}
	Sea $A$ un anillo y sea $p_0 \subsetneqq p_1 \subsetneqq ... \subsetneqq p_r$ una cadena estrictamente creciente de ideales primos en $A$. En tal caso, diremos que la longitud de la cadena es $r$, y definimos $\dim_{Krull} A$ como el máximo (si existe) de las longitudes tras considerar todas las posibles cadenas de ideales primos en $A$:
	
	$$ \dim_{Krull}A = \max \left\{ r: \exists \text{ cadena } p_0 \subsetneqq p_1 \subsetneqq ... \subsetneqq p_r \text{ con } p_i \text{ primo } \subset A \right\} $$
\end{defn}

\begin{example}
	Sea $\K$ un cuerpo (su único ideal primo es el $\{0\}$), la cadena más larga que puedo montar es $\{0\}$, por tanto: $\dim_{Krull} \K = 0$
\end{example}

\begin{example}
	Sea $\ent$, los ideales primos en $\ent$ son el $\{0\}$ y los generados por números primos, por tanto las cadenas serán:
	
	$$ \{0\} \subsetneqq \gen{p} $$
	
	Pero $\gen{p}$ es maximal así que ya no puedo hacer la cadena más larga, luego $\dim_{Krull} \ent = 1$
\end{example}

\begin{example}
	Sea $\K[x]$ con $\K$ cuerpo. Construyo la cadena:
	
	$$\{0\} \subsetneqq  \gen{p(x)} $$
	
	Con $p(x)$ irreducible, pero $\gen{p(x)}$ es maximal así que ya no puedo seguir. Luego  $\dim_{Krull} \K[x] = 1$
	
	Además $\K[x]$ es el anillo de coordenadas de $\A^1_K$, que esperamos que tenga dimensión 1.
\end{example}

\begin{example}
	Sea $\K[x,y]$:
	
	$$\{0\} \subsetneqq  \gen{x} \subsetneqq \gen{x,y} $$
	
	Por tanto  $\dim_{Krull} \K[x,y] \geq 2$
\end{example}

	Nos vamos a creer lo siguiente: \textbf{Teorema: Sea $\K$ un cuerpo, entonces $\dim_{Krull} \K[x_1,...,x_n]=n$}
	
\section{Dimensión de Krull en el caso de extensiones enteras de anillos}

Recordatorio extensiones enteras:

Tenemos $A \hookrightarrow B$ ($A \subset B$) extensión de anillos. Cada elemento de b es raíz de un polinomio mónico de elementos en A.
\begin{itemize}
	\item Si $A \subset B$ es una extensión entera y si $J \subset B$ es un ideal, entonces $\quot{A}{J \cap A} \hookrightarrow \quot{B}{J}$ es entera.
	\item Si $S \subset A$ es una parte multiplicativa (conjunto multiplicativamente cerrado) entonces $S^{-1}A \hookrightarrow S^{-1}B$ es una extensión entera.
	\item Si fijamos un ideal primo $p \subset A$ y tomamos $S=A \setminus p$, entonces $A_p=S^{-1}A$ es un anillo local con maximal $pA_p$
\end{itemize}






\begin{itemize}
	\item 
\end{itemize}