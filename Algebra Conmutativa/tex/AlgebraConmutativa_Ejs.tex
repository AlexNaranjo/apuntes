% -*- root: ../AlgebraConmutativa.tex -*-
\chapter{Ejercicios}
\setcounter{section}{-1} % Start counting from 0
\section{Hoja 0}

\begin{problem}
\ppart
\ppart
\ppart
\solution
\end{problem}

\begin{problem}

\solution
\end{problem}

\begin{problem}
{\bf Sobre las unidades en anillos de polinomios}
\ppart
\ppart
\solution
\end{problem}

\begin{problem}

\solution
\end{problem}

\begin{problem}

\solution
\end{problem}

\begin{problem}
\ppart
\ppart
\ppart
\solution
\end{problem}

\begin{problem}
\ppart
\ppart
\ppart
\solution
\end{problem}

\begin{problem}
\ppart
\ppart
\ppart
\ppart
\ppart
\ppart
\ppart
\solution
\end{problem}
%%%%%%%%%%%%%%%%%%%%%%%%%%%%%%%%%%%%%%%%%%%%%%%%%%%%%%%%%%%%%%%%%%%%%%%
\section{Hoja 1: Anillos, ideales}

\begin{problem}
{\bfseries El radical de un ideal.}

Sea $I\subset R$ un ideal  y sea $\sqrt{I}$ su radical.

\ppart Demuestra que $\sqrt{I}$ es un ideal que adem\'as contiene a $I$.

\ppart Demuestra que todo ideal primo es radical.

\ppart Describe todos lo ideales de ${\mathbb Z}$ que sean radicales.

\ppart Sea $k$ un cuerpo. Describe todos lo ideales de $k[x]$ que sean radicales.

{\em Sugerencia: utiliza el ejercicio 1  de la parte de problemas para entregar.}

\solution
\end{problem}

\begin{problem}
{\bfseries Unión de ideales.}
\solution
\end{problem}

\begin{problem}
{\bfseries Intersección de ideales.}
\solution
\end{problem}

\begin{problem}
{\bfseries ¿Cuándo es la intersección de ideales igual al producto?}
\solution
\end{problem}

\begin{problem}
{\bfseries La geometría que veremos.}
\solution
\end{problem}

\begin{problem}
{\bfseries Miramos ejemplos en $\ent$.}
\solution
\end{problem}

\begin{problem}
{\bfseries Anillos locales.}
\solution
\end{problem}

\begin{problem}
\solution
\end{problem}

\noindent \fbox{{\bfseries  Problemas para entregar} }

\begin{problem}
{\bfseries Sobre los ideales en $K[x]$.}

\ppart Demuestra que $K[x]$ es un dominio euclídeo.
\ppart Demuestra que en $K[x]$ todos los ideales son principales.
\ppart Demuestra que en $K[x]$ un ideal es maximal si y solo si está generado por un polinomio irreducible.
\ppart Demuestra que en $k[x,y]$ no todos los ideales son principales. {\em Sugerencia: considera el conjunto $I$ formado por todos los polinomios $p(x,y)\in k[x,y]$ que se anulan en $(0,0)$; demuestra que es un ideal y que no es principal.}
\solution
\end{problem}

\begin{problem}
Decide de manera razonada para qué valores de $n$ el anillo ${\mathbb Z}_n$ es un anillo local. En aquellos casos en los que se trate de un anillo local indica cuál es el (único) ideal maximal.
\solution
\end{problem}

\begin{problem}
Considera el siguiente subanillo de ${\mathbb Q}(x)$:
$$R ≝\left\{\frac{p(x)}{q(x)}: \frac{p(x)}{q(x)} =\frac{r(x)}{s(x)} \text{ con }  s(0)\neq 0\right\}. $$
\ppart  Demuestra que  ${\mathfrak m} ≝ \left\{\frac{p(x)}{q(x)}\in R: p(0)=0\right\}$ es un ideal en $R$.
\ppart  Demuestra que $R$ es un anillo local con ideal maximal $\mathfrak m$.

{\em Sugerencia: puedes usar el ejercicio 7.}
\solution
\end{problem}

\begin{problem}
{\bfseries Conductores.}
Sean $I, J\subset R$ ideales. Definimos el conductor de $J$ en $I$ \index{Conductor} como:
$$(I:J) ≝\{a\in R: aJ\subset I).$$

\ppart Demuestra que $(I:J)$ es un ideal en $R$ que además contiene a $I$.

\ppart Calcula $(\langle 12\rangle : \langle n \rangle)$ para todo  $n\in {\mathbb Z}$.

\ppart Intuitivamente,  ?`qué es el ideal $(I: \langle a\rangle)$ para $a\in R$?

\ppart Sea ${\mathfrak p}\subset R$ un ideal primo. Calcula $({\mathfrak p}: I)$ para todo $I\subset R$.
?`Por qué crees que obtienes ése   resultado?

\ppart Demuestra que si $J=\langle a_1,\ldots, a_r\rangle$ entonces $(I:J)=\cap_{i=1}^r(I: \langle a_i\rangle).$
\solution
\end{problem}
