% -*- root: ../AlgebraConmutativa.tex -*-
\chapter{Ejercicios}

Últimamente nos llegan comentarios de gente que dice que "los apuntes están muy bien, pero me daba vergüenza deciros que el ejercicio $X.Y$ estaba mal". Gracias.

Pero si tú te beneficias de estos ejercicios y crees que están mal, por favor háznoslo saber por email o en persona.

El profesor no va a tener ningún reparo en ponernos un 0 en el examen.

\setcounter{section}{-1} % Start counting from 0
\section{Hoja 0}

\begin{problem}
Sea $R$ un anillo. Demuestra que:

\ppart Si $a \in R$ es un divisor de cero, entonces $a$ no puede ser una unidad.
\ppart Si $R$ es finito, entonces todo $r \in R \setminus \set{0}$ es o bien unidad, o bien un divisor de cero
\ppart Si $R$ no es finito, el enunciado anterior no es necesariamente cierto.
\solution

\doneby{Guille}

\spart

Si $a$ es un divisor de cero, entonces $∃b ∈ R$, con $b ≠ 0$ tal que $ab = 0$. Si $a$ fuese una unidad, podríamos multiplicar por su inverso a ambos lados, y tendríamos que $a'ab = a' 0 \implies b = 0$, contradicción.

\spart

Dado un $r ∈ R^*$, consideramos todas las posibles multiplicaciones por $b ∈ R^*$. Si existe algún $b ∈ R^*$ tal que $rb = 1$, entonces $r$ es una unidad. Si, por otra parte, $rb = 0$, entonces $r$ es divisor de cero.

Vamos a demostrar ahora que el último caso ($r$ no es unidad ni divisor de cero) no se puede dar. Dado que $R$ es finito, por el principio del palomar tiene que haber dos multiplicaciones por elementos distintos que repitan resultado, esto es, $∃b,c ∈ R^*$ tales que $rb = rc$ y $b ≠ c$. Sin embargo, simplemente operando tenemos que $rb - rc = 0$, y por la propiedad asociativa $r(b-c) = 0$, por lo que $r$ sí es divisor de cero, contradicción.

\spart

El principio del palomar no nos vale en cuerpos infinitos. Por ejemplo, en $ℤ$ sólo hay dos unidades y no hay divisores de cero.
\end{problem}

\begin{problem}
Encuentra todas las unidades y los divisores de cero en $\ent$, $\ent_n$ y en $\K[x]$ donde $\K$ es un cuerpo. % de ahí las dos patitas que diría Quirós
\solution
\end{problem}

\begin{problem}
{\bf Sobre las unidades en anillos de polinomios}

Sea $R$ un anillo y sea $R[x]$ el anillo de polinomios con coeficientes en $R$.
\ppart Si $R$ es un dominio de integridad demuestra que
\[ \set{\text{Unidades de } R[x]} = \set{\text{Unidades de } R} \]
\ppart Demuestra, dando un contraejemplo, que el enunciado anterior no es cierto si $R$ no es un dominio de integridad.
\solution

\doneby{Guille}

\spart

Un dominio de integridad no tiene divisores de cero, por lo que nunca nos vamos a poder quitar los monomios de $x^n$ y sólo nos van a quedar las unidades de $R$.

Más formalmente, sea $p(x) = a_0 + a_1 x + \dotsb + a_nx^n$ y $q(x) = b_0 + b_1 x + \dotsb + b_mx^m$ unidades, tales que $p(x) q(x) = 1$. Sin embargo, $p(x) q(x)$ va a tener como coeficiente más alto $a_n b_n x^{m+n}$, y si $p$ y $q$ son unidades el coeficiente de $x^{m+n}$ ha de ser cero para $m+n ≥ 1$. Por lo tanto, o bien $m = n = 0$ y ambos polinomios son de grado cero (luego unidades de $R$) o bien $a_n b_n = 0$, que ya hemos dicho que no ocurre por ser $R$ dominio de integridad.

\spart

Tomamos $ℤ_4[x]$ y $p(x) = 2x + 1$, $q(x) = -2x + 1$. Multiplicando, $p(x) · q(x) = -4x^2 +2x -2x + 1 = 1 \mod 4$, por lo que son unidades de $R[x]$ a pesar de no ser polinomios de grado 0.

\end{problem}

\begin{problem}
Demuestra, dando un contraejemplo, que el conjunto de todos los divisores de cero de un anillo, junto con el 0 no forman, en general, un ideal.
\solution

\doneby{Guille}

Tomamos $ℤ_{6}$, que tiene como divisores de cero $\set{2,3,4}$. Sin embargo, no es un grupo con la suma ya que $2 + 3 = 5$, que no es un divisor de cero.


\end{problem}

\begin{problem}
Demuestra, dando un contraejemplo, que el conjunto de todos los elementos de un anillo que no son unidades no forman, en general, un ideal.
\solution
\end{problem}

\begin{problem}
Demuestra que las siguientes afirmaciones son equivalentes:
\ppart $R$ es un dominio de integridad
\ppart el ideal $\zerogen$ es primo en $R$
\ppart el ideal $\zerogen$ es primo minimal en $R$
\solution
\end{problem}

\begin{problem}
Demuestra que las siguientes afirmaciones son equivalentes:
\ppart $R$ es un cuerpo
\ppart $R$ solo tiene dos ideales, $\zerogen$ y $R$
\ppart El único maximal de $R$ es $\zerogen$
\solution
\end{problem}

\begin{problem}
Sea $\K$ un cuerpo finito
\ppart Demuestra que $\K$ es una extensión finita de $\field_p$ con $p = $char$(\K)$. Concluye que $|\K| = p^n$ para algún $n \geq 1$, $n \in \nat$.
\ppart Demuestra que el homomorfismo de Frobenius que eleva cada elemento a la potencia p es un isomorfismo. Concluye que $\K$ es un cuerpo perfecto.
\ppart Demuestra que $\K$ es un cuerpo de descomposición para el polinomio $x^{p^n}-x$.

Sugerencia: observa que $\K^*$ es un grupo multiplicativo con $p^n - 1$ elementos, concluye que todos los elementos de $\K$ son raíces de $x^{p^n}-x$.
\ppart Deduce del apartado anterior que todos los cuerpos con $p^n$ elementos son isomorfos y que para cada natural $n \geq 1$ hay un cuerpo con $p^n$ elementos. Denotamos por $\field_{p^n}$ al único cuerpo salvo isomorfismo con $p^n$ elementos.
\ppart Usa el apartado c) para demostrar que si $n|m$ entonces $\field_{p^n} \subset \field_{p^m}$ y la extensión es de grado $m/n$.
\ppart Si $\field_{p^n} \subset \field_{p^m}$ demuestra que la extensión es simple.

Sugerencia: usa el Teorema del elemento primitivo.
\ppart Concluye que $\K$ es finito, entonces para cada natural $m \geq 1$, existe al menos un polinomio irreducible de grado m.

Sugerencia: usa los dos apartados anteriores.
\solution
\end{problem}
%%%%%%%%%%%%%%%%%%%%%%%%%%%%%%%%%%%%%%%%%%%%%%%%%%%%%%%%%%%%%%%%%%%%%%%
\section{Hoja 1: Anillos, ideales}

\begin{problem}
{\bfseries El radical de un ideal.}

Sea $I\subset R$ un ideal  y sea $\sqrt{I}$ su radical.

\ppart Demuestra que $\sqrt{I}$ es un ideal que además contiene a $I$.

\ppart Demuestra que todo ideal primo es radical.

\ppart Describe todos lo ideales de ${\mathbb Z}$ que sean radicales.

\ppart Sea $k$ un cuerpo. Describe todos lo ideales de $k[x]$ que sean radicales.

{\em Sugerencia: utiliza el ejercicio 1  de la parte de problemas para entregar.}

\solution
\end{problem}

\begin{problem}
{\bfseries Unión de ideales.}

La unión de ideales no siempre es un ideal.

\ppart Sean $I,J\subset R$ dos ideales. Demuestra que, en general, $I\cup J$ no es un ideal.

\ppart  Sea $I_1\subset I_2\subset \ldots \subset I_n \subset \ldots$  una sucesión creciente (no necesariamente finita) de ideales en $R$.
Demuestra que la unión $\cup_{n}I_n$ es un ideal en $R$.

\ppart Demuestra que en ${\mathbb Z}$, $\langle n\rangle +\langle m\rangle=\langle k\rangle$ con $k=\text{gcd}(n,m)$.
\solution
\end{problem}

\begin{problem}
{\bfseries Intersección de ideales.}

La intersección de ideales siempre es un ideal que no hay que confundir con el producto de ideales.

\ppart Sea $\{I_i\}_{i\in {\mathcal I}}$ una colección de ideales.  Demuestra que $\cap_{i\in {\mathcal I}} I_i$ es un ideal en $R$.

\ppart  Demuestra que en ${\mathbb Z}$, $\langle n\rangle \cap \langle m\rangle=\langle k\rangle$ con $k=\text{m.c.m.}(n,m)$.

\ppart  Sean $I, J\subset R$ dos ideales. Demuestra que $I\cdot J\subset I\cap J$; encuentra un ejemplo para el que la inclusión sea estricta.
\solution
\end{problem}

\begin{problem}
{\bfseries ¿Cuándo es la intersección de ideales igual al producto?}

Se dice que dos ideales $I, J\subset R$ son coprimos si $I+J=R$.

\ppart Encuentra dos ideales coprimos en
$Z$, y otros dos en ${\mathbb R}[x,y]$.

\ppart Demuestra que si $I$ y $J$ son coprimos, entonces  $I\cdot J=I\cap J$.

\ppart Sean $I_1,\ldots I_n$ ideales en $R$. Demuestra, usando inducción en $n$, que si los ideales son primos dos a dos entonces $\prod_iI_i=\cap_iI_i$.

\noindent{\em Sugerencia para (a): si $I$ y $J$ son coprimos, entonces existen $x\in I$ e $y\in J$ tales que $x+y=1$;
ahora multiplica la expresión anterior por  $a\in I\cap J$.

Sugerencia para (b): define $J=\prod_{I=1}^{n-1}I_i$; como los ideales $I_i$ son coprimos dos a dos, existen $y_i\in I_n$, y
$x_i\in I_i$, $i=1,\ldots, n-1$, tales que $x_i+y_i=1$; ahora utiliza el producto $\prod_{i=1}^{n-1}x_i$ para probar que
$I_n$ y $J$ son coprimos. }
\solution
\end{problem}

\begin{problem}
{\bfseries La geometría que veremos.}

Sean $I_1,\ldots, I_n$ ideales en $R$, y sea $\mathfrak p \subset R$ un ideal primo que contiene a
$\cap_iI_i$, i.e., $\cap_iI_i\subset \mathfrak p$. Demuestra que:

\ppart $I_i\subset \mathfrak p$ para algún $i\in \{1,\ldots,n\}$;

\ppart Si además  $\cap_iI_i=\mathfrak p$ entonces $I_i=\mathfrak p$ para algún $i\in \{1,\ldots,n\}$.

\noindent{\em {Sugerencia para (a)}: si $ I_i \nsubseteq \mathfrak p$ para $i=1,\ldots,n$ entonces para cada $i$ existe un elemento $b_i\in I_i $ con $b_i\notin \mathfrak p$; ahora considera $\prod_ib_i$. }
\solution
\end{problem}

\begin{problem}
{\bfseries Miramos ejemplos en $\ent$.}

En ${\mathbb Z}$ observa que $\langle 36\rangle \subset \langle 3\rangle$, $\langle 2\rangle$;
encuentra todas las maneras de escribir $\langle 36\rangle$ como intersección de ideales. A la vista del problema anterior,  ¿qué crees que va a suceder?
\solution
\end{problem}

\begin{problem}
{\bfseries Anillos locales.}

Demuestra que las siguientes afirmaciones son equivalentes:

\ppart $R$ es un anillo local con ideal maximal $\mathfrak m$;

\ppart $R$ contiene un ideal  $\mathfrak m$ tal que todo elemento de $R\setminus \mathfrak m$ es una unidad.
\solution
\end{problem}

\begin{problem}
Sea $k$ un cuerpo. Definimos el anillo de series formales en una variable sobre $k$ como el conjunto
$$k[|t|]:=\left\{\sum_{i=0}^{\infty}a_it^i: a_i\in k\right\}$$
con la suma y el producto habituales. Demuestra que $\alpha=\sum_{i=0}^{\infty}a_it^i
  \in k[|t|] $ es una unidad si y sólo si $a_0\neq 0$. Concluye que $k[|t|]$ es un anillo local cuyo único ideal maximal es además  principal.
\solution
\end{problem}

\noindent \fbox{{\bfseries  Problemas para entregar} }

\begin{problem}
{\bfseries Sobre los ideales en $K[x]$.}

\ppart Demuestra que $K[x]$ es un dominio euclídeo.
\ppart Demuestra que en $K[x]$ todos los ideales son principales.
\ppart Demuestra que en $K[x]$ un ideal es maximal si y solo si está generado por un polinomio irreducible.
\ppart Demuestra que en $k[x,y]$ no todos los ideales son principales. {\em Sugerencia: considera el conjunto $I$ formado por todos los polinomios $p(x,y)\in k[x,y]$ que se anulan en $(0,0)$; demuestra que es un ideal y que no es principal.}
\solution
\end{problem}

\begin{problem}
Decide de manera razonada para qué valores de $n$ el anillo ${\mathbb Z}_n$ es un anillo local. En aquellos casos en los que se trate de un anillo local indica cuál es el (único) ideal maximal.
\solution
\end{problem}

\begin{problem}
Considera el siguiente subanillo de ${\mathbb Q}(x)$:
$$R ≝\left\{\frac{p(x)}{q(x)}: \frac{p(x)}{q(x)} =\frac{r(x)}{s(x)} \text{ con }  s(0)\neq 0\right\}. $$
\ppart  Demuestra que  ${\mathfrak m} ≝ \left\{\frac{p(x)}{q(x)}\in R: p(0)=0\right\}$ es un ideal en $R$.
\ppart  Demuestra que $R$ es un anillo local con ideal maximal $\mathfrak m$.

{\em Sugerencia: puedes usar el ejercicio 7.}
\solution
\end{problem}

\begin{problem}
{\bfseries Conductores.}
Sean $I, J\subset R$ ideales. Definimos el conductor de $J$ en $I$ \index{Conductor} como:
$$(I:J) ≝\{a\in R: aJ\subset I).$$

\ppart Demuestra que $(I:J)$ es un ideal en $R$ que además contiene a $I$.

\ppart Calcula $(\langle 12\rangle : \langle n \rangle)$ para todo  $n\in {\mathbb Z}$.

\ppart Intuitivamente,  ?`qué es el ideal $(I: \langle a\rangle)$ para $a\in R$?

\ppart Sea ${\mathfrak p}\subset R$ un ideal primo. Calcula $({\mathfrak p}: I)$ para todo $I\subset R$.
?`Por qué crees que obtienes ése   resultado?

\ppart Demuestra que si $J=\langle a_1,\ldots, a_r\rangle$ entonces $(I:J)=\cap_{i=1}^r(I: \langle a_i\rangle).$
\solution
\end{problem}
