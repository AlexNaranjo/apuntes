% -*- root: ../AlgebraConmutativa.tex -*-

% Clase 7/3/2016
Vamos a ver algunas propiedades de estas variedades generadas por ideales. Por ejemplo, si tomamos dos ideales $J, I ⊂ K[x_1, \dotsc, x_n]$, si $J ⊂ I$ entonces $\V(J) ⊃ \V(I)$.

Los radicales también nos darán resultados interesantes. A priori, un radical nos debería dar una variedad más pequeña al ser un ideal más grande. La cuestión es que lo que nos dará en realidad será la misma variedad.

\begin{prop} Sea $I ⊂ K[x_1, \dotsc, x_n]$ un ideal, y consideramos su radical $\sqrt{I}$. Entonces $\V(I) = \V(\sqrt{I})$.
\end{prop}

\begin{proof}

\proofpart{$⊃$}

Como $I ⊂ \sqrt{I}$, es obvio que $\V(I) ⊃ \V(\sqrt{I})$.

\proofpart{$⊂$}

Lo primero que tenemos que hacer es tener cuidado cuando el ideal es el vacío. Ahora bien, si $\V(I) = ∅$, entonces $\V(\sqrt{I}) = ∅$ igualmente, ya que hemos demostrado antes que $\V(\sqrt{I}) ⊂ \V(I)$.

Podemos suponer entonces que $\V(I) ≠ ∅$. Sea $(a_1, \dotsc, a_n) ∈ \V(I)$ o, en otras palabras, que $∀p(x_1, \dotsc, x_n) ∈ I$ se tiene que $p(a_1, \dotsc, a_n) = 0$.

Queremos demostrar que $(a_1, \dotsc, a_n) ∈ \V(\sqrt{I})$ y por lo tanto que $∀q(x_1, \dotsc, x_n) ∈ \sqrt{I}$ tengamos que $q(a_1, \dotsc, a_n) = 0$. Ahora bien, por ser $q$ elemento del radical, elevado a una potencia $m > 0$ nos dará cero, luego en particular $q(a_1, \dotsc, a_n)^m = 0$. Pero $q(a_1, \dotsc, a_n) ∈ K$ que es un cuerpo y dominio de integridad, así que si elevado a una potencia nos da $0$ tiene que ser él mismo cero, y por lo tanto $(a_1, \dotsc, a_n) ∈ \V(\sqrt{I})$.
\end{proof}

\subsection{La operación $\I(S)$}

Vamos a tomar un subconjunto $S ⊂ \A_K^n$. Nos fijaremos en todos los polinomios en $n$ variables que se anulan en $S$. Es decir, \[ \I(S) = \set{p(x_1, \dotsc, x_n) ∈ K[x_1, \dotsc, x_n] \tq ∀(a_1, \dotsc, a_n) ∈ S \; p(a_1, \dotsc, a_n) = 0} \]

No es difícil ver que $\I(S)$ es un ideal: no es vacío ($0 ∈ \I(S)$), está cerrado con la suma y la propiedad de absorción también se cumple trivialmente.

\begin{wrapfigure}{R}{0.4\textwidth}
\centering
\begin{tikzpicture}
\draw[->] (-0.5, 0) -- (3,0);
\draw[->] (0, -0.5) -- (0,2);

\draw[|-|, very thick, blue] (0,0) node[anchor = north west, yshift = -0.05cm, xshift = -0.05cm] {$0$} -- (1,0) node[below, yshift = -0.05cm] {$1$};
\end{tikzpicture}
\caption{El intervalo $[0,1]$ no es una variedad algebraica.}
\label{fig:Interv01}
\end{wrapfigure}

Por ejemplo, podemos buscar todos los polinomios que se anulan en el intervalo $S= [0,1]$. Está claro que $\gen{y} = \I(S)$, pero no vamos a poder encontrar un polinomio que se anule sólo en $S$. La razón es que $S = [0,1]$ no es una variedad algebraica. Esto nos va a dar un lema interesante.

\begin{lemma} Sea $S ⊂ \A_K^n$ un subconjunto del espacio afín. Entonces $\V(\I(S)) ⊃ S$, y la igualdad $\V(\I(S)) = S$ se da si y sólo si $S$ es una variedad algebraica afín.
\end{lemma}

\begin{proof} Por definición, siempre tenemos que $S ⊂ \V(\I(S))$. Supongamos que $S = \V(\I(S))$. Es claro que $S$ es una variedad algebraica afín, porque la igualdad indica que $S$ es el conjunto de ceros del ideal $\I(S)$.

Vamos a ir con la demostración al otro lado. Supongamos que $S$ es una variedad algebraica afín, y queremos demostrar que $S = \V(\I(S))$.

Como $S$ es una v.a.a., entonces $∃ J ⊂ K[x_1, \dotsc, x_n]$ tal que $S = \V(J)$, luego $J ⊂ \I(S)$. Entonces podemos hacer una cadena \[ S = \V(J) ⊃ \V(\I(S)) ⊃ S \], por lo que todas las inclusiones tienen que ser igualdades.
\end{proof}

Este lema lo podemos relacionar con la topología de Zariski que veíamos antes, que nos decía que los cerrados eran las variedades algebraicas.

\begin{corol} Sea $S ⊂ \A_K^n$. Entonces la \concept{Clausura\IS de Zariski} de $S$ es $\adh{S} = \V(\I(S))$, y entonces $\adh{S}$ es la variedad algebraica más pequeña que contiene a $S$.
\end{corol}

\begin{proof}
Sabemos que $S ⊂ \V(\I(S))$, donde $\V(\I(S))$ es una variedad algebraica afín. Tenemos que probar que además es la variedad más pequeña que contiene a $S$.

Hacemos la demostración por reducción al absurdo, suponiendo que existe una variedad algebraica afín $X ⊃ S$ y además $X \subsetneq \V(\I(S))$. Ahora bien, es obvio que $\I(S) ⊃ \I(X) ⊃ \I(\V(\I(S)))$. Podemos volver a tomar la variedad y entonces \[ \V(\I(S)) ⊂ \underbracket{\V(\I(X))}_{ = X} ⊂ \underbracket{\V(\I(\V(\I(S))))}_{= \V(\I(S)))} \], luego $X = \V(\I(S))$, contradicción.
\end{proof}

Dos observaciones:

\begin{prop}
Dado $S ⊂ \A_K^n$, el ideal $\I(S)$ es un ideal radical.
\end{prop}

\begin{proof} Ejercicio para el lector.
\end{proof}

\begin{prop} Sea $X ⊂ \afesp$ una variedad algebraica afín. Entonces existe $J ⊂ K[x_1, \dotsc, x_n]$ tal que $\V(J) = X$.
\end{prop}

El ideal $J$ no es el único que determina a $X$. Sin embargo, el ideal $\I(X)$ es el ideal más grande de $K[x_1, \dotsc, x_n]$ cuyos ceros son $X$, esto es, es el ideal más grande tal que $X = \V(\I(X))$.

Esto nos permite asociar un ideal radical a $X$ sin ninguna ambigüedad. Diremos entonces que $I(X)$ es el ideal que determina a $X$.
