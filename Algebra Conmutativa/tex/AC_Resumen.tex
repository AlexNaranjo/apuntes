% -*- root: ../AlgebraConmutativa.tex -*-

\chapter{Resumen rápido}

\section{Anillos I: ideales, homomorfismos y cocientes}

\subsection{Anillos}

Partimos del anillo (\fref{def:Anillo}), un conjunto con operaciones suma y producto (ambas conmutativas), asociativo y distributivo con respecto al producto y al que nosotros pediremos además que tenga una unidad multiplicativa. Podemos poner adjetivos a los elementos de anillos según su comportamiento:

\begin{itemize}[itemsep = -2pt]
\item \textbf{Unidad}: Elemento con inverso multiplicativo.
\item \textbf{Divisor de cero}: Elemento que multiplicado por otro da cero.
\item \textbf{Nilpotente}: Elemento que elevando a una cierta potencia da cero.
\end{itemize}

Igualmente, podemos poner clasificar los anillos según sus comportamientos:

\begin{itemize}[itemsep = -2pt]
\item \textbf{Dominio de integridad}: sin divisores de cero.
\item \textbf{Anillo reducido}: sin nilpotentes distintos de cero
\item \textbf{Cuerpo}: Todo elemento no nulo tiene inverso multiplicativo.
\item \textbf{Anillo local}: Sólo con un ideal maximal.
\item \textbf{Anillo semilocal}: Con un número finito de maximales.
\end{itemize}

\subsection{Ideales}

Como siempre, nos interesará estudiar los subanillos (anillo contenido dentro de otro) y especialmente los ideales (\fref{def:Ideal}). Un subconjunto $I ⊆ R$ será ideal si no es vacío, es subgrupo con la suma y tiene la propiedad de absorción ($r∈I,\,i ∈ I \implies ri ∈ I$). A la hora de las demostraciones, no hará falta demostrar que es subgrupo con la suma, nos valdrá demostrar que $a,b ∈ S \implies a+b ∈S$ ó $a-b ∈ S$ (\fref{prop:def_ideal2} y  \ref{prop:def_ideal3}). Sobre los ideales podemos sacar varias operaciones:

\begin{itemize}[itemsep = -2pt]
\item \textbf{Generación}: Dado $T ⊂ R$, construimos el ideal que lo contiene como $\gen{T} = \set{\sum r_i t_i\tq r_i ∈ R\,t_i ∈ T}$.
\item \textbf{Intersección}: La intersección de ideales es ideal.
\item \textbf{Unión/suma de ideales}: La unión en general no es ideal, así que definimos la suma (\fref{def:IdealSuma}) como $I+J = \set{a+b \tq a ∈ I, \, b∈J}$ que sí es un ideal.
\item \textbf{Producto}: De forma análoga a la suma (\fref{def:IdealProducto}).
\item \textbf{Radical de un ideal}: (\fref{def:RadicalIdeal}) Elementos del anillo que tienen alguna potencia en el ideal. En particular, el nilradical será el ideal de los nilpotentes. Además, $\sqrt{I} = \bigcap \pideal$ donde $\pideal$ son ideales primos de $R$ que contienen a $I$.
\end{itemize}

Al igual que con los anillos, podemos poner adjetivos a los ideales:

\begin{itemize}[itemsep = -2pt]
\item \textbf{Propio}: Distinto del todal.
\item \textbf{Radical}: Si su radical es él mismo.
\item \textbf{Primo}: $I \subsetneq R$ tal que $a · b ∈ I \implies a∈I$ ó $b ∈I$.
\item \textbf{Maximal}: Ideal propio maximal con respecto a la inclusión de ideales (\fref{def:IdealMaximal}).
\item \textbf{Principal}: Generado por un único elemento.
\item \textbf{Primo minimal}: Primo y minimal con respecto a la inclusión de primos.
\end{itemize}

Un teorema importante (\ref{thm:IdealContenidoMaximal}) es ver que cualquier ideal propio está contenido en un maximal de $R$. Para la demostración se usa el \nref{lem:Zorn}, que nos dice que un conjunto no vacío parcialmente ordenado con cualquier cadena creciente acotada tiene un maximal.

En general, los ideales maximales nos dan cosas interesantes: cualquier maximal es primo, y además si en un anillo todos los ideales son primos es un cuerpo.

\subsection{Homomorfismos}

Un homomorfismo de anillos (\fref{def:Homomorfismo}) es una aplicación $f$ compatible con la suma y el producto, y que además nos lleva la unidad multiplicativa en unidad multiplicativa. Una propiedad muy útil es que $f$ es inyectiva si y sólo si $\ker f = \zerogen$. La imagen de ideales por $f$ no tiene por qué ser ideal, pero la inversa de ideales sí que es ideal.

\subsection{Cocientes}

Dado un ideal $I$, se define una relación de equivalencia $a \sim b \iff a - b ∈ I$ que nos permite definir el conjunto cociente $\quot{R}{I}$ como el conjunto de clases de equivalencia de $R$ módulo $I$. El conjunto cociente es igualmente anillo conmutativo con unidad, y la aplicación paso al cociente π que lleva cada elemento a su clase es un homomorfismo. π es sobreyectiva y $\ker π = I$.

Además de las propiedades de homomorfismos, π nos lleva ideales de $R$ en ideales de $\quot{R}{I}$, y $\inv{π}(π(L)) = L + I$. También respeta la inclusión de ideales que contienen a $I$ (si $I ⊆ M \subsetneq L$ entonces $π(M) \subsetneq π(L)$) y la primalidad (ideales primos en $R$ son primos en el cociente).

El anillo cociente nos permite estudiar ideales. $I$ es primo sii $\quot{R}{I}$ es dominio de integridad; es maximal sii $\quot{R}{I}$ es cuerpo y es radical sii $\quot{R}{I}$ es reducido.

\subsection{Teoremas de isomorfía}

Los teoremas de isomorfía nos dan, atención, isomorfías. Dado un homomorfismo $\appl{f}{R}{S}$, el \nref{thm:IsomorfiaAnillos1} nos dice que hay un isomorfismo $\quot{R}{\ker f} \simeq \img f$ y un homomorfismo inducido $\appl{\bar{f}}{\quot{R}{\ker f}}{S}$.

Más generalmente (\fref{prop:FactorizacionHomomorfismo} y \ref{prop:FactorizacionHomomorfismo2}) $I ⊂ \ker f$ si y sólo si $f$ factoriza por el homomorfismo $\appl{f'}{\quot{R}{I}}{S}$ (en otras palabras, si y sólo si $f = f' ○ π$ donde π es el paso al cociente).

El \nref{thm:IsomorfiaAnillos2} nos dide que dados $I ⊂ J ⊂ R$ ideales de un anillo, entonces $\quot{R}{J} \simeq \frac{\quot{R}{I}}{\quot{J}{I}}$ o, en otras palabras, que el $I$ se puede ``cancelar'' y es lo mismo. Este teorema nos da una demostración sencilla para ver que la imagen inversa de un ideal primo es otro primo.

Por último, el \nref{thm:IsomorfiaAnillos3} nos dice que dados $I,J$ ideales en $R$ entonces $\quot{I+J}{I} \simeq \quot{J}{I ∩ J}$.

\section{Anillos II: Módulos y más}

\subsection{Módulos}

Dado $R$ un anillo, un $R$-módulo es un grupo abeliano con la suma con una operación externa $R×M \to M$ que cumple unas propiedades sensatas, a saber, distributividad por la izquierda y derecha, asociatividad ($r(r'm) = (rr')m$) y unidad ($\one_R · m = m$).

Un $R$-módulo estará finitamente generado si cualquier elemento es combinación lineal finita de elementos de $r$ por una ``base'' de elementos de $M$.

\subsection{Localización multiplicativa}

Una parte multiplicativa es un conjunto cerrado con el producto, que nos permite construir el localizado de $R$, dado por $\inv{S}R = \quot{R×S}{\sim}$ donde $(r,s) \sim (r',s') \iff ∃s'' ∈ S \tq s''(rs'-r's) = 0$. Una primera propiedad es ver que $\inv{S}R = \set{0}$ si y sólo $S$ contiene nilpotentes.

\seprule[Fin temario parcial 1]

Dos propiedades rápidas sobre la localización: tomando el homomorfismo de inclusión $ψ(r) = (r,1)$ de $R$ en $\inv{S}R$, tenemos las siguientes propiedades (ver \fref{thm:PropsLocalizacion}):

\begin{itemize}
\item $ψ(s)$ es invertible para $s ∈ S$.
\item $ψ(a) = 0$ si y sólo si $as = 0$ para algún $s ∈ S$.
\item Cualquier elemento de $\inv{S}R$ se puede expresar como $ψ(r) · \inv{ψ}(s)$.
\end{itemize}

Además, esas propiedades determinan completamente a $\inv{S}R$: si tenemos otro homomorfismo $\appl{f}{R}{B}$ que cumple esas propiedades entonces $B \simeq \inv{S}R$.

\begin{wrapfigure}[9]{R}{0.4\textwidth}
\vspace{-15pt}
\centering
\begin{tikzpicture}
\matrix (m) [matrix of math nodes,row sep=4em,column sep=2em,minimum width=2em]
{
	R &  &  B\\
	& S^{-1}R &  \\};
\path[-stealth]
(m-1-1) edge node [above] {$f$} (m-1-3)
(m-1-1) edge node [below] {$\psi$} (m-2-2)
(m-2-2) edge node [right] {$g$} (m-1-3);
\end{tikzpicture}

\caption{Diagrama conmutativo para la propiedad universal de la localización.}
\label{fig:Resumen:PropUnivLocalizacion}
\end{wrapfigure}


Sin embargo, el teorema verdaderamente relevante de esta parte es la \nlref{thm:PropUniversalLoc}, que nos dice que existe una única aplicación $g$ que hace conmutar el diagrama de la \fref{fig:Resumen:PropUnivLocalizacion}.

\subsection{Ideales y localización}

La localización funciona decentemente con los ideales, y se cumplen varias propiedades interesantes:

\begin{itemize}
\item Dado un ideal $J ⊂ \inv{S}R$, entonces $\psi(\psi^{-1}(J))^e = J$.
\item Dado un ideal $I ⊂ R$, entonces $\psi^{-1}(\psi(I)^e)=\bigcup_{s\in S}(I:s)$ con $(I:s)=\set{ r \in R \tq r\cdot s \in I }$.
\item Dado un ideal $I ⊂ R$, entonces $\psi^{-1}(\psi(I)^e)=R \iff I \cap S \neq \emptyset$.
\item Todo ideal de $\inv{S} R$ es el extendido de algún ideal de $R$.
\end{itemize}

Para ideales primos, se cumplen las siguientes tres propiedades:

\begin{itemize}
\item Si $\pideal \cap S \neq \emptyset \implies \psi(\pideal)^e=S^{-1}R$.
\item Si $\pideal \cap S = \emptyset \implies \psi(\pideal)^e$ es un ideal primo en $S^{-1}R$.
\item Hay una correspondencia biyectiva entre ideales primos $\pideal ⊂ R$ que no intersecan con $S$ ($\pideal ∩ S = ∅$) y los ideales primos de $\inv{S} R$.
\end{itemize}

Para terminar, dos formas de construir localizados. Dado un $a ∈ R$, construimos la parte multiplicativa $S = \set{a^n}_{n ∈ ℕ}$ y entonces definiremos $R_a ≝ R_{\set{a}} ≝ \inv{S} R$ como $R$ localizado en $a$.

Por otra parte, si tomamos $\pideal ⊂ R$ un ideal primo en $R$, entonces definimos $S = R \setminus \pideal$ y $R_\pideal ≝ \inv{S}R$ como $R$ localizado en el ideal primo \pideal. En particular, $R_\pideal$ es un \nlref{def:AnilloLocal}.

\subsection{Anillos noetherianos}

Un \nlref{def:AnilloNoetheriano} es un anillo para el cual todo ideal es finitamente generado como $R$-módulo. Los dominios de ideales principales (donde todos los ideales están generados por un único elemento) o los cuerpos son dos ejemplos de anillos noetherianos.

Hay dos caracterizaciones útiles para anillos noetherianos (\fref{prop:caracterizacion_noetheriano}). Que $R$ sea noetheriano es equivalente a

\begin{itemize}
\item Toda cadena creciente de ideales estabiliza: $I_1 ⊆ I_2 ⊆ \dotsb I_n = I_{n+1} = I_{n+2} = \dotsb$.
\item Todo conjunto no vacío de ideales en $R$ tiene un elemento maximal (no tiene por qué ser un ideal maximal).
\end{itemize}

Además, el \nref{thm:tma_base_hilbert} nos dice que si $R$ es noetheriano, entonces $R[x]$ es noetheriano.

Por último, los pasos al cociente y al localizado conservan ``noetherianidad'': si $R$ es noetheriano y $I ⊂ R$ un ideal, entonces $\quot{R}{I}$ es noetheriano igualmente (\fref{prop:NoetherianoCociente}), y si $S ⊂ R$ es una parte multiplicativa entonces $\inv{S}R$ también es noetheriano (\fref{prop:NoetherianoLocalizacion}).

\section{Variedades algebraicas afines}

Para esta parte, haremos geometría con álgebra, ¡yupi! La herramienta básica será la \nlref{def:variedad_algebraica}, subconjuntos del espacio afín $\afesp$ cuyos puntos son todos los ceros de un conjunto $F ⊂ \kbb[x_1, \dotsc,x_n]$ de polinomios. Los conjuntos finitos de puntos son varieades, como lo son las rectas. Como observación, un conjunto infinito de puntos aislados no puede ser variedad algebraica afín.

Nos interesará saber que toda v.a.a. es el conjunto de solución de un número finito de ecuaciones polinómicas y es igual a $\V(I)$ para $I ⊂ \kbb[x_1, \dotsc, x_n]$ un ideal.

En cuanto a operaciones, sean $X_1, X_2$ dos variedades algebraicas afines con $\V(I_1) =X_1, \, \V(I_2) = X_2$. Entonces:

\begin{itemize}
\item La unión es v.a.a.: $X_1 ∪ X_2 = \V(I_1 ∩ I_2) = \V(I_1 · I_2)$, no necesariamente válido para uniones infinitas.
\item La intersección es v.a.a.: $X_1 ∩ X_2 = \V(I_1 ∪ I_2) = \V(I_1 + I_2)$, válido también para intersecciones infinitas.
\item $\V(I_1) = \V(\sqrt{I_1})$.
\end{itemize}

En las variedades algebraicas se puede definir una topología (la \textbf{Topología de Zariski}) en la cual los cerrados son variedades algebraicas afines, y cumplen los axiomas de los cerrados de la topología.

También se puede definir la operación $\I(S)$ para $S ⊂ \afesp$, definida como el ideal más grande de los polinomios en $n$ variables que se anulan en $S$. Es un ideal radical, no vacío, y que cumple que $\V(\I(S)) ⊇ S$, co igualdad si y sólo si $S$ es v.a.a.. Además, en particular nos permite definir la \textbf{clausura de Zariski} como $\adh{S} = \V(\I(S)))$.

Se puede discutir la \textbf{irreducibilidad} de las v.a.a.. Estaremos ante una \fref{def:VaaIrreducible} si no se puede expersar como unión de dos variedades algebraicas propias. Una variedad $X$ será irreducible si y sólo si $\I(X)$ es primo (\fref{prop:VaaIrreducibleIdPrimo}). Un ejemplo de cálculo de un ideal primo y variedad irreducible se puede ver en el \fref{ej:H5:VariedadIrreducible}.

Además, toda v.a.a. se puede escribir como unión finita de variedades irreducibles de forma única (salvo orden y redundancias).

\section{Teorema de los ceros de Hilbert}

En esta sección trateremos de construir y estudiar cuándo hay correspondencias 1 a 1 entre ideales radicales y variedades algebraicas afines. El \nref{thm:tma_0_hilbert} nos dice que $J = \I(\V(J))$ si y sólo si $J$ es radical y el cuerpo base es algebraicamente cerrado.

La demostración\footnote{Que resumo aquí por si entra en el examen.} pasa por dos teoremas auxiliares para cuerpos $\K[x_1, \dotsc, x_n]$ con $\K$ algebraicamente cerrado: el \nref{thm:Basico}, que nos dice que los ideales maximales en cuerpos  son de la forma $\gen{x_1 - a_1, \dotsc, x_n - a_n}$; y el \nref{thm:Debil}, que nos dice que para ideales propios $I ⊂ \K[x_1, \dotsc, x_n]$ entonces $\V(I) ≠ ∅$.

El teorema débil sale trivialmente del teorema básico. A su vez, el teorema débil implica el teorema de los ceros de Hilbert. Para ello, se busca demostrar que $\I(\V(J)))$ es radical, y se añade una variable más al cuerpo de polinomios y otra componente $g(x_1, \dotsc, x_n)t - 1$ a los polinomios $f_i$ que generan $J$ (que sabemos que son finitos por ser $J$ noetheriano) con $g ∈ \I(\V(J))$. $g$ se anula en $\V(J)$, luego la componente que hemos añadido nunca se anula y ese ideal será el total.

En particular, podremos encontrar una combinación lineal de los generadores $f_i$ de $J$ y $g·t - 1$ que nos de como resultado $1$. Los coeficientes son polinomios en las variables $x_1, \dotsc, x_n,t$, aunque como $gt - 1 = 0$, se despeja y se reemplaza $t = \frac{1}{g}$. Dado que todos los coeficientes son polinomios, podemos tomar un $N$ suficientemente grande de tal forma que cancelamos los denominadores.

Así, nos quedará que habremos expresado $g^N$ como combinación lineal de elementos de $J$ por polinomios en $\K[x_1, \dotsc, x_n]$ para todo $g ∈ J$, luego $J$ es radical y la única opción es que $J = \I(\V(J))$.

La prueba del teorema básico requiere bastante más teoría, que pasamos a ver a continuación.

\subsection{Extensiones enteras de anillos}

Las extensiones enteras serán extensiones (inclusiones $A ⊂ B$) tales que todos los elementos $b ∈ B$ son enteros sobre $A$ (\fref{def:ElementoEntero}, expresables como raíz $p(b) = 0$ de un polinomio mónico no nulo en $A[x]$). En cuerpos, es equivalente que un elemento sea entero y que sea algebraico, aunque no ocurrirá en anillos generales (p.e., $\frac{1}{2}$ es algebraico sobre $ℤ$ pero no entero).

Una caracterización interesante es que $b$ es entero sobre $A$ si y sólo si $A(b)$ es un $A$-módulo finitamente generado. Además, dada una extensión entera $A ⊂ B$, $A$ es cuerpo si y sólo si $B$ también lo es.

Las extensiones algebraicas $K ⊂ L$ se pueden caracterizar de la siguiente forma: $α ∈ L$ será algebraico sobre $K$ si y sólo si $K[α]$ es un cuerpo.

\textit{Me falta la última parte de la prueba del teorema básico}

\seprule[Fin temario parcial 2]

\section{Anillos de coordenadas y morfismos entre variedades}

En esta parte pasamos a estudiar las funciones que se pueden definir sobre variedades algebraicas afines. La base será la \nlref{def:FuncionRegular}, función que se comporta como un polinomio; e identificaremos el \nlref{def:AnilloCoordenadas} como el anillo de las funciones regulares de una variedad $X$ al cuerpo base $\K$, que es igual a $\quot{\K[x_1, \dotsc, x_n]}{\I(X)}$.

También definiremos el \nlref{def:MorfismoVAA}, que es una función que lleva de una variedad $X$ a otra $Y$, y cuyas coordenadas son funciones regulares (polinomios). Además, induce un homomorfismo entre anillos de coordenadas $\appl{φ^*}{\K[Y]}{\K[X]}$, y de hecho también ocurre al revés: un homomorfismo de anillos de coordenadas induce un morfismo de v.a.a. (\fref{prop:HomomorfismoInduceMorfismo}). Todo esto también funciona bien con la composición ($(ψ○φ)$ es morfismo y $(ψ○φ)^* = ψ^* ○ φ^*$).

Un lema importante es el \fref{lem:morf}, que nos dice que $\img φ = ``\V(\ker φ^*)''$, con las comillas porque el núcleo está en $\K[Y]$ y no en el anillo de polinomios, aunque hay una correspondencia biyectiva entre esos ideales.

Otra relación entre morfismo y homomorfismo inducido es la siguiente:
\begin{itemize}
\item $φ$ dominante (con $\adh{\img φ} = Y$) si y sólo si $φ^*$ inyectiva (\fref{crl:morf}).
\item $φ$ es inyectivo sólo si $φ^*$ es sobreyectivo (\fref{lem:MorfismoInducidoSobreyectivo}).
\end{itemize}

\subsection{Morfismos y fibras}

La idea es, dado un morfismo $\appl{φ}{X}{Y}$, estudiar qué puntos de $X$ llevan a una subvariedad $W ⊂ Y$, esto es, estudiar $\inv{φ}(W)$.

La descripción algebraica detallada está en la \fref{sec:AlgebraFibras}, aunque rápidamente tenemos que $\inv{φ}(W) = \V(ψ(\I(W)))$ donde $ψ = φ^* ○ π$ con $π$ la proyección $\K[y_1, \dotsc, y_m] \to \K[Y]$.

\section{Dimensión de variedades algebraicas afines}

A las variedades algebraicas se les da una noción de dimensión a través de la \nref{def:DimKrull}, que se define como la máxima longitud de cadenas estrictamente crecientes de ideales primos, menos 1. Así, un cuerpo tiene dimensión de Krull cero (sólo hay un ideal primo, luego la cadena tiene longitud uno y la dimensión es $0$); y $ℤ$ tiene dimensión $1$ por la cadena $\set{0} \subsetneq \gen{p}$ con $p$ primo.

Con esto, definiremos la dimensión de una variedad como la dimensión de Krull de su anillo de coordenadas, creyéndonos por el camino que $\dk \K[x_1, \dotsc, x_n] = n$.

\subsection{Dimensión de Krull en extensiones enteras}

Las extensiones enteras $A ⊂ B$ nos dan algunas cosas interesantes sobre los primos, como por ejemplo que $\mideal ⊂ B$ es maximal si y sólo si $\mideal ∩ A$ lo es en $A$ (\fref{prop:ExtensionMaximal}) o que si $\pideal ⊂ A$ es primo entonces hay otro primo $\aideal$ en $B$  tal que $\aideal ∩ A = \pideal$ (\fref{prop:ExtensionEnteraPrimos}), y además ese $\aideal$ es único (\fref{prop:ExtensionPrimosUnico}).

Todo eso nos permite probar el \fref{crl:DimKrullExtension} ($\dk A ≥ \dk B$) y el \nref{thm:GoingUp}, que nos permite aumentar una cadena de primos en $B$ a partir de una cadena más larga de primos en $A$ que a su vez da la igualdad en la dimensión de Krull (\fref{crl:DimKrullExtensionIgual}): \[ \dk A = \dk B \]

\subsubsection{Morfismos finitos}

Todo eso de antes tiene una interpretación geométrica a través del \nlref{def:MorfismoFinito} ($\appl{φ}{X}{Y}$ es finito si y sólo si $\K[Y] ⊂ \K[X]$  es una extensión finita). El \fref{thm:MorfismoFinito} dirá que φ es sobreyectivo y que además el número de preimágenes de un punto de $Y$ es finito.
