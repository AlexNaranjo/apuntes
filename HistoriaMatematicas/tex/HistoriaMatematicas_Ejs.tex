% -*- root: ../HistoriaMatematicas.tex -*-
\section{Ejercicios mandados en clase}
\begin{problem}[1]
Demostrar que cada uno de los ángulos de un triángulo equilátero es de $\frac{π}{3}$.
\solution
\end{problem}

\begin{problem}[2]
Demostrar que cada una semicircunferencia, cualquier punto de la misma nos sirve para construir un triángulo rectángulo (con el ángulo recto en el punto dado) que tiene como base el diámetro de la circunferencia.
\solution
\end{problem}

\begin{problem}[3]
Demostrar que dado el triángulo del ejercicio anterior, el ángulo formado entre base y el segmento que une el centro con el punto de la semicircunferencia es el doble del ángulo izquierdo del triángulo.
\solution
\end{problem}

\begin{problem}[4]
Buscar la fracción continua de $\sqrt[3]{2}$.

\textbf{Pista:} Calcular $\sqrt[3]{2}$ por el método de Newton con 4 decimales y ver cuántos pasos tarda.
\solution
\end{problem}

\section{Hoja 1}
\subsection{Babilonia, Egipto y China}
\begin{problem}[1]
Ejemplos de números escritos por el método babilónico de cuñas

\solution

\end{problem}

\begin{problem}[2]
Ejemplos de números escritos por el método chino tradicional usado hoy día

\solution

\end{problem}

\begin{problem}[3]
¿Cuál es la mayor terna pitagórica que conocían los babilónicos? Extraer la información de la tablilla Plimton 322

\solution

\end{problem}

\begin{problem}[4]
Raíz cuadrada por el \textbf{método de Heron}.

Sea
\[x_{n+1} = \frac{1}{2} \left(x_n + \frac{R}{x_n} \right)\]
demostrar que existe el límite y que la serie converge a  $\sqrt{R}$. Describir el conjunto de los datos iniciales que llevan a convergencia.
\solution

\end{problem}

\subsection{Grecia}
\begin{problem}[5]
Elaborar una tabla con el alfabeto griego, los símbolos, la fonética y el nombre y pronunciación usual de cada letra.

\obs Epsilon y eta; i e ípsilon; o y omega; zeta y theta; nu, xi, phi, psi y los diptongos ai=e, ei=i

\solution

\end{problem}

\subsection{Thales}
\begin{problem}[6]
\ppart Demostrar que en un triángulos con dos lados iguales hay dos ángulos iguales y viceversa

\ppart Demostrar que la suma de ángulos de un triángulo es un ángulo plano

\ppart Demostrar el siguiente teorema de Tales: el ángulo inscrito en una semicircunverencia es recto. Demostrar el inverso

\ppart Demostrar que el ángulo inscrito en una circunferencia vale la mitad del ángulo central correspondiente

\solution

\end{problem}

\subsection{Pitágoras}
\begin{problem}[7]
Escribir la definición de las ternas pitagóricas. Describir todas las ternas pitagórigas mediante dos fórmulas. Listar todas aquellas ternas que tienen todos los números menores que 500

\solution

\end{problem}

\begin{problem}[8]
Método de Diofanto para calcular las ternas pitagóricas. Describir el método y el uso de los números fraccionarios.

Aplicar el método disparando de $N=(0,-1)$ o desde $N'=(1,0)$. Comparar

\solution

\end{problem}

\begin{problem}[9]
Leer el capítulo 1 de Stillwell (3rd Edition). Hacer los ejercicios 1.2.{1,3,4}, 1.2.4, 1.3.{1,2,3}.

Especialmente el último

\solution

\end{problem}

\begin{problem}[10]
Sea $(a,b,c)$ una terna pitagórica generada por $(p,q)$ con $a,b,c,p,q$ enteros positivos, demostrar que si $p=q+1$, entonces $c=b+1$.

\solution

\end{problem}

\begin{problem}[11]
Demuestra que al menos uno de los tres términos de una terna pitagórica es par.

\solution

\end{problem}