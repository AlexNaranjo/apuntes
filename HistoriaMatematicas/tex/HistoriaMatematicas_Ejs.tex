% -*- root: ../HistoriaMatematicas.tex -*-
\section{Ejercicios mandados en clase}
\begin{problem}[1]
Demostrar que cada uno de los ángulos de un triángulo equilátero es de $\frac{π}{3}$.
\solution
\end{problem}

\begin{problem}[2]
Demostrar que cada una semicircunferencia, cualquier punto de la misma nos sirve para construir un triángulo rectángulo (con el ángulo recto en el punto dado) que tiene como base el diámetro de la circunferencia.
\solution
\end{problem}

\begin{problem}[3]
Demostrar que dado el triángulo del ejercicio anterior, el ángulo formado entre base y el segmento que une el centro con el punto de la semicircunferencia es el doble del ángulo izquierdo del triángulo.
\solution
\end{problem}

\begin{problem}[4]
Buscar la fracción continua de $\sqrt[3]{2}$.

\textbf{Pista:} Calcular $\sqrt[3]{2}$ por el método de Newton con 4 decimales y ver cuántos pasos tarda.
\solution
\end{problem}
