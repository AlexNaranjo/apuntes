% -*- root: ../AprendizajeAutomatico.tex -*-

\section{Hoja 1}

\begin{problem}[1]

Un taxi golpea a una persona de noche y huye. En la ciudad operan dos compañías de taxis. Una verde y otra azul. El 85\% de los taxis son verdes. 

\ppart
¿De qué color es más probable que sea el taxi?

\ppart
Un testigo dice que es azul con fiabilidad del 80\%

\solution
\paragraph{Notación} H de hipótesis, D de Datos e I de información.

\spart
Se deja como ejercicio para el lector experimentado.

\spart
P(H=verde) = 85\%
P(H=azul) = 15\%

\[
P\left( H=azul | D=azul, I \right) = \frac{P(D=a|H=a,I)·P(H=A|I)}{P(D=a|I)}
\]

Para calcularlo, necesitamos saber la probabilidad con la que el testigo dice que un taxi es azul.\footnote{Suponemos que el testigo se equivoca igual de azul a verde que de verde a azul}
\[
P(D=a|I) = P(D=a|H=a)·P(H=a) + P(D=verde | H=v) · P(H=v) = 0.8·0.15 + 0.2·0.85 = 0.29
\]

Con esto ya tenemos:

\[
 P\left( H=azul | D=azul, I \right) = \frac{0.8·0.15}{0.29} = 0.41
\]

Por el contrario, la probabilidad de que el taxi sea verde (utilizando la información del testigo) es de un 0.59.

\paragraph{Notación} Observamos que todas las probabilidades son condicionadas a I, que no influye en nada y omitiremos cuando no sea influyente
\end{problem}


\begin{problem}[4]
Tenemos 2 bolsas. La bolsa A tiene 2 bolas negras y 3 blancas. La bolsa B tiene 3 bolas negras y 2 blancas.

\ppart Se elige una de las bolsas al azar, y escogemos al azar una bola. ¿Cuál es la probabilidad de que sea negra?

\ppart Habiendo extraido una bola negra. ¿Cuál es la bolsa más probable?
\solution

\spart 
\[
P(negra) = P(A,negra) + P(B,negra) = P(negra|A,I)·P(A) + P(negra|B,I)·P(B)\]
\[
P(negra) = 0.4*0.5+0.6*0.5 = 0.5
\]
\spart La intuición nos dice que será de la bolsa B, ya que esta tiene más bolas negras, y las 2 bolsas son equiprobables.

Vamos a justificarlo matemáticamente:

P(H=A) = 50\%

D=negra

\[
P(A|D=negra) = \frac{P(negra|A)·P(A)}{P(negra)}
\]




\end{problem}