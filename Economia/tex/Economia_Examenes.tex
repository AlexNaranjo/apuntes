% -*- root: ../Economia.tex -*-
\section{Final Junio 2011}
\begin{problem}[1]
Consideramos una opción que pagará a vencimiento (dentro de un año) $(S_4-A_4)^+$ siendo $A_4=\frac{1}{4}\sum_{i=1}^4S_i$, la media de los valores del subyacente dentro de tres, seis, nueve y doce meses.

\ppart Calcular el valor de esta opción. Para ello supondremos que la dinámica para los doce próximos meses viene descrita por un modelo binomial de cuatro periodos. En cada periodo el activo puede subir o bajar un 3\%. El tipo libre de riesgo es el 5\% anual para composición continua. El activo no paga dividendos y su valor acutal es 10.

\ppart ¿Cuál sería el impacto en este valor si el subyacente pagase dividendos en el noveno mes?

\solution

\end{problem}

\begin{problem}[2]
Consideremos el subyacente $S_0=50$ cuya dinámica para los próximos seis meses viene descrita por un modelo binomial de dos periodos. En cada periodo el activo puede subir o bajar un 20\%. El tipo libre de riesgo es el 12\% anual para composición continua. El activo no paga dividendos.

\ppart Calcular la probabilidad riesgo neutro de pasar al estdo alto

\ppart Calcular el valor de la put americana de precio de ejeercicio 52

\ppart Si el activo pagase dividendos, ¿el precio de la put sería mayor o menos?

\ppart Consideremos ahora un derivad sobre dicho subyacente que paga, al final de los seis meses, $M_2(ω)-m_2(ω)$ donde
\[M_2(ω)=\max(S_0(ω),S_1(ω),S_2(ω)), \;\;\; m_2(ω)=\min(S_0(ω),S_1(ω),S_2(ω))\]
Calcular la caretara de cobertura en cada nodo.

\solution

\end{problem}

\begin{problem}[3]
Una instituçión financiera entró en su día en un swap de tipos en el que cordó pagar un 6\% anual y recibir LIBOR a tres meses sobre un principal de 100 millones de euros con pagos intercambiados cada tres meses. Al swap le quedan 14 meses de vida. El tipo swap para intercambio por el LIBOR a tres meses en estos momentos es del 9\% anual par atodos los vencimientos. El tipo LIBOR a tres meses publicado hace dos meses era del 12\%. Todos los tipos son compuestos trimestralmente. ¿Cuál es el valor actual del swap?
\solution

\end{problem}

\begin{problem}[4]
Una institución financiera acaba de vender 1000 opciones europeas de compra a seis meses sobre
un activo S. El valor actual del mismo es de 10 euros y el precio de ejercicio es de 11 euros. El tipo
de interés libre de riesgo para el periodo es del 4\% anual para composición continua.
\ppart ¿cómo debería cubrirse esta institución financiera?
\ppart ¿Cuál sería la variación del precio del derivado si el precio del subyacente sube un 5\% a lo largo de la sesión, quedando invariantes los demás parámetros que inciden en su precio?
\ppart ¿Cuál sería el impacto en dicho precio si, con el subyacente en 10 euros, su volatilidad crece un 10\%?

\solution
\end{problem}

\section{Recuperación Septiembre 2011}
\begin{problem}[1]
Explicar el principio de valoración por ausencia de oportunidad de arbitraje
\solution
\end{problem}

\begin{problem}[2]
Calcula la curva cupón-cero (tiempo continuo) determinada por los siguientes bonos:

\begin{center}
\begin{tabular}{cccc}
\hline
\textbf{Principal} & \textbf{T} & \textbf{Cupón Anual} & \textbf{P} \\
\hline
100 & 0.25 & 0 & 97.50\\
100 & 0.50 & 0 & 94.90\\
100 & 1.00 & 0 & 90.00\\
100 & 1.50 & 8$^*$ & 96.00\\
100 & 2.00 & 12 & 101.6\\
\hline
\end{tabular}
\end{center}
(*) Este tipo es anual para composición semestral.
\solution

\end{problem}

\begin{problem}[3]
Consideramos un activo, de valor inicial $S_0 = 10$ cuya dinámica viene descrita por un árbol binomial
de cuatro periodos correspondientes a un horizonte temporal de seis meses. La volatilidad del
subyacente es del 25\% y el tipo libre de riesgo para cada periodo es del 0,6%

\ppart Calcular el coeficiente $a$ del árbol binomial
\ppart Calcular la probabilidad riesgo neutro asociada al modelo
\ppart Dar el valor de la opción lookback cuyo pago final es $T_{0.5}=S_4-\min\{S_i \mid 1 \leq i \leq 4\}$

\ppart ¿Qué pasaría si el activo pagase dividendos en el inicio del cuarto periodo?
\solution

\end{problem}

\begin{problem}[4]
Una compañía de aviación tendrá que comprar un millón de galones de keroseno dentro de tres
meses. La desviación típica de las variaciones de los precios del galón de keroseno para un periodo
de tres meses se ha estimado en 0,032. La compañía decide cubrirse con contratos de futuros sobre
gasóleo de calefacción. La desviación típica de los cambios en los precios de los futuros, para un
periodo de tres meses, se estima en 0,040 y la correlación lineal entre los cambios de ambos precios
es de 0,8.

\ppart Explicar el riesgo que supone elegir una tal cobertura
\ppart Calcular el ratio de cobertura
\ppart Cada contrato de futuros sobre keroseno es por 42000 galones, ¿cuál es el número de contratos que deberá usar la compañía? Exlicar si usará una posición larga o corta.
\solution

\end{problem}

\begin{problem}[5]
Los swaps de divisas son instrumentos muy usados en los mercados internacionales

\ppart A las empresas $A$ y $B$ les han ofrecido los siguientes tipos anuales en euros y en dólares

\begin{center}
\begin{tabular}{ccc}
\hline
& Euro & Dólar \\
Empresa A & 3.5 \% & 6.6\% \\
Empresa B & 5 \% & 7\% \\
\hline
\end{tabular}
\end{center}

A la empresa A le interesa un préstamos en dólares, mientras que a la empresa B le interesa que
sea en euros. Diseñar un swap igualmente atractivo para ambas empresas con un intermediario
financiero que se queda con 10 puntos básicos a cambio de asumir todo el riesgo de tipo de
cambio.(1 punto)

\ppart Una institución financiera A entrado en un swap de tipos, con pagos semestrales y a cinco,
años con la empresa B. Bajo las condiciones del swap, A recibe un 5\% anual a cambio del
LIBOR a 6 meses sobre un principal de 100 millones de euros. La empresa deja de pagar
al final del tercer año cuando le tocaba pagar la sexta liquidación. El LIBOR a 6 meses
publicado medio año antes era del 6,6\% anual. La media del tipo fijo ofertado y demandado
intercambiada actualmente por un LIBOR a 6 meses es del 7\% ¿Cuál será el coste para la
institución financiera? (Todos los tipos se componen semestralmente). (1 punto)

\solution

\end{problem}