\documentclass{apuntes}

\title{Topología I}
\author{Guillermo Julián Moreno \\ Cristina Kasner Tourné}
\date{14/15 C1}
% Paquetes adicionales
\usepackage{tikztools}
\usepackage{fastbuild}

\usetikzlibrary{arrows}
% --------------------

\precompileTikz

\begin{document}
\pagestyle{plain}
\maketitle

\tableofcontents
\newpage

\chapter{Conceptos básicos}

\section{Introducción}

En Topología buscamos extender conceptos importantes como continuidad o convergencia. Si partimos del concepto de continuidad en los reales, teníamos que

\begin{defn}[Continuidad] Dada $\appl{f}{(a,b)}{ℝ}$, se dice que es continua en $x_0 ∈ (a,b)$ si $∀ ε > 0 \; ∃δ>0 $ tal que $\abs{x-x_0} < δ \implies \abs{f(x) - f(x_0)} < ε$.
\end{defn}

¿Cómo podemos extender esto a conjuntos que no sean $ℝ$? Lo primero es que necesitamos una distancia. Y la propiedad central de la distancia debería ser \[ \abs{x+y} ≤ \abs{x} + \abs{y} \]. Esta propiedad es la desigualdad triangular, y es ciertamente natural. La extensión de la distancia la tendremos en los espacios métricos.

\begin{defn}[Espacio\IS métrico]
Un espacio métrico es un par $(X, d)$, con $X$ un conjunto y $d$ una aplicación $\appl{d}{X×X}[0, ∞)$ tal que 

\begin{enumerate}
\item $\dst(x,x) = 0\;∀x∈X$.
\item $\dst(x,y) ≥ 0\;∀x,y∈X$.
\item $\dst(x,y) = 0 \dimplies x=y$.
\item \concept{Desigualdad\IS triangular}: $\dst(x,z) \leq \dst(x,y) + d(y,z)\; ∀x,y,z∈X$.
\end{enumerate}
\end{defn}

Tenemos varios ejemplos de distancias:

Por ejemplo, en $ℝ^m$, tenemos $\dst (x,y) = \md{\vx-\vy} = \sqrt{(x_1-y_1)^2 + \dotsb + (x_m-y_m)^2}$.


Si consideramos el conjunto de funciones continuas $C([0,1]) \equiv \{ \appl{f}{[0,1]}{ℝ}, \text{f continua} \}$, $\md{f} ≝ \max_{x∈[0,1]} \abs{f(x)}$. Con esta noción, el conjunto de funciones continuas se comporta de forma similar a $ℝ^m$ con dimensión infinita, y podemos hacer cosas parecidas a las del espacio euclídeo.

Así, podemos definir $\dst(f,g) ≝ \md{f-g}$, y llegar a una definición de convergencia uniforme: $f_n \to f$ en esa distancia implica una convergencia uniforme en $[0,1]$.

Podemos definir una distancia algo artificial. Sea $X$ un conjunto cualquiera, definimos
\[ \dst(x,y) ≝ \begin{cases}
0 & \text{si}\; x = y \\
1 & \text{si}\; x ≠ y \\
\end{cases} \] 
que cumple las 3 primeras propiedades de distancia y su comprobación es trivial y para la comprobación de la desigualdad triangular, basta comprobarlo por casos.

\begin{defn}[Bola] Dado $(X,\dst)$ un espacio métrico, con $x∈X$ y $r∈(0,∞)$, definimos la bola $\bola$ centrada en $x$ de radio $r$ como 

\[ \bola(x,r) ≝ \{ y∈X \tq \dst(x,y) < r \} \]

En ocasiones querremos especificar la distancia ($\bola_{\dst}$) o el conjunto ($\bola_X$) con el subíndice.
\end{defn}

Las bolas tienen ciertas propiedades muy sencillas. Dados \sdst, $x∈X$, $r>0$, $y∈\bola(x,r)$ entonces $\bola(y,r-\dst(x,y)) ⊆ \bola(x,r)$.

\begin{wrapfigure}{r}{0.4\textwidth}
\inputtikz{I_BolaContenida}
\caption{La bola verde ($\bola(y, r-\dst(x,y)$) contenida dentro de $\bola(x, r)$.}
\label{figBolaContenida}
\end{wrapfigure}

Esto se puede demostrar con un dibujo (\ref{figBolaContenida}), pero tenemos que demostrarlo más formalmente: 

\begin{proof}
$∀z∈\bola(x, r-\dst(x,y))$ tenemos que $\dst(x,z) ≤ \dst(x,y) + \dst(y,z) < \dst(x,y) + r - \dst(x,y) = r$, y por lo tanto $z ∈ \bola(x,r)$.
\end{proof}

Por supuesto, el dibujo es una guía. Si tomásemos la distancia rara de antes que sólo tomaba valores 1 ó 0, la bola no sería una bola como en $ℝ$.

Vamos a definir ahora el cierre, aunque sólo como notación:

\begin{defn}[Cierre] Dado \sdst espacio métrico, $x∈X$, $r≥0$, definimos \[ \overline{\bola}(x,r) ≝ \{ y∈ X\tq \dst(x,y) ≤ r\} \] como la bola cerrada de centro $x$ y radio $r$.\end{defn}

\begin{defn}[Conjunto\IS abierto] Sea \sdst un espacio métrico. Entonces damos dos definiciones

\begin{enumerate}
\item $A⊆X$ es abierto en \sdst si $∀x∈A\; ∃δ=δ_x > 0$ tal que $\bola(x,δ_x) ⊆ A$.
\item La familia de abiertos es $\topl_d \equiv \{ A ⊆ X \tq A\, \text{abierto} \}$.
\end{enumerate}
\end{defn}

La familia de abiertos que acabamos de definir es una \concept[Topología]{topología}, y cumple las siguientes propiedades.

\begin{enumerate}
\item $\emptyset, X ∈ \topl_d$.
\item $A,B ∈ \topl_d \implies A \cap B ∈ \topl_d$.
\item $A_j ∈ \topl_d\; ∀j∈ J \implies \bigcup_{j∈J} A_j ∈ \topl_d$
\end{enumerate}

Demostremos las dos últimas propiedades:

\begin{proof} \paragraph{Propiedad 2} Sea $x∈A\cap B$. Entonces $x∈A$ y $x∈B$, luego existen $δ_x^A, δ_x^B$ tales que $\bola(x, δ_x^A) ⊆ A$ y $\bola(x, δ_x^B) ⊆ B$ respectivamente. Sea ahora $δ=\min(δ_x^A, δ_x^B)$. Entonces $\bola(x,δ) ⊆ A\cup B$.

\paragraph{Propiedad 3} La propiedad es equivalente a la pregunta de, si dado $x∈\bigcup_{j∈J}$, se cumple que $∃δ > 0 \tq \bola(x,δ)⊆\bigcup_{j∈J} A_j$.

Es obvio que $∃j_x ∈ J\tq x∈ A_{j_x}$, luego \[ ∃ δ > 0 \tq \bola(x,δ) ⊆ A_{j_x} ⊆ \bigcup_{j∈J} A_j \]
\end{proof}

Por otra parte, también podemos hacer una observación: por inducción, la intersección de una familia \textit{finita} de conjuntos también es un abierto.

\subsection{Topologías}

Una vez hecho esto, ya podemos pasar a definir qué es un espacio topológico y una topología.

\begin{defn}[Topología]\label{defTopología}
Sea $X$ un conjunto. Entonces una familia $\topl$ de subconjuntos de $X$ es una topología en $X$ si y sólo si cumple las tres propiedades que acabamos de ver:

\begin{enumerate}
\item $\emptyset, X ∈ \topl$.
\item $A,B ∈ \topl \implies A \cap B ∈ \topl$.
\item $A_j ∈ \topl\; ∀j∈ J \implies \bigcup_{j∈J} A_j ∈ \topl$
\end{enumerate}
\end{defn}

\begin{defn}[Espacio\IS topológico] Un espacio topológico es un par $(X, \topl)$ donde $\topl$ es una topología en $X$.
\end{defn}

Los elementos de \topl son los \textit{abiertos} de la topología. 

También podemos definir el conjunto cerrado:

\begin{defn}[Conjunto\IS cerrado]
Dado un espacio topológico \stopl, $F⊆X$ es cerrado si y sólo si $F^C \equiv X \setminus F$ es abierto.
\end{defn}

Podemos definir dos topologías "comunes", por así decirlo, las obvias para cualquier conjunto. Tenemos la \concept[Topología!trivial]{topología\IS trivial} (el mínimo) dada por \[ \topl_{triv.} = \{ \emptyset, X \} \], y la \concept[Topología!discreta]{topología\IS discreta}, que sería el máximo: \[ \topl_{disc.} = \parts{X} \].

Y volviendo a nuestro bonito mundo de los reales, tenemos las \concept[Topología\IS usual]{topologías usuales} en $ℝ^m$ o $\topl_{ℝ^m}$. Para $m=1$, diremos que \[ A ∈ \topl_ℝ ≝ ∀x∈A\; ∃a,b∈ℝ \tq x∈ (a,b) ⊆ A \] Esto es, que siempre podemos encontrar un intervalo contenido en $A$ que a su vez contenga a $x$. Equivalentemente, $A$ será una unión de intervalos abiertos.

Para dimensión $m>1$, definimos su topología usual como \[	ℝ^m ⊇ A ∈ \topl_{R^m} ≝ ∀x∈A\; ∃ \begin{matrix} a_1, \dotsc, a_m \\ b_1, \dotsc, b_m \end{matrix} ∈ ℝ \] tales que $ x∈ (a_1, b_1) × \dotsb × (a_m, b_m) ⊆ A$.


\subsubsection{Topologías metrizables}

\begin{defn}[Espacio\IS topológico metrizable] Dado \stopl un espacio topológico, se dice que es metrizable si existe una distancia $\dst$ en $X$ tal que $\topl = \topl_{\dst}$. 

$\dst$ no es necesariamente única.
\end{defn}

Por ejemplo, $\topl_ℝ$ es metrizable. Coincide $\topl= \topl_{\dst}$ con $\dst(x,y) = \abs{x- y}$.

Expandiendo un poco más sobre lo que significa que una topología coincide con otra, o lo que significa que una topología \textit{sea inducida} por una aplicación. 

\begin{defn}[Topología\IS inducida] Dado un espacio topológico \stopl, entonces la topología inducida por una función $f$ es \[ \topl_f = \{ \inv{f} (A) \tq A ∈ \topl \}\footnote{La prueba de porqué es topología se encuentra en los ejercicios (H1.E6)} \]
\end{defn} 

¿Qué significa entonces que una topología sea igual a otra? Si nos remitimos a la definición de topología (\ref{defTopología}), vemos que es un conjunto de subconjuntos de $X$. Luego dos topologías son iguales o equivalentes si y sólo si tienen los mismos elementos. Es decir, que si un conjunto es abierto en $X$ según una topología, también lo es según la otra y viceversa.

Volviendo al caso concreto, una topología inducida por la distancia es igual a otra topología si los abiertos según la distancia (esto es, las bolas) son también abiertos según la otra topología que estemos considerando.

Hagamos algunos ejemplos sobre topologías metrizables. ¿Son $\topl_{disc.}, \topl_{triv.}$ metrizables para $X$ un conjunto cualquiera?

En el caso de $\topl_{disc}$ sí lo es. Definimos la distancia como \[ \dst (x,y) = \begin{cases} 0 & x = y \\ 1 & x ≠ y \end{cases} \], luego $\bola(x, 1/2) = \{ x \}$, luego $\{ x \}$ es abierto en $\topl_{\dst}$. Entonces, si $A ⊆ X$, entonces $A= \bigcup_{x∈A} \{ x \}$ es abierto también.

La topología trivial es más interesante de estudiar. Si $\card{X}≥2$, entonces $\topl_{triv} ≠ \topl_{\dst}$ para cualquier distancia $\dst(x,y)$. ¿Por qué?

En la topología trivial sólo hay dos abiertos (vacío y total). Sin embargo, en la topología inducida por la distancia, los abiertos son las bolas. 

Si hay más de dos elementos en $X$, existen $x,y∈X$ distintos, y por lo tanto existe una distancia $r=\dst(x,y) > 0$. Con $δ=\frac{r}{2}$, las bolas $\bola(x,δ), \bola(y,δ)$ son distintas y disjuntas. Ninguna de ellas es el vacío y el total así que no son abiertos en $\topl_{triv}$, pero sí que son abiertos en $\topl_{\dst}$. Por lo tanto, tenemos que $\topl_{\dst} = \topl_{triv}$.

\subsubsection{Topologías generadas por una base}

Además de por la distancia, podemos considerar las \textbf{topologías generadas por una base}.

Recordemos cómo definíamos una topología en $\topl_ℝ$. Decíamos que $A∈\topl_ℝ$ si y sólo si $∀x∈A\; ∃(a,b)$ tales que $x∈(a,b) ⊆ A$. 

\begin{defn}[Base]\label{defBase}
Sea $X$ un conjunto y $\base$ una familia de subconjuntos de $X$ (e.d. $\base ⊆ \parts{X}$). Entonces $\base$ es una base si y sólo si 

\begin{enumerate}
\item $∀x∈X\;∃B∈\base \tq x∈B$. Dicho de otra forma, $\bigcup_{B∈\base} B = X$.
\item $∀B_1,B_2∈\base;\, ∀x∈B_1 ∩ B_2$, existe $B_3 ∈ \bola \tq x∈B_3 ⊆ B_1∩B_2$.
\end{enumerate}
\end{defn}


\begin{defn}[Topología\IS generada por una base] \label{TopologiaGeneradaBase} La topología $\topl_\base$ se define por \[ A ∈ \topl_\base \iff ∀x ∈ A\; \exists B∈\base \tq x∈B⊆A \]
\end{defn}

Tenemos que demostrar, eso sí, que eso que hemos definido ahí es realmente una topología.

\begin{prop} $\topl_\base$ es una topología en $X$.\end{prop}

\begin{proof} Tenemos que comprobar las tres propiedades de una topología (\ref{defTopología}). Sabemos que $\emptyset ∈ \topl_\base$. Además, tal y como hemos definido la topología generada, también sabemos que $X∈ \topl_\base$.

\paragraph{Propd. 2} Tenemos que demostrar que $A_1, A_2 ∈ \topl_\base \implies A_1∩A_2 ∈ \topl_\base$. Si $x∈ A_1∩A_2$, entonces $∃B_1, B_2 ∈ \base$ tales que $x∈B_1⊆A_1$ y $x∈B_2⊆A_2$ respectivamnete. 

Según la segunda propiedad de la base, existe un $B_3∈\base$ tal que $x∈B_3 ⊆ B_1∩B_2 ⊆A_1∩A_2$, luego $A_1∩A_2 ∈ \topl_\base$.

\paragraph{Propd. 3} Demostramos que $A_j ∈ \topl_\base\; ∀j∈J \implies \bigcup_{j∈J} A_j ∈ \topl_\base$. Si $x∈\bigcup_{j∈J} A_j \implies ∃i=i_x∈J\tq x∈A_i$. Luego como $A_i ∈ \topl_\base$ tenemos que $\exists B∈ \base$ tal que $x∈B ⊆ A_i ⊆  \bigcup A_j$.
\end{proof}

Nos fijamos que en la demostración de la tercera propiedad no hemos usado nada sobre cómo hemos definido la base. Es decir, que siempre que defininamos una topología $\topl$ como \[ A ∈ \topl \iff ∀x∈ A\;∃U ∈ \mathcal{F} \tq x∈ U ⊆ A \], donde $\mathcal{F}$ es una familia de subconjuntos de $X$, la propiedad tercera de la definición de topología \textbf{está garantizada}. Es para la primera y segunda propiedad para las que se necesita que $\mathcal{F}$ cumpla algún tipo de propiedad.

\begin{defn}[Topología\IS fina]
Dado un espacio $X$ y dos topologías $\topl_1, \topl_2$, si $\topl_1⊆\topl_2$ (todo abierto de $\topl_1$ es abierto de $\topl_2$) se dice que $\topl_2$ es \textbf{más fina} que $\topl_1$.
\end{defn}

\begin{prop} Sea $X$ un espacio topológico y $\base$ una base. Entonces 

\begin{enumerate}
\item $\base⊆\topl_\base$ (todos los elementos de $\base$ son abiertos en $\topl_\base$.
\item $A∈\topl_\base$ si y sólo si $A$ es unión de elementos de $\base$.
\end{enumerate}
\end{prop}

\begin{proof}
\paragraph{1)} Recordamos que \[ V ∈ \topl_\base ≝ ∀x∈V \; ∃B=B_x∈\base\tq  x∈B⊆V \]. Sea $M∈\base$, quiero demostrar que $M∈ \topl_\base$. He de comprobar que \[ ∀x ∈ M\; ∃B∈\base \tq x∈ B ⊆ M \], lo cual es obvio si tomamos $B=M$, ya que los elementos de la base son siempre abiertos.

\paragraph{2)} Partiendo de la afirmación de antes, sabemos que si $A ∈ \topl_\base$, entonces $∀x∈A\; ∃B_x∈\base$ tal que $x∈ B_x⊆A$. Como cada uno de esos conjuntos está en $A$, su unión también lo está. Y por otra parte, dado que consideramos todos los puntos $x$ de $A$, nos queda que \[ A = \bigcup_{x∈A}B_x \], demostrando así el primer lado de la implicación.

La implicación a la izquierda se resuelve por la primera parte de esta proposición: si $B_j∈B$, entonces $B_j∈\topl_\base$ y por la tercera propiedad de la topología (\ref{defTopología}), nos queda que \[ \bigcup_{j∈J} B_j ∈ \topl_\base \]

\end{proof}

Ahora que ya sabemos cómo generar una topología a partir de una base, podemos hacernos una pregunta. Consideramos una serie de conjuntos que queremos que sean abiertos en nuestro espacio. Obviamente, la topología discreta cumple lo que buscamos, pero, ¿hay una topología más pequeña? ¿Cuál es la topología \textit{"mínima"}?

\begin{prop} Sea $X$ un conjunto. \label{propTopologiaMinima}

\begin{enumerate}
\item Si $\topl_k$ es una topología en $X$, $∀k∈K$ entonces \[ \topl ≝\bigcap_{k∈K} \topl_k \].

\item Sea $D$ una familia de subconjuntos de $X$ ($D⊆\parts{X}$) y sea \[ \topl_D ≝ \bigcap_{D⊆\topl} \topl \] donde $\topl$ es una topología en $X$. 

Entonces $\topl_D$ es una topología en $X$, $D⊆\topl_D$ y $\topl_D$ es la topología menos fina que cumple $D⊆\topl_D$.
\end{enumerate}
\end{prop}

\begin{proof}
\paragraph{1)} La primera propiedad de la topología es trivial. Vamos con la segunda. Si $V_1V_2 ∈ \topl$, tenemos que $V_1, V_2 ∈ \topl_k\,∀k∈K$. Luego como $\topl_k$ es topología, $V_1∩ V_2 ∈\topl_k\, ∀k∈K$, y entonces $V_1∩V_2 ∈ \bigcap \topl_k = \topl$.

\paragraph{2)} Sabemos que $\topl_D$ es topología por lo que acabamos de demostrar. Ahora bien, ¿es la más pequeña? Es obvio, viendo que es la intersección de todas las topologías que contienen a $D$.\footnotemark
\end{proof}
\footnotetext{Relacionado con el ejercicio 9-c.}

Tenemos que tener cuidado cuando $D$ es una base: hay que asegurarse de que la topología coincida en ese caso.\footnote{Y nos preocupamos nosotros de eso.} 

\subsubsection{Topología del orden}

Hasta ahora hemos visto cómo generar topologías a partir de la distancia, y también tratando de extrapolar el concepto de los intervalos de $ℝ$ con las bases. Ahora vamos a ver cómo hacerlos a través de otra visión de los intervalos como elementos de orden. Recordemos brevemente qué es un orden total: a grandes rasgos es uno donde podemos comparar todos los elementos.

\begin{defn}[Orden\IS total] Dado un conjunto $X$, un orden total en $X$ es una relación $x < y$ tal que 

\begin{enumerate}
\item $x<y, y < z\implies x < z$.
\item $∀x∈X$, $x < x$ es falso.
\item $∀x,y∈X$ con $x≠y$ entonces se cumple una y sólo una de $x< y$ ó $y<x$. 
\end{enumerate}
\end{defn}

Dado un conjunto $X$ y un orden total $<$ se puede construir una topología $\topl_<$ de la misma forma que en $ℝ$: intervalos $(a,b)$. Empecemos con ejemplos.

\paragraph{Orden lexicográfico en $ℝ^2$} Este ejemplo es una topología muy visual (ver la figura \ref{figOrdenLex}), importante y rara. Empezamos definiendo qué es ese orden

\begin{defn}[Orden\IS lexicográfico] Denotamos como $<_{Lex}$ al orden que, dado $x=(x_1,x_2), y=(y_1, y_2)$ ambos en $ℝ^2$, se dice que $x<_{Lex} y$ si $x_1 < y_1$ o bien, si $x_1 = y_1$, entonces $x_2 < y_2$.
\end{defn}

\begin{wrapfigure}{r}{0.4\textwidth}
\inputtikz{I_OrdenLexicografico}
\caption{Ilustración del orden lexicográfico en $ℝ^2$. Cualquier punto en $r_2$ es mayor que todos los de $r_1$. En la misma vertical, tenemos que $a<_{Lex}b$.}
\label{figOrdenLex}
\end{wrapfigure}

A partir de esto podemos definir el intervalo lexicográfico de la forma obvia:

\begin{defn}[Intervalo\IS lexicográfico] Si $a,b∈ℝ^2$ con $a<_{Lex}b$ entonces
\[ (a,b)_{Lex} ≝ \{  x ∈ℝ^2\tq a <_{Lex} x <_{Lex} b \} \]
\end{defn}

\begin{prop} \[ \base_{Lex}=\{ (a,b)_{Lex} \tq a,b∈ℝ^2, a<_{Lex}b \} \] es una base para una topología en $ℝ^2$.\end{prop}

\begin{proof}
Como ejercicio, pero queda claro que si $B_1,B_2∈\base$ entonces $B_1∩B_2∈\base$, lo que es todavía mejor que la definición de base.
\end{proof}

\begin{defn}[Topología\IS lexicográfica en $ℝ^2$] En $ℝ^2$, definimos la topología lexicográfica como 

\[ \topl_{Lex} ≝ \topl_{\base_{Lex}} \]
\end{defn}

Podemos ver un ejemplo, considerando en $[0,1]^2$ el intervalo acotado lexicográficamente por $\mathbbold{0} = (0,0)$ y $\mathbbm{1} = (1,1)$, luego
\[ [0,1] × [0,1] = [\mathbbold{0}, \mathbbm{1}]_{Lex} \] 

En esta topología, el abierto más sencillo que contiene a un punto "en el medio" (por ejemplo, el $(0.5, 0.5)$) sería un intervalo "vertical". Sin embargo, para un punto en el borde superior o inferior, el abierto más sencillo sería un rectángulo que se expande hacia la derecha (ver figura \ref{figIntervalosLex}).

\begin{figure}[hbtp]
\centering
\begin{subfigure}[b]{0.4\textwidth}
\inputtikz{I_OrdenLex_AbiertoVertical}
\caption{En un punto dentro del cuadrado, el abierto más sencillo es un intervalo vertical}
\end{subfigure}
~
\begin{subfigure}[b]{0.4\textwidth}
\inputtikz{I_OrdenLex_AbiertoTop}
\caption{En un punto en un borde del cuadrado, el abierto más sencillo es un rectángulo, el intervalo $(a,b)_{Lex}$.}
\end{subfigure}

\caption{Intervalos más sencillos en $[\mathbbold{0}, \mathbbm{1}]_{Lex}$.}
\label{figIntervalosLex}
\end{figure}

Más ejemplos: ¿una bola $A$ en el sentido habitual (ver figura \ref{figBolaLex}) de $ℝ^2$ es abierto en esta topología? Efectivamente: podemos expresarlo como unión de abiertos de la topología, las líneas verticales. 

\begin{figure}[hbtp]
\centering
\inputtikz{I_OrdenLex_Bola}
\caption{La bola $A$ es abierto en la topología lexicográfica si la expresamos como unión de intervalos verticales}
\label{figBolaLex}
\end{figure}

Si lo expresamos de forma simbólica también llegamos a lo mismo. Tomamos

\[ A = \{ (x,y) \tq \left(x-\frac{1}{2}\right)^2 + \left(y-\frac{1}{2}\right)^2 < \frac{1}{100} \]

Así, tendríamos que 

\[ A = \bigcup (a_x, b_x)_{Lex} \]

con 

\begin{gather*}
a_x = \left(x, \frac{1}{2} - \sqrt{\frac{1}{100} - \left(x-\frac{1}{2}\right)^2}\right) \\
b_x = \left(x, \frac{1}{2} + \sqrt{\frac{1}{100} - \left(x-\frac{1}{2}\right)^2}\right)
\end{gather*}

\subsection{Convergencia}

Ahora sigamos con más definiciones.

\begin{defn}[Entorno\IS abierto] Dado \stopl un espacio topológico y $x∈X$, un entorno abierto de $x$ es un abierto $U∈\topl$ tal que $x∈U$.
\end{defn}

\begin{defn}[Convergencia\IS de sucesiones] Sea \stopl un espacio topológico y $\{x_n\}_{n∈ℕ}$ una sucesión en $X$, y $x∈X$. Se dice que $x_n$ converge a $x$ si todo entorno de $x$ contiene todos los términos de la sucesión a partir de un índice determinado.

Dicho simbólicamente

\[ ∀U∈\topl,\; x∈U\; ∃n_U∈ℕ \tq x_n ∈ U \,∀n≥n_U \]
\end{defn}

\paragraph{Ejercicio} En un espacio métrico \sdst con la topología $\topl_{\dst}$ inducida por la distancia, demuestra que \[ x_n \convs x \iff \dst(x,x_n)\convs 0 \iff ∀ε> 0\, ∃n_ε\tq \dst(x,x_n) < ε \]

\paragraph{Ejemplo 1} Veamos ejemplos de convergencia en topologías raras. Tomemos $ℝ$ con $\topl_{[, )}$, y la sucesión $x_n= \frac{-1}{n}$ para $n≥1$. 

Esta sucesión converge en el sentido usual (euclídeo) a cero, pero no en esta topología. Y es que existe un intervalo que es entorno de $0$ (por ejemplo, $[0, 1)$ que no contiene a ningún punto de la sucesión.

\paragraph{Ejemplo 2} Tomemos la sucesión $x_n=\left(\frac{1}{n}, 1\right)$ para $n≥1$ en el espacio topológico $(ℝ^2, \topl_{Lex})$. Converge en la topología usual a $(0,1)$, pero no en la lexicográfica: un entorno vertical del $(0,1)$ no contiene a puntos de la sucesión.


Ahora vamos a ver algunos conceptos en relación a los cerrados, que recordemos eran los complementarios de los abiertos. 

\begin{prop} $ $
\begin{enumerate}
\item $\emptyset, X$ son cerrados.
\item La unión finita de cerrados es cerrado.
\item La intersección de una familia de cerrados es cerrado.
\end{enumerate}
\end{prop}

\paragraph{Ejemplos}  Si tomamos $R_a ≝ \{ (a,y) \tq y ∈ ℝ \}$, es abierto y cerrado en $(ℝ^2, \topl_{Lex})$. 

Es cerrado porque $R_a^c$ es abierto: para todo punto puedo coger un entorno abierto contenido en el conjunto. También podemos verlo como que $R_a^c = \bigcup_{b∈ℝ\setminus \{a\}} R_b$, que es unión de abiertos.

Otro conjunto interesante es $[a,b)$ en $\topl_{[,)}$, que también es abierto (es un elemento de la base) y cerrado (su complementario es $(-∞, a) ∪ [b, ∞)$, ambos abiertos (podemos expresarlos como unión de conjuntos de la base).

Curiosamente, ese conjunto en $(ℝ, \topl_ℝ)$ no es ni abierto ni cerrado: no hay ningún abierto que contenga a $a$ y que esté contenido en $[a,b)$ por lo que no es abierto; y tampoco es cerrado porque $[a,b)^c=(-∞,a) ∪ [b,∞)$ que no es abierto.

\seprule

Vamos a hacer ahora ciertas observaciones sobre convergencia en topologías definidas por una base, para darnos cuenta de los interesantes que pueden resultar al permitirnos probar cosas mirando sólo los elementos de la base.

\begin{prop} Dado un conjunto $X$ y $\base$ una base para una topología en $X$ $\topl_\base$, entonces 

\[ x_n\convs x \iff ∀B∈\base \tq x ∈ B,\; ∃n_B \tq  x_n∈ B ∀ n ≥ n_B \].

Es decir, basta comprobar la definición para entornos de $x$ que son elementos de la base.\end{prop}

\begin{proof}
La implicación a la derecha es trivial: si se cumple para todos los abiertos, se cumple para algunos en particular.

Ahora tenemos que demostrar la implicación a la izquierda. Si $U∈ \topl_\base$ y $x∈U$, entonces por la definición de $\topl_\base$ $∃B∈ \base$ tal que $x∈B⊆U$. Por hipótesis, $∃n_B$ tal que $x_n∈B\; ∀n≥ n_B$, y como $B⊆U$ entonces $x_N∈U \; ∀n≥ n_B$.
\end{proof}

\subsection{Interior, adherencia y frontera de conjuntos}

\subsubsection{Interior}

Sea \stopl un espacio topológico y $W⊆X$ un conjunto cualquiera. Entonces, vamos a definir varios conceptos

\begin{defn}[Interior] Decimos que $x ∈ \mathop{Int}(W)$ si existe un entorno $U$ de $x$ tal que $U⊆W$. Es decir, existe un $U$ abierto tal que $x∈U ⊆ W$.

El interior de un conjunto $W$ se denota como $\intr{W}$.
\end{defn}

Por ejemplo, $\intr{ℚ}$ es vacío tanto en la topología usual como en $\topl_{[,)}$.

El intervalo $[a,b]$ en la topología usual tiene como interior el abierto $(a,b)$. Pero, ¿y en la topología $\topl_{[,)}$? En este caso es: $\intr{[a,b]} = [a,b)$.

Razonando, si $a≤x<b$, entonces $x∈[a,b) ⊆ [a,b]$, luego $x$ está en el interior. Si $x = b$, entonces no existe un intervalo abierto $U$ con $b∈U⊆[a,b]$, porque si $b∈U$ con $U$ abierto entonces existiría un $[α,β)$ con $b∈[α,β) ⊆ U$. En ese caso, el punto medio $\frac{b+β}{2} ∈ [α,β)$, luego entonces también pertenecería a $U$. Sin embargo, es claro que $\frac{b+β}{2} > b$, así que no puede estar en $U$.

\begin{figure}[hbtp]
\centering
\inputtikz{I_ConjuntoWLex}
\caption{Conjunto $W$ en la topología lexicográfica.}
\label{figConjuntoWLex}
\end{figure}

Otro ejemplo: tomemos $([0,1]^2, \topl_{Lex})$. ¿Cuál es el interior del conjunto $W$ que aparece en la figura \ref{figConjuntoWLex}?

Está claro que los puntos de dentro $p_i$ son del interior, ya que siempre podemos encontrar un abierto. También están dentros los puntos $p_b$ de los bordes inferior y superior (con $0≤x<b$), ya que podemos encontrar una banda (como la banda azul) contenida en $W$. Los puntos del lateral izquierdo $p_l$ también están en el interior.

Sin embargo, los puntos $p_d$ en el borde de la diagonal o en el borde inferior con $b ≥ x$ no están en el interior: cualquier abierto que cojamos será una banda como la roja, que se sale de $W$. Tampoco están los puntos del borde lateral derecho.

En definitiva, en la figura \ref{figConjuntoWLex} el interior sería el interior de naranja con los bordes naranjas, excluyendo los marcados en rojo.

\begin{prop} En el caso de una topología generada por una base $\topl_\base$, si $W⊆X$, decimos que $x∈\intr{W}$ si y sólo si existe un elemento $B$ de la base tal que $x∈B⊆W$.
\end{prop}

\begin{proof} Recordamos la definición: \[ x∈ \intr{W} \iff ∃A∈\topl \tq x∈A⊆W \]

La implicación a la izquierda la demostramos diciendo que si $B∈\base$, entonces $B∈\topl_\base$ y $A=B$.

Hacia la derecha, si $x∈\intr{W}$ entonces sabemos que $∃A∈\toplb$ tal que $x∈A⊆W$. Como $A∈\toplb$, entonces $∃B∈\base \tq x∈B⊆A$, y nos queda $x∈B⊆A⊆W$.
\end{proof}

\begin{prop} Sea $\stopl$ un espacio topológico y $W⊆X$. Entonces

\begin{enumerate}
\item $\intr{W} ⊆ W$.
\item $\intr{W} = \bigcup A$ tales que $A∈\topl, A⊆W$ (los abiertos contenidos en $W$.
\item $W$ es abierto si y sólo si $\intr{W} = W$.
\end{enumerate}
\label{propInterior}
\end{prop}

\begin{proof} La primera parte es trivial.

La segunda proposición, si $U$ es abierto y $U⊆W$, entonces $A⊆\intr{U}$, de forma bastante obvia. Por otra parte, si $x∈\intr{W}$, entonces $∃A∈\topl\tq x∈A⊆W$. Entonces $x∈A⊆W⊆\bigcup U$ donde $U$ son abiertos contenidos en $W$.

Y por úlitmo la tercera parte. Como $\intr{W}$ es unión de abiertos (lo acabamos de demostrar) también es abierto. La demostración a la izquierda resulta trivial de esta forma: si $W=\intr{W}$ y $\intr{W}$ es abierto, $W$ es abierto.

Ahora queremos demostrar que si $W$ es abierto, entonces $\intr{W}$ es abierto también. Sabemos que siempre $\intr{W} ⊆ W$, así que sólo nos falta la otra inclusión $W⊆\intr{W}$. Simplemente tenemos que recordar la demostración anterior: si $\intr{W}$ es la unión de todos los abiertos $U$ contenidos en $W$, entonces $W$ es uno de esos abiertos y por lo tanto $W⊆\bigcup U = \intr{W}$.
\end{proof}

\begin{remark}
$\intr{W}$ es el abierto más grande contenido en $W$.
\end{remark}

\subsubsection{Adherencia}

\begin{defn}[Adherencia] Sea \stopl un espacio topológico, $W⊆X$. La adherencia $\adh{W}$ de $W$ se define como todos los entornos de $X$ que "cortan" a $W$. Más formalmente

\[ x ∈ \adh{W} \iffdef A∩W ≠ \emptyset \; ∀ A ∈ \topl \tq x∈A \]
\label{defAdherencia}
\end{defn}

Como ocurría con otras propiedades de conjuntos, si la topología está generada por una base nos vale con comprobar la definición para los elementos de la base.

\begin{prop} En una topología generada por una base \toplb se tiene que \[ x ∈ \adh{W} \iff B∩W≠\emptyset \; ∀B∈\base \tq x∈B \]
\end{prop}

\paragraph{Ejemplos} $ℚ⊆\topl_ℝ$ y en $\topl_{[,)}$. En ambos casos $\adh{ℚ} = ℝ$: cualquier $x∈(a,b)⊆ℝ$ que cojamos se tiene que $(a,b) ∩ ℚ$ no es vacío (y cualquier intervalo $(a,b)$ que escojamos no es vacío porque $x$ está en él). La razón es análoga para $\topl_{[,)}$.

Este ejemplo nos sirve como introducción a la definición de conjunto "denso".

\begin{defn}[Conjunto\IS denso] Dado \stopl espacio topológico y $W⊆X$, se dice que $W$ es denso en $X$ para \topl si $\adh{W} = X$.
\end{defn}

Sigamos con ejemplos. Tomamos $X=[0,1]^2$ con la topología lexicográfica, y $W=\{(x,y) ∈ [0,1]^2 \tq x+y < \frac{3}{2} \}$.

\begin{figure}[hbtp]
\inputtikz{I_AdhConjuntoWLex}
\caption{Adherencia (azul) del conjunto $W$ (naranja) en la topología lexicográfica}
\label{figAdhWLex}
\end{figure}

Está claro que $W⊆\adh{W}$. La diagonal (puntos $p_d$ en la figura \ref{figAdhWLex}) también está en la adherencia: cualquier abierto que cojamos alrededor de ese punto interseca con $W$. Además, cualquier punto $p_b$ del borde superior (salvo la esquina $(1,1)$) está en la adherencia, ya que los abiertos serán bandas como la verde, que intersecan igualmente con $W$.

\begin{prop} Dado \stopl un espacio topológico y $W⊆X$, se tiene que 
\begin{enumerate}
\item $W⊆\adh{W}$.
\item $\adh{W} = \left(\mop{Int}(W^c)\right)^c$. Como consecuencia $\adh{W}$ es cerrado.
\item $\adh{W} = \bigcap_{F⊇W} F$ donde $F$ son todos los cerrados que contienen a $W$.
\item $W$ es cerrado si y sólo si $W = \adh{W}$.\\
\end{enumerate}\end{prop}

Para demostrar estas propiedades, muy similares a las del interior (\ref{propInterior}), usaremos la dualidad abierto-cerrado, unión-intersección e interior-adherencia, que nos será muy útil en el futuro.

\begin{proof}
\begin{enumerate}
\item Claro a partir de la definición (\ref{defAdherencia}).
\item Lo que estamos diciendo es que $\adh{W}^c = \mop{Int}(W^c)$. Empezamos diciendo que si $x∉\adh{W}$, entonces $∃A∈\topl \tq x∈A$ y $A∩W=\emptyset$. Es decir, que $A⊆W^c$, luego $x∈\mop{Int}(W^c)$.
\item Lo deja como ejercicio, usando la segunda definición y las propiedades del interior. 
\item Ídem.
\end{enumerate}
\end{proof}

\subsubsection{Frontera de un conjunto}

\begin{defn}[Frontera] Dado \stopl un espacio topológico y $W⊆X$, se dice que 

\[ x∈\mop{Fr}(W) \iffdef A∩W ≠ \emptyset \y A∩W^c ≠ \emptyset\; ∀A ∈ \topl \tq x∈A \]

Es decir, la frontera son todos los puntos cuyos entornos cortan a $W$ y $W^c$. 
\label{defFrontera}
\end{defn}

Como viene siendo habitual, si tenemos una topología generada por una base, bastan comprobar para los elementos de la base.

\begin{prop} Sea \stopl un espacio topológico y $W⊆X$, entonces

\begin{enumerate}
\item $\mop{Fr}(W) = \adh{W} ∩ \adh{W^c}$, y por lo tanto es cerrado.
\item $\mop{Fr}(W) = \adh{W} \setminus \intr{W}$.
\end{enumerate}
\end{prop}

\begin{proof}
\begin{enumerate}
\item Claro a partir de la definición (\ref{defFrontera}).
\item Antes hemos visto que $\adh{W^c} = \left(\mop{Int}((W^c)^c)\right)^c$, o dicho de otra forma $\adh{W^c} = (\intr{W})^c = X \setminus \intr{W}$. Entonces $\mop{Fr}(W) = \adh{W} ∩ \adh{W^c} = \adh{W} ∩ (X\setminus \intr{W}) = \adh{W} \setminus \intr{W}$.
\end{enumerate}
\end{proof}

\begin{remark} La definición nos permite escribir la adherencia como unión de conjuntos disjuntos: \[ \adh{W} = \intr{W} ∪ \mop{Fr}(W) \]
\end{remark}

\paragraph{Ejemplos} Empezamos con el simple: $ℚ$ en $ℝ$ con la topología usual y con $\topl_{[,)}$. En ambos casos tenemos que \[ \mop{Fr}(ℚ) = \adh{ℚ} \setminus \intr{ℚ} = ℝ \], ya que en cualquier intervalo de longitud mayor que cero hay racionales e irracionales.

Y volvemos a la lexicográfica: sea $W=\{ (x,y) ∈ [0,1]^2 \tq x+y < 3/2 \}$. Sabemos que $\mop{Fr}(W) = \adh{W} \setminus \intr{W}$, y ya habíamos calculado la adherencia (azul en la figura \ref{figFrontWLex}) como

\[ \adh{W} = \left\{ (x,y) ∈ [0,1]^2 \tq x + y ≤ \frac{3}{2} \right\} υ \left\{ (x,1) \tq \frac{1}{2} ≤ x < 1 \right\} \]

Por otra parte (ver figura \ref{figConjuntoWLex}), el interior era

\[ \intr{W} = W \setminus \left\{ (x,0) \tq \frac{1}{2} < x \right\} \]

De esta forma, nos queda que \[ \mop{Fr}(W) = \{ (x,y) ∈ [0,1]^2 \tq \} \]

\begin{figure}[hbtp]
\inputtikz{I_FronteraConjuntoWLex}
\caption{Frontera del conjunto $W$, en verde, con los dibujos de la adherencia (azul, figura \ref{figAdhWLex}) y el interior (rojo, figura \ref{figConjuntoWLex}).}
\label{figFrontWLex}
\end{figure}

Más ejemplos: sea \sdst un espacio métrico y $\topl_{\dst}$ la topología inducida por la distancia. Consideramos $x∈X$ y $r>0$, y $W=\bola(x,r)$. ¿Cuáles son el interior, adherencia y frontera del conjunto?

Está claro que $\intr{W} = \bola(x,r)$ por ser las bolas abiertas. La adherencia es algo más complicada. Es obvio que \[ \adh{W} ⊆ \adh{\bola}(x,r) ≝ \{ y∈X \tq \dst(x,y) ≤ r \} \] (que es la bola cerrada, no la adherencia de la bola), pero pueden ser distintos. En $ℝ^n$ con la distancia euclídea sí son iguales. Pero consideremos la distancia \[ \dst(x,y) = \begin{cases} 0 & x = y \\ 1 & x ≠ y \end{cases} \] que nos genera la topología discreta, en la que todos los conjuntos son a la vez abiertos y cerrados. Si $r=1$, entonces \[ W = \bola(x,1) = \{ x \} \implies \adh{W} = W = \{ x \} \] 

¿Y cuál es la bola cerrada? $\adh{\bola}(x,1) = X$, obviamente no es lo mismo aunque desde luego $\adh{W} ⊆ \adh{\bola}(x,1)$.

Además, en esta topología, tendríamos que $\mop{Fr}(\{ x \} ) = \emptyset$.


\begin{prop} Una proposición trivial: dado \stopl espacio topológico, $A⊆B⊆X$. Entonces se tiene que $\int{A}⊆\int{B}$ y que $\adh{A}⊆\adh{B}$.
\end{prop}

\subsection{Puntos aislados y puntos de acumulación}

\begin{defn}[Punto\IS aislado] Dado \stopl un espacio topológico y $W⊆X$, se dice que $x∈W$ es un punto aislado de $W$ si y sólo si existe un abierto $A$ con $x∈A$ y tal que $A∩W=\{x\}$.
\end{defn}

El punto que no es aislado es un punto de acumulación:

\begin{defn}[Punto\IS de acumulación] Dado \stopl un espacio topológico y $W⊆X$, se dice que $x∈X$ (no es necesario que esté en $W$) es un punto de acumulación de $W$ si y sólo si $∀A∈\topl$ con $x∈A$ se tiene que $A∩(W\setminus\{x\}) ≠ \emptyset$.

El conjunto de todos los puntos de acumulación de $W$ se denota como $W'$. En algún caso (no durante estas clases) se le llamará el \concept[Conjunto\IS derivado]{conjunto derivado} de $W$.
\end{defn}

\paragraph{Ejemplos} Empezamos por lo simple, como siempre. Consideramos a $(a,b)$ en $ℝ$ con la topología usual. No hay ningún punto aislado, y los puntos de acumulación son $[a,b]$.

Si pensamos en $W=(a,b]$ en $ℝ$ con la topología de intervalos $\topl_{[,)}$, vemos que $b$ es un punto aislado. Cualquier abierto  $[b, b + ε)$ sólo interseca en $b$ con $W$. Por otra parte, los puntos de acumulación son $[a,b)$. 

Como siempre, vamos ahora a la topología lexicográfica. Consideramos $W=\{ (x, 0.5) \tq 0≤x ≤ 1\}$. Todos los puntos son aislados. Los puntos de acumulación desde luego no serán puntos del intervalo: son los bordes superiores e inferiores, cuyos entornos son bandas (salvo los que coincidan con bordes laterales). Luego \[ W' = \{ (x,1) \tq x < 1 \} ∪ \{ (x,0) \tq x > 0 \} \] 

Y por último, consideremos $W=\left\{ \frac{1}{n} \tq n ≥ 1, n∈ℕ\right\}$ con la topología usual en $ℝ$. Todos los puntos de $W$ son aislados, y el $0$ es punto de acumulación.

\begin{prop} Sea \stopl un espacio topológico y $W⊆X$. Entonces

\begin{enumerate}
\item $x∈W' \iff x∈ \adh{W\setminus \{x\}}$. Como consecuencia, $W'⊆\adh{W}$.
\item $\adh{W} = W ∪ W'$.
\item $W$ es cerrado si y sólo si $W'⊆W$.
\end{enumerate}
\end{prop}

\subsection{Espacios métricos}

En los espacios métricos con distancias razonables, la adherencia, interior y demás puntos se vuelven más interesantes y fáciles de estudiar.

\begin{prop} Sea \sdst un espacio métrico con la topología inducida por la distancia $\topl_{\dst}$, y sea $W⊆X$. 

\begin{enumerate}
\item $x∈\intr{W} \iff ∃δ> 0 \tq \bola(x,δ) ⊆ W$.
\item $x∈\adh{W} \iff ∀ε>0\; \bola(x,ε) ∩ ≠ \emptyset$.
\item $x∈\mop{Fr}(W) \iff ∀ε>0\; \bola(x,ε) ∩W ≠ \emptyset \y \bola(x, ε) ∩W^c ≠ \emptyset$.
\item $x$ es punto aislado de $W$ si y sólo si $∃δ>0 \tq \bola(x,δ) ∩ W = \{ x\}$.
\item $x∈W' \iff ∀ε>0\; \bolac(x,ε) ∩ W ≠ \emptyset$, donde $\bolac(x,ε) = \bola(x,ε) \setminus \{x\}$.
\item $x∈ \adh{W}$ si y sólo si existe una sucesión $\{x_n\}⊆W\tq x_n\convs x$.
\item $x∈W'$ si y sólo si existe una sucesión $\{x_n\}⊆W$ con $x_n\convs x$ y $x_n≠x\; ∀n∈ℕ$ (excluimos sucesiones constantes a partir de un término.
\end{enumerate}

En las proposiciones 2,3 y 5 basta con comprobarlo para una sucesión $ε_n \to 0$, por ejemplo $ε_n=\frac{1}{n}$.
\end{prop}

\begin{proof}
\begin{enumerate}
\item Queremos demostrar que $x∈\intr{W} \iff ∃δ> 0 \tq \bola(x,δ) ⊆ W$, y para ello recordamos la definición: $x∈\intr{W} \iffdef ∃A∈\topl \tq x∈A ⊆ W$. La implicación hacia la izquierda es sencilla si tomamos simplemente el abierto $A = \bola(x,δ)$.

En la otra dirección, la hipótesis es que $∃A∈\topl \tq x∈A⊆W$. Al ser una topología inducida por la distancia, podemos encontrar una bola $\bola(x,δ)⊆A⊆W$.

\item Son bastante sencillas.
\item 
\item 
\item
\item $x∈\adh{W} \iff ∃x_n∈W \tq x_n\convs x$. Hacia la derecha, podemos ir cogiendo bolas $\bola(x,1/n)∩W ≠ \emptyset$, cada vez más pequeñas, y escoger puntos $x_n∈\bola(x,1/n)$, de tal forma que $\dst(x_n, x) \convs 0$, o de otra forma $x_n \convs x$.

Para el otro lado, es cierto y además para cualquier espacio topológico. Tenemos que $x_n\convs x$ para $x_N∈W$, queremos probar que $x∈\adh{W}$. Por la propia definición de convergencia, podemos encontrar un $n_A$ tal que $x_n∈A$ para todo $n≥n_A$, luego $A∩W ≠ \emptyset$. 

\item Y ya para acabar, hay que demostrar que $x∈W' \iff ∃x_n∈W, x_n≠x \; ∀n∈ℕ$ y $x_n\convs x$. Hacia la izquierda, sabemos que $\bolac(x,\frac{1}{n})∩W ≠ \emptyset$, luego $∃x_n∈W∩\bolac(x, 1/n)$, es decir, que $x_n∈W$ y además $x_n≠x$, luego ya hemos encontrado la sucesión que buscábamos.

La implicación hacia la izquierda es igual en cualquier espacio topológico. $∀A∈\topl$ con $x∈A$, tenemos que $∃n_A \tq x_n∈A \; ∀n≥a$ por definición de convergencia. Luego es claro que $(A\setminus \{ x\}∩W ≠ \emptyset$ y entonces $x∈W'$.
\end{enumerate}
\end{proof}

\begin{remark} Si \sdst es un espacio métrico y $x_n\to x$ y  $x≠z$, entonces $x_n$ no converge a $z$. Es decir, el límite es único.\end{remark}

\begin{proof}
Si $x≠z$, entonces $\dst(x,z)>0$. Sea $δ$ con $0<δ<\frac{\dst(x,z)}{2}$. Por definición de convergencia, existe $n_δ \tq x_n ∈ \bola(x,δ) \; ∀ n≥n_δ$. Por otra parte, $\bola(x,δ) ∩ \bola(x,δ) = \emptyset$, y entonces $x_n∈\bola(z,δ) \,∀n≥n_δ$, luego es imposible que $x_n$ converja a $z$.
\end{proof}

Atención porque esto sólo pasa en espacios métricos: si cogemos un espacio raro el límite puede dejar de ser único. 

Lo que importa realmente del argumento no es que sea métrico, si no que podamos coger dos bolas disjuntas para puntos distintos. Formalicémoslo:


\begin{defn}[Espacio\IS Hausdorff]\label{defHausdorff}
Un espacio topológico \stopl es Hausdorff si $∀x,y \in X$ con $x≠y$ existen $V_x$ entorno de $x$ y $V_y$ entorno de y tal que $V_x∩V_y≠\emptyset$
\end{defn}

\begin{prop} Sea \stopl espacio topológico de Hausdorff. Entonces
\begin{enumerate}
\item Si $x_n\to x$ entonces $x_n$ no converge a $y$ $∀y≠x$ (es decir, el límite de una sucesión, si existe, es único).
\item $\{x\}$ es cerrado $∀x \in X$.
\end{enumerate}
\end{prop}

\begin{proof}
\begin{enumerate}
\item Supongo $x_n\to x$ y $x≠y$

	\begin{enumerate}
	\item Por definición, $\exists V_x,V_y$ entornos de $x$ e $y$ con $V_x∩V_y≠\emptyset$.
	\item Si $x_n\to x$ y $x\in V_x$, por definición de convergencia $\exists n_0$ tal que $x_n\in V_x ∀n ≥n_0$. Luego $x_n\notin V_y ∀n ≥ n_0$ y por lo tanto $ x_n$ no converge a $y$.
	\end{enumerate}

\item Queremos demostrar que  $\adh{\{x\}} = \{x\}$, y lo haremos por doble contenido. El contenido a la izquierda es trivial ($\{x\} ⊆ \adh{\{x\}}$) así que sólo tenemos que hacerlo a la derecha.

Si $y≠x$, entonces existen $V_y$ entorno de $y$ y $V_x$ entorno de $x$ tales que $V_x∩V_y≠\emptyset$, por lo tanto $V_y∩\{x\}=\emptyset$ , de tal forma que $y\notin \adh{\{x\}}$ y $\adh{\{x\}} ⊆ \{x\}$.
\end{enumerate}
\end{proof}

\begin{remark}
La propiedad 2 ($\{x\}$ cerrado $∀x \in X$) es equivalente a decir que $∀ x,y \in X$ con $x≠y$ existen entornos respectivos $V_x$ y $V_y$ tal que $x\notin V_y$ , $y\notin V_x$.
Espacios topológicos con esa propiedad se llaman $T_1$ (Hausdoff es $T_2$)
\end{remark}

Por ejemplo, $(ℝ^m, \topl_{usual})$, $(ℝ^2, \topl_{Lex.})$, o \sdst son espacios Hausdorff.

\subsection{Topología de subespacios}

\begin{defn}[Topología\IS de subespacio]
Dado \stopl espacio topológico y  $S⊆X$, se define la topología de subespacio en $S$ por:
$V\in \topl^{sub} \equiv \topl^{sub}_S \equiv \exists A\in \topl$ tal que $V = A∩S$.
\end{defn}

Ejercicio: comprobar que es una topología.

\begin{prop}
\stopl e.t, $S⊆X$
\begin{enumerate}
\item $C(⊆S)$ es cerrado en $\topl^{sub} \iff \exists F$ cerrado en \stopl tal que $C=F∩S$.
\item Si $\topl = \toplb$, la topología generada por una base $\base$, entonces
	\begin{enumerate}
	\item $\base_S \equiv \{B∩S : B\in \base\}$ es una base para una top en $S$
	\item $\topl^{sub} = \topl_{\base_S}$
	\end{enumerate}
\item Si $\topl = \topl_d$ es la topología inducida por una distancia $\dst(x,y)$ en $X$, entonces $\topl^{sub} = \topl_{\dst|_{S×S}}$, tomando $\dst|_{S×S}$ como la restricción de la distancia al conjunto $S$:
\begin{align*}
	\appl{\dst|_{S×S}}{S×S&}{[0,\infty)} \\
	(x,y)&\longmapsto \dst(x,y)
\end{align*}

\end{enumerate}
\end{prop}

\begin{proof}
\begin{enumerate}
\item $C$ cerrado en $\topl^{sub}$ si y sólo si $S\setminus C ∈ \topl^{sub}$, lo cual es equivalente a a que $∃A∈\topl$  tal que $S\setminus C = A∩S$. 

Si $F≝X\setminus A$, entonces $A=X\setminus F$, luego volviendo a lo que teníamos nos queda que $S\setminus C = (X\setminus F) ∩ S = S\setminus (F∩S)$, equivalente a $C=F∩S$.
\item
\item
\end{enumerate}
\end{proof}

\section{Continuidad} 

En esta sección estudiaremos las funciones $f$ entre dos espacios topológicos $(X,\topl_X)$ y $(Y, \topl_Y)$. Usaremos la forma habitual $\appl{f}{X}{Y}$, aunque para ser más concretos las denotaremos como $\appl{f}{(X,\topl_X)}{(Y, \topl_Y)}$.

\begin{defn}[Función\IS continua]
Sea $x_0∈ X$. Se dice que $f$ es continua en $x_0$ si y sólo si $∀V$ entorno de $f(x_0)$, existe un $U∈\topl_X$ entorno de $x_0$ tal que $f(x)∈V\; ∀x∈U$.

Es decir, que $f(U)⊆ V$ y que $x_0∈ U ⊆\inv{f}(V)$.

Por otra parte, y como hacíamos en otras asignaturas, se dice que una función es continua si es continua en todos los puntos de su dominio.
\end{defn}

\begin{remark} En el caso de espacios métricos $(X, \dst_X), (Y, \dst_Y)$, se dice que $f$ es continua en $x_0$ si y sólo si $∀ε>0\;∃δ>0 \tq \dst_X(x,x_0) < δ \implies \dst_Y(f(x), f(x_0)) < ε$, que es decir de forma más general la forma que hemos puesto de continuidad.

Además, para topologías generadas por bases, basta considerar entornos que están en la base, como siempre.
\end{remark}

Veamos un ejemplo simple sobre continuidad de funciones. Más concretamente, continuidad de funcionales usando sólo la definición de continuidad basada en topologías.

\begin{example} Tomamos $X$ como el conjunto de las funciones continuas $X=C([0,1])≝\{ \appl{f}{[0,1]}{ℝ} \tq f ∈ C^1\}$, y definimos la distancia como
\begin{align*}
\md{f}_{∞} &≝ \max_{x∈[0,1]} \abs{f(x)} \\ 
\dst(f,g) &≝ \md{f-g}_{\infty}
\end{align*}

A partir de esto definimos el siguiente funcional: 
\begin{align*}
\appl{F}{(X, \dst_X)&}{(ℝ, \topl_{usual})} \\
f&\longmapsto F(f) ≝ \int_0^1f(x) \dif x 
\end{align*}

Afirmamos que $F$ es continua es $f_0 ∈ X$. Podemos encontrar $\epsilon$ y $\delta$ de la siguinte forma:

\begin{align*}
\dst_{ℝ}(F(f), F(f_0)) &= \abs{F(f) - F(f_0)} = \abs{\int_0^1 f(x) \dif x - \int_0^1 f_0(x)\dif x} = \\
&= \abs{\int_0^1 f(x) - f_0(x) \dif x} \leq \int_0^1 \abs{f(x)-f_0(x) \dif x} \leq \\
&\leq \int_0^1 \md{f(x) - f_0(x)}_{\infty} \dif x \leq \md{f - f_0}_{\infty} = \dst_{X}(f, f_0) < \delta
\end{align*}

Luego, como $\dst_{\mathbb{R}}(F(f), F(f_0)) < \epsilon$, $\forall \epsilon > 0$ , tenemos que:

\begin{align*}
\delta = \epsilon \wedge dst_{X}(f, f_0) < \delta = \epsilon \implies \dst_{\mathbb{R}}(F(f), F(f_0)) \leq dst_{X}(f, f_0) - \delta
\end{align*} \qed

De hecho, hemos obtenido un resultado más fuerte: el $\delta$ no depende del punto, luego F es \underline{uniformemente contínua}.

\end{example}

A partir del ejemplo, podemos desarrollar nuevas propiedades:

\begin{prop} Sean $(X_1, \topl_1), (X_2, \topl_2)$ espacios topológicos y $\appl{f}{X_1}{X_2}$. Entonces

\begin{enumerate}
\item $f$ es continua si y sólo si la imagen inversa de un abierto en $X_2$ es abierta en $X_1$. Es decir, si y sólo si $∀A∈\topl_2 \; \inv{f}(A) ∈ \topl_1$.
\item $f$ es continua si y sólo si la imagen inversa de un cerrado en $X_2$ es cerrada en $X_2$.
\item $f$ es continua si y sólo si $f(\adh{W}) ⊆ \adh{f(W)}\; ∀W⊆X_1$.
\end{enumerate}
\end{prop}

\begin{proof}
\begin{enumerate}
\item Empezamos tomando como hipótesis que $f$ es continua en $x\; ∀x∈X_1$. Tomamos $A∈\topl_2$ y estudiamos su imagen inversa $\inv{f}(A)$ para ver si está en$\topl_1$. Si $x∈\inv{f}(A)$, entonces $f(x) ∈ A$.  Si $A$ es abierto en $X_2$ y $f(x)∈A$, entonces por la definición de continuidad de $f$, $∃V_x$ entorno de $X$ val que $f(V_x)⊆A$. 

Es decir, que $x∈V_x⊆\inv{f}(A)$. Como esto pasa para todo punto, podemos escribir $\inv{f}(A)$ como la unión de todos los $V_x$, la unión de abiertos es abierta y entonces ya tenemos que $A$ es abierto.

En el otro sentido, tomamos como hipótesis que la imagen inversa de un abierto es abierta. Sea $x∈X$, queremos saber si $f$ es continua en $x$. 

La definición de continuidad nos decía que dado un entorno $W$ cualquiera de $f(x)$, existía un entorno $V_x$ de $x$ tal que $f(V_x) ⊆ W$. Dicho de otra forma, cercanía en el dominio implica cercanía en la imagen.

Entonces, partiendo de dos premisas ($W∈\topl_2 \implies \inv{f}(W) ∈ \topl_1$, $f(x)∈W \iff x∈\inv{f}(W)$) nos queda que $V=\inv{f}(W)∈ \topl_1$, $x∈V$ y $f(V)⊆W$. % No me queda muy claro esto pero lo dejo así.

\item Queremos demostrar que la imagen inversa de cerrados es cerrada, y vamos a usar la propiedad de que el complementario de un cerrado es abierto. Vemos claramente que $\inv{f}(X_2\setminus A ) = X_1 \setminus \inv{f}(A)$. A partir de esto es muy sencillo probarlo tomando complementarios, y no lo voy a copiar.

\item Tomamos como hipótesis que $f$ es continua, y queremos demostrar que $f(\adh{W}) ⊆ \adh{f(W)}\; ∀W⊆X_1$. Está claro que $\adh{f(W)}$ es cerrado, y que al ser $f$ continua entonces $\inv{f}(\adh{f(W)})$ es cerrado también. Además, es seguro que $W⊆\inv{f}(\adh{f(W)})$. Uniendo las dos cosas, tenemos que $\adh{W} ⊆ \inv{f}(\adh{f(W)})$. 
\end{enumerate}
\end{proof}

Vamos a ir ahora a por algunas propiedades sobre la composición de funciones.

\begin{prop} Sean $(X,\topl_X), (Y, \topl_Y), (Z, \topl_Z)$ espacios topológicos. Entonces 

\begin{enumerate}
\item Si $\appl{f}{X}{Y}$ y $\appl{g}{Y}{Z}$  son continuas entonces $\appl{g○f}{X}{Z}$ también lo es.
\item Si $\appl{f}{X}{Y}$ es constante entonces $f$ es continua.
\item Si $\appl{f}{X}{Y}$continua y $S$ subespacio de $X$ (es decir, $S⊆X$ con $\topl^{sub}$), entonces $\appl{f|_S}{S}{Y}$ es continua.
\item Sea $\appl{f}{X}{Y}$ y $W$ tales que $f(X) ⊆ W ⊆ Y$. Denotemos $\appl{f^W}{X}{W}$ con $W$ en la topología de subespacios. Entonces $f$ es continua si y sólo si $f^W$ es continua.
\end{enumerate}
\end{prop}

\begin{proof}
\begin{enumerate}
\item Sea $A$ abierto en $Z$, entonces tenemos que demostrar que $(g○f)^{-1}(A)$ es abierto en $X$. Sabemos que $\inv{(g○f)}(A) = \inv{f} ()$ y nosequé.
\end{enumerate}
\end{proof}

\begin{remark} Si tenemos una aplicación $\appl{f}{(X,\topl_X)}{(Y, \topl_Y)}$ donde $\topl_Y = \toplb$ con $\base$ una base, entonces $f$ es continua si y sólo si $\inv{f}(B)∈\topl_X\;∀B∈\base$.
\end{remark}

\begin{defn}[Homeomorfismo] Un homeomorfismo entre espacios topológicos $(X,\topl_X)$, $(Y, \topl_Y)$ es una aplicación biyectiva $\appl{f}{X}{Y}$ y tal que tanto $f$ como $\inv{f}$ son continuas. En este caso se dice que $(X,\topl_X)$, $(Y, \topl_Y)$ son homeomorfos.
\end{defn}

\begin{defn}[Propiedad\IS topológica] Una propiedad de un espacio topológico $(X, \topl_X)$ es topológica si la comparten todos los espacios topológicos homeomorfos a $(X, \topl_X)$.\end{defn}

\begin{remark} La topología de subespacios usual en $\bola(0,1)$ es $\topl_{\dst}$ donde $\dst$ es la distancia euclídea en $\bola(0,1)$. \end{remark}

Podemos demostrar que $ℝ^2$ y $\bola(0,1)$ son homeomorfos, tomando la biyección 

\begin{align*}
\appl{f}{ℝ^2&}{\bola(0,1)} \\
\vx&\longmapsto \frac{\vx}{1+\md{\vx}} = \vy
\end{align*}

Demostramos que es biyectiva: \[ \md{\vy} = \frac{\md\vx}{1 + \md{\vx}} \implies \md{\vx} = \frac{\md{\vy}}{1- \md{\vy}} \], luego \[ \vx = \frac{\vy}{1-\md{\vy}} \]. 

Otro ejemplo: vamos a demostrar que $[0,1]$ y $(0,1)$ no son homeomorfos. La propiedad que nos va a interesar sería la compacidad, pero todavía no tenemos la maquinaria para hacerlo. 

Lo que haremos será ver qué ocurre si quitamos un punto. Cualquier punto que quitemos en $(0,1)$ nos dejará dos trozos no conexos. Ahora bien, si lo hacemos en $[0,1]$, podemos quedarnos con un trozo conexo si el punto es $0$ o $1$.

Vamos a demostrar que no son homeomorfos viendo que si $\appl{f}{[0,1]}{(0,1)}$ es continua, entonces no es sobreyectiva.

Un pequeño aparte: tenemos que entender que cada intervalo está con la topología del subespacio. En el caso de $\topl_{[0,1]}^{sub}$, la topología del subespacio sería la generada por los elementos de la base de $\topl_ℝ$ intersección $[0,1]$. Es decir, la base es \begin{multline*} \base = \{ (a,b) ∩ [0,1] \tq a < b, a,b∈ℝ \} \equiv \\ \equiv \left\{ [0,1], [0,b), (a, 1], (c,d) \tq 0 < b,a < 1,\; 0<c<d<1 \right\} \end{multline*}

Volviendo a lo que íbamos: no tendremos problema por parte de la imagen inversa. En este caso, la imagen inversa de un abierto $(0,1)$ es un abierto: $[0,1]$ es abierto en la topología de su subespacio (es el total).

Donde sí vamos a lograr algo es viendo que $f$ continua en $[0,1]$ alcanza un máximo. Es decir, $∃x_0 ∈ [0,1]$ tal que $f(x) ≤ f(x_0)\; ∀x∈[0,1]$. 

Por otra parte, es obvio que $f(x_0)∈(0,1)\implies f(x_0) < 1$. Juntando estas dos conclusiones llegamos a que $(f(x_0), 1) ∩ f([0,1]) = \emptyset$. Es decir, $f$ no es sobreyectiva (por ejemplo, $\frac{1+f(x_0)}{2}$ no está en la imagen).

Otra forma sería ver que si $f$ fuese inyectiva, tendría que ser monótona creciente o decreciente. Podríamos suponer sin pérdida de generalidad que fuese creciente, y entonces $f(0) < f(x)\; ∀x∈(0,1]$. Sin embargo, siempre podríamos encontrar un $y ∈ (0, f(0))$ por lo que no sería sobreyectiva.

\paragraph{Propiedad topológica Hausdorff} Recordamos los espacios Hausdorff (ver \ref{defHausdorff}).  Queremos demostrar que si $(X,\topl_X)$ es Hausdorff y  $(X,\topl_X)$ es homeomorfo a $(Y, \topl_Y)$, entonces $(Y, \topl_Y)$ es Hausdorff igualmente.

Tomamos $x_1∈V_1$ y $x_2∈V_2$, ambos en $X$, con $y_j=f(x_j)$. Hay que comprobar que $y_j∈f(V_j)$, que $f(V_j)$ es abierto en $Y$ y que la intersección $f(V_1) ∩ F(V_2) = \emptyset$.

\begin{remark} Si $f$ es un homemorfismo entonces $\inv{f}$ es continua. Dicho de otra forma, la imagen inversa de la inversa de un abierto es abierto. % WTF.

También se puede decir que $f$ es homeomorfismo si $f$ es biyectiva, y que la imagen inversa de $\inv{f}$ de un abierto es abierto, y que $f$ de un abierto es abierto también.
\end{remark}

\begin{defn}[Función\IS abierta] Sean $(X,\topl_X)$, $(Y, \topl_Y)$ espacios topológicos y $\appl{f}{X}{Y}.$ Se dice que $f$ es abierta si y sólo si $∀A∈\topl_X$ se tiene que $f(A) ∈ \topl_Y$.

Por otra parte, se dice que una función es una \concept[Función! cerrada]{función cerrada} si y sólo si para todo $C$ cerrado en $X$, $f(C)$ es cerrado en $Y$.
\end{defn}

Por ejemplo, si $Y$ es Hausdorff (\ref{defHausdorff}) y $f$ es constante, entonces es cerrada. La razón es que en un espacio Hausdorff, un único punto siempre es un conjunto cerrado.

Las dos definiciones no son excluyentes. Un homeomorfismo es aplicación abierta y cerrada, por ejemplo. También podríamos construir una aplicación $f$ abierta y cerrada sin que sea necesario homeomorfismo. Podemos coger una $f$ constante e $Y$ con la topología discreta, o con $\topl_Y = \{ \emptyset, \{p\}, \{p\}^c, Y\}$.

\section{Topología producto}

Ahora consideraremos dos espacios topoloógicos $(X_1, \topl_1), (X_2, \topl_2)$, y estudiaremos la topología en $X_1×X_2$.

\begin{prop} Sea $\base = \{ V_1 × V_2 \tq V_i ∈ \topl_i \}$. $\base$ es una base para una topología en $X_1 × X_2$. $\topl_\base$ es la topología producto y se denota $\topl_1 \otimes \topl_2$ (habitualmente $\topl_1× \topl_2$).
\end{prop}

\begin{proof}
Vamos a demostrar que efectivamente es una topología. Está claro que $\bigcup_{B∈\base} B = X_1 × X_2$, ya que $X_1$ y $X_2$ son el total y están en sus respectivas topologías, por lo que $X_1× X_2∈\base$.

Por otra parte, queremos demostrar que la intersección también está.  Si tenemos $B, C ∈ \base$ y  $x∈B∩C$, entonces existe un $\tilde{B}∈\base$ tal que $X∈\tilde{B}⊆B∩C$.

Para ello usaremos que $(V_1×V_2) ∩ (\hat{V}_1 × \hat{V}_2) = (V_1 ∩ \hat{V}_1) × (V_2 ∩ \hat{V}_2)$, luego $B∩C ∈ \base$. Podría copiar la demostración pero no me da tiempo.

\end{proof}

La topología producto no se limita a sólo dos espacios topológicos.

\begin{prop} Más generalmente, si $\{X_j,\topl_j\}_{j=1,\dotsc, m}$ es un conjunto de espacios topológicos, entonces $\base = \{ V_1× V_2 × \dotsb × V_m \tq V_j ∈ \topl_j\}$ es una base para una topología producto $\topl_1 \otimes \topl_2 \dotsb \otimes \topl_m$.
\end{prop}

Como ejercicio, podemos considerar tres espacios topológicos, y estudiar si la topología producto es asociativa, es decir, si 

\[ \topl_1 \otimes \topl_2 \otimes \topl_3 = (\topl_1 \otimes \topl_2)\otimes \topl_3 = \topl_1 \otimes (\topl_2 \otimes \topl_3 ) \]

Otro ejemplo, ¿son iguales $\topl_{ℝ^2} = \topl_ℝ \otimes \topl_ℝ$? En el primer caso, la base $\base_1$ de $ℝ^2$ son elementos de la forma $(a_1,b_1) × (a_2, b_2)$, es decir, rectángulos. La segunda base $\base_2$ está formada por los elementos $V_1 × V_2$ donde $V_i ∈ \topl_ℝ$.

Para demostrarlo vamos a ver lo de siempre: las bases molan. Si tenemos $\base$ una base y $\topl$ una topología con $B⊆\topl$, entonces $\topl_\base⊆\topl$. Es decir, $\topl_\base ⊆ \topl \iff \base ⊆ \topl$. 

Como consecuencia de esa observación, si $\base_1,\base_2$ son bases entonces $\topl_{\base_1} ⊆ \topl_{\base_2} \iff \base_1 ⊆ \topl_{\base_2}$ y, por tanto \[ \topl_{\base_1} = \topl_{\base_2} \iff \base_1 ⊆ \topl_{\base_2} \y \base_2 ⊆ \topl_{\base_1} \]

Con esto ya podemos volver a nuestra vida normal y demostrar que $\topl_{ℝ^2} = \topl_ℝ \otimes \topl_ℝ$. Si $B∈\base_1$ con $B=(a_1,b_1) × (a_2, b_2)$, entonces como $(a_1,b_1)$ y $(a_2, b_2)$ son abiertos en $\topl_ℝ$, $B∈\base_2$ y $\base_1 ⊆ \base_2 ⊆ \topl_{\base_2}$. Con esto hemos demostrado el contenido hacia la izquierda ( $\topl_{ℝ^2} ⊆ \topl_ℝ \otimes \topl_ℝ$).

Vamos ahora para el otro lado. Si $B∈\base_2$ con $B=V_1 ×V_2$, con $V_i ∈ \topl_ℝ$, tenemos que $V_1 = \bigcup_{j∈ J_1}(a_j, b_j)$ y análogamente con $V_2 = \bigcup_{k∈K_2} (c_k, d_k)$. Entonces 

\[ V_1 × V_2 = \bigcup_{\substack{j∈J_1 \\ k∈K_2}} (a_j, b_j) ×(c_k, d_k) \], por lo tanto $V_1×V_2$ es unión de elementos de $\base_1$ y por lo tanto está en $\topl_{\base_1}$, luego $\base_2 ⊆ \topl_{\base_1}$ y tenemos la inclusión para el otro lado, y entonces $\topl_{\base_1} = \topl_{\base_2}$.

Pero la justificación de esto no sólo nos vale para $ℝ$, sino que puede ser mucho más general si no escribimos explícitamente los intervalos de $ℝ$, que no son más que abiertos de la topología.

\begin{prop} Sean $(X, \topl_{\base_X})$, $(Y, \topl_{\base_Y})$. Entonces \[ \topl_{\base_X} \otimes  \topl_{\base_Y} = \topl_\base \] donde \[ \base =\left\{ B × \hat{B} \tq B ∈ \base_X, \hat{B} ∈ \base_Y \right\} \].
\end{prop}

\begin{prop} Sean $(X_1, \dst_1)$ y $(X_2, \dst_2)$ espacios métricos. Entonces $\topl_{\dst_1} \otimes \topl_{\dst_2} = \topl_{\dst}$ donde \[ \dst\left((x_1, x_2), (y_1, y_2)\right) = \dst_1(x_1, y_1) + \dst(x_2, y_2)\]
\end{prop}

La demostración se deja como ejercicio para el lector, aunque es parecido al ejercicio 18 de la hoja 1.

\subsection{Funciones continuas y topología producto}

La gran ventaja de la topología producto nos va a venir a la hora de comprobar continuidad de funciones en espacios producto. Vamos a verlo.

\begin{prop} Sean $(X_1, \topl_1), (X_2, \topl_2)$ espacios topológicos. Entonces

\begin{enumerate}
\item Sean $\appl{P_j}{(X_1×X_2, \topl_1×\topl_2)}{(X_j, \topl_j)}$ con $P_1((x_1, x_2)) = x_1$ y $P_2((x_1, x_2)) = x_2$ las proyecciones. Entonces $P_1$ y $P_2$ son continuas.

\item Sea $\appl{f}{(X, \topl)}{(X_1×X_2, \topl_1 \otimes \topl_2)}$ tal que $x\longmapsto f(x) = (f_1(x), f_2(x))$. Entonces $f$ es continua si y sólo si $f_1, f_2$ son continuas.

\item Si $\appl{f, g}{(X, \topl)}{(ℝ, \topl_ℝ)}$ son continuas entonces $f\pm g$, $fg$ y $\frac{f}{g}$ son continuas, en el último caso suponiendo que $g(x) ≠ 0$.
\end{enumerate}

Las propiedades se mantienen para un producto finito de espacios.
\end{prop}

\begin{proof}
\begin{enumerate}
\item Si $P_1$ es continua, entonces $\inv{P_1}(A_1) ∈ \topl_1 \otimes \topl_2\quad ∀ A_1 ∈ \topl_1$. Tenemos que ver primero qué es $\inv{P_1}(A_1)$. Al ser la proyección, $\inv{P_1}(A_1) = A_1 × X_2$, que es claramente un abierto en la topología producto (es un elemento de la base).

\item Podemos decir que $f_1 = P_1 ○ f$ y $f_2 = P_2 ○ f$. La implicación a la derecha es obvia: si $f$ es continua, como $P_1$ y $P_2$ son continuas entonces $P_1 ○ f$ y $P_2 ○ f$ son continuas.
\end{enumerate}
\end{proof}
 


\appendix
\chapter{Ejercicios}

% -*- root: ../TopologiaI.tex -*-
\section{Hoja 1}

\subsection{Definición de topología y ejemplos}

\begin{problem}[1]
Sea $X=\{a,b,c\}$ un conjunto de 3 elementos. Encontrar todas las topologías sobre $X$.
\solution

Las topologías más sencillas son: $\topl_1 = \{\emptyset,X,a\}$. Es trivial comprobar que cumple las 3 propiedades de topología. Lo mismo con $b$, con $c$, con ${a,b}$, con ${b,c}$ y ${a,c}$.

El siguiente nivel de complejidad son aquellas con 2 elementos: $\topl_4 = \{\emptyset,a,b,X\}$. Para que esto fuera topología faltaría $\{a\} \cup \{b\}$, con lo que $\topl_4 = {\emptyset,a,b,\{a,b\}X}$

Y así acaban saliendo todas.
\end{problem}

\begin{problem}[2]
En $\real$ se considera la $\topl =\{(-\infty,a): -\infty \leq a \leq \infty\}$. Demostrar que es una topología.
\solution

Tenemos que demostrar las 3 propiedades de topología:

\begin{enumerate}
\item $\emptyset,\real \in \topl$. Basta tomar $a=\pm \infty$ para tener ambas pertenencias.
\item $A,B\in\topl \implies A\cap B\in\topl$. Dados $a_1,a_2\in\real$, con $a_1\neq a_2$ tenemos $A_1 = (-\infty,a_1)$ y $A_2 = (-\infty,a_2)$.

Entonces, $A_1\cap A_2 = (.-\infty,\min(a_1,a_2)) = \left\{\begin{array}{cc}
A_1\cap A_2 = A_1 & si\; a_1 = \min(a_1,a_2)\\A_1\cap A_2 = A_2  & si\; a_2 = \min(a_1,a_2)
\end{array}\right.$

Hemos demostrado que $A_1 \cap A_2 \in \topl$.

\item $A,B\in\topl \implies A\cup B\in\topl$. Lo mismo que el apartado anterior tomando máximos en vez de mínimos.
\end{enumerate}
\end{problem}


\begin{problem}[3]
Sean $X$ un conjunto infinito y $\topl$ una topología sobre $X$ en la que todos los subconjuntos infinitos son abiertos. Demostrar que $\topl$ es la topología discreta de $X$.
\solution

Tenemos que demostrar que $\topl = \topl_{disc} = \mathcal{P}(X)$
\end{problem}

\begin{problem}[4]
 Sea $X$ un conjunto con más de dos elementos.
\ppart Definir dos topologías $\topl_1,\topl_2$ sobre $X$ de modo que $\topl_1 \cup \topl_2$ no sea una topología.
\ppart Sea $\topl_j , j ∈ J$ una familia de topologías sobre $X$. Probar que
 $\bigcap_{j∈J} Tj$ es también una topología sobre $X$.

\solution
\spart Sea $X = \{a,b,c\}$. Definiendo $\topl_1 = \{\emptyset,X,a\}$ y $\topl_2=\{\emptyset,X,b\}$.

$\topl_1 \cup \topl_2 = \topl_{\cup}= \{\emptyset,X,a,b\}$ no es topología, porque $\{a\} \cup \{b\} = \{a,b\} \notin \topl_{\cup}$
\spart
Probándolo para 2 topologías cualesquiera, lo habremos probado para todas, pues por inducción, si se cumple para 2 se cumple para una cantidad finita.

Tenemos que comprobar las 3 propiedades de topología:

\begin{enumerate}
\item $\emptyset,\real \in \topl$. Como $\emptyset,X $ pertenecen a ambas $\topl_1,\topl_2 \implies \emptyset,X \in \topl_{\cap}$

\item $A_1,A_2\in\topl_{\cap} \implies A_1\cap A_2\in\topl_{\cap}$.\\
\[\left.\begin{array}{c}
A_1\in\topl_{\cap}\implies
\left\{
	\begin{array}{cc}
		A_1\in\topl_1\\A_1\in\topl_2
	\end{array}
\right.\\
 \text{Lo mismo con } A_2.
\end{array}\right\}\implies \begin{array}{c}
A_1\cap A_2\in\topl_1\\
A_1\cap A_2\in \topl_2\end{array}\implies A_1\cap A_2 \in \topl_{\cap}\]


\item $A_1,A_2\in\topl_{\cup} \implies A\cup A_2\in\topl_{\cup}$. El mismo razonamiento es válido para la unión.
\end{enumerate}
\end{problem}

\begin{problem}[5]
 En el plano $\real^2$ se considera la familia $\topl$ de todos los subconjuntos $U$ tales que para cada punto
$(a, b) \in U\; \exists  ε > 0 \tlq ((a − ε, a + ε) × b) ∪ (a × (b − ε, b + ε)) \subset U$.

Estudiar si $\topl$ es una topología en $\real^2$.
\solution

Gráficamente es fácil de contestar. Los subconjuntos pedidos son cruces y la intersección de 2 cruces no es una cruz, con lo que no puede ser topología.
\end{problem}


\begin{problem}[6] Sea $\appl{g}{X}{Y}$ una aplicación entre dos conjuntos.

\ppart Demostrar que si $\topl$ es una topología en $X$ entonces \[ \mathcal{S} = \{ E ⊆ Y \tq \inv{g}(E) ∈ \topl \} \] es una topología en $Y$.
\ppart Demostrar que si $\mathcal{S}$ es una topología en $Y$ entonces \[ \mathcal{U} = \{ \inv{g}(E) \tq E ∈ \mathcal{S} \} \]es una topología en $X$.

\solution
\spart Vamos a demostrar que es una topología, para lo cual tenemos que comprobar las 3 propiedades (ver \ref{defTopologia}):

Es importante saber que las aplicaciones entre conjuntos se definen en todo el dominio, no en un subconjunto, es decir, $∀ x ∈ X, ∃g(x)∈Y$

\begin{enumerate}
\item Tomando $E=Y$, tenemos $Y∈ \mathcal{S}$ por ser $g$ una aplicación de conjuntos tal que $g^{-1}(Y)=X$,=. (La imagen inversa de todo Y pertenece a la topología de X, ya que es X.)

Tomando $E=∅$, tenemos que $g^{-1}(Ø) = Ø \in \topl$, porque no puede existir un $x\in X\tlq g(x)=Ø$ por ser $g$ aplicación de conjuntos (todos los elementos tienen que tener una imagen del cunjunto destino).

\item $A,B \in \mathcal{S} \dimplies g^{-1}(A),g^{-1}(B) \in \mathcal{T}$. (1)

$A\cap B \in\mathcal{S} \dimplies g^{-1}(A\cap B) \in \mathcal{T}$.(2)

Si tuvieramos que (1) $\implies$ (2) ya lo tendríamos demostrado. Vamos a demostrar que $g^{-1}(A),g^{-1}(B) \in \mathcal{T} \implies g^{-1}(A\cap B) \in \mathcal{T}$.

Para ello: $g^{-1}(A\cap B) = g^{-1}(A)\cap g^{-1}(B)$. No es difícil convencernos de esta igualdad. Para resolver las dudas, vamos a demostrar las 2 inclusiones (una en cada sentido).

\paragraph{$\subset$}
\begin{gather*}
g^{-1}(A\cap B) \subset g^{-1}(A)\cap g^{-1}(B) \implies
x∈g^{-1}(A∩B) \implies ∃y∈A∩B\tq g^{-1}(y)=x\\
\implies\left| \begin{array}{c}
y∈A \implies g^{-1}(y)=x∈g^{-1}(A)\\
y∈B \implies g^{-1}(y)=x∈g^{-1}(B)
\end{array}
\right.
\end{gather*}

\paragraph{$⊃$}


\begin{gather*}
g^{-1}(A∩B) ⊃ g^{-1}(A)∩g^{-1}(B) \implies x∈g^{-1}(A)∩g^{-1}(B)\implies \\
\implies
\left|\begin{array}{cc}
x∈g^{-1}(A) \implies ∃ y_a ∈ A\tq g^{-1}(y_a)=x\\
x∈g^{-1}(B) \implies ∃ y_b ∈ B\tq g^{-1}(y_b)=x
\end{array}\right. \underset{g\, aplicacion}{\implies} y_a=y_b\\
\end{gather*}
Hemos demostrado: $$∀x∈g^{-1}(A)∩g^{-1}(B)∃y∈A∩B \tq g^{-1}(y)=x∈g^{-1}(A∩B)$$


Es importante el comentario de que esto para imágenes directas no funciona.

\item Demostramos ahora que la unión de abiertos está en la topología. Si $A, B ∈ \mathcal{S}$, entonces $\inv{g}(A), \inv{g}(B) ∈ \topl$. Como $\topl$ es topología, tenemos que $\inv{g}(A) ∪ \inv{g}(B) ∈ \topl$, lo que implica (por el mismo razonamiento que antes) que \\$\inv{g}(A∪B) ∈ \topl$ y por lo tanto $A∪B ∈ \mathcal{S}$.
\end{enumerate}

\spart

\end{problem}


\begin{problem}[7]
Sean $X$ un conjunto y $a$ un elemento de $X$. Se considera la familia $\topl_a$ de los subconjuntos $U\subset X$ tales que o bien $U = \emptyset$, o bien $a ∈ U$ . Estudiar si $\topl_a$ es una topología en $X$.
\solution

Comprobamos las 3 propiedades de topología:
\begin{enumerate}
\item
\begin{itemize}
\item $∅∈\topl_a$: Sí, por definición de $\topl_a$.
\item Tomando $U=X$, tenemos que $X\in\topl_a \impliedby a\in X$.
\end{itemize}
\item Sean $U₁,U₂∈X.$  Tenemos que demostrar si $a∈U₁∩U₂$. En este caso está claro dado que a∈U₁ y a∈U₂.
\item Lo mismo con la unión.
\end{enumerate}

Podemos concluir que $\topl_a$ es una topología en $X$.

\end{problem}

\subsection{Bases y entornos}

\begin{problem}[9] Se consideran las siguientes familias de conjuntos en $ℝ$:

\begin{gather*}
\base_{\leftarrow} = \{ (-∞, b) \tq b ∈ ℝ \} \\
\base_{\rightarrow} = \{ (a,∞) \tq  ∈ ℝ \}
\end{gather*}

\ppart Demostrar que cada familia es una base de una topología sobre $ℝ$.
\ppart Comparar esas topologías.
\ppart Demostrar que la topología generada por $\base_{\leftarrow} ∪ \base_{\rightarrow}$ es la usual.
\solution
\spart Recordamos las propiedades que tiene que cumplir para ser base: \ref{defBase}

En este caso, $$\begin{array}{cc}
∀x∈ℝ, x∈\mathcal{B}_1 & \text{ basta con tomar } b=x+1\\
∀x∈ℝ, x∈\mathcal{B}_2 & \text{ basta con tomar } a=x-1
\end{array}$$

La segunda propiedad la tenemos sin darle muchas vueltas: $b_1,b_2∈\mathcal{B}_1\; b_1∩b_2 = max\{b_1,b_2\} ∈\mathcal{B}_1$. Ni siquiera tenemos que encontrar un elemento de la base que lo contenga. Lo mismo para $\mathcal{B}_2$.

\textbf{Importante: } Observamos que es imprescindible mencionar que tenemos que añadir el vacío y el total a ambas bases para que sean realmente bases.

\spart
\spart Dado que $\base_{\leftarrow} ∪ \base_{\rightarrow}$  es una base (sino, no podríamos hablar de topología generada por $\base_{\leftarrow} ∪ \base_{\rightarrow}$) tenemos que la intersección de 2 elementos está contenido en un elemento de la base, que a su vez está contenido en la intersección (\ref{defTopologiaGeneradaBase})

Si $(c,d) ∈ \toplb \implies ∀x∈(c,d) ∃ (-∞,b)\tlq x∈B\subseteq (c,d)$. Es imposible para los 2 casos, con lo que, para que sea base, elementos del tipo $(c,d)$ tienen que estar incluidos en $\base_{\leftarrow} ∪ \base_{\rightarrow}$, dando lugar a la topología usual (tienen la misma base).

\end{problem}


\begin{problem}[10]
Probar que si $\mathcal{B}$ es una base para una topología sobre $X$, entonces la topología $\toplb$ generada por
$\mathcal{B}$ es igual a la intersección de todas las topologías sobre $X$ que contienen a $\mathcal{B}$.
\solution

Sea $\topl$ una topología sobre $X$. Queremos demostrar que si $\mathcal{B}$ base tal que $\toplb = \topl \implies \toplb = \bigcap \{\topl_X \tlq \mathcal{B}∈\topl_X\quad ∀x\}$.


Para demostrar la igualdad, demostraremos $\toplb \subseteq \bigcap \topl_X$ y $\bigcap \topl_X \subseteq \toplb$. Recordamos que todas las cosas que tengamos que probar para una topología generada por una base, basta con probarla para los elementos de la base.

$\toplb \subseteq \bigcap \topl_X$ $$∀A\in\toplb \implies A∈\bigcap\topl_x \impliedby  A ∈\mathcal{B} \subset \topl_x ∀x$$


$\bigcap \topl_X \subseteq \toplb$ $$\text{Esta propiedad es obvia, dado que } \toplb \subseteq \{\bigcap \topl_x\}$$
\end{problem}

\begin{problem}[11]
Sea $\topl_j$, $j∈J$ una familia de topologías sobre $X$. Demostrar que existe una topología que contiene a todas las $\topl_j$, para $j∈J$ y además es la menos fina de todas las que verifican esta propiedad.
\solution

Aplicamos directamente la proposición \ref{propTopologiaMinima}: la topología que contiene a todas ellas es \[ \topl = \bigcap_{j∈J} \topl_j \]
\end{problem}

\paragraph{Observación útil para el 12 (y para el 5):}
\begin{enumerate}
\item $x \in C(x,\varepsilon)$
\item $\varepsilon_1 > \varepsilon_2 \implies C(x,\varepsilon_2) \subset C(x,\varepsilon_1)$
\end{enumerate}

Y podemos aplicar la propiedad:
\[
A\in\topl \dimplies \forall a\in A \exists \varepsilon > 0 \tlq C(x,\varepsilon)\subseteq A
\]

Haciendo caso al enunciado y haciendo el dibujo vemos que se cumplen las propiedades de base.

Esta topología contiene a la usual pero al revés no, porque para el punto de intersección de las diagonales no existe un abierto de la usual que le contenga.

\subsection{Espacios métricos}

\begin{problem}[13]
 Sea $(X, d)$ un espacio métrico. Demostrar que, para cualesquiera x, y, x e y elementos de X, se cumple
$$|d(x, y) − d(x' , y' )| ≤ d(x, x' ) + d(y, y' )$$
Deducir de ello que $\displaystyle\lim_{n→∞} d(x_n , y_n ) = d(x, y)$ cuando $\displaystyle\lim_{n→∞} d(x_n , x) = 0 = \lim{n→∞} d(y_n , y).$
\solution

$$|a-b| = |a+(-b)| ≤ |a| + |-b| = |a|+|b|$$

\end{problem}

\begin{problem}[14]
Estudiar si $(ℝ, d)$ es un espacio métrico, donde $\appl{\dst}{ℝ × ℝ}{ℝ}$ está definida como
$$d(x, y) = \left\{
	\begin{array}{cr}
		0 & si \quad x=y\\
		|x| + |y| & si \quad x≠y
	\end{array}
\right.$$


Dibujar la bola $B(x, r)$ para los casos
\ppart x = 0 y radio r = 1/2
\ppart x = 1/2 y r = 1.

\solution

Para comprobar que es una distancia tenemos que comprobar \ref{defEspacioMetrico}

\begin{itemize}
\item $d(x,x) = 0$. Sí
\item $d(x,y) ≥ 0$. Sí, de hecho con $x≠y$ tenemos $d(x,y)>0$.
\item Desigualdad triangular: $d(x,z) \leq d(x,y) + d(y,z)\; ∀x,y,z∈X$. En este caso,  $\dst(x,y) + \dst(y,z) = |x|+|y| + |y| + |z| ≥ |x|+|z|$
\end{itemize}

Hemos comprobado que tenemos un espacio métrico. Vamos a dibujar las bolas.

\spart

\begin{figure}[hbtp]
\centering
\inputtikz{E_H1_14_A}
\caption{$B(0,\frac{1}{2})$}
\label{H1_E14}
\end{figure}

\spart
\begin{figure}[hbtp]
\centering
\inputtikz{E_H1_14_B}
\caption{$B(\frac{1}{2},1)$}
\label{H1_E14}
\end{figure}
\end{problem}

\begin{problem}[15]
Demostrar que si $d_1$ es una distancia entonces $d_2(x, y) = \min\{ (d_1(x, y), 1)\}$ también lo es y que ambas distancias inducen la misma topología.
\solution

Es inmediato comprobar las 3 propiedades de distancia.

Vamos a comprobar que inducen la misma topología. Para ello, basta comprobar que las 2 métricas generan 2 bases que inducen la misma topología. Al empezar a tratar con bases, conseguimos hacer la demostración solamente para los elementos de la base.

Dada $B_{d₁}(x.ε)\quad ∀y∈B_{d₁}(x,ε)$ es posible encontrar una bola $B_{d₂}(y,δ)$ tal que $$y∈B_{d₂}(y,δ) \subseteq B_{d₁}(x,ε)$$ y viceversa, con lo que las bases generadas por las métricas son iguales $\implies$ inducen la misma topología.
\end{problem}

\paragraph{Pistas para los siguientes}

(16) Si tengo $d$, una distancia no acotada, puedo definir $d'=\frac{d}{1+d}$, que sigue siendo una distancia, parecida y además acotada.

(17) $\sum \frac{1}{2n} \leq 1$. La clave está en aplicar la desigualdad triangular a cada término del sumatorio. La clave para este problema es el 16.



\section{Hoja 2}

\subsection{Convergencia de sucesiones}

\begin{problem}[1]
 ¿Qué sucesiones convergen en la topología trivial? ¿y en la topología de los complementos contables?
\solution
\spart En $\topl_{triv}$ todas las sucesiones convergen y lo hacen a todos los puntos. ¿Porqué? porque si x∈X, el único abierto que contiene a x es el total.

\spart
La topología de los complementos contables se define así: $A∈\topl_{cocont} \dimplies (A=Ø) \;o\; (X\ A$ es contable (es decir, es finito o numerable).

\obs si $X$ es contable, $\topl_{cocont} = \topl_{disc}$.

\obs Las topologías se podrían definir a partir de los cerrados.

En este caso estamos pidiendo que los cerrados sean los subconjuntos contables  y el total. Comprobamos que es topología utilizando que la unión de numerables es numerables y la intersección de conjuntos numerables (en realidad, con al menos uno numerable es suficiente), es numerable.

Vamos a poner algún ejemplo de $\topl_{cocont}$ (porque le parece interesante comentarlo). Tomamos $X=ℝ$.

Si B es contable, $B ∩ (0,1) ≠ Ø$, pues de lo contrario, $(0,1)\subseteq B$ lo cual es imposible por ser $(0,1)$ infinito no numerable y $B$ contable.

Esta topología es otro ejemplo de NO Hausdorff, ya que la intersección de 2 abiertos nunca es vacía (si el total es no numerable).



Una vez acabados comentarios, vamos a contestar a la pregunta que nos hace el ejercicio que es sobre la convergencia de sucesiones.

$x_n \to x \dimplies ∃ n₀ \tq x_n = x ∀n≥ n₀$ (las sucesiones constantes a partir de un término).

Faltaría demostrar las 2 implicaciones.
\end{problem}

\begin{problem}[3]
 Estudiar la convergencia de las sucesiones $x_n = \left(\frac{1}{n}, \frac{1}{n}\right)$ e $y_n = \left(\frac{1}{n}, 1 − \frac{1}{n}\right)$ en $(ℝ^2 , \topl_{ℝ²} ) $, en $(ℝ^2, T_{lex} )$ y en $([0, 1]^2 , \topl_{lex})$.

\solution
\begin{itemize}

\item $x_n = \left(\frac{1}{n}, \frac{1}{n}\right)$ en $(ℝ^2 , \topl_{ℝ²} ) $

\item $x_n = \left(\frac{1}{n}, \frac{1}{n}\right)$  en $(ℝ^2, T_{lex} )$

\item $x_n = \left(\frac{1}{n}, \frac{1}{n}\right)$  en $([0, 1]^2 , \topl_{lex})$.

\item $y_n = \left(\frac{1}{n}, 1 − \frac{1}{n}\right)$ en $(ℝ^2 , \topl_{ℝ²} ) $

\item $y_n = \left(\frac{1}{n}, 1 − \frac{1}{n}\right)$ en $(ℝ^2, T_{lex} )$

\item $y_n = \left(\frac{1}{n}, 1 − \frac{1}{n}\right)$ en $([0, 1]^2 , \topl_{lex})$.
\end{itemize}
\end{problem}

\begin{problem}[6]
 Sea X un espacio topológico y A, D ⊂ X. Demostrar que:
\ppart $\mop{Fr} (A) = A \ \mop{Int} (A)$.
\ppart $\mop{Fr} (A) = ∅$ si y sólo si $A$ es simultáneamente abierto y cerrado.
\ppart Si $A ∩ D = ∅$ entonces $\mop{Fr} (A ∪ D) = \mop{Fr} (A) ∪ \mop{Fr} (D)$.
\ppart $\mop{Int} (A) ∪ \mop{Int} (D) ⊂ \mop{Int} (A ∪ D)$.
\ppart La inclusión en el apartado anterior puede ser estricta.


\textbf{Indicación: } encontrar $A, D \subset ℝ$ tales que $\mop{Int} (A) ∪ \mop{Int} (D) = \mop{Int} (A ∪ D)$.

\solution

Utilizamos la propiedad $\bar{A ∪ B} = \bar{A} ∪ \bar{B}$
(que se está como ejercicio la demostración en el 11.b)


$\bar{A} ∩ \bar{D} = \emptyset$.

\spart  $B=\bar{ D }^c$ es abierto y $\bar{A}\subseteq B$

Si $x∈\bar{A}$ y $V$ es un entorno de $x$, entonces $V∩B$ es un entorno de $x$, con $V∩B∩\bar{D} = \emptyset$.

\spart

\spart

\spart


\end{problem}

\begin{problem}[11]
\ppart
\ppart

\solution
\spart
\spart  Demostrar que si I es finito entonces $$\bigcup_{i∈I} \bar{A_i} = \bar{\bigcup_{i∈I} A_i}$$

Demostramos las 2 inclusiones.

$\bar{A}∪\bar{B} \subseteq \bar{A∪B}$. Está bastante claro.

$A\subseteq A∪B \subseteq \bar{A∪B}$. Como $\bar{A∪B}$ es cerrado y es más grande que $A$, tiene que contener a $\bar{A}$ (dado que éste es el cerrado más pequeño que lo contiene). El mismo razonamiento para $B$.

La inclusión en el otro sentido se puede demostrar así:

$A∪B \subseteq \underbrace{\bar{A}∪\bar{B}}_{cerrado}$. Entonces, $\bar{A∪B}\subseteq \bar{A}∪\bar{B}$.
\end{problem}

\subsection{Topología de subespacio y funciones continuas}

\begin{problem}[14] Explica si las siguientes funciones son continuas.

\ppart[c] Tenemos $X=[0,1]$ con la topología usual, y $Y=[0,1]^2$ con la topología $\topl_{Lex}$, la del orden lexicográfico. Sea $\appl{h}{X}{Y}$ tal que $h(t) = (t,1)$.
\solution

\spart[c] Lo primero que hay que hacer es entender las topologías de los conjuntos. En $X$, los abiertos serán elementos de la base de $ℝ$ intersección $X$, es decir, que la topología será generada por la base \begin{multline*} \base = \{ (a,b) ∩ [0,1] \tq a < b, a,b∈ℝ \} \equiv \\ \equiv \left\{ [0,1], [0,b), (a, 1], (c,d) \tq 0 < b,a < 1,\; 0<c<d<1 \right\} \end{multline*}

En $Y$, la topología será la generada por otra base $\tilde{\base}$. Para comprobar la continuidad, basta con comprobar los elementos de la base: que para todo elemento $B∈\tilde{\base}$, $\inv{h}(B)$ es abierto en la topología usual de $[0,1]$.

Vemos que la imagen de elementos de la base que son abiertos verticales, tanto dentro ($B_2$) como en el borde izquierdo ($B_1$), son el vacío. Ahora bien, la imagen inversa del conjunto $B_3$, en el borde derecho, es $\{1\}$ que no es abierto en la topología usual, de la misma forma que tampoco lo es la imagen inversa de $B_4$, que es un intervalo $[a,b)$ (ver imagen \ref{figH2Ej14})

\begin{figure}[hbtp]
\centering
\inputtikz{E_H2_14}
\caption{Conjuntos de la base en la topología del orden lexicográfico.}
\label{figH2Ej14}
\end{figure}


Hay que tener cuidado, eso sí, en distinguir entre la topología del orden lexicográfico en $[0,1]^2$ y la topología de subespacio del orden lexicográfico en $ℝ^2$. Por ejemplo, un intervalo vertical $(a,b)$ con $0 < a < 1$ y $b > 1$, es un abierto en $\topl_{Lex}$ de $ℝ^2$. Pero al hacer la intersección para la topología del subespacio, nos quedaría que el intervalo $(a, 1]$ es  un abierto, pero este intervalo no está en $\topl_{Lex\;[0,1]^2}$.

Una duda que surge: ¿es una de las topologías menos fina que la otra? Ya hemos visto que la topología del subespacio no está contenida en la del orden lexicográfico, así que faltaría comprobar si $\topl_{Lex\;[0,1]^2} ⊆ \topl_{Lex\; ℝ^2}^{sub}$. Para ello, nos bastaría comprobar si los elementos de la base de $\topl_{Lex\;[0,1]^2}$ son abiertos en $\topl_{Lex\; ℝ^2}^{sub}$. Es decir, hay que comprobar si \[ \base_{Lex\;[0,1]^2} ⊆ \topl_{Lex\;ℝ^2}^{sub} \]

La base del orden lexicográfico son los abiertos verticales (las bandas se generan como unión de intervalos verticales).

\end{problem}

\begin{problem}[6] Sea $X$ un espacio topológico y $A, D ⊂ X$. Demostrar que
\ppart[c] Si $\adh{A}∩\adh{D} = \emptyset$, entonces $\mop{Fr}(A∪D) = \mop{Fr}(A) ∪ \mop{Fr}(D)$
\ppart[e] $\intr{A} ∪ \intr{D} ⊂ \intr{A∪D}$.
\solution
\spart[c]

Si $\adh{A}∩\adh{D} = \emptyset$, entonces $\mop{Fr}(A∪D) = \mop{Fr}(A) ∪ \mop{Fr}(D)$. REcordamos que $\mop{Fr}(M) = \adh{M} ∩ \adh{M^c}$. Entonces

\[ \mop{Fr}(A∪D) = \adh{A∪D} ∩ \adh{\left(A∪D\right)^c} = (\adh{A}∪\adh{D}) ∩ \adh{(A^c∩D^c)} = \adh{A}∩(\adh{A^c∩D^c}) ∪ (\adh{D}∩ \adh{A^c∩D^c}) \]

Por simetría basta comprobar que $\mop{Fr}(A) = \adh{A} ∩ \adh{(A^c∩D^c)}$. El contenido hacia la izquierda es sencillo de demostrar

\[ A^c∩D^c ⊆ A^c \implies \adh{A^c∩D^c} ⊆ \adh{A^c}\implies \adh{A} ∩ \adh{A^c∩D^c} ⊆ \adh{A}∩ \adh{A^c} \]

Por el otro lado, si $x∈\adh{A} ∩ \adh{A^c}$, ¿tenemos que $x∈\adh{A} ∩ \adh{A^c∩ D^c}$?

Hay que comprobar que $x∈\adh{A^c∩D^c}$, o que $∀U$ abierto con $x∈U$, entonce $U∩A^c∩D^c≠\emptyset$.  Vemos que \[ U∩A^c∩D^c⊇ U ∩ A^c∩ \adh{D}^c ≠  \emptyset \], porque si $\adh{D}$ es cerrado, entonces $\adh{D}^c$ es abierto y entonces $U∩\adh{D}^c$ es abierto. Luego si $x∈\adh{A}$. tenemos que $x∉\adh{D}$ y $x∈\adh{D}^c$, luego $x∈U∩\adh{D}^c$, entonces $U∩\adh{D}^c$ es un entorno de $x$.

\spart[e] En la topología usual en $ℝ$, podemos escoger $A=ℚ$ y $D=ℝ\setminus ℚ$, por lo que el contenido puede ser estricto.
\end{problem}

\begin{problem}[8] Encontrar $A ⊂ ℝ$ tal que en la topología usual de $ℝ$ se tenga $\mop{Fr}(A) = \left\{ [1 , 2] ∪ \{ 0 \} ∪ \{1/n \tq n ∈ ℤ^+ \}\right\}$
\solution
\end{problem}

\begin{problem}[9] Indica razonadamente si estas afirmaciones son verdaderas o falsas.
\ppart[e] Si $A∩\mop{Fr}(A) = \emptyset$ entonces $A$ es abierto.
\solution
\spart[e] Cierto. Podemos hacerlo operando o por la definición. Si $A$ no fuese abierto, entonces $∃x∈A$ tal que $∀V$ entorno de $x$, $V\nsubseteq A$, es decir $V∩A^c ≠ \emptyset$. Como además $V ∩ A ≠ \emptyset$, entonces $x∈ \mop{Fr}(A)$, contradicción.
\end{problem}


\begin{problem}[12] Sea \stopl un espacio topológico y $W⊂X$ con la topología de subespacio.
\ppart Sea $D⊂W$, ¿cómo se relaciona la adherencia de $D$ en $W$ con la adherencia de $D$ en $X$?
\ppart Sea $\{x_n\}$ una sucesión en $W$ y $x∈W$. Demostrar que $x_n \to x$ en $W$si y sólo si $x_n \to x$ en $X$.

\solution
\spart Queda como ejercicio, pero parece que en la adherencia de $D$ en $X$ podría haber puntos que no estén en $W$.

\spart Si $x_n\to x$ en $X$, entonces $∀V∈\topl_X$ entorno de $x$ se tiene que $∃n_V$ tal que $∀n≥n_V,\; x_n∈V$.

Por otra parte, si $x_n\to x$ en $W$, entonces $∀V^W ∈ \topl_{W}^{sub}$ de $x$ en $W$ $∃n_W$ tal que $∀n≥ n_W\; x_n ∈ V^W$. Si $V^W$ es abierto en $W$, por definición $V^W = V∩W$ para un $V∈\topl_X$. Luego $x∈V$, por lo que $V$ es entorno de $x$ en $W$, por lo que $∃n_B\tlq ∀n≥n_v$ se tiene que $x_n∈V, x_n∈W$ por lo que $x_n ∈ V^W$.
\end{problem}

\begin{problem}[15] Probar que existen funciones de $(ℝ, \topl_{[,)})$ en $ℕ$ con la topología discreta que son sobreyectivas y continuas, pero que no existen funciones de $(ℝ, \topl_{[,)})$ en $ℝ$ con la topología discreta que tengan tales propiedades.
\solution

En el primer caso, vemos qué es la topología discreta en $ℕ$, que no es más que una topología $\topl_\base$ con $\base = \left\{ \{n\} \tq n∈ℕ\right\}$. $f$ será continua si y sólo si $\inv{f}(B) ∈ \topl_{[,)}\; ∀B∈\base$, y sobreyectiva si y sólo si $\inv{f}(B) ≠ \emptyset\; ∀B∈\base$. Además, los $\inv{f}(B)$ tienen que ser disjuntos, y por otra parte $\bigcup \inv{f}(B) = ℝ$.

Es decir, las imágenes inversas de los elementos de la base (esto es, de los naturales) nos van a dar una partición de $ℝ$ por los abiertos de $\topl_{[,)}$. Podemos encontrar entonces una función que los cumpla, como por ejemplo

\begin{align*}
\inv{f}(2k) &= [k-1, k)  \\
\inv{f}(2k-1) &= [-k, -k+1)
\end{align*}

Ahora bien, ¿qué ocurre en el segundo caso? No es cuestión de cardinales, en $ℝ$ tenemos ``espacio'' de sobra (podemos descomponer $ℝ$ en una familia no numerable de conjuntos no numberables).

Más bien, el problema tiene que ver con la numerabilidad. Si pudiese definir una función sobreyectiva de $ℝ$ con la topología $\topl_{[,)}$ a $ℝ$ con la topología discreta, entonces tendríamos imágenes inversas $\inv{f}(\{c\}) = A_c$, con $A_c$ disjuntos, no vacíos y abiertos en $\topl_{[,)}$, y por supuesto con $ℝ = \bigcup_{c∈ℝ} A_c$.

El conflicto está en encontrarlos disjuntos y que cubran todo $ℝ$. Al ser $A_c ≠ \emptyset$, entonces $∃[a_c, b_c) ⊆ A_c$. Es imposible tener $[a_c, b_c)$ disjuntos no vacíos con $c∈ℝ$.

Si $[a_c, b_c) ≠ \emptyset$, entonces $∃q_c∈ℚ$ con $q_c∈[a_c, b_c)$. Al ser los $[a_c, b_c)$ disjuntos, los $q_c$ son todos distintos, luego debería de haber una aplicación $\appl{ℝ}{ℚ}$ tal que $g(c) = q_c$ inyectiva, pero sería una contradicción porque entonecs $\card{ℚ} ≥ \card{ℝ}$.
\end{problem}


\begin{problem}[18] Sea $A=(-∞, 0] ∪ (2, +∞)$ y $\appl{f}{A}{ℝ}$ dada por \[ f(x) = \begin{cases}
-x^2 & x≤0 \\
x - 2 & x > 2 \end{cases} \]

Demostrar que $f$ es continua si $A$ tiene la topología del orden o la de subespacio, pero que sólo es un homeomorfismo con la del orden.
\solution

En ambos casos es continua ($f(0) = f(2)$). Para ver que es homeomorfismo, tenemos que ir un poco más allá.

Empezamos viendo que la topología del orden en $A$ no es más que la generada por una base \[ \base = \{ (a,b)_A \tq a<b,\; a,b ∈ A \}\]. Da la casualidad que $(5,7)_A = (5,7)_{ℝ}$, pero \[ (-1,3)_A = \{ x∈ A \tq -1 < x < 3\} = (-1, 0]_{ℝ} ∪ (2,3)_{ℝ} \]. Hay que tener cuidado con eso.

Por otra parte, la topología del subespacio es $\topl_{A}^{sub} = \{ V ∩ A \tq V ∈ \topl_ℝ \}$. Podemos encontrar un $W∈\topl_A^{sub}$ tal que $W∉\topl_{ord_A}$, como por ejemplo \[ W = (-1,1) ∩ A = (-1, 0]_ℝ \]

\end{problem}


\begin{problem}[0]
¿Cuál es la topología producto de $ℝ_{disc} × ℝ_{usual}$?
\solution

¿Cómo construimos la topología producto? Nos basta con coger los elementos de la base de ambos. Como recordatorio:

\[ \topl_{\base_1} \otimes \topl_{\base_2} = \topl_{\base_1 \otimes \base_2} \] donde $\base_1 \otimes \base_2 = \left\{ B_1 × B_2 \tq B_1 ∈ \base_1, B_2 \base_2 \right\}$.

La base $\base_1$ de la topología discreta es $\base_1=\left\{\{x\} \tq x∈ℝ\right\}$, y la base $\base_2$ de la topología usual son los abiertos $(a,b)$. Entonces $B_1 × B_2 = \{x\} × (a,b)$, intervalos abiertos verticales. Casualmente, esta es la topología $\topl_{Lex}$ del orden lexicográfico en $ℝ^2$.
\end{problem}

\begin{problem}[0] Sea $\appl{f}{ℝ}{ℝ}$ continua y biyectiva. ¿Es un homeomorfismo?
\solution

Lo único que le falta para ser homeomorfismo es que la inversa sea continua. Si $\inv{f}$ es continua, entonces la imagen inversa por $\inv{f}$ de un abierto es abierta. Es decir, hay que demostrar que $f$ de un abierto es abierto. Y como tenemos una base, nos basta comprobar sólo para los elementos de la base.

Una función $f$ continua e inyectiva en $ℝ$ debe ser monótona estrictamente creciente o decreciente. Supongamos que es creciente, entonces $f((a,b)) = (f(a), f(b))$.

\end{problem}

\section{Hoja 3}

\subsection{Funciones continuas, espacios de Hausdorff y topología producto}

\begin{problem}[2] En el espacio producto $\topl_{[,)} \otimes \topl_{[,)}$ en $ℝ^2$, describir la topología inducida en los subconjuntos \begin{gather*}
X = \{ (x, -x) \tq x∈ ℝ\} \\
Y = \{ (x, x) \tq x∈ ℝ \}
\end{gather*}
\solution

Estudiamos primero la topología producto, que podemos expresar como la generada por una base \[ \base = \{ [a_1, b_1) × [a_2,b_2) \tq a_i < b_i \} \]. Es decir, los elementos de la base son rectángulos en los que los bordes inferior e izquierdo (salvo los vértices) están incluidos.

\begin{figure}[hbtp]
\centering
\inputtikz{E_H3_2}
\caption{Topología del subespacio (en rojo) para $X$ e $Y$.}
\label{figEH32}
\end{figure}

Empezamos, por llevar la contraria, con $Y$. En ese caso, los abiertos serán los intervalos $[\vec{c},\vec{d})$ con $\vec{c} = (c_1, c_2)$ y $\vec{d} = (d_1, d_2)$ tales que $d_1 - c_1 = d_2 - c_2$. Y en el caso de $X$, serán los puntos (la topología discreta).
\end{problem}

Un comentario del profesor para los ejercicios 8, 9 y 10.
\begin{prop} Si $\appl{f}{X}{Y}$ es continua e $Y$ es Hausdorff (\ref{defHausdorff}) entonces \[ G_f = \{ (x, f(x))\tq x∈X \} \] es cerrado en $X×Y$.\label{propHausdorffAppl}
\end{prop}

\begin{proof}

\begin{wrapfigure}{r}{0.3\textwidth}
\centering
\inputtikz{E_H3_PropI}
\caption{Conjunto $G_f ⊆ X×Y$, el grafo de una función.}
\label{figE_H3_PropI}
\end{wrapfigure}

Queremos ver que $G_f^c$ es abierto. Para ello, queremos encontrar un entorno para todo punto $(x,y)∉G_f$ que no corte a $G_f$ (es decir, $y≠f(x)$). Como estamos en la topología producto, necesitamos encontrar simplemente abiertos de la base.

Al ser $Y$ Hausdorff, tenemos que $∃V_1, V_2$ abiertos en $Y$ tales que $f(x) ∈ V_1$, $y∈V_2$ y $V_1 ∩ V_2 = \emptyset$. Definimos ahora \[ U ≝ \inv{f}(V_1)\]. $U$ es abierto en $X$ por ser $f$ continua, y además $x∈U$.

Entonces afirmamos que $(U×V_2)∩G_f = \emptyset$. La razón es que si $(\tilde{x}, f(\tilde{x})) ∈ U × V_2$, entonces $f(\tilde{x})$ pertenece a $V_2$ y también a $f(U)⊆V_1$. Entonces tendría $V_1 ∩ V_2 ≠\emptyset$, lo que sería una contradicción.

Sabemos que $(x,y) ∈ U × V_2 ⊆ G_f^c$, y por tanto $G_f^c$ es abierto en $X×Y$.
\end{proof}

\begin{problem}[4]Se considera la topología $\topl_\base$ en $ℝ^2$ generada por la base $\base$ del ejercicio 9 de la hoja 1: Para cada punto $(x, y )$ de $ℝ^2$ y cada $r ∈ ℝ$ con $r > 0$ se considera el siguiente conjunto $B((x, y), r)$: el cuadrado con lados paralelos a los ejes, centrado en $(x, y)$ y de lado $2r$ , del que se ha excluido los lados y los puntos de las diagonales que no sean el punto $(x, y)$, de tal forma que  \[ \base = \{ B((x, y), r) : (x, y) ∈ ℝ^2 , r > 0\} \]

¿Existen topologías en $ℝ$ de modo que su producto coincida con la topología $\topl_\base$?

Indicación: Prueba que ambas topologías deben ser menos finas que la usual.
\solution

No puede darse el caso de que $\topl_\base$ sea una topología producto de $\topl_1\otimes\topl_2$ en $ℝ$. Si lo fuese, se podría obtener una base de $\topl_1$ como $p_1(\base)$ para la topología usual. Es decir, $\topl_1$ tendría que ser $\topl_{usual}$. Por simetría, entonces $\topl_2 = \topl_{usual}$, y entonces $\topl_1 \otimes \topl_2 = \topl_{usual}^{ℝ^2}$, cosa que es imposible porque está claro que $\topl_\base ≠ \topl_{usual}$.
\end{problem}

\begin{problem}[5] Sean $X$ e $Y$ dos conjuntos no vacíos. Sea $\topl$ la topología producto en $X × Y$ construida a partir de las topologías $\topl_1$ de $X$ y $\topl_2$ de $Y$ . Prueba que si $\base$ es una base de $\topl$ (no necesariamente la “base producto”) entonces $p_1(\base) = \{ p_1(B) \tq B ∈\base \}$ es base de $\topl_1$ y $p_2(\base) = \{ p_2 (B) \tq B ∈\base\}$ es base de $\topl_2$ . ¿Se puede usar este hecho para resolver el ejercicio anterior?

\solution

Sabemos que $\base$ es una base para $\topl$. Entonces \[ \base_1 ≝ \{ p_1(B) \tq B ∈ \base \}\] es base de $\topl_1$. Es obvio que como $p_1$ es abierta, si $B∈\base$ entonces $p_1(B) ∈ \topl_1$.

Además, si $A∈\topl_1$, entonces \[ A = \bigcup_{j∈J} B^j_1 \] con $B^j_1 ∈ \base_1$. Queremos pasar de la topología producto a la proyección, así que consideramos \[ A × Y ∈ \topl_1 \otimes \topl_2\; p_1(A×Y) = A \]. Por otra parte, podemos expresar $A×Y$ como unión de elementos de la base $\base$: \[ A×Y = \bigcup_{j∈J} B^j\] con $B^j∈\base$. Y como la imagen de la unión es la unión de imágenes, tenemos que \[ p_1(A×Y) = \bigcup_{j∈J} p_1(B^j) = \bigcup_{j∈J} B_1^j \] con $B_1^j∈\base_1$.

\end{problem}

\begin{problem}[7]
\ppart Demostrar que un espacio topológico \tops es un espacio de Hausdorff si y sólo si la diagonal $Δ=\{(x,x) \tq x ∈ X\}$ es un cerrado en el espacio topológico $X×X$.
\ppart Demuestra que si \stdf es continua e $Y$ es un espacio de Hausdorff, entonces \[ K = \{ (x_1, x_2) \tq f(x_1) = f(x_2) \}\] es cerrado en $X × X$.
\solution
\spart La implicación a la derecha se puede hacer usando la proposición \ref{propHausdorffAppl}: si $X$ es Hausdorff, entonces la identidad $\appl{I}{X}{X}$ es una función continua que podemos usar para generar el conjunto \[ G_I = \{ (x,I(x)) \tq x∈X \} = Δ \], que según la proposición es cerrado en $X×X$.

\begin{figure}[hbtp]
\centering
\inputtikz{E_H3_E7b}
\caption{$A$ y $B$ son abiertos disjuntos, así que podemos encontrar entornos de los dos puntos y hacer su proyeccción para demostrar que $X$ es Hausdorff.}
\label{figH3_7b}
\end{figure}

Vamos ahora al otro lado: tenemos que $Δ$ es cerrado en $X×X$ y queremos demostrar que $X$ es Hausdorff, esto es, que $∀x_1, x_2$ distintos existen dos entornos disjuntos $V_1, V_2$. Consideramos los puntos $\va = (x_1, x_2),\, \vb = (x_2, x_1)$, que son distintos; y los subconjuntos de $X×X$ $A=\{ (x,y) \tq x > y\}$ y $B=\{(x,y) \tq x < y\}$. Está claro que ambos son abiertos, que $A∩B = ∅$ y que $A∪B = Δ^c$.

Si suponemos sin pérdida de generalidad que $x_1 > x_2$, entonces $\va ∈ A$ y $\va ∈ B$, luego podemos encontrar entornos $E_a ⊆ A$ y $E_b ⊆ B$ de $\va$ y $\vb$ respectivamente, que por fuerza han de ser disjuntos. Entonces podemos construir los entornos $V_1, V_2$ como $V_1 = p_1(E_b),\, V_2=p_1(E_a)$.

¿Podría darse el caso de que $p_1(E_b) ∩ p_1(E_a) ≠ ∅$?\footnote{Duda planteada por Jorge} No lo sabemos. En todo caso, si eso ocurriese podríamos tomar las proyecciones por $p_2$, que ya por fuerza sí tienen que ser disjuntas (si no lo fuesen, la intersección de $E_a$ o $E_b$ con Δ no sería vacía, contradicción).

En cualquier caso, podemos construir los intervalos $V_1, V_2$ que buscamos, disjuntos, para cualqueir par $x_1, x_2$, luego $X×X$ es Hausdorff.

\spart Si $p ∈ K^c$, entonces $f(p_1) ≠ f(p_2)$. Entonces existen dos entornos $V_1, V_2$ en $Y$ tales que $f(p_i) ∈ V_i$. Entonces $W_j = \inv{f}(V_j)$ es abierto en $X$, y $W_1×W_2 ⊆ K^c$, ya que $∀x_j ∈ W_j \implies f(x_j) ∈ V_j$ y entonces $f(x_1) ≠ f(x_2)$ al ser $V_1, V_2$ disjuntos.
\end{problem}

\begin{problem}[9] Demostrar que si $f$ y $g$ son funciones continuas definidas de $X$ en $Y$ siendo $Y$ un espacio de Hausdorff, entonces el conjunto $C= \{ x \tq f(x) = g(x) \}$ es cerrado en $X$.

\solution

\begin{figure}[hbtp]
\centering
\inputtikz{E_H3_E9}
\caption{Ejercicio 9.}
\label{figH3_E9}
\end{figure}

Comprobamos si $C^c$ es abierto. Cogemos $x_0∈C^c$, entonces $f(x_0) ≠ g(x_0)$, luego existen $V_f, V_g$ aiertos tales que $f(x_0) ∈ V_f,\,g(x_0) ∈ V_g$ y $V_f ∩ V_g=∅$ (se puede ver un dibujo en la figura \ref{figH3_E9}).

Hay que encontrar ahora un entorno $U$ de $x_0$ tal que $f(x) ≠ g(x)\; ∀x∈U$. Ese entorno es \[ U = \inv{f}(V_f) ∩ \inv{g}(V_g) \].
\end{problem}

\begin{problem}[13]
\ppart
\ppart Sea $Y$ el subespacio $(ℝ^+ × ℝ) ∪ (ℝ × \{0\})$ de $ℝ×ℝ$. Sea $h$ la restricción de $p_1$ a $Y$. Demostrar que la aplicación $h$ no es ni abierta ni cerrada.

Indicación : $\inv{h}(U) ∩ (ℝ × \{0\}) = U × 0$.

\solution
\spart

\spart

\begin{figure}[hbtp]
\centering
\inputtikz{E_H3_E13}
\caption{Esquema de un abierto en $Y$ (sombreado rojo) y su imagen por $h$.}
\label{figH3_13}
\end{figure}

Tenemos que darnos cuenta de que restringir una aplicación abierta no tiene por qué darnos una aplicación abierta de nuevo.

Vamos a ver cómo son los abiertos en $Y$ (figura \ref{figH3_13}). cogemos un abierto $A⊆ℝ^2$ del total y cortamos con $Y$, y nos queda un conjunto cerrado por la izquierda. Entonces, $h(A∩Y)$ es un intervalo cerrado por la izquierda.

\end{problem}

\section{Hoja 4}

\subsection{Espacios conexos y conexos por caminos}

\begin{problem}[6] Sean $A$ y $D$ dos conjuntos cerrados no vacíos de un espacio topológico $X$. Demuestra que si $A∪D$ y $A∩D$ son conexos entonces $A$ y $D$ también lo son. ¿Qué pasa si $A$ ó $D$ no son cerrados?
\solution
Nos dan una idea: si $A$ no es conexo, podría descomponer $A∪D$. Podemos suponer $X=A∪D$. Si $A$ no es conexo, podríamos escribirlo como $A=F_1∪F_2$ con $F_1, F_2$ cerrados en $A$ y, como $A$ es cerrado, cerrados también en $X$; disjuntos y no vacíos. Entonces si $A∩D$ es conexo, se tiene que $A∩D⊆F_1$ o bien $A∩D⊆F_2$. Supomgamos sin pérdida de generalidad el segundo caso. Entonces $A∪D= F_1 ∪ F_1^c$ ($F_1^c = D∪F_2$).
\end{problem}

\begin{problem}[9] Sea \stdf una función continua y sobreyectiva de un espacio topológico $X$ sobre un espacio topológico $Y$ que tiene $n$ componentes conexas. Prueba que $X$ tiene como mínimo $n$ componentes conexas.
\solution
Sabemos que la imagen de una componente conexa de $X$ tiene que estar contenida en una componente conexa de $Y$. Luego si $C_i^X$ son las componentes conexas de $X$ y $C_i^Y$ las de $Y$, tenemos que \[ f(C_i^X)⊆C_{α(i)}^Y \], donde $\appl{α}{\{1,\dotsc,m\}}{\{1,\dotsc,n\}}$ es una aplicación sobreyectiva que lleva el índice de una componente conexa en $X$ al de la componente correspondiente en $Y$.

Sea $y∈C_k^Y$, entonces $x∈\inv{f}(y)$. No sé cómo lo ha escrito pero la idea básica es que como la aplicación es sobreyectiva, todas las componentes conexas de $Y$ deben de ser imagen de al menos una componente conexa de $X$, y entonces es bastante fácil.
\end{problem}

\begin{problem}[10]
Demuestra que $A=(ℝ ×\{0\}) ∪ (\{0\}× ℝ)$ no es homeomorfo a $ℝ$.
\solution
Si fuesen homeomorfos, entonces la restricción \[g=\appl{f}{A\setminus\{(0,0)\}}{ℝ\setminus\{f((0,0))\}}\] sería también homemomorfismo, luego la imagen de una componente conexa tendría que ser componente conexa. Sin embargo, eso es imposible ya que en el dominio tenemos cuatro componentes conexas y en la imagen sólo dos.

\end{problem}
\begin{problem}[13] Demuestra que todo subconjunto conexo de $ℝ^n$ con más de un punto es no numerable.
\solution

Sea $A$ conexo y con $\card{A}≥2$. Entonces $∃x,y∈A\tq x≠y$, dos puntos distintos. Sea $r=\dst(x,y) > 0$, luego $∀ε$ con $0<ε<r$, entonces $∂\bola(x,ε) ∩ A ≠ ∅$, donde $∂\bola$ es la frontera o borde de la bola. Si fuese vacío, entonces podríamos escribir \[ A = (\bola(x,ε) ∩ A ) ∪ \left(\adh{\bola}^c (x,ε) ∩ A\right) \], unión de dos conjuntos distintos, abiertos de $A$, y entonces $A$ no sería conexo.

Esto quiere decir que podemos construir una aplicación $\appl{ξ}{(0,r)}{A}$ tal que a todo $ε∈(0,r)$ le asigne $z_ε∈A$ tal que $\dst(x,z_ε) = ε$. Esta aplicación es inyectiva, luego $\card{A} ≥ \card{(0,r)} = \card{ℝ}$, y entonces $A$ no es numerable.
\end{problem}

\begin{problem}[14]
\ppart Probar que un espacio $X$ es conexo si y sólo si no existe ninguna aplicación continua y sobreyectiva $\appl{f}{X}{Y}$ donde $Y = \{0,1\}$ con la topología discreta.
\ppart Usar el apartado anterior para probar que si $S$ es un subconjunto conexo de un espacio $X$ y $K$ satisface $S ⊂ K ⊂ \adh{S}$ entonces $K$ es conexo.

\solution
\spart
\spart Idea: si $K$ no es conexo, entonces $∃\appl{f}{K}{\{0,1\}}$ como en el apartado anterior. Si $S$ es conexo, tiene que ser que $f(S) = \{p\}$ con $p=0$ ó $p=1$, ya que los únicos conexos en la topología discreta son los puntos. Consideremos ahora $\adh{S}^K = \adh{S}∩K = K$, ¿cuál es su imagen por $f$?
\end{problem}

\begin{problem}[16] En el plano con la topología usual, sea \[ S = \left\{ (r \cos t,r \sin t ) \tq r = 1 − \frac{1}{t} , t ≥ 1 \right\} \]. Probar que $X = S ∪ \mathbb{S}^1$ es conexo pero no es conexo por caminos, donde $\mathbb{S}^1$ es la circunferencia de radio 1.
\solution

Queremos demostrar que es conexo, y para ello queremos demostrar que $\mathbb{S}^1 ⊆ \adh{S}$. Es fácil construir una sucesión para un $t$ fijo: \[ \va_n = \left(1 - \frac{1}{t_n}\right)\left(\cos(t + 2πn), \sin(t + 2πn)\right)\] donde $t+2πn = t_n$. Entonces la sucesión de $\va_n$ tiende a $(\cos t, \sin t)∈\mathbb{S}^1$ cuando $n\to∞$, entonces $\mathbb{S}^1 ⊆\adh{S}$.

La parte de demostrar que no es cpc no la hacemos, pero implicaría ver que no podemos encontrar un camino desde el primer punto de la espiral (por ejemplo) hasta un punto de la circunferencia.
\end{problem}

\begin{problem}[20] Demostrar que si $X$ e $Y$ son conexos y $A, B$ son subconjuntos propios no vacíos de $X$ e $Y$ respectivamente entonces $X × Y \setminus A × B$ es conexo. En la situación anterior, ¿es cierto que si $X$ e $Y$ son conexos por caminos entonces $X × Y \setminus A × B$ también lo es?
\solution
\begin{figure}[hbtp]
\centering
\inputtikz{E_H4_E20}
\caption{Demostración visual de que $X×Y \setminus A×B$ es cpc.}
\label{figH4_E20}
\end{figure}

Queremos poder conectar dos puntos cualesquiera $\va = (x_1, y_1)$ y $\vb = (x_2, y_2)$. Podemos construir dos aplicaciones $\appl{φ_1, φ_2}{[0,1]}{X × Y \setminus A × B}$ tales que $φ_1(0)=(x_1, y_1)$, $φ_1(1) = (x_1, y_2)$ y $φ_2(0) = (x_1, y_2), φ_2(1) = (x_2, y_2)$ continuas, ya que siempre podemos encontrar una aplicación continua entre $y_1$ y $y_2$ y $x_1$ y $x_2$ porque tanto $Y$ como $X$ son conexos por caminos.

Entonces $φ_1\ast φ_2$ es continua y conecta $\va$ con $\vb$. Hay que ser un poquillo riguroso con la ordenación para que no pase la aplicación por $A×B$ pero siempre se puede encontrar.

Demostración más rigurosa. Podemos considerar $y_2∉B, x_1∉A$ y el conjunto $D_1 = \{x_1\} × Y$, que es conexo por ser homeomeorfo a $X_2$. De la misma manera, podemos coger $D_2 = X ×\{y_2\}$, igualmente conexo. Está claro que $D_1 ∪ D_2$ es conexo (la intersección es no vacía).

Sea $D_0 = D_1 ∪ D_2 ⊆ X × Y \setminus A × B$. Entonces podemos escribir $X × Y \setminus A × B$ como unión de los conjuntos $D_0$ y $D_c = \{c\}×Y$ y $D_r =X × \{ r\}$ como

\[ X × Y \setminus A × B  = D_0 ∪ \left(\bigcup_{r∉B} D_r\right) ∪ \left(\bigcup_{c∉A} D_c\right) \]

La intersección de cada $D_c$ y $D_r$ con $D_0$ es no vacía, luego estamos en la situación de la proposición \ref{propUnionConexa} y entonces $ X × Y \setminus A × B$ es conexo.
\end{problem}

\begin{problem}[26] Sea $\mathbb{S}^1 = \{(x,y) \tq x^2 + y^2 = 1 \}$ con la topología usual heredada del plano y sea $Y$ el intervalo $[a,b]$ también con la topología usual. Supongamos que $\appl{f}{\mathbb{S}^1}{Y}$ es continua y sobreyectiva. Probar que, para cada $c ∈ (a,b)$ , el conjunto $\inv{f}(c)$ contiene más de un punto.
\solution

\end{problem}


\section{Hoja 5}

\subsection{Compacidad}

\begin{problem}[1] Decide si los siguientes conjuntos son compactos
\ppart $\{(-1)^n + \frac{1}{n} \tq n \in \nat\} \subset \real$
\ppart Los racionales: $ℚ$.
\ppart El intervalo $[0,1]$.
\ppart El segmento $[0,1]×\{1\}$ en $\topl_{Lex}$.

\solution

\spart Podemos dividirlo en dos conjuntos, tomando por un lado los puntos generados con un n par y por otro los impares.

Así, cada conjunto constituiría una sucesión que converge a $\pm$1 respectivamente.

Por tratarse de una sucesión sabemos que es compacto y la unión de compactos es compacto.

\spart $ℚ$ no es compacto. Tenemos 2 formas de demostrarlo:

1) Como $(ℝ,\topl_{usu})$, es Hausdorff si $ℚ$ fuera compacto, entonces sería cerrado en $(ℝ,\topl_{usu})$, pero $\bar{ℚ} = ℝ ≠ ℚ \implies $ NO es cerrado.

2) La otra posibilidad es construir un recubrimiento del que no se puede extraer un subrecubrimiento finito.

\spart No es compacto. Si tomamos \[ [0,1) = \bigcup_{n=1}^∞ \left[0, 1 - \frac{1}{n}\right)\], no podemos coger un SRF luego no es compacto. Sería trivial añadir un abierto cualquiera para cubrir el 1 y tener un recubrimiento abierto de $[0,1]$ sin SRF.

\spart No lo es. Tomamos $R_a ≝ \{a\} × ℝ$, abierto en $\topl_{Lex}$. Entonces el recubrimiento $\{ R_a \tq 0 ≤ a ≤ 1 \}$ es un recubrimiento abierto de $W$ sin SRF. De hecho, no hay ningún recubrimiento más pequeño que eso.
\end{problem}

\begin{problem}[8] Sean $X_1 = \set{(x,y) ∈ ℝ^2 \tq x^2 + y^2 < 1}$, $X_2 = \set{(x,y) ∈ ℝ^2 \tq x^2 + y^2 ≤ 1}$ . Demostrar que $X_1$ es homeomorfo a $R_2$ y que $X_1$ y $X_2$ no son homeomorfos.
\solution

Vamos a demostrar esto usando el grupo fundamental, por risas y diversión\footnote{Podríamos hacerlo viendo que uno es compacto y el otro no, pero eso es aburrido.}


Desde luego, los grupos fundamentales de ambos son el mismo, el trivial, por ser convexo. Pero si quitamos un punto del borde en $X_2$, en $X_1$ se quitará un punto del interior, y entonces en $X_1$ el grupo fundamental sería el grupo cíclico mientras que en $X_2$ seguiría siendo el trivial.

\end{problem}

\begin{problem}[11] Demostrar que si $Y$ es compacto entonces $\appl{p_1}{X×Y}{Y}$ es cerrada.

\textit{Indicación}: Si $A$ es cerrado y $x∉p_1(A)$, hallamos un ``tubo'' $T = U_x × Y$ tal que $T∩A = ∅$. Dar un ejemplo de un conjunto no compacto en $ℝ^2$ cuyas proyecciones sean compactas.

\label{ejH5E11}
\solution

Queremos ver que $Y$ compacto implica que la proyección $\appl{p_1}{X×Y}{X}$ es cerrada. $∀y ∈ Y$, $(x,y) ∈ \set{x} × Y$ existe un entorno de $(x,y)$, digamos $U_x × V_y$ que no toca a $A$ por ser este cerrado.

Entonces, el conjunto $\set{V_y}_{y∈Y}$ es un recubrimiento abierto de $Y$. Por ser $Y$ compacto, podemos extraer un recubrimiento finito. Sea \[ U_x = U_x^{(y_1)} ∩ U_x^{(y_2}) ∩ \dotsb ∩ U_x^{(y_n)} \]

Como $U_x ⊆ U_x^{(y_j)}$ entonces $U_x × V_y ⊆ U_x^{(y_1)} × V_{y_i}$ y entonces y entonces y entonces

\[ U_x × Y ⊆ \bigcup_{j=1}^n U_x × V_{y_j} ⊆ U_{j=1}^n U_x^{(y_j)} × V_{(y_j)} \] y entonces el ``tubo'' $T_x$ tiene intersección vacía con $A$ y entonces $U_x ∩ p_1(A) = ∅$, luego $x∈\adh{p_1(A)}$ y entonces $p_1(A)$ es cerrado.\footnote{Esto igual está fatal escrito y se me han escapado cosas por todas partes.}

\end{problem}

\begin{problem}[12]
Sea $X$ un espacio topológico e $Y$ un espacio de Hausdorff compacto. Probar que $\appl{f}{X}{Y}$ es continua si y sólo si la gráfica de $f$, $Γ_f$ , es cerrada en $X × Y$. Si $X$ es también un espacio de Hausdorff compacto, entonces $f$ es continua si y sólo si $Γ_f$ es compacta.
\solution

Empezamos con la implicación a la derecha. Queremos ver si dado un $(x,y) ∉ Γ_f$, existe un entorno $A$ de $(x,y)$ con $A∩Γ_f = ∅$. Y esto es lo mismo que tratar de demostrar que $(x,y)∉ \adh{Γ_f}$.

Como estamos en la topología producto, entonces $A = U_x × V_y$, producto de entornos básicos en $X$ e $Y$. Si $(x,y) ∉ Γ_f$, entonces $y ≠ f(x)$.

Por ser $Y$ Hausdorff, entonces $∃V_y$ entorno de $y$ y $V_{f(x)}$ entorno de $f(x)$ tales que $V_y ∩ V_{f(x)} = ∅$.

Sea $U_x = \inv{f}(V_{f(x)})$, como $f$ es continua entonces $U_x$ es abierto y como $f(x) ∈ V_{f(x)}$ entonces $x∈U_x$, es decir $U_x$ es entorno de $x$. Luego como $f(U_x) ⊆ V_{f(x)}$, entonces $f(U_x) ∩ V_y = ∅$. Afirmamos entonces que $U_x × V_y ∩ Γ_f = ∅$, ya que si $z∈U_x$ entonces $f(z) ∈ V_{f(x)}$, por lo que $f(z) ∉ V_y$ y entonces $(z, f(z)) ∉ U_x × V_y$.

Vamos ahora con la implicación al otro lado.

Como indicación para verla, vamos a usar esta igualdad:

\[ \inv{f}(C) = p_1 \left((X×C) ∩ Γ_f\right)\]

Como $C$ es cerrado en $Y$, entonces $X×C$ es cerrado en $X×Y$. Y como $Γ_f$ es cerrado, entonces $(X×C) ∩ Γ_f$ es cerrado. Sabemos que la proyección es continua (demostrada en ejercicio \ref{ejH5E11}), así que entonces $ p_1 \left((X×C) ∩ Γ_f\right) = \inv{f}(C)$ es cerrado y ya tenemos la demostración de continuidad.
\end{problem}

\begin{problem}[13]
\ppart  $([0,1]×[0,1],\topl_{lex})$ ¿compacto?
\ppart  $([0,1)×[0,1],\topl_{lex})$ ¿compacto?

\ppart$[0,1]×[0,1]$ como subespacio de $(ℝ^2,\topl_{lex})$ ¿es compacto?
\solution
\spart $([0,1]×[0,1],\topl_{lex})$ es compacto. Se puede hacer de manera simiar a $ℝ$.

Si no hay un SRF (subrecubrimiento finito) en todo el conjunto, si dividimos por la mitad el cuadrado, entonces en alguno de los 2 no puede haber un SRF.  Seguimos dividiendo por la mitad y tomamos la intersección de las mitades de esos conjuntos, que es una línea vertical, homeomorfa a $[0,1]$ que es compacto.

Como la línea vertical es compacta, existe un SRF para la línea, que tiene que cubrir también una banda, que es contradictorio con el razonamiento anterior.


Veremos en teoría una demostración más general.

\spart No, porque no llegamos a coger los puntos (1,1) y (1,0), construyendo $$\bigcup_{k=1}^n [(0,0), \left(1-\frac{1}{n_k},0\right)]$$


\spart
No, el contraejemplo es  una línea vertical.

En $ℝ^2$ la vertical es abierta. Entonces podemos definir el recubrimiento: $A_a = \{(a,y): y∈ℝ\}∈\topl_{lex}$.

$$[0,1]×[0,1] \subseteq \bigcup_{0≤a≤1} A_a$$ y no exsite un subrecubrimiento finito, ya que los $A_a$ son disjuntos.

\end{problem}

\begin{problem}[15] Sea $X$ un espacio compacto.

\ppart Sea $F$ una familia de funciones continuas de $X$ en $[0 , 1]$ tales que si $f,g ∈F$ entonces $f·g ∈F$, y para cada $x ∈ X$ existe $f∈F$ y un entorno $U_x$ con $f (U_x) = 0$ . Probar que $F$ contiene a la función nula.

\ppart Sea $F$ una familia de funciones continuas de $X$ en $ℝ^+$ tales que si $f,g ∈F$ entonces existe $h ∈F$ con $h ≤ \min \set{f,g}$, y para todo $x ∈ X$, $\inf\set{f(x) \tq f ∈F} = 0$. Demuestra que para todo $ε > 0$ , existe $f ∈F$ tal que $f (x ) < ε$ para todo $x ∈ X$.
\solution

\spart Si cogemos todos los entornos para cada $x∈X$, tenemos un recubrimiento de $X$. Por ser este compacto, podemos extraer un SRF, esto es, existe un conjunto finito $J ⊂ X$ tal que $\set{U_x}_{x∈J}$ es un SRF. Consideramos entonces el conjunto $A$ de las funciones que se anulan en un entorno de $x$ para cada $x∈J$. El producto de todas estas funciones será la función nula, ya que para cada punto hay al menos una que vale cero. Por ser producto finito, tendremos además que está en $F$.

\spart Para cada $x∈X$, tomamos $f_x$ tal que $f_x(x) < ε$, que sabemos que existe pues el ínfimo de $F$ para cada punto es 0. Consideramos entonces $V = (0,ε) ⊆ ℝ^+$. Entonces $∀x∈X$y por ser cada $f_x$ continua, existe un abierto $U_x ⊆ X$ entorno de $x$ tal que $f(U_x) ⊆ V$.

El conjunto $\set{U_x}_{x∈X}$ es un recubrimiento de $X$, y como éste es compacto tenemos que existe un $J⊂X$ finito tal que $\set{U_x}_{x∈J}$ es un SRF. Además, $\set{f_x(U_x)}_{x∈J} ⊆ V = (0,ε)$ por cómo hemos escogido las $f_x$ (continuas) y $U_x$.

Por hipótesis, dado que tenemos un conjunto finito de funciones, existe una $h ≤ \min \set{f_x}_{x∈J}$, es decir, que $h(x) < ε\; ∀x∈X$, y esta es la función que buscábamos.

\end{problem}

\begin{problem}[16] Demostrar que los conjuntos compactos en la recta de Sorgenfrey $(X, \topl_{[,)})$ son necesariamente numerables.

Sugerencia: usar el hecho de que en un conjunto no numerable hay siempre una sucesión estrictamente creciente.
\solution

Tenemos que demostrar que si $W$ es infinito no numerable, entonces no es compacto.

Vamos a tratar de buscar un recubrimiento a partir de sucesiones infinitas y usar eso para ver que no hay subrecubrimiento finito. Sea $\{ x_n\}_{n∈ℕ} ⊆ W$ una sucesión estrictamente creciente ($∀n ∈ ℕ\, x_n < x_{n+1}$).

Consideramos entonces los siguientes intervalos $[x_1, x_2), [x_2, x_3), \dotsc$. Pueden darse dos casos: si $\sup x_n = ∞$, entonces el recubrimiento es infinito y no hay ninguna forma de sacar un subrecubrimiento finito de esa unión infinita.

Pero, ¿y si $\sup x_n = x < ∞$? En ese caso añadimos algo y no sé qué más. \footnote{Completing necesiting.}

\end{problem}

\begin{problem}[17] Sea $X$ Hausdorff y $K_1 ⊇K_2 ⊇ K_3 ⊇ \dotsb $ compactos no vacíos. Entonces demuestra que $\bigcap_{n∈ℕ} K_n ≠ ∅$.
\solution

Si $X$ es Hausdorff y $K_i$ es compacto, entonces $K_i$ es cerrado también. Además, la sucesión $K_2, K_3, \dotsc$ son subconjuntos cerrados de $K_1$ con la propiedad de intersección finita.  Es decir, \[ K_{n_1} ∩ K_{n_2} ∩ \dotsb ∩ K_{n_m}  = K_{\max n_k} ≠ ∅\]

Además, por ser $K_1$ compacto, entonces $\bigcap_{n=2}^∞ K_n ≠ ∅$, y vamos a ver por qué.

Tomamos $A_i = X \setminus K_i$, que es abierto en $X$ por ser $K_i$ cerrado. Si la intersección de los $K_n$ fuese vacía, entonces la unión de todos los $A_i$ sería el total $X$. Entonces tendríamos un recubrimiento abierto de $K_1$ sin SRF: si tomásemos un SRF $A_1 ∪ A:2 ∪ \dotsb ∪ A_m = A_m$, el conjunto $K_m$ no estaría en ese conjunto, luego no sería un SRF de $K_1$.
\end{problem}

\subsection{Axiomas de numerabilidad y espacios separables}

\begin{problem}[18] Si un espacio es IAN con cierta topología \stopl, ¿lo es necesariamente con una menos fina?
\solution

No, no lo es. Podemos tomar $\topl = \topl_{disc}$, que es IAN ($\{x\}$ es una base de entornos de $x$).  Basta tomar entonces una topología cualquiera $(X, \topl')$ que no sea IAN.

Pero, ¿y si $\topl'$ es más fina que $\topl$? Supongamos que tenemos $\{V_n(x)\}$, una base de entornos de $x$ en $\topl$. Entonces $V_n(x)$ es entorno de $x$ en $\topl'$.

Pero para que $\{V_n(x)\}$ sea base de entornos (ver definición \ref{defBaseEntornos}), tiene que cumplirse que $∀A ∈ \topl'$ abierto, exista un $n$ tal que $V_n(x) ⊆ A$. En concreto, tiene que que cumplirse también para los abiertos en $\topl' \setminus \topl$. Podemos añadir entonces abiertos raros (por ejemplo, $x$ y un intervalo abierto $(a,b)$ con $b > a > 100$) y entonces ningún elemento de la base anterior contendría a estos nuevos monstruos interválicos.\footnote{Hay que explicar y formalizar mejor esto.}

\end{problem}

\begin{problem}[19] Se considera el espacio topológico $(X, \topl_{cofinita})$. ¿Es un espacio separable?
\solution

Recordemos la definición de topología cofinita (\ref{defTopCofinita}): $A$ es abierto si es el vacío o si $X \setminus A$ es finito. Por otra parte, $(X, \topl)$ es separable (\ref{defEspacioSeparable}) si y sólo si existe un subconjunto numerable o finito denso en $X$.

Si $X$ es finito o numerable, tenemos que $D = X$ cumple $\adh{D} = X$ para cualquier topología. Las dificultades vendrán si $X$ es infinito no numerable.

Pero cualquier $D⊆X$ numerable cumple que $\adh{D} = X$: si $A ∈ \topl_{cofinita}$, $A≠∅$, entonces $A ∩ D ≠ ∅$ (si fuese vacía, entonces $D⊆ A^c$ pero $A^c$ no es finito).

\end{problem}

\begin{problem}[20] Sea $X$ un espacio IAN. Sean $A⊆X$ y $x∈X$. Demuestra que

\ppart $x ∈ \mop{Fr}(A)$ si y sólo si existen $\{x_n\}_{n>0} ⊆ A$ y $\{y_n\}_{n>0} ⊆ X\setminus A$ ambas con límite $x$.
\ppart $x∈A'$ (es punto de acumulación) si y sólo si existe $\{x_n\}_{n>0} ⊆ A$ con límite $x$.

\solution

\spart

\spart

La implicación a la izquierda es siempre cierta. Tenemos que $∀V$ entorno de $x$, entonces como $x_n \to x$, existe $n_V$ tal que $∀n≥n_V$ y $x_n ∈ V$, y por lo tanto $V\setminus\set{x} ∩ A ≝ ∅$ y entonces por la propia definición de punto de acumulación $x∈A'$.

Para la implicación al otro lado, sabemos que $∃V_n$ entorno de $x$ (siendo $n∈ℕ$) tal que $∀V$ entorno de $x$ existe un $n$ tal que $x ∈ V_n ⊆ V$.

Podemos suponer que $V_1 ⊇ V_2 ⊇ \dotsb$. En caso contrario, se usa $V_1 ⊇ V_1 ∩ V_2 ⊇ V_1 ∩ V_2 ∩ V_3 \dotsb$ que también es base decreciente.

Como $x∈A'$ y $V_n$ es entorno de $X$, entonces por la definición $V_n \setminus\set{x} ∩ A ≠ ∅$. Es decir, $∃x_n ∈ A∩V_n$ y $x_n ≠ x$.

Afirmamos que $x_n \convs x$. Para demostrarlo, sea $V$ un entorno de $x$, y nos preguntamos si existe un $n_V$ tal que $x_n ∈ V\; ∀n≥n_V$. Por ser $V$ entorno de $x$, entonces $∃n_V$ tal que $V_{n_V} ⊆ V$. Así, $V_n ⊆ V$ y entonces $∀n≥n_V$ $x_n ∈ V$.

\end{problem}

\begin{problem}[21] Probar que si un espacio topológico $X$ es separable (es decir, existe $A ⊂ X$ numerable y tal que $\adh{A} = X$ ) entonces toda familia de abiertos disjuntos es numerable.
\solution

Si $\adh{A} = X$,entonces $∀V$ abierto no vacío, $V∩A ≠ ∅$. Supongamos que $\set{V_i}_{i∈I}$ son abiertos no vacíos y disjuntos. Entonces $V_i ∩ ≠ ∅$ implica que $∃a_i ∈ V_i ∩ A$. Por otra parte si $V_i ∩ V_j = ∅$, entonces $a_i ≠ a_j$ luego existe una aplicación inyectiva $\appl{ζ}{I}{A}$ tal que $ζ(i) = a_i$. Así, $\card{I} ≤ \card{A}$ y $I$ tiene que ser por fuerza como mucho numerable.
\end{problem}

\begin{problem}[23] Probar que si un espacio topológico es IIAN entonces cualquier unión de abiertos $U = \bigcup_{i ∈ I} U_i$ se puede expresar como una unión numerable, esto es, \[ U = \bigcup_{n ∈ ℕ} U_{i_n} \] con $i_n ∈ I$ para cada $n ∈ ℕ$ (la anterior propiedad se llama de Lindelof).

Usar esto para probar que si un espacio es IIAN entonces cualquier base contiene una base numerable. Esto último puede servir para probar que $(ℝ, \topl_{[,)})$ no es IIAN.

\solution

Por ser IIAN, entonces $\base = \set{B_n\tq n ∈ ℕ}$ es una base de $\topl$. Buscamos demostrar \[ U = \bigcup_{i∈I} U_i \qeq \bigcup_{n∈ℕ}U_{i_n} \]

Por ser $U_i$ abierto, entonces $∃J_i ⊆ ℕ$ tal que $U_i = \bigcup_{n∈J_i} B_n$, luego \[ U = \bigcup U_i = \bigcup_{i∈I} \bigcup_ {n∈J_i} B_n  = \bigcup_{n∈J} B_n \] siendo $J=\bigcup_{i∈I} J_i$.

Entonces $∀m∈J$ existe un $i_m$ tal que $B_m ⊆ U_{i_m}$, y entonces  \[ U = \bigcup_{m∈J} B_m  ⊆ \bigcup_{m∈J} U_{i_m} ⊆ U \implies U = \bigcup_{m∈J} U_{i_m} \]
\end{problem}

\section{Hoja 6}

\subsection{Grupo fundamental y retractos de deformación}

\begin{problem}[1]
Encontrar dos espacios que tengan el mismo grupo fundamental pero que no sean homeomorfos.
\solution

$ℝ$ y $ℝ^2$ tienen el mismo grupos fundamentales isomorfos ($\{1\}$, puesto que son convexos) pero no son homeomorfos.

Vamos a comprobar que no son homeomorfos por reducción al absurdo.

Si fuese homeomorfos con un homeomorfismo $f$, tendríamos que $\real - \{(0,0)\}$ sería homeomorfo a $\real^2 - \{f((0,0))\}$ pero esto no es cierto puesto que $\real - \{(0,0)\}$  no es conexo mientras que $\real^2 - \{f((0,0))\}$ si.

\end{problem}

\begin{problem}[2]
Decidir, razonadamente, si las siguientes afirmaciones son verdaderas o falsas
\ppart Si $A$ y $B$ son subespacios simplemente conexos con $A \cap B \neq \emptyset$, entonces $A \cup B$ también lo es.

\ppart Si $X$ es homeomorfo a la frontera de $[0,1]\times [0,1]$, el grupo fundamental de $X$ es isomorfo a $\ent$

\ppart Si el grupo fundamental de $X$ es isomorfo a $\ent$ y $X$ es conexo por caminos, entonces $X$ es homeomorfo a $\mathbb{S}^1$.

\ppart Si $A$ y $B$ son retractos por deformación fuerte de espacios homeomorfos, entonces $A$ y $B$ son homeomorfos.
\solution

\spart
\textbf{Falso}
\begin{figure}[hbtp]
\centering
\inputtikz{E_H6_E2}
\caption{La intersección de dos conjuntos simplemente conexos no tiene por qué ser simplemente conexa.}
\label{figH6E2}
\end{figure}

Podemos ver dos mitades de un disco que se solapan (figura \ref{figH6E2}). Ambos son simplemente conexos (podemos deformarlos a una sección del disco que sea estrellada) y su intersección es no vacía, por lo que cumplen las hipótesis del enunciado.

Sin embargo, su unión no es simplemente conexa puesto que la corona circular pintada tiene a una circunferencia como retracto por deformación fuerte, lo que implica que su grupo fundamental es $\ent$.

\spart
\textbf{Verdadero}

Podemos dibujar una circunferencia contenia en el cuadrado y, tomando como función la clásia proyección radial vemos que es un retracto por deformación fuerte del cuadrado, por lo que comparten grupo fundamental

\spart
\textbf{Falso}

Podemos verlo fácilmente con un contraejemplo, como podría ser el del cilindro, cuyo grupo fundamental es $\ent$ y es conexo por caminos, mas no es homeomorfo a $\mathbb{S}^1$

\spart
\textbf{Falso}

Veamos un contraejemplo.

Tomemos los espacios $X=Y=\real - \{(0,0)\}$ que, lógicamente son homeomorfos.

Tomemos ahora los retractos por deformación fuerte: $A=X$ y $B=\mathbb{S}^1$.

Con ellos satisfacemos las condiciones iniciales del enunciado pero $A$ y $B$ no son homeomorfos.

\obs Para ver que no son homeomorfos repetimos la idea del primer ejercicio. Si quitamos un punto del borde de $\mathbb{S}^1$ su grupo fundamental pasa a ser $\{1\}$ mientras que quitarle un punto a $\real - \{(0,0)\}$ hace que su grupo fundamental sea $F_2$

% Si aquí añadimos una referencia a los grupos libres, triunfamos como los chichos

\end{problem}

\begin{problem}[3] Decidir si los siguientes espacios son homeomorfos
\ppart
\[X_1 = \{(x,y)\in \real^2 | (x-1)^2 + y^2 \leq 1\} \cup \{(x,y) \in \real^2 | (x+1)^2 + y^2 \leq 1\}\]
\ppart
\[X_2 = \{(x,y)\in \real^2 | x^ + y^2 < 1\}\]
\ppart
\[X_3 = \{(x,y)\in \real^2 | x^2 + y^2 \leq 1\}\]
\ppart
\[X_4 = \{(0,y) \in \real^2 | -1 < y < 1\}\]
\solution

Lo primero es ver qué es cada uno.

$X_1$ son dos discos, centrados en $(-1, 0)$ y $(1,0)$, de radio $1$ y por lo tanto tangentes en $(0,0)$. $X_2$ es el disco abierto centrado en el origen de radio $1$, $X_3$ es el mismo disco pero cerrado, y $X_4$ es el plano sin el segmento vertical $(-1, 1)$.

$X_1$ es cerrado, pero $X_2$ no lo es así que no puede ser homeomorfo. Eso sí, no es algo trivial de demostrar. Podemos verlo mejor por compacidad: $X_1$ es compacto (cerrado y acotado), $X_2$ no lo es. En ese caso, según la proposición \ref{propCompactos} no puede existir una $\appl{f}{X_1}{X_2}$ continua y entonces no puede haber homomorfismo.

Para ver si $X_1$ y $X_2$ son homeomorfos ya no podemos usar compacidad. Por conexión sería fácil: quitamos $(0,0)$ en $X_1$ y ya tendría dos componentes conexas, mientras que $X_2$ seguiría teniendo una única componente.

Podemos demostrarlo también por grupos fundamentales. Si existiese un homeomorfismo, los puntos del borde de $X_1$ tendrían que ir por fuerza a puntos del interior de $X_2$. Si quitamos un punto del borde, el grupo fundamental sigue siendo el grupo trivial, pero en $X_2$ ya lo habríamos cambiado y tendríamos el grupo fundamental $ℤ$.

$X_2$ y $X_3$ no son homeomorfos por el mismo argumento de compacidad.

Por último, $X_4$ no es homemorfo a ninguno de los otros espacios: $X_4$ es un RDF de $\crc$, con $π(X_4) \simeq ℤ$. Sin embargo, todos los demás tienen como grupo fundamental $\set{1}$ así que no pueden ser homemorfos.

\end{problem}

\begin{problem}[4] Hallar el grupo fundamental de $A = \set{1≤x^2+y^2≤4}$ y de $B = \set{x^2+y^2≥4}$.

\solution

En el segundo caso, es un RDF de la circunferencia de radio $2$ con la transformación \[ H(t,p) = 2t\frac{p}{\md{p}}  + (1-t) p \]

Podemos usar esa misma deformación, $H$, para demostrar que $A$ es un RDF de la circunferencia de radio dos, la de fuera de la corona.

Así, el grupo fundamental de ambos es isomorfo a $ℤ$ ya que los dos son homomorfos a la circunferencia de radio $1$.
\end{problem}

\begin{problem}[5]
Demostrar que la relación \textit{ser un retracto por deformación fuerte} es transitiva, esto es, si $A$ es RDF de $B$ y $B$ es RDF de $C$, entonces $A$ es RDF de $C$.
\solution

Si $A$ es RDF de $B$, entonces existe una aplicación $\appl{G}{[0,1]×X}{X}$ que nos lleva del uno a otros, y similarmente existe otra $H$ que nos lleva de $B$ a $C$.

Vamos a buscar ahora la aplicación $F$ que nos lleve de $A$ a $C$ \[ \appl{F}{[0,1]×C}{C} \] tal que $F(0,x) = x,\, F(1,x) ∈ A\;∀x∈C$ y $F(t,x) = x\;∀x ∈ A$.

Parece que lo obvio será buscar una aplicación que primero nos lleve a $B$ con $H$ y luego a $C$ con $G$, es decir,

\[ F(t,x) = \begin{cases}
H(2t,x) & t≤\frac{1}{2} \\
G(2t - 1, H(1, x)) & \frac{1}{2} ≤ t ≤ 1
\end{cases} \]

Demostrar sus propiedades es fácil. La continuidad se hace usando una proposición que no me acuerdo cuál es, que dice algo de cerrados y cosas.
\end{problem}

\begin{problem}[6]
Hallar el grupo fundamental del toro sólido $\disc × \crc$ donde $\disc =\set{(x,y) ∈ ℝ^2 \tq x^2 +y^2 < 1}$.
\solution

Sabemos que $π(X_1 × X_2) \simeq π(X_1) × π(X_2)$. Como el grupo fundamental del disco es el trivial y el de $\crc$ es $ℤ$, el grupo fundamental del toro sólido es $ℤ$.

De hecho, podemos ver cómo sería la deformación que nos llevaría el toro sólido a una circunferencia en la figura \ref{figH6E6}.

\begin{figure}[hbtp]
\centering
\inputtikz{E_H6_E6}
\caption{Un RDF del toro sólido a la circunferencia en $ℝ^3$.}
\label{figH6E6}
\end{figure}

\end{problem}

\begin{problem}[7] Probar que $\disc = \set{(x,y) ∈ ℝ^2 \tq x^2 + y^2 < 1}$ y $D = \disc ∪ \set{(1,0)}$ no son homeomorfos.
\solution

Supongamos que son homeomorfos. Entonces $∃\appl{φ}{D}{\disc}$ homeomorfismo. En ese caso, $φ$ restringida a $D \setminus \set{(1,0)}$, es decir, $\appl{φ}{\disc}{\disc \setminus \set{φ((1,0))}}$ debería seguir siéndolo\footnote{Ver el ejercicio \ref{ejH6E12} para una demostración rigurosa de esto. Tiene que ver con que la restricción de una función continua es continua.}

Esto querría decir que el grupo fundamental $π(\disc)$ y $π(\disc \setminus \set{p})$ (con $p = φ(1,0)$) son isomorfos. Ahora bien, $p$ sólo puede ser un punto del interior, luego $π(\disc\setminus\set{p}) \simeq π(\crc) \simeq ℤ$, y sin embargo $π(\disc) \simeq \set{1}$ por ser convexo, contradicción.
\end{problem}

\begin{problem}[8] Decidir, razonadamente, si los siguientes espacios topológicos son homeomorfos:
\begin{gather*}
ℝ×\crc × (\bbs^2 \setminus\set{(0,0,1)}) \\
ℝ^2 × (\bbs^2 \setminus\set{(0,0,1)}) \\
ℝ^4
\end{gather*}
\solution

Empezamos por lo fácil: el grupo fundamental de $ℝ^n$ es el trivial por ser convexo. El grupo fundamental de $(\bbs^2 \setminus\set{(0,0,1)})$, la esfera hueca, es también el trivial: la proyección estereográfica es un homemorfismo (ya que no tenemos el polo que nos molesta para continuidad) y nos lleva al plano. Por último, sabemos que el grupo fundamental de la circunferencia es $ℤ$. Así, tenemos que

\begin{align*}
π\left(ℝ×\crc × (\bbs^2 \setminus\set{(0,0,1)})\right) &\simeq ℤ \\
π\left(ℝ^2 × (\bbs^2 \setminus\set{(0,0,1)})\right) &\simeq \set{1} \\
π\left(ℝ^4\right) &\simeq \set{1}
\end{align*}

\end{problem}


\begin{problem}[11]
Sea $\set{U_i}_{i \in I}$ un recubrimiento por abiertos del espacio $X$ que verifique las siguientes condiciones:
\begin{itemize}
\item Existe un punto $x_0$ tal que $x_0 \in U_i\; \forall i \in I$.
\item Para cada $i\in I, U_i$ es simplemente conexo.
\item Si $i \neq j$, entonces $U_i \cap U_j$ es conexo por caminos.
\end{itemize}

\ppart Probar que $X$ es simplemente conexo
\ppart Deducir que $\mathbb{S}^1$ es simplemente conexo si n $\geq$ 2

\textit{Indicación:} Para probar que todo lazo $\appl{α}{I}{X}$ con base en $x_0$ es trivial, considerese primero el recubrimiento abierto $\{α^{-1}(U_i)\}$ del compacto $I=[0,1]$ y, con ayuda del número de Lebesgue de este recubrimiento, escribir $α = α_1\cdot α_2 \cdot \dotsb \cdot α_n$ tal que $α_j(I)$ es subconjunto de algún $U_j$.
\solution

\spart

Un esbozo de la solución. El número de Lebesgue de un recubrimiento de $(Y,\dst)$ espacio métrico compacto.

Si $\set{V_j}_{j∈J}$ es un recubrimiento abierto de $Y$ entonces $∃δ>0$ tal que $∀B ⊆ Y$ con $\mop{diam}(B) < δ$ existe un $j_B ∈ J$ tal que $B ⊆ V_{j_B}$.

\seprule

Definimos $V_i ≝ \inv{α}(U_i)$, la imagen inversa de los abiertos que recubren $X$. Como α es continua, entonces $V_i$ es abierto en $[0,1]$.

Además, como $U_i$ es un recubrimiento de $X$ y $α([0,1]) ⊆ X$, entonces la unión de los $V_i$ es un recubrimiento de $[0,1]$.

$∃δ> 0$ tal que $B⊆[0,1]$, $\mop{diam}(B) < δ$, entonces $∃i_B$ tal que $B⊆V_{i_B}$.

Cogemos $n$ tal que $\frac{1}{n} < δ$, y tomamos $t_k = \frac{k}{n}$. Es decir, dividimos el intervalo $[0,1]$ en muchos intervalos pequeñitos.

Entonces, $∃i_j$ tal que $[t_j, t_{j+1}] ⊆ V_{i_j} = \inv{α}(U_{i_j})$, esto es, $α([t_j, t_{j+1}]) ⊆ U_{i_j}$. Por comodidad, denotaremos $α([t_j, t_{j+1}]) = A_j$. En la figura \ref{figH6E11} tenemos un esquema de qué está pasando.

\begin{figure}[hbtp]
\centering
\inputtikz{E_H6_E11}
\caption{Esquema de lo que estamos haciendo. Los $A_j$ tienen que ser como los azules, acabando en la intersección con el siguiente $U_i$. No puede ser como el $A_j$ rojo, que acaba en un punto que no es de la intersección. La intersección (rayas naranjas) de $U_i ∩ U_j$ es cpc, luego $β_j$ siempre existe. Expresando $α$ como esta composición podemos llegar a deformarlo continuamente a un punto fijo.}
\label{figH6E11}
\end{figure}

Definimos entonces \[ α_j ≝ \restr{α}{[t_j, t_{j+1}]} \], la restricción de $α$ al intervalo ese. Entonces parece claro que $α \dfm α_0 * α_1 * \dotsb * α_{n-1}$.

Además, podemos meter entre medias el camino que nos lleva a $x_0$ y luego volver a donde estábamos. Es decir, como $β_0^- * β_0 = δ_0$, entonces \[ α \dfm α_0 * β_0^- * β_0 * α_1 * \dotsb \] teniendo en cuenta, eso sí, que $t_{n+1} = t_0$ (el camino es cíclico).

La idea de todo esto es que $γ_j = β_j * α_{j+1} * β_{j+1}^-$ está entonces contenido en $U_{i_{j+1}}$. Como por hipótesis todos los $U_i$ son simplemente conexos, entonces $γ_j \dfm δ_{x_0}$ en $U_{i_{j+1}}$, y en particular en $X$.

Además, tendremos que \[ α = δ_{x_0} * δ_{x_0} * \dotsb = δ_{x_0} \]

Es decir, lo que hemos hecho ha sido descomponer el camino $α$ en varios trozos, demostrar que podemos deformar cada uno de esos trozos en un punto y así nos queda que la deformación al punto es muy sencillita.

\spart ¿Qué abiertos podemos coger en $\bbs^n$? Pues por ejemplo \[ U_1 = \set{x∈ \bbs^n \tq x_{n+1} > \frac{-1}{2} } \] y \[ U_2 = \set{x∈ \bbs^n \tq x_{n+1} < \frac{1}{2} } \], siendo $x_{n+1}$ la última coordenada del punto (por ejemplo, si estamos en $ℝ^3$, la coordenada $z$).

Estos $U_i$ son simplemente conexos (lo podemos ver haciendo la proyección estereográfica, que es un disco en el plano). Además, la intersección $U_1 ∩ U_2$ es cpc. De hecho, podemos ir de un punto $p$ a otro $q$ moviéndonos desde $p$ hasta el ecuador por el meridiano e ir por el ecuador hasta que podemos subir (o bajar) por el meridiano a $q$.

Tenemos todas las condiciones del apartado anterior, así que efectivamente la esfera $\bbs^n$ es simplemente conexa en $ℝ^{n+1}$ para $n≥2$.

\spart

Esto ya porque nos apetece. Veamos qué ocurre con dos esferas tangentes en un punto. Podemos coger $U_i$ como una esfera y un casquete de la otra. Este conjunto es RDF de la esfera (movemos los puntos del casquete al punto de tangencia a través de las geodésicas), luego es simplemente conexo, y claramente la intersección $U_1 ∩ U_2$, los dos casquetes, será cpc, y estamos de nuevo en las condiciones del apartado anterior.

\end{problem}

\begin{problem}[12] Demostrar que $\bbs^n$, la esfera $n$-dimensional, es un RDF de $ℝ^{n+1}\setminus\set{0}$. Utilizar este hecho para demostrar que $ℝ^2$ no es homeomorfo a $ℝ^n$ con $n≠2$.
\label{ejH6E12}
\solution

\spart Simplemente cogemos la función \[ H(t,x) = (1-t) x + t \frac{x}{\md{x}} \], que es continua y cumple las condiciones que necesitamos, luego $\bbs^n$ es un RDF de $ℝ^{n+1}\setminus\set{0}$.

\spart Hay que distinguir primero el caso $ℝ^1$. Aquí, $ℝ^2$ no es homeomorfo a $ℝ^1$. Si lo fuese, entonces existiría un homomorfismo $\appl{h}{ℝ^2}{ℝ^1}$ y entonces $\appl{\hat{h}}{ℝ^2\setminus\set{0}}{ℝ\setminus\set{0}}$ tal que $\hat{h}(x) = h(x)$ también lo sería. Vamos a justificarlo también:

$\hat{h}$ es continua por ser la restricción de $h$, función continua, y su inversa también lo es por ser la restricción de $\inv{\hat{h}}$ .

En este caso, podríamos tomar un camino φ cerrado en $ℝ^2\setminus\set{0}$, y entonces $\hat{h}○φ$ sería igualmente cerrado en $ℝ\setminus\set{0}$. La imagen de $\hat{h}○φ$ está en $ℝ_+$ o en $ℝ_-$, y se puede deformar entonces al camino trivial ($ℝ_+, ℝ_-$ son simplemente conexos).

Entonces, volviendo por $\inv{\hat{h}}$ tendríamos que φ se puede deformar al camino trivial, luego $π(ℝ^2\setminus\set{0})$ sería trivial, contradicción.

Lo intereseante es ver $n>2$. Vamos a demostrar que no existe un homomorfismo $h$. Si existe, tomamos $\appl{\hat{h}}{ℝ^2\setminus\set{0}}{ℝ^n\setminus\set{0}}$, y entonces tendríamos que \[ ℤ\simeq π(ℝ^2\setminus\set{0})\simeq π(ℝ^n\setminus\set{h(0)})\simeq \set{1} \], contradicción.

\end{problem}


\printindex

\end{document}
