% -*- root: ../TopologiaI.tex -*-
\chapter{Exámenes}

\section{2013/2014}

\subsection{Parcial I}

\begin{problem} Consideramos $ℝ$ con la topología generada por la base $\base = \{[a,b) \tq a < b\}$. Hallar el interior y adherencia del conjunto \[ A = \left\{1 - \frac{1}{n} \tq n ∈ ℕ \right\} ∪ [4,5) ∪ [7,8) ∪ (9,10) \]
\solution

Recuperemos la definición de interior (\ref{defInterior}): $x ∈ \mop{Int}(W) = \intr{W}$ si existe un entorno $U$ de $x$ tal que $U⊆W$. Por otra parte, la adherencia (\ref{defAdherencia}) se define de la siguiente forma \[ x ∈ \adh{W} \iffdef A∩W ≠ \emptyset \; ∀ A ∈ \topl \tq x∈A \]

Vamos a estudiar por partes la adherencia y el interior de cada uno de los subconjuntos disjuntos de $A$: $A_1 = \left\{1 - \frac{1}{n} \tq n ∈ ℕ \right\}$, $A_2 = [4,5)$, $A_3 = [7,8)$ y $A_4 = (9,10)$. Dado que estamos trabajando con una base, nos bastará comprobar las definiciones para los elementos de la base.

\paragraph{Interior} Para $A_1$ lo tenemos fácil: no hay interior. Si $x = 1 - \frac{1}{n}∈\intr{A_1}$, entonces $∃[a,b) ⊆ A_1$ con $a ≤ x < b$. Ahora bien, podemos encontrar un $ε ∈ (0,1)$ suficientemente pequeño tal que $β = 1 - \frac{1}{1+ε} < b$, por lo que $[a, β) ⊆ [a,b)$. Pero tal y como hemos construido el intervalo, $[a, β) ∩ A_1 = \{ x \}$, pero está claro que $[a,β) ≠ \{x\}$, lo que quiere decir que $[a,β) \nsubseteq A_1$, contradicción.\footnote{No sé si se podría haber hecho más fácil o si me he pasado de riguroso.}

En $A_2$ y $A_3$ también es sencillo: $[4,5)$ y $[7,8)$ son abiertos y por lo tanto para cualquier punto contenido en ellos existe un entorno abierto (ellos mismos). Es decir, que $A_2$ y $A_3$ están en el interior. También podemos usar la propiedad de que $W$ es abierto si y sólo si $W=\intr{W}$.

Y por último, $A_4$ también es todo interior: $∀x ∈ (9,10)$, el elemento de la base $[x, x+ε)$ con $0 < ε < 10 - x (< 1)$ es un abierto contenido en $A_4$ que contiene al punto.

\paragraph{Adherencia} De $A_1$, la adherencia es $A_1$, y lo mismo ocurre con $A_2$ y $A_3$. En $A_4 = (9,10)$, la adherencia es $[9,10)$ porque cualquier abierto de la base $[9, 9+ε)$\footnote{Los abiertos que se extienden más por la izquierda no hace falta considerarlos porque contienen un intervalo del tipo $[9, 9+ε)$.} con $ε>0$ tiene intersección no vacía con $A_4$.
\end{problem}

\begin{problem} Sean $X$ un espacio topológico, $Y$ un espacio topológico de Hausdorff y \stdf continua. Probar que el grafo $\{(x,f(x)) \tq x ∈ X\}$ es cerrado en $X×Y$.
\solution
\end{problem}

\begin{problem} Demostrar que cualquiera de los siguientes subconjuntos de $ℝ^2$, con la topología usual, no son homeomorfos:

\centering \Huge{A, O, H}
\solution
\end{problem}

\begin{problem} Se considera $X = \{x∈ℝ^2 \tq 1 ≤ x_1^2 + x_2^2 ≤ 2\}$ con la topología de subespacio indicida por la del orden lexicográfico en $ℝ^2$.

\ppart Demostrar que $X$ no es conexo.
\ppart Encontrar las componentes conexas de $X$

\solution
\spart
\spart
\end{problem}

\section{2014-2015}

\subsection{Parcial I}

\begin{problem} Sea $\rel = \{( -∞,b] \tq b ∈ ℝ\}$. 
\ppart ¿Es una base de una topología en $ℝ$?
\ppart Dado \[ A = \]
\solution

\spart Tenemos que demostrar que la unión de todos los elementos de la base está en $ℝ$, \[ \bigcup_{b∈ℝ} (-∞, b] = ℝ \], obvio.

Además, dados $B_1, B_2 ∈ \rel$ y $x∈B_1∩B_2$ entonces tiene que existir $B_3∈\rel$ tal que $x∈B_3 ⊆ B_1 ∩ B_2$. Si $B_1 = (-∞, b_1]$ y $B_2 = (-∞, b_2]$, entonces \[ B_1∩B_2 = (-∞, \min (b_1, b_2)] \], que es elemento de la base y entonces ya tenemos lo que necesitábamos.

\spart Para comprobar $x∈\adh{A}$ o $x∈\intr{A}$ basta considerar en la definición elementos de la base.

Si $x≤1$, no está en la adherencia porque $x∈(-∞, x]$, cuya intersección con $A$ es vacío.

Los puntos del intervalo $(1, 2]$ están en la adherencia por ser puntos de $A$.

Si $x>2$, entonces $x∈\adh{A}$ pues $∀B∈\rel$ con $x∈B$ se tiene que $B∩A ≠ ∅$.

Vamos ahora con el interior, que es vacío ya que si $x∈B ∈ \rel$, entonces $B\nsubseteq A$.

\end{problem}

\begin{problem} Sea $\appl{f}{([0,1]×[0,1], \topl_{Lex})}{(ℝ, \topl_{usual})}$ con $f(x,y) = x$. ¿Es continua? ¿Es abierta?
\solution

Sobre la continuidad, $f$ es continua si y sólo si la imagen inversa de un abierto es abierto. Es decir, tenemos que ver que $\inv{f}((a,b))$ es abierto en $([0,1]×[0,1], \topl_{Lex})$.

Si $a≥1$ o $b≤0$, $\inv{f}((a,b)) = ∅$. Tenemos que estudiar el caso de $a<1, b>0$. 

Si todo el intervalo está ($a < 0$ y $b > 1$) entonces la imagen inversa es el total.

Si $b > 1$ y $a ≥ 0$, entonces $\inv{f}((a,b)) = \{ (x_1, x_2) \tq a < x_1 ≤ 1 \}$ que es abierto por ser $((a,1), (1,1)]$, que está en la topología por definición. El caso $a < 0, b ≤ 1$ es similar.

El último caso a estudiar es cuando $a ≥ 0$ y $b ≤ 1$. Aquí, $\inv{f}((a,b)) = \{ (x_1, x_2) \tq a < x_1 < b\} = ((a,1), (b,0))_{Lex} ∈ \topl_{Lex}$ por definición.

Ahora toca ver si es abierta, que no lo es. Por ejemplo $f([(0,0), (0.5, 0)) = [0, 0.5)$ en $ℝ$ que no es abierto. Otro ejemplo sería un intervalo vertical, cuya imagen es un punto que es cerrado.
\end{problem}

\begin{problem} Dado $A⊆X$, 
\ppart Demostrar $\mop{Fr}(\adh{A}) ⊆ \mop{Fr}(A)$ y $\mop{Fr}(\intr{A}) ⊆ \mop{Fr}(A)$.
\ppart Buscar ejemplos en los que se de las inclusiones anteriores estrictas.
\solution
\spart Sabemos que $\mop{Fr}(B) = \adh{B} \setminus \intr{B}$. Luego \[ \mop{Fr}(\adh{A}) = \adh{\adh{A}} \setminus \intr{\adh{A}} \], pero como $A⊆\adh{A}$ entonces $\intr{A} ⊆ \intr{\adh{A}}$, por lo que $\adh{A} \setminus \intr{\adh{A}} ⊆ \adh{A} \setminus \intr{A}$ y entonces $\mop{Fr}(\adh{A}) ⊆ \mop{Fr}(A)$.

Para la frontera del interior, podemos hacerlo usando la dualidad entre complementarios: $\intr{A} = (\adh{A^c})^c$ y $\mop{Fr}(A) = \mop{Fr}(A^c)$, y con la propiedad anterior es inmediato.\footnote{Esto merecería algo más de expansión.}

\spart Buscamos ejemplos extremos que nos faciliten la vida. En el primero, podemos ver que si cogemos el total, su frontera siempre es vacía. De hecho, la frontera de un conjunto es vacía si y sólo si es abierto y cerrado a la vez.

Buscamos entonces un conjunto cuya adherencia sea el total pero que él mismo no sea el total. Por ejemplo, $A=ℚ$ en $ℝ$ con la topología usual. Aquí, $\mop{Fr}(ℚ) = ℝ$, pero $\mop{Fr}(\adh{ℚ}) = \mop{Fr}(ℝ) = ∅$. De hecho, este mismo ejemplo nos vale para la otra inclusión estricta: $\mop{Fr}(\intr{ℚ}) = \mop{Fr}(∅) = ∅$, pero $\mop{Fr}(ℚ) = ℝ$.

\end{problem}

\begin{problem} En $([0,1]×[0,1], \topl_{Lex})$, cogemos el ``borde'' del cuadrado, esto es, \[W = \left(\{0,1\} × [0,1]\right) ∪ \left([0,1] × \{0,1\}\right) \]. Encuentra las componentes conexas de $W$ en $\topl_{Lex}^{sub}$.
\solution

Las componentes conexas han de ser los subconjuntos conexos de $W$ más grandes posibles. ¿Cómo son los conexos en $\topl_{Lex}$ $[0,1]×[0,1]$? Son los intervalos.

En $W$, el conexo más grande que contiene al $(0,0)$ es $[(0,0), (0,1)]_{Lex}$. Si $x = (x_1, x_2) > (0,1)$ entonces $x_1 > 0$ y $[(0,0), x] \nsubseteq W$.

Por el mismo argumento, la componente conexa más grande que contiene al $(1,1)$ es el intervalo $[(1,0), (1,1)]$. 

¿Cuáles son las componentes que contienen a los puntos de los intervalos horizontales? Si $x = (a,0)$ con $0<a<1$, la componente conexa que lo contiene es $\{(a,0)\}$ pues cualquier intervalo $(α,β)$ con $(a,0) ∈ (α,β)$ implica que $α < a$, y entonces $(α,β) \nsubseteq W$.
\end{problem}