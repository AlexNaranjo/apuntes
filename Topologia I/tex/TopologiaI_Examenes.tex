% -*- root: ../TopologiaI.tex -*-
\chapter{Exámenes}

\section{2013/2014}

\subsection{Parcial I}

\begin{problem} Consideramos $ℝ$ con la topología generada por la base $\base = \{[a,b) \tq a < b\}$. Hallar el interior y adherencia del conjunto \[ A = \left\{1 - \frac{1}{n} \tq n ∈ ℕ \right\} ∪ [4,5) ∪ [7,8) ∪ (9,10) \]
\solution

Recuperemos la definición de interior (\ref{defInterior}): $x ∈ \mop{Int}(W) = \intr{W}$ si existe un entorno $U$ de $x$ tal que $U⊆W$. Por otra parte, la adherencia (\ref{defAdherencia}) se define de la siguiente forma \[ x ∈ \adh{W} \iffdef A∩W ≠ \emptyset \; ∀ A ∈ \topl \tq x∈A \]

Vamos a estudiar por partes la adherencia y el interior de cada uno de los subconjuntos disjuntos de $A$: $A_1 = \left\{1 - \frac{1}{n} \tq n ∈ ℕ \right\}$, $A_2 = [4,5)$, $A_3 = [7,8)$ y $A_4 = (9,10)$. Dado que estamos trabajando con una base, nos bastará comprobar las definiciones para los elementos de la base.

\paragraph{Interior} Para $A_1$ lo tenemos fácil: no hay interior. Si $x = 1 - \frac{1}{n}∈\intr{A_1}$, entonces $∃[a,b) ⊆ A_1$ con $a ≤ x < b$. Ahora bien, podemos encontrar un $ε ∈ (0,1)$ suficientemente pequeño tal que $β = 1 - \frac{1}{1+ε} < b$, por lo que $[a, β) ⊆ [a,b)$. Pero tal y como hemos construido el intervalo, $[a, β) ∩ A_1 = \{ x \}$, pero está claro que $[a,β) ≠ \{x\}$, lo que quiere decir que $[a,β) \nsubseteq A_1$, contradicción.\footnote{No sé si se podría haber hecho más fácil o si me he pasado de riguroso.}

En $A_2$ y $A_3$ también es sencillo: $[4,5)$ y $[7,8)$ son abiertos y por lo tanto para cualquier punto contenido en ellos existe un entorno abierto (ellos mismos). Es decir, que $A_2$ y $A_3$ están en el interior. También podemos usar la propiedad de que $W$ es abierto si y sólo si $W=\intr{W}$.

Y por último, $A_4$ también es todo interior: $∀x ∈ (9,10)$, el elemento de la base $[x, x+ε)$ con $0 < ε < 10 - x (< 1)$ es un abierto contenido en $A_4$ que contiene al punto.

\paragraph{Adherencia} De $A_1$, la adherencia es $A_1$, y lo mismo ocurre con $A_2$ y $A_3$. En $A_4 = (9,10)$, la adherencia es $[9,10)$ porque cualquier abierto de la base $[9, 9+ε)$\footnote{Los abiertos que se extienden más por la izquierda no hace falta considerarlos porque contienen un intervalo del tipo $[9, 9+ε)$.} con $ε>0$ tiene intersección no vacía con $A_4$.
\end{problem}

\begin{problem} Sean $X$ un espacio topológico, $Y$ un espacio topológico de Hausdorff y \stdf continua. Probar que el grafo $\{(x,f(x)) \tq x ∈ X\}$ es cerrado en $X×Y$.
\solution
\end{problem}

\begin{problem} Demostrar que cualquiera de los siguientes subconjuntos de $ℝ^2$, con la topología usual, no son homeomorfos:

\centering \Huge{A, O, H}
\solution
\end{problem}

\begin{problem} Se considera $X = \{x∈ℝ^2 \tq 1 ≤ x_1^2 + x_2^2 ≤ 2\}$ con la topología de subespacio indicida por la del orden lexicográfico en $ℝ^2$.

\ppart Demostrar que $X$ no es conexo.
\ppart Encontrar las componentes conexas de $X$

\solution
\spart
\spart
\end{problem}