% -*- root: ../TopologiaI.tex -*-
\section{Hoja 1}

\begin{problem}[1]
Sea $X=\{a,b,c\}$ un conjunto de 3 elementos. Encontrar todas las topologías sobre $X$.
\solution

Las topologías más sencillas son: $\topl_1 = \{\emptyset,X,a\}$. Es trivial comprobar que cumple las 3 propiedades de topología. Lo mismo con $b$, con $c$, con ${a,b}$, con ${b,c}$ y ${a,c}$.

El siguiente nivel de compejidad son aquellas con 2 elementos: $\topl_4 = \{\emptyset,a,b,X\}$. Para que esto fuera topología faltaría $\{a\} \cup \{b\}$, con lo que $\topl_4 = {\emptyset,a,b,\{a,b\}X}$

Y así acaban saliendo todas.
\end{problem}

\begin{problem}[2]
En $\real$ se considera la $\topl =\{(-\infty,a): -\infty \leq a \leq \infty\}$. Demostrar que es una topología.
\solution

Tenemos que demostrar las 3 propiedades de topología:

\begin{enumerate}
\item $\emptyset,\real \in \topl$. \\Basta tomar $a=\pm \infty$ para tener ambas pertenencias.
\item $A,B\in\topl \implies A\cap B\in\topl$.\\ Dados $a_1,a_2\in\real$, con $a_1\neq a_2$ tenemos $A_1 = (-\infty,a_1)$ y $A_2 = (-\infty,a_2)$.

Entonces, $A_1\cap A_2 = (.-\infty,min(a_1,a_2)) = \left\{\begin{array}{cc}
A_1\cap A_2 = A_1 & si\; a_1 = min(a_1,a_2)\\A_1\cap A_2 = A_2  & si\; a_2 = min(a_1,a_2)
\end{array}\right.$

Hemos demostrado que $A_1 \cap A_2 \in \topl$.

\item $A,B\in\topl \implies A\cup B\in\topl$. Lo mismo que el apartado anterior tomando máximos en vez de mínimos.
\end{enumerate}
\end{problem}


\begin{problem}[3]
Sean $X$ un conjunto infinito y $\topl$ una topología sobre $X$ en la que todos los subconjuntos infinitos
son abiertos. Demostrar que $\topl$ es la topología discreta de $X$.
\solution

Tenemos que demostrar que $\topl = \topl_{disc} = \mathcal{P}(X)$

(No se me ocurre nada medio formal. Intuitivamente está claro)
\end{problem}

\begin{problem}[4]
 Sea $X$ un conjunto con más de dos elementos.
\ppart Definir dos topologías $\topl_1,\topl_2$ sobre $X$ de modo que $\topl_1 \cup \topl_2$ no sea una topología.
\ppart Sea $\topl_j , j ∈ J$ una familia de topologías sobre $X$. Probar que
 $\bigcap_{j∈J} Tj$ es también una topología sobre $X$.

\solution
\spart Sea $X = \{a,b,c\}$. Definiendo $\topl_1 = \{\emptyset,X,a\}$ y $\topl_2=\{\emptyset,X,b\}$.

$\topl_1 \cup \topl_2 = \topl_{\cup}= \{\emptyset,X,a,b\}$ no es topología, porque $\{a\} \cup \{b\} = \{a,b\} \notin \topl_{\cup}$
\spart
Probándolo para 2 topologías cualesquiera, lo habremos probado para todas, pues por inducción, si se cumple para 2 se cumple para una cantidad finita.

Tenemos que comprobar las 3 propiedades de topología:

\begin{enumerate}
\item $\emptyset,\real \in \topl$. \\ Como $\emptyset,X $ pertenecen a ambas$ \topl_1,\topl_2 \implies \emptyset,X \in \topl_{\cap}$

\item $A_1,A_2\in\topl_{\cap} \implies A_1\cap A_2\in\topl_{\cap}$.\\
\[\left.\begin{array}{c}
A_1\in\topl_{\cap}\implies 
\left\{
	\begin{array}{cc}
		A_1\in\topl_1\\A_1\in\topl_2
	\end{array}
\right.\\
 \text{Lo mismo con } A_2.
\end{array}\right\}\implies \begin{array}{c}
A_1\cap A_2\in\topl_1\\
A_1\cap A_2\in \topl_2\end{array}\implies A_1\cap A_2 \in \topl_{\cap}\]


\item $A_1,A_2\in\topl_{\cup} \implies A\cup A_2\in\topl_{\cup}$. \\
El mismo razonamiento es válido para la unión.
\end{enumerate}
\end{problem}

\begin{problem}[5]
 En el plano $\real^2$ se considera la familia $\topl$ de todos los subconjuntos $U$ tales que para cada punto
$(a, b) \in U\; \exists  ε > 0 \tlq ((a − ε, a + ε) × b) ∪ (a × (b − ε, b + ε)) \subset U$.\\ Estudiar si
$\topl$ es una topología en $\real^2$.
\solution

Gráficamente es fácil de contestar. Los subconjuntos pedidos son cruces y la intersección de 2 cruces no es una cruz, con lo que no puede ser topología.
\end{problem}


\begin{problem}[6] Sea $\appl{g}{X}{Y}$ una aplicación entre dos conjuntos.

\ppart Demostrar que si $\topl$ es una topología en $X$ entonces \[ \mathcal{S} = \{ E ⊆ Y \tq \inv{g}(E) ∈ \topl \} \] es una topología en $Y$.
\ppart Demostrar que si $\mathcal{S}$ es una topología en $Y$ entonces \[ \mathcal{U} = \{ \inv{g}(E) \tq E ∈ \mathcal{S} \} \]es una topología en $X$.

\solution
\spart Vamos a demostrar que es una topología, para lo cual tenemos que comprobar las 3 propiedades (ver \ref{defTopología}):

Es importante saber que las aplicaciones entre conjuntos se definen en todo el dominio, no en un subconjunto, es decir, $∀ x ∈ X, ∃g(x)∈Y$

\begin{enumerate}
\item Tomando $E=Y$, tenemos $Y∈ \mathcal{S}$ por ser $g$ una aplicación de conjuntos tal que $g^{-1}(Y)=X$,=. (La imagen inversa de todo Y pertenece a la topología de X, ya que es X.)

Tomando $E=Ø$, tenemos que $g^{-1}(Ø) = Ø \in \topl$, porque no puede existir un $x\in X\tlq g(x)=Ø$ por ser $g$ aplicación de conjuntos (todos los elementos tienen que tener una imagen del cunjunto destino).

\item $A,B \in \mathcal{S} \dimplies g^{-1}(A),g^{-1}(B) \in \mathcal{T}$. (1)

$A\cap B \in\mathcal{S} \dimplies g^{-1}(A\cap B) \in \mathcal{T}$.(2)

Si tuvieramos que (1) $\implies$ (2) ya lo tendríamos demostrado. Vamos a demostrar que $g^{-1}(A),g^{-1}(B) \in \mathcal{T} \implies g^{-1}(A\cap B) \in \mathcal{T}$.

Para ello: $g^{-1}(A\cap B) = g^{-1}(A)\cap g^{-1}(B)$. No es difícil convencernos de esta igualdad. Para resolver las dudas, vamos a demostrar las 2 inclusiones (una en cada sentido). 

\paragraph{$\subset$}
\begin{gather*}
g^{-1}(A\cap B) \subset g^{-1}(A)\cap g^{-1}(B) \implies
x∈g^{-1}(A∩B) \implies ∃y∈A∩B\tq g^{-1}(y)=x\\
\implies\left| \begin{array}{c}
y∈A \implies g^{-1}(y)=x∈g^{-1}(A)\\
y∈B \implies g^{-1}(y)=x∈g^{-1}(B)
\end{array}
\right.
\end{gather*}

\paragraph{$⊃$}


\begin{gather*}
g^{-1}(A∩B) ⊃ g^{-1}(A)∩g^{-1}(B) \implies x∈g^{-1}(A)∩g^{-1}(B)\implies \\
\implies
\left|\begin{array}{cc}
x∈g^{-1}(A) \implies ∃ y_a ∈ A\tq g^{-1}(y_a)=x\\
x∈g^{-1}(B) \implies ∃ y_b ∈ B\tq g^{-1}(y_b)=x
\end{array}\right. \underset{g\, aplicacion}{\implies} y_a=y_b\\
\end{gather*}
Hemos demostrado: $$∀x∈g^{-1}(A)∩g^{-1}(B)∃y∈A∩B \tq g^{-1}(y)=x∈g^{-1}(A∩B)$$


Es importante el comentario de que esto para imágenes directas no funciona.

\item Demostramos ahora que la unión de abiertos está en la topología. Si $A, B ∈ \mathcal{S}$, entonces $\inv{g}(A), \inv{g}(B) ∈ \topl$. Como $\topl$ es topología, tenemos que $\inv{g}(A) ∪ \inv{g}(B) ∈ \topl$, lo que implica (por el mismo razonamiento que antes) que \\$\inv{g}(A∪B) ∈ \topl$ y por lo tanto $A∪B ∈ \mathcal{S}$.
\end{enumerate}

\spart

\end{problem}


\begin{problem}[7]
Sean $X$ un conjunto y $a$ un elemento de $X$. Se considera la familia $\topl_a$ de los subconjuntos $U\subset X$ tales que o bien $U = \emptyset$, o bien $a ∈ U$ . Estudiar si $\topl_a$ es una topología en $X$.
\solution

Comprobamos las 3 propiedades de topología:
\begin{enumerate}
\item
\begin{itemize}
\item Ø∃∈$\topl_a$: Sí, por definición de $\topl_a$.
\item Tomando $U=X$, tenemos que $X\in\topl_a \impliedby a\in X$.
\end{itemize}
\item Sean $U₁,U₂∈X.$  Tenemos que demostrar si $a∈U₁∩U₂$. En este caso está claro dado que a∈U₁ y a∈U₂.
\item Lo mismo con la unión.
\end{enumerate}

Salvo que dejuan sea un tolai, podemos concluir que $\topl_a$ es una topología en $X$.

\end{problem}

\begin{problem}[9] Se consideran las siguientes familias de conjuntos en $ℝ$:

\begin{gather*}
\base_{\leftarrow} = \{ (-∞, b) \tq b ∈ ℝ \} \\
\base_{\rightarrow} = \{ (a,∞) \tq  ∈ ℝ \} 
\end{gather*}

\ppart Demostrar que cada familia es una base de una topología sobre $ℝ$.
\ppart Comparar esas topologías.
\ppart Demostrar que la topología generada por $\base_{\leftarrow} ∪ \base_{\rightarrow}$ es la usual.
\solution
\spart Si añadimos $\emptyset$ y el total, entonces tenemos una topología generada por $\base_{\rightarrow}$ y otra generada por $\base_{\leftarrow}$.

\spart  

\spart
\end{problem}

\begin{problem}[11]
Sea $\topl_j$, $j∈J$ una familia de topologías sobre $X$. Demostrar que existe una topología que contiene a todas las $\topl_j$, para $j∈J$ y además es la menos fina de todas las que verifican esta propiedad. 
\solution

Aplicamos directamente la proposición \ref{propTopologiaMinima}: la topología que contiene a todas ellas es \[ \topl = \bigcap_{j∈J} \topl_j \]
\end{problem}

\paragraph{Observación útil para el 5 y el 12:}  
\begin{enumerate}
\item $x \in C(x,\varepsilon)$
\item $\varepsilon_1 > \varepsilon_2 \implies C(x,\varepsilon_2) \subset C(x,\varepsilon_1)$
\end{enumerate}

Y podemos aplicar la propiedad:
\[
A\in\topl \dimplies \forall a\in A \exists \varepsilon > 0 \tlq C(x,\varepsilon)\subseteq A
\]

Haciendo caso al enunciado y haciendo el dibujo vemos que se cumplen las propiedades de base.

Esta topología contiene a la usual pero al revés no, porque para el punto de intersección de las diagonales no existe un abierto de la usual que le contenga.


\paragraph{Pistas para espacios métricos}

(16) Si tengo $d$, una distancia no acotada, puedo definir $d'=\frac{d}{1+d}$, que sigue siendo una distancia, parecida y además acotada.

(17) $\sum \frac{1}{2n} \leq 1$. La clave está en aplicar la desigualdad triangular a cada término del sumatorio. La clave para este problema es el 16.

\section{Hoja 2}

\begin{problem}[1]
 ¿Qué sucesiones convergen en la topología trivial? ¿y en la topología de los complementos contables?
\solution
\spart En $\topl_{triv}$ todas las sucesiones convergen y lo hacen a todos los puntos. ¿Porqué? porque si x∈X, el único abierto que contiene a x es el total.

\spart
La topología de los complementos contables se define así: $A∈\topl_{cocont} \dimplies (A=Ø) \;o\; (X\ A$ es contable (es decir, es finito o numerable).

\obs si $X$ es contable, $\topl_{cocont} = \topl_{disc}$.

\obs Las topologías se podrían definir a partir de los cerrados.

En este caso estamos pidiendo que los cerrados sean los subconjuntos contables  y el total. Comprobamos que es topología utilizando que la unión de numerables es numerables y la intersección de conjuntos numerables (en realidad, con al menos uno numerable es suficiente), es numerable.
 
Vamos a poner algún ejemplo de $\topl_{cocont}$ (porque le parece interesante comentarlo). Tomamos $X=ℝ$.

Si B es contable, $B ∩ (0,1) ≠ Ø$, pues de lo contrario, $(0,1)\subseteq B$ lo cual es imposible por ser $(0,1)$ infinito no numerable y $B$ contable.

Esta topología es otro ejemplo de NO Hausdorff, ya que la intersección de 2 abiertos nunca es vacía (si el total es no numerable).



Una vez acabados comentarios, vamos a contestar a la pregunta que nos hace el ejercicio que es sobre la convergencia de sucesiones.

$x_n \to x \dimplies ∃ n₀ \tq x_n = x ∀n≥ n₀$ (las sucesiones constantes a partir de un término).

Faltaría demostrar las 2 implicaciones.
\end{problem}

\begin{problem}[3]
 Estudiar la convergencia de las sucesiones $x_n = \left(\frac{1}{n}, \frac{1}{n}\right)$ e $y_n = \left(\frac{1}{n}, 1 − \frac{1}{n}\right)$ en $(ℝ^2 , \topl_{ℝ²} ) $, en $(ℝ^2, T_{lex} )$ y en $([0, 1]^2 , \topl_{lex})$.
 
\solution
\begin{itemize}

\item $x_n = \left(\frac{1}{n}, \frac{1}{n}\right)$ en $(ℝ^2 , \topl_{ℝ²} ) $

\item $x_n = \left(\frac{1}{n}, \frac{1}{n}\right)$  en $(ℝ^2, T_{lex} )$

\item $x_n = \left(\frac{1}{n}, \frac{1}{n}\right)$  en $([0, 1]^2 , \topl_{lex})$.

\item $y_n = \left(\frac{1}{n}, 1 − \frac{1}{n}\right)$ en $(ℝ^2 , \topl_{ℝ²} ) $

\item $y_n = \left(\frac{1}{n}, 1 − \frac{1}{n}\right)$ en $(ℝ^2, T_{lex} )$ 

\item $y_n = \left(\frac{1}{n}, 1 − \frac{1}{n}\right)$ en $([0, 1]^2 , \topl_{lex})$.
\end{itemize}
\end{problem}

\begin{problem}[6]
 Sea X un espacio topológico y A, D ⊂ X. Demostrar que:
\ppart Fr (A) = A \ Int (A).
\ppart Fr (A) = Ø si y sólo si A es simultáneamente abierto y cerrado.
\ppart Si A ∩ D = Ø entonces Fr (A ∪ D) = Fr (A) ∪ Fr (D).
\ppart Int (A) ∪ Int (D) ⊂ Int (A ∪ D).
\ppart La inclusión en el apartado anterior puede ser estricta.


\textbf{Indicación: } encontrar $A, D \subset ℝ$ tales que Int (A) ∪ Int (D) = Int (A ∪ D).

\solution

Utilizamos la propiedad $\bar{A ∪ B} = \bar{A} ∪ \bar{B}$
(que se está como ejercicio la demostración en el 11.b)


$\bar{A} ∩ \bar{D} = \emptyset$.

\spart  $B=\bar{ D }^c$ es abierto y $\bar{A}\subseteq B$

Si $x∈\bar{A}$ y $V$ es un entorno de $x$, entonces $V∩B$ es un entorno de $x$, con $V∩B∩\bar{D} = \emptyset$.

\spart 

\spart 

\spart 


\end{problem}

\begin{problem}[11]
\ppart
\ppart

\solution
\spart
\spart  Demostrar que si I es finito entonces $$\bigcup_{i∈I} \bar{A_i} = \bar{\bigcup_{i∈I} A_i}$$

Demostramos las 2 inclusiones.

$\bar{A}∪\bar{B} \subseteq \bar{A∪B}$. Está bastante claro.

$A\subseteq A∪B \subseteq \bar{A∪B}$. Como $\bar{A∪B}$ es cerrado y es más grande que $A$, tiene que contener a $\bar{A}$ (dado que éste es el cerrado más pequeño que lo contiene). El mismo razonamiento para $B$.

La inclusión en el otro sentido se puede demostrar así:

$A∪B \subseteq \underbrace{\bar{A}∪\bar{B}}_{cerrado}$. Entonces, $\bar{A∪B}\subseteq \bar{A}∪\bar{B}$.
\end{problem}

\begin{problem}[14] Explica si las siguientes funciones son continuas.

\ppart[c] Tenemos $X=[0,1]$ con la topología usual, y $Y=[0,1]^2$ con la topología $\topl_{Lex}$, la del orden lexicográfico. Sea $\appl{h}{X}{Y}$ tal que $h(t) = (t,1)$.
\solution

\spart[c] Lo primero que hay que hacer es entender las topologías de los conjuntos. En $X$, los abiertos serán elementos de la base de $ℝ$ intersección $X$, es decir, que la topología será generada por la base \begin{multline*} \base = \{ (a,b) ∩ [0,1] \tq a < b, a,b∈ℝ \} \equiv \\ \equiv \left\{ [0,1], [0,b), (a, 1], (c,d) \tq 0 < b,a < 1,\; 0<c<d<1 \right\} \end{multline*}

En $Y$, la topología será la generada por otra base $\tilde{\base}$. Para comprobar la continuidad, basta con comprobar los elementos de la base: que para todo elemento $B∈\tilde{\base}$, $\inv{h}(B)$ es abierto en la topología usual de $[0,1]$.

Vemos que la imagen de elementos de la base que son abiertos verticales, tanto dentro ($B_2$) como en el borde izquierdo ($B_1$), son el vacío. Ahora bien, la imagen inversa del conjunto $B_3$, en el borde derecho, es $\{1\}$ que no es abierto en la topología usual, de la misma forma que tampoco lo es la imagen inversa de $B_4$, que es un intervalo $[a,b)$ (ver imagen \ref{figH2Ej14})

\begin{figure}[hbtp]
\centering
\inputtikz{E_H2_14}
\caption{Conjuntos de la base en la topología del orden lexicográfico.}
\label{figH2Ej14}
\end{figure}


Hay que tener cuidado, eso sí, en distinguir entre la topología del orden lexicográfico en $[0,1]^2$ y la topología de subespacio del orden lexicográfico en $ℝ^2$. Por ejemplo, un intervalo vertical $(a,b)$ con $0 < a < 1$ y $b > 1$, es un abierto en $\topl_{Lex}$ de $ℝ^2$. Pero al hacer la intersección para la topología del subespacio, nos quedaría que el intervalo $(a, 1]$ es  un abierto, pero este intervalo no está en $\topl_{Lex\;[0,1]^2}$.

Una duda que surge: ¿es una de las topologías menos fina que la otra? Ya hemos visto que la topología del subespacio no está contenida en la del orden lexicográfico, así que faltaría comprobar si $\topl_{Lex\;[0,1]^2} ⊆ \topl_{Lex\; ℝ^2}^{sub}$. Para ello, nos bastaría comprobar si los elementos de la base de $\topl_{Lex\;[0,1]^2}$ son abiertos en $\topl_{Lex\; ℝ^2}^{sub}$. Es decir, hay que comprobar si \[ \base_{Lex\;[0,1]^2} ⊆ \topl_{Lex\;ℝ^2}^{sub} \]

La base del orden lexicográfico son los abiertos verticales (las bandas se generan como unión de intervalos verticales). 

\end{problem}

\begin{problem}[6] Sea $X$ un espacio topológico y $A, D ⊂ X$. Demostrar que
\ppart[c] Si $\adh{A}∩\adh{D} = \emptyset$, entonces $\mop{Fr}(A∪D) = \mop{Fr}(A) ∪ \mop{Fr}(D)$
\ppart[e] $\intr{A} ∪ \intr{D} ⊂ \intr{A∪D}$. 
\solution
\spart[c]

Si $\adh{A}∩\adh{D} = \emptyset$, entonces $\mop{Fr}(A∪D) = \mop{Fr}(A) ∪ \mop{Fr}(D)$. REcordamos que $\mop{Fr}(M) = \adh{M} ∩ \adh{M^c}$. Entonces

\[ \mop{Fr}(A∪D) = \adh{A∪D} ∩ \adh{\left(A∪D\right)^c} = (\adh{A}∪\adh{D}) ∩ \adh{(A^c∩D^c)} = \adh{A}∩(\adh{A^c∩D^c}) ∪ (\adh{D}∩ \adh{A^c∩D^c}) \]

Por simetría basta comprobar que $\mop{Fr}(A) = \adh{A} ∩ \adh{(A^c∩D^c)}$. El contenido hacia la izquierda es sencillo de demostrar

\[ A^c∩D^c ⊆ A^c \implies \adh{A^c∩D^c} ⊆ \adh{A^c}\implies \adh{A} ∩ \adh{A^c∩D^c} ⊆ \adh{A}∩ \adh{A^c} \]

Por el otro lado, si $x∈\adh{A} ∩ \adh{A^c}$, ¿tenemos que $x∈\adh{A} ∩ \adh{A^c∩ D^c}$? 

Hay que comprobar que $x∈\adh{A^c∩D^c}$, o que $∀U$ abierto con $x∈U$, entonce $U∩A^c∩D^c≠\emptyset$.  Vemos que \[ U∩A^c∩D^c⊇ U ∩ A^c∩ \adh{D}^c ≠  \emptyset \], porque si $\adh{D}$ es cerrado, entonces $\adh{D}^c$ es abierto y entonces $U∩\adh{D}^c$ es abierto. Luego si $x∈\adh{A}$. tenemos que $x∉\adh{D}$ y $x∈\adh{D}^c$, luego $x∈U∩\adh{D}^c$, entonces $U∩\adh{D}^c$ es un entorno de $x$.

\spart[e] En la topología usual en $ℝ$, podemos escoger $A=ℚ$ y $D=ℝ\setminus ℚ$, por lo que el contenido puede ser estricto.
\end{problem}

\begin{problem}[8] Encontrar $A ⊂ ℝ$ tal que en la topología usual de $ℝ$ se tenga $\mop{Fr}(A) = \left\{ [1 , 2] ∪ \{ 0 \} ∪ \{1/n \tq n ∈ ℤ^+ \}\right\}$
\solution
\end{problem}

\begin{problem}[9] Indica razonadamente si estas afirmaciones son verdaderas o falsas.
\ppart[e] Si $A∩\mop{Fr}(A) = \emptyset$ entonces $A$ es abierto.
\solution
\spart[e] Cierto. Podemos hacerlo operando o por la definición. Si $A$ no fuese abierto, entonces $∃x∈A$ tal que $∀V$ entorno de $x$, $V\nsubseteq A$, es decir $V∩A^c ≠ \emptyset$. Como además $V ∩ A ≠ \emptyset$, entonces $x∈ \mop{Fr}(A)$, contradicción.
\end{problem}


\begin{problem}[12] Sea \stopl un espacio topológico y $W⊂X$ con la topología de subespacio.
\ppart Sea $D⊂W$, ¿cómo se relaciona la adherencia de $D$ en $W$ con la adherencia de $D$ en $X$?
\ppart Sea $\{x_n\}$ una sucesión en $W$ y $x∈W$. Demostrar que $x_n \to x$ en $W$si y sólo si $x_n \to x$ en $X$.
 
\solution
\spart Queda como ejercicio, pero parece que en la adherencia de $D$ en $X$ podría haber puntos que no estén en $W$.

\spart Si $x_n\to x$ en $X$, entonces $∀V∈\topl_X$ entorno de $x$ se tiene que $∃n_V$ tal que $∀n≥n_V,\; x_n∈V$. 

Por otra parte, si $x_n\to x$ en $W$, entonces $∀V^W ∈ \topl_{W}^{sub}$ de $x$ en $W$ $∃n_W$ tal que $∀n≥ n_W\; x_n ∈ V^W$. Si $V^W$ es abierto en $W$, por definición $V^W = V∩W$ para un $V∈\topl_X$. Luego $x∈V$, por lo que $V$ es entorno de $x$ en $W$, por lo que $∃n_B\tlq ∀n≥n_v$ se tiene que $x_n∈V, x_n∈W$ por lo que $x_n ∈ V^W$.
\end{problem}

\begin{problem}[15] Probar que existen funciones de $(ℝ, \topl_{[,)})$ en $ℕ$ con la topología discreta que son sobreyectivas y continuas, pero que no existen funciones de $(ℝ, \topl_{[,)})$ en $ℝ$ con la topología discreta que tengan tales propiedades. 
\solution

En el primer caso, vemos qué es la topología discreta en $ℕ$, que no es más que una topología $\topl_\base$ con $\base = \left\{ \{n\} \tq n∈ℕ\right\}$. $f$ será continua si y sólo si $\inv{f}(B) ∈ \topl_{[,)}\; ∀B∈\base$, y sobreyectiva si y sólo si $\inv{f}(B) ≠ \emptyset\; ∀B∈\base$. Además, los $\inv{f}(B)$ tienen que ser disjuntos, y por otra parte $\bigcup \inv{f}(B) = ℝ$. 

Es decir, las imágenes inversas de los elementos de la base (esto es, de los naturales) nos van a dar una partición de $ℝ$ por los abiertos de $\topl_{[,)}$. Podemos encontrar entonces una función que los cumpla, como por ejemplo 

\begin{align*}
\inv{f}(2k) &= [k-1, k)  \\
\inv{f}(2k-1) &= [-k, -k+1) 
\end{align*}

Ahora bien, ¿qué ocurre en el segundo caso? No es cuestión de cardinales, en $ℝ$ tenemos ``espacio'' de sobra (podemos descomponer $ℝ$ en una familia no numerable de conjuntos no numberables).

Más bien, el problema tiene que ver con la numerabilidad. Si pudiese definir una función sobreyectiva de $ℝ$ con la topología $\topl_{[,)}$ a $ℝ$ con la topología discreta, entonces tendríamos imágenes inversas $\inv{f}(\{c\}) = A_c$, con $A_c$ disjuntos, no vacíos y abiertos en $\topl_{[,)}$, y por supuesto con $ℝ = \bigcup_{c∈ℝ} A_c$.

El conflicto está en encontrarlos disjuntos y que cubran todo $ℝ$. Al ser $A_c ≠ \emptyset$, entonces $∃[a_c, b_c) ⊆ A_c$. Es imposible tener $[a_c, b_c)$ disjuntos no vacíos con $c∈ℝ$.

Si $[a_c, b_c) ≠ \emptyset$, entonces $∃q_c∈ℚ$ con $q_c∈[a_c, b_c)$. Al ser los $[a_c, b_c)$ disjuntos, los $q_c$ son todos distintos, luego debería de haber una aplicación $\appl{ℝ}{ℚ}$ tal que $g(c) = q_c$ inyectiva, pero sería una contradicción porque entonecs $\card{ℚ} ≥ \card{ℝ}$.
\end{problem}

\begin{problem}[0]
¿Cuál es la topología producto de $ℝ_{disc} × ℝ_{usual}$?
\solution

¿Cómo construimos la topología producto? Nos basta con coger los elementos de la base de ambos. Como recordatorio:

\[ \topl_{\base_1} \otimes \topl_{\base_2} = \topl_{\base_1 \otimes \base_2} \] donde $\base_1 \otimes \base_2 = \left\{ B_1 × B_2 \tq B_1 ∈ \base_1, B_2 \base_2 \right\}$.

La base $\base_1$ de la topología discreta es $\base_1=\left\{\{x\} \tq x∈ℝ\right\}$, y la base $\base_2$ de la topología usual son los abiertos $(a,b)$. Entonces $B_1 × B_2 = \{x\} × (a,b)$, intervalos abiertos verticales. Casualmente, esta es la topología $\topl_{Lex}$ del orden lexicográfico en $ℝ^2$.
\end{problem}

\begin{problem}[0] Sea $\appl{f}{ℝ}{ℝ}$ continua y biyectiva. ¿Es un homeomorfismo?
\solution

Lo único que le falta para ser homeomorfismo es que la inversa sea continua. Si $\inv{f}$ es continua, entonces la imagen inversa por $\inv{f}$ de un abierto es abierta. Es decir, hay que demostrar que $f$ de un abierto es abierto. Y como tenemos una base, nos basta comprobar sólo para los elementos de la base.

Una función $f$ continua e inyectiva en $ℝ$ debe ser monótona estrictamente creciente o decreciente. Supongamos que es creciente, entonces $f((a,b)) = (f(a), f(b))$.

\end{problem}