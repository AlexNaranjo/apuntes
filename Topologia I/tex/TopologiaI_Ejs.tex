
% -*- root: ../TopologiaI.tex -*-
\section{Hoja 1}

\begin{problem}[1]
Sea $X=\{a,b,c\}$ un conjunto de 3 elementos. Encontrar todas las topologías sobre $X$.
\solution

Las topologías más sencillas son: $\topl_1 = \{\emptyset,X,a\}$. Es trivial comprobar que cumple las 3 propiedades de topología. Lo mismo con $b$, con $c$, con ${a,b}$, con ${b,c}$ y ${a,c}$.

El siguiente nivel de complejidad son aquellas con 2 elementos: $\topl_4 = \{\emptyset,a,b,X\}$. Para que esto fuera topología faltaría $\{a\} \cup \{b\}$, con lo que $\topl_4 = {\emptyset,a,b,\{a,b\}X}$

Y así acaban saliendo todas.
\end{problem}

\begin{problem}[2]
En $\real$ se considera la $\topl =\{(-\infty,a): -\infty \leq a \leq \infty\}$. Demostrar que es una topología.
\solution

Tenemos que demostrar las 3 propiedades de topología:

\begin{enumerate}
\item $\emptyset,\real \in \topl$. Basta tomar $a=\pm \infty$ para tener ambas pertenencias.
\item $A,B\in\topl \implies A\cap B\in\topl$. Dados $a_1,a_2\in\real$, con $a_1\neq a_2$ tenemos $A_1 = (-\infty,a_1)$ y $A_2 = (-\infty,a_2)$.

Entonces, $A_1\cap A_2 = (.-\infty,\min(a_1,a_2)) = \left\{\begin{array}{cc}
A_1\cap A_2 = A_1 & si\; a_1 = \min(a_1,a_2)\\A_1\cap A_2 = A_2  & si\; a_2 = \min(a_1,a_2)
\end{array}\right.$

Hemos demostrado que $A_1 \cap A_2 \in \topl$.

\item $A,B\in\topl \implies A\cup B\in\topl$. Lo mismo que el apartado anterior tomando máximos en vez de mínimos.
\end{enumerate}
\end{problem}


\begin{problem}[3]
Sean $X$ un conjunto infinito y $\topl$ una topología sobre $X$ en la que todos los subconjuntos infinitos son abiertos. Demostrar que $\topl$ es la topología discreta de $X$.
\solution

Tenemos que demostrar que $\topl = \topl_{disc} = \mathcal{P}(X)$
\end{problem}

\begin{problem}[4]
 Sea $X$ un conjunto con más de dos elementos.
\ppart Definir dos topologías $\topl_1,\topl_2$ sobre $X$ de modo que $\topl_1 \cup \topl_2$ no sea una topología.
\ppart Sea $\topl_j , j ∈ J$ una familia de topologías sobre $X$. Probar que
 $\bigcap_{j∈J} Tj$ es también una topología sobre $X$.

\solution
\spart Sea $X = \{a,b,c\}$. Definiendo $\topl_1 = \{\emptyset,X,a\}$ y $\topl_2=\{\emptyset,X,b\}$.

$\topl_1 \cup \topl_2 = \topl_{\cup}= \{\emptyset,X,a,b\}$ no es topología, porque $\{a\} \cup \{b\} = \{a,b\} \notin \topl_{\cup}$
\spart
Probándolo para 2 topologías cualesquiera, lo habremos probado para todas, pues por inducción, si se cumple para 2 se cumple para una cantidad finita.

Tenemos que comprobar las 3 propiedades de topología:

\begin{enumerate}
\item $\emptyset,\real \in \topl$. Como $\emptyset,X $ pertenecen a ambas $\topl_1,\topl_2 \implies \emptyset,X \in \topl_{\cap}$

\item $A_1,A_2\in\topl_{\cap} \implies A_1\cap A_2\in\topl_{\cap}$.\\
\[\left.\begin{array}{c}
A_1\in\topl_{\cap}\implies 
\left\{
	\begin{array}{cc}
		A_1\in\topl_1\\A_1\in\topl_2
	\end{array}
\right.\\
 \text{Lo mismo con } A_2.
\end{array}\right\}\implies \begin{array}{c}
A_1\cap A_2\in\topl_1\\
A_1\cap A_2\in \topl_2\end{array}\implies A_1\cap A_2 \in \topl_{\cap}\]


\item $A_1,A_2\in\topl_{\cup} \implies A\cup A_2\in\topl_{\cup}$. El mismo razonamiento es válido para la unión.
\end{enumerate}
\end{problem}

\begin{problem}[5]
 En el plano $\real^2$ se considera la familia $\topl$ de todos los subconjuntos $U$ tales que para cada punto
$(a, b) \in U\; \exists  ε > 0 \tlq ((a − ε, a + ε) × b) ∪ (a × (b − ε, b + ε)) \subset U$.

Estudiar si $\topl$ es una topología en $\real^2$.
\solution

Gráficamente es fácil de contestar. Los subconjuntos pedidos son cruces y la intersección de 2 cruces no es una cruz, con lo que no puede ser topología.
\end{problem}


\begin{problem}[6] Sea $\appl{g}{X}{Y}$ una aplicación entre dos conjuntos.

\ppart Demostrar que si $\topl$ es una topología en $X$ entonces \[ \mathcal{S} = \{ E ⊆ Y \tq \inv{g}(E) ∈ \topl \} \] es una topología en $Y$.
\ppart Demostrar que si $\mathcal{S}$ es una topología en $Y$ entonces \[ \mathcal{U} = \{ \inv{g}(E) \tq E ∈ \mathcal{S} \} \]es una topología en $X$.

\solution
\spart Vamos a demostrar que es una topología, para lo cual tenemos que comprobar las 3 propiedades (ver \ref{defTopología}):

Es importante saber que las aplicaciones entre conjuntos se definen en todo el dominio, no en un subconjunto, es decir, $∀ x ∈ X, ∃g(x)∈Y$

\begin{enumerate}
\item Tomando $E=Y$, tenemos $Y∈ \mathcal{S}$ por ser $g$ una aplicación de conjuntos tal que $g^{-1}(Y)=X$,=. (La imagen inversa de todo Y pertenece a la topología de X, ya que es X.)

Tomando $E=∅$, tenemos que $g^{-1}(Ø) = Ø \in \topl$, porque no puede existir un $x\in X\tlq g(x)=Ø$ por ser $g$ aplicación de conjuntos (todos los elementos tienen que tener una imagen del cunjunto destino).

\item $A,B \in \mathcal{S} \dimplies g^{-1}(A),g^{-1}(B) \in \mathcal{T}$. (1)

$A\cap B \in\mathcal{S} \dimplies g^{-1}(A\cap B) \in \mathcal{T}$.(2)

Si tuvieramos que (1) $\implies$ (2) ya lo tendríamos demostrado. Vamos a demostrar que $g^{-1}(A),g^{-1}(B) \in \mathcal{T} \implies g^{-1}(A\cap B) \in \mathcal{T}$.

Para ello: $g^{-1}(A\cap B) = g^{-1}(A)\cap g^{-1}(B)$. No es difícil convencernos de esta igualdad. Para resolver las dudas, vamos a demostrar las 2 inclusiones (una en cada sentido). 

\paragraph{$\subset$}
\begin{gather*}
g^{-1}(A\cap B) \subset g^{-1}(A)\cap g^{-1}(B) \implies
x∈g^{-1}(A∩B) \implies ∃y∈A∩B\tq g^{-1}(y)=x\\
\implies\left| \begin{array}{c}
y∈A \implies g^{-1}(y)=x∈g^{-1}(A)\\
y∈B \implies g^{-1}(y)=x∈g^{-1}(B)
\end{array}
\right.
\end{gather*}

\paragraph{$⊃$}


\begin{gather*}
g^{-1}(A∩B) ⊃ g^{-1}(A)∩g^{-1}(B) \implies x∈g^{-1}(A)∩g^{-1}(B)\implies \\
\implies
\left|\begin{array}{cc}
x∈g^{-1}(A) \implies ∃ y_a ∈ A\tq g^{-1}(y_a)=x\\
x∈g^{-1}(B) \implies ∃ y_b ∈ B\tq g^{-1}(y_b)=x
\end{array}\right. \underset{g\, aplicacion}{\implies} y_a=y_b\\
\end{gather*}
Hemos demostrado: $$∀x∈g^{-1}(A)∩g^{-1}(B)∃y∈A∩B \tq g^{-1}(y)=x∈g^{-1}(A∩B)$$


Es importante el comentario de que esto para imágenes directas no funciona.

\item Demostramos ahora que la unión de abiertos está en la topología. Si $A, B ∈ \mathcal{S}$, entonces $\inv{g}(A), \inv{g}(B) ∈ \topl$. Como $\topl$ es topología, tenemos que $\inv{g}(A) ∪ \inv{g}(B) ∈ \topl$, lo que implica (por el mismo razonamiento que antes) que \\$\inv{g}(A∪B) ∈ \topl$ y por lo tanto $A∪B ∈ \mathcal{S}$.
\end{enumerate}

\spart

\end{problem}


\begin{problem}[7]
Sean $X$ un conjunto y $a$ un elemento de $X$. Se considera la familia $\topl_a$ de los subconjuntos $U\subset X$ tales que o bien $U = \emptyset$, o bien $a ∈ U$ . Estudiar si $\topl_a$ es una topología en $X$.
\solution

Comprobamos las 3 propiedades de topología:
\begin{enumerate}
\item
\begin{itemize}
\item $∅∈\topl_a$: Sí, por definición de $\topl_a$.
\item Tomando $U=X$, tenemos que $X\in\topl_a \impliedby a\in X$.
\end{itemize}
\item Sean $U₁,U₂∈X.$  Tenemos que demostrar si $a∈U₁∩U₂$. En este caso está claro dado que a∈U₁ y a∈U₂.
\item Lo mismo con la unión.
\end{enumerate}

Podemos concluir que $\topl_a$ es una topología en $X$.

\end{problem}


\begin{problem}[9] Se consideran las siguientes familias de conjuntos en $ℝ$:

\begin{gather*}
\base_{\leftarrow} = \{ (-∞, b) \tq b ∈ ℝ \} \\
\base_{\rightarrow} = \{ (a,∞) \tq  ∈ ℝ \} 
\end{gather*}

\ppart Demostrar que cada familia es una base de una topología sobre $ℝ$.
\ppart Comparar esas topologías.
\ppart Demostrar que la topología generada por $\base_{\leftarrow} ∪ \base_{\rightarrow}$ es la usual.
\solution
\spart Recordamos las propiedades que tiene que cumplir para ser base: \ref{defBase}

En este caso, $$\begin{array}{cc}
∀x∈ℝ, x∈\mathcal{B}_1 & \text{ basta con tomar } b=x+1\\
∀x∈ℝ, x∈\mathcal{B}_2 & \text{ basta con tomar } a=x-1
\end{array}$$

La segunda propiedad la tenemos sin darle muchas vueltas: $b_1,b_2∈\mathcal{B}_1\; b_1∩b_2 = max\{b_1,b_2\} ∈\mathcal{B}_1$. Ni siquiera tenemos que encontrar un elemento de la base que lo contenga. Lo mismo para $\mathcal{B}_2$.

\textbf{Importante: } Observamos que es imprescindible mencionar que tenemos que añadir el vacío y el total a ambas bases para que sean realmente bases.

\spart  
\spart Dado que $\base_{\leftarrow} ∪ \base_{\rightarrow}$  es una base (sino, no podríamos hablar de topología generada por $\base_{\leftarrow} ∪ \base_{\rightarrow}$) tenemos que la intersección de 2 elementos está contenido en un elemento de la base, que a su vez está contenido en la intersección (\ref{defTopologiaGeneradaBase})

Si $(c,d) ∈ \toplb \implies ∀x∈(c,d) ∃ (-∞,b)\tlq x∈B\subseteq (c,d)$. Es imposible para los 2 casos, con lo que, para que sea base, elementos del tipo $(c,d)$ tienen que estar incluidos en $\base_{\leftarrow} ∪ \base_{\rightarrow}$, dando lugar a la topología usual (tienen la misma base).

\end{problem}


\begin{problem}[10]
Probar que si $\mathcal{B}$ es una base para una topología sobre $X$, entonces la topología $\toplb$ generada por
$\mathcal{B}$ es igual a la intersección de todas las topologías sobre $X$ que contienen a $\mathcal{B}$.
\solution

Sea $\topl$ una topología sobre $X$. Queremos demostrar que si $\mathcal{B}$ base tal que $\toplb = \topl \implies \toplb = \bigcap \{\topl_X \tlq \mathcal{B}∈\topl_X\quad ∀x\}$.


Para demostrar la igualdad, demostraremos $\toplb \subseteq \bigcap \topl_X$ y $\bigcap \topl_X \subseteq \toplb$. Recordamos que todas las cosas que tengamos que probar para una topología generada por una base, basta con probarla para los elementos de la base.

$\toplb \subseteq \bigcap \topl_X$ $$∀A\in\toplb \implies A∈\bigcap\topl_x \impliedby  A ∈\mathcal{B} \subset \topl_x ∀x$$


$\bigcap \topl_X \subseteq \toplb$ $$\text{Esta propiedad es obvia, dado que } \toplb \subseteq \{\bigcap \topl_x\}$$
\end{problem}

\begin{problem}[11]
Sea $\topl_j$, $j∈J$ una familia de topologías sobre $X$. Demostrar que existe una topología que contiene a todas las $\topl_j$, para $j∈J$ y además es la menos fina de todas las que verifican esta propiedad. 
\solution

Aplicamos directamente la proposición \ref{propTopologiaMinima}: la topología que contiene a todas ellas es \[ \topl = \bigcap_{j∈J} \topl_j \]
\end{problem}

\paragraph{Observación útil para el 12 (y para el 5):}  
\begin{enumerate}
\item $x \in C(x,\varepsilon)$
\item $\varepsilon_1 > \varepsilon_2 \implies C(x,\varepsilon_2) \subset C(x,\varepsilon_1)$
\end{enumerate}

Y podemos aplicar la propiedad:
\[
A\in\topl \dimplies \forall a\in A \exists \varepsilon > 0 \tlq C(x,\varepsilon)\subseteq A
\]

Haciendo caso al enunciado y haciendo el dibujo vemos que se cumplen las propiedades de base.

Esta topología contiene a la usual pero al revés no, porque para el punto de intersección de las diagonales no existe un abierto de la usual que le contenga.



\begin{problem}[13]
 Sea $(X, d)$ un espacio métrico. Demostrar que, para cualesquiera x, y, x e y elementos de X, se cumple
$$|d(x, y) − d(x' , y' )| ≤ d(x, x' ) + d(y, y' )$$
Deducir de ello que $\displaystyle\lim_{n→∞} d(x_n , y_n ) = d(x, y)$ cuando $\displaystyle\lim_{n→∞} d(x_n , x) = 0 = \lim{n→∞} d(y_n , y).$
\solution

$$|a-b| = |a+(-b)| ≤ |a| + |-b| = |a|+|b|$$

\end{problem}




\begin{problem}[14]
Estudiar si $(ℝ, d)$ es un espacio métrico, donde $d : ℝ × ℝ → ℝ$ está definida como
$$d(x, y) = \left\{
	\begin{array}{cr} 
		0 & si \quad x=y\\
		|x| + |y| & si \quad x≠y
	\end{array}
\right.$$


Dibujar la bola B(x, r) para los casos
\ppart x = 0 y radio r = 1/2
\ppart x = 1/2 y r = 1.

\solution

Para comprobar que es una distancia tenemos que comprobar \ref{defEspacioMetrico}

\begin{itemize}
\item $d(x,x) = 0$. Sí
\item $d(x,y) ≥ 0$. Sí, de hecho con $x≠y$ tenemos $d(x,y)>0$.
\item Desigualdad triangular: $d(x,z) \leq d(x,y) + d(y,z)\; ∀x,y,z∈X$. En este caso,  $\dst(x,y) + \dst(y,z) = |x|+|y| + |y| + |z| ≥ |x|+|z|$
\end{itemize}

Hemos comprobado que tenemos un espacio métrico. Vamos a dibujar las bolas.

\spart

\begin{figure}[hbtp]
\centering
\inputtikz{E_H1_14_A}
\caption{$B(0,\frac{1}{2})$}
\label{H1_E14}
\end{figure}

\spart
\begin{figure}[hbtp]
\centering
\inputtikz{E_H1_14_B}
\caption{$B(\frac{1}{2},1)$}
\label{H1_E14}
\end{figure}
\end{problem}

\begin{problem}[15]
Demostrar que si $d_1$ es una distancia entonces $d_2(x, y) = \min\{ (d_1(x, y), 1)\}$ también lo es y que ambas distancias inducen la misma topología.
\solution

Es inmediato comprobar las 3 propiedades de distancia.

Vamos a comprobar que inducen la misma topología. Para ello, basta comprobar que las 2 métricas generan 2 bases que inducen la misma topología. Al empezar a tratar con bases, conseguimos hacer la demostración solamente para los elementos de la base.

Dada $B_{d₁}(x.ε)\quad ∀y∈B_{d₁}(x,ε)$ es posible encontrar una bola $B_{d₂}(y,δ)$ tal que $$y∈B_{d₂}(y,δ) \subseteq B_{d₁}(x,ε)$$ y viceversa, con lo que las bases generadas por las métricas son iguales $\implies$ inducen la misma topología.
\end{problem}

\paragraph{Pistas para los siguientes}

(16) Si tengo $d$, una distancia no acotada, puedo definir $d'=\frac{d}{1+d}$, que sigue siendo una distancia, parecida y además acotada.

(17) $\sum \frac{1}{2n} \leq 1$. La clave está en aplicar la desigualdad triangular a cada término del sumatorio. La clave para este problema es el 16.



\section{Hoja 2}

\begin{problem}[1]
 ¿Qué sucesiones convergen en la topología trivial? ¿y en la topología de los complementos contables?
\solution
\spart En $\topl_{triv}$ todas las sucesiones convergen y lo hacen a todos los puntos. ¿Porqué? porque si x∈X, el único abierto que contiene a x es el total.

\spart
La topología de los complementos contables se define así: $A∈\topl_{cocont} \dimplies (A=Ø) \;o\; (X\ A$ es contable (es decir, es finito o numerable).

\obs si $X$ es contable, $\topl_{cocont} = \topl_{disc}$.

\obs Las topologías se podrían definir a partir de los cerrados.

En este caso estamos pidiendo que los cerrados sean los subconjuntos contables  y el total. Comprobamos que es topología utilizando que la unión de numerables es numerables y la intersección de conjuntos numerables (en realidad, con al menos uno numerable es suficiente), es numerable.
 
Vamos a poner algún ejemplo de $\topl_{cocont}$ (porque le parece interesante comentarlo). Tomamos $X=ℝ$.

Si B es contable, $B ∩ (0,1) ≠ Ø$, pues de lo contrario, $(0,1)\subseteq B$ lo cual es imposible por ser $(0,1)$ infinito no numerable y $B$ contable.

Esta topología es otro ejemplo de NO Hausdorff, ya que la intersección de 2 abiertos nunca es vacía (si el total es no numerable).



Una vez acabados comentarios, vamos a contestar a la pregunta que nos hace el ejercicio que es sobre la convergencia de sucesiones.

$x_n \to x \dimplies ∃ n₀ \tq x_n = x ∀n≥ n₀$ (las sucesiones constantes a partir de un término).

Faltaría demostrar las 2 implicaciones.
\end{problem}

\begin{problem}[3]
 Estudiar la convergencia de las sucesiones $x_n = \left(\frac{1}{n}, \frac{1}{n}\right)$ e $y_n = \left(\frac{1}{n}, 1 − \frac{1}{n}\right)$ en $(ℝ^2 , \topl_{ℝ²} ) $, en $(ℝ^2, T_{lex} )$ y en $([0, 1]^2 , \topl_{lex})$.
 
\solution
\begin{itemize}

\item $x_n = \left(\frac{1}{n}, \frac{1}{n}\right)$ en $(ℝ^2 , \topl_{ℝ²} ) $

\item $x_n = \left(\frac{1}{n}, \frac{1}{n}\right)$  en $(ℝ^2, T_{lex} )$

\item $x_n = \left(\frac{1}{n}, \frac{1}{n}\right)$  en $([0, 1]^2 , \topl_{lex})$.

\item $y_n = \left(\frac{1}{n}, 1 − \frac{1}{n}\right)$ en $(ℝ^2 , \topl_{ℝ²} ) $

\item $y_n = \left(\frac{1}{n}, 1 − \frac{1}{n}\right)$ en $(ℝ^2, T_{lex} )$ 

\item $y_n = \left(\frac{1}{n}, 1 − \frac{1}{n}\right)$ en $([0, 1]^2 , \topl_{lex})$.
\end{itemize}
\end{problem}

\begin{problem}[6]
 Sea X un espacio topológico y A, D ⊂ X. Demostrar que:
\ppart $\mop{Fr} (A) = A \ \mop{Int} (A)$.
\ppart $\mop{Fr} (A) = ∅$ si y sólo si $A$ es simultáneamente abierto y cerrado.
\ppart Si $A ∩ D = ∅$ entonces $\mop{Fr} (A ∪ D) = \mop{Fr} (A) ∪ \mop{Fr} (D)$.
\ppart $\mop{Int} (A) ∪ \mop{Int} (D) ⊂ \mop{Int} (A ∪ D)$.
\ppart La inclusión en el apartado anterior puede ser estricta.


\textbf{Indicación: } encontrar $A, D \subset ℝ$ tales que $\mop{Int} (A) ∪ \mop{Int} (D) = \mop{Int} (A ∪ D)$.

\solution

Utilizamos la propiedad $\bar{A ∪ B} = \bar{A} ∪ \bar{B}$
(que se está como ejercicio la demostración en el 11.b)


$\bar{A} ∩ \bar{D} = \emptyset$.

\spart  $B=\bar{ D }^c$ es abierto y $\bar{A}\subseteq B$

Si $x∈\bar{A}$ y $V$ es un entorno de $x$, entonces $V∩B$ es un entorno de $x$, con $V∩B∩\bar{D} = \emptyset$.

\spart 

\spart 

\spart 


\end{problem}

\begin{problem}[11]
\ppart
\ppart

\solution
\spart
\spart  Demostrar que si I es finito entonces $$\bigcup_{i∈I} \bar{A_i} = \bar{\bigcup_{i∈I} A_i}$$

Demostramos las 2 inclusiones.

$\bar{A}∪\bar{B} \subseteq \bar{A∪B}$. Está bastante claro.

$A\subseteq A∪B \subseteq \bar{A∪B}$. Como $\bar{A∪B}$ es cerrado y es más grande que $A$, tiene que contener a $\bar{A}$ (dado que éste es el cerrado más pequeño que lo contiene). El mismo razonamiento para $B$.

La inclusión en el otro sentido se puede demostrar así:

$A∪B \subseteq \underbrace{\bar{A}∪\bar{B}}_{cerrado}$. Entonces, $\bar{A∪B}\subseteq \bar{A}∪\bar{B}$.
\end{problem}

\begin{problem}[14] Explica si las siguientes funciones son continuas.

\ppart[c] Tenemos $X=[0,1]$ con la topología usual, y $Y=[0,1]^2$ con la topología $\topl_{Lex}$, la del orden lexicográfico. Sea $\appl{h}{X}{Y}$ tal que $h(t) = (t,1)$.
\solution

\spart[c] Lo primero que hay que hacer es entender las topologías de los conjuntos. En $X$, los abiertos serán elementos de la base de $ℝ$ intersección $X$, es decir, que la topología será generada por la base \begin{multline*} \base = \{ (a,b) ∩ [0,1] \tq a < b, a,b∈ℝ \} \equiv \\ \equiv \left\{ [0,1], [0,b), (a, 1], (c,d) \tq 0 < b,a < 1,\; 0<c<d<1 \right\} \end{multline*}

En $Y$, la topología será la generada por otra base $\tilde{\base}$. Para comprobar la continuidad, basta con comprobar los elementos de la base: que para todo elemento $B∈\tilde{\base}$, $\inv{h}(B)$ es abierto en la topología usual de $[0,1]$.

Vemos que la imagen de elementos de la base que son abiertos verticales, tanto dentro ($B_2$) como en el borde izquierdo ($B_1$), son el vacío. Ahora bien, la imagen inversa del conjunto $B_3$, en el borde derecho, es $\{1\}$ que no es abierto en la topología usual, de la misma forma que tampoco lo es la imagen inversa de $B_4$, que es un intervalo $[a,b)$ (ver imagen \ref{figH2Ej14})

\begin{figure}[hbtp]
\centering
\inputtikz{E_H2_14}
\caption{Conjuntos de la base en la topología del orden lexicográfico.}
\label{figH2Ej14}
\end{figure}


Hay que tener cuidado, eso sí, en distinguir entre la topología del orden lexicográfico en $[0,1]^2$ y la topología de subespacio del orden lexicográfico en $ℝ^2$. Por ejemplo, un intervalo vertical $(a,b)$ con $0 < a < 1$ y $b > 1$, es un abierto en $\topl_{Lex}$ de $ℝ^2$. Pero al hacer la intersección para la topología del subespacio, nos quedaría que el intervalo $(a, 1]$ es  un abierto, pero este intervalo no está en $\topl_{Lex\;[0,1]^2}$.

Una duda que surge: ¿es una de las topologías menos fina que la otra? Ya hemos visto que la topología del subespacio no está contenida en la del orden lexicográfico, así que faltaría comprobar si $\topl_{Lex\;[0,1]^2} ⊆ \topl_{Lex\; ℝ^2}^{sub}$. Para ello, nos bastaría comprobar si los elementos de la base de $\topl_{Lex\;[0,1]^2}$ son abiertos en $\topl_{Lex\; ℝ^2}^{sub}$. Es decir, hay que comprobar si \[ \base_{Lex\;[0,1]^2} ⊆ \topl_{Lex\;ℝ^2}^{sub} \]

La base del orden lexicográfico son los abiertos verticales (las bandas se generan como unión de intervalos verticales). 

\end{problem}

\begin{problem}[6] Sea $X$ un espacio topológico y $A, D ⊂ X$. Demostrar que
\ppart[c] Si $\adh{A}∩\adh{D} = \emptyset$, entonces $\mop{Fr}(A∪D) = \mop{Fr}(A) ∪ \mop{Fr}(D)$
\ppart[e] $\intr{A} ∪ \intr{D} ⊂ \intr{A∪D}$. 
\solution
\spart[c]

Si $\adh{A}∩\adh{D} = \emptyset$, entonces $\mop{Fr}(A∪D) = \mop{Fr}(A) ∪ \mop{Fr}(D)$. REcordamos que $\mop{Fr}(M) = \adh{M} ∩ \adh{M^c}$. Entonces

\[ \mop{Fr}(A∪D) = \adh{A∪D} ∩ \adh{\left(A∪D\right)^c} = (\adh{A}∪\adh{D}) ∩ \adh{(A^c∩D^c)} = \adh{A}∩(\adh{A^c∩D^c}) ∪ (\adh{D}∩ \adh{A^c∩D^c}) \]

Por simetría basta comprobar que $\mop{Fr}(A) = \adh{A} ∩ \adh{(A^c∩D^c)}$. El contenido hacia la izquierda es sencillo de demostrar

\[ A^c∩D^c ⊆ A^c \implies \adh{A^c∩D^c} ⊆ \adh{A^c}\implies \adh{A} ∩ \adh{A^c∩D^c} ⊆ \adh{A}∩ \adh{A^c} \]

Por el otro lado, si $x∈\adh{A} ∩ \adh{A^c}$, ¿tenemos que $x∈\adh{A} ∩ \adh{A^c∩ D^c}$? 

Hay que comprobar que $x∈\adh{A^c∩D^c}$, o que $∀U$ abierto con $x∈U$, entonce $U∩A^c∩D^c≠\emptyset$.  Vemos que \[ U∩A^c∩D^c⊇ U ∩ A^c∩ \adh{D}^c ≠  \emptyset \], porque si $\adh{D}$ es cerrado, entonces $\adh{D}^c$ es abierto y entonces $U∩\adh{D}^c$ es abierto. Luego si $x∈\adh{A}$. tenemos que $x∉\adh{D}$ y $x∈\adh{D}^c$, luego $x∈U∩\adh{D}^c$, entonces $U∩\adh{D}^c$ es un entorno de $x$.

\spart[e] En la topología usual en $ℝ$, podemos escoger $A=ℚ$ y $D=ℝ\setminus ℚ$, por lo que el contenido puede ser estricto.
\end{problem}

\begin{problem}[8] Encontrar $A ⊂ ℝ$ tal que en la topología usual de $ℝ$ se tenga $\mop{Fr}(A) = \left\{ [1 , 2] ∪ \{ 0 \} ∪ \{1/n \tq n ∈ ℤ^+ \}\right\}$
\solution
\end{problem}

\begin{problem}[9] Indica razonadamente si estas afirmaciones son verdaderas o falsas.
\ppart[e] Si $A∩\mop{Fr}(A) = \emptyset$ entonces $A$ es abierto.
\solution
\spart[e] Cierto. Podemos hacerlo operando o por la definición. Si $A$ no fuese abierto, entonces $∃x∈A$ tal que $∀V$ entorno de $x$, $V\nsubseteq A$, es decir $V∩A^c ≠ \emptyset$. Como además $V ∩ A ≠ \emptyset$, entonces $x∈ \mop{Fr}(A)$, contradicción.
\end{problem}


\begin{problem}[12] Sea \stopl un espacio topológico y $W⊂X$ con la topología de subespacio.
\ppart Sea $D⊂W$, ¿cómo se relaciona la adherencia de $D$ en $W$ con la adherencia de $D$ en $X$?
\ppart Sea $\{x_n\}$ una sucesión en $W$ y $x∈W$. Demostrar que $x_n \to x$ en $W$si y sólo si $x_n \to x$ en $X$.
 
\solution
\spart Queda como ejercicio, pero parece que en la adherencia de $D$ en $X$ podría haber puntos que no estén en $W$.

\spart Si $x_n\to x$ en $X$, entonces $∀V∈\topl_X$ entorno de $x$ se tiene que $∃n_V$ tal que $∀n≥n_V,\; x_n∈V$. 

Por otra parte, si $x_n\to x$ en $W$, entonces $∀V^W ∈ \topl_{W}^{sub}$ de $x$ en $W$ $∃n_W$ tal que $∀n≥ n_W\; x_n ∈ V^W$. Si $V^W$ es abierto en $W$, por definición $V^W = V∩W$ para un $V∈\topl_X$. Luego $x∈V$, por lo que $V$ es entorno de $x$ en $W$, por lo que $∃n_B\tlq ∀n≥n_v$ se tiene que $x_n∈V, x_n∈W$ por lo que $x_n ∈ V^W$.
\end{problem}

\begin{problem}[15] Probar que existen funciones de $(ℝ, \topl_{[,)})$ en $ℕ$ con la topología discreta que son sobreyectivas y continuas, pero que no existen funciones de $(ℝ, \topl_{[,)})$ en $ℝ$ con la topología discreta que tengan tales propiedades. 
\solution

En el primer caso, vemos qué es la topología discreta en $ℕ$, que no es más que una topología $\topl_\base$ con $\base = \left\{ \{n\} \tq n∈ℕ\right\}$. $f$ será continua si y sólo si $\inv{f}(B) ∈ \topl_{[,)}\; ∀B∈\base$, y sobreyectiva si y sólo si $\inv{f}(B) ≠ \emptyset\; ∀B∈\base$. Además, los $\inv{f}(B)$ tienen que ser disjuntos, y por otra parte $\bigcup \inv{f}(B) = ℝ$. 

Es decir, las imágenes inversas de los elementos de la base (esto es, de los naturales) nos van a dar una partición de $ℝ$ por los abiertos de $\topl_{[,)}$. Podemos encontrar entonces una función que los cumpla, como por ejemplo 

\begin{align*}
\inv{f}(2k) &= [k-1, k)  \\
\inv{f}(2k-1) &= [-k, -k+1) 
\end{align*}

Ahora bien, ¿qué ocurre en el segundo caso? No es cuestión de cardinales, en $ℝ$ tenemos ``espacio'' de sobra (podemos descomponer $ℝ$ en una familia no numerable de conjuntos no numberables).

Más bien, el problema tiene que ver con la numerabilidad. Si pudiese definir una función sobreyectiva de $ℝ$ con la topología $\topl_{[,)}$ a $ℝ$ con la topología discreta, entonces tendríamos imágenes inversas $\inv{f}(\{c\}) = A_c$, con $A_c$ disjuntos, no vacíos y abiertos en $\topl_{[,)}$, y por supuesto con $ℝ = \bigcup_{c∈ℝ} A_c$.

El conflicto está en encontrarlos disjuntos y que cubran todo $ℝ$. Al ser $A_c ≠ \emptyset$, entonces $∃[a_c, b_c) ⊆ A_c$. Es imposible tener $[a_c, b_c)$ disjuntos no vacíos con $c∈ℝ$.

Si $[a_c, b_c) ≠ \emptyset$, entonces $∃q_c∈ℚ$ con $q_c∈[a_c, b_c)$. Al ser los $[a_c, b_c)$ disjuntos, los $q_c$ son todos distintos, luego debería de haber una aplicación $\appl{ℝ}{ℚ}$ tal que $g(c) = q_c$ inyectiva, pero sería una contradicción porque entonecs $\card{ℚ} ≥ \card{ℝ}$.
\end{problem}


\begin{problem}[18] Sea $A=(-∞, 0] ∪ (2, +∞)$ y $\appl{f}{A}{ℝ}$ dada por \[ f(x) = \begin{cases}
-x^2 & x≤0 \\
x - 2 & x > 2 \end{cases} \]

Demostrar que $f$ es continua si $A$ tiene la topología del orden o la de subespacio, pero que sólo es un homeomorfismo con la del orden.
\solution

En ambos casos es continua ($f(0) = f(2)$). Para ver que es homeomorfismo, tenemos que ir un poco más allá.

Empezamos viendo que la topología del orden en $A$ no es más que la generada por una base \[ \base = \{ (a,b)_A \tq a<b,\; a,b ∈ A \}\]. Da la casualidad que $(5,7)_A = (5,7)_{ℝ}$, pero \[ (-1,3)_A = \{ x∈ A \tq -1 < x < 3\} = (-1, 0]_{ℝ} ∪ (2,3)_{ℝ} \]. Hay que tener cuidado con eso. 

Por otra parte, la topología del subespacio es $\topl_{A}^{sub} = \{ V ∩ A \tq V ∈ \topl_ℝ \}$. Podemos encontrar un $W∈\topl_A^{sub}$ tal que $W∉\topl_{ord_A}$, como por ejemplo \[ W = (-1,1) ∩ A = (-1, 0]_ℝ \]

\end{problem}


\begin{problem}[0]
¿Cuál es la topología producto de $ℝ_{disc} × ℝ_{usual}$?
\solution

¿Cómo construimos la topología producto? Nos basta con coger los elementos de la base de ambos. Como recordatorio:

\[ \topl_{\base_1} \otimes \topl_{\base_2} = \topl_{\base_1 \otimes \base_2} \] donde $\base_1 \otimes \base_2 = \left\{ B_1 × B_2 \tq B_1 ∈ \base_1, B_2 \base_2 \right\}$.

La base $\base_1$ de la topología discreta es $\base_1=\left\{\{x\} \tq x∈ℝ\right\}$, y la base $\base_2$ de la topología usual son los abiertos $(a,b)$. Entonces $B_1 × B_2 = \{x\} × (a,b)$, intervalos abiertos verticales. Casualmente, esta es la topología $\topl_{Lex}$ del orden lexicográfico en $ℝ^2$.
\end{problem}

\begin{problem}[0] Sea $\appl{f}{ℝ}{ℝ}$ continua y biyectiva. ¿Es un homeomorfismo?
\solution

Lo único que le falta para ser homeomorfismo es que la inversa sea continua. Si $\inv{f}$ es continua, entonces la imagen inversa por $\inv{f}$ de un abierto es abierta. Es decir, hay que demostrar que $f$ de un abierto es abierto. Y como tenemos una base, nos basta comprobar sólo para los elementos de la base.

Una función $f$ continua e inyectiva en $ℝ$ debe ser monótona estrictamente creciente o decreciente. Supongamos que es creciente, entonces $f((a,b)) = (f(a), f(b))$.

\end{problem}

\section{Hoja 3}

\begin{problem}[2] En el espacio producto $\topl_{[,)} \otimes \topl_{[,)}$ en $ℝ^2$, describir la topología inducida en los subconjuntos \begin{gather*}
X = \{ (x, -x) \tq x∈ ℝ\} \\ 
Y = \{ (x, x) \tq x∈ ℝ \}
\end{gather*}
\solution

Estudiamos primero la topología producto, que podemos expresar como la generada por una base \[ \base = \{ [a_1, b_1) × [a_2,b_2) \tq a_i < b_i \} \]. Es decir, los elementos de la base son rectángulos en los que los bordes inferior e izquierdo (salvo los vértices) están incluidos.

\begin{figure}[hbtp]
\centering
\inputtikz{E_H3_2}
\caption{Topología del subespacio (en rojo) para $X$ e $Y$.}
\label{figEH32}
\end{figure}

Empezamos, por llevar la contraria, con $Y$. En ese caso, los abiertos serán los intervalos $[\vec{c},\vec{d})$ con $\vec{c} = (c_1, c_2)$ y $\vec{d} = (d_1, d_2)$ tales que $d_1 - c_1 = d_2 - c_2$. Y en el caso de $X$, serán los puntos (la topología discreta).
\end{problem}

Un comentario del profesor para los ejercicios 8, 9 y 10.
\begin{prop} Si $\appl{f}{X}{Y}$ es continua e $Y$ es Hausdorff (\ref{defHausdorff}) entonces \[ G_f = \{ (x, f(x))\tq x∈X \} \] es cerrado en $X×Y$.\label{propHausdorffAppl}
\end{prop}

\begin{proof}

\begin{wrapfigure}{r}{0.3\textwidth}
\centering
\inputtikz{E_H3_PropI}
\caption{Conjunto $G_f ⊆ X×Y$, el grafo de una función.}
\label{figE_H3_PropI}
\end{wrapfigure}

Queremos ver que $G_f^c$ es abierto. Para ello, queremos encontrar un entorno para todo punto $(x,y)∉G_f$ que no corte a $G_f$ (es decir, $y≠f(x)$). Como estamos en la topología producto, necesitamos encontrar simplemente abiertos de la base.

Al ser $Y$ Hausdorff, tenemos que $∃V_1, V_2$ abiertos en $Y$ tales que $f(x) ∈ V_1$, $y∈V_2$ y $V_1 ∩ V_2 = \emptyset$. Definimos ahora \[ U ≝ \inv{f}(V_1)\]. $U$ es abierto en $X$ por ser $f$ continua, y además $x∈U$.

Entonces afirmamos que $(U×V_2)∩G_f = \emptyset$. La razón es que si $(\tilde{x}, f(\tilde{x})) ∈ U × V_2$, entonces $f(\tilde{x})$ pertenece a $V_2$ y también a $f(U)⊆V_1$. Entonces tendría $V_1 ∩ V_2 ≠\emptyset$, lo que sería una contradicción.

Sabemos que $(x,y) ∈ U × V_2 ⊆ G_f^c$, y por tanto $G_f^c$ es abierto en $X×Y$.
\end{proof}

\begin{problem}[4]Se considera la topología $\topl_\base$ en $ℝ^2$ generada por la base $\base$ del ejercicio 9 de la hoja 1: Para cada punto $(x, y )$ de $ℝ^2$ y cada $r ∈ ℝ$ con $r > 0$ se considera el siguiente conjunto $B((x, y), r)$: el cuadrado con lados paralelos a los ejes, centrado en $(x, y)$ y de lado $2r$ , del que se ha excluido los lados y los puntos de las diagonales que no sean el punto $(x, y)$, de tal forma que  \[ \base = \{ B((x, y), r) : (x, y) ∈ ℝ^2 , r > 0\} \]

¿Existen topologías en $ℝ$ de modo que su producto coincida con la topología $\topl_\base$?

Indicación: Prueba que ambas topologías deben ser menos finas que la usual.
\solution

No puede darse el caso de que $\topl_\base$ sea una topología producto de $\topl_1\otimes\topl_2$ en $ℝ$. Si lo fuese, se podría obtener una base de $\topl_1$ como $p_1(\base)$ para la topología usual. Es decir, $\topl_1$ tendría que ser $\topl_{usual}$. Por simetría, entonces $\topl_2 = \topl_{usual}$, y entonces $\topl_1 \otimes \topl_2 = \topl_{usual}^{ℝ^2}$, cosa que es imposible porque está claro que $\topl_\base ≠ \topl_{usual}$.
\end{problem}

\begin{problem}[5] Sean $X$ e $Y$ dos conjuntos no vacíos. Sea $\topl$ la topología producto en $X × Y$ construida a partir de las topologías $\topl_1$ de $X$ y $\topl_2$ de $Y$ . Prueba que si $\base$ es una base de $\topl$ (no necesariamente la “base producto”) entonces $p_1(\base) = \{ p_1(B) \tq B ∈\base \}$ es base de $\topl_1$ y $p_2(\base) = \{ p_2 (B) \tq B ∈\base\}$ es base de $\topl_2$ . ¿Se puede usar este hecho para resolver el ejercicio anterior?

\solution

Sabemos que $\base$ es una base para $\topl$. Entonces \[ \base_1 ≝ \{ p_1(B) \tq B ∈ \base \}\] es base de $\topl_1$. Es obvio que como $p_1$ es abierta, si $B∈\base$ entonces $p_1(B) ∈ \topl_1$.

Además, si $A∈\topl_1$, entonces \[ A = \bigcup_{j∈J} B^j_1 \] con $B^j_1 ∈ \base_1$. Queremos pasar de la topología producto a la proyección, así que consideramos \[ A × Y ∈ \topl_1 \otimes \topl_2\; p_1(A×Y) = A \]. Por otra parte, podemos expresar $A×Y$ como unión de elementos de la base $\base$: \[ A×Y = \bigcup_{j∈J} B^j\] con $B^j∈\base$. Y como la imagen de la unión es la unión de imágenes, tenemos que \[ p_1(A×Y) = \bigcup_{j∈J} p_1(B^j) = \bigcup_{j∈J} B_1^j \] con $B_1^j∈\base_1$.

\end{problem}

\begin{problem}[7]
\ppart Demostrar que un espacio topológico \tops es un espacio de Hausdorff si y sólo si la diagonal $Δ=\{(x,x) \tq x ∈ X\}$ es un cerrado en el espacio topológico $X×X$.
\ppart Demuestra que si \stdf es continua e $Y$ es un espacio de Hausdorff, entonces \[ K = \{ (x_1, x_2) \tq f(x_1) = f(x_2) \}\] es cerrado en $X × X$.
\solution
\spart La implicación a la derecha se puede hacer usando la proposición \ref{propHausdorffAppl}: si $X$ es Hausdorff, entonces la identidad $\appl{I}{X}{X}$ es una función continua que podemos usar para generar el conjunto \[ G_I = \{ (x,I(x)) \tq x∈X \} = Δ \], que según la proposición es cerrado en $X×X$.

\begin{figure}[hbtp]
\centering
\inputtikz{E_H3_E7b}
\caption{$A$ y $B$ son abiertos disjuntos, así que podemos encontrar entornos de los dos puntos y hacer su proyeccción para demostrar que $X$ es Hausdorff.}
\label{figH3_7b}
\end{figure}

Vamos ahora al otro lado: tenemos que $Δ$ es cerrado en $X×X$ y queremos demostrar que $X$ es Hausdorff, esto es, que $∀x_1, x_2$ distintos existen dos entornos disjuntos $V_1, V_2$. Consideramos los puntos $\va = (x_1, x_2),\, \vb = (x_2, x_1)$, que son distintos; y los subconjuntos de $X×X$ $A=\{ (x,y) \tq x > y\}$ y $B=\{(x,y) \tq x < y\}$. Está claro que ambos son abiertos, que $A∩B = ∅$ y que $A∪B = Δ^c$.

Si suponemos sin pérdida de generalidad que $x_1 > x_2$, entonces $\va ∈ A$ y $\va ∈ B$, luego podemos encontrar entornos $E_a ⊆ A$ y $E_b ⊆ B$ de $\va$ y $\vb$ respectivamente, que por fuerza han de ser disjuntos. Entonces podemos construir los entornos $V_1, V_2$ como $V_1 = p_1(E_b),\, V_2=p_1(E_a)$. 

¿Podría darse el caso de que $p_1(E_b) ∩ p_1(E_a) ≠ ∅$?\footnote{Duda planteada por Jorge} No lo sabemos. En todo caso, si eso ocurriese podríamos tomar las proyecciones por $p_2$, que ya por fuerza sí tienen que ser disjuntas (si no lo fuesen, la intersección de $E_a$ o $E_b$ con Δ no sería vacía, contradicción).

En cualquier caso, podemos construir los intervalos $V_1, V_2$ que buscamos, disjuntos, para cualqueir par $x_1, x_2$, luego $X×X$ es Hausdorff.

\spart Si $p ∈ K^c$, entonces $f(p_1) ≠ f(p_2)$. Entonces existen dos entornos $V_1, V_2$ en $Y$ tales que $f(p_i) ∈ V_i$. Entonces $W_j = \inv{f}(V_j)$ es abierto en $X$, y $W_1×W_2 ⊆ K^c$, ya que $∀x_j ∈ W_j \implies f(x_j) ∈ V_j$ y entonces $f(x_1) ≠ f(x_2)$ al ser $V_1, V_2$ disjuntos.
\end{problem}

\begin{problem}[9] Demostrar que si $f$ y $g$ son funciones continuas definidas de $X$ en $Y$ siendo $Y$ un espacio de Hausdorff, entonces el conjunto $C= \{ x \tq f(x) = g(x) \}$ es cerrado en $X$.

\solution

\begin{figure}[hbtp]
\centering
\inputtikz{E_H3_E9}
\caption{Ejercicio 9.}
\label{figH3_E9}
\end{figure}

Comprobamos si $C^c$ es abierto. Cogemos $x_0∈C^c$, entonces $f(x_0) ≠ g(x_0)$, luego existen $V_f, V_g$ aiertos tales que $f(x_0) ∈ V_f,\,g(x_0) ∈ V_g$ y $V_f ∩ V_g=∅$ (se puede ver un dibujo en la figura \ref{figH3_E9}).

Hay que encontrar ahora un entorno $U$ de $x_0$ tal que $f(x) ≠ g(x)\; ∀x∈U$. Ese entorno es \[ U = \inv{f}(V_f) ∩ \inv{g}(V_g) \].
\end{problem}

\begin{problem}[13]
\ppart
\ppart Sea $Y$ el subespacio $(ℝ^+ × ℝ) ∪ (ℝ × \{0\})$ de $ℝ×ℝ$. Sea $h$ la restricción de $p_1$ a $Y$. Demostrar que la aplicación $h$ no es ni abierta ni cerrada. 

Indicación : $\inv{h}(U) ∩ (ℝ × \{0\}) = U × 0$.

\solution
\spart

\spart 
 
\begin{figure}[hbtp]
\centering
\inputtikz{E_H3_E13}
\caption{Esquema de un abierto en $Y$ (sombreado rojo) y su imagen por $h$.}
\label{figH3_13}
\end{figure}

Tenemos que darnos cuenta de que restringir una aplicación abierta no tiene por qué darnos una aplicación abierta de nuevo.

Vamos a ver cómo son los abiertos en $Y$ (figura \ref{figH3_13}). cogemos un abierto $A⊆ℝ^2$ del total y cortamos con $Y$, y nos queda un conjunto cerrado por la izquierda. Entonces, $h(A∩Y)$ es un intervalo cerrado por la izquierda.

\end{problem}

\section{Hoja 4}

\begin{problem}[6]
\solution
Nos dan una idea: si $A$ no es conexo, podría descomponer $A∪D$. Podemos suponer $X=A∪D$. Si $A$ no es conexo, podríamos escribirlo como $A=F_1∪F_2$ con $F_1, F_2$ cerrados en $A$ y, como $A$ es cerrado, cerrados también en $X$; disjuntos y no vacíos. Entonces si $A∩D$ es conexo, se tiene que $A∩D⊆F_1$ o bien $A∩D⊆F_2$. Supomgamos sin pérdida de generalidad el segundo caso. Entonces $A∪D= F_1 ∪ F_1^c$ ($F_1^c = D∪F_2$).
\end{problem}

\begin{problem}[9] Sea \stdf una función continua y sobreyectiva de un espacio topológico $X$ sobre un espacio topológico $Y$ que tiene $n$ componentes conexas. Prueba que $X$ tiene como mínimo $n$ componentes conexas.
\solution
Sabemos que la imagen de una componente conexa de $X$ tiene que estar contenida en una componente conexa de $Y$. Luego si $C_i^X$ son las componentes conexas de $X$ y $C_i^Y$ las de $Y$, tenemos que \[ f(C_i^X)⊆C_{α(i)}^Y \], donde $\appl{α}{\{1,\dotsc,m\}}{\{1,\dotsc,n\}}$ es una aplicación sobreyectiva que lleva el índice de una componente conexa en $X$ al de la componente correspondiente en $Y$.

Sea $y∈C_k^Y$, entonces $x∈\inv{f}(y)$. No sé cómo lo ha escrito pero la idea básica es que como la aplicación es sobreyectiva, todas las componentes conexas de $Y$ deben de ser imagen de al menos una componente conexa de $X$, y entonces es bastante fácil. 
\end{problem}

\begin{problem}[10]
Demuestra que $A=(ℝ ×\{0\}) ∪ (\{0\}× ℝ)$ no es homeomorfo a $ℝ$.
\solution
Si fuesen homeomorfos, entonces la restricción \[g=\appl{f}{A\setminus\{(0,0)\}}{ℝ\setminus\{f((0,0))\}}\] sería también homemomorfismo, luego la imagen de una componente conexa tendría que ser componente conexa. Sin embargo, eso es imposible ya que en el dominio tenemos cuatro componentes conexas y en la imagen sólo dos.

\end{problem}
\begin{problem}[13] Demuestra que todo subconjunto conexo de $ℝ^n$ con más de un punto es no numerable.
\solution

Sea $A$ conexo y con $\card{A}≥2$. Entonces $∃x,y∈A\tq x≠y$, dos puntos distintos. Sea $r=\dst(x,y) > 0$, luego $∀ε$ con $0<ε<r$, entonces $∂\bola(x,ε) ∩ A ≠ ∅$, donde $∂\bola$ es la frontera o borde de la bola. Si fuese vacío, entonces podríamos escribir \[ A = (\bola(x,ε) ∩ A ) ∪ \left(\adh{\bola}^c (x,ε) ∩ A\right) \], unión de dos conjuntos distintos, abiertos de $A$, y entonces $A$ no sería conexo. 

Esto quiere decir que podemos construir una aplicación $\appl{ξ}{(0,r)}{A}$ tal que a todo $ε∈(0,r)$ le asigne $z_ε∈A$ tal que $\dst(x,z_ε) = ε$. Esta aplicación es inyectiva, luego $\card{A} ≥ \card{(0,r)} = \card{ℝ}$, y entonces $A$ no es numerable.
\end{problem}

\begin{problem}[14]
\ppart Probar que un espacio $X$ es conexo si y sólo si no existe ninguna aplicación continua y sobreyectiva $\appl{f}{X}{Y}$ donde $Y = \{0,1\}$ con la topología discreta. 
\ppart Usar el apartado anterior para probar que si $S$ es un subconjunto conexo de un espacio $X$ y $K$ satisface $S ⊂ K ⊂ \adh{S}$ entonces $K$ es conexo.

\solution
\spart
\spart Idea: si $K$ no es conexo, entonces $∃\appl{f}{K}{\{0,1\}}$ como en el apartado anterior. Si $S$ es conexo, tiene que ser que $f(S) = \{p\}$ con $p=0$ ó $p=1$, ya que los únicos conexos en la topología discreta son los puntos. Consideremos ahora $\adh{S}^K = \adh{S}∩K = K$, ¿cuál es su imagen por $f$?
\end{problem}

\begin{problem}[16] En el plano con la topología usual, sea \[ S = \left\{ (r \cos t,r \sin t ) \tq r = 1 − \frac{1}{t} , t ≥ 1 \right\} \]. Probar que $X = S ∪ \mathbb{S}^1$ es conexo pero no es conexo por caminos, donde $\mathbb{S}^1$ es la circunferencia de radio 1.
\solution

Queremos demostrar que es conexo, y para ello queremos demostrar que $\mathbb{S}^1 ⊆ \adh{S}$. Es fácil construir una sucesión para un $t$ fijo: \[ \va_n = \left(1 - \frac{1}{t_n}\right)\left(\cos(t + 2πn), \sin(t + 2πn)\right)\] donde $t+2πn = t_n$. Entonces la sucesión de $\va_n$ tiende a $(\cos t, \sin t)∈\mathbb{S}^1$ cuando $n\to∞$, entonces $\mathbb{S}^1 ⊆\adh{S}$.

La parte de demostrar que no es cpc no la hacemos, pero implicaría ver que no podemos encontrar un camino desde el primer punto de la espiral (por ejemplo) hasta un punto de la circunferencia.
\end{problem}

\begin{problem}[20] Demostrar que si $X$ e $Y$ son conexos y $A, B$ son subconjuntos propios no vacíos de $X$ e $Y$ respectivamente entonces $X × Y \setminus A × B$ es conexo. En la situación anterior, ¿es cierto que si $X$ e $Y$ son conexos por caminos entonces $X × Y \setminus A × B$ también lo es?
\solution
\begin{figure}[hbtp]
\centering
\inputtikz{E_H4_E20}
\caption{Demostración visual de que $X×Y \setminus A×B$ es cpc.}
\label{figH4_E20}
\end{figure}

Queremos poder conectar dos puntos cualesquiera $\va = (x_1, y_1)$ y $\vb = (x_2, y_2)$. Podemos construir dos aplicaciones $\appl{φ_1, φ_2}{[0,1]}{X × Y \setminus A × B}$ tales que $φ_1(0)=(x_1, y_1)$, $φ_1(1) = (x_1, y_2)$ y $φ_2(0) = (x_1, y_2), φ_2(1) = (x_2, y_2)$ continuas, ya que siempre podemos encontrar una aplicación continua entre $y_1$ y $y_2$ y $x_1$ y $x_2$ porque tanto $Y$ como $X$ son conexos por caminos.

Entonces $φ_1\ast φ_2$ es continua y conecta $\va$ con $\vb$. Hay que ser un poquillo riguroso con la ordenación para que no pase la aplicación por $A×B$ pero siempre se puede encontrar.

Demostración más rigurosa. Podemos considerar $y_2∉B, x_1∉A$ y el conjunto $D_1 = \{x_1\} × Y$, que es conexo por ser homeomeorfo a $X_2$. De la misma manera, podemos coger $D_2 = X ×\{y_2\}$, igualmente conexo. Está claro que $D_1 ∪ D_2$ es conexo (la intersección es no vacía). 

Sea $D_0 = D_1 ∪ D_2 ⊆ X × Y \setminus A × B$. Entonces podemos escribir $X × Y \setminus A × B$ como unión de los conjuntos $D_0$ y $D_c = \{c\}×Y$ y $D_r =X × \{ r\}$ como 

\[ X × Y \setminus A × B  = D_0 ∪ \left(\bigcup_{r∉B} D_r\right) ∪ \left(\bigcup_{c∉A} D_c\right) \]

La intersección de cada $D_c$ y $D_r$ con $D_0$ es no vacía, luego estamos en la situación de la proposición \ref{propUnionConexa} y entonces $ X × Y \setminus A × B$ es conexo.
\end{problem}

\begin{problem}[26] Sea $\mathbb{S}^1 = \{(x,y) \tq x^2 + y^2 = 1 \}$ con la topología usual heredada del plano y sea $Y$ el intervalo $[a,b]$ también con la topología usual. Supongamos que $\appl{f}{\mathbb{S}^1}{Y}$ es continua y sobreyectiva. Probar que, para cada $c ∈ (a,b)$ , el conjunto $\inv{f}(c)$ contiene más de un punto.
\solution

\end{problem}


\section{Hoja 5}

\begin{problem}[1]
\ppart Los racionales: $ℚ$.

\solution

\spart $ℚ$ no es compacto. Tenemos 2 formas de demostrarlo:

1) Como $(ℝ,\topl_{usu})$, si $ℚ$ fuera compacto, entonces sería cerrado en $(ℝ,\topl_{usu})$, pero $\bar{ℚ} = ℝ ≠ ℚ \implies $ NO es cerrado.

2) La otra posibilidad es construir un recobrimiento del que no se puede extraer un subrecubrimiento finito.
\end{problem}

\begin{problem}[13]
\ppart  $([0,1]×[0,1],\topl_{lex})$ ¿compacto?
\ppart  $([0,1)×[0,1],\topl_{lex})$ ¿compacto?

\ppart$[0,1]×[0,1]$ como subespacio de $(ℝ^2,\topl_{lex})$ ¿es compacto?
\solution
\spart $([0,1]×[0,1],\topl_{lex})$ es compacto. Se puede hacer de manera simiar a $ℝ$.

Si no hay un SRF (subrecubrimiento finito) en todo el conjunto, si dividimos por la mitad el cuadrado, entonces en alguno de los 2 no puede haber un SRF.  Seguimos dividiendo por la mitad y tomamos la intersección de las mitades de esos conjuntos, que es una línea vertical, homeomorfa a $[0,1]$ que es compacto.

Como la línea vertical es compacta, existe un SRF para la línea, que tiene que cubrir también una banda, que es contradictorio con el razonamiento anterior.


Veremos en teoría una demostración más general.

\spart No, porque no llegamos a coger los puntos (1,1) y (1,0), construyendo $$\bigcup_{k=1}^n [(0,0), \left(1-\frac{1}{n_k},0\right)]$$


\spart
No, el contraejemplo es  una linea vertical.

En $ℝ^2$ la vertical es abierta. Entonces podemos definir el recubrimiento: $A_a = \{(a,y): y∈ℝ\}∈\topl_{lex}$. 

$$[0,1]×[0,1] \subseteq \bigcup_{0≤a≤1} A_a$$ y no exsite un subrecubrimiento finito, ya que los $A_a$ son disjuntos.

\end{problem}