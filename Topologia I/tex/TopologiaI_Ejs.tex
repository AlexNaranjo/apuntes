\begin{problem}[11]
He llegado tarde
\solution
asique no lo he copiado.
\end{problem}

\begin{problem}[6]
\ppart
\ppart
\solution
\spart
Vamos a demostrar que es una topología, para lo cual tenemos que comprobar las 3 propiedades:

\begin{itemize}
\item $Y\in \mathbb{S} \dimplies g^{-1}(x)\in \mathbb{T}$, por ser $g^{-1}(y)=x$

El razonamiento de porqué $\emptyset \in \mathbb{S}$ es igual.

\item $A,B \in \mathbb{S} \dimplies_{def} g^{-1}(A),g^{-1}(B) \in \mathbb{T}$.

$A\cap B \in\mathbb{S} \dimplies g^{-1}(A\cap B) \in \mathbb{T}$.

Hemos llegado a que para demostrar la segunda propiedad, tenemos que demostrar $g^{-1}(A),g^{-1}(B) \in \mathbb{T} \implies g^{-1}(A\cap B) \in \mathbb{T}$.

Para ello: $g^{-1}(a\cap B) = g^{-1}(A)\cap g^{-1}(B)$. No es difícil convencernos de esta igualdad. En caso de tener dudas, demostrar las 2 inclusiones (una en cada sentido). Esto para imágenes directas no funciona.
\end{itemize}
\end{problem}


\begin{problem}[9]
Lo vamos a resolver de palabra y a medias.
\ppart
\ppart
\ppart
\solution
\spart Si añadimos $\emptyset$ y el total, entonces tenemos una topología generada por $B_{\rightarrow}$ y otra generada por $B_{\leftarrow}$.
\spart  
\spart
\end{problem}

\paragraph{Observación útil para el 5  el 12:}  
\begin{enumerate}
\item $x \in C(x,\varepsilon)$
\item $\varepsilon_1 > \varepsilon_2 \implies C(x,\varepsilon_2) \subset C(x,\varepsilon_1)$
\end{enumerate}

Y podemos aplicar la propiedad:
\[
A\in\topl \dimplies \forall a\in A \exists \varepsilon > 0 \tlq C(x,\varepsilon)\subseteq A
\]

Haciendo caso al enunciado y haciendo el dibujo vemos que se cumplen las propiedades de base.

Esta topología contiene a la usual pero al revés no, porque para el punto de intersección de las diagonales no existe un abierto de la usual que le contenga.


\paragraph{Pistas para espacios métricos.}

(16) Si tengo $d$, una distancia no acotada, puedo definir $d'=\frac{d}{1+d}$, que sigue siendo una distancia, parecida y además acotada.
(17) $\sum \frac{1}{2n} \leq 1$. La clave está en aplicar la desigualdad triangular a cada término del sumatorio. La clave para este problema es el 16.