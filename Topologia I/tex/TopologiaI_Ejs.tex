\section{Hoja 1}


\begin{problem}[6] Sea $\appl{g}{X}{Y}$ una aplicación entre dos conjuntos.

\ppart Demostrar que si $\topl$ es una topología en $X$ entonces \[ \mathcal{S} = \{ E ⊆ Y \tq \inv{g}(E) ∈ \topl \} \] es una topología en $Y$.
\ppart Demostrar que si $\mathcal{S}$ es una topología en $Y$ entonces \[ \mathcal{U} = \{ \inv{g}(E) \tq E ∈ \mathcal{S} \} \]es una topología en $X$.

\solution
\spart Vamos a demostrar que es una topología, para lo cual tenemos que comprobar las 3 propiedades (ver \ref{defTopología}):

\begin{enumerate}
\item $Y\in \mathcal{S} \dimplies g^{-1}(x)\in \mathcal{T}$, por ser $g^{-1}(y)=x$

El razonamiento de porqué $\emptyset \in \mathcal{S}$ es igual.

\item $A,B \in \mathcal{S} \dimplies g^{-1}(A),g^{-1}(B) \in \mathcal{T}$.

$A\cap B \in\mathcal{S} \dimplies g^{-1}(A\cap B) \in \mathcal{T}$.

Hemos llegado a que para demostrar la segunda propiedad, tenemos que demostrar $g^{-1}(A),g^{-1}(B) \in \mathcal{T} \implies g^{-1}(A\cap B) \in \mathcal{T}$.

Para ello: $g^{-1}(a\cap B) = g^{-1}(A)\cap g^{-1}(B)$. No es difícil convencernos de esta igualdad. En caso de tener dudas, demostrar las 2 inclusiones (una en cada sentido). Esto para imágenes directas no funciona.

\item Demostramos ahora que la unión de abiertos está en la topología. Si $A, B ∈ \mathcal{S}$, entonces $\inv{g}(A), \inv{g}(B) ∈ \topl$. Como $\topl$ es topología, tenemos que $\inv{g}(A) ∪ \inv{g}(B) ∈ \topl$, lo que implica de forma obvia que $\inv{g}(A∪B) ∈ \topl$ y por lo tanto $A∪B ∈ \mathcal{S}$.
\end{enumerate}

\spart

\end{problem}

\begin{problem}[9] Se consideran las siguientes familias de conjuntos en $ℝ$:

\begin{gather*}
\base_{\leftarrow} = \{ (-∞, b) \tq b ∈ ℝ \} \\
\base_{\rightarrow} = \{ (a,∞) \tq  ∈ ℝ \} 
\end{gather*}

\ppart Demostrar que cada familia es una base de una topología sobre $ℝ$.
\ppart Comparar esas topologías.
\ppart Demostrar que la topología generada por $\base_{\leftarrow} ∪ \base_{\rightarrow}$ es la usual.
\solution
\spart Si añadimos $\emptyset$ y el total, entonces tenemos una topología generada por $\base_{\rightarrow}$ y otra generada por $\base_{\leftarrow}$.

\spart  

\spart
\end{problem}

\begin{problem}[11]
Sea $\topl_j$, $j∈J$ una familia de topologías sobre $X$. Demostrar que existe una topología que contiene a todas las $\topl_j$, para $j∈J$ y además es la menos fina de todas las que verifican esta propiedad. 
\solution

Aplicamos directamente la proposición \ref{propTopologiaMinima}: la topología que contiene a todas ellas es \[ \topl = \bigcap_{j∈J} \topl_j \]
\end{problem}

\paragraph{Observación útil para el 5 y el 12:}  
\begin{enumerate}
\item $x \in C(x,\varepsilon)$
\item $\varepsilon_1 > \varepsilon_2 \implies C(x,\varepsilon_2) \subset C(x,\varepsilon_1)$
\end{enumerate}

Y podemos aplicar la propiedad:
\[
A\in\topl \dimplies \forall a\in A \exists \varepsilon > 0 \tlq C(x,\varepsilon)\subseteq A
\]

Haciendo caso al enunciado y haciendo el dibujo vemos que se cumplen las propiedades de base.

Esta topología contiene a la usual pero al revés no, porque para el punto de intersección de las diagonales no existe un abierto de la usual que le contenga.


\paragraph{Pistas para espacios métricos}

(16) Si tengo $d$, una distancia no acotada, puedo definir $d'=\frac{d}{1+d}$, que sigue siendo una distancia, parecida y además acotada.

(17) $\sum \frac{1}{2n} \leq 1$. La clave está en aplicar la desigualdad triangular a cada término del sumatorio. La clave para este problema es el 16.