\begin{tikzpicture}[scale=4]
\begin{scope}
\draw[opacity=0] (0,-0.2) rectangle (0, 1.2); % Chapuza para que tikz no se coma cosas por los bordes de arriba y abajo.
\fill[orange!20!white] (0,0) -- (1, 0) -- (1, 0.5) -- (0.5, 1) -- (0, 1) -- cycle;
\draw[gray] (0,0) rectangle (1, 1);

\draw[opacity=0.5, thin, pattern color=blue, pattern=horizontal lines] (0,0) -- (1, 0) -- (1, 0.5) -- (0.5, 1) -- (0, 1) -- cycle;

\draw[very thick, blue] (0,0) -- (1, 0) -- (1, 0.5) -- (0.5, 1) -- (0, 1) -- cycle;
\draw[very thick, blue, -)] (0.5, 1) -- (1, 1);
\end{scope}

\begin{scope}[xshift=70]
\fill[orange!20!white] (0,0) -- (1, 0) -- (1, 0.5) -- (0.5, 1) -- (0, 1) -- cycle;
\draw[red, opacity=0.5, thin, pattern color=red, pattern=vertical lines] (0,0) -- (1, 0) -- (1, 0.5) -- (0.5, 1) -- (0, 1) -- cycle;
\draw[gray] (0,0) rectangle (1, 1);

\draw[red, very thick] (0.5, 0) -- (0,0) -- (0, 1) -- (0.5, 1);
\end{scope}

\begin{scope}[xshift=35]
\fill[orange!20!white] (0,0) -- (1, 0) -- (1, 0.5) -- (0.5, 1) -- (0, 1) -- cycle;
\draw[gray] (0,0) rectangle (1, 1);

\draw[[-), green, very thick] (0.5, 1) -- (1,1);
\draw[[-, green, very thick] (1,0.5) -- (0.5, 1);
\draw[[-), green, very thick] (1,0) -- (0.5, 0);
\end{scope}
\end{tikzpicture}