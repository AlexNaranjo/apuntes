% -*- root: ../EstadisticaII.tex -*-
\section{Hoja 1}

\begin{problem}[1]
Sea $Y = (Y_1,Y_2,Y_3)' ≡ N_3(µ,Σ)$, donde \[µ = (0,0,0)'\;
Σ =\begin{pmatrix}
1&0&0\\
0&2&−1\\
0&−1&2
\end{pmatrix}
\]


\ppart  Calcula la distribución del vector $(X_1,X_2)$, donde $X_1 = Y_1 + Y_3$ y $X_2 = Y_2 + Y_3$.
\ppart ¿Existe alguna combinación lineal de las variables aleatorias $Y_i$ que sea independiente de $X_1$?

\solution

\spart 
\[
\begin{pmatrix}X_1 \\ X_2 \end{pmatrix} = \begin{pmatrix} Y_1 + Y_3 \\ Y_2 + Y_3 \end{pmatrix} = \begin{pmatrix} 1&0&1\\0&1&1 \end{pmatrix} \begin{pmatrix} Y_1\\Y_2\\Y_3 \end{pmatrix} \equiv N_1\left( \begin{pmatrix}0\\0 \end{pmatrix},\begin{pmatrix}3&1\\1&2\end{pmatrix} \right)
\]

\spart 
\[
\begin{pmatrix} Ay\\By \end{pmatrix} = \begin{pmatrix} A\\B \end{pmatrix} Y \equiv N_{q+r} \left( \begin{pmatrix} Aμ\\Bμ \end{pmatrix},\begin{pmatrix} A\\B \end{pmatrix} Σ(A',B') \right)
\]

Entonces \[cov\left(a'y,(1,0,1)y\right) = (a_1,a_2,a_3) \begin{pmatrix} 1&0&0\\0&2&-1\\0&-1&2\end{pmatrix} \begin{pmatrix} 1&0&1 \end{pmatrix}\]

\end{problem}


\begin{problem}[2]

\solution

\end{problem}

\begin{problem}[3]


\solution

\end{problem}

\begin{problem}[5]

Calcula la distribución condicionada de $X$ dado $Y$ = $y$, y la de $Y$ dado $X$ = $x$.

\solution


\[
\begin{pmatrix}X\\Y \end{pmatrix} \equiv N_2\left(\begin{pmatrix}0\\0\end{pmatrix},\begin{pmatrix}1&-1\\-1&2\end{pmatrix}^{-1}\right)
\]

Aplicando las fórmulas vistas en teoría \ref{form::EspVarCondicionada}

\[
E(X|Y=y) = μ_y + Σ_{21}Σ_{11}^{-1}(X-μ_x) = 0 + \frac{1}{1}(y-0) = y
\]
\[
E(Y|X=x) = μ_x + Σ_{21}Σ_{11}^{-1}(Y-μ_y) = 0 + \frac{1}{2}(x-0) = \frac{x}{2}
\]

\end{problem}

\begin{problem}[7]
Sea $X = (X1,X2,X3)'$ un vector aleatorio con distribución normal tridimensional con vector de medias $(0,0,0)'$ y matriz de covarianzas
\[
Σ =
\begin{pmatrix}
1&2&−1\\
2&6&0\\
−1&0&4
\end{pmatrix}
\]


Definamos las v.a. $Y_1 = X_1 + X_3, Y_2 = 2X_1 − X_2 e Y_3 = 2X_3 − X_2$. Calcula la distribución de $Y_3$ dado que $Y_1=0$ e $Y_2=1$.

\solution

Lo primero es descubrir la matriz de la combinación lineal, esto es:
\[
\begin{pmatrix} Y_1\\Y_2\\Y_3\end{pmatrix} = \begin{pmatrix}1&0&1\\2&-1&0\\0&-1&2\end{pmatrix}\begin{pmatrix}X_1\\X_2\\X_3\end{pmatrix} \equiv N_3 \left( \begin{pmatrix}0\\0\\0\end{pmatrix}, \begin{pmatrix}3&-2&4\\-2&2&-2\\4&-2&22 \end{pmatrix} \right)
 \]

Llamamos \[A=\begin{pmatrix}3&-2&4\\-2&2&-2\\4&-2&22 \end{pmatrix}\]

¿De dónde sale esta matriz? Elena opina (y Jorge lo confirma) que $A = ΣBΣ'$, donde $B$ es la matriz de la combinación lineal, es decir: 
\[B=\begin{pmatrix}1&0&1\\2&-1&0\\0&-1&2\end{pmatrix}\]

\[
E(Y_3|y_1 = 0, y_2 = 1) = 0 + (4,-2) \begin{pmatrix} 3&-2\\-2&2\end{pmatrix}^{-1} \begin{pmatrix}0-0\\1-0\end{pmatrix} = ... = 1
\]

\[
V(Y_3|y_1=0,y_2=1) = 22 - (4,-2) \begin{pmatrix} 3&-2\\-2&2 \end{pmatrix}^{-1} \begin{pmatrix}4\\-2\end{pmatrix} = ... = 16
\]


Entonces, la distribución de $Y = (Y_1,Y_2,Y_3)' = N_3(1,16)$

\end{problem}
