% -*- root: ../EstadisticaII.tex -*-
\section{Hoja 1}

\begin{problem}[1]
Sea $Y = (Y_1,Y_2,Y_3)' ≡ N_3(µ,Σ)$, donde \[µ = (0,0,0)'\;
Σ =\begin{pmatrix}
1&0&0\\
0&2&−1\\
0&−1&2
\end{pmatrix}
\]


\ppart  Calcula la distribución del vector $(X_1,X_2)$, donde $X_1 = Y_1 + Y_3$ y $X_2 = Y_2 + Y_3$.
\ppart ¿Existe alguna combinación lineal de las variables aleatorias $Y_i$ que sea independiente de $X_1$?

\solution

\spart 
\[
\begin{pmatrix}X_1 \\ X_2 \end{pmatrix} = \begin{pmatrix} Y_1 + Y_3 \\ Y_2 + Y_3 \end{pmatrix} = \begin{pmatrix} 1&0&1\\0&1&1 \end{pmatrix} \begin{pmatrix} Y_1\\Y_2\\Y_3 \end{pmatrix} \equiv N_1\left( \begin{pmatrix}0\\0 \end{pmatrix},\begin{pmatrix}3&1\\1&2\end{pmatrix} \right)
\]

\spart 
\[
\begin{pmatrix} Ay\\By \end{pmatrix} = \begin{pmatrix} A\\B \end{pmatrix} Y \equiv N_{q+r} \left( \begin{pmatrix} Aμ\\Bμ \end{pmatrix},\begin{pmatrix} A\\B \end{pmatrix} Σ(A',B') \right)
\]

Entonces \[cov\left(a'y,(1,0,1)y\right) = (a_1,a_2,a_3) \begin{pmatrix} 1&0&0\\0&2&-1\\0&-1&2\end{pmatrix} \begin{pmatrix} 1&0&1 \end{pmatrix}\]

\end{problem}


\begin{problem}[2]

\solution

\end{problem}

\begin{problem}[3]


\solution

\end{problem}

\begin{problem}[5]

Calcula la distribución condicionada de $X$ dado $Y$ = $y$, y la de $Y$ dado $X$ = $x$.

\solution


\[
\begin{pmatrix}X\\Y \end{pmatrix} \equiv N_2\left(\begin{pmatrix}0\\0\end{pmatrix},\begin{pmatrix}1&-1\\-1&2\end{pmatrix}^{-1}\right)
\]

Aplicando las fórmulas vistas en teoría \ref{form::EspVarCondicionada}

\[
E(X|Y=y) = μ_y + Σ_{21}Σ_{11}^{-1}(X-μ_x) = 0 + \frac{1}{1}(y-0) = y
\]
\[
E(Y|X=x) = μ_x + Σ_{21}Σ_{11}^{-1}(Y-μ_y) = 0 + \frac{1}{2}(x-0) = \frac{x}{2}
\]

\end{problem}

\begin{problem}[7]
Sea $X = (X1,X2,X3)'$ un vector aleatorio con distribución normal tridimensional con vector de medias $(0,0,0)'$ y matriz de covarianzas
\[
Σ =
\begin{pmatrix}
1&2&−1\\
2&6&0\\
−1&0&4
\end{pmatrix}
\]


Definamos las v.a. $Y_1 = X_1 + X_3, Y_2 = 2X_1 − X_2 e Y_3 = 2X_3 − X_2$. Calcula la distribución de $Y_3$ dado que $Y_1=0$ e $Y_2=1$.

\solution

Lo primero es descubrir la matriz de la combinación lineal, esto es:
\[
\begin{pmatrix} Y_1\\Y_2\\Y_3\end{pmatrix} = \begin{pmatrix}1&0&1\\2&-1&0\\0&-1&2\end{pmatrix}\begin{pmatrix}X_1\\X_2\\X_3\end{pmatrix} \equiv N_3 \left( \begin{pmatrix}0\\0\\0\end{pmatrix}, \begin{pmatrix}3&-2&4\\-2&2&-2\\4&-2&22 \end{pmatrix} \right)
 \]

Llamamos \[A=\begin{pmatrix}3&-2&4\\-2&2&-2\\4&-2&22 \end{pmatrix}\]

¿De dónde sale esta matriz? Elena opina (y Jorge lo confirma) que $A = ΣBΣ'$, donde $B$ es la matriz de la combinación lineal, es decir: 
\[B=\begin{pmatrix}1&0&1\\2&-1&0\\0&-1&2\end{pmatrix}\]

\[
E(Y_3|y_1 = 0, y_2 = 1) = 0 + (4,-2) \begin{pmatrix} 3&-2\\-2&2\end{pmatrix}^{-1} \begin{pmatrix}0-0\\1-0\end{pmatrix} = ... = 1
\]

\[
V(Y_3|y_1=0,y_2=1) = 22 - (4,-2) \begin{pmatrix} 3&-2\\-2&2 \end{pmatrix}^{-1} \begin{pmatrix}4\\-2\end{pmatrix} = ... = 16
\]


Entonces, la distribución de $Y = (Y_1,Y_2,Y_3)' = N_3(1,16)$

\end{problem}

\section{Hoja 2}


\begin{problem}[1] Calcula la distribución exacta bajo la hipótesis nula del estadístico de Kolmogorov-Smirnov para muestras de tamaño 1.

\solution

La hipótesis sería $H_0 : F = F_0$ continua, con $X \sim F$

En este caso,

\[D=||F_1 - F_0||_{\inf} = (1) = \max\{F_0(x), 1 - F_0(x)\}\]

$(1)$ hay 2 posibles caminos. Al dibujar lo que nos dicen (una muestra de tamaño 1) podemos sacarlo por intuición. Sino, aplicamos la fórmula de los estadísticos.

Ahora calculamoms:

\[ P_{F_0}(D\leq x) = P_{F_0} = \left\{\max \{ ... \}\leq x\right\} = P_{F_0} = P_{F_0}\{ 1-x \leq F_0(x) \leq x \}\]

Resolvemos la desigualdad, aplicando que $F_0$ es una uniforme.

\[
P\{1-x \leq U \leq x\} = \left\{ \begin{array}{cc} 0 & x\leq \frac{1}{2} \\ 2x-1 & x\geq \frac{1}{2}\end{array} \right. \implies D \sim \mathcal{U}\left(\frac{1}{2},1\right)
\]

Ya que $1-x > x \dimplies x\le \frac{1}{2}$

\end{problem}
\begin{problem}[2] Se desea contrastar la hipótesis nula de que una única observación X procede de una distribución N(0,1). Si se utiliza para ello el contraste de Kolmogorov-Smirnov, determina para qué valores de X se rechaza la hipótesis nula a nivel α = 0,05.
\solution

Este ejercicio está muy relacionado con el primero. Es una aplicación al caso de la normal.


Mirando en la tabla, encontramos que para $α = 0.05$, entonces $d_α = 0.975$. Con esta inormación podemos construir la región crítica:
\[ R = \left\{\max\{\Phi(x), 1 - \Phi(x))\} > 0.975\right\} = \{\Phi(x) > 0.975\} \cup \{1 - \Phi(x) > 0.975\} =\]
\[ \{ X>\Phi^{-1}(0.975)\} \cup \{X < \Phi^{-1}(0.025)\}\]

Consultando las tablas, vemos que $\Phi^{-1}(0.975) = 1.96$ y por simetría, $\Phi^{-1}(0.025) = -1.96$

\[R = \{|X| > 1.96\}\]


\obs Es interesante saber que, al ser simétrica la normal, la interpretación gráfica es muy fácil. Si dividimos la normal en 3 intervalos, $(-∞ , -1.96) , (-1.96,1.96) , (1.96, ∞)$, el área encerrada en las colas es el nivel de significación, en este caso: \[\text{Area }\left((-∞ , -1.96)\cup (1.96, ∞)\right) = 0.05\]

\end{problem}
\begin{problem}[3] Da una demostración directa para el caso k = 2 de que la distribución del estadístico del contrast $\chi^2$ de bondad de ajuste converge a una distribución $\chi_1^2$ , es decir,
\[
T = \frac{(O1 − E1)^2}{E1} +
\frac{(O2 − E2)^2}{E2} \convs[d] \chi_1^2\]

\label{ej::2.3}

[Indicación: Hay que demostrar que $T = X^2_n$ , donde $X_n\convs[d] N(0,1)$. Para reducir los dos sumandos a uno, utilizar la relación existente entre O1, E1 y O2, E2.]
\solution

Si tenemos $n$ datos, vamos a construir la tabla de contingencia. Creo que consideramos una binomial porque, al sólo tener 2 clases, o eres de una o eres de la otra con una probabilidad $p$.

\begin{center}
\begin{tabular}{c|cc}
 & $A_1$ & $A_2$ \\\hline
 Obs & $n\gor{p}$ & $n(1-\gor{p})$\\
 Esp  & $np_0$ & $n(1-p_0)$\\
\end{tabular}
\end{center}

\[ T = \sum_{i=1}^2 \frac{(O_i - E_i)^2}{E_i} = \frac{n^2(\gor{p}-p_0)^2}{n} + \frac{n^2(\gor{p}-p_0)}{n(1-p_0)}  = ... \]
Simplificando, llegamos a:

\[
T = \left(\frac{|\gor{p}-p_0|}{\sqrt{\frac{p_0(1-p_0)}{n}}} \right)
\]

Está contando un montón de cosas interesantes que me estoy perdiendo.



Entre ellas, tenemos que $\sqrt{T} \convs[d]N(0.1)$ por el teorema central del límite ( es el caso particular para una binomial), con lo que $T\convs[d] \chi^2$. ¿Porqué 1 grado de libertad? Porque sólo estamos estimando 1 parámetro, el $\gor{p}$.

Esto responde también al problema 11. 

\end{problem}
\begin{problem}[4] El número de asesinatos cometidos en Nueva Jersey cada día de la semana durante el año 2003 se muestra en la tabla siguiente:

\begin{center}
\begin{tabular}{c|ccccccc}
Día & Lunes & Martes & Miércoles & Jueves & Viernes & Sábado & Domingo \\\hline
Frecuencia & 42 & 51 & 45 & 36 & 37 & 65 & 53
\end{tabular}
\end{center}

\ppart Contrasta a nivel α = 0,05, mediante un test $χ2$, la hipótesis nula de que la probabilidad de que se cometa un asesinato es la misma todos los días de la semana.

\ppart ¿Podría utilizarse el test de Kolmogorov-Smirnov para contrastar la misma hipótesis? Si tu
respuesta es afirmativa, explica cómo. Si es negativa, explica la razón.


\ppart Contrasta la hipótesis nula de que la probabilidad de que se cometa un asesinato es la misma desde el lunes hasta el viernes, y también es la misma los dos días del fin de semana (pero no es necesariamente igual en fin de semana que de lunes a viernes).

\solution

\spart $n = 329$, $E_i = \frac{329}{7}$ y $H_0 : p_i = \frac{1}{7}$

Calculamos el estadístico $T = ... = \sum_{i=1}^7 ... = ... = 13.32$

Por otro lado, $\chi^2_{6;0.05} = 12.59$, con lo que rechazamos la hipótesis.

\spart No podría utilizarse al tratarse de algo discreto y KS sólo sirve para continuas.

\spart

Tenemos la siguiente tabla:

\begin{center}
\begin{tabular}{c|ccccccc}
Día & Lunes & Martes & Miércoles & Jueves & Viernes & Sábado & Domingo \\\hline
Frecuencia & p & p & p & p & p & q & q
\end{tabular}
\end{center}

Con $5p + 2q = 1 \implies q = \frac{11-5p}{2}$

Entonces, tenemos \[ e.m.v.(p) =L(p)= p^{42+51+...+37} \left( \frac{11-5p}{2} \right)^{65+53} \]

Ahora, despejamos tomando $l(p) = ln(L(p)) = 211 ln(p) + 118ln\left(\frac{11-5p}{2}\right)$ y maximizamos:

\[
l'(p) = 0 \implies ... \left\{\begin{array}{c} \gor{p} = 0.128\\ \gor{q} = 0.179 \end{array}\right.
\]


Ahora construimos el estadístico:

\[
T = \sum_{i=1}^7 \frac{O_i^2}{\gor{E}_i^2} - n = ... = 5.4628
\]

Y comparamos con la $\chi^2$. ¿Cuántos grados de libertad? Si tenemos $7$ clases, siempre perdemos uno, con lo que serían 6. Sin embargo hemos estimado un parámetro, con lo que son $5$ grados de libertad. Entonces: $ c = \chi^2_{5;0.05} = 11.07$

Como $T < c$, no podemos rechazar la hipótesis.

\obs
Podríamos plantearnos contrastar que es uniforme de lunes a viernes ($H_1$) y otra uniforme distinta en fines de semana ($H_2$). Entonces tendríamos $H_0 : H_1 \cap H_2$, y construir la región $R = R_1 \cup R_2$. ¿Cuál es el problema de este camino?

El nivel de significación, ya que $P_{H_0}(R_1 \cup R_2) = P_{H_0}(R_1) + P_{H_0}(R_2) - P_{H_0}(R_1\cap R_2) = 2α - α^2 \sim 2α$. 

Podríamos tomar, chapucerillamente $α = \frac{α}{2}$ para que al final, $P_{H_0} ( R_1 \cup R_2) = α$. Aquí surge otro problema, que es que estamos despreciando la probabilidad de la intersección y tomándolo como independiente cuando no tiene porqué serlo. Es una aproximación ``buena'' que a veces se utiliza, pero pudiendo hacerlo bien... 

\end{problem}



\begin{problem}[5] Para estudiar el número de ejemplares de cierta especie en peligro de extinción que viven en un
bosque, se divide el mapa del bosque en nueve zonas y se cuenta el número de ejemplares de cada
zona. Se observa que 60 ejemplares viven en el bosque repartidos en las 9 zonas de la siguiente
forma:


\begin{center}
\begin{tabular}{|c|c|c|}
\hline
8&7&3 \\\hline
5&9&11 \\\hline
6&4&7 \\\hline
\end{tabular}
\end{center}

Mediante un contraste de hipótesis, analiza si estos datos aportan evidencia empírica de que los
animales tienen tendencia a ocupar unas zonas del bosque más que otras.

Tomamos $α = 0.01$
\solution

$T = 7.47$, $\chi^2_{8;0.001} = 20.09$

Aceptamos la hipótesis $H_0 : $ la especie se reparte uniformemente.

\end{problem}
\begin{problem}[6] Se ha desarrollado un modelo teórico para las diferentes clases de una variedad de moscas. Este
modelo nos dice que la mosca puede ser de tipo L con probabilidad p
2
, de tipo M con probabilidad
q
2 y de tipo N con probabilidad 2pq (p + q = 1). Para confirmar el modelo experimentalmente
tomamos una muestra de 100 moscas, obteniendo 10, 50 y 40, respectivamente.
(a) Hallar la estimación de máxima verosimilitud de p con los datos obtenidos.
(b) ¿Se ajustan los datos al modelo teórico, al nivel de significación 0’05?
\solution

\end{problem}
\begin{problem}[7] (a) Aplica el test de Kolmogorov-Smirnov, al nivel 0.05, para contrastar si la muestra (3.5, 4, 5,
\solution

\end{problem}
\begin{problem}[5]2, 6) procede de la U(3,8).
(b) Aplica el test de Kolmogorov-Smirnov, al nivel 0.05, para contrastar la hipótesis de que la
muestra (0, 1.2, 3.6) procede de la distribución N(µ = 1;σ = 5).
\solution

\end{problem}
\begin{problem}[8] Se ha clasificado una muestra aleatoria de 500 hogares de acuerdo con su situación en la ciudad
(Sur o Norte) y su nivel de renta (en miles de euros) con los siguientes resultados:
Renta Sur Norte
0 a 10 42 53
10 a 20 55 90
20 a 30 47 88
más de 30 36 89
(a) A partir de los datos anteriores, contrasta a nivel α = 0,05 la hipótesis nula de que en el sur los
hogares se distribuyen uniformemente en los cuatro intervalos de renta considerados.
(b) A partir de los datos anteriores, ¿podemos afirmar a nivel α = 0,05 que la renta de los hogares
es independiente de su situación en la ciudad?
\solution

\end{problem}
\begin{problem}[9] A finales del siglo XIX el físico norteamericano Newbold descubrió que la proporción de datos
que empiezan por una cifra d, p(d), en listas de datos correspondientes a muchos fenómenos
naturales y demográficos es aproximadamente:
p(d) = log10
d + 1
d
!
, d = 1,2,...,9.
Por ejemplo, p(1) = log10 2 ≈ 0,301030 es la frecuencia relativa de datos que empiezan por 1. A raíz
de un artículo publicado en 1938 por Benford, la fórmula anterior se conoce como ley de Benford.
El fichero poblacion.RData incluye un fichero llamado poblaciones con la población total de los
municipios españoles, así como su población de hombres y de mujeres.
(a) Contrasta a nivel α = 0,05 la hipótesis nula de que la población total se ajusta a la ley de Benford.
(b) Repite el ejercicio pero considerando sólo los municipios de más de 1000 habitantes.
(c) Considera las poblaciones totales (de los municipios con 10 o más habitantes) y contrasta a nivel
α = 0,05 la hipótesis nula de que el primer dígito es independiente del segundo.
(Indicación: Puedes utilizar, si te sirven de ayuda, las funciones del fichero benford.R).
\solution

\end{problem}
\begin{problem}[10] Se ha llevado a cabo una encuesta a 100 hombres y 100 mujeres sobre su intención de voto. De
las 100 mujeres, 34 quieren votar al partido A y 66 al partido B. De los 100 hombres, 50 quieren
votar al partido A y 50 al partido B.
(a) Utiliza un contraste basado en la distribución χ
2 para determinar si con estos datos se puede
afirmar a nivel α = 0,05 que el sexo es independiente de la intención de voto.
(b) Determina el intervalo de valores de α para los que la hipótesis de independencia se puede
rechazar con el contraste del apartado anterior.
\solution

\end{problem}
\begin{problem}[11] Sea X1,...,Xn una muestra de una distribución Bin(1, p). Se desea contrastar H0 : p = p0.
Para ello hay dos posibilidades: (a) Un contraste de proporciones basado en la región crítica
R = {|pˆ − p0| > zα/2
p
p0(1 − p0)/n}, y (b) un contraste χ
2 de bondad de ajuste con k = 2 clases. ¿Cuál
es la relación entre ambos contrastes?
\solution

Consultar el ejercicio \ref{ej::2.3}.

\end{problem}
\begin{problem}[12] En un estudio de simulación se han generado 10000 muestras aleatorias de tamaño 10 de una
distribución N(0,1). Para cada una de ellas se ha calculado con R el estadístico de KolmogorovSmirnov
para contrastar la hipótesis nula de que los datos proceden de una distribución normal
estándar, y el correspondiente p-valor.
(a) Determina un valor x tal que la proporción de estadísticos de Kolmogorov-Smirnov mayores
que x, entre los 10000 obtenidos, sea aproximadamente igual a 0.05. ¿Cuál es el valor teórico al que
se debe aproximar la proporción de p-valores menores que 0.1 entre los 10000 p-valores obtenidos?
(b) ¿Cómo cambian los resultados del apartado anterior si en lugar de considerar la distribución
normal estándar se considera una distribución uniforme en el intervalo (0,1)?
\solution

\end{problem}