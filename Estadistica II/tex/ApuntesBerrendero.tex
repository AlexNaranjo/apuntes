% ----------------------------------------------------------------
% Presentaciones con Beamer ---------------------------------------
% ----------------------------------------------------------------

\documentclass{beamer}


%\usepackage{beamerthemeplain}
\usepackage{graphicx}
%\usepackage{beamerthemesplit}
\beamertemplateshadingbackground{red!0}{blue!0}
\usepackage[utf8]{inputenc}

%----------------------------------------------------------------------
% Para que aparezcan varias transparencias en la misma p\'{a}gina
\usepackage{pgfpages}
%\pgfpagesuselayout{4 on 1}[a4paper,border shrink=5mm,landscape]
%----------------------------------------------------------------------
% Suprime los s\'{\i}mbolos de navegaci\'{o}n
\setbeamertemplate{navigation symbols}{}
%---------------------------------------------------------------------

%----------------------------------------------------------------
\newcommand{\ep}{\epsilon}
\newcommand{\real}{{\rm I\kern-.17em R}}
\newcommand{\pro}{\mbox{P}}
%-----------------------------------------------------------------

\title[Tema 1]{Estad\'{\i}stica II\\
Tema 4: Clasificación y regresión logística}
\author[J.R. Berrendero]
{Jos\'{e} R. Berrendero}
\date{}
\institute{Departamento de Matem\'{a}ticas\\
 Universidad Aut\'{o}noma de Madrid}

%------------------------------------------------------------

\begin{document}

\frame{\titlepage}
%\section[Outline]{}
%\frame{\tableofcontents}

%------------------------------------------------------------------
% Aqu\'{\i} empieza la presentaci\'{o}n
%------------------------------------------------------------------

\begin{frame}
\frametitle{Estructura de este tema}

\begin{itemize}
\item Planteamiento del problema de clasificación supervisada

\item Regla de Mahalanobis

\item Regla lineal de Fisher

\item Regresión logística

\item Optimalidad: la regla de Bayes

\end{itemize}


\end{frame}
%---------------------------------------------------------------------
\begin{frame}[plain]
\frametitle{El problema de clasificación supervisada}

Disponemos de una muestra de $k$ variables medidas en $n$ unidades u objetos que pertenecen a dos grupos o poblaciones \textit{(training data)}.

\

Cada observación $i=1,\ldots,n$ consiste en un vector $(x'_i,y_i)'$, donde $x_i\in\mathbb{R}^k$ son las $k$ variables e    $y\in\{0,1\}$ indica el grupo al que pertenece la unidad en la que se han obtenido. 

\

\textbf{Objetivo:} Asignar una nueva unidad con valores $x$ (e $y$ desconocida) a uno de los dos grupos (\textbf{obtener una regla de clasificación}).


\

 Este problema tiene diferentes nombres en la literatura en inglés: ``supervised classification", ``statistical learning", ``discrimination", ``machine learning", ``pattern recognition", etc.







\end{frame}
%-----------------------------------------------------------------
\begin{frame}[plain,fragile]
\frametitle{Ejemplo: infartos de miocardio}

En un estudio de factores de riesgo en enfermedades coronarias, se dispone de datos de 462 personas (de las que 160 habían sufrido infartos y 302 eran controles). Para cada una de ellas se midieron las siguientes variables:

\

\begin{center}
{\scriptsize

\begin{tabular}{|c|l|}
\hline Nombre variable & Descripci\'{o}n  \\
\hline {\tt sbp} & Tensión sanguínea sistólica \\
{\tt tobacco} & Consumo de tabaco\\
{\tt ldl} & Colesterol\\
{\tt adiposity} & Medida de adiposidad\\
{\tt typea} & Comportamiento ``tipo A"\\
{\tt obesity} & Medida de la obesidad\\
{\tt alcohol} & Consumo de alcohol\\
{\tt age} & Edad \\
 \hline
 \end{tabular}

 }
\end{center}
 
 \

Representación gráfica de los datos: verde (controles)  y rojo (casos).


\end{frame}

%------------------------------------------------------------------
\begin{frame}[plain,fragile]
\frametitle{Los datos}

\centerline{\includegraphics[height=8cm,width=11cm]{pairs-SAheart}}

\end{frame}
%------------------------------------------------------------------
\begin{frame}[plain,fragile]
\frametitle{Las variables \textit{edad} y \textit{obesidad}}

\centerline{\includegraphics[height=8cm]{edadobesidad-SAheart}}

\end{frame}
%---------------------------------------------
\begin{frame}[plain]
\frametitle{Regla de Mahalanobis}

Para $i=0,1$ denotamos $P_i$  a la distribución condicionada  $X|Y=i$. Suponemos que $P_i$ es una distribución con vector de medias $\mu_i$ y matriz de covarianzas $\Sigma_i$.




\

Asignar al grupo cuyo centro es más cercano en el sentido de la distancia de Mahalanobis.

\

\textbf{Regla de Mahalanobis:} Clasificar $x$ en el grupo 1 (i.e. $Y=1$) si y solo si
\[
(x-\mu_0)'\Sigma_0^{-1}(x-\mu_0) >  (x-\mu_1)'\Sigma_1^{-1}(x-\mu_1).
\]

\

En la práctica se usan los vectores de medias y las matrices de covarianzas muestrales. 


\end{frame}
%---------------------------------------------------
\begin{frame}
 \frametitle{Regla de Mahalanobis}

\centerline{$\mu_0=(1,0)^\prime$, $\mu_1=
(5,0)^\prime$, $\Sigma_1=\left[
  \begin{array}{cc}
    1 & 0 \\
    0 & 1 \
  \end{array} \right]$,
$\Sigma_2=\left[
  \begin{array}{cc}
    5 & 0 \\
    0 & 5 \
  \end{array} \right]$.}

\centerline{{\includegraphics[width=9cm]{mahalanobis-8}}}




\end{frame}
%---------------------------------------------
\begin{frame}[plain]
\frametitle{Regla lineal de Fisher}

$P_0$, $P_1$ poblaciones homocedásticas ($\Sigma_0=\Sigma_1$).
\medskip

\textbf{El enfoque de Fisher:} Proyectar los datos en la direcci\'{o}n $a$ m\'{a}s conveniente  y
utilizar las proyecciones $a'x_i$ para discriminar.


\centerline{{\includegraphics[width=9 cm]{fisher}}}



\end{frame}
%---------------------------------------------
\begin{frame}[plain]
\frametitle{Regla lineal de Fisher}

\vspace{-1.7 cm}

\centerline{{\includegraphics[width=7 cm,angle=-50]{proyeccion}}}

\end{frame}
%--------------------------------------------------------------------------------------
\begin{frame}
\frametitle{Regla lineal de Fisher}

¿Cuál es la direcci\'{o}n \'{o}ptima?

\begin{itemize}
  \item Una buena direcci\'{o}n debe separar bien los centros de los grupos. La distancia entre las medias $(a'\mu_0-a'\mu_1)^2 = a'Ba$, donde $B=(\mu_0-\mu_1)(\mu_0-\mu_1)'$, debe ser grande.
\end{itemize}
\centerline{{\includegraphics[width=7 cm]{Fisher-normales-1}}}
\vspace{-2mm}

\begin{itemize}
  \item La varianza de las proyecciones dentro de los grupos ($a'\Sigma a$) debe ser lo menor
  posible.
\end{itemize}
\centerline{{\includegraphics[width=8 cm]{Fisher-normales-2}}}


\end{frame}

%---------------------------------------------
\begin{frame}[plain]
\frametitle{Regla lineal de Fisher}

\textbf{Problema:} Encontrar la dirección $a$ que maximiza
\[
f(a) = \frac{a'Ba}{a'\Sigma a}\qquad (\textbf{cociente de Rayleigh}).
\]

\

Para cualquier $\lambda\ne 0$, $f(\lambda a) =f(a)$, por lo que
es necesario imponer alguna condici\'{o}n de normalizaci\'{o}n. 

\

En R se impone $a'\Sigma a=1$.

\

La soluci\'{o}n es proporcional al vector discriminante $w=\Sigma^{-1}(\mu_1-\mu_0)$.



\end{frame}
%---------------------------------------------
\begin{frame}[plain]
\frametitle{Regla lineal de Fisher}

\centerline{{\includegraphics[width=11cm]{mahalanobis-11}}}

\textbf{Regla de Fisher:} Clasificar $x$ en el grupo 1 (i.e. $Y=1$) si y solo si
\[
w'\left(x-\frac{\mu_0 + \mu_1}{2}\right)>0,
\]
donde $w=\Sigma^{-1}(\mu_1-\mu_0).$

\end{frame}
%---------------------------------------------
\begin{frame}[plain]
\frametitle{Regla lineal de Fisher}

\textbf{Ejercicio} 

Si en la regla de Mahalanobis se supone $\Sigma_1=\Sigma_2$, se obtiene la regla de Fisher.


\

\

\textbf{Estimación de los parámetros}

En la práctica, para aplicar la regla de Fisher se usan los vectores de medias muestrales y la matriz de covarianzas estimada combinada siguiente:
\[
\hat{\Sigma} = \frac{n_0-1}{n_0+n_1-2} S_0  + \frac{n_1-1}{n_0+n_1-2} S_1,
\]
donde $S_i$ es la matriz de covarianzas muestral del grupo $i$, $i=0,1$.

\




\

 

\end{frame}
%------------------------------------------------------------------
\begin{frame}[plain,fragile]
\frametitle{Datos simulados}

\centerline{\includegraphics[height=8cm]{sim-plot}}

\end{frame}

%------------------------------------------------------------------
\begin{frame}[plain,fragile]
\frametitle{Datos simulados: reglas de Mahalanobis y de Fisher}

\centerline{\includegraphics[height=8cm]{sim-plot-reglas}}

\end{frame}
%------------------------------------------------------------------
\begin{frame}[plain,fragile]
\frametitle{Infartos de miocardio (edad y obesidad)}

\centerline{\includegraphics[height=8cm]{edadobesidad-reglas}}

\end{frame}
%----------------------------------------------------
\begin{frame}
\frametitle{Estimación del error de clasificación}

Es importante estimar el error que comete la regla de clasificación.

La \textbf{tasa de error aparente} (TEA) es:
$$\text{TEA}:=\frac{\text{Total de mal clasificados en la muestra}}{n}100\%.$$
\vspace{-4 mm}

\centerline{{\includegraphics[width=8.5 cm]{errores-1}}}



\end{frame}
%------------------------------------------------------------------
\begin{frame}
\frametitle{Estimación del error de clasificación}

La TEA tiende a infraestimar el verdadero error ya que los datos se utilizan tanto para calcular la regla de clasificación como para evaluarla.
\medskip


Existen diversos procedimientos para resolver este problema:

\begin{itemize}
  \item[$\blacktriangleright$] Dividir la muestra en dos partes: \textbf{training data} y \textbf{test data}. Utilizar la primera parte para construir la regla de clasificación y estimar el error mediante la segunda.
   \medskip

  \item[$\blacktriangleright$] \textbf{Validaci\'{o}n cruzada:} Omitimos un dato de los $n$ observados y generamos la regla de clasificación con los $n-1$ restantes. Clasificamos la observación apartada y repetimos el procedimiento para cada una de las observaciones.
\end{itemize}\vspace{-5 mm}

$$\text{TEVC}:=\frac{\text{Total de mal clasificados en la muestra por VC}}{n}100\%.$$


\end{frame}


%------------------------------------------------------------------
\begin{frame}[plain,fragile]
\frametitle{Regla de Fisher con R}

{\scriptsize
\begin{verbatim}
library(MASS)
resultado.lda <- lda(datos.edob, clase, prior=c(0.5,0.5))

resultado.lda$scaling	# Coeficientes para cada variable
plot(resultado.lda)   # Representa las puntuaciones discriminantes
clasificacion <- predict(resultado.lda)$class   # Predicciones para la muestra
sum(clasificacion!= clase)/length(clase)   # Proporcion de errores
0.3528139
\end{verbatim}
}

\centerline{\includegraphics[height=4cm]{edadobesidad-puntuaciones}}

\end{frame}
%---------------------------------------------------------------
\begin{frame}[plain]
\frametitle{El modelo de regresión logística}


Disponemos de $n$ observaciones. Cada observación $(x_{i1},\ldots,x_{ik},y_i)'$ está formada por un vector de variables regresoras $x_i =(x_{i1},\ldots,x_{ik})'$ y el valor de la 
variable respuesta $y_i$.


\


Las variables $Y_1,\ldots,Y_n$ son independientes y tienen distribución de Bernoulli. 

\

Denotamos
$p_i=\mathbb{P}(Y_i=1 \mid x_i)$.
La probabilidad de ``éxito'' depende de las variables regresoras.



\

Una relación lineal $p_i=\beta_0+\beta_1x_{i1}+\cdots+\beta_kx_{ik}$ no es adecuada\\ (¿por qué?)


\end{frame}
%---------------------------------------------------------------
\begin{frame}[plain]
\frametitle{El modelo de regresión logística}



Suponemos que la relación entre $p_i$ y $x_i$ viene dada por 
\[
p_i = \frac{1}{1+e^{-\beta_0-\beta_1x_{i1}-\cdots-\beta_kx_{ik}}},
\]
es decir,
\[
p_i = F(\beta_0+\beta_1x_{i1}+\cdots+\beta_kx_{ik}),
\]
donde $F(x)=1/(1+e^{-x})$ es la \textbf{función logística}.


\begin{center}
\includegraphics[height=4cm,width=6cm]{funcion-logistica}
\end{center}


\end{frame}
%---------------------------------------------------------------
\begin{frame}[plain]
\frametitle{Interpretación de los parámetros del modelo}

Llamamos $O_i$ a la \textbf{razón de probabilidades} para la observación $i$:
\[
 O_i=\frac{p_i}{1-p_i} 
\]
¿Cómo se interpreta el valor de $O_i$? ¿Qué significa, por ejemplo, $O_i=2$?




Si se cumple el modelo de regresión logística, entonces
\[
O_i =  e^{\beta_0  + \beta_1x_{i1} + \cdots + \beta_kx_{ik}}
\]


¿Cómo varía la razón de probabilidades si la variable regresora $x_{ij}$ se incrementa una unidad?
\[
\frac{O'_i}{O_i} = \frac{e^{\beta_0+\cdots+\beta_j(x+1)+\cdots+\beta_kx_{ik}}}{e^{\beta_0+\cdots+\beta_jx+\cdots+\beta_kx_{ik}}}= e^{\beta_j}.
\]  
Por tanto $e^{\beta_j}$ es la variación de la razón de probabilidades cuando la variable regresora $j$ se incrementa en una unidad y el resto de variables permanece constante. 


\end{frame}
%----------------------------------------------------------------
\begin{frame}[plain]
\frametitle{Estimación}

Para estimar los parámetros se usa el método de máxima verosimilitud.

\

Por ejemplo, si observamos los datos

\begin{center}
\begin{tabular}{c|ccc}
$x_i$ & 2 & 1 & 3 \\  \hline
$Y_i$ & 0 & 1 & 1
\end{tabular}
\end{center}

entonces $\hat{\beta}_0$ y $\hat{\beta}_1$ son los valores que maximizan la función de verosimilitud
\[
L(\beta_0,\beta_1) = P(Y=0 \mid x = 2)P(Y=1 \mid x = 1)P(Y=1 \mid x = 3)
\]
\[
L(\beta_0,\beta_1) = \left(1-\frac{1}{1+e^{-\beta_0-2\beta_1}}\right)
\left(\frac{1}{1+e^{-\beta_0-\beta_1}}\right)
\left(\frac{1}{1+e^{-\beta_0-3\beta_1}}\right)
\]

\

Se pueden aplicar métodos numéricos estándar de optimización.

\end{frame}
%-------------------------------------------------------------------
\begin{frame}[plain,fragile]
\frametitle{Regresión logística: datos simulados}

\begin{verbatim}
# Numero de observaciones
n = 100

# Parametros
beta0 = 0
beta1 = 3

# Genera los datos
x = rnorm(n)
p = 1/(1+exp(-beta0-beta1*x))
y = rbinom(n,1,p)

# Ajusta el modelo
reg = glm(y~x,family=binomial)
summary(reg)
\end{verbatim}

\end{frame}
%------------------------------------------------------------------
\frame<all:1>[plain]{\frametitle{Probabilidades de \'{e}xito estimadas}

\centerline{\includegraphics[height=8cm,width=8cm]{sim-probest}}

}
%------------------------------------------------------------------
\begin{frame}[plain,fragile]
\frametitle{Salida {\tt R}}

{\scriptsize
\begin{verbatim}
Call:
glm(formula = y ~ x, family = binomial)

Deviance Residuals:
    Min       1Q   Median       3Q      Max
-1.8340  -0.7598  -0.1568   0.7623   2.4224

Coefficients:
            Estimate Std. Error z value Pr(>|z|)
(Intercept)  -0.1157     0.2585  -0.447    0.655
x             2.7083     0.5854   4.627 3.71e-06 ***
---
Signif. codes:  0 '***' 0.001 '**' 0.01 '*' 0.05 '.' 0.1 ' ' 1

(Dispersion parameter for binomial family taken to be 1)

    Null deviance: 137.628  on 99  degrees of freedom
Residual deviance:  90.396  on 98  degrees of freedom
AIC: 94.396

Number of Fisher Scoring iterations: 5
\end{verbatim}
}

\end{frame}
%------------------------------------------------------------------
\begin{frame}[plain,fragile]
\frametitle{Infartos de miocardio: regresión logística simple}

Como variable regresora, vamos a usar s\'{o}lo la edad.

\

Ajustamos el modelo:
\[
\mbox{P}(Y=1) =
\frac{1}{1+\exp(-\beta_0-\beta_1\mbox{{\tt edad}})}
\]

\

\begin{verbatim}
> reg1 = glm(Y ~ edad, family='binomial')
> summary(reg1)
\end{verbatim}

\end{frame}
%------------------------------------------------------------------
\begin{frame}[plain,fragile]
\frametitle{Salida {\tt R}}

{\scriptsize
\begin{verbatim}
Call:
glm(formula = Y ~ edad, family = "binomial")

Deviance Residuals: 
    Min       1Q   Median       3Q      Max  
-1.4321  -0.9215  -0.5392   1.0952   2.2433  

Coefficients:
             Estimate Std. Error z value Pr(>|z|)    
(Intercept) -3.521710   0.416031  -8.465  < 2e-16 ***
edad         0.064108   0.008532   7.513 5.76e-14 ***
---
Signif. codes:  0 ‘***’ 0.001 ‘**’ 0.01 ‘*’ 0.05 ‘.’ 0.1 ‘ ’ 1 

(Dispersion parameter for binomial family taken to be 1)

    Null deviance: 596.11  on 461  degrees of freedom
Residual deviance: 525.56  on 460  degrees of freedom
AIC: 529.56

Number of Fisher Scoring iterations: 4
\end{verbatim}
}

\end{frame}
%------------------------------------------------------------------
\begin{frame}[plain,fragile]
\frametitle{Probabilidades estimadas}

Se representan los datos junto con la curva de probabilidades
estimadas:
\[
\widehat{\mbox{P}(Y=1)} =
\frac{1}{1+\exp(-\hat{\beta}_0-\hat{\beta}_1\mbox{{\tt edad}})}
\]

\centerline{\includegraphics[height=5cm,width=7cm]{edad-probest}}

\end{frame}
%------------------------------------------------------------------
\begin{frame}[plain,fragile]
\frametitle{Dos variables regresoras}

Ajustamos el modelo con dos variables regresoras, la edad y la obesidad:

\[
\mbox{P}(Y=1) =
\frac{1}{1+\exp(-\beta_0-\beta_1\mbox{{\tt edad}}-\beta_2\mbox{{\tt obesidad}})}
\]

\begin{verbatim}
> reg = glm(Y ~ edad+obesidad, family='binomial')
> summary(reg)
\end{verbatim}

\end{frame}
%------------------------------------------------------------------
\begin{frame}[plain,fragile]
\frametitle{Salida {\tt R}}

{\scriptsize
\begin{verbatim}
Call:
glm(formula = Y ~ edad + obesidad, family = "binomial")

Deviance Residuals: 
    Min       1Q   Median       3Q      Max  
-1.4401  -0.9227  -0.5384   1.0905   2.2497  

Coefficients:
             Estimate Std. Error z value Pr(>|z|)    
(Intercept) -3.581465   0.742611  -4.823 1.42e-06 ***
edad         0.063958   0.008674   7.374 1.66e-13 ***
obesidad     0.002523   0.025934   0.097    0.923    
---
Signif. codes:  0 ‘***’ 0.001 ‘**’ 0.01 ‘*’ 0.05 ‘.’ 0.1 ‘ ’ 1 

(Dispersion parameter for binomial family taken to be 1)

    Null deviance: 596.11  on 461  degrees of freedom
Residual deviance: 525.55  on 459  degrees of freedom
AIC: 531.55

Number of Fisher Scoring iterations: 4
\end{verbatim}
}

\end{frame}

%-------------------------------------------------------------------
\begin{frame}[plain,fragile]
\frametitle{Comparación de modelos mediante ANOVA}

\begin{verbatim}
> anova(reg1, reg, test='Chisq')
Analysis of Deviance Table

Model 1: Y ~ edad
Model 2: Y ~ edad + obesidad
  Resid. Df Resid. Dev Df  Deviance P(>|Chi|)
1       460     525.56                       
2       459     525.55  1 0.0094552    0.9225
\end{verbatim}

\end{frame}
%------------------------------------------------------------------
\begin{frame}[plain,fragile]
\frametitle{Regla de clasificación logística}

\textbf{Regla de clasificación logística: }
Clasificar $x$ en el grupo 1 (i.e. $Y=1$) si y solo si
\[
\widehat{\mathbb{P}(Y=1|x)}>\widehat{\mathbb{P}(Y=0|x)}
\]

\

Sustituyendo por la función logística tenemos una regla lineal (diferente en general a la de Fisher): Clasificar $x$ en el grupo 1 (i.e. $Y=1$) si y solo si
\[
\hat\beta_0 +\hat\beta_k x_1 + \ldots + \hat\beta_k x_k > 0.
\]

\

En el \textbf{ejemplo}, clasificamos a un individuo como enfermo si y solo si
\[
-3.58 + 0.064 \cdot \tt{edad} + 0.0025 \cdot \tt{obesidad} > 0.
\]



\end{frame}
%------------------------------------------------------------------
\begin{frame}[plain,fragile]
\frametitle{Regla de clasificación logística}

\centerline{\includegraphics[height=8cm]{edadobesidad-reglas-log}}

\end{frame}
%------------------------------------------------------------------
\begin{frame}[plain,fragile]
\frametitle{Infartos de miocardio: regresión logística múltiple}

{\scriptsize
\begin{verbatim}
Call:
glm(formula = Y ~ sbp + tobacco + ldl + adiposity + typea + obesity + 
    alcohol + age, family = "binomial")
 

Coefficients:
             Estimate Std. Error z value Pr(>|z|)    
(Intercept) -6.066864   1.271443  -4.772 1.83e-06 ***
sbp          0.005641   0.005611   1.005 0.314721    
tobacco      0.072716   0.026326   2.762 0.005742 ** 
ldl          0.192492   0.059429   3.239 0.001199 ** 
adiposity    0.017066   0.028433   0.600 0.548355    
typea        0.040467   0.012078   3.350 0.000807 ***
obesity     -0.057931   0.042980  -1.348 0.177703    
alcohol      0.001446   0.004403   0.328 0.742627    
age          0.050650   0.011766   4.305 1.67e-05 ***
---

(Dispersion parameter for binomial family taken to be 1)

    Null deviance: 596.11  on 461  degrees of freedom
Residual deviance: 488.89  on 453  degrees of freedom
AIC: 506.89
\end{verbatim}
}

\end{frame}
%----------------------------------------------------
\begin{frame}
\frametitle{Regla Bayes}



\textbf{Regla Bayes:}
$x$ se clasifica en $P_1$ si y solo si 
\[
\mathbb{P}(Y=1|x)>\mathbb{P}(Y=0|x) 
\]

\



En el caso en que 
\begin{enumerate}
\item $P_0$ tiene densidad $f_0$ y  $P_1$ tiene densidad $f_1$,
\item las \textbf{probabilidades a priori} de las poblaciones son
$$\mathbb{P}(P_0)=\pi_0,\quad \mathbb{P}(P_1)=\pi_1\quad (\pi_0+\pi_1=1).$$
\end{enumerate}
se tiene
\[
\mathbb{P}(Y=1|x)>\mathbb{P}(Y=0|x) \Leftrightarrow \pi_0 f_0(x)>\pi_1 f_1(x).
\]

\

La regla Bayes es óptima (su error de clasificación es el mínimo posible).


\end{frame}
%----------------------------------------------------
\begin{frame}
\frametitle{Regla Bayes bajo normalidad}

Supongamos que $f_0$ y $f_1$ son normales: para $x\in\mathbb{R}^p$,
\[
f_i(x) = |\Sigma_i|^{-1/2}(2\pi)^{-p/2}\exp\left\{-\frac{1}{2}(x-\mu_i)'\Sigma_i^{-1}(x-\mu_i)    \right\},\quad i=0,1.
\]

\

\textbf{Regla Bayes bajo normalidad}\\
$x$ en $P_0$ si $\displaystyle  d_{M_0}^2(x,\mu_0)<d^2_{M_1}(x,\mu_1)+2\log\left(\frac{\pi_0|\Sigma_1|^{1/2}}{\pi_1|\Sigma_0|^{1/2}}\right)$
donde $\displaystyle d_{M_i}^2(x,\mu_i)=(x-\mu_i)'\Sigma_i^{-1}(x-\mu_i)$ es el cuadrado de la distancia de Mahalanobis entre $x$ y $\mu_i$ ($i=0,1$).
\medskip


\

\textbf{Regla bajo normalidad y homocedasticidad ($\Sigma_0=\Sigma_1$)}\\
$x$ se clasifica en $P_0$ si
$\displaystyle  w^\prime  x < w^\prime  \left(\frac{\mu_0+\mu_1}{2}\right)+  \log\left(\frac{\pi_0}{\pi_1}\right)$





\end{frame}

%-----------------------------------------------------------------
\end{document}
% ----------------------------------------------------------------

%---------------------------------------------
\begin{frame}[plain]
\frametitle{}



\end{frame}

 