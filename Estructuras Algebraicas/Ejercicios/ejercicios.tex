\documentclass{apuntes}

\begin{document}
\section{Hoja 1}
\begin{problem}[1]
 Decide de manera razonada si los siguientes conjuntos son grupos con la operación definida.
 
 \begin{itemize}
  \item a) $(\real,+)$
  \item b) Fijado $n \in \mathbb{Z}_{n>0}$, el conjunto de los enteros módulo $n$ con la suma.
  \item c) $(C^*,\cdot)$
  \item d) $(U (n), ·)$, donde $U (n)$ denota los restos módulo $n$ de enteros coprimos con $n$.
  \item e) Dado un conjunto no vacío $X$, el conjunto $G$ de las biyecciones de $X$ con la composición, $(G, \circ)$.Calcula el cardinal de G si X es un conjunto finito.
 \end{itemize}

 \solution

\paragraph{a)}
$$\left. \begin{matrix}
\text{¿Es asociativo? Sí}\\
\text{¿Tiene elmento neutro? Sí: } \\a + 0 = a, \forall a \in \real\\
\text{¿Existe el inverso de todo elemento? Sí: } \\a^{-1} = -a, \forall a \in \real
\end{matrix}\right\} \implies \text{Sí es un grupo.}$$

\paragraph{b)}
$$\left. \begin{matrix}
\text{¿Es asociativo? Sí}\\
\text{¿Tiene elmento neutro? Sí: } \\\gor{a} + \gor{n} = \gor{a}, \forall \gor{a} \in \mathbb{Z}_n\\
\text{¿Existe el inverso de todo elemento? Sí: } \\ \gor{a} + (\gor{n} - \gor{a}) = \gor{n} = e \implies a^{-1} = \gor{n} - \gor{a}, \forall a \in \real
\end{matrix}\right\} \implies \text{Sí es un grupo.}$$

\end{problem}

\section{Hoja 2}
\subsection{Problema 4}
Sea G un grupo. Demostrar que $Z(G) = \{X \tq X\in G (\forall Y \in G) XY = YX\}$

\begin{itemize}
\item $1 \in Z(G)$
\item {$X_1,X_2 \in G X_1Y = YX_1 \rightarrow X^{-1}XYX^{-1} = X^{-1}YXX^{-1} \implies X^{-1}Y=YX^{-1}$, es decir, el inverso también conmuta, por lo que los inversos de $X_1,X_2 \in Z(G)$}
\item {$X_1 X_2 Y = X_1 Y X_2 = YX_1 X_2, X_1 \cdot X_2 \in Z(G)$ el producto de 2 elementos del grupo está en el centro, por lo que es cerrado por la operación.}
\end{itemize}
\paragraph{a)}
$$Z(D_3), D_3 = \left\{\begin{matrix} 1,a,a^2\\b,ab,a^2b\end{matrix}\right\},\left\{\begin{matrix}a^3 = 1 = b^2\\ ba^j = a^{-j}b\end{matrix}\right\}$$

\paragraph{b)}
$$Z(D_4) =\left\{\begin{matrix} 1,a,a^2,a^3\\b,ab,a^2b,a^3b\\a^4=1=b^2\\ba=a^{-1}b\end{matrix}\right\}$$

$$Z(D_4) = \left\{x \tq x\in D_4 axa^{-1} = x = bxb^{-1}\right\}$$

Sin sentido...
$a^ib \notin Z(D_4). (a^2 \neq 1)$

$a^i \in Z(D_4) \dimplies (a^{2i} = 1)$

$Z(D_4) = \{1,a^2\} = <a^2>$
\subsection{5}

\paragraph{b)}
$C_{D_4}(b)$. Basta con comprobar la conmutación con $a^j$ y con $a^jb$ siendo $j = 0,1,2,3$, ya que con eso podemos ver la conmutación con todos los elementos. Se puede demostrar la conmutatividad multiplicando a derecha e iezquierda por $b$ y $b^{-1}$ y si nos queda $=1$, es conmutativo.

$$\left\{\begin{matrix}b(a^j)b^{-1} = a^{-j}, a^j \in C_{D_4}(b) \dimplies a^2j = 1\\
b(a^jb)b^{-1} = a^{-j} = a^{-j}b, a^jb \in C_{D_4}(b) \dimplies a^2j = 1\end{matrix}\right.$$

\subsection{9}
\paragraph{a)}
\paragraph{b)}

$$f(x,y) = \int_a^xy g(s)ds$$
Aplicando el teorema fundamental del cálculo $\left(f \text{ continua } \implies\displaystyle\int_a^b f(x)dx = F(b)-F(a)\right)$
$$\dpa{f}{x} = g(xy)\underbrace{\dpa{xy}{x}}_{=y} - \underbrace{g(a)\dpa{a}{x}}_{=0}  = g(xy)y$$
$$\dpa{f}{y} = \cdots  = g(xy)x$$

\subsection{Ejercicio de examen:}
$\appl{g}{\real}{\real}$ continua, con $g(1) = 4$.

Sea $f(x,y,z)=\displaystyle \int_0^{x^2ye^z} g(t)dt$.

Demostrar que $f$ es diferenciable y calcular $\nabla f(1,1,0)$.

\section{Hoja 3}

\subsection{Problema 1:}
\paragraph{a)}
Todo subgrupo de $\mathbb{Z}$ es cíclico.\\
$H\leq \mathbb{Z} \implies (\exists r \in N) \tq H = r\mathbb{Z} = \{ r_j\tq j\in \mathbb{Z}\} $\\
$H = \{ \hdots -3r, -2r, -r, 0, r, 2r, 3r, \hdots \}$\\
$\mathbb{N} = \{ 0,1,2,\hdots \}$\\
$\mathbb{N^*} = \{ 1,2,\hdots \}$\\
$r = min (H\cap \mathbb{N}^*)$
Si $H = \{ 0 \}$ ya está ($r = 0$)
Supongo $H \neq \{ 0 \}$ y entonces $H\cap \mathbb{N} \neq \emptyset$\\
$x\in H\implies -x \in H$\\
$r\mathbb{Z} \subset H$\\
Si $x\in H$ puedo tomar $j\in \mathbb{Z}$ y $k\in \mathbb{Z}$ con $x=jr+k, 0\leq k < r$ \footnote{Si a alguien le apetece completarlo que lo haga, no me ha dado tiempo...}

\paragraph{c)}
$r\mathbb{Z} \subset s\mathbb{Z} \iff s|r$\\
$r, s$ enteros no negativos.\\
$r\mathbb{Z} \subset \mathbb{Z} \implies r\in s\mathbb{Z} \implies \exists j \in \mathbb{Z} \tq r=sj \implies s|r$\\
$s|r, x\in r\mathbb{Z},\\ r=js,\ x=rk,\ x=jsk=jks\ \in s\mathbb{Z}$

\paragraph{b)}\footnote{Y por qué el b está después que el c, pues no lo se porque no le entiendo hablar.}\\
$r\mathbb{Z} = s\mathbb{Z} \implies r|s, s|r, \implies r=s$

\paragraph{d)}
$H\leq \mathbb{Z}, 6\mathbb{Z} \subset H = r\mathbb{Z},\\
r|6,\ r=1,2,3,6$\\
$H = \mathbb{Z}, 2\mathbb{Z}, 3\mathbb{Z}, 6\mathbb{Z}$

\subsection{Problema 3:}
\paragraph{a)}

$\appl{f}{\real}{\real} \exists f'(x) \neq 0 \implies f$ inyectiva.

Solución: aplicando el teorema del valor medio.

\paragraph{b)}
$\appl{f}{\real^2}{\real^2}$

$f(x,y) =( e^xcos(y) + 2e^xsen(y),-e^xcos(y))$

$$J = \begin{pmatrix}
       e^xcos(y)+2e^xsen(y) & e^xcos(y)+2e^xsen(y) \\
       -e^xcos(y) & e^xsen(y)
      \end{pmatrix}
$$

Calculamos $$det(J) = (e^xcos(y)+2e^xsen(y))+e^xsen(y) + e^xcos(y)sen(y)(-e^xsen(y)+2e^xcos(y)) = $$
$$ = ... = 2e^x > 0 \forall x \in \real$$

Aunque el jacobiano sea siempre positivo, $f$ no es inyectiva porque si tomamos $f(0,0) = (1,-1) = f(0,2\pi)$.

\subsection{Problema 5}
\paragraph{a)}
$H\leq S_3,\ K\leq S_3 \tq HK \nleq S_3$\\
$<(23)><(13)> = \{ 1,(23)\} \{ 1, (13)\} = \{1, (13), (23), (123) \}$ No es subgrupo de $S_3$

\paragraph{b)}
$H\leq G,\ K\lhd G \implies HK = KH \leq G$
$hk = hkh^{-1}h = k'h$\\
$kh=hh^{-1}kh=hk''$


\subsection{inventado:}
\label{inventado}
Sea $F(x,y) = (x^2-y^2,2xy)$. Encontrar los puntos en los que la siguiente aplicación es localmente inversible de clase $C^1$.

\begin{itemize}
 \item 1) $F \in C^1$ por ser $F_1,F_2$ polinomios.
 \item 2)$det(J)>0 \forall (x,y)\in \real^2$. 
 
 En este caso: $$det\begin{pmatrix}
                  2x&-2y\\
                  2x&2y
                 \end{pmatrix} = 4x^2 + 4y^2 = 0 \dimplies (x,y) = (0,0)$$           
 \item 3) Por el teorema de la funcion inversa, existe una inversa local de $F,C^1$ en todo entorno de $(x,y) \in \real^2$ con $(x,y)\neq (0,0)$. 
 
 Está la posibilidad de que exista la función inversa, pero no podemos deducir nada del teorema. Para verlo, recurrimos a la definición de inyectividad, y en este caso, no es inyectiva porque es una función par.
 \end{itemize}
 \subsection{5}
 \paragraph{a)}
 $f\in C^1(\real), f'\neq0$.
 No tiene sentido...
 $$\left\{\begin{matrix} u(x,y) =f(x)\\v(x,y) = -y + f(x)\end{matrix}\right.$$
 Probar que tiene inversa global.
 
 Mismos pasos que en el ejercicio anterior:
 \begin{itemize}
 \item 1) $F \in C^1$ por ser $F_1,F_2$ , porque $f\in C^1$.
 \item 2)$det(J)>0 \forall (x,y)\in \real^2$.
 
 $$det(J) = det\begin{pmatrix} f'(x)&0\\f(x)+xf'(x)&-1\end{pmatrix} = -f'(x) \neq 0\text{ por hipótesis}$$
 
 Como nos piden calcular las derivadas parciales de la función inversa. (La inversa de la matriz jacobiana, es la jacobiana de la matriz inversa)
 $$J(0,0) = \begin{pmatrix}f'(0) & 0 \\f(0) & -1\end{pmatrix}$$
 Lo que buscamos en la matriz inversa, que en este caso es ella misma.
 
 El teorema solo nos demuestra la existencia de la inversa local (contraejemplo:\ref{inventado}. Hay que ver la inyectividad para hablar de inversa global.
 
\begin{gather*}
F(x,y) = (u(x,y),v(x,y))\\
\text{Condición: }F(x_1,y_1) = F(x_2,y_2) \implies x_1=x_2, y_1=y_2\\
u(x_1,y_1) = u(x_2,y_2) \implies f(x_1) = f(x_2)\\
f' \text{ no se anula } \implies \text{f es inyectiva} \implies x_1=x_2\\
v(x_1,y_1) = v(x_2,y_2) -y_1 + x_1f(x_1) = -y_2 + x_2f(x_2) \underbrace{\implies}_{x_1=x_2}\\
y_1=y_2
\end{gather*}
Hemos demostrado que $F$ es inyectiva y por lo tanto admite inversa global.
 
\end{itemize}


\subsection{Problema 6:}

$$F(x,y,z) = \left\{\begin{matrix}u = 2x+2x^2y+2x^2z+2xy^2+2xyz\\v=x+y+2xy+2x^2\\w=4x+y+z+3y^2+3z^2+6yz\end{matrix}\right.$$
\begin{itemize}
 \item $u,v,w \in C^1$ por se surma de polinomios. 
 \item \begin{gather*}
\dpa{u}{x} = ... \implies \dpa{u}{x}(0,0) = 2\\
\dpa{u}{y} = ... \implies \dpa{u}{y}(0,0) = 0\\
\dpa{u}{z} = ... \implies \dpa{u}{z}(0,0) = 0\\
\dpa{v}{x} = ... \implies \dpa{v}{x}(0,0) = 1\\
\dpa{v}{y} = ... \implies \dpa{v}{y}(0,0) = 0\\
\dpa{v}{z} = ... \implies \dpa{v}{z}(0,0) = 0\\
\dpa{w}{x} = ... \implies \dpa{w}{x}(0,0) = 4\\
\dpa{w}{y} = ... \implies \dpa{x}{y}(0,0) = 1\\
\dpa{w}{z} = ... \implies \dpa{w}{z}(0,0) = 1
       \end{gather*}
       
   $det(J) =\begin{pmatrix}
             2&0&0\\
             1&1&0\\
             4&1&1
            \end{pmatrix}
 = 2 \neq 0 \implies \exists $ inversa local de clase $C^1$ en un entonrno del origen.
\end{itemize}

\subsection{8:}
\paragraph{a)}

$$det(J) = det\begin{pmatrix}
       cos(\varphi)&-rsen(\varphi)&0\\
       sen(\varphi)&rcos(\varphi)&0\\
       0&0&1
      \end{pmatrix} = rcos^2(\varphi) + rsen^2(\varphi) = r$$
      
      Por tanto, por el teorema de la función inversa, existe una inversa de clase $C^1, \forall (r,h,\varphi) \dimplies r\neq 0$.

 \subsection{9:}
 
 \paragraph{b: Calcular la inversa en (2,-2$\sqrt{3}$)}
 
 Resolver: $$\left\{\begin{matrix} 2 = rcos(\varphi)\\-2\sqrt{3} = rsen(\varphi)\end{matrix}\right.$$
 
 Hay que hallar la inversa de: $$\begin{pmatrix}
                                  \frac{1}{2}&2\sqrt{3}\\
                                  \frac{-\sqrt{3}}{2}&2
                                 \end{pmatrix}$$

                                 
  \subsection{13}
  
  $$det(J) = \begin{pmatrix}
              \dpa{f_1}{x}&\dpa{f_1}{y}\\
              \dpa{f_2}{x}&\dpa{f_2}{y}
             \end{pmatrix} = 
             \begin{pmatrix}
              \dpa{f_1}{x}&-\dpa{f_2}{x}\\
              \dpa{f_2}{x}& \dpa{f_1}{x}
             \end{pmatrix}
	    = \left(\dpa{f_1}{x}\right)^2 + \left(\dpa{f_2}{x}\right)^2 \implies \left(\dpa{f_1}{x},\dpa{f_2}{x}\right)$$
Esto es aplicando la primera ecuación de Cauchy-Riemman. Obteniendo una condición

Aplicando la otra condición en el jacobiano llegamos a $\displaystyle\left(\dpa{f_1}{y},\dpa{f_2}{y}\right)\neq (0,0)$
\paragraph{c)}

Queremos ver que $g(x,y) = (f_1(x,y)^2-f_2(x,y)^2,2f_1(x,y)f_2(x,y))$ cumple las ecuaciones de Cauchy-Riemman. Facilito.
\end{document}