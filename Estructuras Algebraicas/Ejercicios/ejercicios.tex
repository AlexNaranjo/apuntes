\documentclass{apuntes}

\begin{document}
\section{Hoja 1}
\section{Hoja 2}
\subsection{Problema 4}
Sea G un grupo. Demostrar que $Z(G) = \{X \tq X\in G (\forall Y \in G) XY = YX\}$

\begin{itemize}
\item $1 \in Z(G)$
\item {$X_1,X_2 \in G X_1Y = YX_1 \rightarrow X^{-1}XYX^{-1} = X^{-1}YXX^{-1} \implies X^{-1}Y=YX^{-1}$, es decir, el inverso también conmuta, por lo que los inversos de $X_1,X_2 \in Z(G)$}
\item {$X_1 X_2 Y = X_1 Y X_2 = YX_1 X_2, X_1 \cdot X_2 \in Z(G)$ el producto de 2 elementos del grupo está en el centro, por lo que es cerrado por la operación.}
\end{itemize}
\paragraph{a)}
$$Z(D_3), D_3 = \left\{\begin{matrix} 1,a,a^2\\b,ab,a^2b\end{matrix}\right\},\left\{\begin{matrix}a^3 = 1 = b^2\\ ba^j = a^{-j}b\end{matrix}\right\}$$

\paragraph{b)}
$$Z(D_4) =\left\{\begin{matrix} 1,a,a^2,a^3\\b,ab,a^2b,a^3b\\a^4=1=b^2\\ba=a^{-1}b\end{matrix}\right\}$$

$$Z(D_4) = \left\{x \tq x\in D_4 axa^{-1} = x = bxb^{-1}\right\}$$

Sin sentido...
$a^ib \notin Z(D_4). (a^2 \neq 1)$

$a^i \in Z(D_4) \dimplies (a^{2i} = 1)$

$Z(D_4) = \{1,a^2\} = <a^2>$
\subsection{5}

\paragraph{b)}
$C_{D_4}(b)$. Basta con comprobar la conmutación con $a^j$ y con $a^jb$ siendo $j = 0,1,2,3$, ya que con eso podemos ver la conmutación con todos los elementos. Se puede demostrar la conmutatividad multiplicando a derecha e iezquierda por $b$ y $b^{-1}$ y si nos queda $=1$, es conmutativo.

$$\left\{\begin{matrix}b(a^j)b^{-1} = a^{-j}, a^j \in C_{D_4}(b) \dimplies a^2j = 1\\
b(a^jb)b^{-1} = a^{-j} = a^{-j}b, a^jb \in C_{D_4}(b) \dimplies a^2j = 1\end{matrix}\right.$$

\end{document}