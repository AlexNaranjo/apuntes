\documentclass[a4paper,10pt]{apuntes}
\usepackage[utf8]{inputenc}
\usepackage{anysize} 
\usepackage{dsfont}
\usepackage{amssymb}
\marginsize{3cm}{2cm}{2cm}{2cm} 

%opening
\title{Estructuras algebraicas}
\author{Pedro Valero Mejía}

\newtheorem{theorem}{Theorem}[section]
\newtheorem{lemma}[theorem]{Lemma}
\newtheorem{proposition}[theorem]{Proposition}
\newtheorem{corollary}[theorem]{Corollary}

\newenvironment{proof}[1][Proof]{\begin{trivlist}
\item[\hskip \labelsep {\bfseries #1}]}{\end{trivlist}}
\newenvironment{definition}[1][Definition]{\begin{trivlist}
\item[\hskip \labelsep {\bfseries #1}]}{\end{trivlist}}
\newenvironment{example}[1][Example]{\begin{trivlist}
\item[\hskip \labelsep {\bfseries #1}]}{\end{trivlist}}
\newenvironment{remark}[1][Remark]{\begin{trivlist}
\item[\hskip \labelsep {\bfseries #1}]}{\end{trivlist}}
\newenvironment{notacion}[1][Notacion:]{\begin{trivlist}
\item[\hskip \labelsep {\bfseries #1}]}{\end{trivlist}}

\newcommand{\qed}{\nobreak \ifvmode \relax \else
      \ifdim\lastskip<1.5em \hskip-\lastskip
      \hskip1.5em plus0em minus0.5em \fi \nobreak
      \vrule height0.75em width0.5em depth0.25em\fi}

\begin{document}

\maketitle

\section{Grupos}
 
 Por definición, sea X un conjunto no vacío, podemos construir el conjunto de pares ordenados X*X=$\{(x,y)/x,y\in X\}$
 Vamos a fijar un conjunto X$\neq \varnothing$  y una función $\varphi$: X*X $\longrightarrow$  X que a cada par (x,y) le 
 asocia un elemento $\varphi$(x,y) $\in$X que expresamos como x*y, siendo '*' cualquier operación.
 \begin{definition}
  Sea S un subconjunto no vacío de G, diremos que S es cerrado por $\varphi$  si la combinación por $\varphi$  de dos elementos de
  S da otro elemento del mismo.
 \end{definition}
 Dados x,y,z $\in$ X puede ser interesante el resultado x*y*z pero, por definición, esto no tiene sentido. Sin embargo, si tendrían 
 sentido (x*y)*z ó x*(y*z). Estas operaciones podrían tener, o no, el mismo resultado. Queremos un conjunto con una operación 
 donde no tengamos que preocuparnos por la colocación de los paréntesis. Para ello debemos buscar una operación asociativa.\\
 
 Dado G$\neq\varnothing$  y $\varphi$: G*G $\longrightarrow$  G, diremos que G es un grupo si:
 \begin{enumerate}
  \item {(G,*) es asociativa}
  \item {$\exists$ e $\in$G t.q. $\forall$x$\in$G: x*e=x}
  \item {$\forall$x$\in$G $\exists$ x' $\in$G t.q. x*x'=x'*x=e}
 \end{enumerate}

 \begin{example}
   ($\mathds{Z}$, +); ($\mathds{R}$, +); ($\mathds{R}$-{0}, $\cdot$); ($\mathds{R}$/x$>$0, $\cdot$) son grupos, mientras que 
   ($\mathds{Z}$, $\cdot$); ($\mathds{R}$, $\cdot$) no lo son.
   
   A partir de un conjunto A definimos B(A) como el conjunto de todas las biyecciones de A en sí mismo. Puesto que la composicón 
   de dos biyecciones es otra biyección, la composición es una operación definida sobre B(A). Si tomamos como elemento 'e' la biyección 
   identidad y como x' la función inversa, podemos comprobar que B(A) es un grupo respecto a la composición.
 \end{example}
  
 En todo grupo se cumplen las propiedades de unicidad del elemento neutro y del inverso. La demostración de estas propiedades
 es bastante trivial.
 
 Sean dos elementos e, y e' dos elementos neutros de nuestro grupo G, se cumple que e*e'=e', pero tambien se cumple que e'*e=e. Esto 
 implica que e'=e.

 Por otro lado, si suponemos la existencia de dos elementos inversos $a',a''\in$G, entonces $e=a*a'=a*a''$. Si multiplicamos por $a$ en ambos lados de la ecuación tenemos: $a*(a*a')=(a*a')*a''$, pudiendo reordenar los paréntesis por la 
 propiedad asociativa. Así pues, obtenemos $a'=a''$.
 En esta última demostración nos hemos apoyado en la propiedad cancelativa, que sólo se presenta cuando trabajamos en un grupo.\\
  
 \begin{definition}
  Transformaciones lineales rígidas son aquellas que conservan las distancias. (En $\mathds{R}^{2}$  sólo están las simetrías y giros).\\
 \end{definition}
 
 \begin{example} 
  Vamos a trabajar con un triángulo equilátero, $\Delta$  en $\mathds{R}^{2}$  y vamos a encontrar el conjunto de todas las aplicaciones lineales rígidas 
  que llevan el triángulo en si mismo. $D_{3}=\{f\in G / f(\Delta)\longrightarrow\Delta\}$.
  Para empezar, dentro de este grupo encontramos todos los giros de águlo 120º. Si defino $a=g_{2\pi/3}$, tenemos las aplicaciones:
  $e$, $a$  y $a*a$, ya que la aplicación $a*a*a=e$, $a*a*a*a=a$  y así sucesivamente, por completar vueltas al círculo unidad.
  Por otro lado, también tenemos las simetrías que tienen como eje las alturas del triángulo. Denotaremos estas simetrías como: $S_{1}$, $S_{2}$  y $S_{3}$.
  Sabemos que la combinación de un giro y una simetría tiene como resultado otra simetría. Si combinamos $a*S_{1}$  obtenemos otra simetría, 
  que también deja el triángulo en si mismo. Así pues, esta simetría, debe tratarse de $S_{2}$  ó $S_{3}$. Lo mismo ocurre con $a*a*S_{1}$.
  
  Así, tenemos: $D_{3}=\{e, a, a*a, S_{1}, a*S_{1}, a^{2}*S_{1}\}$
 \end{example}

 La representación geométrica de un grupo consiste en la descripción de los elementos geométricos que lo constituyen. En el 
 caso del ejemplo, consistiría en indicar qué giros y simetrías constituyen el grupo.
 La representación abstracta o algebraica de un grupo suele realizarse por medio de una tabla, o una serie de restricciones sobre las operaciones
 de combinación de los elementos del grupo, sin necesidad de indicar qué es realmente cada elemento.
 \begin{example}
  La representación abstracta de $D_{3}$  viene dada por tres condiciones:
 \begin{itemize}
  \item ord(g)=3
  \item ord(s)=2
  \item $g*s=s*g^{2}$
 \end{itemize}
 Ya que con estas condiciones podríamos construir una tabla con todas las combinaciones 2 a 2 de elementos del grupo sin necesidad
 de saber nada acerca de esos elementos.
 \end{example}
 
 \begin{definition}
  Se dice que un elemento $a\in G$, siendo G un grupo, tiene orden finito si $\exists k\in\mathds{N}$  t.q  $a^{k}=e$
 \end{definition}
 
 \begin{definition}
  Dado un elemento de orden finitio, decimos que su orden es el menor entero positivo con el que se cumple $a^{k}=e$
 \end{definition}

 Sea G un conjunto y $\varphi$  la operación con la que forma un grupo, vamos a ver cuando un subconjunto de G es un grupo de forma natural,
 esto es lo que denominaremos subgrupo.
 
 \begin{definition}
  Diremos que un subconjunto no vacío S es un subgrupo si:
  \begin{enumerate}
   \item S es cerrado por la operación
   \item e $\in$S
   \item s $\in$S$\Rightarrow$ $s^{-1}\in$S.
  \end{enumerate}
 \end{definition}
 
  \begin{theorem}
   Dados $S_{1}$, $S_{2}$ subgrupos de G $\Rightarrow$  $S_{1}\cap S_{2}$  es un subgrupo de G.
  \end{theorem}

  Este teorema también puede aplicarse con una intersección numerable de grupos.
  
  \begin{definition}
   Fijado un elemento g $\in$G, definimos el grupo generado por g como: $<g>$=$\{$..., $g^{-2}$, $g^{-1}$, e, g, $g^{2}$,...$\}$. Este grupo
  es un subgrupo de G
  \end{definition}
  
  \begin{theorem}
   Si H es un grupo de G y g$\in$H $\Rightarrow$ $<g>\subset$H.   
  \end{theorem}

  A partir de un conjunto C=$\{g_{1}$, $g_{2}$, $g_{3}$... $g_{s}\}$, contenido en G, vamos a buscar e menor subgrupo que los contiene.
  $<g_{1}$, $g_{2}$, $g_{3}$... $g_{s}>$=$\displaystyle \cup_{k\in\mathds{N}}$  $\{$ $a_{1}$* $a_{2}$* $a_{3}$... $a_{k}$/($a_{i}\in$C)$\wedge$($a_{i}\in C^{-1}$)$\}$. Es un subgrupo que contiene a 
  todos los elementos de C.
  
  Si H es un subgrupo que contiene a los elementos de C, el grupo generado por esos elementos se contiene en H.
  
  \begin{notacion}
   H$<$G denota H subgrupo de G
  \end{notacion}
  
  \begin{definition}
   Dado H$<$G se dice que H es cícilico si existe g$\in$G t.q. H=$<g>$ 
  \end{definition}

  \begin{theorem}
   Si G es un grupo finito y S$\cap$G es un subconjunto no vacío $\Rightarrow$ S es un grupo $\Leftrightarrow$ S es cerrado por la operación.   
  \end{theorem}
  \begin{proof}
   La implicación hacia la derecha es obvia por la propia definición de subgrupo.
   Para la implicación hacia la izquierda partimos de que S es cerrado y finito. Por tanto $\exists$ d $\in$ S t.q. ord(d)=n y $<d>\subset S$
   $\Rightarrow$ $d^{n}=e\in S$ y $d^{n-1}=d^{-1}\in S$ $\Rightarrow$ S es un grupo
  \end{proof}

  Dentro de los grupos podríamos realizar una claisficación según fueran finitos o finitos, por ejemplo. No obstante, nos resultará
  más interesante la clasificación de grupos según sean abelianos o no.
  
  \begin{definition}
   Un grupo es abeliano si cumple la propiedad conmutativa.
  \end{definition}

  \begin{example}
   ($\mathds{Z}$, +); ($\mathds{Z}$/$n\mathds{Z}$, +) y $<s,g^{2}>$=$\{1,g^{2},s,sg^{2}\}$ son abelianos
  \end{example}
  
  \begin{lemma}
   Todo subgrupo cíclico de un grupo G es abeliano. Por tanto un grupo no abeliano no puede ser cíclico.
  \end{lemma}

  \begin{example}
   Dado $D_{4}=\{id, g, g^{2}, g^{3}, s, sg, sg^{2},sg^{3}\}$, (Recordemos que era el conjunto de aplicaciones que mantenían un
   cuadrado invariable), vamos a ver los grupos cíclicos contenidos en él.
   $<1>={1}$, $<g>$=$\{1,g,g^{2}, g^{3}\}$, $<g^{2}>$=$\{1,g^{2}\}$, $<g^{3}>=\{1,g, g^{3}\}$, $<s>$=$\{1,s\}$, $<sg>$=$\{1, sg\}$, $<sg^{2}>$=$\{1, sg^{2}\}$, $<sg>$=$\{1, sg^{3}\}$.
   
   Además podemos destacar el caso de ($\mathds{Z}$, +), un grupo cíclico para el cual, todo subgrupo es también cíclico. Esto se demuestra
   de forma general considerando que un grupo no cíclico estaría generado por varios elementos. En este caso, el máximo común
   divisor de estos número sería generador del grupo. Por tanto, el grupo sería cíclico.
 
   De forma más estricta podemos decir que dado un subgrupo H$<\mathds{Z}$ podrá ser H=$\{0\}$ ó H$\neq\{0\}$. 
   En el primer caso, H ya sería cíclico. En el segundo caso, tenemos que $\exists$  d $\in$  H t.q. d$\neq$0. Lo que implica
   que $<d>\subset H$. 
   
   La duda sería si es cierto o no H$\subset <d>$. Sea h $\in$H  $\Rightarrow$  h=qd+r. Puesto que tanto h como d pertenecen a H,
   tenemos que r pertenece a H tambien. Esto implica que r puede expresarse r=dp. Lo que conlleva h=qd+pd=(q+p)d. Por tanto h $\in <d>$
   
  \end{example}

  \begin{theorem}
   En un grupo finito G todo elemento tiene orden finito. Además si g$\in$G tiene ord(g)=k $\Rightarrow$  $<g>$  tiene k elementos.
  \end{theorem}

  \begin{theorem}
   $\forall$  g $\in$ G ord(g)=k $\Rightarrow$  card($<g>$)=k
  \end{theorem}

  \begin{definition}
   Un retículo de subgrupos es aquel retículo (estructura algebraica parcialmente ordenada) formado por subgrupos de un determinado grupo
   con una relación de contención. En este retículo, la unión de dos subgrupos es el subconjunto generado por su conexión.
  \end{definition}

  \begin{example}
   Tomamos una vez más el grupo $D_{4}=\{1,g,g^{2}, g^{3}, s, sg, sg^{2}, sg^{3}\}$. Aquí obtenemos el retículo:
   
  \end{example}

\end{document}
