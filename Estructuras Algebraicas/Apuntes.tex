\documentclass[a4paper,10pt]{apuntes}
\usepackage{anysize} 
\usepackage{dsfont}
\usepackage{amssymb}
\usepackage{textcomp}
\usepackage{plain}

%opening
\title{Estructuras algebraicas}
\author{Pedro Valero Mejía\\Guillermo Ruiz Álvarez}

\newenvironment{notacion}[1][Notación:]{\begin{trivlist}
\item[\hskip \labelsep {\bfseries #1}]}{\end{trivlist}}

\begin{document}

\maketitle

\section{Grupos}
 
 Por definición, sea X un conjunto no vacío, podemos construir el conjunto de pares ordenados $X\ast X=\{(x,y)\tq x,y\in X\}$
 
 Vamos a fijar un conjunto $X\neq \emptyset$  y una función $\appl{\varphi}{X\ast X}{X}$ que a cada par $(x,y)$ le 
 asocia un elemento $\varphi(x,y) \in X$ que expresamos como $x\ast y$, siendo '$\ast$' cualquier operación.
 
 \begin{defn}[Subconjunto\IS cerrado]
  Sea S un subconjunto no vacío de G, diremos que S es cerrado por $\varphi$  si la combinación por $\varphi$  de dos elementos de
  S da otro elemento del mismo.
 \end{defn}
 
 Dados $x,y,z \in X$ puede ser interesante el resultado $x\ast y\ast z$ pero, por definición, esto no tiene sentido. Sin embargo, si tendrían 
 sentido $(x\ast y)\ast z$ ó $x\ast (y\ast z)$. Estas operaciones podrían tener, o no, el mismo resultado. Queremos un conjunto con una operación 
 donde no tengamos que preocuparnos por la colocación de los paréntesis. Para ello debemos buscar una operación asociativa.
 
 \begin{defn}[Grupo]
 Dado $G\neq \emptyset$  y $\appl{\varphi}{G\ast G}{G}$, diremos que $G$ es un grupo si:
 \begin{enumerate}
  \item $\ast$ es asociativa
  \item \textbf{Existencia del neutro} $\exists  e \in G \tq \; \forall\, x\in G\; x\ast e=x$
  \item \textbf{Existencia del inverso} $\forall x \in G \exists x' \in G \tq x\ast x'=x'\ast x=e$
 \end{enumerate}
\end{defn}


 \begin{example}
   $ (\ent, +); (\real, +); (\real, \cdot); (\real/x>0, \cdot)$ son grupos, mientras que 
   $(\ent, \cdot); (\real, \cdot)$ no lo son.
   
   A partir de un conjunto A definimos B(A) como el conjunto de todas las biyecciones de A en sí mismo. Puesto que la composicón 
   de dos biyecciones es otra biyección, la composición es una operación definida sobre B(A). Si tomamos como elemento 'e' la biyección 
   identidad y como x' la función inversa, podemos comprobar que B(A) es un grupo respecto a la composición.
 \end{example}
  
  
  \begin{theorem}[Unicidad\IS del neutro e inverso]
   En todo grupo se cumplen las propiedades de unicidad del elemento neutro y del inverso.
  \end{theorem}
  
  \begin{proof}
 Primero, demostramos que el neutro es único. Sean dos elementos $e$, y $e'$ dos elementos neutros de $G$, se cumple que $e\ast e'=e'$, pero tambien se cumple que $e'\ast e=e$. Esto implica que $e'=e$.

 Por otro lado, si suponemos la existencia de dos elementos inversos $a',a''\in G$, entonces $e=a\ast a'=a\ast a''$. Si multiplicamos por $a$ en ambos lados de la ecuación tenemos: $a\ast (a\ast a')=(a\ast a')\ast a''$, pudiendo reordenar los paréntesis por la 
 propiedad asociativa. Así pues, obtenemos $a'=a''$.
  \end{proof}
  
 \begin{defn}[Transformación\IS lineal rígida]
  Transformaciones lineales rígidas son aquellas que conservan las distancias. (En $\real^{2}$  sólo están las simetrías y giros).
 \end{defn}
 
 \begin{example} 
  Vamos a trabajar con un triángulo equilátero, $\Delta$  en $\real^{2}$  y vamos a encontrar el conjunto de todas las aplicaciones lineales rígidas 
  que llevan el triángulo en si mismo. $D_{3}=\{f\in G / f(\Delta)\longrightarrow\Delta\}$.
  
  Para empezar, dentro de este grupo encontramos todos los giros de ángulo 120º. Si defino $a=g_{2\pi/3}$, tenemos las aplicaciones:
  $e$, $a$  y $a\ast a$, ya que la aplicación $a\ast a\ast a=e$, $a\ast a\ast a\ast a=a$  y así sucesivamente, por completar vueltas al círculo unidad.
  
  Por otro lado, también tenemos las simetrías que tienen como eje las alturas del triángulo. Denotaremos estas simetrías como: $S_{1}$, $S_{2}$  y $S_{3}$.
  
  Sabemos que la combinación de un giro y una simetría tiene como resultado otra simetría. Si combinamos $a\ast S_{1}$  obtenemos otra simetría, que también deja el triángulo en si mismo. Así pues, esta simetría, debe tratarse de $S_{2}$  ó $S_{3}$. Lo mismo ocurre con $a\ast a\ast S_{1}$.
  
  Así, tenemos: $D_{3}=\{e, a, a\ast a, S_{1}, a\ast S_{1}, a^{2}\ast S_{1}\}$
 \end{example}

 La representación geométrica de un grupo consiste en la descripción de los elementos geométricos que lo constituyen. En el caso del ejemplo, consistiría en indicar qué giros y simetrías constituyen el grupo.
 
 La representación abstracta o algebraica de un grupo suele realizarse por medio de una tabla, o una serie de restricciones sobre las operaciones de combinación de los elementos del grupo, sin necesidad de indicar qué es realmente cada elemento.
 \begin{example}
  La representación abstracta de $D_{3}$  viene dada por tres condiciones:
 \begin{itemize}
  \item $ord(g)=3$
  \item $ord(s)=2$
  \item $g\ast s=s\ast g^{2}$
 \end{itemize}
 
Con estas condiciones podríamos construir una tabla con todas las combinaciones 2 a 2 de elementos del grupo sin necesidad
 de saber nada acerca de esos elementos.
 \end{example}
 
 \begin{defn}[Orden\IS finito]
  Se dice que un elemento $a\in G$, siendo G un grupo, tiene orden finito si $\exists k\in\mathds{N}$  t.q  $a^{k}=e$
 \end{defn}
 
 \begin{defn}[Orden]
  Dado un elemento de orden finito, decimos que su orden es el menor entero positivo con el que se cumple $a^{k}=e$
 \end{defn}
 
\subsection{Subgrupos}
 Sea G un conjunto y $\varphi$  la operación con la que forma un grupo, vamos a ver cuando un subconjunto de G es un grupo de forma natural,
 esto es lo que denominaremos subgrupo.
 
 \begin{defn}[Subgrupo]
  Diremos que un subconjunto no vacío S es un subgrupo si:
  \begin{enumerate}
   \item $S$ es cerrado por la operación, esto es $x,y \in S \implies \varphi(x,y) \in S$
   \item $e \in S$
   \item $s \in S \implies s^{-1}\in S$.
  \end{enumerate}
 \end{defn} 
 
  \begin{theorem}
   Dados $S_{1}$, $S_{2}$ subgrupos de G, entonces $S_{1}\cap S_{2}$  es un subgrupo de G.
  \end{theorem}

  Este teorema también puede aplicarse con una intersección numerable de grupos.
  
  \begin{defn}[Grupo\IS generado]
   Fijado un elemento $g \in G$, definimos el grupo generado por $g$ como:
   \[ \gen{g} =\{\dotsc, g^{-2}, g^{-1}, e, g, g^{2},\dotsc \}\]. Este grupo es un subgrupo de G
  \end{defn}

  \begin{theorem}
   Si H es un subgrupo de G y $g\in H$ entonces $\gen{g}\subset H$.   
  \end{theorem}
\begin{proof}
  A partir de un conjunto \[ C=\{g_{1}, g_{2}, g_{3}, \dotsc g_{s} \}\], contenido en $G$, vamos a buscar el menor subgrupo que lo contiene.
  \[ \gen{g_{1}, g_{2}, g_{3}g_{s}} =\bigcup_{k\in\mathds{N}}\; \{ a_{1} \ast a_{2}\ast a_{3} \ast \dotsb  a_{k} \tq (a_{i}\in C) \y (a_{i}\in C^{-1})\} \]. Es un subgrupo que contiene a  todos los elementos de C.
  
  Si H es un subgrupo que contiene a los elementos de C, el grupo generado por esos elementos se contiene en H.
  \end{proof}
  
  \begin{notacion}
   $H<G$ denota H subgrupo de G
  \end{notacion}
  
  \begin{defn}[Grupo\IS cíclico]
   Dado $H<G$ se dice que $H$ es cícilico si existe $g\in G \tq H=\gen{g}$.
  \end{defn}

  \begin{theorem}
   Si G es un grupo finito y $S\subset G$ es un subconjunto no vacío entonces $S$ es un subgrupo si y sólo si $S$ es cerrado por la operación.   
  \end{theorem}
  
  \begin{proof}
   La implicación hacia la derecha es obvia por la propia definición de subgrupo.
   
   Para la implicación hacia la izquierda partimos de que S es cerrado y finito. Por tanto $\exists d \in S \tq \ord{d}=n$ y $\gen{d} \subset S$. Entonces
   
\[ d^{n}=e\in S \y d^{n-1}=d^{-1}\in S\]

Por lo tanto, $S$ es un grupo
  \end{proof}

  Dentro de los grupos podríamos realizar una claisficación según fueran finitos o infinitos, por ejemplo. No obstante, nos resultará
  más interesante la clasificación de grupos según sean abelianos o no.
  
  \begin{defn}[Grupo\IS abeliano]
   Un grupo es abeliano si cumple la propiedad conmutativa.
  \end{defn}

  \begin{example}  $(\ent, +); (\ent/n\ent, +); \gen{s,g^{2}} =\{1,g^{2},s,sg^{2}\}$ son abelianos
  \end{example}
  
  \begin{lemma}
   Todo subgrupo cíclico de un grupo G es abeliano. Por tanto un grupo no abeliano no puede ser cíclico.
  \end{lemma}

  \begin{example}
   Dado $D_{4}=\{id, g, g^{2}, g^{3}, s, sg, sg^{2},sg^{3}\}$, (Recordemos que era el conjunto de aplicaciones que mantenían un cuadrado invariante), vamos a ver los grupos cíclicos contenidos en él.
   
  \begin{gather*}
  \gen{1}=\{1\} \\
  \gen{g} =\{1,g,g^{2}, g^{3}\}\\
  \gen{g^{2}}=\{1,g^{2}\}\\
  \gen{g^{3}}=\{1,g, g^{3}\}\\
  \gen{s} =\{1,s\}\\
  \gen{sg}=\{1, sg\}\\
  \gen{sg^{2}} =\{1, sg^{2}\}\\
  \gen{sg}=\{1, sg^{3}\}
  \end{gather*}
   
   Además podemos destacar el caso de ($\ent$, +), un grupo cíclico para el cual todo subgrupo es también cíclico. Esto se demuestra
   de forma general considerando que un grupo no cíclico estaría generado por varios elementos. En este caso, el máximo común
   divisor de estos número sería generador del grupo. Por tanto, el grupo sería cíclico.
 
   De forma más estricta podemos decir que dado un subgrupo $H<\ent$ podrá ser $H=\{0\}$ ó $H\neq\{0\}$. 
   En el primer caso, H ya sería cíclico. En el segundo caso, tenemos que $\exists  d \in  H$ t.q. $d\neq 0$. Lo que implica
   que $\gen{d} \subset H$. 
   
   La duda sería si es cierto o no $H\subset \gen{d}$. Sea $h \in H$, entonces $h=qd+r$. Puesto que tanto $h$ como $d$ pertenecen a $H$,
   tenemos que $r$ pertenece a $H$ también. Esto implica que $r$ puede expresarse $r=dp$, lo que conlleva $h=qd+pd=(q+p)d$. Por tanto $h \in \gen{d}$.
  \end{example}

  \begin{theorem}
   En un grupo finito G todo elemento tiene orden finito. Además si $g\in G$ tiene $\ord{g}=k$, entonces $\gen{g}$ tiene $k$ elementos.
  \end{theorem}

  \begin{defn}[Retículo\IS de subgrupos]
   Un retículo de subgrupos es aquel retículo (estructura algebraica parcialmente ordenada) formado por subgrupos de un determinado grupo
   con una relación de contención. En este retículo, la unión de dos subgrupos es el subconjunto generado por su conexión.
  \end{defn}

  \begin{example}
   Tomamos una vez más el grupo $D_{4}=\{1,g,g^{2}, g^{3}, s, sg, sg^{2}, sg^{3}\}$. Aquí obtenemos el retículo:
  % \easyimgw{Reticulo.png}{Leyenda}{lblEtiqueta}{0.3}
  
  Aquí iría una imagen.
  \end{example}

  \begin{theorem}[Teorema de Lagrange]
   Si G es un grupo finito y H un subgrupo de G, entonces el número de elementos de H divide el número de elementos de G.
  \end{theorem}
  
  \begin{lemma}
   Sea $\varphi: G\rightarrow G$, son equivalentes:
   \begin{enumerate}
    \item $\varphi$  es inyectiva.
    \item $\varphi$  es biyectiva.
    \item $\varphi$  es sobreyectiva.
   \end{enumerate}
  \end{lemma}
  
  \begin{proof}
  La demostración de este lema es totalmente trivial. Con apoyarnos en que el conjunto G es finito, puede observarse que una de
  esas propiedades implica directamente las demas.
  \end{proof}
  
  \begin{lemma}
   Si $\varphi$  es un biyección y A,B son subconjuntos de G:
   \begin{enumerate}
    \item $\text{card}(A)=\text{card}(\varphi(A))$
    \item $A=B \dimplies \varphi(A)=\varphi(B)$
    \item $\varphi(A\cap B)=\varphi(A)\cap\varphi(B)$
   \end{enumerate}
  \end{lemma}
  
  \begin{lemma}
   Sea G un grupo finito y $g \in G$, entonces $\varphi_{g}(x)=g\ast x$. Además, por la propiedad cancelativa, podemos ver que $\varphi_{g}$ es inyectiva.
   
   Además, por ser G finito, sabemos que $\varphi_{g}$ es biyectiva.
  \end{lemma}
  También podemos definir $\varphi_{g} H=\{g\ast h\tq h\in H\}$
  
  \begin{defn}[Clase\IS lateral]
  \[  gH=\varphi_{g}(H) \]
  \end{defn}

  \begin{corol} 
  \label{corol1}
  Extraemos las siguientes conclusiones:
   \begin{enumerate}
    \item $g \in gH$
    \item $\text{card}(H)=\text{card}(gH)$
    \item $H=gH \dimplies g \in H$ 
   \end{enumerate}
  \end{corol}

  \begin{proof}
   \begin{enumerate}
    \item $e \in  H$. Por tanto $g\ast e =g$ que por hipótesis está en $gH$.
    \item Se puede ver apoyándonos en los lemas anteriores puesto que H es finito.
    \item $\Rightarrow$):  $e \in H \implies e=gh$ con $h \in H$. Por lo tanto $h=g^{-1} \in H$ y entonces $g \in H$.\\
	  $\Leftarrow$):  $g \in H \implies g^{-1}\in H$ y todo elemento $h \in H$ cumple $h=g(g^{-1}h)\in H$.
   \end{enumerate}
  \end{proof}
  
  \begin{prop}
   Dados $g_{1}, g_{2}\in G$ y $H<G$, entonces \[ g_{1}H=g_{2}H \] y \[g_{1}H\cap g_{2}H=\emptyset \]
  \end{prop}
  
  \begin{proof}
   La intersección $g_{1}H\cap g_{2}H$ no será vacía si y sólo si $\exists\, h_{1},h_{2} \in H$ tal que $g_{1}h_{1}=g_{2}h_{2}$. Operando 
   \begin{gather*}
   h_{1}=g_{1}^{-1}g_{2}h_{2} \\
   h_{2}^{-1}h_{1}=g_{1}^{-1}g_{2}
   \end{gather*}
   
   Por ser $h_2$ y $h_1$ pertenecientes a $H$, tenemos que $g_{1}^{-1}g_{2} \in H$. Usando la propiedad 3 de (\ref{corol1}), nos queda que
    \[ g_{1}^{-1}g_{2}H=H  \] y por lo tanto \[ g_{2}H=g_{1}H \]
  \end{proof}
  
  \begin{example}
   Tomando el famoso grupo $D_{4}$, podemos obtener H=$\{1,g,g²,g³\}$; sH=$\{s,sg,sg²,sg³\}$
  \end{example}
  
  
  
  \subsection{Particiones}

  \begin{defn}[Relación\IS de equivalencia]
   Fijado un conjunto $G\neq\varnothing$. Una relación $\rel$ en $G$ es de equivalencia si cumple las siguientes propiedades:
  
   \begin{description}   
    \item[Reflexiva] $\forall x\in G \;  x\rel x$
    \item[Simétrica] $\forall x,y \in G\;  x\rel y \dimplies y \rel x$
    \item[Transitiva] $\forall x,y,z \in G\;  x\rel y \y  y\rel z \implies x\rel z$
   \end{description}  
  \end{defn}
 
  \begin{defn}[Partición]
   Familia de subconjuntos disjuntos dos a dos tales que su unión constituye el total. 
  \end{defn}
  
  Una partición define una relación de equivalencia y viceversa. Si $\rel$ es una relación de equivalencia en un grupo $G$, 
  definimos una partición en la que los subconjuntos (clases de equivalencia) son de la forma: \[ S_{x}=\{y\in G \tq x\rel y\} \]
  
  \begin{proof} Definimos una relación de equivalencia $\rel$ en $G$ a partir del grupo $H$:
  
   \[ g_{1}\rel g_{2} \Leftrightarrow g_{1}^{-1}g_{2}\in H \]  y comprobamos que, efectivamente, esta relación es de equivalencia.
   
   Tenemos $S_{e}=H$. Así $S_{g}$  o ya cubre junto con H todo G, o cojo otro $g^{'}$  y repito el proceso con $S_{g^{'}}$.
   Así formare una serie de grupos de la forma $S_{x}$  disjuntos dos a dos y cuya unión me da G.\\
   De esta forma podemos ver que $\card{G}=\card{H}  s$
  \end{proof}
  
  \begin{corol}[Teorema de Lagrange]
   Dado $g \in  G$, siendo $G$ un grupo finito, entonces $\ord{g}$ divide a $\card{G}$. Así, suponiendo que el cardinal de $G$ es primo, los únicos subgrupos que tiene $G$ son los triviales: $\gen{e}, G$
  \end{corol}
  
  \begin{theorem}
   Si el cardinal de $G$ es primo, entonces $G$ es cíclico. Además, si $|G|=n$, entonces \[ \ord{g} | \, n \; \forall g\in G\] y $g^n = e$.
  \end{theorem}
  
  \begin{example}
   Con ayuda de este teorema puede demostrarse el pequeño teorema de Fermat. 
   Definimos $\ent/p\ent=\{\bar{0},\bar{1}...,\bar{p-1}\}$.
   Si tomamos el grupo de las unidades de $\ent/p\ent$ tenemos $\{\bar{1}, \bar{2},....\bar{p-1}\}$ $\Rightarrow$
   $\bar{a}\in\ent/p\ent \wedge \bar{a}\neq 0 \Rightarrow a^{p-1}=\bar{1}$.
  \end{example}
  
  \begin{theorem}
   Si $|G|=p^{2}$  con $p$ primo entonces $\exists g \in G \tq \ord{g}=p$
  \end{theorem}
  
  \begin{proof}
   Tomamos $g\in G \y g\neq e$. Entonces sólo hay dos posibilidades, o bien $\ord{g}=p$ o bien $\ord{g}=p^{2}$. En el primer caso ya lo tenemos, y en el segundo tenemos que existe $g^p$ que tiene orden $p$.
  \end{proof}
  
  \begin{example}
   Por todo lo explicado anteriormente, si tomamos el anillo de polinomios $\ent/p\ent[x]$, tenemos \[ X^{p-1}-\bar{1}=
   \prod_{\bar{a}\in\ent/p\ent \wedge \bar{a}\neq 0}(x-\bar{a}) \]
  \end{example}
  
  \begin{theorem}[Teorema de Lagrange]
   Dados $H<G$,  se cumple que \[ \exists a_{1},\dotsc,a_{r}\in G\tq G=a_{1}H \cup a_{2}H \cup \dotsb \cup a_{r}H\] y que \[ a_{i}H\cap a_{j}H=\emptyset \forall i\neq j\] 
   Es decir, $\card{G}=r\card{H}$
  \end{theorem}

  Definimos ahora otra relación a partir de $H<G$: \[ c\rel^{'}d \Leftrightarrow cd^{-1}\in H \]. La comprobación de que esto es una relación
  la omitiremos por ser trivial. En esta partición, el conjunto de elementos relacionados con d es: $Hd=\{hd\tq h\in H\}$
  
  \begin{defn}[Subgrupo\IS normal]
   Decimos que $H<G$  es un subgrupo normal si \[ gH=Hg \; \forall g \in G \]. Por tanto, si $G$ es un grupo abeliano, $H$ también lo será y cualquier  subgrupo será normal. Un subgrupo normal se expresa como $H\lhd G$.
  \end{defn}
  
  \begin{example}
   Tomamos G=$\ent$  y H=$4\ent$. 
   El subgrupo de los elementos que son equivalentes a n es nH=$\{n+4k\tq k \in \ent\}$.
   En este caso tenemos 4 subgrupos según esta condición: $\bar{0},\bar{1},\bar{2},\bar{3}$
  \end{example}
  
  Si $H\lhd G$  podemos definir una estructura de grupo en el subconjunto de clases, lo que se llamará el grupo cociente.
  
  \begin{defn}[Grupo\IS cociente] Sea $H \lhd G$. Definimos $G / H$ como el conjunto de todas las clases laterales izquierdas, es decir
  
  \[ G/H = \{ aH \tq a \in G\} \]
 \end{defn}
  
  
  Dados $g_{1}H$  y $g_{2}H$  podemos formar el conjunto $\{h_{1}\ast h_{2}\tq h_{1}\in g_{1}H \wedge h_{2}\in g_{2}H\}$. Si operamos   $g_{1}H\ast g_{2}H=(g_{1}\ast g_{2})H$  que es otra clase. Por tanto el grupo de clases es cerrado.
  
  El neutro sería la caja $H$ puesto que $gHH=gH$ para cualquier $g$ que escojamos. Además el inverso de $gH$ es $g^{-1}H$.
  
  Por último, vemos que $(g_{1}H\ast g_{2}H)\ast g_{3}H=g_{1}H\ast Hg_{2}\ast g_{3}H=g_{1}H\ast Hg_{3}\ast g_{2}H$, por ser G asociativo. Queda probado pues, que el conjunto de las clases de equivalencia, forma un grupo.
  
\begin{defn}[Índice\IS de un subgrupo] El índice de un subgrupo $H$ de $G$ es el \textit{tamaño relativo} de $H$ en $G$, el número de \textit{copias} de $H$ que llenarían $G$.  

Estrictamente, se define $[G:H]$ es el número de clases laterales de $H$ en $G$. Si $H$ y $G$ son finitos, entonces 

\[ [G:H] = \frac{\card{G}}{\card{H}} \]
\end{defn}
  
  \begin{theorem}
   \[ H< G \y [G:H]=2 \implies H \lhd G \]
  \end{theorem}
  
  \begin{proof}
   Por un lado tenemos que $g\notin H \implies gH= H^{c}$. Por otro lado, si tomamos la otra relación de equivalencia, a partir de $g \notin H obtenemos Hg=H^{c}$.
   
   Uniendo estos dos ejemplos, donde mantenemos H como una mitad en ambos, tenemos Hg=gH. Por tanto H es un subgrupo normal.
   
   Guille: Me lo creo pero me convence cero. ¿Qué es $H^c$? ¿Cuál es la otra relación de equivalencia?
  \end{proof}
  
  \begin{lemma}
   Dadas dos particiones formadas por los subconjuntos $aH$ y $Ha$ para un elemento $a \in G$ y con $H <G$, podemos definir dos aplicaciones $\pi$ y $\pi '$ que lleven cada elemento de G a su respectiva caja \textit{¿qué es una caja?}:
   \[ \appl{\pi}{G}{G/H} \]
   \[ \appl{\pi'}{G}{G/H} \]
   
   Tenemos entonces que:
   
   \[ H \text{subgrupo normal} \implies a_iH = Ha_i \]
   \[ H \text{subgrupo normal} \implies \pi \ =\ \pi ' \]
   
   Además, si H es normal
   \begin{enumerate}
	\item Podemos definir una operación en el conjunto cociente.     
	\item $\appl{\pi}{G}{G/H}$ es compatible con las operaciones.
   \end{enumerate}
  \textit{Pues vale.}
  \end{lemma}
  
\begin{defn}[Centro de un grupo]
	Llamamos centro del grupo $G$ al grupo formado por todos los elementos que conmutan con todos los elementos de $G$.:\\
	\[ Z(G) = \{ a\in G\tq a\ast g = g\ast a\ \forall g\in G\} \]
	Además, tenemos que:
	\[ aZ(G)=Z(G)a \implies Z(G)\lhd G \]
  \end{defn}  
  
  \begin{example}
	Sea $Q$ el grupo formado por las raíces cuartas de 1:
	\[ Q = \{ 1, i, j, k, -1, -i, -j, -k\} \] 
	
	Podemos ver que $Q$ se genera a partir de los elementos $i, j, k$: $Q = \gen{i,j,k}$.
	Además vemos que $-1$ conmuta con todos los elementos del grupo: $-1 \in Z(Q)$
	Analizamos el retículo del grupo y vemos que esta formado por los subgrupos:
	
	\begin{gather*}
	Orden\; 1 = \{ 1\}\\
	Orden\; 2 = \{ 1, -1\} \\
	Orden\; 4 = \{ 1, i, i^2, i^3\},\
			  \{ 1, j, j^2, j^3 \},\
			  \{ 1, k, k^2, k^3 \}.\\
			  \end{gather*}
			  
		  
	El grupo no es abeliano, sin embargo, todos sus subgrupos son normales porque todo subgrupo $H<G$ de índice 2 (es decir, el conjunto de clases $G/H$ tiene dos elementos) es normal.
	
  \end{example}
  
 
 Dados un grupo G y un subconjunto no vacío S, defino $S^{-1}=\{s^{-1}\tq s\in S\}$. Dados $S_{1},S_{2}<G$, defino  \[ (S_{1}\ast S_{2})^{-1}=\{s_{1}\ast s_{2}\tq s_{1}\in S_{1} \wedge s_{2}\in S_{2}\} \]
 
 que operando, nos queda que 
 
 \begin{align*}
 (S_1 \ast S_2)^{-1} &= \{s_{1}^{-1}\ast s_{2}^{-1}\tq s_{1}\in S_{1} \wedge s_{2}\in S_{2}\}\\
 &= \{s_{1}^{-1}\ast s_{2}^{-1}\tq s_{1}^{-1}\in S_{1}^{-1} \wedge s_{2}^{-1}\in S_{2}^{-1}\} \\
 & =S_{2}^{-1}\ast S_{1}^{-1}
 \end{align*}
 
 Sea H un subgrupo de G, entonces $H^{-1}=H$.
 
 \begin{theorem}
  Fijado un grupo G y $H_{1}, H_{2}$  subgrupos de G entonces: $H_{1} \lhd G \implies H_{1}\ast H_{2}<G$. 
 \end{theorem}
 
 \begin{proof}
  Sabemos que $H_{1}\ast H_{2}=H_{2}\ast H_{1}$. Para ver que $H_{1}\ast H_{2}<G$  hay que comprobar que está
  \begin{enumerate}
   \item Cerrado por la operación: $(H_{1}\ast H_{2})\ast (H_{2}\ast H_{1})=H_{1}\ast (H_{2}\ast H_{2})\ast H_{1}=H_{1}\ast H_{1}\ast H_{2}\ast H_{2}=H_{1}\ast H_{2}$
   \item Todo elemento tiene inverso, es decir: $\alpha\in H_{1}\ast H_{2}\Rightarrow \alpha^{-1}\in H_{1}\ast H_{2}$. Pero entonces tenemos:
   $\alpha^{-1}\in (H_{1}\ast H_{2})^{-1} = H_{2}^{-1}\ast H_{1}^{-1}=H_{2}\ast H_{1}=H_{1}\ast H_{2}$ que también contiene a $\alpha$ \textit{por ejemplo.}
  \end{enumerate}

 \end{proof}

 \begin{example}
  Tomamos el ya famoso grupo $D_{4}=\{1,g,g^{2},g^{3}, s, sg, sg^{2}, sg^{3}\}=\gen{s,g}$.
  
  Dentro de este grupo, el elemento 1 tiene orden dos, los elementos $g$, $g^{3}$  tienen orden 4 y el resto, 2.
  \begin{gather*}
  Z(D_{4})= \{\alpha\in D_{4} \tq \alpha\beta=\beta\alpha \ \forall \beta \in D_{4}\}= \\
  = \{\alpha\in D_{4}\tq \alpha s = s\alpha \wedge \alpha g = g\alpha\}=\{1, g^{2}\}=\gen{g^{2}}
  \end{gather*} 
  
  Este subgrupo es normal.
  
  Para contruir el retículo de subgrupos tomamos, en primer lugar, el único grupo de un elemento: 
  \[ \{1\} \]
  Ahora tomamos los grupos de orden dos, que serán todos aquellos generados por elementos que tienen orden dos:\\
  \[ \gen{sg^{3}}, \gen{sg}, \gen{g^{2}}, \gen{s}, \gen{sg^{2}} \]
  
  Los subgrupos de orden cuatro son los formados por elementos de orden 4 más los obtenidos al combinar el centro con los demás subgrupos de orden 2:
  \[ \{1,g,g^{2},g^{3}\}, \ \gen{g}, \ \{1, g^{2}, s, sg^{2}\} \]
 \end{example}
 

 \begin{proof}
 $1 \dimplies 2$\\
 $\forall a\in G,\ aHa^{-1}$ COMPLETAR (mucho, completar mucho).
 \end{proof}
 
 \subsection{Homomorfismos de grupos}
 \begin{defn}[Homomorfismo]
 Sea f $\appl{f}{(G_1, \cdot )}{(G_2, \ast )}$. $f$ es un homomorfismo de grupos si 
 \[ f(a\cdot b) = f(a)\ast f(b) \]
\end{defn}
 
 \begin{example}
 Si consideramos la aplicación $\appl{\pi}{G}{G/N}$ que lleva los elementos de $G$ a su respectivo conjunto cociente, la función sobreyectiva $\pi$ es un homomorfismo de grupos si $N\lhd G$.
 \end{example}
 
 \begin{props} 
 Si $\appl{f}{G_1}{G_2}$ es un homomorfismo de grupos, entonces
 \begin{enumerate}
 \item $f(e_1) = e_2$ con $e_1$ neutro de $G_1$ y $e_2$ neutro de $G_2$
 \item $f(x^{-1}) = f(x)^{-1}$
 \end{enumerate}
 \end{props}
 
 \begin{proof} 
 1) 
 \begin{align*}
 e_1\cdot e_1 &= e_1 \\
 f(e_1)& =f(e_1\cdot e_1) = f(e_1)\ast f(e_1) \\
 f(e_1) &= e_2 \ast e_2 = e_2
 \end{align*}

 2) \begin{align*}
 f(x\cdot x^{-1}) &= f(e_1) = e_2 \\ 
 f(x)\ast f(x^{-1}) &= e_2 \\
  f(x^{-1}) &= f(x)^{-1}
  \end{align*}
 
 \end{proof}
 
 

\begin{remark} Si $G$ es un grupo finito \[\ord{f(a)}\ |\;\ord{a}\, \forall a \in G_1 \] \end{remark}

\begin{lemma}
La composición de dos homomorfismos de grupos es un homomorfismo de grupos.
\end{lemma}

\begin{defn}[Núcleo\IS de un homomorfismo]
Dado $\appl{f}{G_1}{G_2}$ el núcleo de $f$ se define como 

\[ \ncl{f} = \{ a\in G_1 \tq f(a) = e_2\} \]

con $e_2$ el elemento neutro de $G_2$. 

Además, se cumple que $\ncl f = \{ e_1 \}$ si y sólo si $f$ es inyectiva.
\end{defn}

\begin{theorem}
 Sea $\appl{f}{G_1}{G_2}$ un homomorfismo de grupo. Entonces $\ncl f$ y $\img f$ son subgrupos de $G_1$ y $G_2$ respectivamente, con $\ncl f$ el núcleo de la aplicación y $\img f$ la imagen de la misma.
 \end{theorem}
 
 \begin{proof} Empezamos demostrando que el núcleo es un subgrupo.
 \begin{gather*}
  \ncl f < G_1 \\
 e_1 \in \ncl f \\
 x, y \in \ncl f \implies xy^{-1}\in \ncl f \\
 f(xy^{-1}) = \underbrace{f(x)}_{e_2} \underbrace{f(y)^{-1}}_{e_2^{-1}} = e_2
 \end{gather*}
 
Pasamos ahora a demostrarlo con la imagen.

\begin{gather*}
\img f < G_2 \\
 e_2\in \img f \implies e_2 = f(e_1)\\
 a\in \img f \implies a^{-1}\in \img f\\
 \exists \alpha \in G_1 \tq a=f(\alpha ) \implies a^{-1} = (f(\alpha ))^{-1} = f(\alpha ^{-1})
 \end{gather*}
 
 \end{proof}

\begin{defn}[Epimorfismo]
Sea $\appl{f}{G_1}{G_2}$ un homomorfismo de grupos. Entonces diremos que $f$ es un epimorfismo si $\img f = G_2$.
\end{defn}

\begin{defn}[Monomorfismo]
Sea $\appl{f}{G_1}{G_2}$ un homomorfismo de grupos. Entonces diremos que $f$ es un monomorfismo si $\ncl f = \{ e_1\} $
\end{defn}

\begin{lemma} Sea $\appl{f}{G_1}{G_2}$ un homomorfismo de grupos. entonces se tiene que

\[ \ncl f \lhd G_1 \]
\end{lemma}

\begin{proof} Recordemos la definición de subgrupo normal:
\[ H \lhd G_1 \dimplies \forall a\in G_1\ aHa^{-1}\subset H \]
En este caso, tenemos que
\[ \forall a\in G_1\ a\ncl{f}a^{-1} = \{aha^{-1} \tq h\in \ncl{f} \} \]

Si tomamos $f\left(a\ncl{f}\inv{a}\right)$, vemos que es igual a \[ f(a) f(\ncl{f}) f(\inv{a}) = \alpha e_2 \inv{\alpha} = e_2 \]
\end{proof}

\begin{remark}El único homomorfismo de $\ent/n\ent$  en $\ent$, y en general de un grupo finito a los enteros, es el trivial. Esto se debe
a que dado $a\in\ent/n\ent$, ord(a)=$\alpha$, pero $\ord{f(a)}$ será infinito, salvo que $f(a)=0$.\end{remark}

\begin{defn}[Homomorfismo\IS inyectivo]
 Un homomorfismo $\appl{f}{G_1}{G_2}$  es inyectivo si su núcleo es el 0. Estos homomorfismos se denominan monomorfismos
\end{defn}

\begin{defn}[Homomorfismo\IS sobreyectivo]
 Un homomorfismo $\appl{f}{G_1}{G_2}$  es sobreyectivo si Img(f)=$G_2$. Estos homomorfismos se denominan epimorfismos
\end{defn}
 
  
\begin{lemma} Si $N\lhd G$ entonces $G/N$ tiene estructura de grupo y $\appl{\pi}{G}{G/N}$  (homomorfismo que manda cada
elemento a su grupo cociente)  es un epimorfismo.
\end{lemma}

\begin{lemma} Dado $\appl{f}{G_1}{G_2}$, con $N=\ncl{F}$ y $N \lhd G$, entonces
\[ b\in aN \dimplies  a^{-1}b\in N \dimplies f(a^{-1}b)=e_2 \ \dimplies f(a)^{-1}f(b)=e_2 \ \dimplies f(a)=f(b) \]
\end{lemma}

\begin{defn}[Isomorfismo]
Un homomorfismo de grupos $\appl{f}{G_1}{G_2}$ es un isoformismo si existe un homomorfismo de grupos $\appl{g}{G_2}{G_1}$ tal que $g\circ f = id_{G_1}$ y $f\circ g = id_{G_2}$. 
\end{defn}


\begin{remark} Dados dos grupos $G_1$ y $G_2$, $\appl{f}{G_1}{G_2}$ es un isomorfismo de grupos si y sólo si f es un homomorfismo y es una biyección de conjuntos.
\end{remark}

\begin{lemma}
Sea $\appl{f}{G_1}{G_2}$, entonces:
\begin{enumerate}
\item Si $H_1$ es un subgrupo de $G_1$, $f(H_1)<G_2$
\item Si $H_2 < G_2$, entonces $f^{-1}(H_2)<G_1$
\item Si $H_2 \lhd G_2$, $f^{-1}(H_2)\lhd G_1$
\item Si $f$ es sobreyectiva y $H_1\lhd G_1$, tenemos que $f(H_1)\lhd G_2$
\end{enumerate}
\end{lemma}

\begin{proof}
\paragraph{Propiedad 1}
\begin{gather*}
e_2 \in f(H_1)\\
x, y\in f(H_1) \implies xy^{-1}\in f(H_1)\\
x=f(a);\; y = f(b);\; a, b\in H_1\\
xy^{-1}=f(a)f(b)^{-1} = \underbrace{f(ab^{-1})}_{H_1} \in f(H_1)
\end{gather*}

\paragraph{Propiedad 2}
\begin{gather*}
e_1 \in f^{-1}(H_2)\\
a, b \in f^{-1}(H_2) \implies ab^{-1} \in f^{-1}(H_2)\\
f(a) \in H_2;\; f(b) \in H_2 \implies \underbrace{f(ab^{-1})}_{f(a)f(b)^{-1}}\in H_2
\end{gather*}

\paragraph{Propiedad 3}
$f^{-1}(H_2) = N(\pi \circ f)$. Como $f^{-1}(H_2)$ es el núcleo de h de g entonces $f^{-1}(H_2)\lhd G_1$.
\end{proof}


Sea $\appl{f}{G_1}{G_2}$ un homomorfismo de grupos, y $f$ sobreyectiva (es decir, un \textit{epimorfismo})
vamos a analizar el retículo de subgrupos de $G_2$.\\
Fijado un subgrupo $H_2<G_2$ le asociamos un subgrupo $f(H_2)<G_1$. Asociamos al retículo de subgrupos de $G_2$ subgrupos de $G_1$ que contienen al núcleo $N(f)$.
\begin{theorem}
Es posible identificar el retículo de subgrupos de $G_2$ con los subgrupos de $G_1$ que contienen al núcleo $N(f)$
\end{theorem}

\obs Sea $\appl{f}{G_1}{G_2}$ sobreyectiva. $H$ subgrupo de $G_1$ que contiene al núcleo $N(f)$. Entonces $H = f^{-1}(f(H))$
Como $f(H)$ es un subgrupo de $G_2$ la observación indica que todo subgrupo de $G_1$ que contiene al núcleo es de la forma $f^{-1}(K)$ para algún subgrupo K de $G_2$

\begin{example}
El retículo de subgrupos de $Q/{\{ -1, 1 \} }$ se identifica con los subgrupos de $Q$ que contienen al núcleo $\{ 1, -1\} $. Siendo $Q$ el grupo de cuaterniones.
\end{example}

\end{document}
