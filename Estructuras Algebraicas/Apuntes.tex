\documentclass[a4paper,10pt]{apuntes}
\usepackage{anysize} 
\usepackage{dsfont}
\usepackage{amssymb}
\usepackage{textcomp}
\usepackage{plain}
\marginsize{3cm}{2cm}{2cm}{2cm} 

%opening
\title{Estructuras algebraicas}
\author{Pedro Valero Mejía}

\newenvironment{example}[1][Ejemplo]{\begin{trivlist}
\item[\hskip \labelsep {\bfseries #1}]}{\end{trivlist}}
\newenvironment{notacion}[1][Notación:]{\begin{trivlist}
\item[\hskip \labelsep {\bfseries #1}]}{\end{trivlist}}

\begin{document}

\maketitle

\section{Grupos}
 
 Por definición, sea X un conjunto no vacío, podemos construir el conjunto de pares ordenados X*X=$\{(x,y)/x,y\in X\}$
 Vamos a fijar un conjunto X$\neq \varnothing$  y una función $\varphi$: X*X $\longrightarrow$  X que a cada par (x,y) le 
 asocia un elemento $\varphi$(x,y) $\in$X que expresamos como x*y, siendo '*' cualquier operación.
 \begin{defn}
  Sea S un subconjunto no vacío de G, diremos que S es cerrado por $\varphi$  si la combinación por $\varphi$  de dos elementos de
  S da otro elemento del mismo.
 \end{defn}
 
 Dados x,y,z $\in$ X puede ser interesante el resultado x*y*z pero, por definición, esto no tiene sentido. Sin embargo, si tendrían 
 sentido (x*y)*z ó x*(y*z). Estas operaciones podrían tener, o no, el mismo resultado. Queremos un conjunto con una operación 
 donde no tengamos que preocuparnos por la colocación de los paréntesis. Para ello debemos buscar una operación asociativa.\\
 
 Dado G$\neq\varnothing$  y $\varphi$: G*G $\longrightarrow$  G, diremos que G es un grupo si:
 \begin{enumerate}
  \item {(G,*) es asociativa}
  \item {$\exists$ e $\in$G t.q. $\forall$x$\in$G: x*e=x}
  \item {$\forall$x$\in$G $\exists$ x' $\in$G t.q. x*x'=x'*x=e}
 \end{enumerate}

 \begin{example}
   ($\mathds{Z}$, +); ($\mathds{R}$, +); ($\mathds{R}$-{0}, $\cdot$); ($\mathds{R}$/x$>$0, $\cdot$) son grupos, mientras que 
   ($\mathds{Z}$, $\cdot$); ($\mathds{R}$, $\cdot$) no lo son.
   
   A partir de un conjunto A definimos B(A) como el conjunto de todas las biyecciones de A en sí mismo. Puesto que la composicón 
   de dos biyecciones es otra biyección, la composición es una operación definida sobre B(A). Si tomamos como elemento 'e' la biyección 
   identidad y como x' la función inversa, podemos comprobar que B(A) es un grupo respecto a la composición.
 \end{example}
  
 En todo grupo se cumplen las propiedades de unicidad del elemento neutro y del inverso. La demostración de estas propiedades
 es bastante trivial.
 
 Sean dos elementos e, y e' dos elementos neutros de nuestro grupo G, se cumple que e*e'=e', pero tambien se cumple que e'*e=e. Esto 
 implica que e'=e.

 Por otro lado, si suponemos la existencia de dos elementos inversos $a',a''\in$G, entonces $e=a*a'=a*a''$. Si multiplicamos por $a$ en ambos lados de la ecuación tenemos: $a*(a*a')=(a*a')*a''$, pudiendo reordenar los paréntesis por la 
 propiedad asociativa. Así pues, obtenemos $a'=a''$.
 En esta última demostración nos hemos apoyado en la propiedad cancelativa, que sólo se presenta cuando trabajamos en un grupo.
  
 \begin{defn}
  Transformaciones lineales rígidas son aquellas que conservan las distancias. (En $\mathds{R}^{2}$  sólo están las simetrías y giros).\\
 \end{defn}
 
 \begin{example} 
  Vamos a trabajar con un triángulo equilátero, $\Delta$  en $\mathds{R}^{2}$  y vamos a encontrar el conjunto de todas las aplicaciones lineales rígidas 
  que llevan el triángulo en si mismo. $D_{3}=\{f\in G / f(\Delta)\longrightarrow\Delta\}$.
  Para empezar, dentro de este grupo encontramos todos los giros de águlo 120º. Si defino $a=g_{2\pi/3}$, tenemos las aplicaciones:
  $e$, $a$  y $a*a$, ya que la aplicación $a*a*a=e$, $a*a*a*a=a$  y así sucesivamente, por completar vueltas al círculo unidad.
  Por otro lado, también tenemos las simetrías que tienen como eje las alturas del triángulo. Denotaremos estas simetrías como: $S_{1}$, $S_{2}$  y $S_{3}$.
  Sabemos que la combinación de un giro y una simetría tiene como resultado otra simetría. Si combinamos $a*S_{1}$  obtenemos otra simetría, 
  que también deja el triángulo en si mismo. Así pues, esta simetría, debe tratarse de $S_{2}$  ó $S_{3}$. Lo mismo ocurre con $a*a*S_{1}$.
  
  Así, tenemos: $D_{3}=\{e, a, a*a, S_{1}, a*S_{1}, a^{2}*S_{1}\}$
 \end{example}

 La representación geométrica de un grupo consiste en la descripción de los elementos geométricos que lo constituyen. En el 
 caso del ejemplo, consistiría en indicar qué giros y simetrías constituyen el grupo.
 La representación abstracta o algebraica de un grupo suele realizarse por medio de una tabla, o una serie de restricciones sobre las operaciones
 de combinación de los elementos del grupo, sin necesidad de indicar qué es realmente cada elemento.
 \begin{example}
  La representación abstracta de $D_{3}$  viene dada por tres condiciones:
 \begin{itemize}
  \item ord(g)=3
  \item ord(s)=2
  \item $g*s=s*g^{2}$
 \end{itemize}
 Ya que con estas condiciones podríamos construir una tabla con todas las combinaciones 2 a 2 de elementos del grupo sin necesidad
 de saber nada acerca de esos elementos.
 \end{example}
 
 \begin{defn}
  Se dice que un elemento $a\in G$, siendo G un grupo, tiene orden finito si $\exists k\in\mathds{N}$  t.q  $a^{k}=e$
 \end{defn}
 
 \begin{defn}
  Dado un elemento de orden finitio, decimos que su orden es el menor entero positivo con el que se cumple $a^{k}=e$
 \end{defn}
\subsection{Subgrupos}
 Sea G un conjunto y $\varphi$  la operación con la que forma un grupo, vamos a ver cuando un subconjunto de G es un grupo de forma natural,
 esto es lo que denominaremos subgrupo.
 
 \begin{defn}
  Diremos que un subconjunto no vacío S es un subgrupo si:
  \begin{enumerate}
   \item S es cerrado por la operación
   \item e $\in$S
   \item s $\in$S$\Rightarrow$ $s^{-1}\in$S.
  \end{enumerate}
 \end{defn}
  \begin{theorem}
   Dados $S_{1}$, $S_{2}$ subgrupos de G $\Rightarrow$  $S_{1}\cap S_{2}$  es un subgrupo de G.
  \end{theorem}

  Este teorema también puede aplicarse con una intersección numerable de grupos.
  
  \begin{defn}
   Fijado un elemento g $\in$G, definimos el grupo generado por g como:\\
   $<g>$=$\{$..., $g^{-2}$, $g^{-1}$, e, g, $g^{2}$,...$\}$. Este grupo es un subgrupo de G
  \end{defn}

  \begin{theorem}
   Si H es un subgrupo de G y g$\in$H $\Rightarrow$ $<g>\subset$H.   
  \end{theorem}

  A partir de un conjunto C=$\{g_{1}$, $g_{2}$, $g_{3}$... $g_{s}\}$, contenido en G, vamos a buscar el menor subgrupo que lo contiene.
  $<g_{1}$, $g_{2}$, $g_{3}$... $g_{s}>$=$\displaystyle \cup_{k\in\mathds{N}}$  $\{$ $a_{1}$* $a_{2}$* $a_{3}$... $a_{k}$/($a_{i}\in$C)$\vee$($a_{i}\in C^{-1}$)$\}$. Es un subgrupo que contiene a 
  todos los elementos de C.
  
  Si H es un subgrupo que contiene a los elementos de C, el grupo generado por esos elementos se contiene en H.
  
  \begin{notacion}
   H$<$G denota H subgrupo de G
  \end{notacion}
  \begin{defn}
   Dado H$<$G se dice que H es cícilico si existe g$\in$G t.q. H=$<g>$ 
  \end{defn}

  \begin{theorem}
   Si G es un grupo finito y S$\subset$G es un subconjunto no vacío $\Rightarrow$ S es un subgrupo $\Leftrightarrow$ S es cerrado por la operación.   
  \end{theorem}
  \begin{proof}
   La implicación hacia la derecha es obvia por la propia definición de subgrupo.
   Para la implicación hacia la izquierda partimos de que S es cerrado y finito. Por tanto $\exists$ d $\in$ S t.q. ord(d)=n y $<d>\subset S$
   $\Rightarrow$ $d^{n}=e\in S$ y $d^{n-1}=d^{-1}\in S$ $\Rightarrow$ S es un grupo
  \end{proof}

  Dentro de los grupos podríamos realizar una claisficación según fueran finitos o infinitos, por ejemplo. No obstante, nos resultará
  más interesante la clasificación de grupos según sean abelianos o no.
  
  \begin{defn}
   Un grupo es abeliano si cumple la propiedad conmutativa.
  \end{defn}

  \begin{example}
   ($\mathds{Z}$, +); ($\mathds{Z}$/$n\mathds{Z}$, +) y $<s,g^{2}>$=$\{1,g^{2},s,sg^{2}\}$ son abelianos
  \end{example}
  
  \begin{lemma}
   Todo subgrupo cíclico de un grupo G es abeliano. Por tanto un grupo no abeliano no puede ser cíclico.
  \end{lemma}

  \begin{example}
   Dado $D_{4}=\{id, g, g^{2}, g^{3}, s, sg, sg^{2},sg^{3}\}$, (Recordemos que era el conjunto de aplicaciones que mantenían un
   cuadrado invariante), vamos a ver los grupos cíclicos contenidos en él.
   $<1>={1}$, $<g>$=$\{1,g,g^{2}, g^{3}\}$, $<g^{2}>$=$\{1,g^{2}\}$, $<g^{3}>=\{1,g, g^{3}\}$, $<s>$=$\{1,s\}$, $<sg>$=$\{1, sg\}$, $<sg^{2}>$=$\{1, sg^{2}\}$, $<sg>$=$\{1, sg^{3}\}$.
   
   Además podemos destacar el caso de ($\mathds{Z}$, +), un grupo cíclico para el cual, todo subgrupo es también cíclico. Esto se demuestra
   de forma general considerando que un grupo no cíclico estaría generado por varios elementos. En este caso, el máximo común
   divisor de estos número sería generador del grupo. Por tanto, el grupo sería cíclico.
 
   De forma más estricta podemos decir que dado un subgrupo H$<\mathds{Z}$ podrá ser H=$\{0\}$ ó H$\neq\{0\}$. 
   En el primer caso, H ya sería cíclico. En el segundo caso, tenemos que $\exists$  d $\in$  H t.q. d$\neq$0. Lo que implica
   que $<d>\subset H$. 
   
   La duda sería si es cierto o no H$\subset <d>$. Sea h $\in$H  $\Rightarrow$  h=qd+r. Puesto que tanto h como d pertenecen a H,
   tenemos que r pertenece a H tambien. Esto implica que r puede expresarse r=dp. Lo que conlleva h=qd+pd=(q+p)d. Por tanto h $\in <d>$
  \end{example}
  \begin{theorem}
   En un grupo finito G todo elemento tiene orden finito. Además si g$\in$G tiene ord(g)=k $\Rightarrow$  $<g>$  tiene k elementos.
  \end{theorem}

  \begin{defn}
   Un retículo de subgrupos es aquel retículo (estructura algebraica parcialmente ordenada) formado por subgrupos de un determinado grupo
   con una relación de contención. En este retículo, la unión de dos subgrupos es el subconjunto generado por su conexión.
  \end{defn}

  \begin{example}
   Tomamos una vez más el grupo $D_{4}=\{1,g,g^{2}, g^{3}, s, sg, sg^{2}, sg^{3}\}$. Aquí obtenemos el retículo:
  % \easyimgw{Reticulo.png}{Leyenda}{lblEtiqueta}{0.3}
  \end{example}

  \begin{theorem}[Teorema de Lagrange]
   Si G es un grupo finito y H un subgrupo de G, entonces el número de elementos de H divide el número de elementos de G.
  \end{theorem}
  \begin{lemma}
   Sea $\varphi: G\rightarrow G$, son equivalentes:
   \begin{enumerate}
    \item $\varphi$  es inyectiva.
    \item $\varphi$  es biyectiva.
    \item $\varphi$  es sobreyectiva.
   \end{enumerate}
  \end{lemma}
  La demostración de este teorema es totalmente trivial. Con apoyarnos en que el conjunto G es finito, puede observarse que una de
  esas propiedades implica directamente las demas.
  
  \begin{lemma}
   Si $\varphi$  es un biyección y A,B son subconjuntos de G:
   \begin{enumerate}
    \item card(A)=card($\varphi(A)$)
    \item A=B $\Leftrightarrow$  $\varphi(A)=\varphi(B)$
    \item $\varphi(A\cap B)=\varphi(A)\cap\varphi(B)$
   \end{enumerate}
  \end{lemma}
  
  \begin{lemma}
   Sea G un grupo finito y g $\in$G. $\varphi_{g}(x)=g*x$. Además, por la propiedad cancelativa, podemos ver que $\varphi_{g}$ es inyectiva.
   Además, por ser G finito, sabemos que $\varphi_{g}$  es biyectiva.
  \end{lemma}
  También podemos definir $\varphi_{g}$=$\{g*h\tq h\in H\}$
  \begin{lemma}
   $H<G\Rightarrow$  H es subconjunto de G.
  \end{lemma}
  
  \begin{notacion}
   gH=$\varphi_{g}(H)$
  \end{notacion}

  \begin{corol} 
  Extraemos las siguientes conclusiones:
   \begin{enumerate}
    \item g$\in$  gH
    \item card(H)=card(gH)
    \item H=gH $\Leftrightarrow$  g $\in$H
   \end{enumerate}
  \end{corol}

  \begin{proof}
   \begin{enumerate}
    \item e $\in$  H. Por tanto g*e$\in$gH. g*e=g$\in$ gH
    \item Se puede ver apoyándonos en los lemas anteriores puesto que H es finito.
    \item $\Rightarrow$:  e $\in$H $\Rightarrow$ e=gh con h $\in$ H $\Rightarrow$ h=$g^{-1}\in$H $\Rightarrow$ g $\in$H.\\
	  $\Leftarrow$:  g $\in$H $\Rightarrow$ $g^{-1}\in$H y todo elemento h $\in$H cumple h=$g(g^{-1}h)\in$H.
   \end{enumerate}
  \end{proof}
  
  \begin{prop}
   Dados $g_{1}$,$g_{2}\in$G y $H<G\Rightarrow g_{1}H=g_{2}H \vee g_{1}H\cap g_{2}H=\varnothing$
  \end{prop}
  \begin{proof}
   $g_{1}H\cap g_{2}H \neq \varnothing \Leftrightarrow$ $\exists h_{1},h_{2}$  t.q. $g_{1}h_{1}=g_{2}h_{2} \Rightarrow h_{1}=g_{1}^{-1}g_{2}h_{2} \Rightarrow h_{2}^{-1}h_{1}=g_{1}^{-1}g_{2}\in$H
   $\Leftrightarrow$ $g_{1}^{-1}g_{2}H=H \Rightarrow g_{2}H=g_{1}H$
  \end{proof}
  \begin{example}
   Tomando el famoso grupo $D_{4}$, podemos obtener H=$\{1,g,g²,g³\}$; sH=$\{s,sg,sg²,sg³\}$
  \end{example}
  \subsection{Particiones}
  \begin{defn}[Relación de equivalencia]
   Fijado un conjunto G$\neq\varnothing$. Una relación R en G es de equivalencia si:
   \begin{enumerate}
    \item $\forall x\in G$  xRx
    \item $\forall x,y \in G$  xRy $\Leftrightarrow$  yRx
    \item $\forall x,y,z \in G$  xRy $\wedge$  yRz $\Rightarrow$  xRz
   \end{enumerate}
  \end{defn}
  \begin{defn}[Partición]
   Familia de subconjuntos disjuntos dos a dos tales que su unión constituye el total.
  \end{defn}
  
  Una partición define una relación de equivalencia y viceversa. Si R es una relación de equivalencia en un grupo G, 
  definimos una partición en la que los subconjuntos son de la forma: $S_{x}=\{y\in G \tq xRy\}$
  
  \begin{proof}[Volvamos a demostrar la proposición 1.22]
   \\Definimos una relación de equivalencia R en G a partir del grupo H.
   $g_{1}Rg_{2} \Leftrightarrow g_{1}^{-1}g_{2}\in H$  y comprobamos que, efectivamente, esta relación es de equivalencia.
   Tenemos $S_{e}=H$. Así $S_{g}$  o ya cubre junto con H todo G, o cojo otro $g^{'}$  y repito el proceso con $S_{g^{'}}$.
   Así formare una serie de grupos de la forma $S_{x}$  disjuntos dos a dos y cuya unión me da G.\\
   De esta forma podemos ver que card(G)=card(H)*s
  \end{proof}
  
  \begin{corol}[Teorema de Lagrange]
   Dado g $\in$ G, siendo G un grupo finito $\Rightarrow$  ord(g) divide a |G| (cardinal de G). Así, suponiendo |G|=p primo,
   los únicos subrgupos que tiene G son los triviales: $<1>, G$
  \end{corol}
  \begin{theorem}
   Si |G|=p primo $\Rightarrow$  G es cíclico. Además, si |G|=n, $\forall$g ord(g) divide a n $\Rightarrow$  $g^{n}=e$
  \end{theorem}
  \begin{example}
   Con ayuda de este teorema puede demostrarse el pequeño teorema de Fermat. 
   Definimos $\mathds{Z}/p\mathds{Z}=\{\bar{0},\bar{1}...,\bar{p-1}\}$.
   Si tomamos el grupo de las unidades de $\mathds{Z}/p\mathds{Z}$ tenemos $\{\bar{1}, \bar{2},....\bar{p-1}\}$ $\Rightarrow$
   $\bar{a}\in\mathds{Z}/p\mathds{Z} \wedge \bar{a}\neq 0 \Rightarrow a^{p-1}=\bar{1}$.
  \end{example}
  \begin{theorem}
   Si |G|=$p^{2}$  con p primo $\Rightarrow$  $\exists g \in G \tq ord(g)=p$
  \end{theorem}
  \begin{proof}
   Tomamos g $\in G \wedge g\neq e \Rightarrow ord(g)=p$  ó $ord(g)=p^{2}$. En el primer caso ya lo tenemos, vamos a por el segundo.
   ord(g)=$p^{2}\Rightarrow ord(g^{p})=p$
  \end{proof}
  \begin{example}
   Por todo lo explicado anteriormente, si tomamos el anillo de polinomios $\mathds{Z}/p\mathds{Z}[x]$, tenemos $X^{p-1}-\bar{1}=
   \prod_{\bar{a}\in\mathds{Z}/p\mathds{Z} \wedge \bar{a}\neq 0}(x-\bar{a})$
  \end{example}
  
  \begin{theorem}[Teorema de Lagrange Bis]
   Dados $H<G$  $\exists a_{1},...,a_{r}\in G\tq$  G=$a_{1}H \cup a_{2}H.... \cup a_{r}H$  $\wedge a_{i}H\cup_{j}H=\varnothing \forall i,j$.
   Es decir, |G|=r|H|
  \end{theorem}

  Definimos ahora otra relación a partir de H$<$G: $cR^{'}d \Leftrightarrow cd^{-1}\in H$. La comprobación de que esto es una relación
  la omitiremos por ser trivial. En esta partción, el conjunto de elementos relacionados con d es: Hd=$\{hd/h\in H\}$
  \begin{defn}
   Decimos que $H<G$  es un subgrupo normal si aH=Ha $\forall$a. Por tanto, si G es un grupo abeliano, H también lo será y cualquier
   subgrupo será normal. Un subgrupo normal se expresa como $H\lhd G$
  \end{defn}
  \begin{example}
   Tomamos G=$\mathds{Z}$  y H=$4\mathds{Z}$. 
   El subgrupo de los elementos que son equivalentes a n es nH=$\{n+4k\tq k \in \mathds{Z}\}$.
   En este caso tenemos 4 subgrupos según esta condición: $\bar{0},\bar{1},\bar{2},\bar{3}$
  \end{example}
  
  Si $H\lhd G$  podemos definir una estructura de grupo en el subconjunto de clases.
  Dados $g_{1}H$  y $g_{2}H$  podemos formar el conjunto $\{h_{1}*h_{2}\tq h_{1}\in g_{1}H \wedge h_{2}\in g_{2}H\}$. Si operamos
  $g_{1}H*g_{2}H=(g_{1}*g_{2})H$  que es otra clase. Por tanto el grupo de clases es cerrado.
  El neutro sería la caja H puesto que gHH=gH para cualquier g que escojamos. Además el inverso de gH es $g^{-1}H$.
  Por último, vemos que $(g_{1}H*g_{2}H)*g_{3}H=g_{1}H*Hg_{2}*g_{3}H=g_{1}H*Hg_{3}*g_{2}H$, por ser G asociativo. Queda probado pues, 
  que el conjunto de las clases de equivalencia, forma un grupo.
  
  \begin{theorem}
   Si definimos [G:H]=nº de cajas=índice, tenemos:
   H$<$G $\wedge[G:H]=2 \Rightarrow H \lhd G$.
  \end{theorem}
  \begin{proof}
   Por un lado tenemos que g$\notin$H $\Rightarrow$  gH= $H^{c}$.
   Por otro lado, si tomamos la otra relación de equivalencia, a partir de g$\notin$H obtenemos Hg=$H^{c}$.
   Uniendo estos dos ejemplos, donde mantenemos H como una mitad en ambos, tenemos Hg=gH. Por tanto H es un subgrupo normal
  \end{proof}

  \newpage
  
  \begin{lemma}
   Dadas dos particiones formadas por los subconjuntos $aH$ y $Ha$ para un elemento $a \in G$ y con $H <G$, podemos definir dos aplicaciones $\pi$ y $\pi '$ que lleven cada elemento de G a su respectiva caja:
   $$\appl{\pi}{G}{G/H}$$
   $$\appl{\pi'}{G}{G/H}$$
   Tenemos entonces que:
   $$H \text{subgrupo normal} \implies \text{la partición izquierda coincide con la derecha}\ (a_iH = Ha_i)$$
   $$H \text{subgrupo normal} \implies \pi \ =\ \pi '$$

   
   Además, si H es normal
   \begin{enumerate}
	\item Podemos definir una operación en el conjunto cociente.     
	\item $\appl{\pi}{G}{G/H}$ es compatible con las operaciones.
   \end{enumerate}
  \end{lemma}
  
  \begin{example}
	Sea $Q$ el grupo formado por las raíces cuartas de 1.\\
	$$Q = \{ 1, i, j, k, -1, -i, -j, -k\} $$\\
	Podemos ver que $Q$ se genera a partir de los elementos $i, j, k$.\\
	$$Q = <i,j,k>$$\\
	Además vemos que $-1$ conmuta con todos los elementos del grupo: $-1 \in Z(G)$
	Analizamos el retículo del grupo y vemos que esta formado por los subgrupos:
	\\
	$Orden\ 1 = \{ 1\} $\\
	$Orden\ 2 = \{ 1, -1\} $\\
	$Orden\ 4 = \{ 1, i, i^2, i^3\},\
			  \{ 1, j, j^2, j^3 \},\
			  \{ 1, k, k^2, k^3 \}.$\\
		  
	El grupo no es abeliano, sin embargo, todos sus subgrupos son normales porque todo subgrupo $H<G$ de indice 2 (es decir, el conjunto de clases $G/H$ tiene dos elementos) es normal.
	
  \end{example}
  
  \begin{defn}
	Llamamos centro del grupo $G$ al grupo formado por todos los elementos que conmutan con todos los elementos de $G$.:\\
	$$Z(G) = \{ a\in G\tq a\ast g = g\ast a\ \forall g\in G\} $$
	Además, tenemos que:
	$aZ(G)=Z(G)a \implies Z(G)\lhd G$
  \end{defn}
 
 \newpage
 
 Dados un grupo G y un subconjunto no vacío S, defino $S^{-1}=\{s^{-1}\tq s\in S\}$.\\
 Dados $S_{1},S_{2}<G$, defino $(S_{1}*S_{2})^{-1}=\{s_{1}*s_{2}\tq s_{1}\in S_{1} \wedge s_{2}\in S_{2}\}=\{s_{1}^{-1}*s_{2}^{-1}\tq s_{1}\in S_{1} \wedge s_{2}\in S_{2}\}=
 \{s_{1}^{-1}*s_{2}^{-1}\tq s_{1}^{-1}\in S_{1}^{-1} \wedge s_{2}^{-1}\in S_{2}^{-1}\}=S_{2}^{-1}*S_{1}^{-1}$.\\
 Sea H un subgrupo de G, entonces $H^{-1}=H$.
 
 \begin{theorem}
  Fijado un grupo G y $H_{1}, H_{2}$  subgrupos de G entonces: $H_{1} \vartriangleleft G$ $\Rightarrow$ $H_{1}*H_{2}<G$. 
 \end{theorem}
 \begin{proof}
  Sabemos que $H_{1}*H_{2}=H_{2}*H_{1}$. Para ver que $H_{1}*H_{2}<G$  hay que:
  \begin{enumerate}
   \item Cerrado por la operación: $(H_{1}*H_{2})*(H_{2}*H_{1})=H_{1}*(H_{2}*H_{1})*H_{1})=H_{1}*H_{1}*H_{2}*H_{2}=H_{1}*H_{2}$
   \item Todo elemento, tiene inverso, es decir: $\alpha\in H_{1}*H_{2}\Rightarrow \alpha^{-1}\in H_{1}*H_{2}$. Pero entonces tenemos:
   $\alpha^{-1}\in (H_{1}*H_{2})^{-1} = H_{2}^{-1}*H_{1}^{-1}=H_{2}*H_{1}=H_{1}*H_{2}\supset\alpha$
  \end{enumerate}

 \end{proof}

 \begin{example}
  Tomamos el ya famoso grupo $D_{4}=\{1,g,g^{2},g^{3}, s, sg, sg^{2}, sg^{3}\}=<s,g>$\\
  Dentro de este grupo, el elemento 1 tiene orden dos, los elementos $g$, $g^{3}$  tienen orden 4 y el resto, 2.\\
  Z($D_{4}$)=$\{\alpha\in D_{4} \tq \alpha\beta=\beta\alpha \ \forall \beta \in D_{4}\}$=$\{\alpha\in D_{4}\tq \alpha s = s\alpha \wedge \alpha g = g\alpha\}=\{1, g^{2}\}=<g^{2}>$. Este subgrupo es normal.
  
  Para contruir el retículo de subgrupos tomamos, en primer lugar, el único grupo de un elemento: \\
  $$\{1\}$$\\
  Ahora tomamos los grupos de orden dos, que serán todos aquellos generados por elementos que tienen orden dos:\\
  $$<sg^{3}>, \ <sg>, \ <g^{2}>, \ <s>, \ <sg^{2}>$$\\
  Los subgrupos de orden cuatro son los formados por elementos de orden 4 más los obtenidos al combinar el centro con los demás subgrupos de orden 2:\\
  $$\{1,g,g^{2},g^{3}\}, \ <g>, \ \{1, g^{2}, s, sg^{2}\}$$
 \end{example}
 
 
 \begin{theorem}
 H es un subgrupo normal si $H<G$\\
 \begin{enumerate}
 \item $\forall a \in G,\ aH = Ha,\ aHa^{-1}=H$
 \item $\forall a \in G,\ aHa^{-1} \subset H$
 \end{enumerate}
 \end{theorem}
 \begin{proof}
 $1 \iff 2$\\
 $\forall a\in G,\ aHa^{-1} COMPLETAR$
 \end{proof}
 
 \begin{defn}
 Sea f $\appl{f}{(G_1, \cdot )}{(G_2, \ast )}$\\
 La función f es un homomorfismo de grupos si 
 $f(a\cdot b) = f(a)\ast f(b)$
 \end{defn}
 
 \begin{example}
 Si consideramos la aplicación $$\appl{\pi}{G}{G/N}$$ que lleva
 los elementos de $G$ a su respectiva caja, la función sobreyectiva $\pi$ es 
 un homomorfismo de grupos si $N\lhd G$.
 \end{example}
 
 \begin{props} 
 Si $\appl{f}{G_1}{G_2}$ es un homomorfismo de grupos.
 \begin{enumerate}
 \item $f(e_1) = e_2$ con $e_1$ neutro de $G_1$ y $e_2$ neutro de $G_2$
 \item $f(x^{-1}) = f(x)^{-1}$
 \end{enumerate}
 \end{props}
 
 \begin{proof} \\
 1) $e_1\cdot e_1 = e_1$\\ 
 $f(e_1)=f(e_1\cdot e_1) = f(e_1)\ast f(e_1) \implies f(e_1) = e_2$\\
 \\
 2) $f(x\cdot x^{-1}) = f(e_1) = e_2$\\ 
 $f(x)\ast f(x^{-1}) = e_2 \implies f(x^{-1}) = f(x)^{-1}$
 
 \end{proof}
 
 \begin{theorem}
 $N(f)$ y $Img(f)$ son subgrupos de $G_1$ y $G_2$ respectivamente.
 Con $N(f)$ el núcleo de la aplicación y $Img(f)$ la imagen de la misma.
 \end{theorem}
 \begin{proof}\\
 1) $N(f) < G$\\
 $e_1 \in N(f)$\\ $x, y \in N(f) \implies xy^{-1}\in N(f)$\\
 $f(xy^{-1}) = \underbrace{f(x)}_{e_2} \underbrace{f(y)^{-1}}_{e_2^{-1}} = e_2$\\
 2) $Img(f) < G_2$\\
 $i)\ e_2\in Img(f) \implies e_2 = f(e_1)$\\
 $ii)\ a\in Img(f) \implies a^{-1}\in Img(f)$\\
 $\exists \alpha \in G_1 \tq a=f(\alpha ) \implies a^{-1} = (f(\alpha ))^{-1} = f(\alpha ^{-1})$
 \end{proof}

\obs $\underbrace{f(a) \ast f(a) \ast \hdots \ast f(a)}_{t\ veces} = f(\underbrace{a \cdot a \cdot \hdots \cdot a}_{t\ veces}) = f(e_1) = e_2$\\ Si $G$ es un grupo finito $\implies\ ord(f(a))\ |\ ord(a)\ \forall a \in G_1$

\begin{example} Homomorfismo de grupos.\\
\begin{gather*}
\appl{h}{\mathbb{Z}}{G}\\
-2 \longrightarrow \alpha ^{-2}\\
-1 \longrightarrow \alpha ^{-1}\\
0 \longrightarrow e\\
1 \longrightarrow \alpha\\
2 \longrightarrow \alpha ^{2}\\
\end{gather*}
\end{example}

\obs El único homomorfismo de grupos de $\mathbb{Z}/n\mathbb{Z}$ en $\mathbb{Z}$ es el trivial.\\

\begin{defn}
Dado $appl{f}{G_1}{G_2}$ el núcleo de $f$ se define como\\
$$N(f) = \{ a\in G_1 \tq f(a) = \underbrace{e_2}_{\text{Núcleo de } G_2}\} $$
Además $N(f) = \{ e_1 \} \iff$ f es inyectiva.
\end{defn}

\begin{defn}
Sea $\appl{f}{G_1}{G_2}$ un homomorfismo de grupos\\
Diremos que $f$ es un epimorfismo si $Img(f) = G_2$\\
Diremos que $f$ es un monomorfismo si $N(f) = \{ e_1\} $
\end{defn}

\begin{lemma} Cuando $f$ es un homomorfismo de grupos

$$N(f) \lhd G_1$$\\
\end{lemma}

\begin{proof}
$\forall a\in G_1\ aHa^{-1}\subset H$\\
$\forall a\in G_1\ aN(f)a^{-1} = \{aha^{-1} \tq h\in N(f) \}\subset N(f)$\\
\end{proof}


$\underbrace{f(a) \ast f(a) \ast \hdots \ast f(a)}_{t\ veces} = f(\underbrace{a \cdot a \cdot \hdots \cdot a}_{t\ veces}) = f(e_1) = e_2$\\ 
Si $G$ es un grupo finito $\implies\ ord(f(a))\ |\ ord(a)\ \forall a \in G_1$\\

El único homomorfismo de $\mathds{Z}/n\mathds{Z}$  en $\mathds{Z}$, y en general de un grupo finito a los enteros, es el trivial. Esto se debe
a que dado $a\in\mathds{Z}/n\mathds{Z}$, ord(a)=$\alpha$, pero ord(f(a)) será infinito, salvo que f(a)=0.

\begin{defn}
 Un homomorfismo $\appl{f}{G_1}{G_2}$  es inyectivo si su núcleo es el 0. Estos homomorfismos se denominan monomorfismos
\end{defn}
\begin{defn}
 Un homomorfismo $\appl{f}{G_1}{G_2}$  es sobreyectivo si Img(f)=$G_2$. Estos homomorfismos se denominan epimorfismos
\end{defn}
 
 \begin{lemma}
  $\appl{f}{G_1}{G_2}$  $\Rightarrow$  N(f)$\vartriangleleft G_1$ .
 \end{lemma}
  \begin{proof}
   Basta con probar que $aHa^{-1}\subset H$, que equivale a decir $aHa^{-1}=H$.\\
   Esto es: $\forall a \in G_{1} aN(f)a^{-1}\subset N(f)$.\\
   Tomamos $f(aN(f)a^{-1})=f(a)f(N(f))f(a^{-1}=\alpha e_{2} \alpha^{-1}=e_{2}$\\
  \end{proof}
  
Si N$\vartriangleleft$G, entonces G/N tiene estructura de grupo y $\appl{\pi}{G}{G/N}$  (homomorfismo que manda cada
elemento a su 'caja'  es un epimorfismo.\\

Dado $\appl{f}{G_1}{G_2}$  N=N(f), N$\vartriangleleft$G, entonces:\\
b$\in aN \Leftrightarrow a^{-1}b\in N \ \Leftrightarrow f(a^{-1}b)=e_2 \ \Leftrightarrow f(a)^{-1}f(b)=e_2 \ \Leftrightarrow f(a)=f(b)$
\end{document}
