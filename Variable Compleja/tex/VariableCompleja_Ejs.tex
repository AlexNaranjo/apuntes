% -*- root: ../VariableCompleja.tex -*-

\section{Hoja 1}
%
\begin{problem}[1]
Realices las operaciones indicadas
\ppart
\[\frac{1}{i}+\frac{1}{1+i}\]
\ppart
\[\frac{2}{(1-3i)^2}\]
\ppart
\[(1+i\sqrt{3})^3\]
\ppart
\[(\overline{1-i})^2+\overline{2+i}\]

\solution

\spart
\[\frac{1}{i}+\frac{1}{1+i} = \frac{1+i+i}{i-1} = -\frac{2i}{1-i} = \frac{2i(1+i)}{(1-i)(1+i)} = \frac{2i-2}{1+1}=i-1\]
\spart
\[\frac{2}{(1-3i)^2} = \frac{2}{1-9-6i} = \frac{2}{-8-6i}=\frac{2(-8+6i)}{(-8-6i)(-8+6i)} = \frac{2(-8+6i)}{64+36} = \frac{-4+3i}{25}\]
\spart
\[(1+i\sqrt{3})^3\ = 1 +3i\sqrt{3}-3\cdot 3-i3^{\frac{3}{2}} = -8 +i (3\sqrt{3}-3\sqrt{3}) = -8\]
\spart
\[(\overline{1-i})^2+\overline{2+i} = \overline{1-1-2i}+2-i = 2i+2-i=2+i\]
\end{problem}

\begin{problem}[2]
Calcule los valores de
\ppart
\[\sum_{k=1}^{2015}i^k\]
\ppart
\[(1+i)^n+(1-i)^n\]
\ppart
\[\left( \cos \left( \frac{\pi}{12} \right) + i \sin \left( \frac{\pi}{12} \right)\right)^{20}\]
\ppart
\[\left(\frac{1+i}{1-i}\right)^{2014}\]
\solution

\spart
Por tratarse de una sucesión geométrica de razón $i$ sabemos que:
\[\sum_{k=1}^{2015}i^k = \frac{1-i^{2016}}{1-i} -1\]

\spart
Primero debemos observar que
\large
\[(1+i) = 2^{\frac{1}{2}}e^{\frac{\pi}{4}i}\]
\normalsize
por tanto
\[ (1+i)^n= 2^{\frac{n}{2}}e^{\frac{\pi}{4}in} = 2^{\frac{n}{2}}\left(\cos\left(\frac{\pi}{4}n\right)+i \sin\left(\frac{\pi}{4}n\right)\right)\]

Teniendo en cuenta esta relación, podemos resolver el ejercicio:
\[(1+i)^n+(1-i)^n = (1+i)^n+\overline{(1+i)^n} = 2 Re((1+i)^n)=2^{\frac{n}{2}+1}\cos\left(\frac{\pi}{4}n\right)\]

\spart
\[\left( \cos \left( \frac{\pi}{12} \right) + i \sin \left( \frac{\pi}{12} \right)\right)^{20}=\left(e^{\frac{\pi}{12}}\right)^20 = e^{\frac{\pi}{12}\cdot 20} = \cos \left( \frac{\pi\cdot 20}{12} \right) + i \sin \left( \frac{\pi\cdot 20}{12} \right) =\]
\[=\cos \left( \frac{\pi\cdot 5}{3} \right) + i \sin \left( \frac{\pi \cdot 5}{3} \right)\]

\spart
Primero vamos a trabajar con el interior del paréntesis para convertirlo en un número complejo en su expresión habitual, sin fracciones.
\[\frac{1+i}{1-i}=\frac{(1+i)^2}{(1-i)(1+i)} = \frac{1-1+2i}{1+1} = i\]
y puesto que el exponente es par, tenemos que
\[\left(\frac{1+i}{1-i}\right)^{2014}=i^{2014}=1\]
\end{problem}

\begin{problem}[3]
Sea $z=x+iy \in \cplex$. Demuestre que $|x|+|y|\leq \sqrt{2}|z|$, y que sólo hay igualdad si $|x|=|y|$.

\textbf{Ayuda:} Si $a,b \in \real$, entonces $2ab \leq a^2 + b^2$ (con igualdad sólo si $a=b$)

\solution

Si calculamos el módulo de z vemos que
\[|z|=\sqrt{x^2+y^2}\]
si $|x|=|y|$ fácilmente vemos que
\[|z|=\sqrt{x^2+x^2}=\sqrt{2x^2}=\sqrt{2}|x| \implies \sqrt{2}|z|=2|x|=|x|+|y|\]

Veamos ahora el caso en que no son iguales. En esta ocasión, nos apoyamos en al ayuda del enunciado y vemos que
\[|z|=\sqrt{x^2+y^2} \geq \sqrt{2xy} \iff \sqrt{2}|z| \geq 2\sqrt{xy}\]
%TODO terminar esto

\end{problem}

\begin{problem}[4]
Compruebe la identidad
\[|z\bar{w}+1|^2+|z-w|^2 = (1+|z|^2)(1+|w|^2)\]
donde $z,w \in \cplex$

\solution

Llamando a $z=a+bi$ y $w=c+di$ tenemos
\[|(a+bi)(c-di)+1|^2+|a+bi+c-di|^2=|ac-adi+bci+bd+1|^2+|a+bi-c+di|^2 = \]
\[=(ac+bd+1)^2+(bc-ad)^2+(a-c)^2+(b+d)^2 =\]
\[= 1 + a²c²+b²d²+2acbd + 2ac+2bd+b²c²+a²d²-2bcad+a^2+c²-2ac+b²+d²-2bd =\]
\[= 1+a²c²+b²d²+b²c²+a²d²+a²+b²+c²+d² = (1+a²+b²)(1+c²+d^2)\]

\end{problem}

\begin{problem}[5]
Demuestra las siguientes afirmaciones
\ppart
\[\text{Si } |z|=1, \text{ entonces para todos } a,b \in \cplex \text{ con } \bar{b}z+\bar{a} \neq 0\text{ se cumple } \left| \frac{az+b}{\bar{b}z+\bar{a}}\right| = 1\]

\ppart
\[\text{Si } |a| < 1, \text{ entonces } |z| <1 \text{ es equivalente a } \left| \frac{z-a}{1-\bar{a}z}\right|<1\]

\solution

\spart
%Vamos a simplificar el complejo dado
%\[\frac{az+b}{\bar{b}z+\bar{a}}= \frac{(az+b)\bar{z}}{(\bar{b}z+\bar{a})\bar{z}}=\frac{a|z|^2+b\bar{z}}{\bar{b}|z|^2+\bar{a}\bar{z}} = \frac{a+b\bar{z}}{\bar{b}+\bar{z}\bar{a}}\]

%Llegados a este punto, podemos observar que $\overline{\bar{b}+\bar{z}\bar{a}}=b+za$
%\[\frac{az+b}{\bar{b}z+\bar{a}} = \frac{(a_1+ia_2)(z_1+iz_2)+b_1+ib_2}{}\]

\spart

\[\left| \frac{z-a}{1-\bar{a}z}\right|<1 \iff |z-a| < |1-\bar{a}z| \iff \underbrace{|z-a|^2}_{(z-a)(\bar{z}-\bar{a})} < \underbrace{|1-\bar{a}z|^2}_{(1-\bar{a}z)(1-a\bar{z})} \]

Por lo que nos queda que debe cumplirse
\[|z|^2-a\bar{z}-\bar{a}z+|a|^2 < 1-\bar{a}z+a\bar{z}+|a|^2|b|^2 \iff |z|^2+|a|^2-2\cdot Re(z\bar{a}) < 1 + |a|^2|z|^2-2\cdot Re(z\bar{a}) \iff\]

\[\iff |z|^2 + |a|^2 < 1 +|a|^2|z|^2 \iff |z|^2(1-|a|^2) < 1 - |a|^2 \iff |z|^2 < 1 \iff |z| < 1\]

\end{problem}

\begin{problem}[6]
Usando la fórmula de A. de Moivre, demuestre que
\ppart
$\sin(3x)=3\sin(x)-4 \sin^3(x)$, para todod $x \in \real$

\ppart
Para todo $n \in \nat$ par, la función $\cos(n \phi)$ es un polinomio de grado $n$ de $\cos(\phi)$.

\solution

\spart
Aquí hay que tener algo de idea feliz, aunque sabiendo que estamos trabajando con complejos, tampoco es demasiado raro de pensar.

Vamos a elevar el complejo $\cos(x)+i \sin(x)$ al cubo de dos formas distintas y a igualar los resultados.

\begin{enumerate}
\item
\[\left( \cos(x)+i \sin (x) \right)^3 = (e^{ix})^3 = e^{3ix} = \left( \cos(3x)+i \sin (3x) \right)\]
\item
\[\left( \cos(x)+i \sin (x) \right)^3 = . . . = \cos^3(x)-3\cos(x)\sin^2(x) + i \left( 3\cos^2(x)\sin(x)-\sin^3(x)\right)\]
\end{enumerate}
Ahora, puesto que deben ser iguales las dos representaciónes del cubo calculado, debemos igualar las partes reales y las imaginarias.

En este caso, en cuanto forzamos la igualdad de las partes imaginarias obtenemos la igualdad buscada.
\[3\cos^2(x)\sin(x)-\sin^3(x) = \sin(3x) \iff 3\sin(x) - 3 \sin^3(x) - \sin^3(x)=\sin(3x) \iff\]
\[\iff 3\sin(x) - 4 \sin^3(x)=\sin(3x) \]

\spart
El procedimiento a seguir es prácticamente igual que en el caso anterior. Vamos a calcular $\left(\cos(\phi)+i\sin(\phi)\right)^n$ de dos formas distintas
\begin{enumerate}
\item
\[\left(\cos(\phi)+i\sin(\phi)\right)^n = \cos(n\phi)+i\sin(n\phi)\]
\item
\[\left(\cos(\phi)+i\sin(\phi)\right)^n = \sum_{k=0}^n { n \choose k} \cos(\phi)\left( i \sin (\phi)\right)^{n-k}\]
\end{enumerate}

Atendiendo al sumatorio, vemos que vamos a obtener reales siempre que $k$ sea par. En otro caso tendremos siempre un múltiplo de $i$. La suma de esos múltiplos de $i$ acabará siendo $\sin (n\phi)$.

Aplicando esto llegamos a:
\[\cos(n\phi) = \sum_{0 \leq k \leq n} { n \choose k} \cos^k(\phi)(-1)^{\frac{n-k}{2}}\sin^{n-k}(\phi)\]

pero, si nos fijamos en el seno, tenemos que

\[\sin^{n-k}(\phi) = \left(\sin^2(\phi)\right)^{\frac{n-k}{2}} = \left(1-\cos^2(\phi)\right)^{\frac{n-k}{2}}\]

y aplicando esta relación a la igualdad anterior, obtenemos
\[\cos(n\phi) = \sum_{0 \leq k \leq n} (n,k) \cos^k(\phi)(-1)^{\frac{n-k}{2}}\left(1-\cos^2(\phi)\right)^{\frac{n-k}{2}}\]

que, efectivamente, se trata de un polinomio de grado $n$ de $\cos(\phi)$
\end{problem}

\begin{problem}[7]
Demuestre que
\[\left( \frac{1+i\tan(\phi)}{1-i\tan(\phi)}\right)^n = \frac{1+i\tan(n\cdot\phi)}{1-i\tan(n\cdot\phi)}\]

\solution

Vamos a autoconvencernos de que la igualdad es cierta con $n=2$
\[\left( \frac{1+i\tan(\phi)}{1-i\tan(\phi)}\right)^2 = \frac{(1+i\tan(\phi))^2}{(1-i\tan(\phi))^2} = \frac{1-\tan(\phi)^2+2i\tan(\phi)}{1-\tan(\phi)^2-2i\tan(\phi)} = \frac{\cos(\phi)^2-\sin(\phi)^2+2i\sin(\phi)\cos(\phi)}{\cos(\phi)^2-\sin(\phi)^2-2i\sin(\phi)\cos(\phi)}=\]
\[=\frac{\cos(\phi)^2-\sin(\phi)^2+i\sin(2\phi)}{\cos(\phi)^2-\sin(\phi)^2-i\sin(2\phi)}\]
Ahora dividimos entre $\cos(\phi)^2-\sin(\phi)^2$ y, sabiendo que
\[\tan(2 \phi)=\frac{2\sin(\phi)\cos(\phi)}{\cos(\phi)^2-\sin(\phi)^2}\]
tenemos que
\[\frac{\cos(\phi)^2-\sin(\phi)^2+i\sin(2\phi)}{\cos(\phi)^2-\sin(\phi)^2-i\sin(2\phi)}=\frac{1+i\tan(2\cdot\phi)}{1-i\tan(2\cdot\phi)}\]

Ahora vamos a aplicar inducción. Suponemos que la igualdad es cierta para $n$ y vamos a ver qué ocurre con $n+1$.
\[\left( \frac{1+i\tan(\phi)}{1-i\tan(\phi)}\right)^{n+1} =  \frac{\left(1+i\tan(n\cdot\phi)\right)\left(1+i\tan(\phi)\right)}{\left(1-i\tan(n\cdot\phi)\right)\left(1-i\tan(\phi)\right)} = \frac{1-\tan(n\phi)\tan(\phi)+i\left(\tan(\phi)+\tan(n\phi)\right)}{1-\tan(n\phi)\tan(\phi)-i\left(\tan(\phi)+\tan(n\phi)\right)} = \]
multiplicando y dividiendo por $\cos(n\phi)\cos(\phi)$ llegamos a
\[=\frac{\cos(n\phi)\cos(\phi)-\sin(n\phi)\sin(\phi)+i\left(\sin(\phi)\cos(n\phi) + \cos(\phi)\sin(n\phi)\right)}{\cos(n\phi)\cos(\phi)-\sin(n\phi)\sin(\phi)-i\left(\sin(\phi)\cos(n\phi) + \cos(\phi)\sin(n\phi)\right)} =\]
\[=\frac{\cos(n\phi)\cos(\phi)-\sin(n\phi)\sin(\phi)+i\left(\sin((n+1)\phi)\right)}{\cos(n\phi)\cos(\phi)-\sin(n\phi)\sin(\phi)-i\left(\sin((n+1)\phi)\right)}\]
Al igual que hicimos en el caso particular de $n=2$, ahora multiplicamos y dividimos por $\cos(n\phi)\cos(\phi)-\sin(n\phi)\sin(\phi)$ y, sabiendo que
\[\tan((n+1)\phi)=\frac{\sin((n+1)\phi)}{\cos((n+1)\phi)}=\frac{\sin(\phi)\cos(n\phi) + \cos(\phi)\sin(n\phi)}{\cos(n\phi)\cos(\phi)-\sin(n\phi)\sin(\phi)}\]
obtenemos directamente el resultado.

\[\frac{\cos(n\phi)\cos(\phi)-\sin(n\phi)\sin(\phi)+i\left(\sin((n+1)\phi)\right)}{\cos(n\phi)\cos(\phi)-\sin(n\phi)\sin(\phi)-i\left(\sin((n+1)\phi)\right)} =  \frac{1+i\tan((n+1)\cdot\phi)}{1-i\tan((n+1)\cdot\phi)}\]
\obs La última igualdad indicada se obtiene calculando $\sen(α+β)$ y $\cos(α+β)$ con las fórmulas habituales, considerando $α=n\phi$ y $β=\phi$

\end{problem}

\begin{problem}[8]
Sin realizar cálculo alguno, razónese que no es posible que alguno de los valores de $\sqrt[1928]{1+i}$ sea $\frac{1-i}{2}$

\solution
Lo más fácil, en este caso, es ver que los módulos no coinciden. Para ello escribimos
\[1+i = 2^{\frac{1}{2}}e^{(\frac{\pi}{4}+2\pi k)i}\]
y al calcular la raíz obtenemos
\[(1+i)^{\frac{1}{1928}} = 2^{\frac{1}{2\cdot 1928}}e^{(\frac{\pi}{4}+2\pi k)\frac{i}{1928}}\]

Llegados a este punto, podemos ver que los módulos no coinciden, pues
\[2^{\frac{1}{1928}}\neq \left|\frac{1-i}{2}\right| = \sqrt{\frac{1}{2}}\]
\end{problem}

\begin{problem}[9]
Demuestre las siguientes afirmaciones
\ppart
Si $z\neq 1$ entonces
\[\sum_{i=0}^n z^i = \frac{1-z^{n+1}}{1-z}\]

\ppart
Si $w\neq 1$ es una raíz n-ésima de la unidad, entonces
\[\sum_{i=0}^{n-1} w^i = \sum_{i=1}^n w^i= 0\]
y
\[\sum_{i=0}^{n-1} i w^i = \frac{n}{w-1}\]

\ppart
si $\sin\left(\frac{\phi}{2}\right)$, entonces
\[\sum_{i=0}^n \cos(i\phi) = \frac{1}{2}\left(1+\frac{\sin((n+\frac{1}{2})\phi)}{\sin\left( \frac{\phi}{2}\right)}\right)\]

y

\[\sum_{i=1}^n \sin(\phi) = \frac{\sin(\frac{n}{2}\phi)\sin(\frac{n+1}{2}\phi)}{\sin(\frac{\phi}{2})}\]

\textbf{Ayuda:} Use el apartado a) con $z=e^{i\phi}$

\solution
\spart
Vamos a demostrarlo por inducción. En este caso, el caso base es trivial, pues sería $n=1$ con lo que tendríamos
\[1+z=\frac{1-z^2}{1-z}=\frac{(1-z)(1+z)}{1-z} = 1+z\]
Ahora suponemos que la fórmula es válida para $n$ y vamos a ver qué ocurre para $n+1$.
\[\sum_{i=0}^{n+1} z^i = \sum_{i=0}^n z^i + z^{n+1} = \frac{1-z^{n+1}}{1-z} + z^{n+1} = \frac{1-z^{n+1}+z^{n+1}-z^{n+2}}{1-z} = \frac{1+z^{n+2}}{1-z}\]
por lo que queda probado que si la ecuación se cumple para $n$ se cumple también para $n+1$ y, puesto que se cumple para 1, podemos concluir que la ecuación es válida.

\spart
Este caso resulta muy sencillo y rápido sin los apoyamos en el anterior y sabemos que $w^n=1$ y que $w^{n+1}=w$ siendo $w$ una raíz n-ésima de la unidad.

Por el apartado anterior sabemos que
\[\sum_{i=0}^{n+1} z^i = \sum_{i=0}^n z^i + z^{n+1} = \frac{1-w^{n+1}}{1-w}-1 = \frac{1-w}{1-w}-1 = 0\]

Para la segunda igualdad símplemente tenemos que darnos cuenta de que
\[\sum_{i=0}^{n-1} i w^i  = \frac{\partial}{\partial w} \sum_{i=0}^{n}w^i = \frac{\partial}{\partial w} \frac{1-w^{n+1}}{1-w} = \frac{(-n-1)(1-w)+(1-w)}{(1-w)^2} = \frac{-n-1+1}{1-w}\]
y llegamos a
\[\sum_{i=0}^{n-1} i w^i = \frac{-n}{1-w}=\frac{n}{w-1}\]
\spart
%TODO completar "Lo hice" en clase


\end{problem}

\begin{problem}[10]
Calcule todos los valores de
\ppart
\[\left(-\sqrt{2}-i\sqrt{2}\right)^{1/3}\]
\ppart
\[\sqrt{1-i\sqrt{3}}\]
\ppart
\[\sqrt[4]{1-i}\]
\ppart
\[\left(\sqrt{-i}\right)^{1/3}\]

\solution

\spart
\[\left(-\sqrt{2}-i\sqrt{2}\right)^{1/3}=\left(\sqrt{2}(-1-i)\right)^{1/3}=(\sqrt{2}\sqrt{2})^{1/3}\left(e^{\frac{-3\pi}{4}+2k\pi}\right)^{1/3}2^{1/3}e^{\frac{-\pi}{4}+\frac{2}{3}k\pi}\]

\spart
\[\sqrt{1-i\sqrt{3}} = \sqrt{4e^{(\pi/3+2k\pi)i}}=2e^{(\pi/6+k\pi)i}\]
\spart
\[\sqrt[4]{1-i} = \sqrt[4]{2e^{7\pi/8+2k\pi}} = \sqrt[4]{2}e^{(7\pi/32 + k\pi /2)i}\]

\spart

\[\left(\sqrt{-i}\right)^{1/3} = \left(1e^{(\pi+2k\pi)i}\right)^{1/6} = e^{(\pi/6+k\pi/3)i}\]
\end{problem}

\begin{problem}[11]
En este ejercicio, consideramos sólo el \textit{valor principal de la raíz cuadrada}, definido como
\[\sqrt[(p)]{z}=\sqrt{r}\left(\cos\frac{\phi}{2}+i\sin\frac{\phi}{2}\right)\]
cuando $z=r(\cos\phi+i\sin\phi)$ con $-\pi < \phi < \pi$. Claramente, $\left( \sqrt[(p)]{z} \right)^2=z$
\ppart Demuestra que las soluciones en $\cplex$ de la ecuación $az^2+bz+c=0$, con $a\neq 0$, son
\[z=\frac{-b\pm \sqrt[(p)]{b^2-4ac}}{2a}\]
\ppart
Calcule
\[\sqrt[(p)]{\left(\sqrt[(p)]{i}\right)^5} \text { y } \sqrt[(p)]{1+\sqrt[(p)]{i}}\]
\solution

\spart
Para resolver este apartado basta son sustituir la fórmula que nos dan para la $z$ en la ecuación dada y comprobar que, efectivamente, la ecuación se verifica.

\spart
La raíz principal puede sonar a algo exótico pero consiste, simplemente, en tomar la raíz del número dado y, en lugar de considerar los varios ángulos posibles, tomamos el menor posible (siempre positivo).

A efectos legales esto nos hace ahorrarnos el típico $+2k\pi$. Veamos a modo e ejemplo los radicales que nos pide calcular el enunciado
\[\sqrt[(p)]{\left( \sqrt[(p)]{i}\right)^5} = \sqrt[(p)]{e^{-\frac{3\pi}{4}i}} = e^{-\frac{3\pi}{8}}i\]
\end{problem}

\begin{problem}[12]
Resuelve las siguientes ecuaciones:
\ppart
\[(z+1)^4+i=0\]
\ppart
\[Re(z^2+5)=0\]
\ppart
\[Re(z+5)=Im(z-i)\]
\solution

\spart
Despejando como hemos hecho siempre tenemos que
\[z=\sqrt[4]{-i}-1 = e^{\pi/4+\pi k / 2} -1 \]

\spart
Considerando $z=x+iy$ tenemos que $z^2=x^2-y^2+2xyi$ con lo que llegamos a
\[x^2-y^2 = -5\]
que nos da una hipérbola
\spart
Considerando $z=x+iy$ tenemos
\[Re(z+5)=Im(z-i) \iff x+5=y-1\]
obteniendo como resultado una recta.

\end{problem}

\begin{problem}[13]
\ppart
Demuestra que si $w$ es solución de $z^n=μ$ (con $μ\in\cplex$ fijo), entonces todas las soluciones son $ww_i$ con $i=0,1,...,n-1$ donde $w_i$ son las raíces n-ésimas de la unidad
\ppart
Encuentre razonadamente las soluciones de $z^6-8=0$
\solution
\textcolor{blue}{Hecho por mi. No fiarse al 100\%}

\spart
Es sencillo e ver que los números de la forma $ww_i$ son soluciones, puesto que
\[(ww_1)^n = w^n w_i^n= μ \cdot 1 = μ\]
Cualquier otra hipotética solución deberá cumplir que al elevarla a $n$ obtengamos μ, por lo que deberá ser $w$ multiplicado por algo que, al elevarlo a $n$ nos de 1. Es decir, no habrá más posibilidades que las indicadas

\spart
Siguiente lo indicado en el apartado anterior las soluciones serán de la forma:
\[z=\sqrt[6]{8}w_i \text{ con } i=0,1,2,3,4,5 \text{ y } w \text{ raíz n-ésima de la unidad}\]

\end{problem}

\begin{problem}[14]
¿Cuándo son colineales tres puntos $z_1,z_2,z_3$ distintos dos a dos?
\solution
Para verlo hacemos como en bachillerato con los reales: escribimos la recta que pasa por dos de esos puntos y forzamos a que el tercero se contenga en dicha recta.

La recta que pasa por $z_1$ y $z_2$ sería:
\[L=\{z_1+t(z_2-z_1) t \in \real \}\]
Si $z_3 \in L \implies \exists t \tq z_3=z_1 + t(z_2-z_1)$ es decir:
\[t = \frac{z_3-z_1}{z_2-z_1}\in \real\]
para que sea real ese resultado necesitamos que el numerador y el denominador tengan el mismo argumento.

Intuitivamente representa que uniendo $z_3$ con $z_1$ obtenemos la misma recta que uniendo $z_2$ con $z_1$
\end{problem}

\begin{problem}[15]

\ppart
Compruebe que la ecuación
\[Re(az+b) = 0 \text{ con } a,b\in \cplex, \ a \neq 0\]
define una recta en el plano y que, recíprocamente, cada recta viene descrita por una ecuación de este tipo

\ppart
Encuentre los números $a,b$ para que la recta pase por dos puntos dados $z_1, z_2 \in \cplex$

\ppart
Demuestre que las rectas determinadas por las ecuaciones $Re(az+b)=0$ y $Re(cz+d)=0$ respectivamente, son perpendiculares si y sólo si $Re(a\bar{c})=0$

\ppart
Demuestre que la ecuación de una recta que pasa por dos puntos dados $z_1$ y $z_2$ puede escribirse de la forma
\[ \left| \begin{array}{ccc}
z  & \bar{z} & 1 \\
z_1 & \bar{z_1}&  1 \\
z_2 & \bar{z_2} & 1 \end{array} \right| = 0\]

\solution
\textcolor{blue}{Hecho por mi. No fiarse al 100\%}

\spart

Siendo cada número complejo $x\in \cplex = x_r+ix_i$, la ecuación que nos dan se traduce en
\[a_rz_r-a_iz_i+b_r=0 \equiv z_i = \frac{a_rz_r+b_r}{a_i}\]
siendo $x_i = y$ t $z_r = x$ obtenemos la ecuación de una recta en el plano.

\spart

Basta cons sutituir en la ecuación los valores $z_1=z_1r+iz_1i$ y $z_2=z_2r+iz_2i$ y obtenemos un sistema de 4 ecuaciones e 4 incógnitas que podremos resolver.

\spart

Basándonos en el apartado a), podemos ver que las pendientes de esas rectas son, respectivamente, $\frac{a_r}{a_i}$ y $\frac{c_r}{c_i}$.

Para que sean perpendiculares, debemos tener
\[\frac{a_r}{a_i}= - \frac{c_i}{c_r} \implies a_rc_r = -a_ic_i \implies Re\left((a_r+ia_i)(c_r-ic_i)\right) = 0\]

\spart
%TODO por hacer

\end{problem}

\begin{problem}[16]
Describa el conjunto del plano complejo determinado por las siguientes relaciones
\ppart
\[|z-2|-|z+2| > 3\]
\ppart
\[Re(z)+Im(z) < 1\]
\ppart
\[|2z|>|1+z^2|\]

\solution
\textcolor{blue}{Hecho por mi. No fiarse al 100\%}

\spart
Si tuviéramos una igualdad, estaríamos hablando de los puntos del plano cuya diferencia de distancias a los puntos $(2,0)$ y $(-2,0)$ es constante. Es decir, tendríamos una hipérbola.

Al tener una desigualdas, estamos cogiendo aquellos puntos situados a la derecha de la hipérbola.

\spart

Esta ecuación representa aquellos puntos del plano que quedan a la izquierda de la recta $y=-x+1$.

\spart


\end{problem}

\begin{problem}[17]
Determine las ecuaciones complejas:
\ppart de la parábola con foco i y directriz $Im(z)=-1$
\ppart de la elipse con focos $\pm 1$ que pasa por $i$
\ppart de la hipérbola con focos $\pm 1$ que pasa por $i+1$

\solution

Este ejercicio es bastante semejante a los apartados b) y c) del ejercicio 1.10
\spart
Recordemos que una parábola se definía a partir del foco y la directriz como el conjunto de puntos del plano que equidistaban de ellos.

Para escribir la ecuación, simplemente aplicamos la definición y vemos a que ecuación nos lleva.

Sea un punto cualquiera $z=x+iy$ su distancia al foco $i$ sería $|z-i|$ mientras que la distancia a la directriz sería $1+Re(z)$

Igualando tenemos la ecuación buscada
\[|z-i|=Re(z)+1\]

\spart
Recordemos que, por definición, la elipse es el conjunto de puntos del plano con suma de distancias a los focos constante.

Conocemos los focos lo que nos lleva a:
\[|z-1|+|z+1|=cte\]

Para determinar la constante nos basamos en que pasa por $i$, lo que nos lleva a concluir que la constante es $2\sqrt{2}$ es decir, nos queda la ecuación
\[|z-1|+|z+1|=2\sqrt{2}\]

\spart
La hipérbola tenía definición similar a la de la elipse salvo que en este caso considerábamos constante la diferencia de distancias a los pocos en lugar de la suma.

De aquí obtenemos que la ecuación buscada será de la forma
\[|z-1|-|z+1|=cte\]
Sabiendo que pasa por el punot $i+1$ podemos calcular la constante
%TODO completar

\end{problem}

\begin{problem}[18]
Esboce el conjunto de puntos $z \in \cplex$ que satisfacen
\ppart \[Re\left( \frac{z}{1+i}\right) = 0\]
\ppart \[|z^2-4z+4| = 4\]
\ppart \[|z^2-2z-1|=1\]

\solution

\spart
Vamos a jugar un poco con el número que nos dan. Siendo $z=x+iy$ tenemos
\[\frac{x+iz}{1+i}\cdot\frac{1-i}{1-i} = \frac{x-y+i(y-x)}{2} \implies Re\left( \frac{z}{1+i}\right) = \frac{x-y}{2}\]

Por tanto, obtenemos la recta $y=x$, la bisectriz del primer cuadrante.

\spart
\[z^2-4z+4| = 4 \iff |z-2|^2 = 4 \]
con lo que tenemos una circunferencia

\spart
Se deja como ejercicio para el lector, que deberá pasar a cordenadas polares con el objetivo de poder esbozar el dibujo pedido.

\textbf{consejo:} Acordarnos de la Lemniscata
\end{problem}

\begin{problem}[19]
\ppart Sea $a \in \cplex$ un número fijo. Encuentre el máximo de $|z^{12}-a|$ cuando $z$ es cualquier número complejo tal que $|z|\leq 1$
\ppart Halle razonadamente el supremo y el ínfimo del siguiente conjunto de números reales
\[\{Re(iz^4+1) \tq |z| \sqrt{2}\}\]

\solution

\end{problem}

\begin{problem}[20]
Describa geométricamente el conjunto de los puntos $w \in \cplex$ que se escriben en la forma $w=iz^2+1$, para $z=x+iy$ con $x>0, y>0, \ x^2+y^2<1$.

\solution

Operando, tenemos que estamos trabajando con el conjunto de números complejos de la forma:
\[w=i(x^2-y^2)-2xy+1\]


\end{problem}

\begin{problem}[21]
Demuestre que, dados $a,c \in \cplex$, la condición necesaria y suficiente para que exista $z \in \cplex$ que verifique $|z+a|+|z-a|=2|c|$ es que sea $|a|\leq|c|$

\textbf{Ayuda:} Si λ>0, el conjunto $ \{z \in \cplex \tq |z+a|+|z-a|=2λ\}$ es una elipse si $λ > |a|$, un segmento si $λ=|a|$ y el conjunto vacío si $λ<|a|$

\solution

Basándonos en la indicación dada es obvio que $|a|\leq|c|$ es condición necesaria y suficiente para que podamos hablar de la solución de la ecuación ya que, en caso contrario, tendríamos el vacío.

Si tenemos $|a|\leq|c|$ el conjunto de puntos solución de la ecuación constituirán una recta o una elipse (según el caso) pero en ambos casos son conjuntos válidos que nos dan solución para la ecuación.
\end{problem}

\begin{problem}[22]
He aquí algunas interpretaciones geométricas de ciertas operaciones con números complejos.

\ppart Si $z=x+iy \in \cplex$ sea $α(z)$ el vector de tres dimensiones $(x,y,0)$. Verifique que para cada $z,w \in \cplex$ se cumple que $α(z)α(w)=Re(z\bar{w})$ y $α(z)\times α(w)=(0,0,Im(\bar{z}w))$

\ppart Si $0,z,w$ son los vértices de un triángulo $T$, compruebe que $Area(T)=\frac{1}{2}|Im(\bar{z}w)|$

\ppart
Si $z_1, z_2,...z_n$ son los vértices de un polígono $P$ que contiene a 0 en su interior, demuestra que $Area(P)=\frac{1}{2}\left|Im\left( \sum_{j=1}^n \bar{z}_jz_{j+1}\right)\right|$, donde se toma $z_{n+1}=z_1$

\solution
\textcolor{blue}{Hecho por mi. No fiarse al 100\%}

\spart
\[α(z)α(w)=z_xw_x+z_yw_y\]
Por otro lado
\[Re(zw)=Re\left(z_xw_x+z_yw_y+i(z_xw_y-z_yw_x)\right) = z_xw_x+z_yw_y \]

Si calculamos el producto vectorial que se nos pide, tenemos que
\[α(z)\times α(w) = (0,0,-z_xw_y+z_yw_x)\]
que podemos comprobar que coincide con la parte imaginaria de
\[Im(\bar{z}w) = Im \left( z_xw_x-z_yw_y+i(-z_xw_y+z_yw_x)\right)\]

\spart

Si ya hemos visto que el producto vectorial coincide con la parte imaginaria, es trivial ver que un medio de esa parte imaginaria nos dará el área del triángulo, pues el producto vectorial nos da el área del paralelogramo generado por los dos vectores.

\spart
Con imagen del producto que se nos da (tras multiplicar por 1/2) obtenemos el área del triángulo formado por los dos puntos dados y el origen.

Puesto que el origen se contiene en la figura cuyo área estamos calculando, al hacer esta operación con todos los vértices tenemos el área de la figura.

\end{problem}

\begin{problem}
Demuestre que la condición necesaria y suficiente para que $\{z_1, z_2, z_3\}$ sea el conjunto de los vértices de un triángulo equilátero es que
\[z_1z_2+z_2z_3+z_3z_1=z_1^2+z_2^2+z_3^2\]
\textbf{Ayuda:} Considere el triángulo $\{z_2, z_3,z_1\}$

\solution
\end{problem}


%%%%%%%%%%%%%%%%%%%%%%%%%%%%%%%%%%%%%%%%%%%%%%%%%%%%%%%%%%%%%%%%%%%%%%%%
%%%%%%%%%%%%%%%%%%%%%%%%%%%%%%%%%%%%%%%%%%%%%%%%%%%%%%%%%%%%%%%%%%%%%%%%
%%                                                                    %%
%%                            HOJA 2                                  %%
%%                                                                    %%
%%%%%%%%%%%%%%%%%%%%%%%%%%%%%%%%%%%%%%%%%%%%%%%%%%%%%%%%%%%%%%%%%%%%%%%%
%%%%%%%%%%%%%%%%%%%%%%%%%%%%%%%%%%%%%%%%%%%%%%%%%%%%%%%%%%%%%%%%%%%%%%%%
\newpage
\section{Hoja 2}
\begin{problem}[1]
(\textit{Esfera de Riemann}) Se considera $\widehat{\cplex} = \cplex = \{\infty\}$ y se definen los entornos de $\infty$ como aquellos que contienen un conjunto de la forma $\{z \in \cplex \tq |z|>R\}$ para algún $R > 0$

Con estos entonrnos $z_n \to \infty$ quiere decir que
\[\forall R > 0 \exists N \tq |z_n| > R \ \forall n > N\]

De manera similar se definen $\lim_{z \to b} f(z)= \infty$ y $\lim_{z \to \infty}f(z)=\infty$.

Sean $\mathbb{S}= \{p \in \real^3 : p_1^2+p_2^2+p_3^2\}$ y consideramos la proyección estereográfica:
\[\appl{\pi}{\mathbb{S}}{\widehat{\cplex}}, \pi(p) = \left\{
\begin{array}{lcc}
    \frac{(p_1+ip_2)}{1-p_3} & si & p \neq N = (0,0,1) \\
 \\ \infty & si & p = N
 \end{array} \right.\]

 \ppart
 Compruebe que
 \[\pi^{-1}\left( \frac{2Re(z)}{|z|^2+1}, \frac{2Im(z)}{|z|^2+1}, \frac{|z|^2-1}{|z|^2+1}\right)\]

 \ppart
 Sea $\rho(z,w)$= distancia (en $\real^3$) entre $\pi^{-1}(z)$ y $\pi^{-1}(w)$ para $z,w \in \widehat{\cplex}$. Entonces:
 \[z_n \to z \text{ en } \widehat{\cplex} \implies \rho(z_n,z) \to 0\]

 \ppart
 Demuestre que
 \[\lim_{n \to \infty}\frac{z^n}{n} = \infty \text{ si } |z| > 1\]

\solution

\end{problem}

\begin{problem}[2]
\ppart
Demuestre que, mediante la proyección estereográfica, las circunferencias sobre la esfera se transforman en circunferencias o rectas del plano. ¿Cuáles son las circunferencias sobre la esfera que se transforman en rectas?

\ppart
¿Qué corresponde en la esfera de Riemann a una familia de rectas paralelas del plano?

\ppart
Halle, en la esfera de Riemann, las imágenes de los conjuntos definidos por las siguientes desigualdades:
\begin{enumerate}
\item $Im(z) > 0$
\item $Re(z) < 1$
\item $|z| < 1$
\item $|z| > 2$
\end{enumerate}

\solution

\end{problem}

\begin{problem}[3]
Decida si las sucesiones $z_n= \left(\frac{1-2i}{3}\right)^n, \ w_n = \left( \frac{3-4i}{5}\right)^n$ tienen límite (finito) o no

\solution

\end{problem}

\begin{problem}[4]
Decida razonadamente si las siguientes funciones tienen límite (finito) o no en el punto indicado

\ppart
\[f(x) = \frac{|z|^2}{z} (\text{ para z}\neq0 )\text{ en el punto } z=0\]

\ppart
\[f(z)= \frac{z^3-8i}{z+2i} ( \text{ para z} \neq -2i) \text{ en el punto } z=-2i\]

\solution

\end{problem}

\begin{problem}[5]
Demuestre las siguientes afirmaciones
\ppart

Si $P(z)=a_nz^n+\cdots + a_0$ y $Q(z)=b_mz^m+\cdots b_0$ son polinomios con $a_n \neq 0 \neq b_m$ entonces se tiene
\[\lim_{z \to \infty} \frac{P(z)}{Q(z)} = \left\{
\begin{array}{lcc}
    0& si & n < m \\
    \\ \frac{a_n}{b_m} & si & n=m \\
 \\ \infty & si & n > m
 \end{array} \right.\]

\ppart
No existe $\lim_{z \to \infty}e^z$
\solution

\end{problem}

\begin{problem}[6]
Halle los puntos de continuidad de las funciones:
\[f(z)=\left\{
\begin{array}{lcc}
    \frac{z^4-1}{z-i}& si & z \neq i \\
 \\ 4i & si & z=i
 \end{array} \right.\]
\[g(z)=\left\{
\begin{array}{lcc}
    z & si & |z| \leq 1 \\
 \\ |z|^2 & si & |z| > 1
 \end{array} \right.\]

 \solution


\end{problem}


\begin{problem}[7]
¿Dónde son holomorfas las siguientes funciones?
\ppart $f(x,y)=x^2-y^2+ixy$
\ppart $f(z)=g(\bar{z})$, donde $g$ es holomorfa en $\Omega$
\ppart $f(z)=\overline{g(z)}$, donde $g$ es holomorfa en $\Omega$
\ppart $f(z)=\overline{g(\bar{z})}$, donde $g$ es holomorfa en $\Omega$
\ppart $f(z)=|g(z)|$, donde $g$ es holomorfa en $\Omega$

\textbf{Ayuda:} en los apartados b)-e) basta con usar la definición de derivada.

\solution

\end{problem}