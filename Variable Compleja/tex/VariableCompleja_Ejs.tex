% -*- root: ../VariableCompleja.tex -*-

\section{Hoja 1}
%
\begin{problem}[1]
Realices las operaciones indicadas
\ppart
\[\frac{1}{i}+\frac{1}{1+i}\]
\ppart
\[\frac{2}{(1-3i)^2}\]
\ppart
\[(1+i\sqrt{3})^3\]
\ppart
\[(\overline{1-i})^2+\overline{2+i}\]

\solution

\end{problem}

\begin{problem}[2]
Calcule los valores de
\ppart
\[\sum_{k=1}^{2015}i^k\]
\ppart
\[(1+i)^n+(1-i)^n\]
\ppart
\[\left( \cos \left( \frac{\pi}{12} \right) + i \sin \left( \frac{\pi}{12} \right)\right)^20\]
\ppart
\[\frac{1+i}{1-i}^{2014}\]
\solution

\spart
\[\sum_{k=1}^{2015}i^k = \frac{1-i^2016}{1-i} -1\]

\spart
Primero debemos observar que
\[(1+i) = 2^{\frac{1}{2}}e^{\frac{\pi}{4}i} \implies (1+i)^n= 2^{\frac{n}{2}}e^{\frac{\pi}{4}in} = 2^{\frac{n}{2}}\left(\cos\left(\frac{\pi}{4}n\right)+i \sin\left(\frac{\pi}{4}n\right)\right)\]
Teniendo en cuenta esta relación, podemos resolver el ejercicio:
\[(1+i)^n+(1-i)^n = (1+i)^n+\overline{(1+i)^n} = 2 Re((1+i)^n)=2^{\frac{n}{2}+1}\cos\left(\frac{\pi}{4}n\right)\]
\end{problem}

\begin{problem}[3]
Sea $z=x+iy \in \cplex$. Demuestre que $|x|+|y|\leq \sqrt{2}|z|$, y que sólo hay igualdad si $|x|=|y|$.

\textbf{Ayuda:} Si $a,b \in \real$, entonces $2ab \leq a^2 + b^2$ (con igualdad sólo si $a=b$)

\solution

\end{problem}

\begin{problem}[4]
Compruebe la identidad
\[|z\bar{w}+1|^2+|z-w|^2 = (1+|z|^2)(1+|w|^2)\]
donde $z,w \in \cplex$

\solution

\end{problem}

\begin{problem}[5]
Demuestra las siguientes afirmaciones
\ppart
\[\text{Si } |z|=1, \text{ entonces para todos } a,b \in \cplex \text{ con } \bar{b}z+\bar{a} \text{ se cumpel } \left| \frac{az+b}{\bar{b}z+\bar{a}}\right| = 1\]

\ppart
\[\text{Si } |a| < 1, \text{ entonces } |z| <1 \text{ es equivalente a } \left| \frac{z-a}{1-\bar{a}z}\right|<1\]

\solution

\spart
\spart

\[\left| \frac{z-a}{1-\bar{a}z}\right|<1 \iff |z-a| < |1-\bar{a}z| \iff \underbrace{|z-a|^2}_{(z-a)(\bar{z}-\bar{a})} < |\underbrace{1-\bar{a}z|^2}_{(1-\bar{a}z)(1-a\bar{z})} \]

Por lo que nos queda que debe cumplirse
\[|z|^2-a\bar{z}-\bar{a}z+|a|^2 < 1-\bar{a}z+a\bar{z}+|a|^2|b|^2 \iff |z|^2+|a|^2-2\cdot Re(z\bar{a}) < 1 + |a|^2|z|^2-2\cdot Re(z\bar{a}) \iff\]

\[\iff |z|^2 + |a|^2 < 1 +|a|^2|z|^2 \iff |z|^2(1-|a|^2) < 1 - |a|^2 \iff |z|^2 < 1 \iff |z| < 1\]

\end{problem}

\begin{problem}[6]
Usando la fórmula de A. de Moivre, demuestre que
\ppart
$\sin(3x)=3\sin(x)-4 \sin^3(x)$, para todod $x \in \real$

\ppart
Para todo $n \in \nat$ par, la función $\cos(n \phi)$ es un polinomio de grado $n$ de $\cos(\phi)$.

\solution

\spart
Aquí hay que tener algo de idea feliz, aunque sabiendo que estamos trabajando con complejos, tampoco es demasiado raro de pensar.

Vamos a elevar el complejo $\cos(x)+i \sin(x)$ al cubo de dos formas distintas y a igualar los resultados.

\begin{enumerate}
\item
\[\left( \cos(x)+i \sin (x) \right)^3 = (e^{ix})^3 = e^{3ix} = \left( \cos(3x)+i \sin (3x) \right)\]
\item
\[\left( \cos(x)+i \sin (x) \right)^3 = . . . = \cos^3(x)-3\cos(x)\sin^2(x) + i \left( 3\cos^2(x)\sin(x)-\sin^3(x)\right)\]
\end{enumerate}
Ahora, puesto que deben ser iguales las dos representaciónes del cubo calculado, debemos igualar las partes reales y las imaginarias.

En este caso, en cuanto forzamos la igualdad de las partes imaginarias obtenemos la igualdad buscada.
\[3\cos^2(x)\sin(x)-\sin^3(x) = \sin(3x) \iff 3\sin(x) - 3 \sin^3(x) - \sin^3(x)=\sin(3x) \iff\]
\[\iff 3\sin(x) - 4 \sin^3(x)=\sin(3x) \]

\spart
El procedimiento a seguir es prácticamente igual que en el caso anterior. Vamos a calcular $\left(\cos(\phi)+i\sin(\phi)\right)^n$ de dos formas distintas
\begin{enumerate}
\item
\[\left(\cos(\phi)+i\sin(\phi)\right)^n = \cos(n\phi)+i\sin(n\phi)\]
\item
\[\left(\cos(\phi)+i\sin(\phi)\right)^n = \sum_{k=0}^n (n,k) \cos(\phi)\left( i \sin (\phi)\right)^{n-k}\]
\end{enumerate}

Atendiendo al sumatorio, vemos que vamos a obtener reales siempre que $k$ sea par. En otro caso tendremos siempre un múltipli de $i$ que acabará siendo 0.

Aplicando esto llegamos a:
\[\cos(n\phi) = \sum_{0 \leq k \leq n} (n,k) \cos^k(\phi)(-1)^{\frac{n-k}{2}}\sin^{n-k}(\phi)\]

pero, si nos fijamos en el seno, tenemos que

\[\sin^{n-k}(\phi) = \left(\sin^2(\phi)\right)^{\frac{n-k}{2}} = \left(1-\cos^2(\phi)\right)^{\frac{n-k}{2}}\]

y aplicando esta relación a la igualdad anterior, obtenemos
\[\cos(n\phi) = \sum_{0 \leq k \leq n} (n,k) \cos^k(\phi)(-1)^{\frac{n-k}{2}}\left(1-\cos^2(\phi)\right)^{\frac{n-k}{2}}\]

que, efectivamente, se trata de un polinomio de grado $n$ de $\cos(\phi)$
\end{problem}

\begin{problem}[7]
Demuestre que
\[\left( \frac{1+i\tan(\phi)}{1-i\tan(\phi)}\right)^n = \frac{1+i\tan(n\cdot\phi)}{1-i\tan(n\cdot\phi)}\]

\solution
TODO
\end{problem}

\begin{problem}[8]
Sin realizar cálculo alguno, razónese que no es posible que alguno de los valores de $\sqrt[1928]{1+i}$ sea $\frac{1-i}{2}$

\solution
Lo más fácil, en este caso, es ver que los módulos no coinciden. Para ello escribimos
\[1+i = 2^{\frac{1}{2}}e^{(\frac{\pi}{4}+2\pi k)i}\]
y al calcular la raíz obtenemos
\[(1+i)^{\frac{1}{1928}} = 2^{\frac{1}{2\cdot 1928}}e^{(\frac{\pi}{4}+2\pi k)\frac{i}{1928}}\]

Llegados a este punto, podemos ver que los módulos no coinciden, pues
\[2^{\frac{1}{1928}}\neq |\frac{1-i}{2}| = \sqrt{\frac{1}{2}}\]
\end{problem}