% -*- root: ../VariableCompleja.tex -*-
%%%%%%%%%%%%%%%%%%%%%%%%%%%%%%%%%%%%%%%%%%%%%%%%%%%%%%%%%%%%%%%%%%%%%%%%
%%%%%%%%%%%%%%%%%%%%%%%%%%%%%%%%%%%%%%%%%%%%%%%%%%%%%%%%%%%%%%%%%%%%%%%%
%%                                                                    %%
%%                            HOJA 1                                  %%
%%                                                                    %%
%%%%%%%%%%%%%%%%%%%%%%%%%%%%%%%%%%%%%%%%%%%%%%%%%%%%%%%%%%%%%%%%%%%%%%%%
%%%%%%%%%%%%%%%%%%%%%%%%%%%%%%%%%%%%%%%%%%%%%%%%%%%%%%%%%%%%%%%%%%%%%%%%

\section{Hoja 1}
%
\begin{problem}[1]
Realice las operaciones indicadas
\ppart
\[\frac{1}{i}+\frac{1}{1+i}\]
\ppart
\[\frac{2}{(1-3i)^2}\]
\ppart
\[(1+i\sqrt{3})^3\]
\ppart
\[(\overline{1-i})^2+\overline{2+i}\]

\solution

\spart
\[\frac{1}{i}+\frac{1}{1+i} = \frac{1+i+i}{i-1} = -\frac{2i}{1-i} = \frac{2i(1+i)}{(1-i)(1+i)} = \frac{2i-2}{1+1}=i-1\]

\spart
\[\frac{2}{(1-3i)^2} = \frac{2}{1-9-6i} = \frac{2}{-8-6i}=\frac{2(-8+6i)}{(-8-6i)(-8+6i)} = \frac{2(-8+6i)}{64+36} = \frac{-4+3i}{25}\]

\newpage
\spart
\[(1+i\sqrt{3})^3\ = 1 +3i\sqrt{3}-3\cdot 3-i3^{\frac{3}{2}} = -8 +i (3\sqrt{3}-3\sqrt{3}) = -8\]

\spart
\[(\overline{1-i})^2+\overline{2+i} = \overline{1-1-2i}+2-i = 2i+2-i=2+i\]
\end{problem}

\begin{problem}[2]
Calcule los valores de
\ppart
\[\sum_{k=1}^{2015}i^k\]
\ppart
\[(1+i)^n+(1-i)^n\]
\ppart
\[\left( \cos \left( \frac{\pi}{12} \right) + i \sin \left( \frac{\pi}{12} \right)\right)^{20}\]
\ppart
\[\left(\frac{1+i}{1-i}\right)^{2014}\]
\solution

\spart
Por tratarse de una sucesión geométrica de razón $i$ sabemos que:
\[\sum_{k=1}^{2015}i^k = \frac{1-i^{2015}}{1-i}=\frac{1-\left((i)^4\right)^{503}i^3}{1-i}=\frac{1+i}{1-i}=\frac{1-1+2i}{2}=i\]

\spart
Primero debemos observar que
\large
\[(1+i) = 2^{\frac{1}{2}}e^{\frac{\pi}{4}i}\]
\normalsize
por tanto
\[ (1+i)^n= 2^{\frac{n}{2}}e^{\frac{\pi}{4}in} = 2^{\frac{n}{2}}\left(\cos\left(\frac{\pi}{4}n\right)+i \sin\left(\frac{\pi}{4}n\right)\right)\]

Teniendo en cuenta esta relación, podemos resolver el ejercicio:
\[(1+i)^n+(1-i)^n = (1+i)^n+\overline{(1+i)^n} = 2 Re((1+i)^n)=2^{\frac{n}{2}+1}\cos\left(\frac{\pi}{4}n\right)\]

\spart
\[\left( \cos \left( \frac{\pi}{12} \right) + i \sin \left( \frac{\pi}{12} \right)\right)^{20}=\left(e^{\frac{\pi}{12}}\right)^{20} = e^{\frac{\pi}{12}\cdot 20} = \cos \left( \frac{\pi\cdot 20}{12} \right) + i \sin \left( \frac{\pi\cdot 20}{12} \right) =\]
\[=\cos \left( \frac{\pi\cdot 5}{3} \right) + i \sin \left( \frac{\pi \cdot 5}{3} \right)\]

\spart
Primero vamos a trabajar con el interior del paréntesis para convertirlo en un número complejo en su expresión habitual, sin fracciones.
\[\frac{1+i}{1-i}=\frac{(1+i)^2}{(1-i)(1+i)} = \frac{1-1+2i}{1+1} = i\]
y puesto que el exponente es par, tenemos que
\[\left(\frac{1+i}{1-i}\right)^{2014}=i^{2014}=-1\]
\end{problem}

\begin{problem}[3]
Sea $z=x+iy \in \cplex$. Demuestre que $|x|+|y|\leq \sqrt{2}|z|$, y que sólo hay igualdad si $|x|=|y|$.

\textbf{Ayuda:} Si $a,b \in \real$, entonces $2ab \leq a^2 + b^2$ (con igualdad sólo si $a=b$)

\solution

\doneby{Pedro}

Si calculamos el módulo de z vemos que
\[|z|=\sqrt{x^2+y^2}\]
Si $|x|=|y|$ fácilmente vemos que
\[|z|=\sqrt{x^2+x^2}=\sqrt{2x^2}=\sqrt{2}|x| \implies \sqrt{2}|z|=2|x|=|x|+|y|\]

Veamos ahora el caso en que no son iguales. En esta ocasión, nos apoyamos en al ayuda del enunciado y vemos que
\[|z|=\sqrt{x^2+y^2} \geq \sqrt{2xy} \iff \sqrt{2}|z| \geq 2\sqrt{xy}\]
%TODO terminar esto

%Sin pérdida de generalidad podemos suponer que $|x| > |y|$ y nos queda:
%\[\sqrt{2}|z| \geq 2 \sqrt{xy} \geq 2\sqrt{y^2} \geq 2|y|\]

\end{problem}

\begin{problem}[4]
Compruebe la identidad
\[|z\bar{w}+1|^2+|z-w|^2 = (1+|z|^2)(1+|w|^2)\]
donde $z,w \in \cplex$

\solution

Llamando a $z=a+bi$ y $w=c+di$ tenemos
\[|(a+bi)(c-di)+1|^2+|a+bi+c-di|^2=|ac-adi+bci+bd+1|^2+|a+bi-c+di|^2 = \]
\[=(ac+bd+1)^2+(bc-ad)^2+(a-c)^2+(b+d)^2 =\]
\[= 1 + a²c²+b²d²+2acbd + 2ac+2bd+b²c²+a²d²-2bcad+a^2+c²-2ac+b²+d²-2bd =\]
\[= 1+a²c²+b²d²+b²c²+a²d²+a²+b²+c²+d² = (1+a²+b²)(1+c²+d^2)\]

\end{problem}

\begin{problem}[5]
Demuestra las siguientes afirmaciones
\ppart
\[\text{Si } |z|=1, \text{ entonces para todos } a,b \in \cplex \text{ con } \bar{b}z+\bar{a} \neq 0\text{ se cumple } \left| \frac{az+b}{\bar{b}z+\bar{a}}\right| = 1\]

\ppart
\[\text{Si } |a| < 1, \text{ entonces } |z| <1 \text{ es equivalente a } \left| \frac{z-a}{1-\bar{a}z}\right|<1\]

\solution

\spart


\spart

\[\left| \frac{z-a}{1-\bar{a}z}\right|<1 \iff |z-a| < |1-\bar{a}z| \iff \underbrace{|z-a|^2}_{(z-a)(\bar{z}-\bar{a})} < \underbrace{|1-\bar{a}z|^2}_{(1-\bar{a}z)(1-a\bar{z})} \]

Por lo que nos queda que debe cumplirse
\[|z|^2-a\bar{z}-\bar{a}z+|a|^2 < 1-\bar{a}z+a\bar{z}+|a|^2|b|^2 \iff |z|^2+|a|^2-2\cdot Re(z\bar{a}) < 1 + |a|^2|z|^2-2\cdot Re(z\bar{a}) \iff\]

\[\iff |z|^2 + |a|^2 < 1 +|a|^2|z|^2 \iff |z|^2(1-|a|^2) < 1 - |a|^2 \iff |z|^2 < 1 \iff |z| < 1\]

\end{problem}

\begin{problem}[6]
Usando la fórmula de A. de Moivre, demuestre que
\ppart
$\sin(3x)=3\sin(x)-4 \sin^3(x)$, para todo $x \in \real$

\ppart
Para todo $n \in \nat$ par, la función $\cos(n \phi)$ es un polinomio de grado $n$ de $\cos(\phi)$.

\solution

\spart
Aquí hay que tener algo de idea feliz, aunque sabiendo que estamos trabajando con complejos, tampoco es demasiado raro de pensar.

Vamos a elevar el complejo $\cos(x)+i \sin(x)$ al cubo de dos formas distintas y a igualar los resultados.

\begin{enumerate}
\item
\[\left( \cos(x)+i \sin (x) \right)^3 = (e^{ix})^3 = e^{3ix} = \left( \cos(3x)+i \sin (3x) \right)\]
\item
\[\left( \cos(x)+i \sin (x) \right)^3 = . . . = \cos^3(x)-3\cos(x)\sin^2(x) + i \left( 3\cos^2(x)\sin(x)-\sin^3(x)\right)\]
\end{enumerate}
Ahora, puesto que deben ser iguales las dos representaciones del cubo calculado, debemos igualar las partes reales y las imaginarias.

En este caso, en cuanto forzamos la igualdad de las partes imaginarias obtenemos la igualdad buscada.
\[3\cos^2(x)\sin(x)-\sin^3(x) = \sin(3x) \iff 3\sin(x) - 3 \sin^3(x) - \sin^3(x)=\sin(3x) \iff\]
\[\iff 3\sin(x) - 4 \sin^3(x)=\sin(3x) \]

\spart
El procedimiento a seguir es prácticamente igual que en el caso anterior. Vamos a calcular $\left(\cos(\phi)+i\sin(\phi)\right)^n$ de dos formas distintas
\begin{enumerate}
\item
\[\left(\cos(\phi)+i\sin(\phi)\right)^n = \cos(n\phi)+i\sin(n\phi)\]
\item
\[\left(\cos(\phi)+i\sin(\phi)\right)^n = \sum_{k=0}^n { n \choose k} \cos(\phi)\left( i \sin (\phi)\right)^{n-k}\]
\end{enumerate}

Atendiendo al sumatorio, vemos que vamos a obtener reales siempre que $k$ sea par. En otro caso tendremos siempre un múltiplo de $i$. La suma de esos múltiplos de $i$ acabará siendo $\sin (n\phi)$.

Aplicando esto llegamos a:
\[\cos(n\phi) = \sum_{0 \leq k \leq n} { n \choose k} \cos^k(\phi)(-1)^{\frac{n-k}{2}}\sin^{n-k}(\phi)\]

pero, si nos fijamos en el seno, tenemos que

\[\sin^{n-k}(\phi) = \left(\sin^2(\phi)\right)^{\frac{n-k}{2}} = \left(1-\cos^2(\phi)\right)^{\frac{n-k}{2}}\]

y aplicando esta relación a la igualdad anterior, obtenemos
\[\cos(n\phi) = \sum_{0 \leq k \leq n} (n,k) \cos^k(\phi)(-1)^{\frac{n-k}{2}}\left(1-\cos^2(\phi)\right)^{\frac{n-k}{2}}\]

que, efectivamente, se trata de un polinomio de grado $n$ de $\cos(\phi)$
\end{problem}

\begin{problem}[7]
Demuestre que
\[\left( \frac{1+i\tan(\phi)}{1-i\tan(\phi)}\right)^n = \frac{1+i\tan(n\cdot\phi)}{1-i\tan(n\cdot\phi)}\]

\solution

Vamos a autoconvencernos de que la igualdad es cierta con $n=2$
\[\left( \frac{1+i\tan(\phi)}{1-i\tan(\phi)}\right)^2 = \frac{(1+i\tan(\phi))^2}{(1-i\tan(\phi))^2} = \frac{1-\tan(\phi)^2+2i\tan(\phi)}{1-\tan(\phi)^2-2i\tan(\phi)} = \] \[ =\frac{\cos(\phi)^2-\sin(\phi)^2+2i\sin(\phi)\cos(\phi)}{\cos(\phi)^2-\sin(\phi)^2-2i\sin(\phi)\cos(\phi)}=\frac{\cos(\phi)^2-\sin(\phi)^2+i\sin(2\phi)}{\cos(\phi)^2-\sin(\phi)^2-i\sin(2\phi)}\]
Ahora dividimos entre $\cos(\phi)^2-\sin(\phi)^2$ y, sabiendo que
\[\tan(2 \phi)=\frac{2\sin(\phi)\cos(\phi)}{\cos(\phi)^2-\sin(\phi)^2}\]
tenemos que
\[\frac{\cos(\phi)^2-\sin(\phi)^2+i\sin(2\phi)}{\cos(\phi)^2-\sin(\phi)^2-i\sin(2\phi)}=\frac{1+i\tan(2\cdot\phi)}{1-i\tan(2\cdot\phi)}\]

Ahora vamos a aplicar inducción. Suponemos que la igualdad es cierta para $n$ y vamos a ver qué ocurre con $n+1$.
\[\left( \frac{1+i\tan(\phi)}{1-i\tan(\phi)}\right)^{n+1} =  \frac{\left(1+i\tan(n\cdot\phi)\right)\left(1+i\tan(\phi)\right)}{\left(1-i\tan(n\cdot\phi)\right)\left(1-i\tan(\phi)\right)} = \frac{1-\tan(n\phi)\tan(\phi)+i\left(\tan(\phi)+\tan(n\phi)\right)}{1-\tan(n\phi)\tan(\phi)-i\left(\tan(\phi)+\tan(n\phi)\right)} = \]
multiplicando y dividiendo por $\cos(n\phi)\cos(\phi)$ llegamos a
\[=\frac{\cos(n\phi)\cos(\phi)-\sin(n\phi)\sin(\phi)+i\left(\sin(\phi)\cos(n\phi) + \cos(\phi)\sin(n\phi)\right)}{\cos(n\phi)\cos(\phi)-\sin(n\phi)\sin(\phi)-i\left(\sin(\phi)\cos(n\phi) + \cos(\phi)\sin(n\phi)\right)} =\]
\[=\frac{\cos(n\phi)\cos(\phi)-\sin(n\phi)\sin(\phi)+i\left(\sin((n+1)\phi)\right)}{\cos(n\phi)\cos(\phi)-\sin(n\phi)\sin(\phi)-i\left(\sin((n+1)\phi)\right)}\]
Al igual que hicimos en el caso particular de $n=2$, ahora multiplicamos y dividimos por $\cos(n\phi)\cos(\phi)-\sin(n\phi)\sin(\phi)$ y, sabiendo que
\[\tan((n+1)\phi)=\frac{\sin((n+1)\phi)}{\cos((n+1)\phi)}=\frac{\sin(\phi)\cos(n\phi) + \cos(\phi)\sin(n\phi)}{\cos(n\phi)\cos(\phi)-\sin(n\phi)\sin(\phi)}\]
obtenemos directamente el resultado.

\[\frac{\cos(n\phi)\cos(\phi)-\sin(n\phi)\sin(\phi)+i\left(\sin((n+1)\phi)\right)}{\cos(n\phi)\cos(\phi)-\sin(n\phi)\sin(\phi)-i\left(\sin((n+1)\phi)\right)} =  \frac{1+i\tan((n+1)\cdot\phi)}{1-i\tan((n+1)\cdot\phi)}\]
\obs La última igualdad indicada se obtiene calculando $\sen(α+β)$ y $\cos(α+β)$ con las fórmulas habituales, considerando $α=n\phi$ y $β=\phi$

\end{problem}

\begin{problem}[8]
Sin realizar cálculo alguno, razónese que no es posible que alguno de los valores de $\sqrt[1928]{1+i}$ sea $\frac{1-i}{2}$

\solution
Lo más fácil, en este caso, es ver que los módulos no coinciden. Para ello escribimos
\[1+i = 2^{\frac{1}{2}}e^{(\frac{\pi}{4}+2\pi k)i}\]
y al calcular la raíz obtenemos
\[(1+i)^{\frac{1}{1928}} = 2^{\frac{1}{2\cdot 1928}}e^{(\frac{\pi}{4}+2\pi k)\frac{i}{1928}}\]

Llegados a este punto, podemos ver que los módulos no coinciden, pues
\[2^{\frac{1}{1928}}\neq \left|\frac{1-i}{2}\right| = \sqrt{\frac{1}{2}}\]
\end{problem}

\begin{problem}[9]
Demuestre las siguientes afirmaciones
\ppart
Si $z\neq 1$ entonces
\[\sum_{k=0}^n z^k = \frac{1-z^{n+1}}{1-z}\]

\ppart
Si $ω\neq 1$ es una raíz n-ésima de la unidad, entonces
\[\sum_{k=0}^{n-1} ω^k = \sum_{k=1}^n ω^k= 0\]
y
\[\sum_{k=0}^{n-1} k ω^k = \frac{n}{ω-1}\]

\ppart
si $\sin\left(\frac{\phi}{2}\right)$, entonces
\[\sum_{k=0}^n \cos(k\phi) = \frac{1}{2}\left(1+\frac{\sin((n+\frac{1}{2})\phi)}{\sin\left( \frac{\phi}{2}\right)}\right)\]

y

\[\sum_{k=1}^n \sin(k\phi) = \frac{\sin(\frac{n}{2}\phi)\sin(\frac{n+1}{2}\phi)}{\sin(\frac{\phi}{2})}\]

\textbf{Ayuda:} Use el apartado a) con $z=e^{i\phi}$

\solution
\spart
Vamos a demostrarlo por inducción. En este caso, el caso base es trivial, pues sería $n=1$ con lo que tendríamos
\[1+z=\frac{1-z^2}{1-z}=\frac{(1-z)(1+z)}{1-z} = 1+z\]
Ahora suponemos que la fórmula es válida para $n$ y vamos a ver qué ocurre para $n+1$.
\[\sum_{k=0}^{n+1} z^k = \sum_{k=0}^n z^k + z^{n+1} = \frac{1-z^{n+1}}{1-z} + z^{n+1} = \frac{1-z^{n+1}+z^{n+1}-z^{n+2}}{1-z} = \frac{1+z^{n+2}}{1-z}\]
por lo que queda probado que si la ecuación se cumple para $n$ se cumple también para $n+1$ y, puesto que se cumple para 1, podemos concluir que la ecuación es válida.

\spart
Este caso resulta muy sencillo y rápido si nos apoyamos en el anterior y sabemos que $ω^n=1$ y que $ω^{n+1}=ω$ siendo $ω$ una raíz n-ésima de la unidad.

Por el apartado anterior sabemos que
\[\sum_{k=0}^{n+1} z^k = \sum_{k=0}^n z^k + z^{n+1} = \frac{1-ω^{n+1}}{1-ω}-1 = \frac{1-ω}{1-ω}-1 = 0\]

Para la segunda igualdad simplemente tenemos que darnos cuenta de que
\[\sum_{k=0}^{n-1} k ω^{k-1}  = \frac{\partial}{\partial ω} \sum_{k=0}^{n}ω^k = \frac{\partial}{\partial ω} \frac{1-ω^{n+1}}{1-ω} = \frac{(-n-1)(1-ω)+(1-ω)}{(1-ω)^2} = \frac{-n-1+1}{1-ω}\]

y llegamos a
\[\sum_{k=0}^{n-1} k ω^k = \frac{-n}{1-ω}=\frac{n}{ω-1}\]
\spart
%TODO completar "Lo hice" en clase


\end{problem}

\begin{problem}[10]
Calcule todos los valores de
\ppart
\[\left(-\sqrt{2}-i\sqrt{2}\right)^{1/3}\]
\ppart
\[\sqrt{1-i\sqrt{3}}\]
\ppart
\[\sqrt[4]{1-i}\]
\ppart
\[\left(\sqrt{-i}\right)^{1/3}\]

\solution

\spart
\[\left(-\sqrt{2}-i\sqrt{2}\right)^{1/3}=\left(\sqrt{2}(-1-i)\right)^{1/3}=(\sqrt{2})^{1/3}\left(e^{\frac{-3\pi}{4}+2k\pi}\right)^{1/3}=2^{1/3}e^{\frac{-\pi}{4}+\frac{2}{3}k\pi}\]

\spart
\[\sqrt{1-i\sqrt{3}} = \sqrt{4e^{(\pi/3+2k\pi)i}}=2e^{(\pi/6+k\pi)i}\]

\spart
\[\sqrt[4]{1-i} = \sqrt[4]{\sqrt{2}e^{7\pi/4+2k\pi}} = \sqrt[8]{2}e^{(7\pi/16 + k\pi )i}\]

\spart

\[\left(\sqrt{-i}\right)^{1/3} = \left(1e^{(\pi+2k\pi)i}\right)^{1/6} = e^{(\pi/6+k\pi/3)i}\]
\end{problem}

\begin{problem}[11]
En este ejercicio, consideramos sólo el \textit{valor principal de la raíz cuadrada}, definido como
\[\sqrt[(p)]{z}=\sqrt{r}\left(\cos\frac{\phi}{2}+i\sin\frac{\phi}{2}\right)\]
cuando $z=r(\cos\phi+i\sin\phi)$ con $-\pi < \phi < \pi$. Claramente, $\left( \sqrt[(p)]{z} \right)^2=z$
\ppart Demuestra que las soluciones en $\cplex$ de la ecuación $az^2+bz+c=0$, con $a\neq 0$, son
\[z=\frac{-b\pm \sqrt[(p)]{b^2-4ac}}{2a}\]
\ppart
Calcule
\[\sqrt[(p)]{\left(\sqrt[(p)]{i}\right)^5} \text { y } \sqrt[(p)]{1+\sqrt[(p)]{i}}\]
\solution

\spart
Para resolver este apartado basta son sustituir la fórmula que nos dan para la $z$ en la ecuación dada y comprobar que, efectivamente, la ecuación se verifica.

\spart
La raíz principal puede sonar a algo exótico pero consiste, simplemente, en tomar la raíz del número dado y, en lugar de considerar los varios ángulos posibles, tomamos el menor posible (siempre positivo).

A efectos legales esto nos hace ahorrarnos el típico $+2k\pi$. Veamos a modo de ejemplo los radicales que nos pide calcular el enunciado
\[\sqrt[(p)]{\left( \sqrt[(p)]{i}\right)^5} = \sqrt[(p)]{e^{-\frac{3\pi}{4}i}} = e^{-\frac{3\pi}{8}}i\]

\[\sqrt[(p)]{1+\sqrt[(p)]{i}} = \sqrt[(p)]{1+\sqrt[(p)]{i}} = \sqrt[(p)]{1+\sqrt{2}/2+i\sqrt{2}/2} = \sqrt{2+\sqrt{2}}\left( \cos\left(\frac{π}{4}\right)+i\sin\left(\frac{π}{4}\right)\right)\]
\end{problem}

\begin{problem}[12]
Resuelve las siguientes ecuaciones:
\ppart
\[(z+1)^4+i=0\]
\ppart
\[Re(z^2+5)=0\]
\ppart
\[Re(z+5)=Im(z-i)\]
\solution

\spart
Despejando como hemos hecho siempre tenemos que
\[z=\sqrt[4]{-i}-1 = e^{\pi/4+\pi k / 2} -1 \]

\spart
Considerando $z=x+iy$ tenemos que $z^2=x^2-y^2+2xyi$ con lo que llegamos a
\[x^2-y^2 = -5\]
que nos da una hipérbola

\spart
Considerando $z=x+iy$ tenemos
\[Re(z+5)=Im(z-i) \iff x+5=y-1\]
obteniendo como resultado una recta.

\end{problem}

\begin{problem}[13]
\ppart
Demuestra que si $α$ es solución de $z^n=μ$ (con $μ\in\cplex$ fijo), entonces todas las soluciones son $αω_i$ con $i=0,1,...,n-1$ donde $ω_i$ son las raíces n-ésimas de la unidad
\ppart
Encuentre razonadamente las soluciones de $z^6-8=0$
\solution
\doneby{Pedro}

\spart
Es sencillo e ver que los números de la forma $αω_i$ son soluciones, puesto que
\[(αω_1)^n = α^n ω_i^n= μ \cdot 1 = μ\]
Cualquier otra hipotética solución deberá cumplir que al elevarla a $n$ obtengamos μ, por lo que deberá ser $α$ multiplicado por algo que, al elevarlo a $n$ nos de 1. Es decir, no habrá más posibilidades que las indicadas

\spart
Siguiendo lo indicado en el apartado anterior las soluciones serán de la forma:
\[z=\sqrt[6]{8}ω_i \text{ con } i=0,1,2,3,4,5 \text{ y } ω \text{ raíz sexta de la unidad}\]

\end{problem}

\begin{problem}[14]
¿Cuándo son colineales tres puntos $z_1,z_2,z_3$ distintos dos a dos?
\solution
Para verlo hacemos como en bachillerato con los reales: escribimos la recta que pasa por dos de esos puntos y forzamos a que el tercero se contenga en dicha recta.

La recta que pasa por $z_1$ y $z_2$ sería:
\[L=\{z_1+t(z_2-z_1) t \in \real \}\]
Si $z_3 \in L \implies \exists t \tq z_3=z_1 + t(z_2-z_1)$ es decir:
\[t = \frac{z_3-z_1}{z_2-z_1}\in \real\]
para que sea real ese resultado necesitamos que el numerador y el denominador tengan el mismo argumento.

Intuitivamente representa que uniendo $z_3$ con $z_1$ obtenemos la misma recta que uniendo $z_2$ con $z_1$
\end{problem}

\begin{problem}[15]

\ppart
Compruebe que la ecuación
\[Re(az+b) = 0 \text{ con } a,b\in \cplex, \ a \neq 0\]
define una recta en el plano y que, recíprocamente, cada recta viene descrita por una ecuación de este tipo

\ppart
Encuentre los números $a,b$ para que la recta pase por dos puntos dados $z_1, z_2 \in \cplex$

\ppart
Demuestre que las rectas determinadas por las ecuaciones $Re(az+b)=0$ y $Re(cz+d)=0$ respectivamente, son perpendiculares si y sólo si $Re(a\bar{c})=0$

\ppart
Demuestre que la ecuación de una recta que pasa por dos puntos dados $z_1$ y $z_2$ puede escribirse de la forma
\[ \left| \begin{array}{ccc}
z  & \bar{z} & 1 \\
z_1 & \bar{z_1}&  1 \\
z_2 & \bar{z_2} & 1 \end{array} \right| = 0\]

\solution
\doneby{Pedro}

\spart

Siendo cada número complejo $x\in \cplex = x_r+ix_i$, la ecuación que nos dan se traduce en
\[a_rz_r-a_iz_i+b_r=0 \equiv z_i = \frac{a_rz_r+b_r}{a_i}\]
siendo $x_i = y$ y $z_r = x$ obtenemos la ecuación de una recta en el plano.

\spart

Basta con sustituir en la ecuación los valores $z_1=z_{1r}+iz_{1i}$ y $z_2=z_{2r}+iz_{2i}$ y obtenemos un sistema de 4 ecuaciones de 4 incógnitas que podremos resolver.

\textcolor{green}{Plan b como alternativa distinta que me convence más}

Utilizando la ecuación $z_i = \displaystyle\frac{a_rz_r+b_r}{a_i}$ escribimos un sistema de 2 ecuaciones con 3 incógnitas, que tiene sentido que salga un sistema indeterminado con infinitas soluciones (puedo coger como $b∈ℂ$ cualquiera que pertenezca a la recta)

\[
\begin{array}{cc}
Im(z_2) = \frac{a_r Re(z_2) - b_i}{a_i}\\
Im(z_1) = \frac{a_r Re(z_1) - b_i}{a_i}
\end{array}
\]

\spart

Basándonos en el apartado a), podemos ver que las pendientes de esas rectas son, respectivamente, $\frac{a_r}{a_i}$ y $\frac{c_r}{c_i}$.

Para que sean perpendiculares, debemos tener
\[\frac{a_r}{a_i}= - \frac{c_i}{c_r} \implies a_rc_r = -a_ic_i \implies Re\left((a_r+ia_i)(c_r-ic_i)\right) = 0\]

\spart
%TODO por hacer

\end{problem}

\begin{problem}[16]
Describa el conjunto del plano complejo determinado por las siguientes relaciones
\ppart
\[|z-2|-|z+2| > 3\]
\ppart
\[Re(z)+Im(z) < 1\]
\ppart
\[|2z|>|1+z^2|\]

\solution
\doneby{Pedro}

\spart
Si tuviéramos una igualdad, estaríamos hablando de los puntos del plano cuya diferencia de distancias a los puntos $(2,0)$ y $(-2,0)$ es constante. Es decir, tendríamos una hipérbola.

Al tener una desigualdad, estamos cogiendo aquellos puntos situados a la derecha de la hipérbola.

\spart

Esta ecuación representa aquellos puntos del plano que quedan a la izquierda de la recta $y=-x+1$.

\spart



\end{problem}

\begin{problem}[17]
Determine las ecuaciones complejas:
\ppart de la parábola con foco i y directriz $Im(z)=-1$
\ppart de la elipse con focos $\pm 1$ que pasa por $i$
\ppart de la hipérbola con focos $\pm 1$ que pasa por $i+1$

\solution

Este ejercicio es bastante semejante a los apartados b) y c) del ejercicio 1.10

\spart
Recordemos que una parábola se definía a partir del foco y la directriz como el conjunto de puntos del plano que equidistaban de ellos.

Para escribir la ecuación, simplemente aplicamos la definición y vemos a que ecuación nos lleva.

Sea un punto cualquiera $z=x+iy$ su distancia al foco $i$ sería $|z-i|$ mientras que la distancia a la directriz sería $1+Re(z)$

Igualando tenemos la ecuación buscada
\[|z-i|=Re(z)+1\]

\spart
Recordemos que, por definición, la elipse es el conjunto de puntos del plano con suma de distancias a los focos constante.

Conocemos los focos lo que nos lleva a:
\[|z-1|+|z+1|=cte\]

Para determinar la constante nos basamos en que pasa por $i$, lo que nos lleva a concluir que la constante es $2\sqrt{2}$ es decir, nos queda la ecuación
\[|z-1|+|z+1|=2\sqrt{2}\]

\spart
La hipérbola tenía definición similar a la de la elipse salvo que en este caso considerábamos constante la diferencia de distancias a los pocos en lugar de la suma.

De aquí obtenemos que la ecuación buscada será de la forma
\[|z-1|-|z+1|=cte\]
Sabiendo que pasa por el punto $i+1$ podemos calcular la constante
%TODO completar

\end{problem}

\begin{problem}[18]
Esboce el conjunto de puntos $z \in \cplex$ que satisfacen
\ppart \[Re\left( \frac{z}{1+i}\right) = 0\]
\ppart \[|z^2-4z+4| = 4\]
\ppart \[|z^2-2z-1|=1\]

\solution

\spart
Vamos a jugar un poco con el número que nos dan. Siendo $z=x+iy$ tenemos
\[\frac{x+iz}{1+i}\cdot\frac{1-i}{1-i} = \frac{x-y+i(y-x)}{2} \implies Re\left( \frac{z}{1+i}\right) = \frac{x-y}{2}\]

Por tanto, obtenemos la recta $y=x$, la bisectriz del primer cuadrante.

\spart
\[z^2-4z+4| = 4 \iff |z-2|^2 = 4 \]
con lo que tenemos una circunferencia

\spart
Se deja como ejercicio para el lector, que deberá pasar a coordenadas polares con el objetivo de poder esbozar el dibujo pedido.

\textbf{consejo:} Acordarnos de la Lemniscata
\end{problem}

\begin{problem}[19]
\ppart Sea $a \in \cplex$ un número fijo. Encuentre el máximo de $|z^{12}-a|$ cuando $z$ es cualquier número complejo tal que $|z|\leq 1$
\ppart Halle razonadamente el supremo y el ínfimo del siguiente conjunto de números reales
\[\{Re(iz^4+1) \tq |z|<\sqrt{2}\}\]

\solution
\doneby{Pedro}

\spart
Si abordamos el ejercicio como un problema en $\real^2$, lo que tenemos es que nos dan un punto cualquiera del plano y debemos buscar el punto del círculo unidad que más diste de él.

Para ello basta con unir el punto dado con el centro y prolongar el segmento que los une hasta que corte a la circunferencia.

Es decir, dado el punto $a=r\left(\cos(\theta)+i\sin(\theta)\right)$, el punto $b$ del círculo unidad más alejado de $a$ es $b=1\left(\cos(\theta +2π)+i\sin(\theta +2π)\right)$

(Posiblemente habría que tener cuidado si el punto $a$ no pertenece al primer cuadrante)

Para calcular ahora el número $z$ pedido, basta con tomar z=$\sqrt[12]{b}$

\spart
Siendo $z=x+iy$ tenemos
\[Re(iz^4+1)=x^4+y^4-6x^2y^2+1\]

Como aprendimos a hacer en Cálculo I, tenemos que calcular el gradiente de esa función e igualarlo a 0 y posteriormente estudiar el comportamiento de la función en la frontera del conjunto que estamos estudiando.

Vamos con el gradiente
\[\nabla \left(Re(iz^4+1)\right) = \left(4x^3-12xy^2, 4y^3-12x^2y\right)\]
al igualarlo a 0 tenemos que los puntos extremos son: el origen y los puntos que satisfacen a la vez las ecuaciones:
\[x^2-3y^2=0\]
\[y^2-3x^2=0\]
que viene a no decir nada y a dejarnos igualmente restringidos al origen

Para estudiar el comportamiento de la función en la frontera del conjunto estudiado tenemos
\[|z| = \sqrt{2} \implies x^2+y^2 =2 \implies y=\sqrt{2-x^2} \]
sustituyendo en la fórmula estudiada tenemos:
\[x^4+4+x^4-4x^2-12x^2+6x^4 +1 =0\]
derivando y simplificando tenemos
\[32x^3-32x=0 \implies x^2-1=0 \implies x=\pm 1 \implies y = \pm \sqrt{3}\]
Los puntos de máximo y mínimo son $(\pm 1, \pm \sqrt{3})$

\end{problem}

\begin{problem}[20]
Describa geométricamente el conjunto de los puntos $w \in \cplex$ que se escriben en la forma $w=iz^2+1$, para $z=x+iy$ con $x>0, y>0, \ x^2+y^2<1$.

\solution

Operando, tenemos que estamos trabajando con el conjunto de números complejos de la forma:
\[w=i(x^2-y^2)-2xy+1\]


\end{problem}

\begin{problem}[21]
Demuestre que, dados $a,c \in \cplex$, la condición necesaria y suficiente para que exista $z \in \cplex$ que verifique $|z+a|+|z-a|=2|c|$ es que sea $|a|\leq|c|$

\textbf{Ayuda:} Si $λ>0$, el conjunto $ \{z \in \cplex \tq |z+a|+|z-a|=2λ\}$ es una elipse si $λ > |a|$, un segmento si $λ=|a|$ y el conjunto vacío si $λ<|a|$

\solution

Basándonos en la indicación dada es obvio que $|a|\leq|c|$ es condición necesaria y suficiente para que podamos hablar de la solución de la ecuación ya que, en caso contrario, tendríamos el vacío.

Si tenemos $|a|\leq|c|$ el conjunto de puntos solución de la ecuación constituirán una recta o una elipse (según el caso) pero en ambos casos son conjuntos válidos que nos dan solución para la ecuación.
\end{problem}

\begin{problem}[22]
He aquí algunas interpretaciones geométricas de ciertas operaciones con números complejos.

\ppart Si $z=x+iy \in \cplex$ sea $α(z)$ el vector de tres dimensiones $(x,y,0)$. Verifique que para cada $z,w \in \cplex$ se cumple que $α(z)α(w)=Re(z\bar{w})$ y $α(z)\times α(w)=(0,0,Im(\bar{z}w))$

\ppart Si $0,z,w$ son los vértices de un triángulo $T$, compruebe que $Area(T)=\frac{1}{2}|Im(\bar{z}w)|$

\ppart
Si $z_1, z_2,...z_n$ son los vértices de un polígono $P$ que contiene a 0 en su interior, demuestra que $Area(P)=\frac{1}{2}\left|Im\left( \sum_{j=1}^n \bar{z}_jz_{j+1}\right)\right|$, donde se toma $z_{n+1}=z_1$

\solution
\doneby{Pedro}

\spart
\[α(z)α(w)=z_xw_x+z_yw_y\]
Por otro lado
\[Re(zw)=Re\left(z_xw_x+z_yw_y+i(z_xw_y-z_yw_x)\right) = z_xw_x+z_yw_y \]

Si calculamos el producto vectorial que se nos pide, tenemos que
\[α(z)\times α(w) = (0,0,-z_xw_y+z_yw_x)\]
que podemos comprobar que coincide con la parte imaginaria de
\[Im(\bar{z}w) = Im \left( z_xw_x-z_yw_y+i(-z_xw_y+z_yw_x)\right)\]

\spart

Si ya hemos visto que el producto vectorial coincide con la parte imaginaria, es trivial ver que un medio de esa parte imaginaria nos dará el área del triángulo, pues el producto vectorial nos da el área del paralelogramo generado por los dos vectores.

\spart
Con imagen del producto que se nos da (tras multiplicar por 1/2) obtenemos el área del triángulo formado por los dos puntos dados y el origen.

Puesto que el origen se contiene en la figura cuyo área estamos calculando, al hacer esta operación con todos los vértices tenemos el área de la figura.

\end{problem}

\begin{problem}
Demuestre que la condición necesaria y suficiente para que $\{z_1, z_2, z_3\}$ sea el conjunto de los vértices de un triángulo equilátero es que
\[z_1z_2+z_2z_3+z_3z_1=z_1^2+z_2^2+z_3^2\]
\textbf{Ayuda:} Considere el triángulo $\{z_2, z_3,z_1\}$

\solution

\end{problem}


%%%%%%%%%%%%%%%%%%%%%%%%%%%%%%%%%%%%%%%%%%%%%%%%%%%%%%%%%%%%%%%%%%%%%%%%
%%%%%%%%%%%%%%%%%%%%%%%%%%%%%%%%%%%%%%%%%%%%%%%%%%%%%%%%%%%%%%%%%%%%%%%%
%%                                                                    %%
%%                            HOJA 2                                  %%
%%                                                                    %%
%%%%%%%%%%%%%%%%%%%%%%%%%%%%%%%%%%%%%%%%%%%%%%%%%%%%%%%%%%%%%%%%%%%%%%%%
%%%%%%%%%%%%%%%%%%%%%%%%%%%%%%%%%%%%%%%%%%%%%%%%%%%%%%%%%%%%%%%%%%%%%%%%
\newpage
\section{Hoja 2}
\begin{problem}[1]
(\textit{Esfera de Riemann}) Se considera $\widehat{\cplex} = \cplex \cup \{\infty\}$ y se definen los entornos de $\infty$ como aquellos que contienen un conjunto de la forma $\{z \in \cplex \tq |z|>R\}$ para algún $R > 0$

Con estos entornos $z_n \to \infty$ quiere decir que
\[\forall R > 0 \exists N \tq |z_n| > R \ \forall n > N\]

De manera similar se definen $\lim_{z \to b} f(z)= \infty$ y $\lim_{z \to \infty}f(z)=\infty$.

Sean $\mathbb{S}= \{p \in \real^3 : p_1^2+p_2^2+p_3^2\}$ y consideramos la proyección estereográfica:
\[\appl{\pi}{\mathbb{S}}{\widehat{\cplex}}, \pi(p) = \left\{
\begin{array}{lcc}
    \frac{(p_1+ip_2)}{1-p_3} & si & p \neq N = (0,0,1) \\
 \\ \infty & si & p = N
 \end{array} \right.\]

 \ppart
 Compruebe que
 \[\pi^{-1}(z)=\left( \frac{2Re(z)}{|z|^2+1}, \frac{2Im(z)}{|z|^2+1}, \frac{|z|^2-1}{|z|^2+1}\right)\]

 \ppart
 Sea $\rho(z,w)$= distancia (en $\real^3$) entre $\pi^{-1}(z)$ y $\pi^{-1}(w)$ para $z,w \in \widehat{\cplex}$. Entonces:
 \[z_n \to z \text{ en } \widehat{\cplex} \implies \rho(z_n,z) \to 0\]

 \ppart
 Demuestre que
 \[\lim_{n \to \infty}\frac{z^n}{n} = \infty \text{ si } |z| > 1\]

\solution
\doneby{Edu}

\spart

Sea $z=0$ el plano sobre el que haremos la proyección estereográfica.

Sea
\[r=\left\{
\begin{pmatrix}x\\y\\z\end{pmatrix} =  \begin{pmatrix}x_0\\y_0\\0\end{pmatrix} t + (1-t) \begin{pmatrix}0\\0\\1\end{pmatrix} = \begin{pmatrix}x_0t\\y_0t\\1-t\end{pmatrix}
\right\}\]
la recta que pasa por el polo norte y el punto $(x_0,y_0,0)$.

Calculemos la intersección con $\mathbb{S}$:
\begin{gather*}
(x_0t)^2 + (y_0t)^2 + (1-t)^2 = 1 \\
x_0^2 t^2 + y_0^2 t^2 + 1 - 2t + t^2 = 1 \\
x_0^2 t^2 + y_0^2 t^2 - 2t + t^2 = 0 \\
t ( x_0^2 t + y_0^2 t - 2 + t) = 0\\
\end{gather*}

Luego las soluciones son
\begin{center}
$t = 0$ (polo norte) y\\
$x_0^2 t + y_0^2 t - 2 + t = 0 \iff t = \frac{2}{x_0^2 + y_0^2 + 1}$.
\end{center}

Luego
\[\begin{pmatrix}x\\y\\z\end{pmatrix} = \begin{pmatrix} \frac{2x_0}{x_0^2 + y_0^2 + 1} \\ \frac{2 y_0}{x_0^2 + y_0^2 + 1} \\ 1-\frac{2}{x_0^2 + y_0^2 + 1} \end{pmatrix} = \begin{pmatrix} \frac{2x_0}{x_0^2 + y_0^2 + 1} \\ \frac{2 y_0}{x_0^2 + y_0^2 + 1} \\ \frac{x_0^2 + y_0^2 - 1}{x_0^2 + y_0^2 + 1} \end{pmatrix}\]

Tomando $w = x_0+iy_0 \implies |w|^2 = x_0^2+y_0^2$, $Re(w) = x_0$, $Im(w) = y_0$:
\[ \begin{pmatrix}x\\y\\z\end{pmatrix} = \begin{pmatrix} \frac{2Re(w)}{|w|^2 + 1} \\ \frac{2 Im(w)}{|w|^2 + 1} \\ \frac{|w|^2 - 1}{|w|^2 + 1} \end{pmatrix}\]
\qed

\textcolor{blue}{Hecho por Pedro}\\
\spart

Ya sabemos que si una sucesión de complejos converge a un complejo dado es por que tanto la parte real como la imaginaria lo hacen por separado. Así mismo, eso implica directamente que la sucesión de los módulos converge al módulo del límite.

Una vez visto esto es obvio ver que la implicación dada es correcta.

\spart

La forma más sencilla de ver esto es trabajando sobre la inversa de la proyección estereográfica. Puesto que es un homeomorfismo (ya lo estudiamos en Análisis) sabemos que $π^{-1}$ es continua por lo que la imagen límite de una sucesión es el límite de las imágenes.

Nos llevamos por tanto los puntos $z_n=\frac{z^n}{n}$ a la esfera y vemos que van creciendo en módulo. Cuando el módulo tiende a infinito, $π^{-1}$ tiene al $(0,0,1)$

\textcolor{blue}{Explicación guarrísima. Trataré de mejorarla.}

\end{problem}

\begin{problem}[2]
\ppart
Demuestre que, mediante la proyección estereográfica, las circunferencias sobre la esfera se transforman en circunferencias o rectas del plano. ¿Cuáles son las circunferencias sobre la esfera que se transforman en rectas?

\ppart
¿Qué corresponde en la esfera de Riemann a una familia de rectas paralelas del plano?

\ppart
Halle, en la esfera de Riemann, las imágenes de los conjuntos definidos por las siguientes desigualdades:
\begin{enumerate}
\item $Im(z) > 0$
\item $Re(z) < 1$
\item $|z| < 1$
\item $|z| > 2$
\end{enumerate}

\solution

\textcolor{blue}{La profesora escribió algunas cuentas pero no me han parecido muy útiles ni novedosas. Aquí doy la idea del ejercicio.}

\spart

Si la circunferencia no pasa por el polo norte, al hacer la proyección estereográfica estamos construyendo un cono e intersecando el mismo con el plano por lo que obtendremos una circunferencia.

No obstante, si hacemos esto mismo con una circunferencia que contiene al polo norte, lo que estamos construyendo es un plano e intersecando dos planos, por lo que obtendremos una recta.

Se transforman en circunferencias en el plano aquellas en la esfera que no pasan por el polo norte.

\spart
Dada una circunferencia que pasa por el polo norte (y cuya imagen es una recta) podemos escribirla como intersección de un plano y la esfera. Todas las circunferencias que podamos obtener haciendo girar el plano sobre una recta horizontal contenida en él y que pasa por el polo norte, nos darán rectas paralelas.

\spart
\begin{enumerate}
\item $Im(z) > 0$
Del cuarto de la esfera que se encuentra en la zona compleja del plano

\item $Re(z) < 1$
De un cuarto de la esfera.

\item $|z| < 1$
Son los puntos de la esfera que se encuentran en la semiesfera inferior (por debajo del plano complejo).

\item $|z| > 2$
Son los puntos de la esfera que se encuentran a una altura mayor que $\frac{1}{5}$
\end{enumerate}

\end{problem}

\begin{problem}[3]
Decida si las sucesiones $z_n= \left(\frac{1-2i}{3}\right)^n, \ w_n = \left( \frac{3-4i}{5}\right)^n$ tienen límite (finito) o no

\solution
Para que tengan límite necesitamos que su módulo converja y para ello necesitamos que este sea menor que 1.

En este caso tenemos:
\[|z_n| = \frac{\sqrt{5}}{3} \implies \lim_{n \to \infty} |z_n|^n = 0 \implies \lim_{n\to\infty} z_n = 0\]

\[|w_n| = \frac{5}{5} \implies \lim_{n \to \infty} |w_n|^n = 1\]

Esto causa que la sucesión no tenga límite, pues tendremos puntos con el mismo módulo pero diferente ángulo por lo que no converge.

\end{problem}

\begin{problem}[4]
Decida razonadamente si las siguientes funciones tienen límite (finito) o no en el punto indicado

\ppart
\[f(x) = \frac{|z|^2}{z} (\text{ para z}\neq0 )\text{ en el punto } z=0\]

\ppart
\[f(z)= \frac{z^3-8i}{z+2i} ( \text{ para z} \neq -2i) \text{ en el punto } z=-2i\]

\solution

Al calcular este tiempo de límites debemos seguir el procedimiento que hacíamos con los reales: probamos a sustituir directamente, nos dará indeterminación y jugamos con el número para evitarla.

\spart
\[\lim_{z \to 0} f(z) = \lim_{z \to 0} \frac{|z|^2}{z} = \lim_{z \to 0}\frac{z \bar{z}}{z}=\lim_{z \to 0} \bar{z} = 0\]

\spart

\[\lim_{z \to -2i} f(z) = \lim_{z \to -2i} \frac{z^3-8i}{z+2i} = \lim_{z \to -2i} z^2-2iz-4 = -12\]
\end{problem}

\begin{problem}[5]
Demuestre las siguientes afirmaciones
\ppart Si $P(z)=a_nz^n+\cdots + a_0$ y $Q(z)=b_mz^m+\cdots b_0$ son polinomios con $a_n \neq 0 \neq b_m$ entonces se tiene
\[\lim_{z \to \infty} \frac{P(z)}{Q(z)} = \left\{
\begin{array}{lcc}
    0& si & n < m \\
    \\ \frac{a_n}{b_m} & si & n=m \\
 \\ \infty & si & n > m
 \end{array} \right.\]

\ppart
No existe $\lim_{z \to \infty}e^z$
\solution
\doneby{Pedro}

\spart
Tenemos que calcular
\[\lim_{z \to \infty}\frac{a_nz^n+\cdots + a_0}{b_mz^m+\cdots b_0}\]
\begin{itemize}
\item Si $n<m$
\[\lim_{z \to \infty}\frac{a_nz^n+\cdots + a_0}{b_mz^m+\cdots b_0} = \lim_{z \to \infty}\frac{z^n(a_n+\frac{a_{n-1}}{z}\cdots + \frac{a_0}{z^n})}{z^n(b_mz^{m-n}+\cdots \frac{b_0}{z^n}}=\lim_{z \to \infty}\frac{(a_n+\frac{a_{n-1}}{z}\cdots + \frac{a_0}{z^n})}{(b_mz^{m-n}+\cdots \frac{b_0}{z^n})} = \]
\[=\lim_{z \to \infty}\frac{a_0}{b_mz^n} = 0\]

\item
Si $n=m$
\[\lim_{z \to \infty}\frac{a_nz^n+\cdots + a_0}{b_mz^m+\cdots b_0} = \lim_{z \to \infty}\frac{z^n(a_n+\frac{a_{n-1}}{z}\cdots + \frac{a_0}{z^n})}{z^n(b_mz^{m-n}+\cdots \frac{b_0}{z^n}}=\lim_{z \to \infty}\frac{(a_n+\frac{a_{n-1}}{z}\cdots + \frac{a_0}{z^n})}{(b_m+\cdots \frac{b_0}{z^n})} = \]
\[=\lim_{z \to \infty}\frac{a_0}{b_m} = \frac{a_n}{b_m}\]

\item Si $n>m$
\[\lim_{z \to \infty}\frac{a_nz^n+\cdots + a_0}{b_mz^m+\cdots b_0} = \lim_{z \to \infty}\frac{z^m(a_nz^{n-m}+\cdots + \frac{a_0}{z^m})}{z^m(b_m+\cdots \frac{b_0}{z^m})}=\lim_{z \to \infty}\frac{(a_n+\frac{a_{n-1}}{z}\cdots + \frac{a_0}{z^n})}{(b_mz^{m-n}+\cdots \frac{b_0}{z^n})} = \]
\[=\lim_{z \to \infty}\frac{a_0z^n}{b_m} = \infty\]
\end{itemize}

\spart

Debemos fijarnos en que
\[e^z=e^{x+iy}=e^xe^{iy}=e^x\left(\cos(y)+i\sin(y)\right)\]
cuando $z$ tiende a infinito, así lo hacen su parte real y su parte imaginaria.

Podemos observar que el módulo del número complejo aquí representado crece hasta infinito y su argumento oscila constantemente de modo que no tiene límite.

\end{problem}

\begin{problem}[6]
Halle los puntos de continuidad de las funciones:
\ppart
\[f(z)=\left\{
\begin{array}{lcc}
    \frac{z^4-1}{z-i}& si & z \neq i \\
 \\ 4i & si & z=i
 \end{array} \right.\]
\ppart
\[g(z)=\left\{
\begin{array}{lcc}
    z & si & |z| \leq 1 \\
 \\ |z|^2 & si & |z| > 1
 \end{array} \right.\]

 \solution
\spart
El único punto con posibles problemas y que deberíamos estudiar es el $z=i$. Vamos a estudiar cuánto vale el límite en ese punto para ver si la función es continua o no:
\[\lim_{z \to i} \frac{z^4 -1}{z-i}=\lim_{z \to i}\frac{(z^2-1)(z-i)(z+i)}{z-i} = \lim_{z \to i} (z^2-1)(z+i) = -4i\]

Al no coincidir con el valor de la función en ese punto, podemos concluir que la función no es continua en ese punto.

\spart
El único lugar donde podemos tener problemas es en los puntos con $[z|=1$.

Para hacernos una idea de lo que podemos esperar de este límite, vamos a observar el caso concreto de $z=e^{iα}$ vemos que
\[\lim_{z \to e^{iα}} g(z) =\left\{
\begin{array}{lcc}
    e^iα & si & |z| \leq 1 \\
 \\ 1 & si & |z| > 1
 \end{array} \right. \]

 Por lo general, vemos que esta función no es continua en los puntos con módulo igual a 1 salvo en el punto $z=1$.

 En general es bastante sencillo ver que esta función no es continua, puesto que todos los puntos del círculo unidad se quedan fijos y los demás van a la recta real según su módulo.

\end{problem}


\begin{problem}[7]
¿Dónde son holomorfas las siguientes funciones?
\ppart $f(x,y)=x^2-y^2+ixy$
\ppart $f(z)=g(\bar{z})$, donde $g$ es holomorfa en $\Omega$
\ppart $f(z)=\overline{g(z)}$, donde $g$ es holomorfa en $\Omega$
\ppart $f(z)=\overline{g(\bar{z})}$, donde $g$ es holomorfa en $\Omega$
\ppart $f(z)=|g(z)|$, donde $g$ es holomorfa en $\Omega$

\textbf{Ayuda:} en los apartados b)-e) basta con usar la definición de derivada.

\solution

Para ver si son holomorfas las funciones, vamos a comprobar si se cumplen las ecuaciones de Cauchy-Riemann:
\[\partial_x f = -i \partial_y f\]
\spart
\[2x+iy=-i\left( -2y +ix\right) \iff 2x+iy = -2yi+x \iff (x,y)=(0,0)\]
Con esto no nos basta para garantizar que la función sea holomorfa en ese punto pero, puesto que tanto la parte real como la imaginaria de $f$ son diferenciables ya sí podemos garantizar que la función es holomorfa en el origen.

\textit{En algunos libros podremos ver que esta función no es considerada holomorfa, puesto que sólo cumple la propiedad en un único punto y, tal y como se hace en variable real, una función sería diferenciable en un entorno del punto, no en un único punto. En general no nos encontraremos con este tipo de funciones en este curso.}

\spart
Tanto en este apartado como en el \textbf{d)}, para que estén bien definidas las $f(z)$ necesitamos que si un punto $z\in Ω$ entonces $\bar{z} \in Ω$, es decir, el conjunto es simétrico con respecto al eje imaginario.

Vamos a considerar $g=u+iv$ y $f=U+iV$ con
\[U(x,y)=u(x,-y)\]
\[V(x,y)=v(x,-y)\]

%TODO completar estas cuentas por que me he liado mientras copiaba
Una vez visto esto, podemos derivar:
\[U_x = u_x; \;\;\; U_y=-u_y; \;\;\; V_x=v_x; \; \; \; V_y=-v_y;\]
Ahora estamos en condiciones de comprobar si se satisfacen las ecuaciones de Cauchy-Riemann:
\[U_x+iV_x=-i\left(U_y+iV_y\right) \iff u_x = -v_y\; \& \; v_x=u_y\]
pero, por ser $g$ Cauchy-Riemann sabemos que cumple las ecuaciones:
\[u_x+iv_x=-i(u_y+iv_y) \implies u_x=v_y\\; \& \; v_x=-u_y\]

y ambas condiciones sólo se darán en caso de que todas las derivadas sean iguales a 0.

\textbf{Aplicando la definición de derivada}
\[g \text{ holomorfa en }Ω \iff \frac{\partial g}{\partial \bar{w}}(w)=0\ \forall w \in Ω \iff \frac{\partial g}{\partial z}(\bar{z})=0 \ \forall z \in Ω \iff \frac{\partial f}{\partial z}(z)=0 \ \forall z \in Ω\]

Es decir, nos queda que $g$ es holomorfa en Ω si y sólo si $f$ es anti-holomorfa en Ω

\spart
\doneby{Pedro}

Si consideramos $g=u+iv$ podemos ver que $f=u-iv$.

Por ser $g$ holomorfa sabemos que
\[g_x+ig_y = 0 \implies u_x+iv_x+iu_y-v_y = 0\implies u_x+iv_x+iu_y-v_y = 0 \implies f_y+if_x=0 \implies\]
\[f \text{ es antiholomorfa }\]


\spart
\textcolor{red}{Hecho por Rual. No fiarse al 100\%}

Sabemos que $f(z) = \overline{g(\overline{z})}$. Por otro lado $g(z) = u(x,y)+iv(x,y)$ con $u,v$ cumpliendo las ecuaciones de Cauchy-Riemann.


Vemos que $$f(z) = \overline{g(\overline{z})} = u(x,-y)-iv(x,-y)$$ luego podemos definir

\begin{equation*}
\left\{
\begin{array}{l l}
U(x,y) = u(x,-y)\\
V(x,y) = -v(x,-y)\\
\end{array}
\right.
\end{equation*}
de forma que $$f(z) = U(x,y)+iV(x,y)$$

De esta manera, si $U(x,y), V(x,y)$ cumplen las ecuaciones de Cauchy-Riemann se tendría que $f(z)$ es holomorfa.
Tenemos, usando la reglita de la cadenita
\begin{equation*}
\left\{
\begin{array}{l l}
U_x(x,y) = u_x(x,-y)\\
U_y(x,y) = -u_y(x,-y)\\
\end{array}
\right.
\left\{
\begin{array}{l l}
V_x(x,y) = -v_x(x,-y)\\
V_y(x,y) = v_y(x,-y)\\
\end{array}
\right.
\end{equation*}
Como sabemos que $u,v$ cumplen las ecuacion de Cauchy Riemann (C-R):
\begin{equation*}
\begin{array}{l l}
U_x=u_x\underbrace{=}_{\text{C-R}}v_y=V_y\\
U_y=-u_y\underbrace{=}_{\text{C-R}}v_x=-V_x
\end{array}
\end{equation*}
Luego tenemos
\begin{equation*}
\left\{
\begin{array}{l l}
U_x=V_y\\
U_y=-V_x
\end{array}
\right.
\end{equation*}
que $U,V$ cumplen C-R.

\spart
\doneby{Pedro}

Si consideraos $g=u+iv$, tendríamos que $f=\sqrt{u^2+v^2}$, que se trata de una función real que no es holomorfa, pues no cumple las ecuaciones de Cauchy-Riemann

\end{problem}


\begin{problem}[8]
¿Dónde son holomorfas las siguientes funciones? ¿Cuál es su derivada?
\ppart[k]
$\log(e^z+1)$

\ppart[l]
$\sqrt{e^z+1}$

\ppart[m]
$\sqrt{z^3-1}$
\solution

\spart[k]
Vamos a ver cómo se comporta esta función.

Primero recordamos que la función $e^z$ nos envía rectas verticales en circunferencias y rectas horizontales en rectas que pasan por el origen.

Al tener $e^z+1$ estamos desplazando hacia la derecha (estamos sumando a la parte real) una unidad las imágenes de esas rectas.

Veamos dos formas distintas de calcular el logaritmo que se pide:
\begin{enumerate}
\item \textbf{Por composición}

Sea $\appl{f}{Ω}{\cplex \setminus \{0\}}$, una rama del logaritmo de $f$ es una función continua $\appl{F}{Ω}{\cplex}$ tal que
\[e^{F(z)}=f(z) \ \forall z \in Ω\]

En este caso tenemos $f(z)=e^z+1$.

Para definirla correctamente debemos quitar del dominio aquellos puntos cuya imagen por $f(z)$ es 0 ya que para esos puntos sería imposible encontrar una $F(z)$ con $e^{F(z)}=0$. Estos puntos son aquellos de la forma $(2n+1)πi$ con $n \in \nat$

También podemos ver que si en el dominio hay una curva α tal que su imagen por $f$ de una vuelta al origen (es decir, la curva debe girar en su dominio en torno a una preimagen de 0) tendremos problemas, pues el argumento irá creciendo hacia $2π$ y acabará valiendo $0$.

Para evitar este caso excluimos del dominio los puntos $(x,y)$ con $|x| > πi$, con lo que ya tenemos el dominio buscado donde la función es holomorfa.

\item \textbf{Por definición}

Para definir ahora una rama del logaritmo de $f$ consideramos
\[F(z)=\log(f(z)) = \log|f(z)|+i\underbrace{\text{arg}}_{\text{una rama del argumento de f}}(f(z))\]

Una vez tengamos esta función $F(z)$ bien definida, tendremos que:
\[e^{F(z)}=f(z)\]

Vamos a definir ahora esa rama del argumento. Para ello recordamos que una rama del argumento de $\appl{f}{Ω}{\cplex}$ es una función $\appl{g}{Ω}{\real}$ continua tal que
\[e^{ig(z)}=\frac{f(z)}{|f(z)|} \ \forall z \in Ω\]

En nuestro caso tendríamos
\[\frac{e^z+1}{|e^z+1|}=e^{ig(z)}\]
pero no sabemos cuál sería el dominio Ω, todavía.

\end{enumerate}

Una vez hecho esto, procedemos a calcular su derivada:

%Se trata de un dominio simplemente conexo lo que implica que existe la rama del logaritmo de $f$.

Ya vimos que si $F(z)$ es rama del logaritmo de $f$ se cumple que $e^{F(z)}=f(z)$ y derivando a ambos lados tenemos
\[e^{F(z)}F'(z) = f'(z) \implies F'(z)=\frac{f(z)}{e^{F(z)}}=\frac{f'(z)}{f(z)}\]
que coincide con la derivada esperada del logaritmo.

\obs Si aplicásemos el segundo procedimiento para encontrar una rama del logaritmo de la función $e^z$, que sabemos es la identidad, obtendríamos que la identidad sólo podría definirse en una parte de los complejos.

La explicación de esto es que, con el este último procedimiento estamos forzando a que exista esa rama como una composición. Al pedir más condiciones, reducimos el conjunto de funciones válidas.

\spart[l]
Una vez que tenemos la rama $F(z)$ del logaritmo de $e^z+1$ (definida en Ω), tenemos también la función
\[G(z)=e^{\frac{1}{2}F(z)} \text{ con } \appl{G}{Ω}{\cplex}\]
que es una rama de la raíz cuadrada de $f(z)=e^z+1$

\spart[m]
Vamos a buscar primero $F(z)$, una rama del logaritmo de $f(z)=z^3-1$.

Para ello primero eliminamos del dominio aquellos puntos que nos llevan al 0. Es decir, quitamos del dominio las tres raíces cúbicas del la unidad.

Nuevamente, el problema nos surje al tener curvas α que den vueltas en torno a alguna de esas raíces. Para evitar que se produzca esto quitamos del dominio tres segmentos infinitos cualesquiera, cada uno de ellos con inicio en una de las raíces cúbicas del aunidad.


\end{problem}

\begin{problem}[9]
Sea $\appl{T}{\real^2}{\real^2}$ dada por
\[T(x,y) = \left(u(x,y), v(x,y)\right)\]
Definimos la derivada de $T$ en la dirección $\overrightarrow{w}=(a,b)$ como:

\[D_{\overrightarrow{w}} T = \lim_{t \to 0} \frac{T(x+ta, y+tb) - T(x,y)}{t}\]
Observe que
\[D_{\overrightarrow{w}} T = \left(D_{\overrightarrow{w}} u, D_{\overrightarrow{w}} v\right)\]

Dada la función compleja $f(z)=u(x,y)+iv(xy), \ z=x+iy$, demuestre que si $f$ es holomorfa, entonces
\[D_{\overrightarrow{w}} f = f'(z)w \text{ donde } w = a+ib\]

\solution
\doneby{Pedro}

Por ser $f$ holomorfa sabemos que
\[\lim_{z\to z_0} \frac{f(z)-f(z_0)}{z-z_0}= f'(z_0)\]

Si consideramos $\appl{f}{\real^2}{\real^2}$, dado $z=x+yi$, tendríamos $f(x,y)=\left(Re(x,y), Im(x,y)\right)$ y el límite de la definición de función holomorfa nos da lugar a dos límites:
\[\lim_{z \to z_0} \frac{Re(x,y)-Re(x_0,y_0)}{x-x_0}=Re'(x,y)\]
\[\lim_{z \to z_0} \frac{Im(x,y)-Im(x_0,y_0)}{y-y_0}=Im'(x,y)\]

Calculamos ahora
\[D_{\overrightarrow{w}} f = \lim_{t \to 0} \frac{f(x+ta, y+tb) - f(x,y)}{t} =\]
\[= \lim_{t \to 0} \frac{Re(x+ta, y+tb)+iIm(x+ta,y+tb)-Re(x,y)-iIm(x,y)}{t} \]

de donde podemos separar dos límites:
\[\lim_{t \to 0} \frac{Re(x+at,y+bt) - Re(x,y)}{t} = Re'(x,y)\]
\[\lim_{t \to 0} \frac{Im(x+at,y+bt) - Im(x,y)}{t} = Im'(x,y)\]

La última igualdad de cada linea viene del hecho de que la función $f$ sea holomorfa.

\end{problem}

\begin{problem}[10]
Sea $f$ una función holomorfa en un dominio $Ω \subset \cplex$. Demuestre que si $|f|$ es constante en Ω, entonces $f$ es constante.

\solution
\doneby{Pedro}

La única posibilidad para que $|f|$ sea constante y no lo sea $f$ es que $f$ alterne de valor entre $|f|$ y $-|f|$.

Salvo que la función sea nula, es obvio ver que en caso de no ser constante, con las condiciones dadas, tampoco será continua.

Sin embargo, ya hemos probado en teoría que si una función es holomorfa es continua, por lo que no puede darse este caso.

\end{problem}

\begin{problem}[11]
Demuestre las siguientes afirmaciones:
\ppart Si $h$ es una fucnión de $\real^2$ en $\real$ de clase $\algb{C}^2$ y $f$ es holomorfa, entonces $\nabla(h\circ f) = (\nabla h \circ f)|f'|^2$

\ppart Si $f$ es holomorfa en un dominio $Ω \subset \cplex$ y $f(z) \neq 0 \forall z \in Ω$, entonces
\[\nabla(|f|)=\frac{|f'|^2}{|f|}\]

\ppart Si $f,g$ son holomorfas en un dominio Ω, y si $|f|+|g|$ es constante en Ω y $f$ y $g$ no se anulan en Ω, entonces $f$ y $g$ son constantes.

\solution

\spart
%\[\nabla (h \circ f)=\left(\partial_{x_1}(h \circ f),\partial_{x_2}(h \circ f)\right)=\left(f_x'\partial_x (h \circ f),f_y'\partial_y (h \circ f)\right)\]

\end{problem}

\begin{problem}[12]
Halle el radio de convergencia de las siguientes series de potencias
\ppart[d] $\displaystyle\sum_{n=0}^{\infty} \cos(in)z^n$
\ppart[f] $\displaystyle\sum_{n=0}^{\infty} (n+a^n)z^n$
\ppart[i] $\displaystyle\sum_{n=0}^{∞} a^{n^2} z^{1+2+...+n}$
\solution

\spart[d]
\[\sum_{n=0}^{\infty} \cos(in)z^n = \sum_{n=0}^{\infty}\frac{1}{2}\left(e^{-n}+e^n \right)z^n\]
Ahora podemos calcular fácilmente el radio de convergencia:
\[R = \lim_{n \to \infty}\left| \frac{\cos(in)}{\cos\left(i(n+1)\right)}\right| = \lim_{n \to \infty}\left| \frac{e^{-n}+e^n}{e^{-(n+1)}+e^{n+1}}\right|=\frac{1}{e}\]

\spart[f]
No podemos simplificar más la fórmula de la serie de modo que procedemos directamente a calcular el radio
\[\lim_{n\to \infty}\left| \frac{n+a^n}{n+1+a^{n+1}}\right|\]
a partir de aquí distinguimos dos casos:
\begin{enumerate}
\item \textbf{ Caso |a| > 1}
\[\lim_{n\to \infty}\left| \frac{n+a^n}{n+1+a^{n+1}}\right| = \lim_{n \to \infty} \left|\frac{\frac{n}{a^n}+1}{\frac{n+1}{a^n}+a} \right| \]

Llegados a este punto, debemos calcular el límite del numerador y del denominador. Vamos a aplicar para ello el lema del Sandwich:
\begin{itemize}
\item \textbf{Numerador}
\[\lim_{n \to \infty} 1-\frac{n}{|a|^n}\leq \lim_{n \to \infty} \left|\frac{n}{a^n}+1 \right| \leq \lim_{n \to \infty} \frac{n}{|a|^n}+1 \implies \lim_{n \to \infty} \left|\frac{n}{a^n}+1 \right|  = 1\]

\item \textbf{Denominador}
\[\lim_{n \to \infty} |a|-\frac{n+1}{|a|^n}\leq \lim_{n \to \infty} \left|\frac{n+1}{a^n}+|a| \right| \leq \lim_{n \to \infty} \frac{n+1}{|a|^n}+|a| \implies \lim_{n \to \infty} \left|\frac{n+1}{a^n}+|a| \right|  = |a|\]
\end{itemize}

Con lo que podemos concluir que
\[\lim_{n\to \infty}\left| \frac{n+a^n}{n+1+a^{n+1}}\right| = \frac{1}{|a|}\]

\item \textbf{Caso |a| $\leq$ 1}
\[\lim_{n\to \infty}\left| \frac{n+a^n}{n+1+a^{n+1}}\right| = \lim_{n \to \infty} \left|\frac{\frac{a^n}{n}+1}{\frac{1}{n}+\frac{a^{n+1}}{n}+1} \right| \]

Como antes, calculamos los límites del numerador y del denominador
\begin{itemize}
\item \textbf{Numerador}
\[\lim_{n\to \infty} 1-\frac{|a|^n}{n} \leq \lim_{n\to \infty}\left| 1+\frac{a^n}{n} \right|  \leq \lim_{n\to \infty}1+\frac{|a|^n}{n} \implies \lim_{n\to \infty} \left| 1+\frac{a^n}{n} \right| = 1\]

\item \textbf{Denominador}
\[\lim_{n\to \infty}1-\frac{1}{n}\frac{|a|^{n+1}}{n} \leq \lim_{n\to \infty}\left| 1 + \frac{1}{n} + \frac{|a|^{n+1}}{n} \right| \leq \lim_{n\to \infty}1 + \frac{1}{n}+\frac{|a|^{n+1}}{n} \implies\]
\[\implies \lim_{n\to \infty}\left| 1 + \frac{1}{n} + \frac{|a|^{n+1}}{n} \right|  = 1\]
\end{itemize}
Por tanto, podemos concluir que, en este caso:
\[\lim_{n\to \infty}\left| \frac{n+a^n}{n+1+a^{n+1}}\right| 1 \]
\end{enumerate}

\spart[i]
\doneby{Dejuan}

En esta serie tenemos $z^1 + z^3 + z^6+...$ y para poder aplicar lo visto en teoría necesitamos $\sum a_n z^n$, asique vamos a transformar esta serie en una de la forma $\sum b_n z^n$ redefiniendo el $a_n$ y el exponente de la $z$.


Para ello, nos damos cuenta que definiendo:

\[b_n = \begin{cases}
1 & si \; ∃k\tq n=\sum_{i=1}^k i\\
0 & otros
\end{cases}
\]

Entonces podríamos escribir: \[
\sum_{n=0}^{∞} a^{n^2} z^{1+2+...+n}=\sum_{n=0}^{∞} b_n · a^{\text{algo}} · z^n
\]

Ese algo será un entero, que si tuvierámos el $k$ que nos da si $b_n$ es 0 o 1, ese algo sería $k^2$. Pero a partir de $n$ podemos sacar $k$


$\sum_{i=1}^k i = n$, y despejamos $k = \displaystyle \frac{1±\sqrt{1+8n}}{2}$. Tomando $c=k^2$ escribimos:

\[
\displaystyle\sum_{n=0}^{∞} a^{n^2} z^{1+2+...+n} = \sum_{n=0}^{∞} b_n·a^{c(n)}\footnote{c depende de n}·z^n
\]

El radio de convergencia será $R = \frac{1}{r_b·r_a}$. Procedemos a calcular $r_a,r_b$.

Como $b_n$ tiene sólo 0's y 1's utilizamos:
\[
r_b = \frac{1}{\lim \sup |b_n|^{\frac{1}{n}}} = 1
\]

Para calcular $r_a$ tenemos que distinguir 2 casos:

\paragraph{$|a| > 1$} no converge ni de blas.

\paragraph{$|a| ≤ 1$}
\[
R=\lim_{n\to ∞}
\frac{
	\displaystyle\frac{
		1+\sqrt{1+8n}
	}{2}
}
{
	\displaystyle\frac{
		1+\sqrt{1+8(n+1)}
	}{2}
} = \text{así a ojo...} = 1
\]

Entonces, el radio de convergecia $R=1$

\end{problem}

\begin{problem}[13]
Supongamos que los radios de convergencia de las series $\sum_{n=0}^{\infty} a_nz^n$ y $\sum_{n=0}^{\infty}b_nz^n$ son iguales a $r_1$ y $r_2$ respectivamente. ¿Qué se puede decir respecto a los radios de convergencia de las series:
\ppart
\[\sum_{n=0}^{\infty} (a_n\pm b_n)z^n\]

\ppart
\[\sum_{n=0}^{\infty}a_nb_nz^n\]

\ppart
\[\sum_{n=0}^{\infty}\frac{a_n}{b_n}z^n\]

\solution

\spart
\[ R = \min\{R_1, R_2\}\]
esto se debe a que podemos escribir la suma como
\[\sum_{n=0}^{\infty}(a_n\pm b_n)z^n = \lim_{n \to \infty} \sum_{i=0}^n (a_i\pm b_i)z^i\]
y ese límite podrá descomponerse en suma de límites cuando ambos límites existan. La forma de garantizar que esto ocurra es tomar el mínimo de los radios de convergencia.

\spart
\[\frac{1}{R}=\limsup_{ n \to \infty}|a_nb_n|^{1/n} = \limsup_{n\to \infty}|a_n|^{1/n}|b_n|^{1/n} \leq\]
\[\leq \limsup_{n\to \infty} |a_n|^{1/n}\cdot \limsup_{n\to \infty} |b_n|^{1/n} = \frac{1}{R_1}\frac{1}{R_2}\]

Por tanto tenemos que $R \geq R_1R_2$ y tendremos la igualdad cuando exista al menos uno de los límites superiores: $\limsup_{n\to \infty} |a_n|^{1/n}$ ó $\limsup_{n\to \infty}|b_n|^{1/n}$.

Para afirmar esto nos hemos basado en el siguiente lema:
\begin{lemma}
Sean $x_n,y_n \geq 0$,
\[\exists \lim_{n \to \infty} x_n \implies \limsup_{n \to \infty }(x_ny_n)=\lim_{n \to \infty}x_n \cdot \limsup_{n \to \infty} y_n\]
\end{lemma}

\spart
\[\sum_{n = 0}^{\infty} \frac{a_n}{b_n}z^n \text{ con } b_n \neq 0\]
Podemos calcular su radio de convergencia de la siguiente forma:
\[\frac{1}{R}=\limsup_{n \to \infty} |a_n|^{1/n} = \limsup_{n \to \infty} \frac{|a_n|^{1/n}}{|b_n|^{1/n}}|b_n|^{1/n} \leq \limsup_{n \to \infty} \frac{|a_n|^{1/n}}{|b_n|^{1/n}}\limsup_{n \to \infty} |b_n|^{1/n}=\frac{1}{R}\cdot \frac{1}{R_2}\]

Así hemos llegado a que $R \leq \frac{R_1}{R_2}$, teniendo la igualdad en caso de que exista el límite de $|b_n|^{1/n}$


\end{problem}

\begin{problem}[14]
Pruebe que para todo $z \in \cplex$ tal que $|z|<1$, se verifican las identidades:
\ppart
\[\frac{1}{1-z}=\sum_{n=0}^{\infty}z^n\]
\ppart
\[\left(\frac{1}{1-z}\right)^2 = \sum_{n=0}^{\infty} nz^{n-1}\]
\solution

\doneby{Pedro}

\spart

\[\sum_{n=0}^{\infty}z^n = \lim_{N \to \infty}\sum_{n=0}^{N}z^n = \lim_{N\to \infty}\frac{1-z^n}{1-z} = \lim_{N \to \infty}\frac{1-|z|^n\left(\cos(α) + i \sin(α)\right)}{1-z} \underbrace{=}_{\text{ por ser } |z|<1} \frac{1}{1-z}\]


\spart

\begin{gather*}
\sum_{n=0}^{\infty} nz^{n-1} = \lim_{N \to \infty}\sum_{n=0}^{N} nz^{n-1} = \lim_{N \to \infty} \frac{d}{dz} (\sum_{n=0}^{N} z^{n}) = \\
= \lim_{N \to \infty} \frac{d}{dz} \left( \frac{1-z^{N+1} }{1-z} \right) = \lim_{N \to \infty} \frac{-(N+1) \cdot z^N \cdot (1-z) - (-1)\cdot(1-z^{N+1})}{(1-z)^2} = \\
= \lim_{N \to \infty} \frac{-(N+1) \cdot z^N + (N+1) \cdot z^{N+1} + 1-z^{N+1}}{(1-z)^2} = \lim_{N \to \infty} \frac{-(N+1) \cdot z^N + N \cdot z^{N+1} + 1}{(1-z)^2} = \\
= \lim_{N \to \infty} \frac{-(N+1) \cdot |z|^N \cdot e^{i\theta} + N\cdot |z|^{N+1} \cdot e^{i\theta} + 1}{(1-z)^2} \underbrace{=}_{\text{ por ser } |z|<1} \frac{1}{(1-z)^2}
\end{gather*}

\end{problem}

\begin{problem}[15]
Desarrolle las siguientes funciones en series de potencias del tipo indicado
\ppart
\[\frac{z}{z^2-5z+6} \text{ y } \frac{z}{(z-1)^2} \text{ en potencias de } z\]
\ppart
\[\frac{2z+3}{z+1} \text{ y } \frac{2z+3}{(z+1)^2} \text{ en potencias de } z-1\]

\solution
\doneby{Pedro}

\spart
\[\frac{z}{z^2-5z+6} = \frac{z}{(z-2)(z-3)} = \frac{A}{z-2}+\frac{B}{z-3}\]
Para calcular estos coeficientes $A$ y $B$, sumamos las dos fracciones de la derecha y damos los valores $z=2$ y $z=3$.

Así obtenemos $A=-2$ y $B=3$ y podemos escribir:
\[\frac{z}{z^2-5z+6}\]

\[\frac{z}{(z-1)^2}= \frac{-2}{z-2}+\frac{3}{z-3} = -\frac{1}{2}\frac{-2}{1-z/2}-\frac{1}{3}\frac{3}{1-z/3} = \sum_{n=0}^{\infty} \left(\frac{z}{2}\right)^n -\frac{1}{2}3\sum_{n=0}^{\infty}\left(\frac{z}{3}\right)^n \]

Para la siguiente función aplicamos el mismo procedimiento aunque con pequeñas variaciones
\[\frac{z}{(z-1)^2}=\frac{A}{(z-1)}+\frac{Bz+C}{(z-1)^2}\]
y procedemos a calcular el valor de las constantes $A$, $B$ y $C$ dando valores. Nos queda el sistema de ecuaciones:
\[\begin{cases}
	1=B+C\\
	A=C \\
	-1=-2A-B+C \\
	\end{cases} \implies \begin{cases}
	1=B+A\\
	A=C \\
	-1=-A-B \\
	\end{cases} \implies \begin{cases}
	1=B+A\\
	A=C \\
	\end{cases}\]
Por comodidad vamos a tomar $A=C=1$, $B=2$ con lo que obtenemos:
\[\frac{z}{(z-1)^2}=\frac{1}{(z-1)}+\frac{2z}{(z-1)^2}+\frac{1}{(z-1)^2} = .... =\]
\[= -\sum_{n=0}^{\infty} z^n+2z\sum_{n=0}^{\infty}z^{2n}+\sum_{n=0}^{\infty}z^{2n}\]

\spart
\[\frac{2z+3}{z+1} = 2 + \frac{1}{z+1}= 2 +\sum_{n=0}^{\infty}(-1)^nz^n\]

\[\frac{2z+3}{(z+1)^2} = \frac{A}{z+1}+\frac{Bz+C}{(z+1)^2}\]
Calculamos $A$, $B$ y $C$ como en el apartado anterior y obtenemos:
\[\begin{cases}
	1=-B+C\\
	3=A+C \\
	5=2A+B+C \\
	\end{cases} \implies \begin{cases}
	1=-B+C\\
	3=A+C \\
	\end{cases} \]
	y vemos que una posible solución sería $A=2$, $B=0$, $C=1$
con lo que obtenemos:
\[\frac{2z+3}{(z+1)^2} = \frac{2}{z+1}+\frac{1}{(z+1)^2} = ... =\]
\[=2\sum_{n=0}^{\infty}(-1)^nz^n+\sum_{n=0}^{\infty}z^2\]
\end{problem}

\begin{problem}[16]
Calcule el radio de convergencia y la suma de
\ppart
\[\sum_{n=0}^{\infty}\frac{z^{2n}}{n!}\]
\ppart
\[\sum_{n=0}^{\infty}n(n-1)z^n\]
\ppart
\[\sum_{n=0}^{\infty}(-1)^n \frac{(z-2πi)^n}{n!}\]
\solution

\spart
\[\sum_{n=0}^{\infty}\frac{z^{2n}}{n!} = \sum_{n=0}^{\infty}\frac{(z^2)^n}{n!} = e^{z^2}\]

\spart

\[\sum_{n=0}^{\infty}n(n-1)z^n = \sum_{n=2}^{\infty}n(n-1)z^n = \frac{2z^2}{(1-z)^3}\]

Para la resolución de este apartado, nos hemos basado en el estudio del sumatorio:
\[\sum_{m=0}^{\infty} (m+1)mz^{m-1} = \frac{2}{(1-z)^3}\]
y multiplicando por $z^2$ a ambos lados de la igualdad, obtenemos la parte derecha necesaria para poder sustituir en nuestro problema inicial.

A su vez, para obtener este sumatorio, nos hemos basado en el producto de las series:
\[\sum_{n=0}^{\infty}z^n = \frac{1}{1-z}\]
\[\sum_{n=0}^{\infty}zz^{n-1} = \frac{1}{(1-z)^2}\]

\spart

\[\sum_{n=0}^{\infty}(-1)^n \frac{(z-2πi)^n}{n!} = \sum_{n=0}^{\infty} \frac{(-z+2πi)^n}{n!} = e^{-z+2πi} = e^{-z}\]
\end{problem}

\begin{problem}[17]
Si $f(z)=\sum_{n=0}^{\infty}a_nz^n$ ¿qué representa $\sum_{n=1}^{\infty}n^2a_nz^n$ en términos de $f$?
\solution

Definimos un nuevo operador de la forma:
\[\left[ z\frac{\partial}{\partial z}\right]\left( f(z) \right) = z f'(z)\]
aplicando este nuevo operador a la función dada obtenemos
\[\left[ z\frac{\partial}{\partial z}\right]\left( f(z) \right) = \sum_{n=0}^{\infty}na_nz^n  \]
y aplicado a la serie que queremos estudiar, podemos ver que
\[\left[ z\frac{\partial}{\partial z}\right]^2\left( f(z) \right)=  \sum_{n=1}^{\infty}n^2a_nz^n\]

Por tanto ya tenemos el resultado pedido. Para acabar, sólo necesitamos calcular el resultado de aplicar el nuevo operador dos veces a la función $f(z)$. Haciéndolo, podemos ver que:

\[zf'(z)+z^2f''(z)= \sum_{n=1}^{\infty}n^2a_nz^n\]
\end{problem}

\begin{problem}[18]
¿Para qué valores de $z$ convergen las siguientes series?
\ppart
\[\sum_{n=0}^{\infty} \left( \frac{z}{1+z}\right)^n\]

\ppart
\[\sum_{n=0}^{\infty} ne^{-nz}\]

\ppart
\[\sum_{n=0}^{\infty} \frac{\sin(nz)}{n^2}\]

\ppart
\[\sum_{n=0}^{\infty} \frac{\sin(nz)}{2^n}\]

\ppart
\[\sum_{n=0}^{\infty} \frac{z^n}{1+z^{2n}}\]

\solution

\spart
\[\sum_{n=0}^{\infty} \left( \frac{z}{1+z}\right)^n\]
Tenemos una progresión geométrica, que sabemos que converge siempre que el módulo de la razón sea menor que 1.

Por tanto sabemos que la serie converge sii
\[\left| \frac{z}{1+z}\right| < 1 \iff |z| < |1+z| \iff x^2+y^2 < 1+x^2+2x+y^2 \iff 2x+1 > 0 \iff Re(z)>-\frac{1}{2}\]

\spart
\[\sum_{n=0}^{\infty} ne^{-nz} \text{ converge } \iff \sum_{n=0}^{\infty} e^{-nz}\text{ converge }\]

y vemos que se trata nuevamente de una progresión geométrica de razón $e^{-z}$ y, al igual que en el apartado anterior, para garantizar la convergencia necesitamos que el módulo de ese valor sea menor que 1. Es decir:
\[|e^{-z}| < 1 \iff 1 < |e^z|=e^{Re(z)} \iff Re(z) > 0\]

\spart
Para poder estudiar esta serie debemos ver antes
\[\sin(nz)=\frac{1}{2i}\left( e^{inz}-e^{-inz}\right) \implies |\sin(nz)|^2 = \left( \frac{e^{ny}-e^{-ny}}{2}\right)^2+\sin^2(nx)\]

Podemos ver que para $n$ grande el segundo sumando oscila entre -1 y 1, por lo que no nos da problemas para la convergencia y tenemos
\[\frac{|\sin(nz)|^2}{n^2} \equiv \frac{e^{n|y|}}{n^2}\]
por lo que los sumandos de la serie convergen a infinito, de modo que la serie también divergen. \textbf{Todo este razonamiento es válido sólo si $|y|\neq 0$}


\spart
\[\sum_{n=0}^{\infty} \frac{\sin(nz)}{2^n}\]
% Nos apoyamos en que modulo de seno cuadrado es seno hiperbolico cuadrado de parte imaginaria mas seno cuadrado de parte real. el seno de la parte real oscila y no dará problemas por lo que la serie acabará siendo igual (convergentemente hablando) al seno hiperbolico al cuadrado.


\end{problem}

\begin{problem}[19]
Se considera la serie $\sum_{n=1}^{\infty}\frac{1}{n^z}$ (conocida como \textbf{Zeta de Riemann}) para $z \in \cplex$
\ppart Demuestre que la serie converge si $Re(z) > 1$
\ppart Demuestre que si $a$ es un número real mayor que 1, entonces la serie converge uniformemente en $\{z \in \cplex \tq Re(z) \geq a\}$

\solution
\doneby{Pedro}

\spart
Siendo $z=x+iy$ tenemos que:
\[\sum_{n=1}^{\infty} \frac{1}{n^z} = \sum_{n=1}^{\infty} \frac{1}{n^{x+iy}} = \sum_{n=1}^{\infty} \frac{1}{n^x} \frac{1}{n^{iy}} = \sum_{n=1}^{\infty} \frac{1}{n^x} \frac{1}{e^{iy\log(n)}} =\footnote{Usando Euler escribimos el segundo factor como combinación de senos y cosenos, que nos dará un número complejo acotado que no afecta a la convergencia de la función} \sum_{n=1}^{\infty}  \frac{1}{n^x}\]

Tenemos entonces que la suma inicial converge si lo hace $\sum_{n=1}^{\infty}  \frac{1}{n^x}$

Usando el teorema de la integral de Cauchy, tenemos que la suma converge si lo hace la integral. Vamos a ver esa integral.

\[\int_1^{\infty}\frac{1}{t^x}dt = \frac{1}{1-x}t^{1-x}|_1^{\infty} = \frac{1}{1-x} \left((\infty)^{1-x}-0\right)\]
Podemos ver fácilmente que para que converga necesitamos que $x>1$

\spart

Este apartado no supone ningún problema. Si en lugar de estudiar $\sum_{n=1}^{\infty} \frac{1}{n^z}$ hubiésemos estudiado $\sum_{n=1}^{\infty}\left| \frac{1}{n^z}\right|$ podríamos haber seguido los mismos pasos, empleando siempre el valor absoluto y escribiendo $\leq$ en lugar de $=$ con lo que habríamos llegado a idéntico resultado, lo que provoca la convergencia absoluta.
\end{problem}

\begin{problem}[20]
Supongamos que la serie de potencias $f(z)=\sum_{k=0}^{\infty}a_kz^k$ tiene un radio de convergencia $R=1$ y que $\sum_{k=0}^{\infty}a_k=0$. Denotemos
\[s_n(z)=\sum_{k=0}^n a_k z^k \text{  y  } s_n=\sum_{k=0}^na_k\]

\ppart Demuestre que $s_n(z)=(1-z)\sum_{k=0}^{n-1}s_kz^k+s_nz^n$ y concluya que $f(z)=(1-z)\sum_{n=0}^{\infty}s_nz^n$

\ppart Demuestre que $f(z) \to 0$ cuando $z$ se aproxima a 1 de tal forma que $\frac{|1-z|}{1-|z|}$ está acotado

\solution
\doneby{Pedro}

\spart
Tenemos que
\[s_n(z) = (1-z)\sum_{k=0}^{n-1}s_kz^k+s_nz^n = \sum_{k=0}^{n-1}s_kz^k -\sum_{k=0}^{n-1}s_kz^{k+1}+s_nz^n = \]
\[=\sum_{k=0}^{n-1}\sum_{i=0}^ka_iz^k -\sum_{k=0}^{n-1}\sum_{i=0}^ka_iz^{k+1}+\sum_{i=0}^na_iz^n = \sum_{k=0}^{n-1} \left( \sum_{i=0}^k a_i - \sum_{i=0}^{k-1} a_i\right) z^k - \sum_{k=0}^{n-1} a_kz^n + \sum_{i=0}^na_iz^n =\]
\[= \sum_{k=1}^{n-1} a_k z^k + a_nz^n = \sum_{k=0}^n a_k z^k\]

Las cuentas son un poco raras si las miras de golpe pero en el fondo es sólo jugar con los índices de los sumatorios.

\spart

\textcolor{blue}{No se escribirlo pero la idea está ahí. Los $s_n$ no nos preocupan por que sabemos que tienden a 0 y el sumatorio que nos queda ignorándolos es una progresión geométrica cuyo valor multiplicado por (1-z) es la fracción que nos dan diciendo que está acotada. El problema es que no se cómo escribirlo bien}

\end{problem}

\begin{problem}[21]
Demuestre las siguientes afirmaciones

\ppart Si $z \in \mathbb{D}$ y $k \in \nat$, entonces $1-|z|^k \leq k(1-|z|)$

\ppart Si $\{a_n\}$ es una sucesión de números complejos tal que $\lim_{n \to \infty} na_n=0$, entonces
\[\lim_{n\to \infty}n \cdot \sup_{k \geq n}|a_k| = 0 \text{  y  } \lim_{n \to \infty}\frac{1}{n}\cdot \sum_{k=0}^nk|a_k|=0\]

\solution
\doneby{Pedro}

\spart
Vamos a demostrarlo por inducción.

El caso base es claro, tomamos $k=1$ y ya lo tenemos

Tomemos como hipótesis que el resultado es cierto el resultado para $k=n$. Queremos ver si es cierto que:
\[1-|z|^k|z| < k(1-|z|)+(1-|z|)\]
Basándonos en la hipótesis tenemos que
\[ k(1-|z|)+(1-|z|) > 1-|z|^k+(1-|z|)\]
Si se cumpliera:
\[1-|z|^k+(1-|z|) > 1-|z|^k|z|\]
ya lo tendríamos todo hecho. Vemos que es cierto que se cumple la desigualdad pues simplificando los 1s que se suman y pasando $|z|^k$ al otro lado llegamos a
\[1-|z| > (1-|z|)|z|^k\]
lo cual es absolutamente cierto pues $|z|^k < 1$

\spart

Para la primera ecuación tenemos que:

\[\lim_{n\to \infty}n \cdot \sup_{k \geq n}|a_k| = \sup_{k \geq n} \lim_{n\to \infty}n |a_k| = \sup_{k \geq n} 0 = 0\]

Vamos con la segunda:
\[\lim_{n \to \infty}\frac{1}{n}\cdot \sum_{k=0}^nk|a_k| \leq \lim_{n \to \infty}\frac{1}{n}\cdot \sup_{k \geq n}|a_k| \sum_{k=0}^nk = \lim_{n \to \infty}\cdot \sup_{k \geq n}|a_k| \frac{1}{n} \frac{n(n+1)}{2}=\]
\[=\lim_{n \to \infty}\cdot \sup_{k \geq n}|a_k|\frac{n+1}{2} = \lim_{n\to \infty}n \cdot \sup_{k \geq n}|a_k| = 0\]
\end{problem}

\begin{problem}[22]
Sea $f$ holomorfa en el disco unidad $\mathbb{D}$, con desarrollo de Taylor alrededor del origen:
\[f(z)=\sum_{n=0}^{\infty}a_nz^n\]
Denotemos por $s_n$ la suma parcial $s_n=\sum_{k=0}^n a_k$

\ppart Prueba que si $z \in \mathbb{D}$, entonces
\[|f(z)-s_n| \leq \sum_{k=0}^n|a_k||1-z^k|+\sup_{k \geq n}|a_k|\frac{1}{1-|z|}\]

\ppart Deduzca que si $z_n=1-1/n$, entonces
\[|f(z_0)-s_n|\leq \frac{1}{n}\sum_{k=0}^n |a_k|k+n\sup_{k \geq n}|a_k|\]

\ppart Concluya finalmente que si $\lim_{n\to\infty}na_n=0$ y que
\[\lim_{x \to 1^-}f(x) = \sum_{n=0}^{\infty}a_n\]
en caso de existir dicho límite


\textbf{Sugerencia:} Use el ejercicio anterior
\solution

\end{problem}
\newpage

%%%%%%%%%%%%%%%%%%%%%%%%%%%%%%%%%%%%%%%%%%%%%%%%%%%%%%%%%%%%%%%%%%%%%%%%
%%%%%%%%%%%%%%%%%%%%%%%%%%%%%%%%%%%%%%%%%%%%%%%%%%%%%%%%%%%%%%%%%%%%%%%%
%%                                                                    %%
%%                            HOJA 3                                  %%
%%                                                                    %%
%%%%%%%%%%%%%%%%%%%%%%%%%%%%%%%%%%%%%%%%%%%%%%%%%%%%%%%%%%%%%%%%%%%%%%%%
%%%%%%%%%%%%%%%%%%%%%%%%%%%%%%%%%%%%%%%%%%%%%%%%%%%%%%%%%%%%%%%%%%%%%%%%
\newpage
\section{Hoja 3}

\begin{problem}[1]
Sea $ω_1,\cdots ω_n$ las n raíces n-ésimas de la unidad, es decir, $ω_j=e^{2πi(j/n)}$ con j=1,...,n.
\ppart Pruebe que si $m$ es un número natural, entonces
\[\frac{1}{n}\sum_{j=1}^n (ω_j)^m = \left\{ \begin{array}{lcc}
             0 &   si  & m \text{ no es múltiplo de  n} \\
             \\ 1 &  si  & m\text{ es múltiplo de n }
             \end{array}
   \right.\]

\ppart Si $P(z) = a_0+a_1z+\cdots a_nz^n+\cdots + a_mz^m, \ m \leq 2n-1$, demuestre que
\[\frac{1}{n}\sum_{j=1}^nP(ω_j)=a_0+a_n\]

\solution

\doneby{Pedro}

\spart
Si $ω_j$ es una raíz n-ésima de la unidad tenemos que $w_j^n=1$ por lo que si $m$ es múltiplo de $n$ tenemos
\[ω_j^m = ω_j^{kn} = (ω_j^n)^k = 1^k = 1 \]

Así el sumatorio que estamos estudiando quedaría
\[\frac{1}{n} \sum_{j=1}^n (ω_j)^m = \frac{1}{n} \sum_{j=1}^n 1 = \frac{n}{n}=1\]

Por otro lado sabemos que la suma de las raíces de la unidad es 1, es decir
\[\sum_{j=0}^n ω_j^n = 1 \implies \sum_{j=1}^n ω_j^n = 0 \ \text{ puesto que } ω_j^0 = 1\]

Para los escépticos, esto puede probarse escribiendo:
\[\sum_{j=1}^n (ω_j)^m = \sum_{j=0}^{n-1} \left(e^{2 \pi i j/n}\right)^m = \frac{\left(e^{2 \pi i n/n}\right)^m - 1}{\left(e^{2 \pi i/n}\right)^m - 1} = \frac{1-1}{e^{2 \pi im/n} - 1}\footnote{Al no ser m múltiplo de n no puede darse el caso de tener un 0 en el denominador} = 0 \]

\spart
\[\frac{1}{n}\sum_{j=1}^nP(ω_j) = \frac{1}{n}\left( \sum_{i=1}^ma_i\left(\sum_{j=1}^n w_j^i\right)\right) = \left( \sum_{i=1}^ma_i\left(\frac{1}{n}\sum_{j=1}^n w_j^i\right)\right) =a_0+a_n\]
Basándonos en el apartado anterior conocemos el valor del sumatorio interior, que será 0 en todos los casos salvo cuando $i$ sea múltiplo de $n$, cosa que sólo puede ocurrir cuando $i$ sea 0 ó $n$ por lo que obtenemos el resultado esperado
\end{problem}

\begin{problem}[2]
\textit{Resolución de la ecuación cúbica}

Consideramos la ecuación cúbica $z^3+az^2+bz+c=0$ con $a,b,c \in \cplex$
\ppart Aplique un cambio de variable $z=w+h$ para obtener una ecuación equivalente de la forma $ω^3+βω+γ=0$
\ppart Haga ahora un cambio $ω=gu$ que nos dé una ecuación de la forma $4u^3-3u+δ=0$
\ppart Sea $v \in \cplex$ tal que $\sin(3v)=δ$. Demuestre que α=$\sin(v)$ es raíz de la ecuación anterior
\ppart Aplique este procedimiento a la ecuación $z^3+3z^2-1=0$

\solution

\doneby{Pedro}

\spart
Siguiendo lo que se indica en el enunciado obtenemos:
\[z^3+az^2+bz+c = 0 \equiv w^3+3w^2h+3wh^2+h^3 + aw^2+a2wh+ah^2+bw+bh+c =0 \equiv\]
\[\equiv w^3+βw+γ = 0\]
siendo

\[\left\{ \begin{array}{l}
             β = 3h^2+2ah+b \\
			 γ = h^3+ah^2+bh+c \\
			 3h = -a
             \end{array}
\right.\]

\spart

Aplicamos el cambio de variable pedido sobre la ecuación del apartado anterior:
\[w^3+βw+γ=0 \equiv g^3u^3+βgu+γ = 0 \equiv 4u^3-3u+δ\]
siendo
\[\left\{ \begin{array}{l}
             g^3 = 4 \\
			 βg + 3 = δ
             \end{array}
\right.\]

\spart

Para comprobar que el α dado es solución de la ecuación vamos a sustituirlo en ella y a comprobar que la igualdad se mantiene.

Para ello vamos a analizar previamente la información que nos dan
\[\sin(3v) = \sin(2v+v)=\sin(v)\cos(2v)+\sin(2v)\cos(v) = \sin(v)\cos^2(v)-\sin^3(v)+2\sin(v)\cos^2(v) = \]
\[=3\sin(v)-3\sin^3(v)-\sin^3(v) = 3\sin(v)-4\sin^3(v)=δ \implies 4\sin^3(v)=-δ+3\sin(v)\]

Veamos ahora cuando vale la ecuación al sustituir el valor dado
\[4\sin^3(v)-3 \sin(v) +δ = -δ+3\sin(v)-3\sin(v) +δ = 0\]

\spart

Basta con resolver las ecuaciones que se han ido planteando en los sucesivos apartados para poder encontrar el cambio de variable adecuado.

Se deja como ejercicio para el lector.
\end{problem}

\begin{problem}[3]
Demuestre que existe un único polinomio $P_n$ de grado $n$ tal que
\[z^n+\frac{1}{z^n}=P_n\left( z + \frac{1}{z}\right)\]

\textbf{Ayuda:} Puede hacerse por inducción. El caso $n=1$ es obvio. Conviene escribir $z^n+\frac{1}{z^n}-\left( z+\frac{1}{z}\right)^n$ y usar que ${n \choose k} = {n \choose n-k}$
\solution

\doneby{Pedro}

Atendiendo al consejo del enunciado, vamos a demostrarlo por inducción.

El caso base es trivial.

Supongamos ahora que se cumple para $n<k$ y veamos que ocurre para el caso $n=k$:

Queremos probar:
\[z^{n}+\frac{1}{z^{n}} = P_{n}\left(z+\frac{1}{z}\right)\]

Atendiendo a la sugerencia del enunciado escribimos
\[z^n+\frac{1}{z^n}-\left(z+\frac{1}{z}\right)^n = - \sum_{i=1}^{n-1}{n \choose i}z^i\frac{1}{z^{n-i}} = - \sum_{i=1}^{n-1}{n \choose i} z^{2i-n}\]

Vamos ahora a sacar del sumatorio el término más grane y el más pequeño con lo que nos queda:
\[- \sum_{i=1}^{n-1}{n \choose i} z^{2i-n} = -{n \choose 1}\frac{1}{z^{n-1}}-{n \choose n-1} z^{n-1} - \sum_{i=2}^{n-2}{n \choose i} z^{2i-n} \]

La ayuda del enunciado nos deja claro que
\[{n \choose 1} = {n \choose n-1}\]
por lo que podemos sacar factor común a los dos sumandos que quedan fuera del sumatorio y expresarlos como un polinomio de grado $n-1$ de la forma indicada en el enunciado\footnote{Por hipótesis de la inducción}.

Si repetimos sucesivamente este procedimiento con el sumatorio que nos queda, vamos obteniendo una serie de polinomios, todos ellos multiplicados por unos ciertos coeficientes y de grado menor que $n$ por lo que su suma sigue siendo un polinomio de grado menor que $n$.

Finalmente nos quedaría
\[z^n+\frac{1}{z^n}-\left(z+\frac{1}{z}\right)^n = -P_{n-1}\left(z+\frac{1}{z}\right) \implies\]
\[\implies z^n + \frac{1}{z^n} = \left(z+\frac{1}{z}\right)^n -P_{n-1}\left(z+\frac{1}{z}\right) = P_n\left(z+\frac{1}{z}\right) \]
\end{problem}

\begin{problem}[4]
El objetivo de este ejercicio es demostrar el resultado conocido como \textit{el teorema de Gauss-Lucas} (apartado \textbf{c)}).

\ppart Demuestre que si $z_1,z_2,\cdots z_n$ son números complejos entonces el polígono convexo más pequeño que los contiene (posiblemente degenerado) viene dado por
\[\{z=t_1z_1+t_2z_2+\cdots + t_nz_n : \ t_i \geq 0 \text{ y } \sum t_i = 1\}\]

\ppart Compruebe que el polígono del apartado anterior es la intersección de todos los semiplanos que contiene a  $z_1,z_2,\cdots, z_n$.

\ppart Demuestre que si $P(z)$ es un polinomio, su derivada $P'(z)$ no puede tener ceros fuera del polígono convexo más pequeño que contiene a las raíces de $P(z)$

\textbf{Ayuda:} Escriba $P(z)=a_n(z-z_1)(z-z_2)\cdots (z-z_n)$ y considere $\frac{P'(z)}{P(z)}$ para expresar los ceros de $P'(z)$ en la forma $m_1z_1+\cdots + m_nz_n$ o use el apartado \textbf{b)}

\solution

\spart

Empezamos con algo de notación. Sea $Z = \set{z_k}_{k=1}^n ⊂ ℂ$ el conjunto de todos esos complejos y sea \[ \mathcal{P} ≝ \set{ z = \sum_{k=1}^n t_k z_k \tq t_k ≥ 0,\; \sum_{k=1}^n t_k = 1} \] el polígono descrito en el enunciado.

Vamos a probar primero que eso es un polígono que contiene a dos los $z_k$ y que además es convexo. La primera parte es obvia: $z_k ∈ \mathcal{P}$ tomando $t_k = 1$ y el resto $t_j = 0$ para $j ≠ k$.

Queremos demostrar ahora que es convexo. Para ello, necesitamos que el segmento entre todos los vértices esté contenido en el polígono. Sean $z_j, z_k ∈ Z$. Podemos parametrizar fácilmente el segmento que los une por $σ_{z_j, z_k} (t) = t z_j + (1-t) z_k$ con $t∈[0,1]$, que obviamente está contenido en $\mathcal{P}$.

Nos falta demostrar que es $\mathcal{P}$ es el más pequeño que cumple estas condiciones. Lo hacemos por reducción al absurdo: suponemos que existe $\mathcal{P}' \subsetneq \mathcal{P}$ convexo y contenido a todos los $z_k$. Entonces $∃ w_0 ∈ \mathcal{P} \setminus \mathcal{P}'$, que estará dado por \[ w_0 = \sum_{k=1}^n t_k^0 z_k \] (lo del superíndice $0$ es sólo notación, que luego viene útil).

Sea ahora \[ w_1 = \sum_{k=2}^n t_k^1 z_k \] con $t_k^1 = t_k^0 + \frac{t_1^0}{n - 1}$. Es claro que no nos hemos salido: $w_1 ∈ \mathcal{P}$. Si tenemos que $w_1 ∈ \mathcal{P}'$, habremos terminado: al ser $\mathcal{P}'$ convexo, el segmento que une dos puntos cualesquiera del polígono tiene que estar necesariamente en el polígono. Pero el segmento que une $w_1$ y $z_1$ contiene a $w_0$ que no está en el polígono, contradicción.

¿Qué ocurre si ese $w_1$ que hemos construido no está en $\mathcal{P}'$? Pues que seguimos el procedimiento (para eso estaban los subíndices). Definimos \[ w_j = \sum_{k=j + 1}^n t_k^j z_k\qquad t_k^j = t_k^{j-1} + \frac{t_j^{j-1}}{n - j} \]. Si $w_2$ no está en $\mathcal{P}'$, miramos $w_3$ y así sucesivamente. La cuestión es que al menos $w_{n-1}$ va a tener que estar en $\mathcal{P}'$ por ser, por definición, $z_n$. Así, $w_{n-2}$ estaría en el segmento que une $z_n$ y $z_{n-1}$. pero acabamos de decir que no estaba en $\mathcal{P}'$, contradicción de nuevo. Hemos demostrado entonces que $\mathcal{P}$ es el polinomio convexo más pequeño que contiene a todos los $z_k$.

\spart

Este es bastante obvio así que no voy a comentarlo mucho, pero básicamente si coges los $z_k$ que son vértices del polígono, los semiplanos serán los generados por los segmentos $z_k, z_{k+1}$. La intersección de todos ellos será el área encerrada por todos los segmentos, que será el polígono que teníamos.

\spart
Siguiendo la ayuda proporcionada escribimos
\[P(z)=a_n(z-z_1)(z-z_2)\cdots (z-z_n)\]
con lo que la fracción pedida queda de la forma:
\[\frac{P'(z)}{P(z)}=\sum_{i=1}^n \frac{1}{z-z_i} \; \forall z \neq z_i \text{ con } i)1,2,...,n\]

Sea $z$ un cero de $P'(z)$ que no anula el polinomio (es distinto de todo $z_i$), tenemos
\[\sum_{i=1}^n \frac{1}{z-z_i}=0\]

Ahora tenemos que comprobar que esta $z$ se contiene dentro del polinomio mencionado en el apartado \textbf{a)}.

Si multiplicamos y dividimos por el conjugado en cada uno de los términos del sumatorio que estamos manejando llegamos a:
\[\sum_{i=1}^n \frac{\bar{z}-\bar{z_i}}{|z-z_i|^2} = 0 \implies  \sum_{i=1}^n \frac{\bar{z_i}}{|z-z_i|^2} = \sum_{i=1}^n \frac{\bar{z}}{|z-z_i|^2}  \]

Si despejamos $z$ llegamos a:
\[z= \sum_{i=1}^n \frac{\frac{1}{|z-z_i|^2}}{\sum_{i=1}^n \frac{1}{|z-z_i|^2}}z_i\]

Si al coeficiente de $z_i$ lo llamamos $t_i$ podemos ver fácilmente que
\[\sum t_i = 1\]
con lo que el cero encontrado pertenece al polígono convexo en el que se encontraban los ceros del polinomio.
\end{problem}

\begin{problem}[5]
Desarrolle en series de potencias (centradas en el origen) las siguientes funciones elementales:
\begin{enumerate}
\item \[(1-z)\cos(z)\]
\item \[\frac{\cos(z)}{1-z^2}\]
\item \[\frac{e^{-z}}{1+z}\]
\end{enumerate}

indicando en cada caso el radio de convergencia. Hágase lo mismo para la función $\sin(2z)=2\sin(z)\cos(z)$ de dos maneras distintas. (Si no resulta fácil encontrar una fórmula general para los coeficientes, basta con escribir los 5 primeros términos e cada serie)
\solution

\begin{enumerate}
\item

\[(1-z)\cos(z) = (1-z)\sum_{n=0}^{\infty}\frac{(-1)^n}{(2n)!}z^{2n}=\sum_{n=0}^{\infty}\frac{(-1)^n}{(2n)!}z^{2n}-\sum_{n=0}^{\infty}\frac{(-1)^n}{(2n)!}z^{2n+1} =\]
\[= \sum_{k=0}^{\infty}a_kz^k\]
siendo
\[a_k= \left\{ \begin{array}{lcc}
             \frac{(-1)^{k/2}}{k!} &   si  & k \text{ es par} \\
             \\ \frac{(-1)^{(k-1)/2-1}}{k!} &  si  & k \text{ es impar}
             \end{array}
   \right.\]


\item
\[\frac{\cos(z)}{1-z^2}\]

En este caso sabemos desarrollar tanto el numerador como el cociente que da lugar al denominador en series de potencias con lo que obtendremos algo de la forma:

\[\frac{\cos(z)}{1-z^2} = \left(\sum a_n z^n \right)\cdot \left( \sum b_nz^n\right)=\sum c_n z^n\]

Vamos a calcular ahora esos sumatorios:
\[\cos(z)=\sum a_n z^n \; \; \; \frac{1}{1-z^2} = \sum b_n z^n\]
siendo
\[a_n= \left\{ \begin{array}{lcc}
             \frac{(-1)^{n/2}}{k!} &   si  & n \text{ es par} \\
             \\0  &  si  & n \text{ es impar}
             \end{array}
   \right.  \; \; b_n= \left\{ \begin{array}{lcc}
             1 &   si  & n \text{ es par} \\
             \\0  &  si  & n \text{ es impar}
             \end{array}
   \right.\]

Por último, sólo nos queda calcular los coeficientes $c_n$. Vamos a ello:
\[c_n = \sum_{k=0}^n a_kb_{n-k} = \sum_{0\leq k \leq n} \frac{(-1)^{n/2}}{k!} = \sum_{j=0}^{n/2} \frac{(-1)^{j}}{(2j)!}\]
siendo esto válido sólo para los $n$ pares. Cuando $n$ sea impar tendremos $c_n=0$


\item
\[\frac{e^{-z}}{1+z}\]

\doneby{Pedro}

En esta ocasión ocurre lo mismo que en el apartado anterior. Tenemos:
\[\frac{1}{1+z}=\frac{1}{1-(-z)} = \sum_{i=0}^{\infty}(-z)^i=\sum (-1)^nz^n\]
\[e^{-z} = \sum_{n=0}^{\infty}\frac{(-z)^n}{n!} = \sum \frac{(-1)^n}{n!}z^n\]

Ahora sólo nos queda multiplicar.
\[\frac{e^{-z}}{1+z}=\sum (-1)^nz^n \cdot \sum \frac{(-1)^n}{n!}z^n = \sum_{n=0}^{\infty}\sum_{i=0}^n(-1)^iz^i\frac{(-1)^{n-i}}{(n-i)!}z^{n-i} = \]
\[=\sum_{n=0}^{\infty}\left(z^n (-1)^n\sum_{i=0}^n\frac{1}{i!} \right)= \sum_{n=0}^{\infty}(-1)^n\cdot e \cdot z^n\]

\end{enumerate}

Vamos a por el $\sin(2z)$ que podemos escribir como serie de la siguiente forma
\[\sin(2z) =\sum_{n=0}^{\infty} \frac{(-1)^n\cdot 2^{2n+1}}{(2n+1)!}z^{2n+1}\footnote{Esta fórmula la plantó la profesora en clase. Puede obtenerse aplicando Taylor o mediante ideas felices que permitan asemejar la función objeto de estudio a otra ya estudiada}\]

la segunda forma de calcularlo a la que se refiere el enunciado es el método empleado al resolver los apartados anteriores de este mismo ejercicio. Desarrollamos el seno por un lado, el coseno por otro y multiplicamos.

\end{problem}

\begin{problem}[6]
Escriba explícitamente la función cuya serie de potencias es
\[f(z)=\sum_{n=0}^{\infty}\frac{z^{4n}}{(4n)!}\]
¿Cuánto vale $\sum_{n=0}^{\infty} \frac{1}{(4n)!}$?

\textbf{Ayuda:} Evalúe la función exponencial en los puntos $\pm z $ y $\pm iz$
\solution

Atendiendo a la sugerencia y evaluamos la función exponencial en los puntos indicados obtenemos:
\[e^z = \sum_{n=0}^{\infty}\frac{1}{n!}z^n; \; \; e^{-z}=\sum_{n=0}^{\infty} \frac{(-1)^n}{n!}z^n ;\;\; e^{iz} = \sum_{n=0}^{\infty}\frac{i^n}{n!}z^n, \;\; e^{-iz} = \sum_{n=0}^{\infty}\frac{(-1)^ni^n}{n!}z^n\]

Si sumamos las dos primeras y las dos últimas obtenemos:
\[\frac{1}{2}(e^z+e^{-z}) =\sum_{k=0}^{\infty} \frac{1}{(2k)!}z^{2k}\]
\[\frac{1}{2}(e^{iz}+e^{-iz}) =\sum_{k=0}^{\infty} \frac{i^{2k}}{(2k)!}z^{2k}=\sum_{k=0}^{\infty} \frac{(-1)^k}{(2k)!}z^{2k}\]

Si sumamos, estamos eliminando aquellas en las que $k$ es par, es decir, el exponente de la $z$ será múltiplo de 4:
\[\frac{1}{2}\left(\frac{1}{2}(e^z+e^{-z})+\frac{1}{2}(e^{iz}+e^{-iz})\right) = \sum_{n=0}^{\infty} \frac{(-1)^n\cdot 2^{2n+1}}{(2n+1)!}z^{2n+1}\]
y vemos que la función pedida, la parte de la izquierda de esta desigualdad, puede escribirse de forma simplificada como:
\[\frac{1}{2} (\cosh(z) + \cos(z))\]
\end{problem}

\begin{problem}[7]
¿Para qué valores de $z \in \cplex$ se cumple que $\overline{e^{iz}} = e^{i\bar{z}}$?
\solution

\doneby{Pedro}

Antes de nada vamos a ver que el conjugado de una exponencial es la exponencial del conjugado.
\[\overline{e^{x+iy}}=\overline{e^xe^{iy}}=\overline{e^x(\cos(y)+i\sin(y))}=\]
\[=e^x(\cos(y)-i\sin(y))=e^x(\cos(-y)+i\sin(-y))=e^xe^{-iy}=e^{x-iy}=e^{\overline{x+iy}}\]

Por tanto tenemos:
\[\overline{e^{iz}} = \overline{e^{i(x+iy)}} = \overline{e^{ix}}\overline{e^{-y}} =e^{-ix}e^{-y}\]

Y por otro:
\[e^{i\bar{z}}=e^{xi+y}=e^{xi}e^{y}\]
si queremos que coincidan estos dos números complejos necesitamos $y=0$ y que
\[\cos(x) = \cos(-x) \ \; \; \sin(x)=\sin(-x) \footnote{Obtenemos estas ecuaciones descomponiendo la exponencial en senos y cosenos} \]

Es decir, necesitamos $x=πk$
\end{problem}

\begin{problem}[8]
Demuestre que
\ppart $\sin^2(z)+\cos^2(z) = 1$
\ppart $\cos(2z)=\cos^2(z)-\sin^2(z)$
\solution

\spart
Veamos primero una forma elegante de hacer esta demostración suponiendo que ya sepamos que la ecuación es cierta en los reales.

Tenemos que la función $\sen^2(z)+\cos^2(z)-1$ se anula en toda la recta de los reales. Por el principio de los ceros aislados una función holomorfa que se hace 0 en todo un eje es la función nula.
%TODO completar esto como se hizo en los apuntes

La otra forma de hacerlo, suponiendo que no sepamos nada de lo que ocurre en los reales, es escribir los senos y cosenos como exponenciales de la siguiente forma:
\[\sin(z)=\frac{1}{2i}\left( e^{iz}-e^{-iz}\right)\]
\[\cos(z)=\frac{1}{2}\left( e^{iz}+e^{-iz}\right)\]

Si ahora sumamos los cuadrados obtenemos directamente un 1.

\spart

Nuevamente, podría hacerse de dos formas al igual que el apartado anterior.

\end{problem}

\begin{problem}[9]
Resuelva las siguientes ecuaciones
\ppart $\cos(z)=2$
\ppart $\sin(z)=\frac{3}{4}-\frac{i}{4}$
\solution

\spart
Escribiéndolo como exponenciales tenemos la ecuación:
\[\cos(z)=2 \iff \frac{1}{2}\left( e^{iz}+e^{-iz}\right) = 2 \iff e^{iz}+\frac{1}{e^iz}=4\]

Haciendo el cambio de variable $e^{iz}=t$ planteamos la ecuación de segundo grado:
\[t^2-4t+1=0 \implies t=2\pm \sqrt{3}\]

Ahora debemos despejar la $z$ a partir de la relación $e^{iz}=2\pm \sqrt{3}$.

\[\begin{cases}
t_1=2+\sqrt{3} \implies e^{iz_1} = 2 + \sqrt{3} \implies iz_1 = \log(2+\sqrt{3})+2kπi \implies z_1=-i\log(2+\sqrt{3})+2kπ\\
t_2=2-\sqrt{3} \implies e^{iz_2} = 2 - \sqrt{3} \implies iz_2 = \log(2-\sqrt{3})+2kπi \implies z_2=-i\log(2-\sqrt{3})+2kπ.
\end{cases}\]

\spart
Escribiéndolo nuevamente como exponenciales tenemos:
\[\frac{1}{2i}\left( e^{iz}-e^{-iz}\right) = \frac{3}{4}-\frac{i}{4} \]

Realizando el mismo cambio de variable que hicimos en el apartado anterior y agrupando los coeficientes llegamos a la ecuación de segundo grado:

\[t^2+\frac{1}{2}(1+3i)t-1=0\]

que nos da las soluciones:
\[t= \frac{1}{2} \left[ \frac{1}{2}(1+3i)\pm \sqrt{\frac{1}{4}(1+i)^2+4}\right]\]

Vamos a trabajar un poco con el número que hay dentro de la raíz para hacerlo lucir mejor.
\[\frac{1}{4}(1+i)^2+4 = \frac{1}{4}(1+6i-9)+4 = 2+\frac{3}{2}i = \frac{5}{2}e^{α} \; \text{ con } \tg(α)=\frac{3}{4}\]

Volvemos ahora al valor de $t$ y aplicamos esta simplificación que acabamos de calcular:
\[t=\frac{1}{2}\left[ \frac{1}{2}(1+3i)\pm \left(\frac{5}{2} \right)^{1/2}e^{iα/2}\right]\]

Escribiendo la exponencial como combinación de senos y cosenos y jugando un poco con la trigonometría legamos fácilmente a:
\[t=\frac{1}{2}\left[ \frac{1}{2}(1+3i)\pm \frac{\sqrt{5}}{\sqrt{2}} \frac{1}{\sqrt{10}}(3+i)\right] = \frac{1}{4}\left[ 1+3i\pm (3+i)\right]\]

Ahora, de forma idéntica al ejercicio anterior, calculamos el valor de $z$ resolviendo las ecuaciones $e^{iz}=t_1$ y $e^{iz}=t_2$.

\[\begin{cases}
t_1 = 1+i \rightarrow e^{iz}=t_1 \implies iz=\log|t_1|+arg(t_1) \implies z = \frac{π}{4}+2kπ-i\frac{1}{2}\log(2) \\
t_2 = 1/2(-1+i) \rightarrow e^{iz}=t_2 \implies iz=\log|t_2|+arg(t_2) \implies z = \frac{3π}{4}+2kπ+i\frac{1}{2}\log(2)
\end{cases}\]
\end{problem}

\begin{problem}[10]
Calcule los siguientes valores:

\ppart
\[e^{iπ/4}, \; e^{5πi/4}, \; e^{-7πi/3}, \; exp\left[π \left(\frac{1+i}{\sqrt{2}} \right)^4 \right], \; \cos(2+3i), \; \sin(1+i)\]

\ppart
\[(1-i)^i, \; 2^{-1+i}, \; i^{\sqrt{2}} \; \text{ tomando la rama principal del logaritmo}\]

\ppart
\[i^{-i}, \; \log(3) , \; \log(\sqrt{3}+1), \; (1+i)^{1+i}, \; 2^{πi} \text{ (calcular todos los posibles valores)}\]

\solution

\spart

\spart
\[(1-i)^i = e^{i\log (1-i)}\]
con la rama principal del logaritmo (como nos indica el enunciado) nos queda
\[\log(1-i) = \log|1-i| + i Arg(1-i) = \log \sqrt{2}-i\frac{π}{4}\]
y sustituyendo:
\[(1-i)^i = e^{i \log \sqrt{2}+\frac{π}{4}}\]


\spart
\end{problem}

\begin{problem}[11]
Denotemos por $\{ arg(z)  \}$ el conjunto de todos los valores posibles para el argumento de $z$, por $\{ \log(z) \}$ el conjunto de todos los valores posibles de $\log(z)$ y por $\{ z^b \}$ el conjunto de todos los valores posibles de $z^b$ con el significado evidente $\{\log(z)\}=\log(|z|)+i\{arg(z)\}$, $\{z^b \} = e^{b\{\log(z)\}}$. Compruebe que:

\ppart
\[\{\log(zw)\} = \{\log(z)\}+\{\log(w)\} \text{ (aquí A+b=\{a+b: a $\in A$, b $\in B$\})}\]
\ppart
\[\{(zw)^b\} = \{z^b\} \{w^b\} \text{ (aquí AB=\{ab: a $\in$ A, b $\in$ B\})}\]
\ppart
\[\{ \log(z^α)\} = \bigcup_{k \in \ent}\left(α\{\log(z)\}+2kπi\right)\]
\solution

\doneby{Pedro}

En el fondo nos están pidiendo la demostración de que dos conjuntos de números son iguales. Vamos a hacerlo por el clásico método del doble contenido. Demostraremos que todo número de un conjunto pertenece al otro y viceversa

\spart

\begin{itemize}
\item
\[\{\log(zw)\} \subset \{\log(z)\}+\{\log(w)\} \text{ (aquí A+b=\{a+b: a $\in A$, b $\in B$\})}\]

Sea $x \in \{\log(zw)\} \implies \exists z,w \tq x=\log(zw)=\log(z)+\log(w) \implies x \in \{\log(z)\}+\{\log(w)\}$
\item
\[\{\log(zw)\} \supset \{\log(z)\}+\{\log(w)\} \text{ (aquí A+b=\{a+b: a $\in A$, b $\in B$\})}\]

Sea $x \in \{\log(z)\}+\{\log(w)\} \implies \exists z,w \tq x=\log(z)+\log(w)=\log(zw)\implies x \in \{\log(zw)\}$
\end{itemize}

\spart

\begin{itemize}
\item
\[\{(zw)^b\} \subset \{z^b\} \{w^b\} \text{ (aquí AB=\{ab: a $\in$ A, b $\in$ B\})}\]

Sea $x \in \{(zw)^b\} \implies \exists z,w \tq x=(zw)^b \implies x=z^bw^b \implies x \in \{z^b\}\{w^b\}$
\item
\[\{(zw)^b\} \supset \{z^b\} \{w^b\} \text{ (aquí AB=\{ab: a $\in$ A, b $\in$ B\})}\]

Sea $x \in \{z^b\}\{w^b\} \implies \exists z,w \tq x=z^bw^b=(zw)^b \implies x \in \{(zw)^b\}$
\end{itemize}

\spart
\textcolor{blue}{Este apartado esta pichí-pichá...}
\begin{itemize}
\item
\[\{ \log(z^α)\} \subset \bigcup_{k \in \ent}\left(α\{\log(z)\}+2kπi\right)\]
Si $x \in \{ \log(z^α)\} \implies \exists z \tq x=\log(|z|^α)+iarg(z^α) = α \log(|z|)+i(Arg(z^α)+2kπ)=α \log(|z|)+iArg(z^α)+i2kπ \implies x \in \bigcup_{k \in \ent}\left(α\{\log(z)\}+2kπi\right)$
\item
\[\{ \log(z^α)\} \supset \bigcup_{k \in \ent}\left(α\{\log(z)\}+2kπi\right)\]
Si $x \in \bigcup_{k \in \ent}\left(α\{\log(z)\}+2kπi\right) \implies \exists k, z \tq x=α\log(|z|)+αarg(z)+2kπi = α\log(|z|)+αArg(z)+2kπi$
\end{itemize}

\end{problem}

\begin{problem}[12]
(Teorema del binomio para exponentes reales) Sea α un número real con $α \notin \nat$ y sea
\[{α \choose 0}=1, \; {α \choose 1}=α, \; {α \choose j} = \frac{α(α-1)\cdots (α-j+1)}{j!} \text{ si } j>1\]

\ppart
Demuestre que el radio de convergencia de la serie $F(z) = \sum_{k=0}^{\infty}{α \choose k}z^k$ es 1

\ppart
Compruebe que $(1+z)F'(z)=αF(z)$

\ppart
Concluya que $F(z)=(1+z)^α$, es decir, $(1+z)^α = \sum_{k=0}^{\infty}{α \choose k} z^k$ si $|z|<1$.
(Aquí se toma la rama principal de $w^α$)

\solution

Aunque no lo indica expresamente el enunciado, es claro que $j \in \ent^+$ ya que de lo contrario no tendrían sentidos los números combinatorios que aparecen en el ejercicio.

\spart

\[R= \lim_{k \to \infty}\left|\frac{a_k}{a_{k+1}} \right| = \lim_{k\to\infty}\left| \frac{k+1}{α-k} \right| = 1\]

\spart
\[(1+z)F'(z)=(1+z)\sum_{k=0}^{\infty}{α \choose k}kz^{k-1} = \sum_{k=0}^{\infty}{α \choose k}kz^{k-1} +\sum_{k=0}^{\infty}{α \choose k}kz^{k} = \]
\[= \sum_{l=0}^{\infty}{α \choose l +1}(l+1)z^{l}+\sum_{k=0}^{\infty}{α \choose k}kz^{k} =\sum_{k=0}^{\infty}\left[ {α \choose k+1}(k+1)+k{α \choose k} \right]z^{k} = \sum_{k=0}^{\infty}{α \choose k}αz^{k} = αF(z)\]

\spart

Vamos a calcular primero $(1+z)^α$ que, por definición sería $(1+z)^α = e^{α\log(1+z)}$ donde $\log(1+z)$ es una rama del logaritmo de $f(z)=1+z$ definida en el disco unidad (donde tenemos convergencia).

Podemos ver que el disco unidad es simplemente conexo y que la función no se anula en él (estamos tomando el disco sin los bordes). Por tanto podemos garantizar la existencia de esta rama del logaritmo.

Podemos tomar pues $\log(1+z)=\log|1+z| + i Arg(1+z)$.

Por tando, en $\mathbb{D}$ se cumple
\[(\log F)'=\frac{F'(z)}{F(z)} = α\frac{f'(z)}{f(z)} = α \log(f)'=(α\log f)' \implies\]
\[\implies \log F - α \log f = \text{ cte }\]

Vamos ahora a determinar cuál es esa constante. Para ello evaluamos en 0.
\[\text{ cte } =\log |F(0)|+i\cdot Arg(F(0))-α\log |f(0)|-i \cdot arg(f(0))\]
puesto que $F(0)=f(0)=$ tenemos que la constante es una diferencia de ángulos equivalentes, es decir, un múltiplo de 2π.

\[\text{ cte } = 2kπi \implies \log F = α\log f +2kπi \implies F = e^{α\log f +2kπi} \implies\footnote{Separando la exponencial en producto y la segunda exponencial en cos+isen}F = e^{α \log f} = (1+z)^α \]
\end{problem}

%%%%%%%%%%%%%%%%%%%%%%%%%%%%%%%%%%%%%%%%%%%%%%%%%%%%%%%%%%%%%%%%%%%%%%%%
%%%%%%%%%%%%%%%%%%%%%%%%%%%%%%%%%%%%%%%%%%%%%%%%%%%%%%%%%%%%%%%%%%%%%%%%
%%                                                                    %%
%%                            HOJA 4                                  %%
%%                                                                    %%
%%%%%%%%%%%%%%%%%%%%%%%%%%%%%%%%%%%%%%%%%%%%%%%%%%%%%%%%%%%%%%%%%%%%%%%%
%%%%%%%%%%%%%%%%%%%%%%%%%%%%%%%%%%%%%%%%%%%%%%%%%%%%%%%%%%%%%%%%%%%%%%%%
\newpage
\section{Hoja 4}

\begin{problem}[1]
Calcule $\int_α |z|\bar{z}dz$, donde $α$ es el camino cerrado compuesto por la semicircunferencia superior de $|z|=1$ y el segmento $-1\leq x \leq 1$; $y=0$, con orientación positiva.

\solution

\doneby{Pedro}

Lo que haremos será separar la curva en dos partes: la semicircunferencia $S$ y la recta $R$. Así, considerando $z=x+iy$, nos queda
\[\int_α |z|\bar{z}dz = \int_S |z|\bar{z}dz + \int_R |z|\bar{z}dz = \]
\[= \int_0^π \left(\cos(t)-i\sin(t)\right)\left(-\sin(t)+i\cos(t)\right)dt + \int_0^2 |1-t|(1-t)dt=πi+0= πi\]

\end{problem}

\begin{problem}[2]
¿Es cierto que $Re\{\int_α f(z) dz\} = \int_α Re\{f(z)\}dz$ para cualquier $f$, función continua compleja?.

Razone la respuesta
\solution

\doneby{Pedro}

No.

Tomamos como contraejemplo la función $f(z)=i$ y como curva $α(t)=(t,t)$. Así obtenemos

\[\int_α f(z)dz = \int_0^1 i(1+i)dt = i-1 \implies Re\{\int_α f(z) dz\}= -1\]
\[\int_α Re\{f(z)\}dz = \int_α 0dz = 0\]

\end{problem}

\begin{problem}[3]
Calcule
\[\int_α \frac{z}{\bar{z}}dz\]
donde $α$ es el camino que va de $-3$ a $-1$ a lo largo del eje real, después va de -1 a 1 siguiendo la semicircunferencia superior del disco unidad, luego va del 1 al 3 a lo largo del eje real y regresa a -3 por la semicircunferencia superior del círculo $|z|=3$.

\solution
\newpage
\doneby{Pedro}

Como en el ejercicio anterior dividimos la curva en partes, según se describe en el enunciado, con lo que nos queda:
\begin{equation*}
\begin{aligned}
\int_α \frac{z}{\bar{z}} dz = \int_0^2dt & +\int_{-π}^0 \frac{\cos(t)+i\sin(t)}{\cos(t)-i \sin(t)}(-\cos(t) + i\sin(t))dt +\int_0^2dt+\\ & \int_0^{π} \frac{\cos(t)+i\sin(t)}{\cos(t)-i \sin(t)}(-\cos(t) + i\sin(t))dt=\\
 =4+0=4
\end{aligned}
\end{equation*}
\end{problem}

\begin{problem}[4]
Demuestra que si $|a|<R$, entonces
\[\int_{|z|=R} \frac{|dz|}{|z-a||z+a|} < \frac{2π R}{R^2-|a|^2} = \]

\solution

\doneby{Pedro}

\[\int_{|z|=R} \frac{|dz|}{|z-a||z+a|} = \int_0^{2π}\frac{Rdt}{|R\cos(t)+iR\sin(t)-a||R\cos(t)+iR\sin(t)+a|} = ... =\]
\[\int_0^{2π}\frac{Rdt}{\sqrt{R^4+a^4+2R^2a^2-4a^2R^2\cos^2(t)}} < \int_0^{2π}\frac{Rdt}{\sqrt{R^4+a^4+2R^2a^2-4a^2R^2}} = \int_0^{2π}\frac{Rdt}{R^2-a^2}=\]
\[=\frac{2πR}{R^2-|a|^2}\]

\end{problem}

\begin{problem}[5]
Sea γ el cuadrado en $\cplex$ con vértices $\pm 1$ $\pm i$. Acote el valor absoluto de las siguientes integrales

\ppart
\[\int_{|z|=1} \frac{dz}{2-z^3}\]
\ppart
\[\int_{|z|=1} \frac{e^z}{z^2}\]
\ppart
\[\int_γ (\cos z)^2dz\]
\solution
\newpage
\spart
\[\int_{|z|=1} \frac{dz}{2-z^3} = 0\]
puesto que la función es holomorfa en $|z| \leq 1$

Por el Teorema de Cauchy, si $f$ es holomorfa en Ω simplemente conexo y α es un camino cerrado en Ω sabemos directamente que la integral sobre el camino será 0.

\spart
\doneby{Pedro}

En esta ocasión no podemos aplicar el Teorema de Cauchy, pues todo Ω simplemente conexo que tomemos que contenga la curva dada contendrá al origen y en ese punto la función no es holomorfa.

Ayudándonos de la fórmula integral de Cauchy tenemos:
\[\int_α \frac{f(w)}{(w-0)^2}dw=Ind_α(0)f'(0)2π = 2π\]


\spart
\[\int_{\text{γ}} (\cos z)^2dz =0\]
Nuevamente tenemos una función holomorfa en un dominio simplemente conexo que estamos integrando sobre una curva. Aplicando el Teorema de Cauchy tenemos directamente que la integral será 0.

\end{problem}

\begin{problem}[6]
Sea γ el arco del círculo $|z|=2$ comprendido en el primer cuadrante. Verifique que
\[\left| \int_γ \frac{dz}{z^2+1}\right|\leq \frac{π}{3}\]

\solution
\[\left| \int_γ \frac{dz}{z^2+1}\right| \leq \int_γ \left| \frac{dz}{z^2+1}\right| = \int_γ \frac{|dz|}{|z^2+1|} \leq\footnote{$|z|^2-1\leq|z^2+1|\leq |z|^2+1 \implies \frac{1}{|z^2+1|} \leq \frac{1}{|z|^2-1}$} \int_γ \frac{|dz|}{|z^2|-1}\leq\footnote{$|z|\leq2,\quad \frac{1}{|z^2|-1} \leq \frac{1}{4-1}$, y $\int_{\gamma} |dz| = \text{long}(\gamma)$} \frac{1}{3}\left(π\right)\footnote{$\gamma$ es un cuarto de circunferencia de radio 2, luego long($\gamma$)$\eqexpl{r=2}\frac{1}{4}\cdot 2 \cdot 2\pi = \pi$} \]

\end{problem}

\begin{problem}[7]
Sea $P(z)$ un polinomio y sea $γ$ el círculo de radio $R$ orientado positivamente, pruebe que
\[\int_γ \overline{P(z)}dz = 2πiR^2\overline{P'(0)}\]

\solution

\doneby{Pedro}

Atendiendo a la fórmula integral de Cauchy tenemos
\[Q'(0)2πi= \int_γ \frac{Q(z)}{z^2}dz\]
y es claro que si llamamos $P(z)=\overline{Q(z)}$ seguimos pudiendo aplicar la función ya que el conjugado de un polinomio en complejos sigue siendo un polinomio y por tanto sigue siendo holomorfo \textcolor{blue}{TRIPLE}.

Por tanto tenemos
\[\overline{P'(0)}2πi=\int_γ\frac{\overline{P(z)}}{z^2}dz=\frac{1}{R^2}\int_0^{2π}\frac{\overline{P(R\cos(t)+iR\sin(t))}}{\cos(2t)+i\sin(2t)}(-R\sin(t)+iR\cos(t))dt\]


%Siendo $P(z)=a_0+a_1z+...+a_nz^n$ un polinomio genérico de grado $n$ cualquiera siendo $z$ un complejo tal que $z=x+iy$

%\[\int_γ \overline{P(z)} dz=\]
%\[= \int_0^{2π}\left(\overline{a_0}+\overline{a_1}(R\cos(t)-Ri\sin(t))+...+\overline{a_n}(R\cos(t)-Ri\sin(t))^n\footnote{Nos basamos en que $overline{P(z)}=P(\bar{z})$}\right)(-R\sin(t)+R\cos(t))dt =\]
%\[= \overline{a_0}(R\cos(t)+i\sin(t))+\frac{\overline{a_1}}{2}(R\cos(t)-i\sin(t))^2+...+\frac{\overline{a_n}}{n+1}(R \cos(t)-i\sin(t))^{n+1}|_{(0,2π)}=0\]


\end{problem}

\begin{problem}[8]
Sea γ un camino simple y cerrado que encierra un área $S$. Demuestre que
\[S = \frac{1}{i}\int_γ xdz = -\int_γ ydz = \frac{1}{2i}\int_γ \bar{z}dz\]
\solution

Recordemos la fórmula de Green:\index{Fórmula de! Green}
\[\int\int_\Omega \left(\frac{\partial Q}{\partial x}-\frac{\partial P}{\partial y}\right)dx dy = \int_\gamma Pdx + Qdy \]

Y recordemos que el Área de una Superficie $\Omega$ es:
\[ \text{Área}(\Omega) = S = \int\int_\Omega 1 dx dy\] \index{Área de una superficie}

Vamos a ello:
\[ \frac{1}{i}\int_\gamma x dz = -i \int_\gamma x dz \eqexpl{dz=dx+idy} \int_\gamma \overbrace{-ix}^P dx + \overbrace{x}^Q dy \eqexpl{Green} \int\int_\Omega (1-0) dx dy \]
\[ - \int_\gamma y dz \eqexpl{dz=dx+idy} -\int_\gamma \overbrace{y}^P dx + \overbrace{iy}^Q dy \eqexpl{Green} - \int\int_\Omega (0-1) dx dy \]
\[ \frac{1}{2i}\int_\gamma \gor{z} dz = \frac{1}{2i} \int_\gamma (x-iy) \cdot (dx+idy) = \frac{1}{2i} \int_\gamma \overbrace{(x-iy)}^P dx + \overbrace{(ix+y)}^Q dy \eqexpl{Green} \int\int_\Omega \frac{1}{2i} (i-(-i))dxdy \]

Otra forma de hacer la última:
\[ \frac{1}{2i}\int_\gamma \gor{z} dz \eqexpl{f(z)=$\gor{z}$} \frac{1}{2i} \int\int_\Omega 2i\cdot \frac{\partial f(z)}{\partial \gor{z}} dxdy = \int\int_\Omega 1 dxdy \]

\end{problem}

\begin{problem}[9]
Sea γ un camino cerrado simple en $\mathbb{D}$, y $f$ una función holomorfa en $\mathbb{D}$ e inyectiva. Demuestre que
\[\int_γ \overline{f(z)}f'(z)dz\]
es un número imaginario puro.

\textbf{Ayuda:}  Escriba $f=u+iv$ y use un cálculo directo(largo) con la fórmula de Green, o bien aplique un cambio de variables adecuado y relaciones la integral con un área.
\solution

\doneby{Pedro}

Atendiendo a la sugerencia del enunciado escribimos
\[f(x,y)=u(x,y)+iv(x,y)\]
De forma similar a lo que se hace en el teorema de Cauchy tenemos:
\[\int_γ (u-iv)(u_x+u_y+i(v_x+v_y)(dx+idy) =\]
\[= \int_γ \left(u(u_x+u_y)+ui(v_x+v_y)-iv(u_x+u_y)+v(v_x+v_y)\right)(dx+idy)=\]

\[= \int_γ \left(u(u_x+u_y)+v(v_x+v_y)\right) dx +\left(v(u_x+u_y)-u(v_x+v_y)\right)dy+\]
\[+i\int_γ \left( u(u_x+u_y)-v(u_x+u_y)\right) dx + \left(v(v_x+v_y)+ u(v_x+v_y)\right)dy=\footnote{Aplicando el teorema de Green}\]
\[=...=\int\int_{\mathbb{D}}0 dxdy+i\int\int_Ω \text{ algo no nulo}\]

De modo que queda claro que se trata de un número imaginario puro.

\textcolor{blue}{La cuenta que nos llevaba al cero en la parte real la he hecho en sucio. Básicamente consiste en escribir la suma enorme que nos da el teorema de Green y empezar a cancelar términos. Algunos se cancelan directamente y otros aplicando que las funciones $u,v$ satisfacen las ecuaciones de Cauchy-Riemann por tratarse de una función $f$ holomorfa.}

\end{problem}

\begin{problem}[10]
Calcule las siguientes integrales

\ppart
\[\int_{|z-1|=2}\frac{dz}{z^2+3i}\]
\ppart
\[\int_{|z|=1}\frac{z^2\sin(z)dz}{(z+a)^3}, \ |a|\neq 1\]
\solution

\spart

Vamos a factorizar el denominador
\[z^2+3i= 0 \implies z^2=-3i \implies z^2=3e^{iπ/2} \implies z=re^{i\theta} \text{ con } r=\sqrt{3} \text{ y } \theta = \frac{-π}{4}+kπ\]

Es decir, las raíces del denominador son
\begin{align}
z_1 &=\sqrt{3}e^{-iπ/4} = \sqrt{3}\left( \frac{\sqrt{2}}{2}-i\frac{\sqrt{2}}{2}\right)\\
z_2 &=\sqrt{3}e^{-i(π/4+π)} = \frac{-\sqrt{6}}{2}(1-i)
\end{align}

Así que podemos escribir:
\[\frac{1}{z^2+3i} = \frac{A}{(z-z_1)}+\frac{B}{z-z_2}\]

Como hacemos con las integrales de bachillerato agrupamos la parte de la derecha en una única fracción, igualamos los numeradores y damos valores  a $z$ con lo que acabamos obteniendo los valores de $A$ y $B$
\[A= \frac{1}{z_1-z_2} \text{ y } B=-A\]

Ahora ya podemos calcular la integral

\[\int_{|z-1|=2}\frac{dz}{z^2+3i} = A\int_{|z-1|=2}\frac{1}{z-z_1}-A\int_{|z-1|=2}\frac{1}{z-z_2}\]

Ahora podemos comprobar que $z_1$ pertenece a la circunferencia de radio 2 centrada en el 1, mientras que $z_2$ no. Así la segunda integral es la integral de una función holomorfa sobre una curva cerrada y sabemos que es 0 por el teorema de Cauchy.

Para la primera integral, recordando la definición de índice y sabiendo que la curva da una única vuelta, tenemos:
\[\int_{|z-1|=2}\frac{dz}{z^2+3i} = 2πiA\]

\spart

Basándonos en la definición de índice y considerando $f(z)=z^2\sin(z)$, calculamos $f^{(2)}(z) = 2z\sin(z) + 2z\cos(z) + 2z\cos(z) -z^2\sin(z)$ y en este caso $z=-a$



\[\int_{|z|=1}\frac{z^2\sin(z)dz}{(z+a)^3} = \frac{2πi}{2!}f^{(2)}(-a) = \frac{2πi}{2!}\left(2\sin(-a)-4a\cos(-a)+a^2\sin(-a)\right)\]

Hemos tomado que el índice es 1, ya que la una circunferencia sólo da una vuelta alrededor del punto. Este resultado es cierto suponiendo $|a| < 1$ y que el punto queda dentro. Si el punto quedase fuera, $|a| > 1$, entonces el índice sería 0. Esto es fácil de comprobar, porque al estar el punto fuera, la función queda holomorfa en el interior y podemos aplicar el teorema de Cauchy.

\end{problem}

\begin{problem}[11]
Calcule las siguientes integrales trigonométricas usando la integración sobre la circunferencia y la fórmula integral de Cauchy

\ppart
\[\int_0^{2π}\frac{1}{2+\cos(t)}dt\]

\ppart
\[\int_0^{2π}\frac{\cos(2t)}{5-4\sin(t)}dt\]
\solution

\spart
\[\int_0^{2π}\frac{1}{2+\cos(t)}dt = \int_0^{2π}F(\cos(t),\sin(t))dt\]

Este tipo de integrales se resuelven fácilmente mediante el cambio de variables:
\[z= e^{it}=\cos(t)+i\sin(t) \ \ \ \frac{1}{z}=e^{-it}=\cos(t)-i\sin(t)\]
\[\cos(t)=\frac{z+\frac{1}{z}}{2} \ \ \sin(t)=\frac{z-\frac{1}{z}}{2i}\]

Así la integral nos queda:
\[\int_0^{2π}\frac{1}{2+\cos(t)}dt = \int_{|z|=1}\frac{1}{2+\frac{z+1/z}{2}}\frac{1}{iz}dz = \frac{2}{i}\int_{|z|=1}\frac{dz}{z^2+4z+1}\]

Las dos raíces del denominador son $-2\pm 2\sqrt{3}$ y ambas se salen del círculo de radio 1 de modo que estamos integrando una función holomorfa sobre una curva cerrada y por el Teorema de Cauchy tenemos que esta integral es 0.

\spart

Aplicamos el mismo procedimiento del ejercicio anterior para resolver la integral:
\[\int_0^{2π}\frac{\cos(2t)}{5-4\sin(t)}dt = \int_{|z|=1} \frac{\frac{z^2+1/z^2+2}{4}-\frac{z^2+1/z^2-2}{-4}}{5-4\frac{z-1/z}{2i}}\frac{1}{iz}dz = \int_{|z|=1}\frac{\frac{z^2+1/z^2}{2}}{5iz-2z^2+2} dz=\]
\[=\frac{1}{2}\int_{|z|=1}\frac{z^4+1}{-2z^4+5iz^3+2z^2} dz= \frac{1}{2}\int_{|z|=1}-\frac{1}{2}+\frac{10iz^3+4z^2+1}{-2z^4+5iz^3+2z^2}dz\]
Podemos factorizar el denominador como:
\[-2z^4+5iz^3+2z^2 = z^2\left(z+\frac{-5+3i}{4}\right)\left(z+\frac{-5-3i}{4}\right)\implies\]
\[\implies \frac{10iz^3+4z^2+1}{-2z^4+5iz^3+2z^2}=\frac{A}{z^2}+\frac{B}{z+\frac{-5+3i}{4}}+\frac{C}{z+\frac{-5-3i}{4}}\]

La integral sobre los dos segundos sumandos será 0 por el Teorema de Cauchy así como sobre la constante $-1/2$, de modo que nos basta con conocer el valor de $A$ que vemos es
\[A= \frac{1}{2.125}\]

Es decir, tenemos que nuestra integral inicial es igual a:
\[\int_0^{2π}\frac{\cos(2t)}{5-4\sin(t)}dt =  2.125\int_{|z|=1}\frac{dz}{z^2} = 2πi \cdot 2.125 \cdot f^{(n)}(0) = 0 \text{ puesto que } f(z)=1\]
\end{problem}

\begin{problem}[12]
Sea γ la circunferencia unidad orientada positivamente. Calcule las siguientes integrales:

\ppart
\[\int_γ z \sin z^2 dz\]

\ppart
\[\int_γ \frac{1-\cos(z)}{z^2}\]

\ppart
\[\int_γ \frac{\sin(z)}{z} dz\]

\ppart
\[\int_γ \frac{e^z}{z^2}dz\]

\ppart
\[\int_γ \frac{2}{1-4z^2}dz\]
\solution

\spart
\[\int_γ z \sin z^2 dz = 0\]
por el teorema de Cauchy, pues se trata de una función holomorfa sobre una curva cerrada.

\spart
Aplicando fórmula integral de Cauchy en el disco.

\[\int_γ \frac{1-\cos(z)}{z^2} = 2πif^{(1)}(0)=2πi·\sin(0) = 0\]

Podemos comprobarlo haciendo las integrales como en el ejercicio y dándoselo a \href{http://www.wolframalpha.com/input/?t=crmtb01&f=ob&i=int%20(1-cos(cos(t)%20%2B%20i%20*%20sin(t)))%2F((cos(t)%20%2B%20i%20*%20sin(t))%5E2)%20(-cos(t)%20%2B%20i*sen(t))%20dt%20from%200%20to%202*pi}{Wolfram} para que integre.

\spart
\[\int_γ \frac{\sin(z)}{z} dz = \sin(0)\cdot 2πi = 0\]
aplicando la fórmula integral de Cauchy en el disco.

\spart
\[\int_γ \frac{e^z}{z^2}dz = e^02πi=2πi\]
aplicando la fórmula integral de Cauchy en el disco.

\spart


\[\int_γ \frac{2}{1-4z^2}dz = \int_γ \frac{1/2}{1/2+z}dz+\int_γ \frac{-1/2}{-1/2+z}dz = πi-πi=0\]
La segunda igualdad se obtiene aplicando la fórmula integral de Cauchy en el disco tomando $f(z)=1/2$

Y sino, volvemos a sustituir $z=\cos(t)+i·\sin(t)$ integrando entre 0 y 2π y se lo damos a \href{http://www.wolframalpha.com/input/?i=int+%281%2F4*%28cos%28t%29+%2B+i+*+sin%28t%29%29%5E2+%28-cos%28t%29+%2B+i*sen%28t%29%29+dt+from+0+to+2*pi}{Wolfram}

\end{problem}

\begin{problem}[13]
Calcule la integral
\[\int_0^{2π}(\cos(t))^{2n}dt\]
¿Cuál es el límite
\[\lim_{n \to \infty}\sqrt{n}\int_0^{2π}(\cos(t))^{2n}dt?\]

\textbf{Sugerencia:} Calcule la integral de línea
\[\int_{|z|=1}\left( z + \frac{1}{z}\right)^{2n}\frac{dz}{z}\]
usando el desarrollo binomial.

\solution

\doneby{Pedro}

Vamos a utilizar nuevamente el truco que aprendimos en el problema \textbf{11}:

\[\int_0^{2π}(\cos(t))^{2n}dt=\int_{|z|=1}\left(\frac{z+1/z}{2}\right)^{2n} \frac{1}{zi}dz = \frac{1}{2^{2n}i}\int_{|z|=1}\left(z+\frac{1}{z}\right)^{2n}\frac{1}{z}dz=\]
\[=\frac{1}{i2^{2n}}\int_{|z|=1}\sum_{j=0}^{2n}{2n \choose j }z^{2n-2j-1}dz=\footnote{Los sumanos con exponente positivo dan como resultado 0 aplicando el Teorema de Cauchy}\frac{1}{i2^{2n}} \int_{|z|=1} \sum_{j=n}^{2n}{2n \choose j }z^{2n-2j-1}dz \eqreason{Cambio de variable: $k=j-n \iff j=k+n$} \]
\[\frac{1}{i2^{2n}} \int_{|z|=1} \sum_{k=0}^{n}{2n \choose k }\frac{1}{z^{2k+1}}dz=\frac{1}{i2^{2n}} \sum_{k=0}^{n}{2n \choose k+n } \int_{|z|=1}\frac{1}{z^{2k+1}}dz \eqreason{salvo el primer término ($k=0$), $1^{(2*k)}=0$} \]
\[=\frac{1}{i2^{2n}}2πi \cdot 1 \cdot 1 {2n \choose n } =\frac{π}{2^{2n-1}} {2n \choose n } \]

Al calcular el límite tenemos 0
\end{problem}

\begin{problem}[14]
Sea Ω un dominio en $\cplex$ y sea $f$ una función holomorfa en Ω tal que para un cierto $M>0$ se tiene que $|f(z)|\leq M$ para todo $z \in Ω$. Pruebe que
\[|f'(z) | \leq \frac{M}{\text{distancia}(z, \partial Ω)}\]
donde $\partial Ω$ denota la frontera de Ω.

\textbf{Sugerencia:} Sea $r$< distancia($z,\partial Ω$). Use la fórmula integral de Cauchy en $\mathbb{D}(z,r)$

\solution
\doneby{Pedro}

Por la fórmula integral de Cauchy sabemos que
\[|f'(z)| = \left| \frac{1}{2πi}\int_{\partial \mathbb{D}(z,r)}\frac{f(w)}{(w-z)^2}dw \right|\leq  \left| \frac{1}{2πi}\right| \left|\int_{\partial \mathbb{D}(z,r)}\frac{M}{(w-z)^2}dw\right|\leq\left|\int_{\partial \mathbb{D}(z,r)}\frac{M}{(w-z)^2}dw\right|=\]
\[=M \left|\int_{\partial \mathbb{D}(z,r)}\frac{dw}{(w-z)^2}\right|\leq M \int_{\partial \mathbb{D}(z,r)}\left|\frac{dw}{(w-z)^2}\right| \leq M \int_{\partial \mathbb{D}(z,r)}\frac{|dw|}{|(w-z)^2|} \leq M \int_{\partial \mathbb{D}(z,r)}\frac{|dw|}{|(w-z)|^2}\]

Llegados a este punto sólo tenemos que ver que la distancia entre $w$ y $z$ siempre será mayor que la distancia desde $z$ a la frontera sobre la cual estamos integrando. Es decir, en el denominador siempre van a quedar distancias desde un punto $z$ a la frontera y estas distancias siempre serán mayores, por definición, que la distancia(z,$\partial Ω$).

Por tanto nos queda
\[|f'(0)| \leq \frac{M}{\text{distancia}^2(z, \partial Ω)} \int_{\partial \mathbb{D}(z,r)}|dw|\leq \frac{M}{\text{distancia}^2(z, \partial Ω)}\text{distancia}^2(z, \partial Ω)\]

con lo que finalmente obtenemos la cota buscada.

\end{problem}

\begin{problem}[15]
Demuestre que si $f$ es holomorfa en un abierto que contiene al disco unidad cerrado $\overline{\mathbb{D}}$, y si $|f(z)|=0$ cuando $|z|=1$, entonces $f(z)=0$ para todo $z \in \overline{\mathbb{D}}$.

\textbf{Ayuda:} Fórmula integral de Cauchy

\solution

\doneby{Pedro}

La única posibilidad de que el módulo de un complejo sea 0 es que el propio complejo sea 0.

Si construimos una sucesión de puntos tal que todos ellos tengan módulo 1 y sus ángulos sea $\frac{1}{n}$, tenemos que en todos ellos la función $f(z)=0$ y convergen al punto $1$, donde la función se anula.

Por el principio de los ceros aislados tenemos que la función debe ser idénticamente nula.

\end{problem}

\begin{problem}[16]
Sea $f$ una función holomorfa en $\{z \in \cplex \tq |z| < R_0\}$. Demuestre las siguientes fórmulas

\ppart
Si γ es la circunferencia de radio $R<R_0$ y $|w|<R$ entonces
\[f(w)=\frac{1}{2πi}\int_γ \frac{R^2-|w|^2}{(z-w)(R^2-z\bar{w})}f(z)dz\]
\ppart
Si $0 \leq r < R < R_0$ entonces
\[f(re^{it})=\frac{1}{2π}\int_0^{2π}\frac{R^2-r^2}{R^2-2Rr\cos(α-t)+r^2}f(Re^{iα})dα\]

\solution

\doneby{Pedro}

\spart
\[\frac{1}{2πi}\int_γ \frac{R^2-|w|^2}{(z-w)(R^2-z\bar{w})}f(z)dz = \frac{1}{2πi}\int_γ \frac{A}{z-w}dz+\frac{1}{2πi}\int_γ \frac{B}{R^2-z\bar{w}}dz\]

Y operando tenemos que
\begin{align}
(R^2-|w|^2) f(w) &= A (R^2-|w|^2) \implies A = f(w)\\
(R^2-|w|^2) f(R^2/\bar{w}) &= B(R^2/\bar{w}-w) \implies B=\bar{w} f(R^2/\bar{w})
\end{align}

Por el Teorema de Cauchy en el disco tenemos que la primera integral vale $f(w)$. Si conseguimos demostrar que la segunda es 0 ya estará resuelto. Para conseguirlo basta con probar que es holomorfa en el dominio que estamos trabajando.

Tenemos que la función
\[\frac{\bar{w} f(R^2/\bar{w}}{R^2-z\bar{w}}\]
es holomorfa en todo punto salvo en $z= R^2/\bar{w} = w R^2/|w|^2 \implies |z|=R^2/|w| \geq R\footnote{|w|<R según el enunciado}$ por tanto este punto no entra dentro del dominio que estamos trabajando. Es decir, la segunda integral es la integral de una función holomorfa sobre una curva cerrada y, por el teorema de Cauchy, sabemos que vale 0.

\spart

\end{problem}

%%%%%%%%%%%%%%%%%%%%%%%%%%%%%%%%%%%%%%%%%%%%%%%%%%%%%%%%%%%%%%%%%%%%%%%%
%%%%%%%%%%%%%%%%%%%%%%%%%%%%%%%%%%%%%%%%%%%%%%%%%%%%%%%%%%%%%%%%%%%%%%%%
%%                                                                    %%
%%                            HOJA 5                                  %%
%%                                                                    %%
%%%%%%%%%%%%%%%%%%%%%%%%%%%%%%%%%%%%%%%%%%%%%%%%%%%%%%%%%%%%%%%%%%%%%%%%
%%%%%%%%%%%%%%%%%%%%%%%%%%%%%%%%%%%%%%%%%%%%%%%%%%%%%%%%%%%%%%%%%%%%%%%%
\newpage
\section{Hoja 5}

\begin{problem}[1]
Sea $\mathbb{D}$ el disco unidad. Demuestre que no hay ninguna función $f \in H(\mathbb{D})$ tal que
\[f \left( \frac{1}{n}\right) = \frac{1}{n} = f\left( -\frac{1}{n} \right)\]
para $n=1,2,3,...$

\solution
La primera igualdad la cumple la función identidad. La sucesión de puntos en que estas dos funciones (la dada y la identidad) valen lo mismo tiene un punto de acumulación en $\mathbb{D}$, el 0, de modo que la función debe ser la identidad.

Está claro que la $f$ buscada no puede ser la identidad por que no cumple la segunda propiedad de modo que no existe ninguna función que cumpla las características pedidas.

\end{problem}

\begin{problem}[2]
Halle razonadamente todas las funciones holomorfas en el disco $\mathbb{D}=\{z : |z-1|<1\}$ y que allí satisfacen la condición
\[f \left( \frac{n}{n+1}\right) = 1-\frac{1}{2n^2+2n+1}\]

\solution
\doneby{Pedro}

\[x=\frac{n}{n+1} \implies nx+x=n\implies n = \frac{x}{1-x}\]

Aplicando este cambio de variable a la función mostrada tenemos:
\[f\left(\frac{n}{n+1}\right) = 1-\frac{1}{2n^2+2n+1}= f(x)=1-\frac{(1-x)^2}{2x^2+2x-2x^2+x^2-2x+1}\]
\[f(x)=\frac{2x}{x^2+1}\]

Es sencillo ver que al evaluar esta función en $\frac{n}{n+1}$ obtenemos la función inicial. Esta función es holomorfa en el disco, pues es holomorfa en todo el plano complejo.

Aplicando ahora una idea similar a la del apartado anterior, vemos que si hubiera otra función $g(x)$ tal que
\[g\left( \frac{n}{n+1}\right) = 1-\frac{1}{2n^2+2n+1}\]
tendríamos que las dos funciones en un conjunto de puntos del intervalo $(0,1)$ tendrían el mismo valor, y en ese conjunto de puntos hay uno de acumulación (el 1) y, por el teorema de continuación única, tenemos que $g=f$.

Es decir, sólo hay una función que cumpla la condición pedida y es la $f(x)$ calculada previamente.

\end{problem}

\begin{problem}[3]
Demuestra que si $f$ es holomorfa en $\mathbb{D}$ y
\[\left| f\left( \frac{1}{n}\right) \right|\leq \frac{1}{2^n} \text{ para } n\geq2\]
entonces $f$ es idénticamente cero en $\mathbb{D}$

\textbf{Sugerencia:} Como $f(0)=0$, entonces $f(z)=z^kg(z)$ con $g(z)$ holomorfa en $\mathbb{D}$ y $g(0)\neq 0$. Compruebe que esto es imposible.

\solution
\doneby{Pedro}

Tomando límites a ambos lados obtenemos, tal como se indica en la sugerencia, $f(0)=0$.
\[\lim_{n \to \infty} f\left(\frac{1}{n}\right) = 0=f(0)\]

Atendiendo a la sugerencia del enunciado tenemos
\[\left|f\left( \frac{1}{n}\right) \right| = \left| \left(\frac{1}{n}\right)^kg\left( \frac{1}{n}\right)\right|\leq \frac{1}{2^n} \implies \left| g\left( \frac{1}{n}\right) \leq \frac{n^k}{2^n}\right| \to 0 \implies g(0)=0\]

Pero decíamos que $f(z)=k^zg(z)$ con $g$ una función holomorfa que no se anula en el 0.


\obs La idea es que si $f$ tiene un cero de multiplicidad $k$ podemos factorizarla escribiendo $f(z)=z^kg(z)$ de modo que es imposible que se anule $g(z)$ en el 0, ya que en caso de hacerlo la multiplicidad del cero sería $k+1$. Como hemos forzado a que $g(0)=0$ llegamos a una contradicción, que implica que no existirá la $f$ con las condiciones pedidas.

\end{problem}

\begin{problem}[4]
Halle todas las funciones enteras tales que
\ppart
\[f(z)=f(z^2) \ \forall z \in \cplex\]
\ppart
\[f(2z)=2f(z) \ \forall z \in \cplex\]

\solution
\spart
\[f(z)=f(z^2) \iff \sum_{n=0}^{\infty}b_nz^{n} = \sum_{n=0}^{\infty}a_nz^{2n}\]
Así nos queda que los coeficientes de la serie deben cumplir:
\[f(x)= \left\{ \begin{array}{lcc}
             0 &   si  & \text{ n impar} \\
             \\ a_{n/2} &  si  & \text{ n par}
             \end{array}
   \right.\]

\spart
\doneby{Pedro}

\[f(2z)=2f(z) \iff \sum_{n=0}^{\infty} b_n2^nz^n = \sum_{n=0}^{\infty}a_n2z^n\]

Igualando los coeficientes de las series obtenemos que esto sólo puede ocurrir cuando la función $f(z)=z+cte$

\end{problem}

\begin{problem}[5]
Halle todas las funciones holomorfas en el disco unidad $\mathbb{D}$ que satisfacen
\[f(z^2)=f(z)+z\]
Compruebe que no existe ninguna función entera que cumpla esta condición

\solution

\doneby{Pedro}

Una función válida sería $g(x)=g(|\sqrt{x}|)+|\sqrt{x}|$ y, aplicando el principio de los ceros aislados vemos que no existen más funciones que la aquí indicada.

Para probar que no existe ninguna función entera que cumpla esta propiedad, basta con ver que esta función no es holomorfa en $\cplex$, lo cual es obvio pues sabemos que la raíz cuadrada no es una función entera.

\end{problem}

\begin{problem}[6]
Sea α un número irracional y $q=e^{2πiα}$. Demuéstrese que las únicas soluciones holomorfas de la ecuación funcional $f(z)=f(qz)$ en la corona $Ω = \{\frac{1}{2}<|z|<\frac{3}{2}\}$ son las funciones constantes.

\solution

\doneby{Pedro}

Si escribimos el desarrollo como serie de la función $f$ tenemos que
\[f(z)=\sum_{n=0}^{\infty}a_n z^n\]

Si forzamos a que la función cumpla la ecuación funcional dada nos queda
\[\sum_{n=0}^{\infty}a_n z^n = \sum_{n=0}^{\infty}a_n e^{2πiαn}z^n\]
y la única forma de que esta igualdad sea cierta es que todos los coeficientes coincidan, es decir:
\[a_n=a_ne^{2πiαn} \iff 1 = e^{2πiαn}\]
pero sabemos que esa última igualdad sólo es cierta si $n=0$. La única explicación posible es que la doble implicación sea falsa puesto que $a_n=0$ para todo $n>0$, con lo que nos queda que nuestra función $f$ es de la forma:
\[f = a_0 \implies f=cte\]
\end{problem}

\begin{problem}[7]
Demuestre que si $f$ es holomorfa en el disco unidad y $|f(z)|\leq 1-|z|$ allí, entonces $f=0$. ¿Puede una función holomorfa satisfacer $|f(z)| \geq 1/(1-|z|)$ para |z| < 1?
\solution
\end{problem}

A partir de aquí no se donde deberían ir los siguientes ejercicios. Ya los ordenaré cuando pueda.

\begin{problem}
Algo

\solution
Toda aplicación holomorfa y biyectiva, $\psi$, de $\mathbb{D}$ en $\mathbb{D}$ es de la forma
\[\psi(z)=e^{iα}\frac{z-a}{1-\bar{a}z}\]
par algún $α \in \real$ y $a \in \cplex$ con $|a|<1$.
..
.
.
.
.
.
y por tanto lleva $\partial \mathbb{D}$ en $\partial \mathbb{D}$ y como $T(0)=-e^{iα}a$ y $|T(0)|=|e^{iα}a|=|a|<1$ lleva $\mathbb{D}$ en $\mathbb{D}$

\obs Toda aplicación holomorfa y biyectiva $\appl{f}{\mathbb{D}}{\mathbb{D}}$
\end{problem}

\begin{problem}[9]
Supongamos que f es entera. Demuestre las siguientes afirmaciones:

\ppart Si $\displaystyle \lim_{z\to ∞} \frac{f(z)}{z} = 0$ entonces $f$ es constante.

\ppart Si existe $M>0$ tal que $|f(z)| \leq M |z^2|$ $∀z∈ℂ$, entonces $f$ es un polinomio de grado $\leq 2$ (de hecho es un múltiplo de $z^2$)
\solution

\doneby{Dejuan}

\spart Si lo pensamos, tiene toda la lógica del mundo, ya si el numerador fuera un polinomio, tendría que tener grado menor y la única posibilidad es constante. ¿Y si no es un polinomio?

Recurrimos al teorema de Liouville. Tenemos que $f$ es entera y que $|f(z)| < |z|$

\[
\lim_{z\to ∞} \frac{f(z)}{z}=0 \implies \lim_{z\to ∞} \frac{|f(z)|}{|z|} = 0
\]

Estamos en condiciones de aplicar el teorema de Liouville ($a=1,n=1,b=$alguno para que se de el límite) entonces $f(z)$ es un polinomio de grado $≤1$.

Si fuera de grado $1$, el límite sería el cociente de los coeficientes principales, por ello queda demostrado que tiene que ser un polinomio de grado 0, es decir, una constante.

\spart Es cierto por el teorema de Liouville. Si pide que demuestre el teorema (que se hace con integrales de Cauchy) no lo voy a hacer.

\end{problem}


\begin{problem}
Demuestre que si $a,b∈ℂ$ y $R>0$ es tal que $|a| < R, |b| < R$, entonces:

\[
\frac{1}{2π}\int_{|z| = R} \frac{f(z)}{(z-a)(z-b)} dz = \frac{f(a) - f(b)}{a-b}
\]

(\textbf{Ayuda}: Fracciones simples.) Use esta fórmula para probar el Teorema de Liouville. (Ayuda:
Haga que R → ∞.

\solution

\end{problem}
%%%%%%%%%%%%%%%%%%%%%%%%%%%%%%%%%%%%%%%%%%%%%%%%%%%%%%%%%%%%%%%%%%%%%%%%
%%%%%%%%%%%%%%%%%%%%%%%%%%%%%%%%%%%%%%%%%%%%%%%%%%%%%%%%%%%%%%%%%%%%%%%%
%%                                                                    %%
%%                            HOJA 6                                  %%
%%                                                                    %%
%%%%%%%%%%%%%%%%%%%%%%%%%%%%%%%%%%%%%%%%%%%%%%%%%%%%%%%%%%%%%%%%%%%%%%%%
%%%%%%%%%%%%%%%%%%%%%%%%%%%%%%%%%%%%%%%%%%%%%%%%%%%%%%%%%%%%%%%%%%%%%%%%
\newpage
\section{Hoja 6}
\subsection{Singularidades aisladas. Series de Laurent}
\begin{problem}
Halle las singularidades, identifique su tipo y calcule los correspondientes residuos para las siguientes funciones
\ppart $\frac{1}{z^2 + 2z + 1}$
\ppart $\frac{1}{z^3 - 1}$
\ppart $\frac{\cos(z) - 1}{z^2}$
\ppart $\frac{1}{z^2 \sin(z)}$
\ppart $\sin(\frac{1}{z^2})$

\solution

\end{problem}


\begin{problem}
Halle los desarrollos de Laurent de las siguientes funciones en los puntos que se indican:
\ppart $\cos (\frac{1}{z^2})$, $a = 0$
\ppart $\frac{1}{z^2 - 3z + 2}$, $a = 2$
\ppart $z^2 e^{1/(1 - z)}$, $a = 1$

\solution
\spart

\spart 
\textcolor{JungleGreen}{Sacado de unos apuntes aleatorios que han dejado unos de mates}

Queremos escribirlo de la manera $\sum a_n (z-2)^n$ porque nos piden en $a=2$. Entonces tenemos que desarrollarlo en potencias de $(z-2)$, como en al principio de curso.

\begin{equation}
\begin{align*}
\frac{1}{z^2 - 3z + 2} &= \frac{1}{(z-2)(z-1)} = (z-2)^{-1}·\frac{1}{z-1} = (z-2)^{-1}\frac{1}{1-(2-z)} = \\
 &(z-2)^{-1} \sum_{k=0}^{∞}(2-z)^k =  (z-2)^{-1} \sum_{k=0}^{∞}(-1)^k(z-2)^k =\\
 &\sum_{n=-1}^{∞}(-1)^{n+1}(z-2)^n 
\end{align*}
\end{equation}

\end{problem}

\begin{problem}
Sean f y g dos funciones holomorfas en ${\mbox D(a, r) = \set{z \in \cplex : |z - a| < r}}$. Demuestre que si f tiene un cero de orden n en a, y g tiene un cero de orden m en a, con $n < m$, entonces $f / g$ tiene un polo de orden $m - n$ en a. Calcule el residuo.
\solution

\end{problem}

\subsection{Teorema de los residuos. Integrales impropias}
\begin{problem}
Calcule las siguientes integrales
\ppart $\int_{\abs{z}=t} \frac{\dif z}{z^2 - z + 1}$, $t \neq 1$
\ppart $\int_{\abs{z}=1} \frac{1 + z}{1 - \cos(z)} \dif z$
\solution

\end{problem}

\begin{problem}
Evalúe $\int_{\gamma} z^n e^{1/z} \dif z$, donde $n \in \nat$ y $\gamma$ es una circunferencia (orientada positivamente) que rodea al origen.
\solution

\end{problem}

\begin{problem}
Demuestre razonadamente que $\int_{0}^{\pi / 2} \frac{\dif \theta}{a + \sin^2(\theta)} = \frac{\pi}{2 \sqrt{a^2 + a}}$, donde $a > 0$
\solution

\end{problem}

\begin{problem}
Demuestre que las siguientes igualdades utilizando el teorema de los residuos (\ref{Th:residuos}) con la elección adecuada de camino
\ppart $\int_{0}^{\infty} \frac{\dif x}{(x^2 + 1)^2} = \frac{\pi}{4}$
\ppart $\int_{-\infty}^{\infty} \frac{x^2}{(x^2 + 2x + 2)^2} \dif x = \pi$
\solution

\end{problem}

\begin{problem}
Calculad\footnote{gusanos!}:
\ppart $\int_{-\infty}^{\infty} \frac{\sin(x)}{(x - 1)^2 +1} \dif x$
\ppart $\int_{0}^{\infty} \frac{\dif x}{1 + x^3}$
\ppart $\int_{-\infty}^{\infty} \frac{\cos(\pi x)}{4x^2 - 1} \dif x$
\solution

\end{problem}

\begin{problem}
(*) Compruebe las igualdades para $\abs{p} < 1$, $q > 0$:
\ppart $\int_{0}^{\infty} \frac{x^p}{1 + x^2} \dif x = \frac{\pi}{2} \left( \cos(\frac{\pi p}{2}) \right)^{-1} $
\ppart $\int_{0}^{\infty} \frac{x^p}{(x + q)^2} \dif x = \frac{\pi \cdot p \cdot q^{p-1}}{\sin{(\pi p)}}$
\solution

\end{problem}

\subsection{Principio del argumento. Teorema de Rouché}
\begin{problem}
Halle el número de ceros\footnote{que sacarás} del polinomio:
\[ p(z) = z^3 - 7z^2 + 2z - 3 \]
en la corona $\set{z \in \cplex: 1 < \abs{z} < 10 }$
\solution

\end{problem}

\begin{problem}

¿Cuántos zeros tiene la ecuación $e^z - 4 z^n + 1 = 0$ en el disco unidad?
\solution

\end{problem}

\begin{problem}
Encuentre razonadamente todos los polinomios mónicos:
\[ p(z) = z^n + a_{n-1} z^{n-1} + ... + a_1 z + a_0 \]
para los cuales $\abs{p(z)} < 1$ $\forall z \in \mathbb{S}^1$.
\solution

\end{problem}

\begin{problem} {\bf Ejercicio de examen}

Demuestre que el polinomio $p(z) = z^4 + iz + 1$ tiene exactamente 2 ceros:
\ppart en el semiplano derecho $D = \set{z \in \cplex : \Re(z) > 0}$
\ppart en el semiplano superior $\Omega = \set{z \in \cplex : Im z > 0}$
\solution

\end{problem}

\begin{problem} {\bf Ejercicio de examen, pero chungo.}

Pruebe que la ecuación $z = \lambda - e^{-z}$, con $\lambda \in \real$, $\lambda > 1$, tiene en $\set{z : \Re(z) > 0} $ una única raíz y, además, ésta es real.
\solution

\end{problem}

\subsection{Teorema de Morera}
\begin{problem}

La función g viene dada por $g(z) = \int_{0}^{\pi} \cos(z + t) \dif t$, para $z \in \cplex$. Demuestre que g es entera.

{\it Hint:} El teorema de Morera nos permite demostrar que funciones definidas mediante ciertas integrales son también holomorfas, lo cual nos proporciona más ejemplos aparte de las fórmula explícitas y series de potencias.
\solution

\end{problem}

\begin{problem}
Sea $f$ una función continua en $\cplex$ y holomorfa en el plano menos un segmento $ [ a, b ] $ del eje real. Utilizando el teorema de Morera, demuestre que $f$ es entera.
\solution

\end{problem}


%%%%%%%%%%%%%%%%%%%%%%%%%%%%%%%%%%%%%%%%%%%%%%%%%%%%%%%%%%%%%%%%%%%%%%%%
%%%%%%%%%%%%%%%%%%%%%%%%%%%%%%%%%%%%%%%%%%%%%%%%%%%%%%%%%%%%%%%%%%%%%%%%
%%                                                                    %%
%%                            HOJA 7                                  %%
%%                                                                    %%
%%%%%%%%%%%%%%%%%%%%%%%%%%%%%%%%%%%%%%%%%%%%%%%%%%%%%%%%%%%%%%%%%%%%%%%%
%%%%%%%%%%%%%%%%%%%%%%%%%%%%%%%%%%%%%%%%%%%%%%%%%%%%%%%%%%%%%%%%%%%%%%%%
\newpage
\section{Hoja 7}
\subsection{Teorema de la aplicación abierta, Principio del módulo máximo y lema de Schwarz}

\begin{problem}
Determine razonadamente todas las funciones $f$, holomorfas en el disco unidad, tales que
$\Re(f(z)) \cdot \Im(f(z)) = 0, \ \forall z \in $ disco unidad.

\solution
\textcolor{red}{Edu: No estoy seguro si esto llega a resolver el ejercicio o no, pero me da que no:}

Toda aplicación holomorfa y biyectiva $\varphi: \mathbb{D} \rightarrow \mathbb{D}$ es de la forma:
\[ \varphi(z) = e^{i\alpha} \frac{z-a}{1-\conj{a}z} \text{, para algún $\alpha \in \real$ y $a \in \cplex$, $\abs{a} < 1$} \]
Lo comprobamos:
\begin{gather*}
\abs{z} = 1 \implies \abs{\varphi} = \abs{\frac{z-a}{1-\conj{a}z}} \eqreason{$z \conj{z} = \abs{z}^2 = 1$} \abs{\frac{z - a z \conj{z}}{1 - \conj{a} z}} = \abs{\frac{z(1 - a \conj{z})}{1 - \conj{a} z}} = \abs{\frac{\conj{(1 - \conj{a} z)}}{1 - \conj{a} z}} = 1
\end{gather*}
Por tanto, lleva $\partial\mathbb{D}$ en $\partial\mathbb{D}$.
Como $\varphi(0) = -e^{i\alpha} a$, y $\abs{\varphi(0)} = \abs{-e^{i\alpha} a} = \abs{a} < 1 \implies$ lleva $\mathbb{D}$ en $\mathbb{D}$.

\doneby{Edu}

Sea $f(z) = u + iv$, $u,v \in \real$.
\begin{gather*}
0 = \Re(f(z)) \cdot \Im(f(z)) = u\cdot v \implies u = 0 \text{ o } v = 0 \ \forall z \in \mathbb{D}
\end{gather*}
Luego tenemos 3 casos:
\begin{itemize}
	\item $v \equiv 0$, ie, $f(z)$ es una función real.
	\item $u \equiv 0$, ie $f(z)$ es un complejo puro.
	\item $u = v = 0$, ie, $f \equiv 0$.
	\item $\forall \in \mathbb{D}$, $u$ y $v$ van alternándose para tomar el valor 0.
\end{itemize}
Por el teorema de la aplicación abierta, como $f$ es holomorfa en el disco (que es un dominio), $f$ debe ser constante (casos 1,2 y 3) o $f$ debe ser una aplicación abierta (caso 4).

\textcolor{red}{Edu: no sé explicar la cuarta opción mejor. Solo sé que debe ser una aplicación abierta en ese caso, que debe ser del tipo como la de la explicación de arriba. Probablemente sean espirales que vayan saltando de un eje en el otro de manera alterna.}
\end{problem}

\begin{problem}
Sea $\Omega$ un dominio acotado en $\cplex$ y $f,g$ funciones holomorfas en $\Omega$ y continuas en $\conj{\Omega}$ % en realidad quiero poner cierre
Demuestre las siguientes cosas:
\ppart Si $\abs{f(z)} = \abs{g(z)}$ en $\partial\Omega$ y $f(z)g(z) \neq 0$ en $\conj{\Omega}$, entonces $f(z) = c g(z)$ con $\abs{c} = 1$.
\ppart Si $\Re(f) = \Re(g)$ en $\partial\Omega$, entonces $f = g + i\alpha$ con $\alpha \in \real$.
\solution

\spart
% Ideas:
% Por el principio del módulo máximo, $f$ y $g$ alcanzan en $\partial\Omega$ su módulo máximo.
% Definimos $h(z) = f(z)g(z)$:
% \[ \abs{h(z)} = \abs{f(z)g(z)} \stackrel{z_0 \in \partial\Omega}{\leq} \abs{f(z_0)}\abs{g(z_0)} \eqreasonup{$\abs{g(z_0)} = \abs{f(z_0)}$} \abs{f(z_0)}^2 \]
% Luego $\abs{h(z)}$ alcanza un máximo en $\partial\Omega$.
% Además, sabemos que $h(z) \neq 0$ en todo $\conj{\Omega}$.
\spart
% aquí hay que aplicar algún teorema que no es ninguno de los 3 del subtitulo de sección. O quizás me falta algo por ver

\end{problem}

\begin{problem}
Sea $\Omega$ un dominio en $\cplex$ y $\gamma$ una curva simple y cerrada, $C^1$ a trozos, contenida en $\Omega$ junto con su dominio interior. Si $f \in \mathcal{H}(\Omega$) y $\abs{f} \equiv 1$ en la curva $\gamma$, demuestre que entonces o bien $f$ tiene algún zero en el interior de $\gamma$, o bien $f$ es constante en $\Omega$.

{\it Hint}: Aplique el Principio del módulo máximo a la función $\frac{1}{f}$.

\solution
\doneby{Edu} Sigamos la indicación.

Sea $g(z) = \frac{1}{f(z)}$. Sea $A =$ la región encerrada por la curva $\gamma$, con $\partial A = \gamma$. Por la definición de $\gamma$, A es un dominio.

Si $z_0 \in A$ tal que $\abs{g(z_0)} \geq \abs{g(z)} \ \forall z \in A \implies g(z)$ es constante en A por el principio del módulo máximo. $\implies c = \frac{1}{f(z)}$

Si $c \neq 0 \implies f(z) = \frac{1}{c} \implies f \equiv $ constante.

\textcolor{red}{Edu: y creo que la siguiente afirmación no es del todo correcta:}\\
Si $c = 0 \implies \lim\limits_{z \to z_0} \frac{1}{\abs{f(z)}} = 0$, con lo cual, $f(z_0) = 0$.

\end{problem}

\begin{problem}
Sea $f$ una función holomorfa en un dominio $\Omega$ simplemente conexo, $\Omega \neq \cplex$, y con valores en ese mismo dominio. Supongamos que en $\Omega$ hay dos puntos a y b, con $a \neq b$ tales que $f(a) = a$ y $f(b) = b$. Pruebe que entonces $f$ es la función identidad.

{\it Hint}: Pase a $\mathbb{D}$.
\solution


\end{problem}

\begin{problem}
Sea $\appl{f}{\mathbb{D}}{\mathbb{D}}$ una función holomorfa no constante. Demuestre:
\ppart $\frac{\abs{f(0)} - \abs{z}}{1 + \abs{f(0)}\abs{z}} \leq \abs{f(z)} \leq \frac{\abs{f(0)} + \abs{z}}{1 - \abs{f(0)}\abs{z}}$, para todo $z \in \mathbb{D}$.
\ppart $\abs{f'(w)} \leq \frac{1 - \abs{f(w)}^2}{1 - \abs{w}^2}$, $\forall z \in \mathbb{D}$.

{\it Hint}: Aplique el Lema de Schwarz a $\varphi_a \circ f$ ($1^{er}$ apartado) o a $\varphi_b \circ f \circ \varphi_a$ ($2^o$ apartado) con un a $\in \mathbb{D}$ apropiado en cada caso, donde $\appl{\varphi_a}{\mathbb{D}}{\mathbb{D}}$ es el automorfismo conforme del disco definido por $\varphi_a(z) = \frac{a - z}{1 - \conj{a} z}$. Conviene observar que $\varphi_a \circ \varphi_a$ es la identidad así que, por ejemplo, $f = \varphi_a \circ (\varphi_a \circ f)$.
\solution

\spart
Se nos ocurre que tenemos que tenemos que construir una aplicación que fije el 0, y como f no lo hace buscamos: $\mathbb{D}\xrightarrow{f}\mathbb{D}\xrightarrow{\varphi}\mathbb{D}$ tal que $\varphi(f(0)) = 0$.
Luego definimos $\appl{g = \varphi \circ f}{\mathbb{D}}{\mathbb{D}}$ holomorfa y tal que $g(0) = 0$:
\begin{gather*}
\abs{(\varphi \circ f)(z)} = \abs{g(z)} \stackrel{\text{Lema Schwarz}}{\leq} \abs{z} \implies \abs{\frac{f(z) - f(0)}{1 - \conj{f(0)} f(z)}} \leq \abs{z} \iff \\
\abs{f(z) - f(0)} \leq \abs{z} \abs{1 - \conj{f(0)} f(z)} \stackrel{\text{desig. triangular}}{\leq} \abs{z} \abs{1 + \conj{f(0)} f(z)}
\end{gather*}

Probamos la primera desigualdad:
\begin{gather*}
\abs{f(0)} - \abs{z} \leq \abs{z}(1 + \abs{f(0)}\abs{f(z)}) \iff\\
\abs{f(z)}(1 - \abs{f(0)}\abs{f(z)}) \leq \abs{z} \abs{f(0)} \iff\\
\abs{f(z)} \leq \frac{\abs{z} + \abs{f(0)}}{1 - \abs{f(0)}\abs{z}} \text{, porque } \abs{f(0)} < 1, \abs{z} < 1
\end{gather*}
Ahora probamos la segunda:
\[ \abs{f(0)} - \abs{f(z)} \leq \abs{z}(1 + \abs{f(0)}\abs{f(z)}) \iff \abs{f(z)} (1 + \abs{z}\abs{f(0)} \leq \abs{z} - \abs{f(0)} \]

$ $\hfill\qed

\spart
Misma idea. Muchas cuentas. Mucho que mirarme.

$ $\hfill\qed
\end{problem}

\begin{problem}
Sea $f$ función holomorfa en el disco unidad $\mathbb{D}$, tal que $\Re(f(z)) > 0$ para todo $z \in \mathbb{D}$ y además $f(0) = 1$. Usando una transformación de Möbius y el Lema de Schwarz, pruebe:
\ppart $\abs{f(z)} \leq \frac{1 + \abs{z}}{1 - \abs{z}}$ para cada $z \in \mathbb{D}$.
\ppart $\abs{f(z)} \geq \frac{1 - \abs{z}}{1 + \abs{z}}$ para cada $z \in \mathbb{D}$.
\solution

\end{problem}



\subsection{Aplicaciones conformes}
En los siguientes problemas denotaremos por $\mathbb{D}$ el disco unidad y por $\mathbb{H}$ al semiplano superior. Es decir, $\mathbb{D} = \set{z \in \cplex: \abs{z} < 1}$ y $\mathbb{H} = \set{z \in \cplex: \Im(z) > 0}$.

\begin{problem}
Describa la imagen mediante la transformación $f(z) = \frac{z - i}{z + i}$ de los siguientes conjuntos:
\ppart $\real$
\ppart $\set{z \in \cplex : \abs{z} < 2}$
\ppart $\mathbb{D}$
\ppart $\set{iy : y \in \real}$
\solution

\end{problem}

\begin{problem}
Halle una transformación de Möbius $T$ tal que $T(i) = -i$, $T(0) = 0$, $T(-1) = \infty$
\solution
\doneby{Edu}

Si utilizando la fórmula de razón doble:
\begin{gather*}
[z,i,0,-1] = [T(z),-i,0,\infty]\\
\frac{z(i + 1)}{(z + 1)i} = \frac{T(z)}{-i} \iff T(z) = -i \frac{z(i + 1)}{(z + 1)i} = -\frac{z(i + 1)}{(z + 1)}
\end{gather*}

\end{problem}

\begin{problem}
Halle una transformación de Möbius $T$ tal que $T(\mathbb{D}) = \mathbb{H}$ y $T(0) = 3 + 2i$
\solution
\textcolor{blue}{Hecho por Jorge}

Cuando hacemos la transformación de llevar el $\mathbb{D}$ a $\mathbb{H}$, no controlamos a dónde van a parar los puntos del interior del disco, ya que solo exigimos que la circunferencia vaya a parar al eje real, y que el interior del disco quede en $\mathbb{H}$.

De modo que la idea es hacer una transformación común $T$ de $\mathbb{D}$ a $\mathbb{H}$, tal que podamos dilatar $T(0)$ hasta $3+2i$ (para ello haremos $k·T(0) = 3+2i$). La dilatación no desplazará verticalmente el eje real, y así seguiremos permaneciendo en $\mathbb{H}$.

Nuestra $T$ llevará la circunferencia unidad en el eje real, y el interior del disco en la parte positiva del plano imaginario:
\begin{align*}
	-1 \to a\\
	i \to b\\
	1 \to -∞
\end{align*}

Con $a,b ∈ ℝ$ y $a>b$ para seguir la orientación. Por tanto:
\[ [z,-1,i,1] = [T(z),a,b,-∞] \]
\[\frac{(z-i)(-1-1)}{(z-1)(-1-i)} = \frac{T(z) - b}{a-b}\]
Luego:
\[T(z) = b + \frac{2(z-i)}{(z-1)(1+i)}(a-b)\]

Nuestro objetivo es que $T(0)$ lo podamos dilatar multiplicando por un $k ∈ ℝ$, para obtener $k·T(0) = 3+2i$:

\[T(0) = b + \frac{2i}{1+i}(a-b) = \frac{3+2i}{k}\]

Esto da un sistema con un grado de libertad (se saca operando e igualando parte real e imaginaria de ambos lados). $k=b=1$, $a=3$ es solución del sistema. Si nos fijamos, $k=1$ y por tanto la dilatación que hemos encontrado consiste en dejar fija $T(z)$. De modo que la solución será nuestra $T$ (enchufando los valores obtenidos):

\[T(z) = 1 + 4 \frac{z-i}{(z-1)(1+i)}\]


\end{problem}

\begin{problem}
Encuentre una aplicación holomorfa y biyectiva de $\Omega_1$ en $\Omega_2$ en los siguientes casos:
\ppart $\Omega_{1} = \set{z \in \cplex: \abs{z-1} < \sqrt{2}, \abs{z+1} < \sqrt{2}}$ y $\Omega_{2} = \mathbb{H}$
\ppart $\Omega_{1} = \set{z \in \cplex: \Re(z) < 0, \abs{\Im(z)} < 1}$ y $\Omega_{2} = \set{z \in \cplex: \abs{\Im(z)} < 1}$
\ppart $\Omega_{1} = \set{z \in \cplex: \abs{\Re(z)} < 1}$ y $\Omega_{2} = \set{z \in \cplex: \Re(z) > 0, \abs{z - \frac{1}{2}} > \frac{1}{2}}$
\ppart $\Omega_{1} = \set{z \in \cplex: \abs{z} < 1, \abs{z - \frac{1}{2}} > \frac{1}{2}}$ y $\Omega_{2} = \set{z \in \cplex: \abs{\Re(z)} < 1}$
\ppart $\Omega_{1} = \set{z \in \cplex: \abs{\Im(z)} < \frac{\pi}{4}}$ y $\Omega_{2} = \mathbb{D}$
\ppart $\Omega_{1} = \mathbb{D}\setminus[0,1)$ y $\Omega_{2} = \mathbb{D}$
\solution

\spart \doneby{Edu}

Tenemos como región $\Omega_1$ la intersección de dos disquitos de radio $\sqrt{2}$. Las intersecciones de los bordes son $\pm i$, luego $\Omega_1$ tiene pinta de balón de rugby. Tomando como orientación de giro la antihoraria, basta con construir la siguiente transformación:
\begin{align*}
	-i \to 0\\
	\sqrt{2} \to 1\\
	i \to \infty
\end{align*}
Luego
\begin{align*}
[z, -i, \sqrt{2}, i] = [T(z), 0, 1, \infty]\\
\frac{(z - \sqrt{2}) (-i - i)}{(-i - \sqrt{2}) (z - i)} = \frac{T(z) - 1}{0 - 1}\\
\frac{(z - \sqrt{2}) 2i}{(-i - \sqrt{2}) (z - i)} + 1 = T(z)
\end{align*}

\spart

\spart

\spart

\spart

\spart

\end{problem}

\begin{problem}
Encuentre una transformación de Möbius $T$ tal que $T(\set{z \in \cplex: \Re(z) > 0, \abs{z - 1} > \frac{1}{2}})$ es de la forma $\set{w \in \cplex: 1 < \abs{w} < h}$, calculando el valor exacto de h.
\solution

\end{problem}

\begin{problem}

Sabiendo que toda aplicación holomorfa biyectiva de $\mathbb{D}$ en $\mathbb{D}$ es de la forma
\[ f(z) = \lambda \frac{a - z}{1 - \conj{a} z} \quad \abs{a} < 1, \ \abs{\lambda} = 1 \]
describa todas las aplicaciones holomorfas biyectivas de $\mathbb{H}$ en $\mathbb{H}$.
\solution
\doneby{Edu}

Queremos con construir $T(z) = \appl{\varphi_a \circ f \circ \varphi_b}{\mathbb{H}}{\mathbb{H}}$, donde $\appl{\varphi_{b}}{\mathbb{H}}{\mathbb{D}}$ y $\appl{\varphi_{a}}{\mathbb{D}}{\mathbb{H}}$ y claramente $\varphi_{b} = \varphi_{a}^{-1}$.
\[ \varphi_{a}(z) = [z, i, 1, -1] = \frac{(z-1)(i+1)}{(z+1)(i-1)} \]
Calculo $\varphi_{a}(i) = 1$, $\varphi_{a}(1) = 0$, $\varphi_{a}(-1) = \infty$
\begin{gather*}
[z, \varphi_{a}(i), \varphi_{a}(1), \varphi_{a}(-1)] = [\varphi_{b}(z), i, 1, -1]\\
z = \frac{(\varphi_{b}(z) - 1) (i + 1)}{(\varphi_{b}(z)+1)(i-1)} \xrightarrow{\text{(sage)}}
\varphi_{b} = \frac{-(i - 1)z - i - 1}{(i - 1)z - i - 1}
\end{gather*}
\end{problem}

\begin{problem}
Sea el poderoso Thor el amo y señor de la galaxia. Calcule la masa del sol sabiendo que su martillo pesa $\infty$ para los mortales pero para él solo 2 kg.

{\it Hint}: Thor es Vikingo.
\solution

\end{problem}
