% -*- root: ../VariableCompleja.tex -*-

\section{Hoja 1}
%
\begin{problem}[1]
Realices las operaciones indicadas
\ppart
\[\frac{1}{i}+\frac{1}{1+i}\]
\ppart
\[\frac{2}{(1-3i)^2}\]
\ppart
\[(1+i\sqrt{3})^3\]
\ppart
\[(\overline{1-i})^2+\overline{2+i}\]

\solution

\spart
\[\frac{1}{i}+\frac{1}{1+i} = \frac{1+i+i}{i-1} = -\frac{2i}{1-i} = \frac{2i(1+i)}{(1-i)(1+i)} = \frac{2i-2}{1+1}=i-1\]
\spart
\[\frac{2}{(1-3i)^2} = \frac{2}{1-9-6i} = \frac{2}{-8-6i}=\frac{2(-8+6i)}{(-8-6i)(-8+6i)} = \frac{2(-8+6i)}{64+36} = \frac{-4+3i}{25}\]
\spart
\[(1+i\sqrt{3})^3\ = 1 +3i\sqrt{3}-3\cdot 3-i3^{\frac{3}{2}} = -8 +i (3\sqrt{3}-3\sqrt{3}) = -8\]
\spart
\[(\overline{1-i})^2+\overline{2+i} = \overline{1-1-2i}+2-i = 2i+2-i=2+i\]
\end{problem}

\begin{problem}[2]
Calcule los valores de
\ppart
\[\sum_{k=1}^{2015}i^k\]
\ppart
\[(1+i)^n+(1-i)^n\]
\ppart
\[\left( \cos \left( \frac{\pi}{12} \right) + i \sin \left( \frac{\pi}{12} \right)\right)^{20}\]
\ppart
\[\left(\frac{1+i}{1-i}\right)^{2014}\]
\solution

\spart
Por tratarse de una sucesión geométrica de razón $i$ sabemos que:
\[\sum_{k=1}^{2015}i^k = \frac{1-i^{2016}}{1-i} -1\]

\spart
Primero debemos observar que
\large
\[(1+i) = 2^{\frac{1}{2}}e^{\frac{\pi}{4}i}\]
\normalsize
por tanto
\[ (1+i)^n= 2^{\frac{n}{2}}e^{\frac{\pi}{4}in} = 2^{\frac{n}{2}}\left(\cos\left(\frac{\pi}{4}n\right)+i \sin\left(\frac{\pi}{4}n\right)\right)\]

Teniendo en cuenta esta relación, podemos resolver el ejercicio:
\[(1+i)^n+(1-i)^n = (1+i)^n+\overline{(1+i)^n} = 2 Re((1+i)^n)=2^{\frac{n}{2}+1}\cos\left(\frac{\pi}{4}n\right)\]

\spart
\[\left( \cos \left( \frac{\pi}{12} \right) + i \sin \left( \frac{\pi}{12} \right)\right)^{20}=\left(e^{\frac{\pi}{12}}\right)^20 = e^{\frac{\pi}{12}\cdot 20} = \cos \left( \frac{\pi\cdot 20}{12} \right) + i \sin \left( \frac{\pi\cdot 20}{12} \right) =\]
\[=\cos \left( \frac{\pi\cdot 5}{3} \right) + i \sin \left( \frac{\pi \cdot 5}{3} \right)\]

\spart
Primero vamos a trabajar con el interior del paréntesis para convertirlo en un número complejo en su expresión habitual, sin fracciones.
\[\frac{1+i}{1-i}=\frac{(1+i)^2}{(1-i)(1+i)} = \frac{1-1+2i}{1+1} = i\]
y puesto que el exponente es par, tenemos que
\[\left(\frac{1+i}{1-i}\right)^{2014}=i^{2014}=1\]
\end{problem}

\begin{problem}[3]
Sea $z=x+iy \in \cplex$. Demuestre que $|x|+|y|\leq \sqrt{2}|z|$, y que sólo hay igualdad si $|x|=|y|$.

\textbf{Ayuda:} Si $a,b \in \real$, entonces $2ab \leq a^2 + b^2$ (con igualdad sólo si $a=b$)

\solution

Si calculamos el módulo de z vemos que
\[|z|=\sqrt{x^2+y^2}\]
si $|x|=|y|$ fácilmente vemos que
\[|z|=\sqrt{x^2+x^2}=\sqrt{2x^2}=\sqrt{2}|x| \implies \sqrt{2}|z|=2|x|=|x|+|y|\]

Veamos ahora el caso en que no son iguales. En esta ocasión, nos apoyamos en al ayuda del enunciado y vemos que
\[|z|=\sqrt{x^2+y^2} \geq \sqrt{2xy} \iff \sqrt{2}|z| \geq 2\sqrt{xy}\]
%TODO terminar esto

\end{problem}

\begin{problem}[4]
Compruebe la identidad
\[|z\bar{w}+1|^2+|z-w|^2 = (1+|z|^2)(1+|w|^2)\]
donde $z,w \in \cplex$

\solution

Llamando a $z=a+bi$ y $w=c+di$ tenemos
\[|(a+bi)(c-di)+1|^2+|a+bi+c-di|^2=|ac-adi+bci+bd+1|^2+|a+bi-c+di|^2 = \]
\[=(ac+bd+1)^2+(bc-ad)^2+(a-c)^2+(b+d)^2 =\]
\[= 1 + a²c²+b²d²+2acbd + 2ac+2bd+b²c²+a²d²-2bcad+a^2+c²-2ac+b²+d²-2bd =\]
\[= 1+a²c²+b²d²+b²c²+a²d²+a²+b²+c²+d² = (1+a²+b²)(1+c²+d^2)\]

\end{problem}

\begin{problem}[5]
Demuestra las siguientes afirmaciones
\ppart
\[\text{Si } |z|=1, \text{ entonces para todos } a,b \in \cplex \text{ con } \bar{b}z+\bar{a} \neq 0\text{ se cumple } \left| \frac{az+b}{\bar{b}z+\bar{a}}\right| = 1\]

\ppart
\[\text{Si } |a| < 1, \text{ entonces } |z| <1 \text{ es equivalente a } \left| \frac{z-a}{1-\bar{a}z}\right|<1\]

\solution

\spart
%Vamos a simplificar el complejo dado
%\[\frac{az+b}{\bar{b}z+\bar{a}}= \frac{(az+b)\bar{z}}{(\bar{b}z+\bar{a})\bar{z}}=\frac{a|z|^2+b\bar{z}}{\bar{b}|z|^2+\bar{a}\bar{z}} = \frac{a+b\bar{z}}{\bar{b}+\bar{z}\bar{a}}\]

%Llegados a este punto, podemos observar que $\overline{\bar{b}+\bar{z}\bar{a}}=b+za$
%\[\frac{az+b}{\bar{b}z+\bar{a}} = \frac{(a_1+ia_2)(z_1+iz_2)+b_1+ib_2}{}\]

\spart

\[\left| \frac{z-a}{1-\bar{a}z}\right|<1 \iff |z-a| < |1-\bar{a}z| \iff \underbrace{|z-a|^2}_{(z-a)(\bar{z}-\bar{a})} < \underbrace{|1-\bar{a}z|^2}_{(1-\bar{a}z)(1-a\bar{z})} \]

Por lo que nos queda que debe cumplirse
\[|z|^2-a\bar{z}-\bar{a}z+|a|^2 < 1-\bar{a}z+a\bar{z}+|a|^2|b|^2 \iff |z|^2+|a|^2-2\cdot Re(z\bar{a}) < 1 + |a|^2|z|^2-2\cdot Re(z\bar{a}) \iff\]

\[\iff |z|^2 + |a|^2 < 1 +|a|^2|z|^2 \iff |z|^2(1-|a|^2) < 1 - |a|^2 \iff |z|^2 < 1 \iff |z| < 1\]

\end{problem}

\begin{problem}[6]
Usando la fórmula de A. de Moivre, demuestre que
\ppart
$\sin(3x)=3\sin(x)-4 \sin^3(x)$, para todod $x \in \real$

\ppart
Para todo $n \in \nat$ par, la función $\cos(n \phi)$ es un polinomio de grado $n$ de $\cos(\phi)$.

\solution

\spart
Aquí hay que tener algo de idea feliz, aunque sabiendo que estamos trabajando con complejos, tampoco es demasiado raro de pensar.

Vamos a elevar el complejo $\cos(x)+i \sin(x)$ al cubo de dos formas distintas y a igualar los resultados.

\begin{enumerate}
\item
\[\left( \cos(x)+i \sin (x) \right)^3 = (e^{ix})^3 = e^{3ix} = \left( \cos(3x)+i \sin (3x) \right)\]
\item
\[\left( \cos(x)+i \sin (x) \right)^3 = . . . = \cos^3(x)-3\cos(x)\sin^2(x) + i \left( 3\cos^2(x)\sin(x)-\sin^3(x)\right)\]
\end{enumerate}
Ahora, puesto que deben ser iguales las dos representaciónes del cubo calculado, debemos igualar las partes reales y las imaginarias.

En este caso, en cuanto forzamos la igualdad de las partes imaginarias obtenemos la igualdad buscada.
\[3\cos^2(x)\sin(x)-\sin^3(x) = \sin(3x) \iff 3\sin(x) - 3 \sin^3(x) - \sin^3(x)=\sin(3x) \iff\]
\[\iff 3\sin(x) - 4 \sin^3(x)=\sin(3x) \]

\spart
El procedimiento a seguir es prácticamente igual que en el caso anterior. Vamos a calcular $\left(\cos(\phi)+i\sin(\phi)\right)^n$ de dos formas distintas
\begin{enumerate}
\item
\[\left(\cos(\phi)+i\sin(\phi)\right)^n = \cos(n\phi)+i\sin(n\phi)\]
\item
\[\left(\cos(\phi)+i\sin(\phi)\right)^n = \sum_{k=0}^n { n \choose k} \cos(\phi)\left( i \sin (\phi)\right)^{n-k}\]
\end{enumerate}

Atendiendo al sumatorio, vemos que vamos a obtener reales siempre que $k$ sea par. En otro caso tendremos siempre un múltiplo de $i$. La suma de esos múltiplos de $i$ acabará siendo $\sin (n\phi)$.

Aplicando esto llegamos a:
\[\cos(n\phi) = \sum_{0 \leq k \leq n} { n \choose k} \cos^k(\phi)(-1)^{\frac{n-k}{2}}\sin^{n-k}(\phi)\]

pero, si nos fijamos en el seno, tenemos que

\[\sin^{n-k}(\phi) = \left(\sin^2(\phi)\right)^{\frac{n-k}{2}} = \left(1-\cos^2(\phi)\right)^{\frac{n-k}{2}}\]

y aplicando esta relación a la igualdad anterior, obtenemos
\[\cos(n\phi) = \sum_{0 \leq k \leq n} (n,k) \cos^k(\phi)(-1)^{\frac{n-k}{2}}\left(1-\cos^2(\phi)\right)^{\frac{n-k}{2}}\]

que, efectivamente, se trata de un polinomio de grado $n$ de $\cos(\phi)$
\end{problem}

\begin{problem}[7]
Demuestre que
\[\left( \frac{1+i\tan(\phi)}{1-i\tan(\phi)}\right)^n = \frac{1+i\tan(n\cdot\phi)}{1-i\tan(n\cdot\phi)}\]

\solution

Vamos a autoconvencernos de que la igualdad es cierta con $n=2$
\[\left( \frac{1+i\tan(\phi)}{1-i\tan(\phi)}\right)^2 = \frac{(1+i\tan(\phi))^2}{(1-i\tan(\phi))^2} = \frac{1-\tan(\phi)^2+2i\tan(\phi)}{1-\tan(\phi)^2-2i\tan(\phi)} = \frac{\cos(\phi)^2-\sin(\phi)^2+2i\sin(\phi)\cos(\phi)}{\cos(\phi)^2-\sin(\phi)^2-2i\sin(\phi)\cos(\phi)}=\]
\[=\frac{\cos(\phi)^2-\sin(\phi)^2+i\sin(2\phi)}{\cos(\phi)^2-\sin(\phi)^2-i\sin(2\phi)}\]
Ahora dividimos entre $\cos(\phi)^2-\sin(\phi)^2$ y, sabiendo que
\[\tan(2 \phi)=\frac{2\sin(\phi)\cos(\phi)}{\cos(\phi)^2-\sin(\phi)^2}\]
tenemos que
\[\frac{\cos(\phi)^2-\sin(\phi)^2+i\sin(2\phi)}{\cos(\phi)^2-\sin(\phi)^2-i\sin(2\phi)}=\frac{1+i\tan(2\cdot\phi)}{1-i\tan(2\cdot\phi)}\]

Ahora vamos a aplicar inducción. Suponemos que la igualdad es cierta para $n$ y vamos a ver qué ocurre con $n+1$.
\[\left( \frac{1+i\tan(\phi)}{1-i\tan(\phi)}\right)^{n+1} =  \frac{\left(1+i\tan(n\cdot\phi)\right)\left(1+i\tan(\phi)\right)}{\left(1-i\tan(n\cdot\phi)\right)\left(1-i\tan(\phi)\right)} = \frac{1-\tan(n\phi)\tan(\phi)+i\left(\tan(\phi)+\tan(n\phi)\right)}{1-\tan(n\phi)\tan(\phi)-i\left(\tan(\phi)+\tan(n\phi)\right)} = \]
multiplicando y dividiendo por $\cos(n\phi)\cos(\phi)$ llegamos a
\[=\frac{\cos(n\phi)\cos(\phi)-\sin(n\phi)\sin(\phi)+i\left(\sin(\phi)\cos(n\phi) + \cos(\phi)\sin(n\phi)\right)}{\cos(n\phi)\cos(\phi)-\sin(n\phi)\sin(\phi)-i\left(\sin(\phi)\cos(n\phi) + \cos(\phi)\sin(n\phi)\right)} =\]
\[=\frac{\cos(n\phi)\cos(\phi)-\sin(n\phi)\sin(\phi)+i\left(\sin((n+1)\phi)\right)}{\cos(n\phi)\cos(\phi)-\sin(n\phi)\sin(\phi)-i\left(\sin((n+1)\phi)\right)}\]
Al igual que hicimos en el caso particular de $n=2$, ahora multiplicamos y dividimos por $\cos(n\phi)\cos(\phi)-\sin(n\phi)\sin(\phi)$ y, sabiendo que
\[\tan((n+1)\phi)=\frac{\sin((n+1)\phi)}{\cos((n+1)\phi)}=\frac{\sin(\phi)\cos(n\phi) + \cos(\phi)\sin(n\phi)}{\cos(n\phi)\cos(\phi)-\sin(n\phi)\sin(\phi)}\]
obtenemos directamente el resultado.

\[\frac{\cos(n\phi)\cos(\phi)-\sin(n\phi)\sin(\phi)+i\left(\sin((n+1)\phi)\right)}{\cos(n\phi)\cos(\phi)-\sin(n\phi)\sin(\phi)-i\left(\sin((n+1)\phi)\right)} =  \frac{1+i\tan((n+1)\cdot\phi)}{1-i\tan((n+1)\cdot\phi)}\]
\obs La última igualdad indicada se obtiene calculando $\sen(α+β)$ y $\cos(α+β)$ con las fórmulas habituales, considerando $α=n\phi$ y $β=\phi$

\end{problem}

\begin{problem}[8]
Sin realizar cálculo alguno, razónese que no es posible que alguno de los valores de $\sqrt[1928]{1+i}$ sea $\frac{1-i}{2}$

\solution
Lo más fácil, en este caso, es ver que los módulos no coinciden. Para ello escribimos
\[1+i = 2^{\frac{1}{2}}e^{(\frac{\pi}{4}+2\pi k)i}\]
y al calcular la raíz obtenemos
\[(1+i)^{\frac{1}{1928}} = 2^{\frac{1}{2\cdot 1928}}e^{(\frac{\pi}{4}+2\pi k)\frac{i}{1928}}\]

Llegados a este punto, podemos ver que los módulos no coinciden, pues
\[2^{\frac{1}{1928}}\neq \left|\frac{1-i}{2}\right| = \sqrt{\frac{1}{2}}\]
\end{problem}

\begin{problem}[9]
Demuestre las siguientes afirmaciones
\ppart
Si $z\neq 1$ entonces
\[\sum_{i=0}^n z^i = \frac{1-z^{n+1}}{1-z}\]

\ppart
Si $w\neq 1$ es una raíz n-ésima de la unidad, entonces
\[\sum_{i=0}^{n-1} w^i = \sum_{i=1}^n w^i= 0\]
y
\[\sum_{i=0}^{n-1} i w^i = \frac{n}{w-1}\]

\ppart
si $\sin\left(\frac{\phi}{2}\right)$, entonces
\[\sum_{i=0}^n \cos(i\phi) = \frac{1}{2}\left(1+\frac{\sin((n+\frac{1}{2})\phi)}{\sin\left( \frac{\phi}{2}\right)}\right)\]

y

\[\sum_{i=1}^n \sin(\phi) = \frac{\sin(\frac{n}{2}\phi)\sin(\frac{n+1}{2}\phi)}{\sin(\frac{\phi}{2})}\]

\textbf{Ayuda:} Use el apartado a) con $z=e^{i\phi}$

\solution
\spart
Vamos a demostrarlo por inducción. En este caso, el caso base es trivial, pues sería $n=1$ con lo que tendríamos
\[1+z=\frac{1-z^2}{1-z}=\frac{(1-z)(1+z)}{1-z} = 1+z\]
Ahora suponemos que la fórmula es válida para $n$ y vamos a ver qué ocurre para $n+1$.
\[\sum_{i=0}^{n+1} z^i = \sum_{i=0}^n z^i + z^{n+1} = \frac{1-z^{n+1}}{1-z} + z^{n+1} = \frac{1-z^{n+1}+z^{n+1}-z^{n+2}}{1-z} = \frac{1+z^{n+2}}{1-z}\]
por lo que queda probado que si la ecuación se cumple para $n$ se cumple también para $n+1$ y, puesto que se cumple para 1, podemos concluir que la ecuación es válida.

\spart
Este caso resulta muy sencillo y rápido sin los apoyamos en el anterior y sabemos que $w^n=1$ y que $w^{n+1}=w$ siendo $w$ una raíz n-ésima de la unidad.

Por el apartado anterior sabemos que
\[\sum_{i=0}^{n+1} z^i = \sum_{i=0}^n z^i + z^{n+1} = \frac{1-w^{n+1}}{1-w}-1 = \frac{1-w}{1-w}-1 = 0\]

Para la segunda igualdad símplemente tenemos que darnos cuenta de que
\[\sum_{i=0}^{n-1} i w^i  = \frac{\partial}{\partial w} \sum_{i=0}^{n}w^i = \frac{\partial}{\partial w} \frac{1-w^{n+1}}{1-w} = \frac{(-n-1)(1-w)+(1-w)}{(1-w)^2} = \frac{-n-1+1}{1-w}\]
y llegamos a
\[\sum_{i=0}^{n-1} i w^i = \frac{-n}{1-w}=\frac{n}{w-1}\]
\spart
%TODO completar "Lo hice" en clase


\end{problem}

\begin{problem}[10]
Calcule todos los valores de
\ppart
\[\left(-\sqrt{2}-i\sqrt{2}\right)^{1/3}\]
\ppart
\[\sqrt{1-i\sqrt{3}}\]
\ppart
\[\sqrt[4]{1-i}\]
\ppart
\[\left(\sqrt{-i}\right)^{1/3}\]

\solution

\spart
\[\left(-\sqrt{2}-i\sqrt{2}\right)^{1/3}=\left(\sqrt{2}(-1-i)\right)^{1/3}=(\sqrt{2}\sqrt{2})^{1/3}\left(e^{\frac{-3\pi}{4}+2k\pi}\right)^{1/3}2^{1/3}e^{\frac{-\pi}{4}+\frac{2}{3}k\pi}\]

\spart
\[\sqrt{1-i\sqrt{3}} = \sqrt{4e^{(\pi/3+2k\pi)i}}=2e^{(\pi/6+k\pi)i}\]
\spart
\[\sqrt[4]{1-i} = \sqrt[4]{2e^{7\pi/8+2k\pi}} = \sqrt[4]{2}e^{(7\pi/32 + k\pi /2)i}\]

\spart

\[\left(\sqrt{-i}\right)^{1/3} = \left(1e^{(\pi+2k\pi)i}\right)^{1/6} = e^{(\pi/6+k\pi/3)i}\]
\end{problem}

\begin{problem}[11]
En este ejercicio, consideramos sólo el \textit{valor principal de la raíz cuadrada}, definido como
\[\sqrt[(p)]{z}=\sqrt{r}\left(\cos\frac{\phi}{2}+i\sin\frac{\phi}{2}\right)\]
cuando $z=r(\cos\phi+i\sin\phi)$ con $-\pi < \phi < \pi$. Claramente, $\left( \sqrt[(p)]{z} \right)^2=z$
\ppart Demuestra que las soluciones en $\cplex$ de la ecuación $az^2+bz+c=0$, con $a\neq 0$, son
\[z=\frac{-b\pm \sqrt[(p)]{b^2-4ac}}{2a}\]
\ppart
Calcule
\[\sqrt[(p)]{\left(\sqrt[(p)]{i}\right)^5} \text { y } \sqrt[(p)]{1+\sqrt[(p)]{i}}\]
\solution

\spart
Para resolver este apartado basta son sustituir la fórmula que nos dan para la $z$ en la ecuación dada y comprobar que, efectivamente, la ecuación se verifica.

\spart
La raíz principal puede sonar a algo exótico pero consiste, simplemente, en tomar la raíz del número dado y, en lugar de considerar los varios ángulos posibles, tomamos el menor posible (siempre positivo).

A efectos legales esto nos hace ahorrarnos el típico $+2k\pi$. Veamos a modo e ejemplo los radicales que nos pide calcular el enunciado
\[\sqrt[(p)]{\left( \sqrt[(p)]{i}\right)^5} = \sqrt[(p)]{e^{-\frac{3\pi}{4}i}} = e^{-\frac{3\pi}{8}}i\]
\end{problem}

\begin{problem}[12]
Resuelve las siguientes ecuaciones:
\ppart
\[(z+1)^4+i=0\]
\ppart
\[Re(z^2+5)=0\]
\ppart
\[Re(z+5)=Im(z-i)\]
\solution

\spart
Despejando como hemos hecho siempre tenemos que
\[z=\sqrt[4]{-i}-1 = e^{\pi/4+\pi k / 2} -1 \]

\spart
Considerando $z=x+iy$ tenemos que $z^2=x^2-y^2+2xyi$ con lo que llegamos a
\[x^2-y^2 = -5\]
que nos da una hipérbola
\spart
Considerando $z=x+iy$ tenemos
\[Re(z+5)=Im(z-i) \iff x+5=y-1\]
obteniendo como resultado una recta.

\end{problem}

\begin{problem}[13]
\ppart
Demuestra que si $w$ es solución de $z^n=μ$ (con $μ\in\cplex$ fijo), entonces todas las soluciones son $ww_i$ con $i=0,1,...,n-1$ donde $w_i$ son las raíces n-ésimas de la unidad
\ppart
Encuentre razonadamente las soluciones de $z^6-8=0$
\solution
\textcolor{blue}{Hecho por mi. No fiarse al 100\%}

\spart
Es sencillo e ver que los números de la forma $ww_i$ son soluciones, puesto que
\[(ww_1)^n = w^n w_i^n= μ \cdot 1 = μ\]
Cualquier otra hipotética solución deberá cumplir que al elevarla a $n$ obtengamos μ, por lo que deberá ser $w$ multiplicado por algo que, al elevarlo a $n$ nos de 1. Es decir, no habrá más posibilidades que las indicadas

\spart
Siguiente lo indicado en el apartado anterior las soluciones serán de la forma:
\[z=\sqrt[6]{8}w_i \text{ con } i=0,1,2,3,4,5 \text{ y } w \text{ raíz n-ésima de la unidad}\]

\end{problem}

\begin{problem}[14]
¿Cuándo son colineales tres puntos $z_1,z_2,z_3$ distintos dos a dos?
\solution
Para verlo hacemos como en bachillerato con los reales: escribimos la recta que pasa por dos de esos puntos y forzamos a que el tercero se contenga en dicha recta.

La recta que pasa por $z_1$ y $z_2$ sería:
\[L=\{z_1+t(z_2-z_1) t \in \real \}\]
Si $z_3 \in L \implies \exists t \tq z_3=z_1 + t(z_2-z_1)$ es decir:
\[t = \frac{z_3-z_1}{z_2-z_1}\in \real\]
para que sea real ese resultado necesitamos que el numerador y el denominador tengan el mismo argumento.

Intuitivamente representa que uniendo $z_3$ con $z_1$ obtenemos la misma recta que uniendo $z_2$ con $z_1$
\end{problem}

\begin{problem}[15]

\ppart
Compruebe que la ecuación
\[Re(az+b) = 0 \text{ con } a,b\in \cplex, \ a \neq 0\]
define una recta en el plano y que, recíprocamente, cada recta viene descrita por una ecuación de este tipo

\ppart
Encuentre los números $a,b$ para que la recta pase por dos puntos dados $z_1, z_2 \in \cplex$

\ppart
Demuestre que las rectas determinadas por las ecuaciones $Re(az+b)=0$ y $Re(cz+d)=0$ respectivamente, son perpendiculares si y sólo si $Re(a\bar{c})=0$

\ppart
Demuestre que la ecuación de una recta que pasa por dos puntos dados $z_1$ y $z_2$ puede escribirse de la forma
\[ \left| \begin{array}{ccc}
z  & \bar{z} & 1 \\
z_1 & \bar{z_1}&  1 \\
z_2 & \bar{z_2} & 1 \end{array} \right| = 0\]

\solution
\textcolor{blue}{Hecho por mi. No fiarse al 100\%}

\spart

Siendo cada número complejo $x\in \cplex = x_r+ix_i$, la ecuación que nos dan se traduce en
\[a_rz_r-a_iz_i+b_r=0 \equiv z_i = \frac{a_rz_r+b_r}{a_i}\]
siendo $x_i = y$ t $z_r = x$ obtenemos la ecuación de una recta en el plano.

\spart

Basta cons sutituir en la ecuación los valores $z_1=z_1r+iz_1i$ y $z_2=z_2r+iz_2i$ y obtenemos un sistema de 4 ecuaciones e 4 incógnitas que podremos resolver.

\spart

Basándonos en el apartado a), podemos ver que las pendientes de esas rectas son, respectivamente, $\frac{a_r}{a_i}$ y $\frac{c_r}{c_i}$.

Para que sean perpendiculares, debemos tener
\[\frac{a_r}{a_i}= - \frac{c_i}{c_r} \implies a_rc_r = -a_ic_i \implies Re\left((a_r+ia_i)(c_r-ic_i)\right) = 0\]

\spart
%TODO por hacer

\end{problem}

\begin{problem}[16]
Describa el conjunto del plano complejo determinado por las siguientes relaciones
\ppart
\[|z-2|-|z+2| > 3\]
\ppart
\[Re(z)+Im(z) < 1\]
\ppart
\[|2z|>|1+z^2|\]

\solution
\textcolor{blue}{Hecho por mi. No fiarse al 100\%}

\spart
Si tuviéramos una igualdad, estaríamos hablando de los puntos del plano cuya diferencia de distancias a los puntos $(2,0)$ y $(-2,0)$ es constante. Es decir, tendríamos una hipérbola.

Al tener una desigualdas, estamos cogiendo aquellos puntos situados a la derecha de la hipérbola.

\spart

Esta ecuación representa aquellos puntos del plano que quedan a la izquierda de la recta $y=-x+1$.

\spart


\end{problem}

\begin{problem}[17]
Determine las ecuaciones complejas:
\ppart de la parábola con foco i y directriz $Im(z)=-1$
\ppart de la elipse con focos $\pm 1$ que pasa por $i$
\ppart de la hipérbola con focos $\pm 1$ que pasa por $i+1$

\solution

Este ejercicio es bastante semejante a los apartados b) y c) del ejercicio 1.10
\spart
Recordemos que una parábola se definía a partir del foco y la directriz como el conjunto de puntos del plano que equidistaban de ellos.

Para escribir la ecuación, simplemente aplicamos la definición y vemos a que ecuación nos lleva.

Sea un punto cualquiera $z=x+iy$ su distancia al foco $i$ sería $|z-i|$ mientras que la distancia a la directriz sería $1+Re(z)$

Igualando tenemos la ecuación buscada
\[|z-i|=Re(z)+1\]

\spart
Recordemos que, por definición, la elipse es el conjunto de puntos del plano con suma de distancias a los focos constante.

Conocemos los focos lo que nos lleva a:
\[|z-1|+|z+1|=cte\]

Para determinar la constante nos basamos en que pasa por $i$, lo que nos lleva a concluir que la constante es $2\sqrt{2}$ es decir, nos queda la ecuación
\[|z-1|+|z+1|=2\sqrt{2}\]

\spart
La hipérbola tenía definición similar a la de la elipse salvo que en este caso considerábamos constante la diferencia de distancias a los pocos en lugar de la suma.

De aquí obtenemos que la ecuación buscada será de la forma
\[|z-1|-|z+1|=cte\]
Sabiendo que pasa por el punot $i+1$ podemos calcular la constante
%TODO completar

\end{problem}

\begin{problem}[18]
Esboce el conjunto de puntos $z \in \cplex$ que satisfacen
\ppart \[Re\left( \frac{z}{1+i}\right) = 0\]
\ppart \[|z^2-4z+4| = 4\]
\ppart \[|z^2-2z-1|=1\]

\solution

\spart
Vamos a jugar un poco con el número que nos dan. Siendo $z=x+iy$ tenemos
\[\frac{x+iz}{1+i}\cdot\frac{1-i}{1-i} = \frac{x-y+i(y-x)}{2} \implies Re\left( \frac{z}{1+i}\right) = \frac{x-y}{2}\]

Por tanto, obtenemos la recta $y=x$, la bisectriz del primer cuadrante.

\spart
\[z^2-4z+4| = 4 \iff |z-2|^2 = 4 \]
con lo que tenemos una circunferencia

\spart
Se deja como ejercicio para el lector, que deberá pasar a cordenadas polares con el objetivo de poder esbozar el dibujo pedido.

\textbf{consejo:} Acordarnos de la Lemniscata
\end{problem}

\begin{problem}[19]
\ppart Sea $a \in \cplex$ un número fijo. Encuentre el máximo de $|z^{12}-a|$ cuando $z$ es cualquier número complejo tal que $|z|\leq 1$
\ppart Halle razonadamente el supremo y el ínfimo del siguiente conjunto de números reales
\[\{Re(iz^4+1) \tq |z|<\sqrt{2}\}\]

\solution
\textcolor{blue}{Hecho por mi. No fiarse al 100\%}

\spart
Si abordamos el ejercicio como un problema en $\real^2$, lo que tenemos es que nos dan un punto cualquiera del plano y debemos buscar el punto del círculo unidad que más diste de él.

Para ello basta con unir el punto dado con el centro y prolongar el segmento que los une hasta que corte a la circunferencia.

Es decir, dado el punto $a=r\left(\cos(\theta)+i\sin(\theta)\right)$, el punto $b$ del círculo unidad más alejado de $a$ es $b=1\left(\cos(\theta +2π)+i\sin(\theta +2π)\right)$

(Posiblemente habría que tener cuidado si el punto $a$ no pertenece al primer cuadrante)

Para calcular ahora el número $z$ pedido, basta con tomar z=$\sqrt[12]{b}$

\spart
Siendo $z=x+iy$ tenemos
\[Re(iz^4+1)=x^4+y^4-6x^2y^2+1\]

Como aprendimos a hacer en Cálculo I, tenemos que calcular el gradiende de esa función e igualarlo a 0 y posteriormente estudiar el comportamiento de la función en la frontera del conjunto que estamos estudiando.

Vamos con el gradiente
\[\nabla \left(Re(iz^4+1)\right) = \left(4x^3-12xy^2, 4y^3-12x^2y\right)\]
al igualarlo a 0 tenemos que los puntos extremos son: el origen y los puntos que satisfacen a la vez las ecuaciones:
\[x^2-3y^2=0\]
\[y^2-3x^2=0\]
que viene a no decir nada y a dejarnos igualmente restringidos al origen

Para estudiar el comportamiento de la función en la frontera del conjunto estudiado tenemos
\[|z| = \sqrt{2} \implies x^2+y^2 =2 \implies y=\sqrt{2-x^2} \]
sustituyendo en la fórmula estudiada tenemos:
\[x^4+4+x^4-4x^2-12x^2+6x^4 +1 =0\]
derivando y simplificando tenemos
\[32x^3-32x=0 \implies x^2-1=0 \implies x=\pm 1 \implies y = \pm \sqrt{3}\]
Los puntos de máximo y mínimo son $(\pm 1, \pm \sqrt{3})$

\end{problem}

\begin{problem}[20]
Describa geométricamente el conjunto de los puntos $w \in \cplex$ que se escriben en la forma $w=iz^2+1$, para $z=x+iy$ con $x>0, y>0, \ x^2+y^2<1$.

\solution

Operando, tenemos que estamos trabajando con el conjunto de números complejos de la forma:
\[w=i(x^2-y^2)-2xy+1\]


\end{problem}

\begin{problem}[21]
Demuestre que, dados $a,c \in \cplex$, la condición necesaria y suficiente para que exista $z \in \cplex$ que verifique $|z+a|+|z-a|=2|c|$ es que sea $|a|\leq|c|$

\textbf{Ayuda:} Si λ>0, el conjunto $ \{z \in \cplex \tq |z+a|+|z-a|=2λ\}$ es una elipse si $λ > |a|$, un segmento si $λ=|a|$ y el conjunto vacío si $λ<|a|$

\solution

Basándonos en la indicación dada es obvio que $|a|\leq|c|$ es condición necesaria y suficiente para que podamos hablar de la solución de la ecuación ya que, en caso contrario, tendríamos el vacío.

Si tenemos $|a|\leq|c|$ el conjunto de puntos solución de la ecuación constituirán una recta o una elipse (según el caso) pero en ambos casos son conjuntos válidos que nos dan solución para la ecuación.
\end{problem}

\begin{problem}[22]
He aquí algunas interpretaciones geométricas de ciertas operaciones con números complejos.

\ppart Si $z=x+iy \in \cplex$ sea $α(z)$ el vector de tres dimensiones $(x,y,0)$. Verifique que para cada $z,w \in \cplex$ se cumple que $α(z)α(w)=Re(z\bar{w})$ y $α(z)\times α(w)=(0,0,Im(\bar{z}w))$

\ppart Si $0,z,w$ son los vértices de un triángulo $T$, compruebe que $Area(T)=\frac{1}{2}|Im(\bar{z}w)|$

\ppart
Si $z_1, z_2,...z_n$ son los vértices de un polígono $P$ que contiene a 0 en su interior, demuestra que $Area(P)=\frac{1}{2}\left|Im\left( \sum_{j=1}^n \bar{z}_jz_{j+1}\right)\right|$, donde se toma $z_{n+1}=z_1$

\solution
\textcolor{blue}{Hecho por mi. No fiarse al 100\%}

\spart
\[α(z)α(w)=z_xw_x+z_yw_y\]
Por otro lado
\[Re(zw)=Re\left(z_xw_x+z_yw_y+i(z_xw_y-z_yw_x)\right) = z_xw_x+z_yw_y \]

Si calculamos el producto vectorial que se nos pide, tenemos que
\[α(z)\times α(w) = (0,0,-z_xw_y+z_yw_x)\]
que podemos comprobar que coincide con la parte imaginaria de
\[Im(\bar{z}w) = Im \left( z_xw_x-z_yw_y+i(-z_xw_y+z_yw_x)\right)\]

\spart

Si ya hemos visto que el producto vectorial coincide con la parte imaginaria, es trivial ver que un medio de esa parte imaginaria nos dará el área del triángulo, pues el producto vectorial nos da el área del paralelogramo generado por los dos vectores.

\spart
Con imagen del producto que se nos da (tras multiplicar por 1/2) obtenemos el área del triángulo formado por los dos puntos dados y el origen.

Puesto que el origen se contiene en la figura cuyo área estamos calculando, al hacer esta operación con todos los vértices tenemos el área de la figura.

\end{problem}

\begin{problem}
Demuestre que la condición necesaria y suficiente para que $\{z_1, z_2, z_3\}$ sea el conjunto de los vértices de un triángulo equilátero es que
\[z_1z_2+z_2z_3+z_3z_1=z_1^2+z_2^2+z_3^2\]
\textbf{Ayuda:} Considere el triángulo $\{z_2, z_3,z_1\}$

\solution

\end{problem}


%%%%%%%%%%%%%%%%%%%%%%%%%%%%%%%%%%%%%%%%%%%%%%%%%%%%%%%%%%%%%%%%%%%%%%%%
%%%%%%%%%%%%%%%%%%%%%%%%%%%%%%%%%%%%%%%%%%%%%%%%%%%%%%%%%%%%%%%%%%%%%%%%
%%                                                                    %%
%%                            HOJA 2                                  %%
%%                                                                    %%
%%%%%%%%%%%%%%%%%%%%%%%%%%%%%%%%%%%%%%%%%%%%%%%%%%%%%%%%%%%%%%%%%%%%%%%%
%%%%%%%%%%%%%%%%%%%%%%%%%%%%%%%%%%%%%%%%%%%%%%%%%%%%%%%%%%%%%%%%%%%%%%%%
\newpage
\section{Hoja 2}
\begin{problem}[1]
(\textit{Esfera de Riemann}) Se considera $\widehat{\cplex} = \cplex \cup \{\infty\}$ y se definen los entornos de $\infty$ como aquellos que contienen un conjunto de la forma $\{z \in \cplex \tq |z|>R\}$ para algún $R > 0$

Con estos entornos $z_n \to \infty$ quiere decir que
\[\forall R > 0 \exists N \tq |z_n| > R \ \forall n > N\]

De manera similar se definen $\lim_{z \to b} f(z)= \infty$ y $\lim_{z \to \infty}f(z)=\infty$.

Sean $\mathbb{S}= \{p \in \real^3 : p_1^2+p_2^2+p_3^2\}$ y consideramos la proyección estereográfica:
\[\appl{\pi}{\mathbb{S}}{\widehat{\cplex}}, \pi(p) = \left\{
\begin{array}{lcc}
    \frac{(p_1+ip_2)}{1-p_3} & si & p \neq N = (0,0,1) \\
 \\ \infty & si & p = N
 \end{array} \right.\]

 \ppart
 Compruebe que
 \[\pi^{-1}\left( \frac{2Re(z)}{|z|^2+1}, \frac{2Im(z)}{|z|^2+1}, \frac{|z|^2-1}{|z|^2+1}\right)\]

 \ppart
 Sea $\rho(z,w)$= distancia (en $\real^3$) entre $\pi^{-1}(z)$ y $\pi^{-1}(w)$ para $z,w \in \widehat{\cplex}$. Entonces:
 \[z_n \to z \text{ en } \widehat{\cplex} \implies \rho(z_n,z) \to 0\]

 \ppart
 Demuestre que
 \[\lim_{n \to \infty}\frac{z^n}{n} = \infty \text{ si } |z| > 1\]

\solution
\textcolor{blue}{Hecho por mi. No fiarse al 1000\%}

\spart
Dado un número complejo $c=a+ib$ podemos verlo en el plano como un punto $c=(a,b)$.

Para poder calcular la inversa de la proyección estereográfica debemos trazar la recta que une este punto con el polo norte ($(0,0,1)$) y calcular la intersección de esta recta con la esfera.

La recta nos queda de la forma:
\[ \left\{
\begin{array}{l}
    x = a+ta\\
 \\ y = b+tb \\
 \\ z = -t
 \end{array} \right.\]

Su intersección con la esfera será aquel punto que cumpla la ecuación de la esfera, es decir:
\[x^2+y^2+z^2=1\]
y sustituyendo tenemos:
\[a^2+t^2a^2+a^2t + b^2+t^2b^2+tb^2+t^2 = 1 \iff\]
\[\iff t^2(a^2+b^2+1)+t(a^2+b^2)+a^2+b^2 -1 \implies \]
\[\implies t = \frac{-(a^2+b^2)\pm \sqrt{a^4+b^4 + 2a^2b^2-4(a^2+b^2+1)(a^2+b^2-1)}}{2(a^2+b^2)}=\]
\[=\frac{-a^2-b^2 \pm \sqrt{-3a^4-3b^4-6a^2b^2}}{2a^2+2b^2} = \frac{-(a^2+b^2) \pm \sqrt{-3(a^2+b^2)^2}}{2(a^2+b^2)} = \frac{-1\pm \sqrt{-3}}{2}\]

\textcolor{blue}{Y en algún punto he metido la pata por que no me salen las cuentas. Una vez tenemos la $t$ sustituimos en la ecuación de la recta y lo tenemos.}

\spart

Ya sabemos que si una sucesión de complejos converge a un complejo dado es por que tanto la parte real como la imaginaria lo hacen por separado. Así mismo, eso implica directamente que la sucesión de los módulos converge al módulo del límite.

Una vez visto esto es obvio ver que la implicación dada es correcta.

\spart

La forma más sencilla de ver esto es trabajando sobre la inversa de la proyección estereográfica. Puesto que es un homeomorfismo (ya lo estudiamos en Análisis) sabemos que $π^{-1}$ es continua por lo que la imagen límite de una sucesión es el límite de las imágenes.

Nos llevamos por tanto los puntos $z_n=\frac{z^n}{n}$ a la esfera y vemos que van creciendo en módulo. Cuando el módulo tiende a infinito, $π^{-1}$ tiene al $(0,0,1)$

\textcolor{blue}{Explicación guarrísima. Trataré de mejorarla.}

\end{problem}

\begin{problem}[2]
\ppart
Demuestre que, mediante la proyección estereográfica, las circunferencias sobre la esfera se transforman en circunferencias o rectas del plano. ¿Cuáles son las circunferencias sobre la esfera que se transforman en rectas?

\ppart
¿Qué corresponde en la esfera de Riemann a una familia de rectas paralelas del plano?

\ppart
Halle, en la esfera de Riemann, las imágenes de los conjuntos definidos por las siguientes desigualdades:
\begin{enumerate}
\item $Im(z) > 0$
\item $Re(z) < 1$
\item $|z| < 1$
\item $|z| > 2$
\end{enumerate}

\solution

\textcolor{blue}{La profesora escribió algunas cuentas pero no me han parecido muy útiles ni novedosas. Aquí doy la idea del ejercicio.}

\spart

Si la circunferencia no pasa por el polo norte, al hacer la proyección estereográfica estamos construyendo un cono e intersecando el mismo con el plano por lo que obtendremos una circunferencia.

No obstante, si hacemos esto mismo con una circunferencia que contiene al polo norte, lo que estamos construyendo es un plano e interescando dos planos, por lo que obtendremos una recta.

Se transforman en circunferencias en el plano aquellas en la esfera que no pasan por el polo norte.

\spart
Las circunferencias en la esfera que pasan por el polo norte se convierten en rectas

\spart
\begin{enumerate}
\item $Im(z) > 0$
Del cuarto de la esfera que se encuentra en la zona compleja del plano

\item $Re(z) < 1$
De un cuarto de la esfera.

\item $|z| < 1$
Son los puntos de la esfera que se encuentran en la semiesfera inferior (por debajo del plano complejo).

\item $|z| > 2$
Son los puntos de la esfera que se encuentran a una altura mayor que $\frac{1}{5}$
\end{enumerate}

\end{problem}

\begin{problem}[3]
Decida si las sucesiones $z_n= \left(\frac{1-2i}{3}\right)^n, \ w_n = \left( \frac{3-4i}{5}\right)^n$ tienen límite (finito) o no

\solution
Para que tengan límite necesitamos que su módulo converja y para ello necesitamos que este sea menor que 1.

En este caso tenemos:
\[|z_n| = \frac{\sqrt{5}}{3} \implies \lim_{n \to \infty} |z_n|^n = 0 \implies \lim_{n\to\infty} z_n = 0\]
\newpage
\[|w_n| = \frac{5}{5} \implies \lim_{n \to \infty} |w_n|^n = 1\]

Esto causa que la sucesión no tenga límite, pues tendremos puntos con el mismo módulo pero diferente ángulo por lo que no converge.

\end{problem}

\begin{problem}[4]
Decida razonadamente si las siguientes funciones tienen límite (finito) o no en el punto indicado

\ppart
\[f(x) = \frac{|z|^2}{z} (\text{ para z}\neq0 )\text{ en el punto } z=0\]

\ppart
\[f(z)= \frac{z^3-8i}{z+2i} ( \text{ para z} \neq -2i) \text{ en el punto } z=-2i\]

\solution

Al calcular este timpo de límites debemos seguir el procedimiento que hacíamos con los reales: probamos a sustituir directamente, nos dará indeterminación y jugamos con el número para evitarla.

\spart
\[\lim_{z \to 0} f(z) = \lim_{z \to 0} \frac{|z|^2}{z} = \lim_{z \to 0}\frac{z \bar{z}}{z}=\lim_{z \to 0} \bar{z} = 0\]

\spart

\[\lim_{z \to -2i} f(z) = \lim_{z \to -2i} \frac{z^3-8i}{z+2i} = \lim_{z \to -2i} z^2-21z-4 = -12\]
\end{problem}

\begin{problem}[5]
Demuestre las siguientes afirmaciones
\ppart

Si $P(z)=a_nz^n+\cdots + a_0$ y $Q(z)=b_mz^m+\cdots b_0$ son polinomios con $a_n \neq 0 \neq b_m$ entonces se tiene
\[\lim_{z \to \infty} \frac{P(z)}{Q(z)} = \left\{
\begin{array}{lcc}
    0& si & n < m \\
    \\ \frac{a_n}{b_m} & si & n=m \\
 \\ \infty & si & n > m
 \end{array} \right.\]

\ppart
No existe $\lim_{z \to \infty}e^z$
\solution
\textcolor{blue}{Hecho por mi. No fiarse al 100 \%}

\spart
Tenemos que calcular
\[\lim_{z \to \infty}\frac{a_nz^n+\cdots + a_0}{b_mz^m+\cdots b_0}\]
\begin{itemize}
\item Si $n<m$
\[\lim_{z \to \infty}\frac{a_nz^n+\cdots + a_0}{b_mz^m+\cdots b_0} = \lim_{z \to \infty}\frac{z^n(a_n+\frac{a_{n-1}}{z}\cdots + \frac{a_0}{z^n})}{z^n(b_mz^{m-n}+\cdots \frac{b_0}{z^n}}=\lim_{z \to \infty}\frac{(a_n+\frac{a_{n-1}}{z}\cdots + \frac{a_0}{z^n})}{(b_mz^{m-n}+\cdots \frac{b_0}{z^n})} = \]
\[=\lim_{z \to \infty}\frac{a_0}{b_mz^n} = 0\]

\item
Si $n=m$
\[\lim_{z \to \infty}\frac{a_nz^n+\cdots + a_0}{b_mz^m+\cdots b_0} = \lim_{z \to \infty}\frac{z^n(a_n+\frac{a_{n-1}}{z}\cdots + \frac{a_0}{z^n})}{z^n(b_mz^{m-n}+\cdots \frac{b_0}{z^n}}=\lim_{z \to \infty}\frac{(a_n+\frac{a_{n-1}}{z}\cdots + \frac{a_0}{z^n})}{(b_m+\cdots \frac{b_0}{z^n})} = \]
\[=\lim_{z \to \infty}\frac{a_0}{b_m} = \frac{a_n}{b_m}\]

\item Si $n>m$
\[\lim_{z \to \infty}\frac{a_nz^n+\cdots + a_0}{b_mz^m+\cdots b_0} = \lim_{z \to \infty}\frac{z^m(a_nz^{n-m}+\cdots + \frac{a_0}{z^m})}{z^m(b_m+\cdots \frac{b_0}{z^m})}=\lim_{z \to \infty}\frac{(a_n+\frac{a_{n-1}}{z}\cdots + \frac{a_0}{z^n})}{(b_mz^{m-n}+\cdots \frac{b_0}{z^n})} = \]
\[=\lim_{z \to \infty}\frac{a_0z^n}{b_m} = \infty\]
\end{itemize}

\spart

Debemos fijarnos en que
\[e^z=e^{x+iy}=e^xe^{iy}=e^x\left(\cos(y)+i\sin(y)\right)\]
cuando $z$ tiende a infinito, así lo hacen su parte real y su parte imaginaria.

Podemos observar que el módulo del número complejo aquí representado crece hasta infinito y su argumento oscila constantemente de modo que no tiene límite.

\end{problem}

\begin{problem}[6]
Halle los puntos de continuidad de las funciones:
\ppart
\[f(z)=\left\{
\begin{array}{lcc}
    \frac{z^4-1}{z-i}& si & z \neq i \\
 \\ 4i & si & z=i
 \end{array} \right.\]
\ppart
\[g(z)=\left\{
\begin{array}{lcc}
    z & si & |z| \leq 1 \\
 \\ |z|^2 & si & |z| > 1
 \end{array} \right.\]

 \solution
\spart
El único punto con posibles problemas y que deberíamos estudiar es el $z=i$. Vamos a estudiar cuánto vale el límite en ese punto para ver si la función es continua o no:
\[\lim_{z \to i} \frac{z^4}{z-i}=\lim_{z \to i}\frac{(z^2-1)(z-i)(z+i)}{z-i} = \lim_{z \to i} (z^2-1)(z+i) = -4i\]

Al no coincidir con el valor de la función en ese punto, podemos concluir que la función no es continua en ese punto.

\spart
El único lugar donde podemos tener problemas es en los puntos con $[z|=1$.

Para hacernos una idea de lo que podemos esperar de este límite, vamos a observar el caso concreto de $z=e^{iα}$ vemos que
\[\lim_{z \to e^{iα}} g(z) =\left\{
\begin{array}{lcc}
    e^iα & si & |z| \leq 1 \\
 \\ 1 & si & |z| > 1
 \end{array} \right. \]

 Por lo general, vemos que esta función no es continua en los puntos con módulo igual a 1 salvo en el punto $z=1$.

 En general es bastante sencillo ver que esta función no es continua, puesto que todos los puntos del círculo unidad se quedan fijos y los demás van a la recta real según su módulo.

\end{problem}


\begin{problem}[7]
¿Dónde son holomorfas las siguientes funciones?
\ppart $f(x,y)=x^2-y^2+ixy$
\ppart $f(z)=g(\bar{z})$, donde $g$ es holomorfa en $\Omega$
\ppart $f(z)=\overline{g(z)}$, donde $g$ es holomorfa en $\Omega$
\ppart $f(z)=\overline{g(\bar{z})}$, donde $g$ es holomorfa en $\Omega$
\ppart $f(z)=|g(z)|$, donde $g$ es holomorfa en $\Omega$

\textbf{Ayuda:} en los apartados b)-e) basta con usar la definición de derivada.

\solution

Para ver si son holomorfas las funciones, vamos a comprobar si se cumplen las ecuaciones de Cauchy-Riemann:
\[\partial_x f = -i \partial_y f\]
\spart
\[2x+iy=-i\left( -2y +ix\right) \iff 2x+iy = -2yi+x \iff (x,y)=(0,0)\]
Con esto no nos basta para garantizar que la función sea holomorfa en ese punto pero, pueso que tanto la parte real como la imaginaria de $f$ son diferenciables ya sí podemos garantizar que la función es holomorfa en el origen.

\textit{En algunos libros podremos ver que esta función no es coniderada holomorfa, puesto que sólo cumple la propiedad en un único punto y, tal y como se hace en variable real, una función sería diferenciable en un entorno del punto, no en un único punto. En general no nos encontraremos con este tipo de funciones en este curso.}

\spart
Tanto en este apartado como en el \textbf{d)}, para que estén bien definidas las $f(z)$ necesitamos que si un punto $z\in Ω$ entonces $z \in Ω$, es decir, el conjunto es simétrico con respecto al eje imaginario.

Vamos a considerar $g=u+iv$ y $f=U+iV$ con
\[U(x,y)=u(x,-y)\]
\[V(x,y)=v(x,-y)\]

%TODO completar estas cuentas por que me he liado mientras copiaba
Una vez visto esto, podemos derivar:
\[U_x = u_x; \;\;\; U_y=-u_y; \;\;\; V_x=v_x; \; \; \; V_y=-v_y;\]
Ahora estamos en condiciones de comprobar si se satisfacen las ecuaciones de Cauchy-Riemann:
\[U_x+iV_x=-i\left(U_y+iV_y\right) \iff u_x = -v_y\; \& \; v_x=u_y\]
pero, por ser $g$ Cauchy-Riemann sabemos que cumple las ecuaciones:
\[u_x+iv_x=-i(u_y+iv_y) \implies u_x=v_y\\; \& \; v_x=-u_y\]

y ambas condiciones sólo se darán en caso de que todas las derivadas sean iguales a 0.

\textbf{Aplicando la definición de derivada}
\[g \text{ holomorfa en }Ω \iff \frac{\partial g}{\partial \bar{w}}(w)=0\ \forall w \in Ω \iff \frac{\partial g}{\partial z}(\bar{z})=0 \ \forall z \in Ω \iff \frac{\partial f}{\partial z}(z)=0 \ \forall z \in Ω\]

Es decir, nos queda que $g$ es holomorfa en Ω si y sólo si $f$ es anti-holomorfa en Ω

\spart
\textcolor{blue}{Hecho por mi. No fiarse al 100\%}

Si consideramos $g=u+iv$ podemos ver que $f=u-iv$.

Por ser $g$ holomorfa sabemos que
\[g_x+ig_y = 0 \implies u_x=v_y \ \& \ v_x=-u_y \implies u_x-iv_x+iu_y+v_y = 0 \implies f_x+if_y=0 \implies\]
\[f \text{ es holomorfa }\]


\spart
\textcolor{blue}{Hecho por mi. No fiarse al 100\%}

Si $f$ fuese holomorfa, tendríamos que $g(\bar{z})$ también lo es, basándonos en el apartado anterior.

Ahora bien, observando el apartado \textbf{b} vemos que esto implicaría que la función $g$ es anti-holomorfa y sabemos que esto no es cierto.

Por tanto esta funciónno es holomorfa.

\spart
\textcolor{blue}{Hecho por mi. No fiarse al 100\%}

Si consideraos $g=u+iv$, tendríamos que $f=\sqrt{u^2+v^2}$, que se trata de una función real.

Por tanto, $f$ será holomorfa siempre y cuando tenga derivada, cosa que ocurre siempre que $g!=0$, es decir, será holomorfa si $g^{-1}(0) \notin Ω$

\end{problem}


\begin{problem}[8]
¿Dónde son holomorfas las siguientes funciones? ¿Cuál es su derivada?
\ppart[k]
$\log(e^z+1)$

\ppart[l]
$\sqrt{e^z+1}$

\ppart[m]
$\sqrt{z^3-1}$
\solution

\spart[k]
Vamos a ver cómo se comporta esta función.

Primero recordamos que la función $e^z$ nos envía rectas verticales en circunferencias y rectas horizontales en rectas que pasan por el origen.

Al tener $e^z+1$ estamos desplazando hacia la derecha (estamos sumando a la parte real) una unidad las imágenes de esas rectas.

Veamos dos formas distintas de calcular el logaritmo que se pide:
\begin{enumerate}
\item \textbf{Por composición}

Sea $\appl{f}{Ω}{\cplex \setminus \{0\}}$, una rama del logaritmo de $f$ es una función continua $\appl{F}{Ω}{\cplex}$ tal que
\[e^{F(z)}=f(z) \ \forall z \in Ω\]

En este caso tenemos $f(z)=e^z+1$.

Para definirla correctamente debemos quitar del dominio aquellos puntos cuya imagen por $f(z)$ es 0 ya que para esos puntos sería imposible encontrar una $F(z)$ con $e^{F(z)}=0$. Estos puntos son aquellos de la forma $(2n+1)πi$ con $n \in \nat$

También podemos ver que si en el dominio hay una curva α tal que su imagen por $f$ de una vuelta al origen (es decir, la curva debe girar en su dominio en torno a una preimagen de 0) tendremos problemas, pues el argumento irá creciendo hacia $2π$ y acabará valiendo $0$.

Para evitar este caso excluimos del dominio los puntos $(x,y)$ con $|x| > πi$, con lo que ya tenemos el dominio buscado donde la función es holomorfa.

\item \textbf{Por definición}

Para definir ahora una rama del logaritmo de $f$ consideramos
\[F(z)=\log(f(z)) = \log|f(z)|+i\underbrace{\text{arg}}_{\text{una rama del argumento de f}}(f(z))\]

Una vez tengamos esta función $F(z)$ bien definida, tendremos que:
\[e^{F(z)}=f(z)\]

Vamos a definir ahora esa rama del argumento. Para ello recordamos que una rama del argumento de $\appl{f}{Ω}{\cplex}$ es una función $\appl{g}{Ω}{\real}$ continua tal que
\[e^{ig(z)}=\frac{f(z)}{|f(z)|} \ \forall z \in Ω\]

En nuestro caso tendríamos
\[\frac{e^z+1}{|e^z+1|}=e^{ig(z)}\]
pero no sabemos cuál sería el dominio Ω, todavía.

\end{enumerate}

Una vez hecho esto, procedemos a calcular su derivada:

%Se trata de un dominio simplemente conexo lo que implica que existe la rama del logaritmo de $f$.

Ya vimos que si $F(z)$ es rama del logaritmo de $f$ se cumple que $e^{F(z)}=f(z)$ y derivando a ambos lados tenemos
\[e^{F(z)}F'(z) = f'(z) \implies F'(z)=\frac{f(z)}{e^{F(z)}}=\frac{f'(z)}{f(z)}\]
que coincide con la derivada esperada del logaritmo.

\obs Si aplicásemos el segundo procedimiento para encontrar una rama del logaritmo de la función $e^z$, que sabemos es la identidad, obtendríamos que la identidad sólo podría definirse en una parte de los complejos.

La explicación de esto es que, con el este último procedimiento estamos forzando a que exista esa rama como una composición. Al pedir más condiciones, reducimos el conjunto de funciones válidas.

\spart[l]
Una vez que tenemos la rama $F(z)$ del logaritmo de $e^z+1$ (definida en Ω), tenemos también la función
\[G(z)=e^{\frac{1}{2}F(z)} \text{ con } \appl{G}{Ω}{\cplex}\]
que es una rama de la raíz cuadrada de $f(z)=e^z+1$

\spart[m]
Vamos a buscar primero $F(z)$, una rama del logaritmo de $f(z)=z^3-1$.

Para ello primero eliminamos del dominio aquellos puntos que nos llevan al 0. Es decir, quitamos del dominio las tres raíces cúbicas del la unidad.

Nuevamente, el problema nos surje al tener curvas α que den vueltas en torno a alguna de esas raíces. Para evitar que se produzca esto quitamos del dominio tres segmentos infinitos cualesquiera, cada uno de ellos con inicio en una de las raíces cúbicas del aunidad.


\end{problem}

\begin{problem}[9]
Sea $\appl{T}{\real^2}{\real^2}$ dada por
\[T(x,y) = \left(u(x,y), v(x,y)\right)\]
Definimos la derivada de $T$ en la dirección $\overrightarrow{w}=(a,b)$ como:

\[D_{\overrightarrow{w}} T = \lim_{t \to 0} \frac{T(x+ta, y+tb) - T(x,y)}{t}\]
Observe que
\[D_{\overrightarrow{w}} T = \left(D_{\overrightarrow{w}} u, D_{\overrightarrow{w}} v\right)\]

Dada la función compleja $f(z)=u(x,y)+iv(xy), \ z=x+iy$, demuestre que si $f$ es holomorfa, entonces
\[D_{\overrightarrow{w}} f = f'(z)w \text{ donde } w = a+ib\]

\solution
\textcolor{blue}{Hecho por mi. No fiarse al 100\%}

Por ser $f$ holomorfa sabemos que
\[\lim_{z\to z_0} \frac{f(z)-f(z_0)}{z-z_0}= f'(z_0)\]

Si consideramos $\appl{f}{\real^2}{\real^2}$, dado $z=x+yi$, tendríamos $f(x,y)=\left(Re(x,y), Im(x,y)\right)$ y el límite de la definición de función holomorfa nos da lugar a dos límites:
\[\lim_{z \to z_0} \frac{Re(x,y)-Re(x_0,y_0)}{x-x_0}=Re'(x,y)\]
\[\lim_{z \to z_0} \frac{Im(x,y)-Im(x_0,y_0)}{y-y_0}=Im'(x,y)\]

Calculamos ahora
\[D_{\overrightarrow{w}} f = \lim_{t \to 0} \frac{f(x+ta, y+tb) - f(x,y)}{t} =\]
\[= \lim_{t \to 0} \frac{Re(x+ta, y+tb)+iIm(x+ta,y+tb)-Re(x,y)-iIm(x,y)}{t} \]

de donde podemos separar dos límites:
\[\lim_{t \to 0} \frac{Re(x+at,y+bt) - Re(x,y)}{t} = Re'(x,y)\]
\[\lim_{t \to 0} \frac{Im(x+at,y+bt) - Im(x,y)}{t} = Im'(x,y)\]

La última igualdad de cada linea viene del hecho de que la función $f$ sea holomorfa.

\end{problem}

\begin{problem}[10]
Sea $f$ una función holomorfa en un dominio $Ω \subset \cplex$. Demuestre que si $|f|$ es constante en Ω, entonces $f$ es constante.

\solution
\textcolor{blue}{Hecho por mi. No fiarse al 100\%}

La única posibilidad para que $|f|$ sea constante y no lo sea $f$ es que $f$ alterne de valor entre $|f|$ y $-|f|$.

Salvo que la función sea nula, es obvio ver que en caso de no ser constante, con las condiciones dadas, tampoco será continua.

Sin embargo, ya hemos probado en teoría que si una función es holomorfa es continua, por lo que no puede darse este caso.

\end{problem}

\begin{problem}[11]
Demuestre las siguientes afirmaciones:
\ppart Si $h$ es una fucnión de $\real^2$ en $\real$ de clase $\algb{C}^2$ y $f$ es holomorfa, entonces $\nabla(h\circ f) = (\nabla h \circ f)|f'|^2$

\ppart Si $f$ es holomorfa en un dominio $Ω \subset \cplex$ y $f(z) \neq 0 \forall z \in Ω$, entonces
\[\nabla(|f|)=\frac{|f'|^2}{|f|}\]

\ppart Si $f,g$ son holomorfas en un dominio Ω, y si $|f|+|g|$ es constante en Ω y $f$ y $g$ no se anulan en Ω, entonces $f$ y $g$ son constantes.

\solution

\spart
%\[\nabla (h \circ f)=\left(\partial_{x_1}(h \circ f),\partial_{x_2}(h \circ f)\right)=\left(f_x'\partial_x (h \circ f),f_y'\partial_y (h \circ f)\right)\]

\end{problem}

\begin{problem}[12]
Halle el radio de convergencia de las siguientes series de potencias
\ppart[d] $\sum_{n=0}^{\infty} \cos(in)z^n$
\ppart[f] $\sum_{n=0}^{\infty} (n+a^n)z^n$

\solution

\spart[d]
\[\sum_{n=0}^{\infty} \cos(in)z^n = \sum_{n=0}^{\infty}\frac{1}{2}\left(e^{--n}+e^n \right)z^n\]
Ahora podemos calcular fácilmente el radio de convergencia:
\[R = \lim_{n \to \infty}\left| \frac{\cos(in)}{\cos\left(i(n+1)\right)}\right| = \lim_{n \to \infty}\left| \frac{e^{-n}+e^n}{e^{-(n+1)}+e^{n+1}}\right|=\frac{1}{e}\]

\spart[f]
No podemos simpliciar más la fórmula de la serie de modo que procedemos directamente a calcular el radio
\[\lim_{n\to \infty}\left| \frac{n+a^n}{n+1+a^{n+1}}\right|\]
a partir de aquí distinguimos dos casos:
\begin{enumerate}
\item \textbf{ Caso |a| > 1}
\[\lim_{n\to \infty}\left| \frac{n+a^n}{n+1+a^{n+1}}\right| = \lim_{n \to \infty} \left|\frac{\frac{n}{a^n}+1}{\frac{n+1}{a^n}+a} \right| \]

Llegados a este punto, debemos calcular el límite del numerador y del denominador. Vamos a aplicar para ello el lema del Sandwich:
\begin{itemize}
\item \textbf{Numerador}
\[\lim_{n \to \infty} 1-\frac{n}{|a|^n}\leq \lim_{n \to \infty} \left|\frac{n}{a^n}+1 \right| \leq \lim_{n \to \infty} \frac{n}{|a|^n}+1 \implies \lim_{n \to \infty} \left|\frac{n}{a^n}+1 \right|  = 1\]

\item \textbf{Denominador}
\[\lim_{n \to \infty} |a|-\frac{n+1}{|a|^n}\leq \lim_{n \to \infty} \left|\frac{n+1}{a^n}+|a| \right| \leq \lim_{n \to \infty} \frac{n+1}{|a|^n}+|a| \implies \lim_{n \to \infty} \left|\frac{n+1}{a^n}+|a| \right|  = |a|\]
\end{itemize}

Con lo que podemos concluir que
\[\lim_{n\to \infty}\left| \frac{n+a^n}{n+1+a^{n+1}}\right| = \frac{1}{|a|}\]

\item \textbf{Caso |a| $\leq$ 1}
\[\lim_{n\to \infty}\left| \frac{n+a^n}{n+1+a^{n+1}}\right| = \lim_{n \to \infty} \left|\frac{\frac{a^n}{n}+1}{\frac{1}{n}+\frac{a^{n+1}}{n}+1} \right| \]

Como antes, calculamos los límites del numerador y del denominador
\begin{itemize}
\item \textbf{Numerador}
\[\lim_{n\to \infty} 1-\frac{|a|^n}{n} \leq \lim_{n\to \infty}\left| 1+\frac{a^n}{n} \right|  \leq \lim_{n\to \infty}1+\frac{|a|^n}{n} \implies \lim_{n\to \infty} \left| 1+\frac{a^n}{n} \right| = 1\]

\item \textbf{Denominador}
\[\lim_{n\to \infty}1-\frac{1}{n}\frac{|a|^{n+1}}{n} \leq \lim_{n\to \infty}\left| 1 + \frac{1}{n} + \frac{|a|^{n+1}}{n} \right| \leq \lim_{n\to \infty}1 + \frac{1}{n}+\frac{|a|^{n+1}}{n} \implies\]
\[\implies \lim_{n\to \infty}\left| 1 + \frac{1}{n} + \frac{|a|^{n+1}}{n} \right|  = 1\]
\end{itemize}
Por tanto, podemos concluir que, en este caso:
\[\lim_{n\to \infty}\left| \frac{n+a^n}{n+1+a^{n+1}}\right| 1 \]
\end{enumerate}

\end{problem}

\begin{problem}[13]
Supongamos que los radios de convergencia de las series $\sum_{n=0}^{\infty} a_nz^n$ y $\sum_{n=0}^{\infty}b_nz^n$ son iguales a $r_1$ y $r_2$ respectivamente. ¿Qué se puede decir respecto a los radios de convergencia de las series:
\ppart
\[\sum_{n=0}^{\infty} (a_n\pm b_n)z^n\]

\ppart
\[\sum_{n=0}^{\infty}a_nb_nz^n\]

\ppart
\[\sum_{n=0}^{\infty}\frac{a_n}{b_n}z^n\]

\solution

\spart
\[ R = \min\{R_1, R_2\}\]
esto se debe a que podemos escribir la suma como
\[\sum_{n=0}^{\infty}(a_n\pm b_n)z^n = \lim_{n \to \infty} \sum_{i=0}^n (a_i\pm b_i)z^i\]
y ese límite podrá descomponerse en suma de límites cuando ambos límites existan. La forma de garantizar que esto ocurra es tomar el mínimo de los radios de convergencia.

\spart
\[\frac{1}{R}=\limsup_{ n \to \infty}|a_nb_n|^{1/n} = \limsup_{n\to \infty}|a_n|^{1/n}|b_n|^{1/n} \leq\]
\[\leq \limsup_{n\to \infty} |a_n|^{1/n}\cdot \limsup_{n\to \infty} |b_n|^{1/n} = \frac{1}{R_1}\frac{1}{R_2}\]

Por tanto tenemos que $R \geq R_1R_2$ y tendremos la igualdad cuando exista al menos uno de los límites superiores: $\limsup_{n\to \infty} |a_n|^{1/n}$ ó $\limsup_{n\to \infty}|b_n|^{1/n}$.

Para afirmar esto nos hemos basado en el siguiente lema:
\begin{lemma}
Sean $x_n,y_n \geq 0$,
\[\exists \lim_{n \to \infty} x_n \implies \limsup_{n \to \infty }(x_ny_n)=\lim_{n \to \infty}x_n \cdot \limsup_{n \to \infty} y_n\]
\end{lemma}

\spart
\[\sum_{n = 0}^{\infty} \frac{a_n}{b_n}z^n \text{ con } b_n \neq 0\]
Podemos calcular su radio de convergencia de la siguiente forma:
\[\frac{1}{R}=\limsup_{n \to \infty} |a_n|^{1/n} = \limsup_{n \to \infty} \frac{|a_n|^{1/n}}{|b_n|^{1/n}}|b_n|^{1/n} \leq \limsup_{n \to \infty} \frac{|a_n|^{1/n}}{|b_n|^{1/n}}\limsup_{n \to \infty} |b_n|^{1/n}=\frac{1}{R}\cdot \frac{1}{R_2}\]

Así hemos llegado a que $R \leq \frac{R_1}{R_2}$, teniendo la igualdad en caso de que exista el límite de $|b_n|^{1/n}$


\end{problem}

\begin{problem}[14]
Pruebe que para todo $z \in \cplex$ tal que $|z|<1$, se verifican las identidades:
\ppart
\[\frac{1}{1-z}=\sum_{n=0}^{\infty}z^n\]
\ppart
\[\left(\frac{1}{1-z}\right)^2 = \sum_{n=0}^{\infty} nz^{n-1}\]
\solution

\textcolor{blue}{Hecho por mi. No fiarse al 100\%}

\spart

\[\sum_{n=0}^{\infty}z^n = \lim_{N \to \infty}\sum_{n=0}^{N}z^n = \lim_{N\to \infty}\frac{1-z^n}{1-z} = \lim_{N \to \infty}\frac{1-|z|^n\left(\cos(α) + i \sin(α)\right)}{1-z} \underbrace{=}_{\text{ por ser } |z|<1} \frac{1}{1-z}\]

\spart

\[\sum_{n=0}^{\infty} nz^{n-1} = \lim_{N \to \infty}\sum_{n=0}^{N} nz^{n-1} = \lim_{N \to \infty} \partial z^n =\footnote{\href{http://www.memedeportes.com/baloncesto/gif-increible-triple-desde-33-metros-que-le-sirve-a-thunder-law-para-romper-un-record-guiness}{triiiiiple}} \partial \lim_{N \to \infty} z^n = \partial \frac{1}{1-z} = \left(\frac{1}{1-z}\right)^2\]

\end{problem}

\begin{problem}[15]
Desarrole las siguientes funciones en series de potencias del tipo indicado
\ppart
\[\frac{z}{z^2-5z+6} \text{ y } \frac{z}{(z-1)^2} \text{ en potencias de } z\]
\ppart
\[\frac{2z+3}{z+1} \text{ y } \frac{2z+3}{(z+1)^2} \text{ en potencias de } z-1\]
\solution
\textcolor{blue}{Hecho por mi. No fiarse al 100\%}

\spart
\[\frac{z}{z^2-5z+6} = \frac{z}{(z-2)(z-3)} = \frac{A}{z-2}+\frac{B}{z-3}\]
Para calcular estos coeficientes $A$ y $B$, sumamos las dos fracciones de la derecha y damos los valores $z=2$ y $z=3$.

Así obtenemos $A=-2$ y $B=3$ y podemos escribir:
\[\frac{z}{z^2-5z+6}\]

\[\frac{z}{(z-1)^2}= \frac{-2}{z-2}+\frac{3}{z-3} = -\frac{1}{2}\frac{-2}{1-z/2}-\frac{1}{3}\frac{3}{1-z/3} = \sum_{n=0}^{\infty} \left(\frac{z}{2}\right)^n -\frac{1}{2}3\sum_{n=0}^{\infty}\left(\frac{z}{3}\right)^n \]

Para la siguiente función aplicamos el mismo procedimiento aunque con pequeñas variaciones
\[\frac{z}{(z-1)^2}=\frac{A}{(z-1)}+\frac{Bz+C}{(z-1)^2}\]
y procedemos a calcular el valor de las constantes $A$, $B$ y $C$ dando valores. Nos queda el sistema de ecuaciones:
\[\begin{cases}
	1=B+C\\
	A=C \\
	-1=-2A-B+C \\
	\end{cases} \implies \begin{cases}
	1=B+A\\
	A=C \\
	-1=-A-B \\
	\end{cases} \implies \begin{cases}
	1=B+A\\
	A=C \\
	\end{cases}\]
Por comodidad vamos a tomar $A=C=1$, $B=2$ con lo que obtenemos:
\[\frac{z}{(z-1)^2}=\frac{1}{(z-1)}+\frac{2z}{(z-1)^2}+\frac{1}{(z-1)^2} = .... =\]
\[= -\sum_{n=0}^{\infty} z^n+2z\sum_{n=0}^{\infty}z^{2n}+\sum_{n=0}^{\infty}z^{2n}\]

\spart
\[\frac{2z+3}{z+1} = 2 + \frac{1}{z+1}= 2 +\sum_{n=0}^{\infty}(-1)^nz^n\]

\[\frac{2z+3}{(z+1)^2} = \frac{A}{z+1}+\frac{Bz+C}{(z+1)^2}\]
Calculamos $A$, $B$ y $C$ como en el apartado anterior y obtenemos:
\[\begin{cases}
	1=-B+C\\
	3=A+C \\
	5=2A+B+C \\
	\end{cases} \implies \begin{cases}
	1=-B+C\\
	3=A+C \\
	\end{cases} \]
	y vemos que una posible solución sería $A=2$, $B=0$, $C=1$
con lo que obtenemos:
\[\frac{2z+3}{(z+1)^2} = \frac{2}{z+1}+\frac{1}{(z+1)^2} = ... =\]
\[=2\sum_{n=0}^{\infty}(-1)^nz^n+\sum_{n=0}^{\infty}z^2\]
\end{problem}

\begin{problem}[16]
Calcule el radio de convergencia y la suma de
\ppart
\[\sum_{n=0}^{\infty}\frac{z^{2n}}{n!}\]
\ppart
\[\sum_{n=0}^{\infty}n(n-1)z^n\]
\ppart
\[\sum_{n=0}^{\infty}(-1)^n \frac{(z-2πi)^n}{n!}\]
\solution

\spart
\[\sum_{n=0}^{\infty}\frac{z^{2n}}{n!} = \sum_{n=0}^{\infty}\frac{(z^2)^n}{n!} = e^{z^2}\]

\spart

\[\sum_{n=0}^{\infty}n(n-1)z^n = \sum_{n=2}^{\infty}n(n-1)z^n = \frac{2z^2}{(1-z)^3}\]

Para la resolución de este apartado, nos hemos basado en el estudio del sumatorio:
\[\sum_{m=0}^{\infty} (m+1)mz^{m-1} = \frac{2}{(1-z)^3}\]
y multiplicando por $z^2$ a ambos lados de la igualdad, obtenemos la parte derecha necesaria para poder sustituir en nuestro problema inicial.

A su vez, para obtener este sumatorio, nos hemos basado en el producto de las series:
\[\sum_{n=0}^{\infty}z^n = \frac{1}{1-z}\]
\[\sum_{n=0}^{\infty}zz^{n-1} = \frac{1}{(1-z)^2}\]

\spart

\[\sum_{n=0}^{\infty}(-1)^n \frac{(z-2πi)^n}{n!} = \sum_{n=0}^{\infty} \frac{(-z+2πi)^n}{n!} = e^{-z+2πi} = e^{-z}\]
\end{problem}

\begin{problem}[17]
Si $f(z)=\sum_{n=0}^{\infty}a_nz^n$ ¿qué representa $\sum_{n=1}^{\infty}n^2a_nz^n$ en términos de $f$?
\solution

Definimos un nuevo operador de la forma:
\[\left[ z\frac{\partial}{\partial z}\right]\left( f(z) \right) = z f'(z)\]
aplicando este nuevo operador a la función dada obtenemos
\[\left[ z\frac{\partial}{\partial z}\right]\left( f(z) \right) = \sum_{n=0}^{\infty}na_nz^n  \]
y aplicado a la serie que queremos estudiar, podemos ver que
\[\left[ z\frac{\partial}{\partial z}\right]^2\left( f(z) \right)=  \sum_{n=1}^{\infty}n^2a_nz^n\]

Por tanto ya tenemos el resultado pedido. Para acabar, sólo necesitamos calcular el resultado de aplicar el nuevo operador dos veces a la función $f(z)$. Haciéndolo, podemos ver que:

\[zf'(z)+z^2f''(z)= \sum_{n=1}^{\infty}n^2a_nz^n\]
\end{problem}

\begin{problem}[18]
¿Para qué valores de $z$ convergen las siguientes series?
\ppart
\[\sum_{n=0}^{\infty} \left( \frac{z}{1+z}\right)^n\]

\ppart
\[\sum_{n=0}^{\infty} ne^{-nz}\]

\ppart
\[\sum_{n=0}^{\infty} \frac{\sin(nz)}{n^2}\]

\ppart
\[\sum_{n=0}^{\infty} \frac{\sin(nz)}{2^n}\]

\ppart
\[\sum_{n=0}^{\infty} \frac{z^n}{1+z^2n}\]

\solution

\spart
\[\sum_{n=0}^{\infty} \left( \frac{z}{1+z}\right)^n\]
Tenemos una progresión geométrica, que sabemos que converge siempre que el módulo de la razón sea menor que 1.

Por tanto sabemos que la serie converge sii
\[\left| \frac{z}{1+z}\right| < 1 \iff |z| < |1+z| \iff x^2+y^2 < 1+x^2+2x+y^2 \iff 2x+1 > 0 \iff Re(z)>-\frac{1}{2}\]

\spart
\[\sum_{n=0}^{\infty} ne^{-nz} \text{ converge } \iff \sum_{n=0}^{\infty} e^{-nz}\text{ converge }\]

y vemos que se trata nuevamente de una progresión geométrica de razón $e^{-z}$ y, al igual que en el apartado anterior, para garantizar la convergencia necesitamos que el módulo de ese valor sea menor que 1. Es decir:
\[|e^{-z}| < 1 \iff 1 < |e^z|=e^{Re(z)} \iff Re(z) > 0\]

\spart
\[\sum_{n=0}^{\infty} \frac{\sin(nz)}{2^n}\]
% Nos apoyamos en que modulo de seno cuadrad es seno hiperbolico cuadrado de parte imaginar mas seno cuadrado de parte real. el seno de la parte real oscila y no dará problemas por lo que la serie acabará siendo igual (convergentemente hablando) al seno hiperbolico al cuadrado.
\spart

\spart

\end{problem}

\begin{problem}[19]
Se considera la serie $\sum_{n=1}^{\infty}\frac{1}{n^z}$ (conocida como \textbf{Zeta de Riemann}) para $z \in \cplex$
\ppart Demuestre que la serie converge si $Re(z) > 1$
\ppart Demuestre que si $a$ es un número real mayor que 1, entonces la serie converge uniformemente en $\{z \in \cplex \tq Re(z) \geq a\}$

\solution
\textcolor{blue}{Hecho por mi. No fiarse al 100\%}

\spart
Siendo $z=x+iy$ tenemos que:
\[\sum_{n=1}^{\infty} \frac{1}{n^z} = \sum_{n=1}^{\infty} \frac{1}{n^{x+iy}} = \sum_{n=1}^{\infty} \frac{1}{n^x} \frac{1}{n^{iy}} = \sum_{n=1}^{\infty} \frac{1}{n^x} \frac{1}{e^{iy\log(n)}} =\footnote{Usando Euler escribimos el segundo factor como combinación de senos y cosenos, que nos dará un número complejo acotado que no afecta a la convergencia de la función} \sum_{n=1}^{\infty}  \frac{1}{n^x}\]

Tenemos entonces que la suma inicial converge si lo hace $\sum_{n=1}^{\infty}  \frac{1}{n^x}$

Usando el teorema de la integral de Cauchy, tenemos que la suma converge si lo hace la integral. Vamos a ver esa integral.

\[\int_1^{\infty}\frac{1}{t^x}dt = \frac{1}{1-x}t^{1-x}|_1^{\infty} = \frac{1}{1-x} \left((\infty)^{1-x}-0\right)\]
Podemos ver fácilmente que para que converga necesitamos que $x>1$

\spart

Este apartado no supone ningún problema. Si en lugar de estudiar $\sum_{n=1}^{\infty} \frac{1}{n^z}$ hubíesemos estudiado $\sum_{n=1}^{\infty}\left| \frac{1}{n^z}\right|$ podríamos haber seguido los mismos pasos, empleando siempre el valor absoluto y escribiendo $\leq$ en lugar de $=$ con lo que habríamos llegado a idéntico resultado, lo que provoca la convergencia absoluta.
\end{problem}

\begin{problem}[20]
Supongamos que la serie de potencias $f(z)=\sum_{k=0}^{\infty}a_kz^k$ tiene un radio de convergencia $R=1$ y que $\sum_{k=0}^{\infty}a_k=0$. Denotemos
\[s_n(z)=\sum_{k=0}^n a_k z^k \text{  y  } s_n=\sum_{k=0}^na_k\]

\ppart Demuestre que $s_n(z)=(1-z)\sum_{k=0}^{n-1}s_kz^k+s_nz^n$ y concluya que $f(z)=(1-z)\sum_{n=0}^{\infty}s_nz^n$

\ppart Demuestre que $f(z) \to 0$ cuando $z$ se aproxima a 1 de tal forma que $\frac{|1-z|}{1-|z|}$ está acotado

\solution
\textcolor{blue}{Hecho por mi. No fiarse al 100\%}

\spart
Tenemos que
\[s_n(z) = (1-z)\sum_{k=0}^{n-1}s_kz^k+s_nz^n = \sum_{k=0}^{n-1}s_kz^k -\sum_{k=0}^{n-1}s_kz^{k+1}+s_nz^n = \]
\[=\sum_{k=0}^{n-1}\sum_{i=0}^ka_iz^k -\sum_{k=0}^{n-1}\sum_{i=0}^ka_iz^{k+1}+\sum_{i=0}^na_iz^n = \sum_{k=0}^{n-1} \left( \sum_{i=0}^k a_i - \sum_{i=0}^{k-1} a_i\right) z^k - \sum_{k=0}^{n-1} a_kz^n + \sum_{i=0}^na_iz^n =\]
\[= \sum_{k=1}^{n-1} a_k z^k + a_nz^n = \sum_{k=0}^n a_k z^k\]

Las cuentas son un poco raras si las miras de golpe pero en el fondo es sólo juegar con los índices de los sumatorios.

\spart

\textcolor{blue}{No se escribirlo pero la idea está ahí. Los $s_n$ no nos preocupan por que sabemos que tienden a 0 y el sumatorio que nos queda ignorándolos es una progresión geométrica cuyo valor multiplicado por (1-z) es la fracción que nos dan diciendo que está acotada. El problema es que no se cómo escribirlo bien}

\end{problem}

\begin{problem}[21]
Demuestre las siguientes afirmaciones

\ppart Si $z \in \mathbb{D}$ y $k \in \nat$, entonces $1-|z|^k \leq k(1-|z|)$

\ppart Si $\{a_n\}$ es una sucesión de números complejos tal que $\lim_{n \to \infty} na_n=0$, entonces
\[\lim_{n\to \infty}n \cdot \sup_{k \geq n}|a_k| = 0 \text{  y  } \lim_{n \to \infty}\frac{1}{n}\cdot \sum_{k=0}^nk|a_k|=0\]

\solution
\textcolor{blue}{Hecho por mi. No fiarse al 100\%}

\spart
Vamos a demostrarlo por inducción.

El caso base es claro, tomamos $k=1$ y ya lo tenemos

Tomemos como hipótesis que el resultado es cierto el resutlado para $k=n$. Queremos ver si es cierto que:
\[1-|z|^k|z| < k(1-|z|)+(1-|z|)\]
Basándonos en la hipótesis tenemos que
\[ k(1-|z|)+(1-|z|) > 1-|z|^k+(1-|z|)\]
Si se cumpliera:
\[1-|z|^k+(1-|z|) > 1-|z|^k|z|\]
ya lo tendríamos todo hecho. Vemos que es cierto que se cumple la desigualdad pues simplificando los 1s que se suman y pasando $|z|^k$ al otro lado llegamos a
\[1-|z| > (1-|z|)|z|^k\]
lo cual es absolutamente cierto pues $|z|^k < 1$

\spart

Para la primera ecuación tenemos que:

\[\lim_{n\to \infty}n \cdot \sup_{k \geq n}|a_k| = \sup_{k \geq n} \lim_{n\to \infty}n |a_k| = \sup_{k \geq n} 0 = 0\]

Vamos con la segunda:
\[\lim_{n \to \infty}\frac{1}{n}\cdot \sum_{k=0}^nk|a_k| \leq \lim_{n \to \infty}\frac{1}{n}\cdot \sup_{k \geq n}|a_k| \sum_{k=0}^nk = \lim_{n \to \infty}\cdot \sup_{k \geq n}|a_k| \frac{1}{n} \frac{n(n+1)}{2}=\]
\[=\lim_{n \to \infty}\cdot \sup_{k \geq n}|a_k|\frac{n+1}{2} = \lim_{n\to \infty}n \cdot \sup_{k \geq n}|a_k| = 0\]
\end{problem}

\begin{problem}[22]
Sea $f$ holomorfa en el disco unidad $\mathbb{D}$, con desarrollo de Taylor alrededor del origen:
\[f(z)=\sum_{n=0}^{\infty}a_nz^n\]
Denotemos por $s_n$ la suma parcial $s_n=\sum_{k=0}^n a_k$

\ppart Prueba que si $z \in \mathbb{D}$, entonces
\[|f(z)-s_n| \leq \sum_{k=0}^n|a_k||1-z^k|+\sup_{k \geq n}|a_k|\frac{1}{1-|z|}\]

\ppart Deduzca que si $z_n=1-1/n$, entonces
\[|f(z_0)-s_n|\leq \frac{1}{n}\sum_{k=0}^n |a_k|k+n\sup_{k \geq n}|a_k|\]

\ppart Concluya finalmente que si $\lim_{n\to\infty}na_n=0$ y que
\[\lim_{x \to 1^-}f(x) = \sum_{n=0}^{\infty}a_n\]
en caso de existir dicho límite


\textbf{Sugerencia:} Use el ejercicio anterior
\solution

\end{problem}
\newpage
\section{Hoja 3}

\begin{problem}[1]
Sea $ω_1,\cdots ω_n$ las n raíces n-ésimas de la unidad, es decir, $ω_j=e^{2πi(j/n)}$ con j=1,...,n.
\ppart Pruebe que si $m$ es un número natural, entonces
\[\frac{1}{n}\sum_{j=1}^n (ω_j)^m = \left\{ \begin{array}{lcc}
             0 &   si  & m \text{ no es múltiplo de  n} \\
             \\ 1 &  si  & m\text{ es múltiplo de n }
             \end{array}
   \right.\]

\ppart Si $P(z) = a_0+a_1z+\cdots a_nz^n+\cdots + a_mz^m, \ m \leq 2n-1$, demuestre que
\[\frac{1}{n}\sum_{j=1}^nP(ω_j)=a_0+a_n\]

\solution


\end{problem}

\begin{problem}[2]
\textit{Resolución de la ecuación cúbica:} Considramos la ecuación cúbica $z^3+az^2+bz+c=0$ con $a,b,c \in \cplex$
\ppart Aplique un cambio de variable $z=w+h$ para obtener una ecuación equivalente de la forma $ω^3+βω+γ=0$
\ppart Haga ahora un cambio $ω=gu$ que nos dé una ecuación de la forma $4u^3-3u+δ=0$
\ppart Sea $v \in \cplex$ tal que $\sin(3v)=δ$. Demuestre que α=$\sin(v)$ es raíz de la ecuación anterior
\ppart Aplique este procedimiento a la ecuación $z^3+3z^2-1=0$

\solution
\end{problem}

\begin{problem}[3]
Demuestre que existe un único polinomio $P_n$ de grado $n$ tal que
\[z^n+\frac{1}{z^n}=P_n\left( z + \frac{1}{z}\right)\]

\textbf{Ayuda:} Puede hacerse por inducción. El caso $n=1$ es obvio. Conviene escribir $z^n+\frac{1}{z^n}-\left( z+\frac{1}{z}\right)^n$ y usar que ${n \choose k} = {n \choose n-k}$
\solution
\end{problem}

\begin{problem}[4]
El objetivo de este ejercicio es demostrar el resultado conocido como \textit{el teorema de Gauss-Lucas} (apartado \textbf{c)}).

\ppart Demuestre que si $z_1,z_2,\cdots z_n$ son números complejos entonces el polígono convexo más pequeño que los contiene (posiblemente degenerado) viene dado por
\[\{z=t_1z_1+t_2z_2+\cdots + t_nz_n : \ t_i \geq 0 \text{ y } \sum t_i = 1\}\]

\ppart Compruebe que el polígono del apartado anterior es la intersección de todos los seimplanos que contiene a  $z_1,z_2,\cdots, z_n$.

\ppart Demuestre que si $P(z)$ es un polinomio, su derivada $P'(z)$ no puede tener ceros fuera del polígono convexo más pequeño que contiene a las raíces de $P(z)$

\textbf{Ayuda:} Escriba $P(z)=a_n(z-z_1)(z-z_2)\cdots (z-z_n)$ y considere $\frac{P'(z)}{P(z)}$ para expresar los ceros de $P'(z)$ en la forma $m_1z_1+\cdots + m_nz_n$ o use el apartado \textbf{b)}

\solution

\spart

\spart

\spart
Siguiendo la ayuda proporcionada escribimos
\[P(z)=a_n(z-z_1)(z-z_2)\cdots (z-z_n)\]
con lo que la fracción pedida queda de la forma:
\[\frac{P'(z)}{P(z)}=\sum_{i=1}^n \frac{1}{z-z_i} \; \forall z \neq z_i \text{ con } i)1,2,...,n\]

Sea $z$ un cero de $P'(z)$ que no anula el polinomio (es distinto de todo $z_i$), tenemos
\[\sum_{i=1}^n \frac{1}{z-z_i}=0\]

Ahora tenemos que comprobar que esta $z$ se contiene dentro del polinomio mencionado en el apartado \textbf{a)}.

Si multiplicamos y dividimos por el conjugado en cada uno de los términos del sumatorio que estamos manejando llegamos a:
\[\sum_{i=1}^n \frac{\bar{z}-\bar{z_i}}{|z-z_i|^2} = 0 \implies  \sum_{i=1}^n \frac{\bar{z_i}}{|z-z_i|^2} = \sum_{i=1}^n \frac{\bar{z}}{|z-z_i|^2}  \]

Si despejamos $z$ llegamos a:
\[z= \sum_{i=1}^n \frac{\frac{1}{|z-z_i|^2}}{\sum_{i=1}^n \frac{1}{|z-z_i|^2}}z_i\]

Si al coeficiente de $z_i$ lo llamamos $t_i$ podemos ver fácilmente que
\[\sum t_i = 1\]
con lo que el cero encontrado pertenece al polígono convexo en el que se encontraban los ceros del polinomio.
\end{problem}

\begin{problem}[5]
Desarrolle en series de potencias (centradas en el origen) las siguientes funciones elementales:
\begin{enumerate}
\item \[(1-z)\cos(z)\]
\item \[\frac{\cos(z)}{1-z^2}\]
\item \[\frac{e^{-z}}{1+z}\]
\end{enumerate}

indicando en cada caso el radio de convergencia. Hágase lo mismo para la función $\sin(2z)=2\sin(z)\cos(z)$ de dos maneras distintas. (Si no resulta fácil encontrar una fórmula general para los coeficientes, basta con escribir los 5 primeros términos e cada serie)
\solution

\begin{enumerate}
\item

\[(1-z)\cos(z) = (1-z)\sum_{n=0}^{\infty}\frac{(-1)^n}{(2n)!}z^{2n}=\sum_{n=0}^{\infty}\frac{(-1)^n}{(2n)!}z^{2n}-\sum_{n=0}^{\infty}\frac{(-1)^n}{(2n)!}z^{2n+1} =\]
\[= \sum_{k=0}^{\infty}a_kz^k\]
siendo
\[a_k= \left\{ \begin{array}{lcc}
             \frac{(-1)^{k/2}}{k!} &   si  & k \text{ es par} \\
             \\ \frac{(-1)^{(k-1)/2-1}}{k!} &  si  & k \text{ es impar}
             \end{array}
   \right.\]


\item
\[\frac{\cos(z)}{1-z^2}\]

En este caso sabemos desarrollar tanto el numerado como el denominador en series de potencias con lo que obtendremos algo de la forma:

\[\frac{\cos(z)}{1-z^2} = \left(\sum a_n z^n \right)\cdot \left( \sum b_nz^n\right)=\sum c_n z^n\]

Vamos a calcular ahora esos sumatorios:
\[\cos(z)=\sum a_n z^n \; \; \; \frac{1}{1-z^2} = \sum b_n z^n\]
siendo
\[a_n= \left\{ \begin{array}{lcc}
             \frac{(-1)^{n/2}}{k!} &   si  & n \text{ es par} \\
             \\0  &  si  & n \text{ es impar}
             \end{array}
   \right.  \; \; b_n= \left\{ \begin{array}{lcc}
             1 &   si  & n \text{ es par} \\
             \\0  &  si  & n \text{ es impar}
             \end{array}
   \right.\]

Por último, sólo nos queda calcular los coeficientes $c_n$. Vamos a ello:
\[c_n = \sum_{k=0}^n a_kb_{n-k} = \sum_{0\leq k \leq n} \frac{(-1)^{n/2}}{k!} = \sum_{j=0}^{n/2} \frac{(-1)^{j}}{(2j)!}\]
siendo esto válido sólo para los $n$ pares. Cuando $n$ sea impar tendremos $c_n=0$


\item
\[\frac{e^{-z}}{1+z}\]


\end{enumerate}

Vamos a por el $\sin(2z)$ que podemos escribir como serie de la siguiente forma
\[\sin(2z) =\sum_{n=0}^{\infty} \frac{(-1)^n\cdot 2^{2n+1}}{(2n+1)!}z^{2n+1}\]

la segunda forma de calcularlo a la que se refiere el enunciado es el método empleado al resolver los apartados anteriores de este mismo ejercicio. Desarrollamos el seno por un lado, el coseno por otro y multiplicamos.

\end{problem}

\begin{problem}[6]
Escriba explícitamente la función cuya serie de potencias es
\[f(z)=\sum_{n=0}^{\infty}\frac{z^{4n}}{(4n)!}\]
¿Cuánto vale $\sum_{n=0}^{\infty} \frac{1}{(4n)!}$?

\textbf{Ayuda:} Evalúe la función exponencial en los puntos $\pm z $ y $\pm iz$
\solution

Atendiendo a la sugerencia y evaluamos la función exponencial en los puntos indicados obtenemos:
\[e^z = \sum_{n=0}^{\infty}\frac{1}{n!}z^n; \; \; e^{-z}=\sum_{n=0}^{\infty} \frac{(-1)^n}{n!}z^n ;\;\; e^{iz} = \sum_{n=0}^{\infty}\frac{i^n}{n!}z^n, \;\; e^{-iz} = \sum_{n=0}^{\infty}\frac{(-1)^ni^n}{n!}z^n\]

Si sumamos las dos primeras y las dos últimas obtenemos:
\[\frac{1}{2}(e^z+e^{-z}) =\sum_{k=0}^{\infty} \frac{1}{(2k)!}z^{2k}\]
\[\frac{1}{2}(e^{iz}+e^{-iz}) =\sum_{k=0}^{\infty} \frac{i^{2k}}{(2k)!}z^{2k}=\sum_{k=0}^{\infty} \frac{(-1)^k}{(2k)!}z^{2k}\]

Si sumamos, estamos eliminando aquellas en las que $k$ es par, es decir, el exponente de la $z$ será múltiplo de 4:
\[\frac{1}{2}\left(\frac{1}{2}(e^z+e^{-z})+\frac{1}{2}(e^{iz}+e^{-iz})\right) = \sum_{n=0}^{\infty} \frac{(-1)^n\cdot 2^{2n+1}}{(2n+1)!}z^{2n+1}\]
y vemos que la función pedida, la parte de la izquierda de esta desigualdad, puede escribirse de forma simplicada como:
\[\frac{1}{2} (\cosh(z) + \cos(z))\]
\end{problem}

\begin{problem}[7]
¿Para qué valores de $z \in \cplex$ se cumple que $\overline{e^{iz}} = e^{i\bar{z}}$?
\solution
Por un lado tenemos:
\[\overline{e^{iz}} = \overline{e^{i(x+iy)}} = \overline{e^{ix}}\overline{e^{-y}} =e^{-ix}e^{-y}\]

Y por otro:
\[e^{i\bar{z}}=e^{xi+y}=e^{xi}e^{y}\]
si queremos que coincidan estos dos números complejos necesitamos $y=0$ y que
\[\cos(x) = \cos(-x) \ \; \; \sin(x)=\sin(-x) \footnote{Obtenemos estas ecuaciones descomponiendo la exponencial en senos y cosenos} \]

%TODO completar
\end{problem}

\begin{problem}[8]
Demuestre que
\ppart $\sin^2(z)+\cos^2(z) = 1$
\ppart $\cos(2z)=\cos^2(z)-\sin^2(z)$
\solution

\spart
Veamos primero una forma elegante de hacer esta demostración suponiendo que ya sepamos que es cierta en los reales.

Tenemos que la función $\sen^2(z)+\cos^2(z)-1$ se anula en toda la recta de los reales. Como ya vimos en algunos ejercicios, una función holomorfa que sea hace 0 en todo un eje es la función nula.
%TODO completar esto como se hizo en los apuntes

La otra forma de hacerlo, suponiendo que no sepamos nada de lo que ocurre en los reales, es escribir los senos y cosenos como exponenciales de la siguiente forma:
\[\sin(z)=\frac{1}{2i}\left( e^{iz}-e^{-iz}\right)\]
\[\cos(z)=\frac{1}{2}\left( e^{iz}+e^{-iz}\right)\]

Si ahora sumamos los cuadrados obtenemos diractemente un 1.

\spart

Nuevamente, podría hacerse de dos formas al igual que el ejercicio anterior.

\end{problem}

\begin{problem}[9]
Resuelva las siguientes ecuaciones
\ppart $\cos(z)=2$
\ppart $\sin(z)=\frac{3}{4}-\frac{i}{4}$
\solution

\spart
Escribiéndolo como exponenciales tenemos la ecuación:
\[\cos(z)=2 \iff \frac{1}{2}\left( e^{iz}+e^{-iz}\right) = 2 \iff e^{iz}+\frac{1}{e^iz}=4\]

Haciendo el cambio de variable $e^{iz}=t$ planteamos la ecuación de segundo grado:
\[t^2-4t+1=0 \implies t=2\pm \sqrt{3}\]

Ahora debemos despejar la $z$ a partir de la relación $e^{iz}=2\pm \sqrt{3}$.

\[\begin{cases}
t_1=2+\sqrt{3} \implies e^{iz_1} = 2 + \sqrt{3} \implies iz_1 = \log(2+\sqrt{3})+2kπi \implies z_1=-i\log(2+\sqrt{3})+2kπ\\
t_2=2-\sqrt{3} \implies e^{iz_2} = 2 - \sqrt{3} \implies iz_2 = \log(2-\sqrt{3})+2kπi \implies z_2=-i\log(2-\sqrt{3})+2kπ.
\end{cases}\]

\spart
Escribiéndolo nuevamente como exponenciales tenemos:
\[\frac{1}{2i}\left( e^{iz}-e^{-iz}\right) = \frac{3}{4}-\frac{i}{4} \]

Realizando el mismo cambio de variable que hicimos en el apartado anterior y agrupando los coeficientes llegamos a la ecuación de segundo grado:

\[t^2+\frac{1}{2}(1+3i)t-1=0\]

que nos da las soluciones:
\[t= \frac{1}{2} \left[ \frac{1}{2}(1+3i)\pm \sqrt{\frac{1}{4}(1+i)^2+4}\right]\]

Vamos a trabajar un poco con el número que hay dentro de la raíz para hacerlo lucir mejor.
\[\frac{1}{4}(1+i)^2+4 = \frac{1}{4}(1+6i-9)+4 = 2+\frac{3}{2}i = \frac{5}{2}e^{α} \; \text{ con } \tg(α)=\frac{3}{4}\]

Volvemos ahora al valor de $t$ y aplicamos esta simplifiación que acabamos de calcular:
\[t=\frac{1}{2}\left[ \frac{1}{2}(1+3i)\pm \left(\frac{5}{2} \right)^{1/2}e^{iα/2}\right]\]

Escribiendo la exponencial como combinación de senos y cosenos y jugando un poco con la trigonometría legamos fácilmente a:
\[t=\frac{1}{2}\left[ \frac{1}{2}(1+3i)\pm \frac{\sqrt{5}}{\sqrt{2}} \frac{1}{\sqrt{10}}(3+i)\right] = \frac{1}{4}\left[ 1+3i\pm (3+i)\right]\]

Ahora, de forma idéntica al ejercicio anterior, calculamos el valor de $z$ resolviendo las ecuaciones $e^{iz}=t_1$ y $e^{iz}=t_2$.

\[\begin{cases}
t_1 = 1+i \rightarrow e^{iz}=t_1 \implies iz=\log|t_1|+arg(t_1) \implies z = \frac{π}{4}+2kπ-i\frac{1}{2}\log(2) \\
t_2 = 1/2(-1+i) \rightarrow e^{iz}=t_2 \implies iz=\log|t_2|+arg(t_2) \implies z = \frac{3π}{4}+2kπ+i\frac{1}{2}\log(2)
\end{cases}\]
\end{problem}

\begin{problem}[10]
Calcule los siguientes valores:

\ppart
\[e^{iπ/4}, \; e^{5πi/4}, \; e^{-7πi/3}, \; exp\left[π \left(\frac{1+i}{\sqrt{2}} \right)^4 \right], \; \cos(2+3i), \; \sin(1+i)\]

\ppart
\[(1-i)^i \; 2^{-1+i}, \; i^{\sqrt{2}} \text{tomando la rama principal del logaritmo}\]

\ppart
\[i^{-i}, \; \log(3) , \; \log(\sqrt{3}+1), \; (1+i)^{1+i}, \; 2^{πi} \text{ (calcular todos los posibles valores)}\]

\solution
\end{problem}

\begin{problem}[11]
Denotemos por $\{ arg(z)  \}$ el conjunto de todos los valores posibles para el argumento de $z$, por $\{ \log(z) \}$ el conjunto de todos los valores posibles de $\log(z)$ y por $\{ z^b \}$ el conjunto de todos los valores posibles de $z^b$ con el significado evidente $\{\log(z)\}=\log(|z|)+i\{arg(z)\}$, $\{z^b \} = e^{b\{\log(z)\}}$. Compruebe que:

\ppart
\[\{\log(zw)\} = \{\log(z)\}+\{\log(w)\} \text{ (aquí A+b=\{a+b: a $\in A$, b $\in B$\})}\]
\ppart
\[\{(zw)^b\} = \{z^b\} \{w^b\} \text{ (aquí AB=\{ab: a $\in$ A, b $\in$ B\})}\]
\ppart
\[\{ \log(z^α)\} = \bigcup_{k \in \ent}(α\{\log(z)\}+2kπi)\]
\solution

\end{problem}

\begin{problem}[12]
(Teorema del binomio para exponentes reales) Sea α un número real con $α \notin \nat$ y sea
\[{α \choose 0}=1, \; {α \choose 1}=α, \; {α \choose j} = \frac{α(α-1)\cdots (α-j+1)}{j!} \text{ si } j>1\]

\ppart
Demuestre que el radio de convergencia de la serie $F(z) = \sum_{k=0}^{\infty}{α \choose k}z^k$ es 1

\ppart
Compruebe que $(1+z)F'(z)=αF(z)$

\ppart
Concluya que $F(z)=(1+z)^n$, es decir, $(1+z)^α = \sum_{k=0}^{\infty}{α \choose k} z^k$ si $|z|<1$.
(Aquí se toma la rama principal de $w^α$)

\solution
Aunque no lo indica expresamente el enunciado, es claroq ue $j \in \ent^+$ ya que de lo contrario no tendrían sentidos los números combinatorios que aparecen en el ejercicio.

\spart

\[R= \lim_{k \to \infty}\left|\frac{a_k}{a_{k+1}} \right| = \lim_{k\to\infty}\left| \frac{k+1}{α-k} \right| = 1\]

\spart


\spart

\end{problem}