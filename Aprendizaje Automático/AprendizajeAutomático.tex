\documentclass{apuntes}

\title{Aprendizaje Automático}
\author{Víctor de Juan}
\date{15/16 C1}

% Paquetes adicionales

% --------------------

\begin{document}
\pagestyle{plain}
\maketitle

\tableofcontents
\newpage
% Contenido.

\chapter{Introducción}


Aristóteles en el siglo IV a.c. dice que el razonamiento deductivo se analiza aplicando 2 silogismos fuertes.

\begin{defn}[Silogismos fuertes] Premisa:  $A \implies B$
\begin{itemize}
\item[1] Si observamos A, deducimos B.
\item[2] Si observamos $\neg B$, deducimos que $\neg A$.
\end{itemize}
\end{defn}

\begin{defn}[Silogismos débiles] 
Si vemos la calle mojada, $\implies$ lo más probable es que haya llovido.
\end{defn}


\section{Introducción a la probabilidad}

Vamos a repasar conceptos de probabilidad para poder utilizar los silogismos débiles.

\begin{defn}[dominio]
Todos los posibles resultados de un experimento aleatorio.
\end{defn}

\begin{defn}[Variable aleatoria]
Variable que identifica el experimento aleatorio
\end{defn}

\begin{defn}[Suceso]
Valor que toma la variable aleatoria al realizar el experimento aleatorio.
\end{defn}

\begin{example} 
Tirar un dado.

\begin{itemize}
	\item Dominio $= \{1,2,3,4,5,6\}$
	\item Variable $X$ representa tirar un dado.
	\item Posibles sucesos: $X=1,X=2,...,X=6$
\end{itemize}

A partir de esto ya podemos calcular probabilidades: 
\[ P\left\(X\text{ par}\right\)\]
\[ P\left\(X\text{ primo}\right\)\]
\end{example}


Más repaso de probabilidad: probabilidad condicionada (regla de la suma) 

\begin{defn}[Regla del producto]
\[P(A,B|I) = P(A|B,I) · P(B|I) = P(B|A,I) · P(A|I)\]
\end{defn}

Vamos a ver cómo utilizar esta regla combinándola con los silogismos:

Podemos incluir el primer silogismo fuerte \ref{silogismofuerte} hacienndo $I= A\implies B$, entonces tenemos:
\[P(A,B|A\implies B) = ... = 1\]
Y aplicando el segundo silogismo fuerte, tenemos:
\[P(A|\bar{B}, A\implies B) = ... = 0\]

Sin embargo, utilizando los silogismos débiles:

\[ P(A|B,A\implies B) = \frac{P(A|A\implies B)}{P(B|A\implies B)} \geq P(A|A\implies B)\]

%% Apéndices (ejercicios, exámenes)
\appendix

\chapter{Ejercicios}
\printindex

\end{document}
