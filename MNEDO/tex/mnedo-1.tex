\section{Hoja 1}

% %Problem
\begin{mathproblem}{1}
Considerar el PVI:
\begin{equation*}
\begin{array}{c c c}
y'=f(t) & t\in[t_0, T], & y(t_0) = 0.
\end{array}
\end{equation*}
Probar utilizando el método de Euler que
$$y_N=\sum_{k=0}^{N-1}f(t_k)h$$
\end{mathproblem}
El método de Euler tiene la expresión:
$$y_{n+1} = y_n + hf(t_n, y_n)$$
En este caso, dado que $f = f(t)$, se tiene 
$$y_{n+1} = y_n + hf(t_n)$$
Vamos a utilizar la inducción para la demostración:

\begin{itemize}
\item Hipótesis:

Suponemos que $$y_N=\sum_{k=0}^{N-1}f(t_k)h$$
se cumple $\forall N\in \mathbb{N}\backslash \{ 0 \}$

\item Base de inducción: \framebox{$N=1$}

Para $N=1$, por hipótesis, se tiene $$y_1=\sum_{k=0}^{N-1}f(t_k)h = hf(t_0)$$

Utilizando el método de Euler obtenemos $$y_1 = y_0 + hf(t_0)$$

Dado que $y_0 = y(t_0) = 0$ tenemos que la hipótesis es cierta para $N=1$.

\item Inducción:

Supongamos la hipótesis cierta para $N$. Vamos a demostrar que es válida para $N+1$.

Por el método de Euler:
$$y_{N+1}=y_{N}+f(t_{N})h$$

Usando la hipótesis, sustituimos el primer término del lado derecho de la igualdad:
$$y_{N+1}=\sum_{k=0}^{N-1}f(t_k)h+f(t_{N})h$$
Por tanto se tiene:
$$y_{N+1}=\sum_{k=0}^{N}f(t_k)h$$

\qed


% % Problem
\begin{mathproblem}{2}
Ver que el método de Euler falla cuando queremos aproximar la solución
$$y(t) = t^{\frac{3}{2}}$$
del PVI
\begin{equation*}
\begin{array}{c c}
y' = \frac{3}{2}y^{\frac{1}{3}} &
y(0) = 0
\end{array}
\end{equation*}
¿Cuál es el problema? Justificar la respuesta.
\end{mathproblem}
El método de Euler tiene la expresión:
$$y_{n+1} = y_n + hf(t_n, y_n)$$
En este caso, dado que $f(t, y(t)) = \frac{3}{2}y^{\frac{1}{3}}$, se tiene 
$$y_{n+1} = y_n + hf(y_n)$$

Veamos si el método aproxima la solución dada:
\begin{equation*}
\begin{array}{l l}
y_0 & = 0\\
y_1 & = y_0 + hf(y_0) = 0 + hf(0) = 0\\
\vdots & \\
y_n & = y_{n-1} + hf(y_{n-1}) = 0 + hf(0) = 0
\end{array}
\end{equation*}
Como se puede observar, el método de Euler aproxima la solución $y = 0$, que también es solución del PVI.

Dado que tenemos dos soluciones distintas vemos que la función $f$ no es Lipschitz con respecto a $y(t)$.

Para que $f$ sea Lipschitz con respecto a $y$, se tiene que cumplir que:
$$|f(y) - f(\bar{y})| \le L|y-\bar{y} |$$
para un cierto $L>0$.
$$|f(y) - f(\bar{y})| = |\frac{3}{2}y^{\frac{1}{3}} - \frac{3}{2}\bar{y}^{\frac{1}{3}}| \le |y^{\frac{1}{3}} - \bar{y}^{\frac{1}{3}}| $$
Si se tuviera que $$|y^{\frac{1}{3}} - \bar{y}^{\frac{1}{3}}| \le L|y -\bar{y}|$$ para un cierto $L>0$, entonces $f$ sería Lipschitz con respecto a $y$.
Por el Teorema del valor medio se tiene:
$$|y^{\frac{1}{3}} - \bar{y}^{\frac{1}{3}}| \le |f'(\xi)||y -\bar{y}|$$
con $\xi \in (y, \bar{y})$. 

Sin embargo, $f'(\xi) = \frac{1}{3\xi{\frac{2}{3}}}$ no está acotada en un entorno de 0. Por lo que no podemos asegurar la unicidad de soluciones en un entorno de dicho punto.
\end{itemize}
\qed


% % Problem
\begin{mathproblem}{3}
Sea el PVI
\begin{equation*}
\begin{array}{c c c}
y' = 1+y^2 & t\in [0,2] & y(0) = 0
\end{array}
\end{equation*}
¿Podemos usar el método de Euler para aproximar la solución $y(t) = tan(t)$?
\end{mathproblem}
Vamos a ver si la función $f(y) = 1+y^2$ es Lipschitz con respecto a $y(t)$ para asegurar la existencia y unicidad de soluciones.
Se tiene que cumplir que
$$|f(y) - f(\bar{y})| < L|y-\bar{y}|$$
Sin embargo, dado que
\begin{equation*}
\begin{array}{l l}
|f(y)-f(\bar{y})| & = |1+y^2-(1-\bar{y}^2)| = |y^2-\bar{y}^2|\\
& = |(y-\bar{y})(y+\bar{y})| = |y-\bar{y}||y+\bar{y}|
\end{array}
\end{equation*}
se tendría que cumplir que $$|y+\bar{y}||y-\bar{y}| \leq L|y-\bar{y}|$$
es decir, que 
$$|y+\bar{y}| \leq L $$
para un cierto $L>0$. Lo cual es imposible.\qed

Por tanto, el método de Euler puede fallar al no estar asegurada la existencia y unicidad de soluciones.

Veamos si falla:
\begin{equation*}
\begin{array}{l l}
y_0 & = 0\\
y_1 & = y_0+hf(y_0) = 0 + hf(0) = h\\
y_2 & = y_1+hf(y_1) = y_1 + h(1+y_1^2) = h + y_1 + hy_1^2\\
\vdots& \\
y_n & = y_{n-1}+hf(y_{n-1}) = h + y_{n-1} + hy_{n-1}^2
\end{array}
\end{equation*}
Dado que cada término es un polinomio de $h$ sin término independiente, vemos que $$\lim_{h\to 0} y_n = 0$$ Por lo que el método de Euler converge a $y=0$, que también es solución del PVI.


% % Problem.
\begin{mathproblem}{4}
Calcular el residuo de los siguientes métodos:

(a) Regla del trapecio:
$$y_{n+1}=y_n+\frac{h}{2}(f(t_{n+1},y_{n+1})+f(t_n, y_n))$$

(b) Euler modificado:
$$y_{n+1}=y_n+hf(t_n+\frac{h}{2}, y_n+\frac{h}{2}f(t_n, y_n))$$

(c) Leap-frog:
$$y_{n+2} = y_n + 2hf(t_{n+1}, y_{n+1})$$
\end{mathproblem}
En este problema se va a utilizar la siguiente notación: 

$y(t_n)$ es el valor exacto de la solución a un PVI en el punto $t_n$, mientras que $y_n$ es el valor aproximado de la función, en el mismo punto, obtenido por el método correspondiente. 

Por tanto, el residuo viene dado por $R_n = y(t_{n+1}) - y_{n+1}$

\vspace{5mm}
\noindent\textbf{Regla del trapecio:}

\begin{equation*}
\begin{array}{l l l l}
R_n = & y(t_{n+1}) - y_{n+1} & = \\
 & y(t_{n+1}) - (y_n+\frac{h}{2}(f(t_{n+1},y_{n+1})+f(t_n, y_n))) & = & ^{(1)}\\
 & y(t_n)+hy'(t_n)+\frac{h^2}{2}y''(t_n)+\frac{h^3}{3!}y''(\bar{\xi_n}) & - & ^{(2)}\\ 
 & (y_n+\frac{h}{2}(f(t_{n+1},y_{n+1})+f(t_n, y_n))) & &  
\end{array}
\end{equation*}

Usando que $y'(t) = f(t, y(t))$ y desarrollando el polinomio de Taylor de orden 2 de $f(t_{n+1}, y_{n+1}) = y'(t_{n+1})$ alrededor de $t_n$ se tiene:

\begin{equation*}
\begin{array}{l l l}
R_n & = y(t_n)+hy'(t_n)+\frac{h^2}{2}y''(t_n)+\frac{h^3}{3!}y''(\bar{\xi_n}) &  \\
	& - (y(t_n) + \frac{h}{2}(y'(t_n) + y'(t_n) + hy''(t_n) + \frac{1}{2}h^2y'''(\bar{\eta_n}))) & ^{(3)}
\end{array}
\end{equation*}

Realizando unas pocas operaciones se obtiene que
$$R_n = h^3\left(\frac{y'''(\bar{\xi_n})}{3!} - \frac{y'''(\bar{\eta_n})}{4}\right)$$

\footnotetext[1]{Se ha desarrollado el polinomio de Taylor de orden 3 de $y(t_{n+1})$ alrededor de $t_n$}
\footnotetext[2]{$\bar{\xi_n} \in [t_n, t_{n+1}]$ puede variar para cada componente.}
\footnotetext[3]{$\bar{\eta_n} \in [t_n, t_{n+1}]$ puede variar para cada componente.}

\vspace{5mm}
\noindent\textbf{Newton Modificado:}

\color{red}{En proceso}
