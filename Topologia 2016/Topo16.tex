\documentclass[palatino]{apuntes}

\title{Topologia 2016}
\author{Pablo Pérez Manso}
\date{16/17 C1}

% Paquetes adicionales

% --------------------

\begin{document}
\pagestyle{plain}
\maketitle

\tableofcontents
\newpage
% Contenido.

\begin{defn}[Topología]
$X \neq \varnothing$, conjunto. Sea $\tau \in \mathcal{P}(X)$. Si $\tau$ satisface:
\begin{enumerate}
\item $\varnothing \in \tau, X \in \tau$
\item Sea $\{A_\lambda\}_{\lambda \in \Lambda}$ con $ A_\lambda \in \tau, \forall \lambda \in \Lambda \implies  (\bigcup_{\lambda \in \Lambda} A_\lambda) \in \tau$
\item $A \in \tau, B \in \tau \implies A \cap B \in \tau$
\end{enumerate}
\end{defn}



\begin{defn}[Abierto]
Sea $K \subset X$, diremos que K es cerrado para $\tau$ si $C_XK\in\tau$.
\end{defn}
\begin{defn}[Cerrado]
Sea $K \subset X$, diremos que K es cerrado para $\tau$ si $C_XK\in\tau$.
\end{defn}

\begin{example}
	Si $\tau = \mathcal{P}(X) \implies \tau $ es una topología ya que cumple las tres propiedades de una topología. (ejercicio)

	\begin{itemize}
		\item Para $x_0 \in X, G=\{x_0\} \in \tau$, luego G es abierto para $\tau$
		\item Para $x_0 \in X, C_XG=X\setminus\{x_0\} \in \tau$, luego $C_XG$ es abierto para $\tau$
	\end{itemize}
\end{example}

\begin{defn}[Topología discreta]
$\tau$ es una topología discreta si todos los subconjuntos son abiertos
\end{defn}


\begin{example}
$\tau = \{\varnothing, X\} \in \mathcal{P}(X) \implies \tau$ es una topología. Cumple las tres propiedades (ejercicio)
\end{example}


\begin{defn}[Topología trivial]
$\tau$ es una topología trivial si sólo tiene dos subconjuntos $\{\varnothing,X\}$
\end{defn}



\begin{example}
$X = \{a,b,c\}$ y $\tau = \{\varnothing, X, \{a\}, \{b,c\}\}$. ¿Es $\tau$ una topología? Sí, ya que cumple las tres propiedades (ejercicio).
\begin{itemize}
	\item $\{a\}, \{b,c\}$ son a la vez abiertos y cerrados ya que están en $\tau$ y su complementario también lo está.
	\item $\{b\}$ no es abierto porque no pertenece a la topología, pero tampoco es cerrado ya que $C_X\{b\}=\{a,c\}$ tampoco pertenece a la topología.
\end{itemize}
\end{example}



\begin{problem}
$\tau = \{\varnothing,X,\{a,b\},\{b,c\}\}$ ¿Es $\tau$ una topología?
\solution
No, ya que las intersecciones finitas deben esar en la topología y $\{a,b\} \cap \{b,c\}$ no pertenece a la topología.
\end{problem}


Faltan un montón de clases, ya me pondré con ello




\section{Topología cociente}

\begin{dfn}[Aplicación cociente]
Dados X,Y espacios topológicos, y p $X\twoheadrightarrow^p Y, p$sobreyectiva, se dice que p es aplicación cociente si:

U abierto en Y $\dimplies p^-1$(U) abierto en X
\end{dfn}

\textbf{Nota:} $(X, \tau_X), (Y, \tau_Y)$ $\{p^{-1}(U: U\ \subset \tau_Y)\} \subset \tau_X$

\textbf{Nota:} Si p es cociente $\implies$ p es continua

\begin{prop}
Sea $p: X \twoheadrightarrow Y$ sobreyectiva, continua y abierta $\implies$ p es una aplicación cociente
\end{prop}

\begin{proof}
Sea U abierto en Y \implies $p^-1(U)$ abierto (por ser p continua)

Sea $\Omega = p^{-1}(V)$ abierto en X $\implies$ V abierto

$p(\Sigma) = pp^{-1}(V) abierto en Y \implies V abierto en Y$
\end{proof}



\begin{defn}
$\mathcal{C}$ \subset X. Diremos que $\mathcal{C}$ es saturado con respecto de la aplicación sobreyectiva $p: X\implies Y$ si $\matcal{C}$ contiene cada fibra que lo corta (es decir, $y\inY Y p^{-1}(y)\cap\matcal{C} \neq \varnothing \implies p^{-1}(y) \subset \mathcal{C} $)
\end{defn}


\begin{example}
$\{(cos(2\pi t), sen (2\pi t)) \in \mathds{S}^1 \tq t \in (0, \fraq{1}{4}) \} = \matcal{C}$

$p^{-1}(\mathcal{C}) = \{ x \in \mathds{R} \tq (cos(2\pi t), sen (2\pi t)) \in \mathcal{C} \} = \bigcup_{x\in(0,1/4)}x + \mathds{Z}  = \bigcup_{x\in(0,1/4)} p^{-1}(p_x) $
\end{example}


\begin{thm}
Son equivalentes:
\begin{enumerate}
	\item p aplicación cociente
	\item p continua que lleva abiertos saturados de X en abiertos de Y
\end{enumerate}
\end{thm}

























%% Apendices (ejercicios, examenes)
\appendix

\chapter{---}
% -*- root: ../Topo16.tex -*-


\printindex
\end{document}
