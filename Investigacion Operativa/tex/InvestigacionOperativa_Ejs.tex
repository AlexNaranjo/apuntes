 -*- root: ../InvestigacionOperativa.tex -*-

\section{Hoja 1}

\begin{problem}[1]

Una empresa de reciclaje usa papel y tela desechados para fabricar dos tipos distintos de papel reciclado.
Cada tanda de papel reciclado de clase A requiere 20 kg de tela y 180 kg de papel y produce un beneficio de 500 euros, mientras que cada tanda de papel reciclado de clase B requiere 10 kg de tela y 150 kg de papel y produce un beneficio de 250 euros. 
La compañía dispone de 100 kg de tela y 660 kg de papel. ¿Cuántas tandas debe fabricar de cada tipo?

\solution

\begin{center}
\begin{tabular}{c|ccc}
& A & B & Disp. \\\hline
Tela & 20 & 10 & 100\\
Papel & 180 & 150 & 660\\
Beneficio & 500 & 250 & 
\end{tabular}
\end{center}

Variables $x = $ calidad de A e $y = $ calidad de B.

\begin{ioprob}
\goal{$\max 500x_1 + 250 x_2$}
\restrictions{$20x_1 + 10x_2 \leq 100$}{$180x_1 + 160x_2 \leq 660$}{$x_i \geq 0$}{}{}{}
\end{ioprob}

Vamos a resolverlo gráficamente en la figura \ref{ej:1.1.a}. Vamos a reprensentar el conjunto del plano que cumple las 3 restricciones.


\begin{figure}[hbtp]
\centering
\begin{tikzpicture}[scale=0.6]
\draw[thick,->] (-1,0) -- (10,0) node[anchor=west] {$x$};
\draw[thick,->] (0,-1) -- (0,10) node[anchor=east] {$y$};
\draw[thick,-] (5,0) -- node[anchor=north west] {\text{ }$20x + 10 y = 100$} (0,10);
\draw[thick,-] (3.8,0) -- node[anchor=west] {$180x + 150y = 660$} (0,4);
\filldraw[fill=blue!40!white, pattern=north west lines, pattern color=blue] (0,0) -- (0,4) -- (3.8,0);
\draw[thick,-,color=red] (2,-2) -- (-1,4);
\draw (3.8,0) node[anchor=north]  {$(3.8,0)$};
\draw [shorten >= 3cm, shorten <= -4cm,color=red,thick] (3.8,0) -- +($(2,-2)-(-1,4)$);
\end{tikzpicture}
\label{ej:1.1.a}
\caption{Representamos las 2 rectas fronteras y el vector gradiente de la función objetivo.}
\end{figure}


La idea es mover la recta roja en su dirección perpendicular todo lo que podamos. En este caso, no podremos alejarnos más que el punto que del punto $(3.8,0)$, con lo que será el óptimo.

\end{problem}


\begin{problem}[2]

La empresa Animales Salvajes S.A. cría faisanes y perdices para repoblar el bosque y dispone de sitio para criar 100 pájaros durante la temporada.
Criar un faisán cuesta 20 euros y criar una perdiz cuesta 30 euros. 
La fundación Vida Animal paga a Animales Salvajes S.A. por los pájaros de forma que se obtiene un beneficio de 14 euros por cada faisán y 16 euros por cada perdiz. 
La empresa dispone de 2400 euros para cubrir costes. ¿Cuántas perdices y cuántos faisanes debe criar?


\solution 


\begin{center}
\begin{tabular}{c|cccc}
&Faisán & Perdiz & Disp. & \# pájaros \\\hline
Coste&20&30&2400&100\\
Beneficio&14&16&
\end{tabular}
\end{center}

Las variables utilizadas son $x$ para Faisán e $y$ para Perdiz.

El problema a resolver sería:

\begin{ioprob}
\goal{$\max 14x + 16y$}
\restrictions{$x+y\leq 100$}{$20x + 30y \leq 2400$}{$x,y > 0$}{}{}{}
\end{ioprob}

En la figura \ref{ej:1.2.a} encontramos la solución gráfica del problema. Vemos que la solución es la intersección de las rectas, así que calculamos la intersección: 
\[
\left.
	\begin{array}{cc}
		x+y = 100 \\ 
		20x+30y = 2400
	\end{array}
\right\} 
\to (x,y) = (40,60)
\]

\begin{figure}[h]
\centering
\begin{tikzpicture}[scale=0.6]
\draw[thick,->] (-1,0) -- (12.5,0) node[anchor=west] {$x$};
\draw[thick,->] (0,-1) -- (0,12.5) node[anchor=east] {$y$};
\draw[thick,-] (10,0) --  node[anchor=west] {$x+y=100$} (0,10);
\draw[thick,-] (8,0) -- (0,12) node[anchor=west] {$20x + 30y = 2400$};
\filldraw[fill=blue!40!white, pattern=north west lines, pattern color=blue] (0,0) -- (0,10) -- (4,6) -- (8,0);
\draw[thick,-,color=red] (3.5,0) -- (0,4);
%\draw[thick,-,color=green] (9,0) -- (0,11);
\draw [shorten <= -4cm,shorten >= -2cm, ,color=green,thick] (4,6) -- +($(3.5,0)-(0,4)$);
\end{tikzpicture}
\label{ej:1.2.a}
\caption{Representamos las 2 rectas fronteras y el vector gradiente de la función objetivo. En rojo la dirección del gradiente y en verde la solución óptima.}
\end{figure}





\end{problem}


\begin{problem}[3]

La siguiente tabla da el porcentaje de proteínas, grasas y carbohidratos, para cinco alimentos
básicos, A, B, C, D y E :


Los precios por 100 g de estos alimentos (dados en el mismo orden de la tabla) son 5, 17, 37, 10,
15. Si una persona necesita consumir como mínimo 75 gramos de proteínas, 90 de grasas y 300 de
hidratos de carbono, plantea el problema de minimización para calcular la dieta alimenticia de mínimo coste.
\solution


\end{problem}


\begin{problem}[9]

Una propiedad conocida de la mediana de un conjunto de datos $y_i$ es que minimiza en $\theta$ el valor de $\sum y_i-\theta$. Plantea el problema de optimización como un problema lineal en forma estándar.
\solution

Queremos minimizar 
\[\sum_{i=1}^n |y_i - \theta|\]

El procedimiento habitual podría ser derivar e igualar a 0, pero en este caso no podemos ir por ese camino, ya que no es derivable.
Vamos a ver que la solución es la mediana y vamos a demostrarlo de 2 maneras distintas. 
Primero, formalmente y después, planteándolo como un problema de optimización lineal.

\begin{proof}

Tomando una muestra de tamaño 2 y un punto interior de $\theta$ el objetivo a minimizar es:
\[\sum |y_i - \theta| = \theta - y_{(1)} + y_{(2)}-\theta = y_{(2)} - y_{(1)}\]
es decir, la longitud del intervalo.

Vamos a tomar una muestra de tamaño $n$. 
En esa muestra,tTomamos una serie de intervalos contenidos de la siguiente manera:

\[ [y_{(1)},y_{(n)}] \supset  [y_{(2)},y_{(n-1)}] \supset [y_{(3)},y_{(n-2)}] \supset ... \supset [y_{(m)},y_{(n-m+1)}]\]

Donde \[m=\left\{ \begin{array}{cc} \frac{n}{2} & n\text{ par}\\ ?? & n\text{ impar} \end{array}\right.\]
Este es un razonamiento casi geométrico de porque la mediana minimiza.
\end{proof}


Ahora, vamos a contestar al enunciado, planteándolo como un problema de regresión.

Definimos $x_i \equiv y_i - \theta = x_i^+ - x_i^-$ donde $x_i^+ = \max\{x_i,0\}$ y $x_i^- = \max\{-x_i,0\}$

De esta manera, $\abs{x_i} = x_i^+ + x_i^-$. Con este truco, hemos conseguido modificar la función objetivo, que de esta manera es una función lineal.

Nuestra función objetivo es:

\[\min \sum_{i=1}^n x_1^+ + x_i^-\]

Y el precio a pagar, es que necesitamos incluir una restricciones, con lo que:


\begin{ioprob}
\goal{\[\min \sum_{i=1}^n x_1^+ + x_i^-\]}
\restrictions{$y_i = x_i^+ - x_i^- + \theta^+ + \theta^-$, $i=1,...,n$}{$x_i^+ \geq 0$}{$x_i^- \geq 0$}{$\theta_i^+ \geq 0$}{$\theta_i^+ \geq 0$}{}
\end{ioprob}


Vamos a escribirlo matricialmente.
Las \textbf{variables de decisión} son $(x_1^+,...,x_n^+,x_1^-,...,x_n^-,\theta^+,\theta^-$. 

\[c = (\underbrace{1,...,1}_{n},\underbrace{1,...,1}_{n},0,0)\]
\[b = (y_1,...,y_n) \]
\[A = \left( I | -I | 1_n | -1_n\right)\to \begin{array}{c}1_n = \begin{pmatrix}1\\1\\\vdots\\1\end{pmatrix}\\ I = \text{ identidad} \end{array}\]




\obs{}
¿Qué utilidad puede tener esto?
Al tomar un modelo de regresión (una recta que pase por una nube de puntos) se suele tomar el criterio de "mínimos cuadrados", que es minimizar $\sum y_i-(β_0+β_1x_i)$ (siendo $β_0+β_1x$ el modelo de regresión). 
Podríamos plantear un modelo de regresión con otro criterio, por ejemplo, el de minimizar el valor absoluto.




\end{problem}



\section{Hoja 3}

\begin{problem}[5]



\solution

El punto medio de un segmento es el límite de una sucesión de puntos dentro del segmento. Vamos a escribirlo formalmente.


Sean $x,y\in S$ y queremos ver si \[λx + (1-λ)y \in S\; ∀λ\in(0,1)\]

Una manera de formalizar esto es escribir $λ$ como:

\[λ = \sum_{i=1}^{∞} \frac{i_1}{2^i}\;\; c_i\in {0,1}\]

Y llamamos $λ_k = \sum^k$. De esta manera,

\[ λ_1x + (1-λ_1)y = \left\{ \begin{array}{cc} \frac{x+y}{2}\in S & λ_1 = \rfrac{1}{2}\\ y\in S & λ_1 = 0 \end{array} \right. \]

Por inducción, \[λ_kx + (1-λ_k) y \in S \implies λ_{k+1}x + (1-λ_{k+1})y \in S\]

Se deja como ejercicio del ejercicio escribir por inducción esta prueba.

\end{problem}


\begin{problem}[6]

\solution

Si $x\in \gor{S}$ e $y\in\text{int}(S)$, entonces $λx + (1-λ)y \in \text{int}(S)$ (es un lema que ya hemos visto).

Utilizando este lema, Tenemos que provar que $S$ convexo $\implies \gor{S}$ convexo.

\[x,y \in \gor{S}, λ\in(0,1) \implies ∃\{x_k\},\{y_k\} \tlq x_k \to x \wedge y_k \to y\]

Por otro lado, tenemos \[λx + (1-λ)y \in \gor{S}\; ¿λ∈(0,1)?\].

Vamos a tomar la combinación convexa con $λ$ y con las sucesiones, es decir:
\[λx_k + (1-λ)y_k \to λx + (1-λ)y \in \gor{S}\]

\end{problem}

\begin{problem}[7]
Tenemos un conjunto $S$ convexo con interior no vacío. 

Probar:

\ppart $\gor{S}=\text{int}(\gor{S})$.

\solution

\spart Una de las 2 inclusiones es obvia y la otra no. Vamos a ver la complicada:

$\gor{S} \subset \text{int}(\gor{S})$:

Utilizando un argumento parecido al anterior, utilizamos $λx_0 + (1-λ)x \in\text{int}({S})$ y haciendo tender $λ\to 0$ obtenemos $x\in \text{int}(\gor{S})$

\spart $\text{int}(\gor{S}) = \text{int}({S})$

Si es vacío, los 2 son vacíos. En caso de no ser vacío, tenemos que probar que:

\[x\in\text{int}(\gor{S}) \implies x \in \text{int}(S)\]

Consideramos el segmento que va de $y$ a $x$ y prolongamos un poco el segmento, hasta un punto $z$ que esté en el cierre, en $\text{int}(\gor{S})$. De esta manera, $x$ queda en un segmento que une un punto del interior ($y$) con uno del cierre $(z)$, con lo que $x$ está en el interior.
\end{problem}

\begin{problem}[8]

\ppart
El cierre convexo de un cerrado no siempre es cerrado.

\ppart 
Si es compacto, entonces es compacto.
\solution

\[S_1 = \{(x_1,x_2) \in \real^2: x_1 \geq 0 \; x_2 = 0\}\]
\[S_2 = \{(x_1,x_2) \in \real^2: x_1 = 0\; 0\leq x_2\leq 1\}\]

\begin{figure}[hbtp]
\centering
\begin{tikzpicture}
\draw[->] (-1,0) -- (4,0);
\draw[->] (0,1) -- (0,4);
\draw[color=red] (0,1) -- (0,0) -- (4,0);
\draw[--] (0,1) -- (4,1);
\end{tikzpicture}
\caption{Representación gráfica de la unión de $S_1$ y $S_2$}
\end{figure}

\spart

\[\conv{S} = \{x = \sum^{n+1}λ_ix_i\;λ_i \geq 0, \sum λ_i = 1\; x_i\in S\}\]

Tomando \[k = \{(λ_1,λ_2,...,λ_{n+1},x_1,...,x_{n+1}) \;λ_i \geq 0, \sum λ_i = 1\; x_i\in S\}\]

Y construimos: $\appl{f}{k}{\conv{S}}$ tal que $f(λ_1,λ_2,...,λ_{n+1},x_1,...,x_{n+1}) = \sum^{n+1}λ_ix_i$ y esta función es continua.

Tenemos $k$ compacto (por ser $S$ compacto) y que  $f(k) =\conv{S}$, siendo $f$ continua. Entonces, $\conv{S}$ es compacto (por ser imagen continua de un compacto).

\end{problem}

\begin{problem}[9]


\solution

\[S_1 = \{ (x_1,x_2) \in \real^2\; x_2\geq \rfrac{1}{x_1} \;x_1 > 0\}\]
\[S_2 = \{(x_1,0) \in \real^2 x_1\in\real\}\]

Vamos a construir $S_1 + S_2$


\begin{figure}[hbtp]
\centering
\begin{tikzpicture}
\draw[->] (-1,0) -- (4,0);
\draw[->] (0,1) -- (0,4);
\end{tikzpicture}
\caption{Representación gráfica de la suma de $S_1$ y $S_2$}
\end{figure}


\[S_1 + S_2  = \{(x_1,x_2)\tlq x_2>0\}\]

\[(0,0) = \lim (\underbrace{x_k}_{\in S_1} + \underbrace{y_k}_{S_2})\]

Para que sea la suma sea cerrada, necesitaríamos $\{y_k\},\{x_k\} \to (0,0)$ y en este caso no es así. Vamos a ver que con que uno de los 2 conjuntos sea acotado, la suma ya es acotada.

\begin{prop}

$S_1,S_2$ convexos y cerrados, con $S_1$ acotado.

Tomamos $x_k + y_k \in S_1 + S_2$ y tenemos $\lim(x_k + y_k) = z$.

Entonces,
\[z\in S_1 + S_2\]
\end{prop}
\begin{proof}
Existe una subsucesión convergente $\{x_{k_i}\}$de $\{x_k\}$, por ser $S_1$ compacto.
Sea $\gor{x} = \lim x_{k_i}$, con $\gor{x}\in S_1$.

La correspondiente subsucesión de $y_k$ también converge (ya que sino la suma no puede converger)

\end{proof}

\end{problem}

\begin{problem}[10]

$S$ convexo es la intersección de todos los subespacios cerrados que lo contienen.

Formalmente:

\[S = \bigcap_{\begin{array}{c}S\subset H_i\\H_i\text{ compacto}\end{array}} H_i\]

\solution

$S\in \bigcap$ es trivial. 

\[\left.\begin{array}{c} x\in\bigcap H_i\\x\not\in S \end{array}\right\} \implies \exists p≠0, α\in\real p^ty\leq α ∀y\in S p^t x >α\]


\end{problem}
