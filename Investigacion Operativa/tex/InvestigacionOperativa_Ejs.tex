% -*- root: ../InvestigacionOperativa.tex -*-

\section{Hoja 1}

\begin{problem}[1]

Una empresa de reciclaje usa papel y tela desechados para fabricar dos tipos distintos de papel
reciclado. Cada tanda de papel reciclado de clase A requiere 20 kg de tela y 180 kg de papel y
produce un beneficio de 500 euros, mientras que cada tanda de papel reciclado de clase B requiere
10 kg de tela y 150 kg de papel y produce un beneficio de 250 euros. La compañía dispone de 100
kg de tela y 660 kg de papel. ¿Cuántas tandas debe fabricar de cada tipo?

\solution

Eeei

\end{problem}


\begin{problem}[2]

La empresa Animales Salvajes S.A. cría faisanes y perdices para repoblar el bosque y dispone
de sitio para criar 100 pájaros durante la temporada. Criar un faisán cuesta 20 euros y criar una
perdiz cuesta 30 euros. La fundación Vida Animal paga a Animales Salvajes S.A. por los pájaros
de forma que se obtiene un beneficio de 14 euros por cada faisán y 16 euros por cada perdiz. La
empresa dispone de 2400 euros para cubrir costes. ¿Cuántas perdices y cuántos faisanes debe criar?
\solution 

eeeeI

\end{problem}


\begin{problem}[3]

La siguiente tabla da el porcentaje de proteínas, grasas y carbohidratos, para cinco alimentos
básicos, A, B, C, D y E :



Los precios por 100 g de estos alimentos (dados en el mismo orden de la tabla) son 5, 17, 37, 10,
15. Si una persona necesita consumir como mínimo 75 gramos de proteínas, 90 de grasas y 300 de
hidratos de carbono, plantea el problema de minimización para calcular la dieta alimenticia de
mínimo coste.
\solution


\end{problem}