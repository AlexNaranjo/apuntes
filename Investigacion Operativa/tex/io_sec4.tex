\section{Máquinas de vectores soporte}

\paragraph{Introducción}

Para más información sobre la clasificación, consultar el último capítulo de \citep{ApuntesEstII}.
%

Disponemos de una muestra de datos bien clasificados (\textit{training data}):

\[
(x_1,y_1),\ldots, (x_n,y_n)
\]
donde $x_i\in \mathbb{R}^d$ son las variables observadas e $y_i\in\{-1,1\}$ es la etiqueta que representa la clase a la que pertenecen las observaciones.

\

Se observa ahora un nuevo vector $x$ independiente de los anteriores.

El objetivo es determinar a qué clase pertenece la observación $x$.

\

La regla óptima (regla Bayes) consiste en asignar a $x$ el valor $y=1$ si y solo si
\[
\mathbb{P}(y=1| x) > \mathbb{P}(y=-1|x)
\]
No es aplicable en la práctica.


\paragraph{Datos separables}

