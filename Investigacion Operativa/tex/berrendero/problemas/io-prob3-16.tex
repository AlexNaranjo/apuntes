
\documentclass[11pt,a4paper,twoside]{article}


%-------------------------------------------------------------

\usepackage[utf8]{inputenc}
\usepackage{graphicx,hyperref,amsmath,natbib,bm,url,microtype}
\usepackage{paralist}
\usepackage[spanish]{babel}
\usepackage[a4paper,text={16.5cm,25.2cm},centering]{geometry}
\usepackage[compact,small]{titlesec}
\setlength{\parskip}{1.2ex}
\setlength{\parindent}{0em}
\clubpenalty = 10000
\widowpenalty = 10000
\usepackage{kpfonts}
\usepackage[T1]{fontenc}
\pagestyle{empty}

\newcounter{problem} \setcounter{problem}{1}
\newcommand{\ex}{\noindent {\sf \bf \theproblem}\addtocounter{problem}{1}.\ }

%-------------------------------------------------------------------
\begin{document}
%-------------------------------------------------------------------
% Encabezamiento



%-----------------------------------
\hrule
\bigskip
\textbf{Investigación Operativa   \hfill Curso 2015/2016\\
Grado en Matemáticas}
\bigskip
\hrule
%--------------------------------

\


\begin{center}  {\bf \large
Relaci\'{o}n 3 de problemas}
\end{center}

\



\ex Resuelve el siguiente problema de optimización lineal utilizando el algoritmo del simplex:

\begin{center}
\begin{tabular}{ll}
Minimizar & $x_2 - 3x_3 + 2x_5$ \\
sujeto a & \\
& $x_1 + 3x_2 - x_3 + 2x_5= 7$\\
& $-2x_2 + 4x_3 + x_4 = 12$\\
& $-4x_2 + 3x_3 + 8x_5 + x_6 = 10$\\
&$x_i\geq 0,\ i=1,\ldots,6$
\end{tabular}
\end{center}

\ 

\ex Una f\'abrica usa tres m\'aquinas para producir dos tipos de
productos. La siguiente tabla indica el n\'umero de horas que cada
m\'aquina necesita para producir una unidad de cada producto y el
tiempo total de disponibilidad para cada m\'aquina durante el
per\'{\i}odo de producci\'on. Los directivos de la f\'abrica quieren
maximizar el n\'umero total de productos fabricados, pero quieren
que la cantidad de producto 1 sea al menos un tercio del total de
la producci\'on.

\begin{center}
\begin{tabular}{|cccc|}
\hline { } & \multicolumn{2}{c}{\small Tiempo necesario} & { }\\
M\'aquina&Producto 1&Producto 2&{Tiempo disponible}\\ \hline
Torno&1.1&2.0&1000\\ Lijadora&3.0&4.5&2000\\
Enceradora&2.5&1.3&1500\\ \hline
\end{tabular}
\end{center}


\begin{compactitem}
\item[(a)] Plantea el correspondiente problema de optimizaci\'on.
\item[(b)] Expresa este problema en una forma adecuada para poder
resolverlo aplicando el algoritmo del simplex. A
continuaci\'on, realiza una sola iteraci\'on de este algoritmo.
\end{compactitem}

\


\ex  Consideremos el siguiente problema de optimizaci\'on
lineal:

\begin{center}
\begin{tabular}{ll}
Minimizar & $c_1x_1+c_2x_2+c_3x_3$ \\
sujeto a & \\
& $x_1 - x_2 + 5x_3\leq 10$\\
& $2x_1 - x_2 + 3x_3 \leq 40$\\
&$x_1\geq 0,\ x_2\geq 0,\ x_3\geq 0$
\end{tabular}
\end{center}

\begin{compactitem}
\item[(a)] Calcula una direcci\'on extrema  y dos puntos extremos
  del conjunto factible.
\item[(b)] Determina unos coeficientes $c_1,c_2,c_3$
de la funci\'on objetivo tales que el problema no tenga
soluci\'on \'optima finita.
\item[(c)] Determina unos coeficientes $c_1,c_2,c_3$
de la funci\'on objetivo tales que el problema tenga
infinitas soluciones \'optimas.
\end{compactitem}

\

\ex Dado un problema de optimización lineal $\min\ c^\top x$ sujeto a $Ax=b$, $x\geq 0$, ¿es posible conseguir un problema en el que no exista soluci\'on
\'optima finita cambiando \'unicamente el vector $b$? Responde a la misma pregunta para el vector $c$.

\

\ex Consideremos el problema de optimizaci\'on
\begin{equation*}
{\rm min}\ z=c^\top x\ \ \hbox{sujeto a}\ \ x\in S=\{x\in\mathbb{R}^n:\, Ax= b,\
x\geq 0\}.
\end{equation*}
donde $b\in\mathbb{R}^m$, $c\in\mathbb{R}^n$ y $A$ es una matriz $m\times n$ de
rango $m$.
\begin{compactitem}
\item[(a)] Supongamos que $S\neq\emptyset$ y que para un cierto vector no
nulo $d\geq 0$, $d\in\mathbb{R}^n$, se verifica $Ad=0$. ?`Puede asegurarse
entonces que existe un vector $c\neq 0$ tal que el problema
 tiene soluci\'on \'optima no acotada?
\item[(b)] ?`Puede ocurrir que una soluci\'on \'optima tenga m\'as de $m$
componentes estrictamente positivas?
\item[(c)] Si $\bar x\in S$ tiene exactamente $m$ componentes
estrictamente positivas, ?`puede asegurarse que $\bar x$ es un punto
extremo de $S$?
\end{compactitem}


\

\ex Consideremos el problema:
maximizar $x_1+3x_2+x_3$, sujeto a $Ax\leq b$, $x\geq 0$,
siendo $A$ una matriz $2\times 3$, $b\in\mathbb{R}^2$ ($b\geq 0$) y
$x=(x_1,x_2,x_3)^\top$.
Se resuelve este problema pas\'andolo a forma est\'andar
(para lo cual se a\~naden dos variables de holgura $x_4$ y $x_5$) y
aplicando el algoritmo del simplex. La tabla final del algoritmo ha
sido:

\begin{center}
\begin{tabular}{c||c|c|c|c|c}
$c_j$&-1&-3&-1&0&0\\
\hline
Variables&$x_1$&$x_2$&$x_3$&$x_4$&$x_5$\\
\hline
$x_1=12$&1&4&3&1&0\\
$x_5=16$&0&6&2&1&1\\
\end{tabular}
\end{center}

\begin{compactitem}
\item[(a)] Calcula la matriz $A$ y el vector $b$.
\item[(b)] Supongamos que el vector de costes $c=(1,3,1,0,0)^\top$ se reemplaza
por $\hat c=c+\lambda \gamma$, siendo $\lambda$ un n\'umero no negativo
y $\gamma=(1,1,1,0,0)^\top$. ?`Para qu\'e valores de $\lambda$ sigue siendo
soluci\'on \'optima $x=(12,0,0,0,16)^\top$?
\end{compactitem}


\

\ex Considera la siguiente tabla del simplex para un
problema  lineal de minimizaci\'on ($a,b,c,d$ y $e$
denotan aqu\'{\i} par\'ametros reales):

\begin{center}
\begin{tabular}{c||c|c|c|c|c}
$c_j$&0&b&e&0&0\\
\hline
Variables&$x_1$&$x_2$&$x_3$&$x_4$&$x_5$\\
\hline
$x_1=a$&1&c&3&0&0\\
$x_4=2$&0&d&2&1&0\\
$x_5=4$&0&-1&1&0&1\\
\end{tabular}
\end{center}

Determina condiciones sobre los par\'ametros $a,b,c,d$ y $e$ de manera que
\begin{compactitem}
\item[(a)] La tabla corresponda a una soluci\'on \'optima.
\item[(b)] La tabla indique la situaci\'on de \textit{soluci\'on \'optima no acotada}.
\item[(c)] La tabla corresponda a una soluci\'on \'optima y la regi\'on
factible  no sea acotada.
\end{compactitem}

\

\ex Demuestra que, en el algoritmo del simplex, una variable que sale de la base en una determinada iteración no puede volver a entrar en la base en la iteración inmediatamente posterior.


\

\ex Estudia c\'omo puede detectarse mediante el algoritmo del simplex
la existencia de ``alternativa \'optima'', es decir, la existencia
de m\'as de una soluci\'on \'optima.




\


\ex En una iteraci\'on del algoritmo del simplex aplicado a un
problema lineal de minimizaci\'on se obtiene la
tabla siguiente:
\[
\begin{array}{c|cccc}
 & -2 & 3 & 0 & 0 \\ \hline
 & x_1 & x_2 & x_3 & x_4 \\ \hline
x_3=4 & c & 0 & 1 & 1/5 \\
x_1=a & d & e & 0 & 2 \\ \hline
 & f & -1 & g & h
\end{array}
\]
El valor del objetivo del problema en la soluci\'on
b\'asica factible a la que corresponde la tabla anterior es
-6. Las variables $x_3$ y $x_4$ son de holgura y formaban
la base inicial.\\
(a) Calcula el valor de las siete inc\'ognitas $a$, $c$,
$d$, $e$, $f$, $g$ y $h$.\\
(b) ?`Corresponde la tabla a una soluci\'on b\'asica factible
\'optima? %\\
%(c) Calcula la soluci\'on del problema dual. ?`Hay alguna
%restricci\'on del problema primal que sea activa en el
%\'optimo?


\

\ex Se consideran los dos problemas siguientes:

\begin{eqnarray*}
 \min z &=& c^tx \\
 Ax &=& b \\
x & \geq & 0
\end{eqnarray*} 
y
\begin{eqnarray*}
 \min z &=& c^tx \\
 Ax &=& b+\theta d \\
  x & \geq & 0,
\end{eqnarray*}
donde $A$ es una matriz $m\times n$, $x\in\mathbb{R}^n$, $b\in\mathbb{R}^m$,
$d\in\mathbb{R}^m$ y $\theta\geq 0$. Dada una soluci\'on b\'asica \'optima
para el primer problema, $\overline{x}$, correspondiente a una base $B$,
 ?`para qu\'e valores de $\theta$ seguir\'a siendo $B$
la base correspondiente a una soluci\'on b\'asica \'optima para el segundo problema?



\

\ex Al final de la primera fase de la resoluci\'on de un algoritmo
del simplex se ha obtenido la siguiente tabla:



\begin{center}
\begin{tabular}{c||c|c|c|c|c|c|c|c|c}
Variables&$x_1$&$x_2$&$x_3$&$x_4$&$x_5$&$x_6$&$x_1^a$&$x_2^a$&$x_3^a$\\
\hline
$x_3=0$&0&0&1&3&0&0&-1/3&0&2/3\\
$x_2=5$&0&1&0&16&1/2&-2&-7/3&3&8/3\\
$x_1=7/3$&1&0&0&14/3&1/3&-1/3&-5/9&1&7/9\\
\end{tabular}
\end{center}



Calcula la matriz $A$ y el vector $b$ del sistema de restricciones
inicial (las variables $x_i^a$ son artificiales).



%------------------------------------------------------------------------
\end{document}
%-------------------------------------------------------------------------

