\documentclass{beamer}

%\usepackage{beamerthemetreebars}
\usepackage{graphicx}
%\usepackage{beamerthemesplit}
%\beamertemplateshadingbackground{red!10}{blue!10}

%----------------------------------------------------------------------
% Para que aparezcan varias transparencias en la misma p\'{a}gina
\usepackage{pgfpages}
%\pgfpagesuselayout{4 on 1}[a4paper,border shrink=5mm,landscape]
%----------------------------------------------------------------------
% Suprime los s\'{\i}mbolos de navegaci\'{o}n
\setbeamertemplate{navigation symbols}{}
%---------------------------------------------------------------------

\usepackage[utf8]{inputenc}
%----------------------------------------------------------------
\newcommand{\ep}{\epsilon}
\newcommand{\real}{{\rm I\kern-.17em R}}
\newcommand{\pro}{\mbox{P}}
%-----------------------------------------------------------------

\title[Estad\'{\i}stica: Tema 5]{Tema 5\\
Dualidad y condiciones de Karush-Kuhn-Tucker}
\author[Berrendero]
{Jos\'{e} R. Berrendero}
\date{}
\institute{Departamento de Matem\'{a}ticas\\
 Universidad Aut\'{o}noma de Madrid}

%------------------------------------------------------------
\begin{document}
%-------------------------------------------------------------------



\frame{\titlepage}
%----------------------------------------------------------------------
\begin{frame}[plain]
\frametitle{Contenidos del tema 5}

\begin{itemize}
 
\item Condiciones de Karush-Kuhn-Tucker (KKT).
\begin{itemize}
\item Problemas con restricciones de desigualdad.
\item Problemas con restricciones de desigualdad y de igualdad.
\end{itemize}
	
  \item El problema dual.
  
  \item Dualidad en optimización lineal. Algoritmo simplex dual.

 \item Relaciones entre el problema dual y las condiciones de KKT.
\end{itemize}


\end{frame}


%----------------------------------------------
\begin{frame}
\frametitle{Problemas que vamos a analizar}


En este tema consideraremos \textbf{problemas generales} de optimización de la forma

\begin{center}
\begin{tabular}{lll}
minimizar & $f(x)$ \\
s.a. & $f_i(x)\leq 0,$  &  $i=1,\ldots,m$ \\
	 & $h_i(x) = 0,$  &  $i=1,\ldots,p.$
\end{tabular}
\end{center}

También consideraremos  \textbf{problemas convexos} de la forma

\begin{center}
\begin{tabular}{lll}
minimizar & $f(x)$ & \\
s.a. & $f_i(x)\leq 0,$  &  $i=1,\ldots,m$ \\
	 & $a^\top_i x = b_i,$  &  $i=1,\ldots,p$,
\end{tabular}
\end{center}
donde las funciones $f,f_1,\ldots, f_n$ son convexas. 



\

En general, denotamos por $D$ la intersección de los dominios de todas las funciones del problema y suponemos que $D$ es un abierto que incluye los puntos que satisfacen las restricciones.

\end{frame}
%-----------------------------------------------
\begin{frame}
\frametitle{Condiciones de Karush-Kuhn-Tucker (solo desigualdades)}

Consideramos el problema  (P):

\begin{center}
\begin{tabular}{lll}
minimizar & $f(x)$ &  \\
s.a. & $f_i(x)\leq 0$ & $i=1,\ldots,m $,
\end{tabular}
\end{center}
donde todas las funciones son diferenciables.

\


Se dice que el punto $\bar{x}\in\mathbb{R}^n$, junto con el vector de multiplicadores $\bar{u}\in\mathbb{R}^m$, verifica las condiciones KKT para (P) si

\begin{align*}
\nabla f(\bar{x}) + \sum_{i=1}^m \bar{u}_i \nabla f_i(\bar{x})  & = 0 \\
f_i(\bar{x}) &\leq 0, & i = 1,\ldots,m \\
\bar{u}_i    & \geq 0, & i = 1,\ldots,m\\
\bar{u}_i f_i(\bar{x}) &  = 0, & i = 1,\ldots,m.
\end{align*}













\end{frame}
%-----------------------------------------------
\begin{frame}
\frametitle{Condiciones KKT necesarias para mínimo local de (P)}



Cualquier mínimo local de (P), que verifique una condición adicional debe satisfacer las condiciones   KKT.

\

\textbf{Cualificación de las restricciones}: Dado un punto factible $\bar{x}$, sea $I=\{i:\, f_i(\bar{x})=0\}$ el conjunto de restricciones activas en $\bar{x}$.

\

Se dice que $\bar{x}$ verifica la cualificación de las restricciones si  los vectores $\nabla f_i(\bar{x})$, $i\in I$, son linealmente independientes.


\

\textbf{Teorema:} Sea $\bar{x}$ un mínimo local de (P) que verifica la cualificación de las restricciones anterior. Entonces, existe un vector $\bar{u}\geq 0$ tal que $\bar{x}$ y $\bar{u}$ verifican las condiciones KKT.

\end{frame}
%-----------------------------------------------
\begin{frame}
\frametitle{Condiciones KKT necesarias para mínimo local de (P)}



\textbf{Demostración:} 

\begin{itemize}
\item Si $\bar{x}$ es un mínimo local de (P), no existe ninguna dirección $d\in\mathbb{R}^n$ tal que $\nabla f(\bar{x})^\top d<0$, $\nabla f_i(\bar{x})^\top d<0$ para $i\in I$.

\

\item Por el teorema de Gordan, existen $u_0\geq 0$, $u_i\geq 0$, $i\in I$, no todos nulos, tales que $u_0\nabla f(\bar{x}) + \sum_{i\in I}u_i\nabla f(\bar{x})=0$.

\

\item Por la cualificación de las restricciones, $u_0 > 0$.

\

\item Definimos  $\bar{u}_i=u_i/u_0$ si $i\in I$ y $\bar{u}_i=0$ si $i\notin I$.

\

\item Es fácil comprobar que $\bar{x}$ y $\bar{u}$ verifican las condiciones KKT. 
\end{itemize}



\end{frame}

%-----------------------------------------------
\begin{frame}
\frametitle{Observaciones}

Para encontrar el mínimo global del problema (P):

\begin{itemize}
\item Hallar los puntos que verifican las condiciones de KKT.

\item Hallar los puntos factibles que no verifican  la cualificación de las restricciones.
\end{itemize}

Si el problema tiene solución, debe encontrarse entre los puntos hallados de alguna de estas dos formas.

\

\textbf{Ejemplo:} Determina los puntos que verifican las condiciones KKT y resuelve el problema:
\begin{center}
\begin{tabular}{ll}
minimizar & $x_1^2+x_2^2+2x_3^2$   \\
s.a. & $2x_1 + 3x_2 + x_3\geq 6$ \\
\end{tabular}
\end{center}



\end{frame}

%-----------------------------------------------
\begin{frame}
\frametitle{Ejemplo}


Considera el siguiente problema de optimización:

\begin{center}
\begin{tabular}{ll}
maximizar & $3x_1+x_2$   \\
s.a. & $x_2 - (1-x_1)^3 \leq 0$ \\
  & $x_1\geq 0$, $x_2\geq 0$
\end{tabular}
\end{center}




\begin{itemize}
\item Resuelve el problema gráficamente
\item Escribe las condiciones KKT
\item ¿Verifican KKT los puntos del interior del conjunto factible?
\item ¿Verifica el máximo global del problema las condiciones KKT?
\item ¿Existen puntos que verifican las condiciones KKT y que no son máximos locales?
\end{itemize}




\end{frame}
%-----------------------------------------------
\begin{frame}
\frametitle{Ejemplo}

\begin{center}
\includegraphics[scale=0.5]{EjemploKKT}
\end{center}

\end{frame}
%-----------------------------------------------
\begin{frame}
\frametitle{Condiciones de Karush-Kuhn-Tucker (también igualdades)}

{\scriptsize
Consideramos el problema  (PI):

\begin{center}
\begin{tabular}{lll}
minimizar & $f(x)$ \\
s.a. & $f_i(x)\leq 0,$  &  $i=1,\ldots,m$ \\
	 & $h_i(x) = 0,$  &  $i=1,\ldots,p.$
\end{tabular}
\end{center}
donde todas las funciones son diferenciables.

\


Se dice que el punto $\bar{x}\in\mathbb{R}^n$, junto con los vectores de multiplicadores $\bar{u}\in\mathbb{R}^m$ y $\bar{v}\in\mathbb{R}^p$, verifica las condiciones KKT para (PI) si

\begin{align*}
\nabla f(\bar{x}) + \sum_{i=1}^m \bar{u}_i \nabla f_i(\bar{x}) + \sum_{i=1}^p \bar{v}_i\nabla h_i(\bar{x}) & = 0 \\
f_i(\bar{x}) &\leq 0, & i = 1,\ldots,m \\
h_i(\bar{x}) &= 0, & i = 1,\ldots,p \\
\bar{u}_i    & \geq 0, & i = 1,\ldots,m\\
\bar{u}_i f_i(\bar{x}) &  = 0, & i = 1,\ldots,m\\
\end{align*}

}










\end{frame}
%-----------------------------------------------
\begin{frame}
\frametitle{Condiciones KKT necesarias para mínimo local de (PI)}



Cualquier mínimo local de (PI), que verifique una condición adicional debe satisfacer las condiciones   KKT.

\

\textbf{Cualificación de las restricciones}: Dado un punto factible $\bar{x}$, sea $I=\{i:\, f_i(\bar{x})=0\}$ el conjunto de restricciones activas en $\bar{x}$.

\

Se dice que $\bar{x}$ verifica la cualificación de las restricciones si  los vectores $\nabla f_i(\bar{x})$, $i\in I$ y $\nabla h_i(\bar{x})$, $i=1,\ldots,p$ son linealmente independientes.


\

\textbf{Teorema:} Sea $\bar{x}$ un mínimo local de (PI) que verifica la cualificación de las restricciones anterior. Entonces, existen vectores $\bar{u}\geq 0$ y $\bar{v}$ tales que $\bar{x}$, $\bar{u}$ y $\bar{v}$ verifican las condiciones KKT.

\end{frame}
%-----------------------------------------------
\begin{frame}
\frametitle{Condiciones KKT necesarias para mínimo local de (PI)}



\textbf{Demostración (esquema):} 

\begin{itemize}
\item Si $\bar{x}$ es un mínimo local de (PI), no existe ninguna dirección $d\in\mathbb{R}^n$ tal que $\nabla f(\bar{x})^\top d<0$; $\nabla f_i(\bar{x})^\top d<0$, para $i\in I$; $\nabla h_i(\bar{x})^\top d = 0$, para $i=1,\ldots,p$.

\

\item Por un teorema de la alternativa, existen $u_0\geq 0$; $u_i\geq 0$, $i\in I$; $v_i$, $i=1,\ldots,p$,  no todos nulos, tales que 
\[
u_0\nabla f(\bar{x}) + \sum_{i\in I}u_i\nabla f(\bar{x})+\sum_{i=1}^p v_i\nabla h_i(\bar{x})=0.
\]

\

\item Por la cualificación de las restricciones, $u_0 > 0$.

\

\item Definimos  $\bar{u}_i=u_i/u_0$, si $i\in I$; $\bar{u}_i=0$, si $i\notin I$ y
$\bar{v}_i=v_i/u_0$, $i=1\ldots,p$.

\

\item Es fácil comprobar que $\bar{x}$, $\bar{u}$ y $\bar{v}$ verifican las condiciones KKT. 
\end{itemize}



\end{frame}

\begin{frame}
\frametitle{La función dual}

Para un problema general (no necesariamente convexo) la \textbf{función lagrangiana} es:
\[
L(x,u,v) = f(x) + \sum_{i=1}^m u_i f_i(x) + \sum_{i=1}^p v_i h_i(x).
\]


\


La \textbf{función dual} es 
\[
g(u,v) = \inf_{x\in D} L(x,u,v).
\] 



\

Si $u\geq 0$, calcular $g(u,v)$ puede interpretarse como resolver una versión menos exigente del problema original (PI).


\

La función dual es cóncava (aunque (PI) no sea convexo).


\end{frame}
%----------------------------------------------
\begin{frame}
\frametitle{El problema dual}

La función dual (es decir el valor óptimo de la versión menos exigente del problema) es una cota inferior del valor óptimo del problema original:

\

{\bf Teorema:} Sea $\bar{p}$ el valor óptimo de un problema de optimización cuya función dual es $g(u,v)$. Para todo $u\geq 0$ y para todo $v$, $g(u,v)\leq \bar{p}$.

\

Resulta natural preguntarse cuál es la mejor de las cotas que pueden obtenerse de esta forma. Esto lleva al \textbf{problema dual}.






\end{frame}
%----------------------------------------------
\begin{frame}
\frametitle{El problema dual}


\begin{center}
\begin{tabular}{ll}
maximizar & $g(u,v)$ \\
s.a. & $u\geq 0$.    
\end{tabular}
\end{center}

\


El problema dual es siempre convexo, independientemente de que lo sea o no el problema primal al que corresponde.

\

Un punto $(u,v)$ con $u\geq 0$ y $g(u,v)>-\infty$ (es decir, en el dominio de la función dual) se llama \textbf{solución factible dual}.

\

Llamamos \textbf{solución factible dual óptima} al punto $(\bar{u},\bar{v})$ que resuelve el problema dual.


\end{frame}
%----------------------------------------------
\begin{frame}
\frametitle{Dualidad débil}

Si $\bar{p}$ y $\bar{d}$ son los valores óptimos de los problemas primal (minimización) y dual (maximización) respectivamente, las definiciones implican la \textbf{desigualdad de dualidad débil}:
\[
\bar{d} \leq \bar{p}.
\]

\

¿Qué ocurre si $\bar{p}=-\infty$? ¿Qué ocurre si $\bar{d}=\infty$?

\

Si el primal es no acotado, entonces el dual es no factible. Si el dual es no acotado, entonces el primal es no factible.

\

La brecha de dualidad es la diferencia $\bar{p}-\bar{d}$.




\end{frame}
%----------------------------------------------
\begin{frame}
\frametitle{El dual de un problema lineal en forma estándar}

\begin{center}
\begin{tabular}{ll}
minimizar & $c^\top x$  \\
s.a. & $Ax=b$,     \\
	 & $x\geq 0$.  
\end{tabular}
\end{center}

\textbf{Función lagrangiana}:
\[
L(x,u,v) = -b^\top v + (c+A^\top v -u)^\top x.
\]

\

La \textbf{función dual} es $g(u,v)=-b^\top v$, si $c+A^\top v -u=0$, y $g(u,v)=-\infty$, en caso contrario.


\

\textbf{Problema dual}:

\begin{center}
\begin{tabular}{ll}
maximizar & $b^\top v$  \\
s.a. & $A^\top v \leq c$       
\end{tabular}
\end{center}

\end{frame}
%----------------------------------------------
\begin{frame}
\frametitle{El dual de un problema lineal en forma canónica}

\textbf{Problema primal}:

\begin{center}
\begin{tabular}{ll}
maximizar & $c^\top x$  \\
s.a. & $Ax\leq b$,     \\
	 & $x\geq 0$.  
\end{tabular}
\end{center}

\textbf{Problema dual}:

\begin{center}
\begin{tabular}{ll}
minimizar & $b^\top v$  \\
s.a. & $A^\top v \geq c$,     \\
	 & $v\geq 0$.  
\end{tabular}
\end{center}


\textbf{Ejercicio:} Escribe el dual de
\begin{center}
\begin{tabular}{ll}
minimizar & $c^\top x$  \\
s.a. & $Ax\geq b$,     \\
	 & $x\geq 0$.  
\end{tabular}
\end{center}



\textbf{Observaciones:}

\begin{itemize}
\item El dual del dual es el primal.
\item Los duales de problemas primales equivalentes son también equivalentes entre sí.
\end{itemize}


\end{frame}
%----------------------------------------------
\begin{frame}
\frametitle{Ejemplos}

Escribe el problema dual de los siguientes problemas de optimización lineal:

\begin{center}
\begin{tabular}{lr}
minimizar & $6x_1 + 8x_2$ \\
s.a. & $3x_1+x_2 \geq 4$    \\
	 & $5x_1 + 2x_2\geq 7$  \\
	 & $x_1\geq 0,\ x_2\geq 0$
\end{tabular}
\end{center}

\

\begin{center}
\begin{tabular}{lr}
minimizar & $8x_1 + 3x_2 - 2x_3$ \\
s.a. & $x_1-6x_2+x_3 \geq 2$    \\
	 & $5x_1 + 7x_2 - 2x_3= -4$  \\
	 & $x_1\geq 0,\ x_2\geq 0,\ x_3\geq 0$
\end{tabular}
\end{center}


\end{frame}
%----------------------------------------------
\begin{frame}
\frametitle{Interpretación económica del problema dual}

 Un pastelero dispone de 150 kg de harina, 22 kg de azúcar y 27.5
 kg de mantequilla para elaborar dos tipos de pasteles ($A$ y
 $B$). Cada caja de pasteles de tipo $A$ requiere 3 kg de harina,
 1 kg de azúcar y 1 kg de mantequilla, y su venta le reporta un
 beneficio de 20 euros. Cada caja de pasteles
  de tipo $B$ requiere 6 kg de harina,
 0.5 kg de azúcar y 1 kg de mantequilla y su venta le reporta un
 beneficio de 30 euros.
 \begin{itemize}
 \item[(a)] ?`Cu\'antas cajas de cada
 tipo debe elaborar el pastelero de manera que se maximicen sus
 ganancias? 
 \item [(b)] Supongamos que la
 cantidad de harina disponible aumenta en un kg. ?`Cu\'anto aumenta
 el beneficio del pastelero? Contesta a la misma cuesti\'on para un
 aumento de un kg en la cantidad de az\'ucar y mantequilla.
 \end{itemize}


\end{frame}
%-------------------------------------------------------------
\begin{frame}
\frametitle{Interpretación económica del problema dual}

\begin{center}
\begin{tabular}{lr}
maximizar & $20x_1 + 30x_2$ \\
s.a. & $3x_1+6x_2 \leq 150$    \\
	 & $x_1 + x_2/2 \leq 22$  \\
	 & $x_1+x_2\leq 27.5$\\
	 & $x_1\geq 0,\ x_2\geq 0$
\end{tabular}
\end{center}

\

\begin{itemize}
\item La solución es $\bar{x}_1=5$ y $\bar{x}_2=22.5$.
\item El beneficio óptimo es $\bar{z}=775$.
\end{itemize}


\

\begin{itemize}
\item Si $\Delta H=1$, tenemos $\Delta \bar{z}=10/3$.
\item Si $\Delta A=1$, tenemos $\Delta \bar{z}= 0$.
\item Si $\Delta M=1$, tenemos $\Delta \bar{z}= 10$.
\end{itemize}

\end{frame}
%----------------------------------------------
\begin{frame}
\frametitle{Interpretación económica del problema dual}

Un comprador está interesado en adquirir el negocio del pastelero (es decir, los recursos de azúcar, harina y mantequilla de los que dispone) y quiere hacerle una oferta de precios por kg de azúcar, harina y mantequilla. Por supuesto el pastelero solo aceptará la oferta del comprador si lo que gana con la venta es al menos lo mismo que ganaría fabricando pasteles. 

\

¿Qué problema debe resolver el comprador para hacerle la oferta más ventajosa al pastelero?

\

Resuelve el problema del comprador. 

\

¿Qué relaciones existen entre las soluciones de los problemas del pastelero y del comprador? 





\end{frame}
%-------------------------------------------------------------
\begin{frame}
\frametitle{Interpretación económica del problema dual}

Las variables $u_1$, $u_2$ y $u_3$ representan los precios de los tres recursos (harina, azúcar y mantequilla) respectivamente.

\

El problema del comprador es:

\begin{center}
\begin{tabular}{lr}
minimizar & $150u_1 + 22u_2 + 27.5u_3$ \\
s.a. & $3u_1+u_2 + u_3 \geq 20$    \\
	 & $6u_1 + u_2/2 + u_3 \geq 30$  \\
	 & $u_1\geq 0,\ u_2\geq 0,\ u_3\geq 0$
\end{tabular}
\end{center}

\

\begin{itemize}
\item La solución es $\bar{u}_1=10/3$, $\bar{u}_2=0$ y $\bar{u}_3=10$.
\item El beneficio óptimo es $\bar{z}=775$.
\end{itemize}


\

\begin{itemize}
\item Los valores óptimos del objetivo son los mismos para ambos problemas (la brecha de dualidad es cero).
\item Los precios óptimos coinciden con los beneficios marginales.
\end{itemize}

\end{frame}
%----------------------------------------------
\begin{frame}
\frametitle{Dualidad fuerte}

Se dice que hay \textbf{dualidad fuerte} cuando $\bar{p}=\bar{d}$. 

\

En problemas lineales factibles siempre hay dualidad fuerte.


\

Puede ocurrir $\bar{d}=-\infty$ y $\bar{p}=\infty$.

\

Escribe el dual de 

\begin{center}
\begin{tabular}{lr}
maximizar & $-x_1 + x_2$ \\
s.a. & $x_1 \leq -1$    \\
	 & $-x_1 - x_2 \leq -1$  \\
	 & $x_1\geq 0,\ x_2\geq 0$
\end{tabular}
\end{center}

\end{frame}
%----------------------------------------------
\begin{frame}
\frametitle{Dualidad fuerte para problemas lineales}

\textbf{Teorema:} Consideremos un problema lineal en forma estándar (P) y el correspondiente problema dual (D). Si (P) tiene solución factible óptima finita, también la tiene (D) y los correspondientes valores óptimos son iguales. 

\

{\scriptsize

\textbf{Demostración:} 

\begin{itemize}
\item Sea $\bar{x}={B^{-1}b\choose 0}$ la solución factible óptima de (P).

\

\item Sea $\bar{u}^\top = c_B^\top B^{-1}$. Se cumple
\[
\bar{u}^\top A \leq c^\top \Leftrightarrow c_B^\top B^{-1}N\leq c_N^\top \Leftrightarrow z_j\leq c_j,\ \ j\in N
\]
Luego $\bar{u}$ es factible para (D).

\

\item Además $\bar{u}^\top b=c_B^\top B^{-1}b=c^\top \bar{x}$.
\end{itemize}
}

 \end{frame}
%----------------------------------------------
\begin{frame}
\frametitle{Dualidad fuerte para problemas lineales}

\begin{itemize}



\item La condición de factibilidad de $\bar{u}$ equivale a la de optimalidad de $\bar{x}$.


\item En el ejemplo, el pastelero ganará lo mismo tanto si fabrica pasteles como si vende el negocio. Si su problema no fuese acotado, no aceptaría ninguna oferta y como consecuencia, el problema del comprador no sería factible.

\item Resolver el primal y el dual es equivalente a partir de la relación $\bar{u}^\top =c_B^\top B^{-1}$.


\item Sensibilidad de soluciones:
$
\bar{z} = \bar{u}^\top b = \sum_{i=1}^m \bar{u}_i b_i.
$
Por lo tanto
\[
\frac{\partial \bar{z}}{\partial b_i} = \bar{u}_i.
\]
La solución del dual puede interpretarse en términos del incremento del valor óptimo del primal ante pequeñas variaciones de los recursos. 



\end{itemize}

\end{frame}
%----------------------------------------------
\begin{frame}
\frametitle{Algoritmo simplex-dual}


\textbf{Punto de partida:}
\begin{itemize}
\item Un punto factible para el dual, pero no para el primal.
\item $z_j-c_j\leq 0$, para todo $j$, pero $\bar{b}=B^{-1}b\ngeqslant 0$.
\end{itemize}

\

\textbf{Criterio de salida:} Sale de la base la variable $x_r$ tal que
\[
\bar{b}_r = \min\{ \bar{b}_i:\, \bar{b}_i < 0\}.
\]

\

\textbf{Criterio de entrada:} Entra en la base la variable $x_k$ tal que 
\[
\frac{z_k-c_k}{y_{rk}} = \min\left\{ \frac{z_j-c_j}{y_{rj}}:\, y_{rj} < 0\right\}.
\]

\

\textbf{Infactibilidad:} Si $y_{rj}\geq 0$ para todo $j$, el problema primal es no factible.


\end{frame}
%----------------------------------------------
\begin{frame}
\frametitle{Ejemplo}

\begin{center}
\begin{tabular}{lr}
minimizar & $3x_1 + 2x_2$ \\
s.a. & $3x_1+x_2 \geq 3$    \\
	 & $4x_1 + 3x_2 \geq 6$  \\
	 & $x_1+x_2\leq 3$\\
	 & $x_1\geq 0,\ x_2\geq 0.$
\end{tabular}
\end{center}

En forma estándar,

\begin{center}
\begin{tabular}{lr}
minimizar & $3x_1 + 2x_2$ \\
s.a. & $-3x_1-x_2 + x_3 = -3$    \\
	 & $-4x_1 - 3x_2 + x_4 = -6$  \\
	 & $x_1+x_2 + x_5 = 3$\\
	 & $x_i\geq 0,\ i=1,\ldots, 5.$
\end{tabular}
\end{center}


\end{frame}
%----------------------------------------------
\begin{frame}
\frametitle{Ejemplo}

\begin{center}
\begin{tabular}{r | rrrrr}
$c$ & 3 & 2 & 0 & 0 & 0 \\ \hline
Variables & $x_1$ & $x_2$ & $x_3$ & $x_4$ & $x_5$ \\ \hline
$x_3 = -3$ & -3 & -1 & 1 & 0 & 0 \\
$x_4=  -6$   & -4 & \textbf{-3} & 0 & 1 & 0  \\
$x_5=3$ & 1 & 1 & 0 & 0 & 1 \\ \hline
$z_j-c_j$ & -3 & -2 & 0 & 0 & 0 
\end{tabular}
\end{center}


\begin{center}
\begin{tabular}{r | rrrrr}
$c$ & 3 & 2 & 0 & 0 & 0 \\ \hline
Variables & $x_1$ & $x_2$ & $x_3$ & $x_4$ & $x_5$ \\ \hline
$x_3 = -1$ & \textbf{-5/3} & 0 & 1 & -1/3 & 0 \\
$x_2=  2$   & 4/3 & 1 & 0 & -1/3 & 0  \\
$x_5=1$ & -1/3 & 0 & 0 & 1/3 & 1 \\ \hline
$z_j-c_j$ & -1/3 & 0 & 0 & -2/3 & 0 
\end{tabular}
\end{center}


\end{frame}
%----------------------------------------------
\begin{frame}
\frametitle{Ejemplo}


\begin{center}
\begin{tabular}{r | rrrrr}
$c$ & 3 & 2 & 0 & 0 & 0 \\ \hline
Variables & $x_1$ & $x_2$ & $x_3$ & $x_4$ & $x_5$ \\ \hline
$x_1 = 3/5$ & 1 & 0 & -3/5 & 1/5 & 0 \\
$x_2=  6/5$   & 0 & 1 & 4/5 & -3/5 & 0  \\
$x_5=6/5$ & 0 & 0 & -1/5 & 2/5 & 1 \\ \hline
$z_j-c_j$ & 0 & 0 & -1/5 & -3/5 & 0 
\end{tabular}
\end{center}

\begin{itemize}
\item La solución del primal es $\bar{x}_1=3/5$, $\bar{x}_2=6/5$ ($\bar{z}=21/5$).
\item La solución del dual es 
\[
\bar{u}^\top = c_B^\top B^{-1} = (3,2,0)\begin{pmatrix}
-3/5 & 1/5 & 0 \\ 
4/5 & -3/5 & 0 \\ 
-1/5 & 2/5 & 1
\end{pmatrix} = (-1/5,-3/5,0).
\]
\end{itemize}



\end{frame}
%----------------------------------------------
\begin{frame}
\frametitle{Algoritmo simplex-dual}

(1) Tras el pivoteo se mantiene la factibilidad dual (optimalidad primal):
\[
\hat{z}_j - \hat{c}_j = (z_j-c_j) - \frac{y_{rj}}{y_{rk}} (z_k-c_k)
\]

\

(2) Tras el pivoteo mejora el objetivo del dual:
\begin{align*}
\sum_{i\in \hat{B}} c_i\bar{b}_i &=  \sum_{i\in B} c_i\left(\bar{b}_i - \frac{\bar{b}_r}{y_{rk}}y_{ik} \right) + c_k\frac{\bar{b}_r}{y_{rk}}\\
\sum_{i\in \hat{B}} c_i\bar{b}_i &= \sum_{i\in B} c_i\bar{b}_i - \frac{\bar{b}_r}{y_{rk}} ( \sum_{i\in B} c_iy_{ik} - c_k)\\
\sum_{i\in \hat{B}} c_i\bar{b}_i &= \sum_{i\in B} c_i\bar{b}_i - \frac{\bar{b}_r}{y_{rk}} ( z_k - c_k) > \sum_{i\in B} c_i\bar{b}_i.
\end{align*}

\end{frame}
%----------------------------------------------
\begin{frame}
\frametitle{Algoritmo simplex-dual}

(3) Si $y_{rj}\geq 0$ para todo $j$, el problema primal no es factible:

\

Si existiera $x\geq 0$ con $Ax=b$ tendríamos:
\[
Ax=b \Leftrightarrow B^{-1}Ax = \bar{b} \Leftrightarrow \sum_{j=1}^n y_j x_j = \bar{b}.
\]

En particular, $\bar{b}_r = \sum_{j=1}^n y_{rj} x_j\geq 0$, lo que no puede ser porque $\bar{b}_r<0$.
\end{frame}
%----------------------------------------------
\begin{frame}
\frametitle{Holgura complementaria en problemas lineales}

Sean $\bar{x}$ y $\bar{u}$ las soluciones de (P) y (D):

(P) max $c^\top x$ s.a. $Ax\leq b$, $x\geq 0$.

(D) min $b^\top u$ s.a. $A^\top u\geq c$, $u\geq 0$.

\

 Sabemos $c^\top \bar{x}=b^\top \bar{u}$ (dualidad fuerte en el caso lineal).
 

\[
c^\top \bar{x} \leq \bar{u}^\top A\bar{x} \leq \bar{u}^\top b = c^\top \bar{x},
\]

\

Por lo tanto, si $a_i$ y $a^j$ denotan las columnas y las filas de $A$,
\begin{itemize}
\item $\bar{u}^\top(A\bar{x}-b)=0 \Leftrightarrow \bar{u}_j(a^j\bar{x}-b_j)=0$, $j=1,\ldots,m$
\item $(c^\top-\bar{u}^\top A)\bar{x}=0 \Leftrightarrow \bar{x}_i(c_i-\bar{u}^\top a_i)=0$, $i=1,\ldots,n$
\end{itemize} 

\

Estas condiciones caracterizan las soluciones de (P) y (D)\\ (¿Por qué?)

\end{frame}
%----------------------------------------------
\begin{frame}
\frametitle{Holgura complementaria en problemas lineales}

\begin{center}
\begin{tabular}{lr}
maximizar & $3x_1 + 4x_2 + 6x_3$ \\
s.a. & $3x_1+x_2 + x_3 \leq 2$    \\
	 & $x_1 + 2x_2 + 6x_3 \leq 1$  \\
	 & $x_i\geq 0,\ i=1,2,3$
\end{tabular}
\end{center}


\


\begin{itemize}
\item Escribe el dual del problema anterior.
\item Resuelve el problema dual gráficamente.
\item Utiliza la condiciones de holgura complementaria para resolver el primal.
\end{itemize}


\end{frame}

%----------------------------------------------
\begin{frame}
\frametitle{Holgura complementaria}

Consideramos un problema general (no necesariamente convexo)
\begin{center}
\begin{tabular}{lll}
minimizar & $f(x)$ \\
s.a. & $f_i(x)\leq 0,$  &  $i=1,\ldots,m$ \\
	 & $h_i(x) = 0,$  &  $i=1,\ldots,p.$
\end{tabular}
\end{center}
para el que se cumple la dualidad fuerte. Sean $\bar{x}$ y $(\bar{u},\bar{v})$ las soluciones óptimas del primal y el dual respectivamente.

\begin{align*}
f(\bar{x}) &= g(\bar{u},\bar{v}) = \inf_x \big(f(x) + \sum_{i=1}^m \bar{u}_i f_i(x) + \sum_{i=1}^p \bar{v}_i h_i(x) \big)\\
&\leq f(\bar{x}) + \sum_{i=1}^m \bar{u}_i f_i(\bar{x}) + \sum_{i=1}^p \bar{v}_i h_i(\bar{x}) \leq f(\bar{x})
\end{align*}


Todas las desigualdades son en realidad igualdades.


\end{frame}
%----------------------------------------------
\begin{frame}
\frametitle{Holgura complementaria}



\begin{align*}
f(\bar{x}) &= g(\bar{u},\bar{v}) = \inf_x \big(f(x) + \sum_{i=1}^m \bar{u}_i f_i(x) + \sum_{i=1}^p \bar{v}_i h_i(x) \big)\\
&= f(\bar{x}) + \sum_{i=1}^m \bar{u}_i f_i(\bar{x}) + \sum_{i=1}^p \bar{v}_i h_i(\bar{x}) 
\end{align*}



Dos consecuencias relevantes:

\begin{itemize}
\item $\bar{x}$ minimiza $L(x,\bar{u},\bar{v})$ en $x$.

\item Se cumple $\sum_{i=1}^m \bar{u}_i f_i(\bar{x})=0$. Como consecuencia:
\[
\bar{u_i} f_i(\bar{x}) = 0, \ \ \ i=1,\ldots,m.
\]
Estas son las condiciones de holgura complementaria.
\end{itemize}


\end{frame}

%--------------------------------------------
\begin{frame}
\frametitle{Dualidad y condiciones de Karush-Kuhn-Tucker}

Supongamos las siguientes condiciones:

\begin{itemize}
\item Las funciones $f,f_1,\ldots, f_m,h_1,\ldots,h_p$ son diferenciables.
\item Las soluciones óptimas de los problemas primal y dual se alcanzan en los puntos
 $\bar{x}$ y $(\bar{u},\bar{v})$.
\item Hay dualidad fuerte: $f(\bar{x})=\bar{p}=\bar{d}=g(\bar{u},\bar{v})$.
\end{itemize}


\

Entonces, los óptimos deben cumplir las condiciones KKT:

\begin{align*}
\nabla f(\bar{x}) + \sum_{i=1}^m \bar{u}_i \nabla f_i(\bar{x}) + \sum_{i=1}^p \bar{v}_i\nabla h_i(\bar{x}) & = 0 \\
f_i(\bar{x}) &\leq 0, & i = 1,\ldots,m \\
h_i(\bar{x}) &= 0, & i = 1,\ldots,p \\
\bar{u}_i    & \geq 0, & i = 1,\ldots,m\\
\bar{u}_i f_i(\bar{x}) &  = 0, & i = 1,\ldots,m\\
\end{align*}

\end{frame}

%--------------------------------------------
\begin{frame}
\frametitle{Condiciones KKT en problemas convexos}


Cuando el problema primal es convexo, las condiciones KKT son suficientes para el óptimo global.

\


\textbf{Teorema:} Consideremos un problema primal convexo y con funciones diferenciables. Sean $\bar{x}$ y $(\bar{u},\bar{v})$ puntos que satisfacen las condiciones KKT. Entonces $\bar{x}$ es la solución del problema primal, $(\bar{u},\bar{v})$ es la solución del dual y  $\bar{p} =  \bar{d}$.

\

\textbf{¿Verdadero o falso?} Un punto $\bar{x}$ es la solución de un problema convexo si y solo si verifica las condiciones KKT.


\end{frame}
%--------------------------------------------
\begin{frame}
\frametitle{Condiciones KKT en problemas convexos}

\textbf{Demostración:}

\

\begin{itemize}
\item $\bar{x}$ es factible.

\

\item $L(x,\bar{u},\bar{v})$ es convexa en $x$.

\

\item Como su gradiente se anula en $\bar{x}$, se tiene que $\bar{x}$ es el mínimo global de $L(x,\bar{u},\bar{v})$.

\

\item  Por tanto,
\[
g(\bar{u},\bar{v}) = L(\bar{x},\bar{u},\bar{v}) = 
 f(\bar{x}) + \sum_{i=1}^m \bar{u}_i  f_i(\bar{x}) + \sum_{i=1}^p \bar{v}_i h_i(\bar{x}) = f(\bar{x}).
\]

\end{itemize}


\end{frame}
%----------------------------------------------
\begin{frame}
\frametitle{Dualidad fuerte para problemas convexos}

En problemas convexos la dualidad fuerte no se cumple en general. Pero puede probarse bajo condiciones adicionales no restrictivas. 

\

\textbf{Condición de Slater}: en un problema convexo
\begin{center}
\begin{tabular}{lll}
minimizar & $f(x)$ & \\
s.a. & $f_i(x)\leq 0,$  &  $i=1,\ldots,m$ \\
	 & $Ax=b,$  &  
\end{tabular}
\end{center}
en el que $\mbox{rango}(A)=p$, se cumple la condición de Slater si existe $\bar{x}\in D$  tal que $f_i(\bar{x})<0$, para $i=1,\ldots,m$, y $A\bar{x}=b$. 

\

(Recordatorio: estamos suponiendo $D$ abierto)

\end{frame}
%----------------------------------------------
\begin{frame}
\frametitle{Dualidad fuerte para problemas convexos}


\textbf{Teorema:} Si en un problema convexo se verifica la condición de Slater, entonces $\bar{p} =  \bar{d}$.

\

{\scriptsize 

{\bf Demostración:} 

\begin{itemize}
\item Como el primal es factible, $\bar{p}<\infty$. ¿Qué ocurre si $\bar{p}=-\infty$?

\

\item Definimos los conjuntos:
\begin{align*}
A &= \{(u,v,t)\in \mathbb{R}^m\times \mathbb{R}^p \times \mathbb{R}:\, \exists x\in D\ \mbox{con}\ 
f_i(x)\leq u_i,\ Ax-b=v,\ f(x)\leq t\} \\
B &= \{(0,0,s)\in \mathbb{R}^m\times \mathbb{R}^p \times \mathbb{R}:\, s<\bar{p}\}
\end{align*}



\item $A$ y $B$ son convexos y disjuntos.

\

\item Teorema de separación: existe $\alpha\in\mathbb{R}$ y $(\bar{u},\bar{v},\mu)\neq 0$ tales que 
\begin{enumerate}
\item Si $(u,v,t)\in A$, $\bar{u}^\top u + \bar{v}^\top v + \mu t \geq \alpha$.
\item Si $(0,0,t)\in B$, $ \mu t \leq \alpha$. Es decir  $\mu t \leq \alpha$ para $t<\bar{p}$, lo que implica $\mu\bar{p}\leq \alpha$.
\end{enumerate}

\

\item Por 1, $\bar{u}\geq 0$ y $\mu\geq 0$.

\end{itemize}


}

\end{frame}
%----------------------------------------------
\begin{frame}
\frametitle{Dualidad fuerte para problemas convexos}




{\scriptsize

\begin{itemize}
\item Además,  por 1 y 2, para todo $x\in D$,
\begin{equation}
\label{eq.fuerte}
\sum_{i=1}^m \bar{u}_i f_i(x) + \bar{v}^\top (Ax-b) + \mu f(x) \geq \mu\bar{p}
\end{equation}


\item Si $\mu>0$, para todo $x\in D$, $L(x,\bar{u}/\mu,\bar{v}/\mu)\geq\bar{p}$. Por lo tanto 
$g(\bar{u}/\mu,\bar{v}/\mu)\geq \bar{p}$ y $\bar{d}=\bar{p}$, que es lo que queríamos probar.

\

\item Usando la condición de Slater, vamos a llegar a una contradicción cuando $\mu=0$. Por (\ref{eq.fuerte}),
$
\sum_{i=1}^m \bar{u}_i f_i(x) + \bar{v}^\top (Ax-b) \geq 0,
$
para todo $x\in D$.


\


\item Si $\bar{x}$ es el punto en el que se cumple la condición de Slater, $\sum_{i=1}^m \bar{u}_i f_i(\bar{x})\geq 0$, por lo que $\bar{u}=0$. Como $(\bar{u},\bar{v},\mu)\neq 0$, tenemos que \textcolor{blue}{$\bar{v}\neq 0$}.

\

\item Pero $\bar{u}=0$ y (\ref{eq.fuerte}) implican $\bar{v}^\top (Ax-b) \geq 0$, para todo $x\in D$.

\

\item Para $\lambda>0$ suficientemente pequeño, $\bar{x}-\lambda A^{\top}\bar{v}\in D$ (abierto).

\

\item $\bar{v}^\top[A(\bar{x}-\lambda A^{\top}\bar{v}) - b] = -\lambda \|A^\top\bar{v}\|^2\geq 0$. Por lo que $A^\top\bar{v}=0$.

\

\item Como $\mbox{rango}(A)=p$, \textcolor{blue}{$\bar{v}=0$}. 


\end{itemize}

}


\end{frame}





%----------------------------------------------------------------------
\end{document}
%----------------------------------------------------------------------

%-----------------------------------------------
\begin{frame}
\frametitle{}

\end{frame}
%----------------------------------------------


MATERIAL SUPRIMIDO:
%----------------------------------------------------------------------
\begin{frame}
\frametitle{Minimización cuadrática con restricciones de igualdad}

Consideramos el problema 

\begin{center}
\begin{tabular}{ll}
minimizar & $(1/2) x^\top P x + q^\top x + r $ \\
s.a. & $Ax=b$,
\end{tabular}
\end{center}
donde $P$ es semidefinida positiva.

\

Las condiciones KKT para este problema se reducen a:
\begin{align*}
A\bar{x} &= b\\
P\bar{x}+q+A^\top\bar{v} &= 0
\end{align*}

\

Para obtener el óptimo basta resolver un sistema de ecuaciones lineales:
\[
\left(\begin{array}{cc}
P & A^\top \\ A & 0
\end{array}\right)
\left(\begin{array}{c}
\bar{x} \\ \bar{v} 
\end{array}\right)=
\left(\begin{array}{c}
-q \\ b 
\end{array}\right)
\]

\end{frame}