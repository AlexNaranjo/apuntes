% -*- root: ../InvestigacionOperativa.tex -*-

\section{Conjuntos convexos}

Vamos a ver qué es un conjunto convexo como caso particular de conjuntos afines.  
Dados 2 puntos $x_1,x_2$, y $\theta$, tomamos 
\[
y = \theta x_1 + (1-\theta)x_2 = x_2 + \theta (x_1-x_2),\ \ \  x_1,x_2\in\mathbb{R}^n.
\]

Vamos a pensar qué ocurre diferentes valores de $\theta$:


\begin{figure}
\begin{center}
\includegraphics[scale=0.8]{tex/berrendero/tema2/combinacion}
\caption{Ejemplo de combinaciones lineales de 2 puntos para ilustrar conjuntos afin y convexo.}
\label{sec2:comb}
\end{center}
\end{figure}


En la figura $\ref{sec2:comb}$, vemos que la recta entera sería el conjunto afín entero, mientras que sólo el intervalo $\theta \in [0,1]$ sería el conjunto convexo. Formalmente,

\begin{defn}[Conjunto\IS afín]
Dados $x_1,x_2\in S$, S es un conjunto afín si $\theta x_1 + (1-\theta)x_2\in S$, para todo $\theta\in \mathbb{R}$.
\end{defn}

\begin{defn}[Conjunto\IS convexo]
Dados $x_1,x_2\in S$, S es un conjunto convexo si $\theta x_1 + (1-\theta)x_2\in S$, para todo $\theta\in [0,1]$.
\end{defn}





Vamos a ver ejemplos de subconjuntos convexos.
\begin{itemize}
\item \textbf{Hiperplanos}: $S=\{x:\, p^\top x = \alpha\}$, donde $p\in\mathbb{R}^n$, $\alpha\in\mathbb{R}$. 

\item \textbf{Semiespacios}: $S=\{x:\, p^\top x \leq \alpha\}$, donde $p\in\mathbb{R}^n$, $\alpha\in\mathbb{R}$. 

\item \textbf{Intersección arbitraria} de convexos: Si $S_i$ es convexo para todo $i\in I$, entonces $S=\bigcap_{i=1}^I S_i$ es un conjunto convexo.

\item Un \textbf{poliedro} (intersección finita de semiespacios) es un conjunto convexo. Por ejemplo, $S=\{x:\, Ax\leq b,\ x\geq 0\}$ es un conjunto convexo.

\item Una \textbf{bola} $B(\bar x,r)=\{x\in\mathbb{R}^n:\, \|x-\bar x\|<r\}$ es un conjunto convexo (para cualquier norma).

\end{itemize}



\subsection{Combinaciones convexas y afines}

Hemos visto lo que son los conjuntos convexos, pero a la hora de trabajar, puede que el conjunto de datos con el que trabajamos, no sea convexo. ¿Tiene sentido hablar del "mínimo conjunto convexo" que contiene al conjunto con el que estamos trabajando? Vamos a verlo, pero para ello necesitamos definir qué es una combinación.

\begin{defn}[Combinación\IS afín]

Sean $x_1,\ldots,x_k \in\mathbb{R}^n$. Una combinación afín de $\{x_i\}$ es
\[
y = \lambda_1 x_1+\cdots +\lambda_k x_k,
\]
donde $\lambda_1+\cdots +\lambda_k=1$.
\end{defn}


En el caso de combinaciones convexas, tenemos alguna restricción más, ya que los conjuntos convexos son subconjuntos de conjuntos afines.

\begin{defn}[Combinación\IS convexa] 
Sean $x_1,\ldots,x_k \in\mathbb{R}^n$. Una combinación convexa de $\{x_i\}$ es
\[
y = \lambda_1 x_1+\cdots +\lambda_k x_k,
\]
donde $\lambda_1+\cdots +\lambda_k=1$ y $\lambda_i \geq 0$, para todo $i=1,\ldots,n$.
\end{defn}


Ya tenemos los ingredientes para construir los \textbf{cierres}, es decir, los mínimos conjuntos convexos/afines que contienen a un conjunto.


\begin{defn}[Cierre\IS afín] 
Definimos $\afin{S}$, el cierre afín de un conjunto $S$ como
\[
\afin{S} = \left\{\sum_{i=1}^k \lambda_i x_i:\, x_i\in S,\  \sum_{i=1}^k \lambda_i = 1\right\}.
\]
\end{defn}

\paragraph{Propiedades:}
\begin{itemize}
\item Un conjunto es afín si y solo $S=\afin{S}$.

\item Un conjunto es afín si y solo si es la traslación de un subespacio vectorial (único)

\item La dimensión afín de un conjunto es la dimensión de su cierre afín (que a su vez es la dimensión del correspondiente subespacio vectorial).
\end{itemize}




\begin{defn}[Cierre convexo]
Definimos $\convx{S}$, el cierre convexo de un conjunto $S$ como
\[
\convx{S} = \left\{\sum_{i=1}^k \lambda_i x_i:\, x_i\in S,\ \lambda_i\geq 0,\ \sum_{i=1}^k \lambda_i = 1\right\}.
\]
\end{defn}

\paragraph{Propiedades:}
\begin{itemize}
\item  S es convexo si y sólo si $S = \convx{S}$

\begin{proof} Vamos a separar las implicaciones:


$\impliedby)$:\\ $S = \convx{S} \implies S$ convexo es trivial, vamos a demostrar la otra implicación:

$\implies)$:\\ Queremos demostrar que si $S$ es convexo, $S\subset \convx{S}$ y $\convx{S}\subset S$. Es trivial que $S\subset\convx{S}$, asique vamos a ver la otra inclusión y vamos a demostrarlo por inducción sobre $k$. Sea

\[x = \sum^k λ_ix_ki \text{ con } x_i \in S,λ\geq 0, \sum λ = 1\]

\subparagraph{Base:} $k=1$ es trivial.

\subparagraph{Paso:} Supongamos, sin pérdida de generalidad que $λ_{k+1} < 1$ y tomamos:
\[\sum^{k+1} λ_ix_i = (1-λ_{k+1})\underbrace{\left(\frac{λ_1}{1-λ_{k+1}}x_1 + ... + \frac{λ_k}{1-λ_{k+1}}x_k \right)}_{(1)}+λ_{k+1}x_{k+1}\]
$(1)$ es una combinación convexa, con lo que este paréntesis $\in S$. 

Por otro lado, $(1-λ_{k+1})a + λ_{k+1}x_{k+1}$ es una combinación convexa, ya que $a,x_{k+1}\in S$ y ambos coeficientes cumplen las restricciones.

Con ello, vemos que $\convx{S}\subset S$.
\end{proof}


\item $\convx{S}$ es el menor conjunto convexo que contiene a $S$

\end{itemize}

