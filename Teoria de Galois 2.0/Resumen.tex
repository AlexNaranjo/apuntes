\documentclass[nochap]{apuntes}

\title{Teoría de Galois}
\author{Pedro Valero}
\date{14/15 C1}

% Paquetes adicionales
\usepackage{tikztools}
\usepackage{fastbuild}
\usetikzlibrary{matrix,arrows}
% --------------------

\begin{document}
\pagestyle{plain}
\maketitle

\tableofcontents
\newpage

\section{Extensiones algebraicas}
\begin{itemize}
\item Dado $F$ subcuerpo de $E$ decimos que $E$ es una \textbf{extensión} de $F$ y la denotamos como $E/F$

\item Un elemento $α\in E$ es \textbf{algebraico} si existe un polinomio $p(x) \in F[x]$ que lo anula. Si todo elemento $α \in E$ es algebraico diremos que se trata de una \textbf{extensión algebraica}. Una extensión generada por elementos algebraicos es algebraica.

\item Podemos ver $E$ como un espacio vectorial sobre $F$. Diremos que $E$ es una \textbf{extensión finita} si es un espacio vectorial de dimensión finita sobre $F$. 

Con [$E$:$F$] denotamos la dimensión de $E$ visto como espacio vectorial sobre $F$ y lo llamamos \textbf{grado de la extensión}

\item Si $E$ es una extensión finita sobre $F$ todo elemento de $E$ es algebraico. Es decir, toda extensión finita es algebraica.

\item Una extensión puede ser algebraica sin ser finita. Por ejemplo: 
\[[\rac_{2^{\frac{1}{2}}, 2^{\frac{1}{3}}, ..., 2^{\frac{1}{n}}}:\rac] = \infty\]

\item Sea $α\in E$ algebraico sobre F y sea $n$ el grado de su polinomio mínimo irreducible sobre $F[t]$, entonces $F_{(α)}$ es una extensión de $F$ de grado $n$, generada por $1, α, α^2, ..., α^{n-1}$.

\item Dado un elemento α$\in E$ algebraico sobre $F$, $F_{(α)}$ es el mínimo cuerpo que contiene a $F$ y α. Por ser α algebraico, $F_{(α)}=F[α]$

\item $[E$:$F]$=1 $\iff$ $E=F$

\item Dadas las extensiones finitas $E_2$/$E_1$ y $E_1/F$, entonces:
\[[E_2:F]=[E_2:E_1]\times [E_1:F]\]

\item Un cuerpo es \textbf{algebraicamente cerrado} si todo polinomio de grado mayor o igual a 1 con coeficientes en él tiene una raiz dentro de ese mismo cuerpo. $\cplex$ es algebraicamente cerrado.Aplicando esta propiedad de forma recursiva vemos que en un cuerpo algebraicamente cerrado todo polinomio tiene tantas raíces en el cuerpo como grado tenga.

\item Dado un cuerpo $F$, llamamos \textbf{cierre algebraico} de $F$ al cuerpo $A$ que es algebraicamente cerrado y algebraico sobre $F$.

\item Definimos como \textbf{característica de un cuerpo} al número de veces que debemos sumar el neutro multiplicativo para obtener el neutro aditivo. Si no se puede obtener (por ejemplo, en $ℚ$ no podemos obtener el $0$ por muchas veces que sumemos $0$), se dice que el cuerpo tiene característica $1$.

\item Si $F$ es un subcuerpo de un cuerpo algebraicamente cerrado con característica 0, todo polinomio irreducible tiene tantas raices \textbf{distintas} como grado tenga.

\end{itemize}

\section{Inmersiones}
\begin{itemize}
\item Una \textbf{inmersión}, σ, es un homomorfismo que va de un cuerpo a otro. Siempre es inyectivo.

\item Sólo hay una inmersión de $\rac$ a cualquier cuerpo.

\item Si una inmersión es, además, sobreyectiva, será biyectiva y pasamos a denominarla \textbf{isomorfismo}

\item Dada una inmersión $\appl{σ}{F}{L}$, sea $p(t)$ un polinomio irreducible en $F[t]$, entonces $σ(p(t))$ es un polinomio irreducible en $L[t]$

\item Dada una inmersión $\appl{σ}{F}{L}$, si α es una raíz de $p(t) \in F[t]$, entonces σ(α) es una raíz en $σ(p(t))\in L[t]$

\item La imagen de un polinomio por una inmersión es el polinomio obtenido al aplicar la inmersión a los coeficientes.

\item Dada una inmersión $\appl{σ}{F}{L}$, sea $p(t)\in F[t]$ un polinomio irreducible y sea α una raíz del mismo en alguna extensión de $F$ y sea $\beta$ una raíz de $σ(f(t))$ en L, existe una inmersión $\appl{\tau}{F_{(α)}}{L}$ con $\tau(α) = \beta$

\item Sea $F$ un cuerpo y sea $p(t)\in F[t]$ un polinomio irreducible, entonces existe una extensión $E/F$ en la que $p(t)$ tiene una raíz.

\item Sea $E/F$ una extensión finita y sea $\appl{σ}{F}{A}$ una inmersión sobre un cuerpo algebraicamente cerrado. Entonces existe una extensión de σ a una inmersión $\appl{σ'}{E}{A}$

\item Sea α un elemento algebraico sobre $F$ y sea $p(t)$ su polinomio irreducible. Denotamos como $α_1,...α_n$ las raíces de $p(t)$ y las denominamos \textbf{conjugados de α sobre F}. Entonces para cada $α_i$ existe una inmersión que cumple $σ_i(α) = α_i$ y deja los elementos de $F$ como están.

\item Dada una extensión con $[E:F]=n$, sea $\appl{α}{F}{A}$ con $A$ algebraicamente cerrado con característica 0, entonces hay $n$ extensiones de $\sigma$ a inmersiones de $E$ en $A$. 

\item Dado un cuerpo con característica 0, y dada una extensión del mismo $E/F$, existe un elemento $\gamma$ tal que $E=F_{(\gamma)}$. Un elemento como este $\gamma$ que nos permite generar toda una extensión se denomina \textbf{primitivo}

\item Las dos últimas proposiciones no se cumplen si el cuerpo $A$ no tiene característica 0. Con el fin de hacer algo similar en estos casos se añade una nueva condición.

Decimos que un polinomio $f(t)\in F[t]$ de grado $n$ es \textbf{separable} si tiene exactamente $n$ raíces distintas. Consecuentemente, un elemento $α\in E$ es \textbf{separable} (siendo $E$ una extensión de $F$) si su polinomio irreducible es separable. Decimos que una extensión finita es \textbf{separable} si el número de extensiones de una inmersión $\appl{α}{F}{A}$ a una inmersión de $E$ en $A$ es igual al grado $[E:F]$. (Siendo $A$ un cuerpo algebraicamente cerrado de característica arbitraria)

Estas son algunas propiedades sobre las extensiones separables que el libro pide demostrar:
\begin{itemize}
\item Si $E = F_{(α_1, α_2,...α_n)}$ y cada $α_i$ es separable sobre $F$, entonces $E$ es separable sobre $F$.
\item Si tenemos una torre de extensiones separables los límites de la torre constituyen una extensión separable.
\item Si $E$ es separable sobre $F$, todo subcuerpo de $E$ que contenga a $F$ será separable sobre $F$.
\item Sea $E$ una extensión finita separable de $F$, existe un elemento $\gamma$ tal que $E=F_{(\gamma)}$
\end{itemize}
\end{itemize}

\section{Cuerpos de descomposición}
\begin{itemize}
\item Sea $f(t) \in F[t]$ se dice que $E/F$ es el \textbf{cuerpo de descomposición} de $f(t)$ si podemos escribir $f(t)=(t-α_1)...(t-α_n) \in E[t]$ con $E=F_{(α_1,...α_n)}$. Es decir, añadimos al cuerpo las raíces del polinomio.

\item Para todo polinomio $f(t)\in F[t]$ existe un cuerpo de descomposición.

\item Si tenemos $K_1, K_2$ cuerpos de descomposición de un mismo polinomio $f(t) \in F[t]$, entonces existe un isomorfismo que extiende la identidad en $F$ y que va de $K_1$ a $K_2$..

\item Una extensión se llama \textbf{normal} si es el cuerpo de descomposición de algún poliniomio $f(t) \in F[t]$.

\item Sea $K$ una extensión normal de $F$. Si $p(t) \in F[t]$ es irreducible sobre $F[t]$ y tiene una raíz en $K$, entonces $p(t)$ tiene todas sus raíces en $K$.
\end{itemize}

\section{Teoría de Galois}
\begin{itemize}

\item Una extensión es \textbf{galoisiana} si es normal y separable. En característica 0 toda extensión normal será una extensión galoisiana.

\item A partir de una extensión galoisiona podemos definir el \textbf{grupo de Galois} como:
\[Gal(E/F) = \left\{\appl{σ}{E}{E} \text{ automorfismo} \tq σ_{|F} = id\right\}\]
Además, $\#(Gal(E/F))=[E:F]$.

\item \textbf{Teorema fundamental de la teoría de Galois}. Sea $K/F$ una extensión galoisiana y finta:
\begin{enumerate}
\item Se tiene la biyección:
\[\{E | F\subset E \subset K\} \rightarrow \text{subgrupos de }Gal(K/F)\]
tal que:
\[E \rightarrow Gal(K/E)\]
\[K^H=\{x \in K | \sigma(x)=x \forall \sigma \in H\}\leftarrow H\]
\item $\forall H < G$, la extensión $K/K^H$ es galoisiana y $[K:K^H]=\abs{H}$
\item $[K^H:F]=\frac{|G|}{|H|}=\frac{[K:F]}{|H|}$
\item $K^H/F$ es galois $\iff H \lhd G$ y además:
\[G/H \longrightarrow Gal(K^H/F)\]
		\[σ \longmapsto σ_{|K^H}\]
\end{enumerate}

\end{itemize}

\end{document}