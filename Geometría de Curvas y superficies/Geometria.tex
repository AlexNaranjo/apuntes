\documentclass[nochap]{apuntes}

\title{Geometría de curvas y superficies}
\author{Por definir}
\date{13 / 14 C2}

\begin{document}

\maketitle
\newpage

\section{Introducción}
Existen varias formas de representar curvas en $\real^2$.

\begin{defn}[Curva parametrizada diferenciabble]
Es una aplicación diferenciable $\appl{sigma}{I=(a,b)}{\real^n}$

Vamos a ver varios ejemplos de curvas parametrizadas:

\begin{example}
Definimos la curva $x^3+y^3 = xy$ y la parametrizamos utilizando la indicación $t = \frac{x}{y}$.

Tras unos cuantos cálculos llegamos a /[\sigma(t)=\left(\frac{t}{1+t^3},\frac{t^2}{1+t^3}\right) /]

Nos damos cuenta que en $t=-1$ no está definida. Vamos a estudiar a ver que sale. Nos fijamos en el intervalo $(-1,1]$.Si $t\to-1^{-1}$ Estudiamos la asíntota


Además es interesante estudiar la simetría. Suponemos que puede ser simétrica respecto de la recta $y=x$. Para ver si una curva parametrizada es simétrica respecto de la recta estudiamos qué pasa con $\sigma\left(\frac{1}{t}\right)$. 

Sustituímos y vemos que $(x(t),y(t)) = (y(1/t),x(1/t))$ con lo que concluimos que es simétrica.

\end{example}
Folium de descartes (análisis completo)


\end{document}