\section{Hoja 1}

\begin{problem} Encuentra el grado de las siguientes extensiones.
\solution
\begin{enumerate}
\item $[ℚ(\sqrt{2}) : ℚ ]= 2$.
\item $[ℚ(\sqrt{6}) : ℚ ]= 2$ con $p(t) = t^2 - 6$: su solución es $\sqrt{6}$ que no está en $ℚ$. El criterio de Eisenstein también funciona con $p=2$: $p$ divide al término independiente pero no al de mayor grado.
\item $[ℚ(\sqrt{16}) : ℚ ]= 1$ con $p(t) = t-4$.
\item $[ℚ(\sqrt{p}) : ℚ] = 2$ con $p$ primo, siendo $p(t) = t^2 - p$.
\item $[ℚ\left(e^{\frac{2πi}{n}}\right) : ℚ]$ con $n=2,3,4,5,6,8$.
\begin{enumerate}
\item Para $n=2$, $e^{\frac{2πi}{n}} = -1$ y $[ℚ(-1):ℚ] =  1$.

\item Para $n=3$, el polinomio que lo anula es $p(t) = t^3 - 1$, pero no es irreducible porque tiene como raíz $1$, y entonces $p(t) = (t-1) (t^2 + t + 1)$. Luego $e^{\frac{2πi}{3}}$ es raíz de $t^2+t+1$, que tiene que ser irreducible (si una raíz,$e^{\frac{2πi}{n}}$, es compleja, la otra es su conjugado, que también es complejo). $[ℚ\left(e^{\frac{2πi}{3}}\right) : ℚ] = 2$.

\item Para $n=4$, $e^{\frac{2πi}{4}} = i$, luego $[ℚ(i) : ℚ ] = 2$ con $p(t) = t^2 + 1$. Si no fuese $p(t)$ irreducible, tendríamos de nuevo que $[ℚ(i) : ℚ]=1$ y $ℚ(i)$ sería igual a $ℚ$, lo cual es absurdo porque $i$ no está en $ℚ$.

\item Para $n=5$, está claro que $h(t) = t^5 - 1$ anula a $e^{\frac{2πi}{5}}$, luego el grado de la extensión es menor o igual que 5. Desde luego no es irreducible porque 1 es raíz $h(t) = (t-1)(t^4 + t^3 + t^2 + t + 1)$. $e^{\frac{2πi}{5}}$ debe ser raíz de $t^4 + t^3 + t^2 + t + 1$. ¿Es este polinomio irreducible? Desarrollaremos un método más general para encontrar los polinomios irreducibles de las raíces de la unidad, pero para este nos quedamos con una sugerencia: sí es irreducible y $[ℚ\left(e^{\frac{2πi}{5}}\right): ℚ ] = 4$. Sólo hay que estudiar la aplicación $ℚ[t] \mapsto ℚ[t]$ que manda $t$ a $t+1$ y por lo tanto $\sum a_i t^i \mapsto \sum a_i (t+1)^i$, que si es un isomorfismo de anillos entonces $p(t)$ es irreducible si y sólo si $p(t+1)$ lo es. En este caso podríamos usar el criterio de Eisenstein para probarlo. Esto se resolvi\'o en un ejemplo de teor\'ia (\ref{Teoria_H1.E1.A5.S_Af})

\item Para $n=6$, $h(t) = t^6 - 1$ lo anula, pero podemos descomponerlo como $h(t) = (t^3 - 1)(t^3 + 1)$. $e^{\frac{2πi}{6}}$ es raíz sólo de $t^3 + 1$, pero este polinomio tampoco es irreducible: $t^3 + 1 = (t-1)(t^2 - t + 1)$. $e^{\frac{2πi}{6}}$ es raíz de $t^2-t+1$, que por el mismo argumento que en el caso $n=4$ tiene que ser irreducible, o si no un número complejo estaría en $ℚ$.

\item Por último, cuando $n=8$, sabemos que $h(t) = t^8 - 1$ anula a $e^{\frac{2πi}{8}}$. Pero descomponiendo de nuevo, $h(t) = (t^4 -1)(t^4 +1)$, $e^{\frac{2πi}{8}}$ es raíz de $t^4 +1$, el grado de la extensión es menor o igual que cuatro. Por suerte, $t^4 + 1$ es irreducible: las raíces deberían ser enteras y dividir a $1$, y ni $1$ ni $-1$ son raíces.
\end{enumerate}
\end{enumerate}
\end{problem}


\begin{problem}Encuentra el grado de las siguientes extensiones

\solution

\begin{itemize}

\item $[\rac(\sqrt{6},i):\rac]$. 

 Para resolver estos ejercicios, el procedimiento es escribirlo desmenuzadas las inclusiones.

$\rac \to \rac(\sqrt{6}) \to \rac(\sqrt{6})(i) = \rac(\sqrt{6},i)$

Y vamos caso por caso:

\begin{itemize}
\item $[\rac: \rac(\sqrt{6})] = 2$
\item $[\rac(i): \rac(\sqrt{6}) ] = 2 (i \notin \rac(\sqrt{6}))$
\end{itemize}

Entonces: $[\rac(\sqrt{6},i):\rac]= 2\cdot 2 = 4$.

\item $[\rac(\sqrt[n]{p},\sqrt{-m}):\rac] = 2n$. Se resuelve generalizando la misma idea del anterior.

\end{itemize}
\end{problem}