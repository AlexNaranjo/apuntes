% -*- root: ../TeoriaGalois.tex -*-
\section{Hoja 1}

\begin{problem} Encuentra el grado de las siguientes extensiones.
\solution
\begin{enumerate}
\item $[ℚ(\sqrt{2}) : ℚ ]= 2$.
\item $[ℚ(\sqrt{6}) : ℚ ]= 2$ con $p(t) = t^2 - 6$: su solución es $\sqrt{6}$ que no está en $ℚ$. El criterio de Eisenstein también funciona con $p=2$: $p$ divide al término independiente pero no al de mayor grado.
\item $[ℚ(\sqrt{16}) : ℚ ]= 1$ con $p(t) = t-4$.
\item $[ℚ(\sqrt{p}) : ℚ] = 2$ con $p$ primo, siendo $p(t) = t^2 - p$.
\item $[ℚ\left(e^{\frac{2πi}{n}}\right) : ℚ]$ con $n=2,3,4,5,6,8$.
\begin{enumerate}
\item Para $n=2$, $e^{\frac{2πi}{n}} = -1$ y $[ℚ(-1):ℚ] =  1$.

\item Para $n=3$, el polinomio que lo anula es $p(t) = t^3 - 1$, pero no es irreducible porque tiene como raíz $1$, y entonces $p(t) = (t-1) (t^2 + t + 1)$. Luego $e^{\frac{2πi}{3}}$ es raíz de $t^2+t+1$, que tiene que ser irreducible (si una raíz,$e^{\frac{2πi}{n}}$, es compleja, la otra es su conjugado, que también es complejo). $[ℚ\left(e^{\frac{2πi}{3}}\right) : ℚ] = 2$.

\item Para $n=4$, $e^{\frac{2πi}{4}} = i$, luego $[ℚ(i) : ℚ ] = 2$ con $p(t) = t^2 + 1$. Si no fuese $p(t)$ irreducible, tendríamos de nuevo que $[ℚ(i) : ℚ]=1$ y $ℚ(i)$ sería igual a $ℚ$, lo cual es absurdo porque $i$ no está en $ℚ$.

\item Para $n=5$, está claro que $h(t) = t^5 - 1$ anula a $e^{\frac{2πi}{5}}$, luego el grado de la extensión es menor o igual que 5. Desde luego no es irreducible porque 1 es raíz $h(t) = (t-1)(t^4 + t^3 + t^2 + t + 1)$. $e^{\frac{2πi}{5}}$ debe ser raíz de $t^4 + t^3 + t^2 + t + 1$. ¿Es este polinomio irreducible? Desarrollaremos un método más general para encontrar los polinomios irreducibles de las raíces de la unidad, pero para este nos quedamos con una sugerencia: sí es irreducible y $[ℚ\left(e^{\frac{2πi}{5}}\right): ℚ ] = 4$. Sólo hay que estudiar la aplicación $ℚ[t] \mapsto ℚ[t]$ que manda $t$ a $t+1$ y por lo tanto $\sum a_i t^i \mapsto \sum a_i (t+1)^i$, que si es un isomorfismo de anillos entonces $p(t)$ es irreducible si y sólo si $p(t+1)$ lo es. En este caso podríamos usar el criterio de Eisenstein para probarlo. Esto se resolvi\'o en un ejemplo de teor\'ia (\ref{Teoria_H1.E1.A5.S_Af})

\item Para $n=6$, $h(t) = t^6 - 1$ lo anula, pero podemos descomponerlo como $h(t) = (t^3 - 1)(t^3 + 1)$. $e^{\frac{2πi}{6}}$ es raíz sólo de $t^3 + 1$, pero este polinomio tampoco es irreducible: $t^3 + 1 = (t-1)(t^2 - t + 1)$. $e^{\frac{2πi}{6}}$ es raíz de $t^2-t+1$, que por el mismo argumento que en el caso $n=4$ tiene que ser irreducible, o si no un número complejo estaría en $ℚ$.

\item Por último, cuando $n=8$, sabemos que $h(t) = t^8 - 1$ anula a $e^{\frac{2πi}{8}}$. Pero descomponiendo de nuevo, $h(t) = (t^4 -1)(t^4 +1)$, $e^{\frac{2πi}{8}}$ es raíz de $t^4 +1$, el grado de la extensión es menor o igual que cuatro. Por suerte, $t^4 + 1$ es irreducible: las raíces deberían ser enteras y dividir a $1$, y ni $1$ ni $-1$ son raíces.
\end{enumerate}
\end{enumerate}
\end{problem}


\begin{problem}Encuentra el grado de las siguientes extensiones

\solution

\begin{itemize}

\item $[\rac(\sqrt{6},i):\rac]$.

 Para resolver estos ejercicios, el procedimiento es escribirlo desmenuzadas las inclusiones.

$\rac \to \rac(\sqrt{6}) \to \rac(\sqrt{6})(i) = \rac(\sqrt{6},i)$

Y vamos caso por caso:

\begin{itemize}
\item $[\rac: \rac(\sqrt{6})] = 2$
\item $[\rac(i): \rac(\sqrt{6}) ] = 2 (i \notin \rac(\sqrt{6}))$
\end{itemize}

Entonces: $[\rac(\sqrt{6},i):\rac]= 2\cdot 2 = 4$.

\item $[\rac(\sqrt[n]{p},\sqrt{-m}):\rac] = 2n$. Se resuelve generalizando la misma idea del anterior.

\item $[ℚ(\sqrt[3]{2},e^{\frac{2 \pi i}{3}}\cdot \sqrt[3]{2}):ℚ]$

Lo primero que nos tenemos que dar cuenta es de que \\
 $ℚ(\sqrt[3]{2},e^{\frac{2 \pi i}{3}}\cdot \sqrt[3]{2}) = ℚ(\sqrt[3]{2},e^{\frac{2 \pi i}{3}})$

Ahora escribimos la cadena de inclusiones:

\begin{gather*}
[ℚ : ℚ(\sqrt[3]{2})] = 3, p(t)=t³-2\\
[ℚ(\sqrt[3]{2}) : ℚ(\sqrt[3]{2})(e^{\frac{2\pi i}{3}})]  [ℚ(e^{\frac{2\pi i}{3}})] = 2; p(t) =  t²+t+1\\
[ℚ(\sqrt[3]{2}) : ℚ(\sqrt[3]{2})(e^{\frac{2\pi i}{3}})] = 2; e^{\frac{2\pi i}{3}}∉ℚ(\sqrt[3]{2})
\end{gather*}

Entonces deducimos que $[ℚ(\sqrt[3]{2},e^{\frac{2 \pi i}{3}}\cdot \sqrt[3]{2}):ℚ] = 2·3 = 6$
\end{itemize}

\end{problem}


\begin{problem}[3]
Sean $E₁$ y $E₂$ dos extensiones distintas de un cuerpo $F$ de grados $p$ y $p'$ respectivamente, donde $p$ y $p'$ son
primos distintos. Demostrar que $E_1 ∩ E_2 = F$.

\solution

Si $p,p'$ son primos con $p≠p'$, entonces $E₁∩E₂ = F$ ($\dimplies m=1$).


$\begin{array}{c}
p=m·a\\
p'=m·b
\end{array}$

Como $p$ es primo, $\implies m=1 || a=1$. Si $m=1$ todo cuadra. Si $a=1 \implies p'=p·b$ que es imposible, porque $p'$ es primo.
\end{problem}

\begin{problem}[4]
Sea $K$ un cuerpo finito. Demostrar $card(K) = p^m$ para algún primo $p$. Además, $p = char(K)$ (característica).

\solution

Lo primero es recordar la \concept[Característica]{característica}: el número de veces que tienes que sumar la unidad para llegar al 0 en cuerpos finitos

Tenemos el siguiente homomorfismo de anillos (φ):
\[
\begin{array}{ccc}
ℤ &\to & K\\
1 &\to & 1_K\\
n &\to & n·1_K = 1_K + ... + 1_K
\end{array}
\]

$Ker(\phi)$ es un ideal de $ℤ \implies Ker(\phi) = (n)$.

Por algo que me he perdido, tenemos \[
\appl{\bar{φ}}{\frac{ℤ}{(n)}}{Im(φ) \subset K}
\]

Afirmamos que $n≠0$ puesto que $K$ es finito y $\frac{ℤ}{(0)} = ℤ$.

Afirmamos que $n$ es primo. Vamos a demostrarlo por reducción al absurdo:

Si $n=a·b$. Entonces en $\frac{ℤ}{(n)} = \bar{a}·\bar{b} = \bar{a·b} = \bar{n} = \bar{0}$. Y esto no puede pasar porque en un cuerpo no hay divisores de 0.


Pongamos $p=n$. Así tenemos:

$\mathbb{F}_p = \frac{ℤ}{(n)}$ un subcuerpo de $K$.


Luego $K$ es un espacio vectorial sobre $\mathbb{F}_p$.

Sea $[K:\mathbb{F}_p] = m$ y sea $\{x₁,...,x_m\}$ es base de $K$ sobre $\mathbb{F}_p$.

$\implies \forall x∈K$, x se escribe de manera única: $a₁ ·x₁ + ... + a_m·x_m\;,a_i∃\mathbb{F}_p \implies card(K) = p^m$.

Observación: \[\mathbb{F}_{pm} ≠ \frac{ℤ}{(p^m)}\]

\paragraph{Conclusión: } ¿Hay un cuerpo con 9 elementos? Sí, porque 9=3².

El problema \ref{H1.E6} va de construirlos.
\end{problem}


%%%%%%%%%% Malvado 2-10-2014 %%%%%%%%%%%%%
\section{Hoja 2}

% Ejercicio 1
\begin{problem}[1] Inmersiones de $ℚ(\alpha)$ en $ℂ$
\solution
\begin{enumerate}
	\item $\alpha \eq \sqrt{2}$.\\
	{\inputtikz{ejercicios/img1-ej1-h2}} \\
	$\sigma(\sqrt{2}) \eq \sqrt{2}  \Rightarrow \sigma_1 \eq Id$ \\
	$\sigma_2(\sqrt{2}) \eq -\sqrt{2}$

	\item $\alpha \eq i \eq \sqrt{2}$ \\
	{\inputtikz{ejercicios/img2-ej1-h2}} \\
	$p(t) = t^2 + 1$\\
	$\sigma_1(i) \eq i \Rightarrow \sigma_1 \eq Id$\\
	$\sigma_2(i) \eq -i \Rightarrow \sigma_2 \eq$ conjugación compleja\\
	$\sigma_2(a+bi) \eq a + b(-i) = a + bi$

	\item $\alpha \eq \sqrt[3]{2}$\\
	{\inputtikz{ejercicios/img3-ej1-h2}} \\
	polinomio mínimo de $\sqrt[3]{2}$ es $p(t) \eq t^3 - 2$\\
	raíces $\sqrt[3]{2}, e^{\frac{2πi}{3}}\sqrt[3]{2}, e^{\frac{4πi}{3}}\sqrt[3]{2}$\\
	$\sigma_1(\sqrt[3]{2}) \eq \sqrt[3]{2} \implies \sigma_1 \eq Id$\\
	$\sigma_2(\sqrt[3]{2}) \eq e^{\frac{2πi}{3}}\sqrt[3]{2}$\\
	$\sigma_3(\sqrt[3]{2}) \eq e^{\frac{4πi}{3}}\sqrt[3]{2}$

	E.G $x \eq 7 + 2\sqrt[3]{2} + 5(\sqrt[3]{2})^2 \in ℚ(\sqrt[3]{2})$\\
	$\sigma_2(x) = 7 + 2e^{\frac{2πi}{3}}\sqrt[3]{2} + 5e^{\frac{4πi}{3}}(\sqrt[3]{2})^2$

	¿Cuáles son entonces?
	$\sigma_2(ℚ(\sqrt[3]{2})) \eq ℚ(e^{\frac{2πi}{3}}\sqrt[3]{2}) \neq ℚ(\sqrt[3]{2})$ con $\sigma_2(ℚ(\sqrt[3]{2})) \subset \real$
\end{enumerate}
\end{problem}


% Ejercicio 2
\begin{problem}[2] Escribir todos los automorfismos de $ℚ(e^{\frac{2πi}{8}})$ que extienden el automorfismo $\sigma: ℚ(i) \rightarrow ℚ(i)$ que envía $i$ a $-i$

\solution
\begin{enumerate}
	\item
	{\inputtikz{ejercicios/img1-ej2-h2}} \\
	en el diagrama $\sigma(i) \eq -i$

	\begin{enumerate}
	\item polinomio irreducible de $e^{\frac{2πi}{8}}$ sobre $ℚ(i)$\\
	$(t^4 + 1)(t^4 - 1) \eq t^8 - 1$ anula a $e^{\frac{2πi}{8}}$\\
	$t^4 + 1$ era el polinomio irreducible de $e^{\frac{2πi}{8}}$ sobre $ℚ$ y sobre $ℚ(i)$\\
	$t^4 + 1 \eq (t^2 - i)(t^2 + i) \implies$ el polinomio irreducible de $e^{\frac{2πi}{8}}$ sobre $ℚ(i)$ es $p(t) = t^2 - i$, pues $(e^{\frac{2πi}{8}})^2 \eq
e^{\frac{πi}{2}} \eq i$\\
	Las dos extensiones son:\\
	$\sigma_1(e^{\frac{2πi}{8}}) \eq w_1 \neq e^{\frac{2πi}{8}}$\\
	$\sigma_2(e^{\frac{2πi}{8}}) \eq w_2 \neq e^{\frac{2πi}{8}}$\\
	donde $w_1$ y $w_2$ son raíces de $p^{\sigma}(t) \eq t^2 + i$
	\end{enumerate}

\end{enumerate}
\end{problem}

\begin{problem}[3]
Respondemos aquí sólo al apartado c) por petición de una alumna de matemáticas, el resto no se hicieron en clase.

El ejercicio pide encontrar todas las inmersiones de $\rac(\sqrt[3]{2},i)$ en $\overline{\rac}$
\solution

Es fácil de comprobar que:
$$[ℚ(\sqrt[3]{2},i):ℚ] = 6$$

Ahora vamos a añadirle algo de complejidad/interés al problema calculando el elemento primitivo de esta extensión.

Un elemento primitivo de $ℚ(\sqrt[3]{2},i) / ℚ$ sería $γ = i + \sqrt[3]{2}$.

Con el fin de simplificar la notación representaremos $ω=\sqrt[3]{2}$
Tenemos varias maneras de comprobar que el elemento es primitivo, podemos utilizar el algoritmo de ir elevando a potencias y acabar calculando el polinomio irreducile y ver su grado, o escribir todas las inmersiones que es lo que vamos a utilizar. Lo mismo hicimos en \ref{tblGaloisT4-2}.


\begin{tabular}{r|c|c}
$\;$  & $i$ & α \\\hline
$σ_1$ & $i$ &  $α$  \\
$σ_2$ & $i$ & $ω_3α$  \\
$σ_3$ & $i$ & $ω_3^2α$\\\hline
$σ_4$ & $.i$ & $α$ \\
$σ_5$ & $-i$ & $ωα$\\
$σ_6$ & $-i$ & $ω^2α$
\end{tabular}

(Donde $ω$ serían números complejos que nos permiten obtener el resto de raíces del polinomio irreducible de α)

Ahora comprobamos que $σ_k (γ) ≠ σ_l(γ) \dimplies k≠l\; k,l=1,...,6$.


\end{problem}

\paragraph{Interesante:}

Las extensiones no tienen polinomios. Las extensiones tienen elementos y son éstos los que tienen polinomios.  Si ese elemento es primitivo, el grado del polinomio del elemento será el grado de la extensión.

%Ejercicio 4
\begin{problem}[6]
 \label{H1.E6}
 Construir un cuerpo con 9 elementos (como un cociente adecuado del anillo de polinomios $\mathbb{F}_3 [x]$ ).

\solution

$9=3² \implies ∃ \mathbb{F}_3 \subset K$

$\displaystyle \frac{\mathbb{F}_3[t]}{(p(t))}$ es un cuerpo si $p(t) \in \mathbb{F}_3[t]$ es irreducible.


Necesitamos un polinomio $p(t)$ con $deg(p(t)) = 2$ (que es el exponente al que hay que elevar el 3)

$p(t) = 1+t+t² ∈ \mathbb{F}_3$. ¿Este polinomio es irreducible? no, porque 1 es raiz ($\overline{1}+\overline{1}+\overline{1}=\overline{3} = \overline{0}$).

Vamos a buscar otro $p(t) = 1+t²$ ¿Es irreducible? Sí (basta prooverline los 3 números ($0,1,2$) del cuerpo para comprooverline que ninguno es raiz)

Entonces, $K = \frac{\mathbb{F}_3}{(t²+1)}$ es un cuerpo (por ser $p(t)$ irreducible). \footnote{Igual que $\displaystyle\frac{ℝ}{(x^2+1)} = ℂ$}

Tenemos la extensión $\mathbb{F} \subset \mathbb{F}_3[\overline{t}], \overline{t}∈ \frac{\mathbb{F}_3[t]}{t²+1}$.

El polinomio mínimo de $\overline{t}$ sobre $\mathbb{F}_3$ es $t²+1$ por ser el polinomio por el que estamos dividiendo.


$p(\overline{t}) = \overline{p(t)} = \overline{t²+1} = 0 \implies \overline{t}$ es raiz de $p(t)$.

Como $deg(p(t)) = 2 \implies $ el grado de la extensión $\mathbb{F}_3[\overline{t}]$ es 2 ($[\mathbb{F}_3(t):\mathbb{F}_3]=2$) y una base es $\{\overline{1},\overline{t}\}$ (no hace falta t² porque $\overline{t}^2 + \overline{1} = 0 \implies \overline{t}^2 = (-1)·1$).

Por tanto, todo elemento $x \in \mathbb{F}_3[\overline{t}]$ puede escribirse de la forma $x=a_1+a_2α$ donde $a_i \in \mathbb{F}_3$, por lo que podemos formar 3²=9 $x$ distintos
\end{problem}

\section{Hoja 3}

\begin{problem}[3]

\ppart Calcular el grado de la extensión $\mathbb{F}_3(x, y)/\mathbb{F}_3(x^3, y^3)$.

\ppart Encontrar todas la extensiones de la inclusión $\mathbb{F}_3 (x^3 , y^3) → \mathbb{F}_3(x, y)$.

\ppart Probar que esta extensión no tiene elementos primitivos.

\solution

Pequeño anexo interesante: $\mathbb{F}_9 ≠ ℤ_9$, $\frac{\mathbb{F}_3[x]}{x^2+1}$



\spart
$$[\mathbb{F}_3(x,y) : \mathbb{F}_3(x^3,y^3)] = 9 \implies \mathbb{F}_3 = \frac{ℤ}{(3)}$$


Lo calculamos haciendo cadenas de inclusiones:

$$\begin{array}{ccccc}\mathbb{F}_3(x^3,y^3) &\subset& \mathbb{F}_3(x^3,y^3)(x) &\subset& \mathbb{F}_3(x,y^3)(y)\\
&\downarrow& & \downarrow&\\
&p_1(t) = t^3-x^3& & p_2(t) = t^3-y^3&\end{array}$$

$p_1$, $p_2$ serán irreducibles sii no tienen raíces en $\mathbb{F}_3(x^3,y^3)$.

Esto es sencillo de comprobar puesto que sabemos que $t^3-x^3 = (t-x)^3$ en $\mathbb{F}_3(x,y)$, y este polinomio solo tiene una raíz, que es $x$.

Ahora la única posibilidad de que no fuese irreducible es que resulte que $x\in \mathbb{F}_3(x^3,y^3)$. En ese caso resultaría que la extensión tendría grado 1 pues estaríamos 'añadiendo' un elemento que ya está en el cuerpo.

Si $x$ perteneciera a $\mathbb{F}_3(x^3,y^3)$, entonces podríamos escribir:
\[x = \frac{\sum a_{ij}(x^3)^i(y^3)^j}{\sum b_{ij}(x^3)^i(y^3)^j}; a_{ij},b_{ij} ∈ \mathbb{F}_3\]
\[\dimplies \sum b_{ij} x^{3i+1}y^{3j} = \sum a_{ij} x^{3i}y^{3j}\]

Para que estos dos polinomios sean iguales debemos igualar sus coeficientes uno a uno. Para cada $j$ el coeficiente de $y^{3j}$ debería ser igual a un lado y al otro de la igualdad, es decir:
\[b_{ij}x^{3i+1}=a_{ij}x^{3i}\]
Y vemos fácilmente que para $i$ que fijemos no existen unas constantes $a,b$ que satisfagan la ecuación.

\spart Esto es paradigmático. Este es un buen ejemplo de algo que veremos más adelante.

En la cadena:
$$\begin{array}{ccccc}\mathbb{F}_3(x^3,y^3) &\subset& \mathbb{F}_3(x^3,y^3)(x) &\subset& \mathbb{F}_3(x,y^3)(y)\\
\end{array}$$

las únicas extensiones posibles en ambas inclusiones son la identidad, ya que por fuerza, $σ(x) = x$ porque $σ$ manda raices de polinomios en raices de polinomios.


Podemos concluir que esta extenión es normal, pero no es separable. Se deja como ejercicio para el lector la comprobación de estas afirmaciones.

\spart Se trata de un ejercicio de examen salvo por un pequeño detalle, y es que en el examen daban una pista extra:
\[γ∈\mathbb{F}_3(x,y) \implies γ^3 ∈\mathbb{F}_3(x^3,y^3)\]

Supongamos que esto es verdad. Entonces:

\[[\mathbb{F}_3(x^3,y^3)(γ) : \mathbb{F}_3(x^3,y^3)] ≤ 3\]

porque $γ$ satisface el polinomio $t^3 -γ^3 ∈ \mathbb{F}_3(x^3,y^3)[t]$

\[γ = \frac{\sum a_{ij}(x^3)^i(y^3)^j}{\sum b_{ij}(x^3)^i(y^3)^j}; a_{ij},b_{ij} ∈ \mathbb{F}_3\]

Vamos a demostrar la pista que, en el examen, nos daba el enunciado.

Para ello, fjémonos en el valor de $γ^3$. Como el cuerpo en el que nos movemos tiene $Ch(\mathbb{F}) = 3$ se cumple que $(a+b)^{3} = a^3 + b^3$.

Por tanto:
\[γ^3 = \frac{\sum a_{ij}^3(x^3)^i(y^3)^i}{\sum b_{ij}^3(x^3)^i(y^3)^i}; a_{ij},b_{ij} ∈ \mathbb{F}_3 \implies γ^3 ∈ \mathbb{F}_3 \implies [\mathbb{F}_3(x^3,y^3)(γ) : \mathbb{F}_3(x^3,y^3)] ≤ 3\]

\paragraph{Nota:} Esta respuesta estaría también bien:
$$γ^3 = \frac{\sum a_{ij}(x^3)^i(y^3)^i}{\sum b_{ij}(x^3)^i(y^3)^i}; a_{ij},b_{ij} ∈ \mathbb{F}_3 \implies γ^3 ∈ \mathbb{F}_3 \implies [... : ...] ≤ 3$$

Es exactamente la misma respuesta solo que aplicamos el pequeño teorema de Fermat (que dice que en $Ch(\mathbb{F}_p) \implies a^p = a, ∀a∈\mathbb{F}_p$)


\paragraph{Conclusión:} No tiene elementos primitivos, por que añadiendo el $γ$ que añada, siempre el grado de la extensión va a ser $≤3$. Si $γ$ fuera primitivo, tendríamos que tener que el grado de esa extensión fuera 9 (por definición de elemento primitivo de una extensión \ref{DefElemPrimitivo}).

\end{problem}


\begin{problem}[Parcial 2]

Demostrar que $K= \rac(ω_n = e^{\frac{2πi}{n}})$ es una extensión galoisianada de grado 2 de $E=ℚ(ω_n+ω_{n}^{-1})$

\solution
Si nos fijamos un poco podemos ver sencillamente que si cojo un polinomio de la forma:
\[p(x)=x-(ω_n+ω^{-1}_n)\]
al evaluarlo en $ω_n$ nos quedará $ω_n^{-1}$. Por tanto si cogemos el polinomio
\[p(x)=(x-(ω_n+ω^{-1}_n))x\]
al evaluarlo en $ω_n$ obtendremos un 1. Así, el candidato a polinomio irreducible sería:
\[p(x)=(x-(ω_n+ω^{-1}_n))x -1\]
%Debemos darnos cuenta de que $ℚ(ω_n) =  ℚ(ω_n+ω_{n}^{-1})(ω_n)$.


%Como tenemos grado 2 y $ω_n$ es raíz, entonces: $$p(t) = (t-ω_n) \cdot \underbrace{(...)}_{\text{Grado 1}}$$

%$$p(t) = (t-ω_n)(t-a) ∈ℚ(ω_n + ω_n^{-1})$$
%$$p(t) = t^2 - (ω_n +a)t + ω_na ∈ℚ(ω_n + ω_n^{-1})$$

%Es lógico probar con $a = ω_n^{-1}$, para que los coeficientes del polinomio pertenezcan al ¿cuerpo?.

Con esto hemos visto que $[ℚ(ω_n) : ℚ(ω_n+ω_{n}^{-1})] ≤ 2$, ya que hemos encontrado un polinomio de grado 2, que no sabemos si es irreducible en el cuerpo pequeño.

Si fuese de grado 1 entonces los 2 cuerpos serían iguales y eso no puede ser porque $ω_n = cos\left(\frac{2π}{n}\right) + i sen (\frac{2π}{n})$. Si calculamos $\bar{ω_n} = cos(\frac{2π}{n}) + isen(\frac{2π}{n})$ tenemos que $ω_n + ω_n^{-1} = ω_n+\bar{ω_n} = 2cos(\frac{2π}{n}) ∉ ℝ$ y $ℚ(ω_n + ω_n^{-1}) \subset ℝ$.


\end{problem}

\section{Hoja 4}

\begin{problem}[4]
$$K = ℚ(ω), ω = e^{\frac{2πi}{7}}$$

Vamos a estudiar la teoría de Galois de esta extensión.

\solution

1) $[K:ℚ] = deg(p(t))$, siendo $p(t)$ el polinomio mínimo de $ω$ sobre $ℚ$.

Sabemos que si tenemos $h(t) = t^7 -1$, $h(ω) = 0$, es decir, $ω≠1$ es raíz de $h(t)$.

Vamos a contruir el polinomio irreducible $h(t) = t^7 - 1 = (t-1)\underbrace{(t^6 + t^5 + t^4 + t^3 + t^2 + t + 1)}_{p(t)}$.

Este polinomio $p(t)$ ya es irreducible sobre $ℚ$ porque si tomamos el isomorfismo de anillos $\appl{σ}{ℚ[t]}{ℚ[t]}$ tal que $ σ(p(t)) = p(t+1)$ tenemos que $p(t+1)$ es irreducible por Einsenstein. Por tratarse de un isomorfismo un elemento será irreducible sii lo es su imagen, por lo que concluimos que $p(t)$ es irreducible. (Ya aplicamos este criterio con más detalle en otros ejercicos).

Concluimos por lo tanto $[K:ℚ] = 6$.

1.2) $ℚ(ω) / ℚ$ es galoisiana, porque es el cuerpo de descomposición de $h(t)$ lo que nos asegura normalidad en la extensión. Sabemos que es separable por $Char(K)=0$.

2) Vamos a calcular el grupo de galois: $Gal(K/ℚ)$.

$$\begin{array}{c|c}
G & ω\\\hline
σ_1 & ω\\
σ_2 & ω^2\\
\vdots&\vdots\\
σ_6 & ω^6
\end{array}$$

El orden del grupo $G = Gal(K/ℚ)$ es 6, por ser 6 el grado de la extensión.


Vamos a estudiar un poco cómo son los elementos de este grupo de 6 elementos. Es interesante saber si es isomorfo a $ℤ_6$ o a $ℤ_2×ℤ_3$. Para ello vamos a estudiar el orden de los elementos. Si hay algún elemento de orden 6 tendrá que ser isomorfo a $ℤ_6$, y sino a $ℤ_2×ℤ_3$ (no hay más posibilidades).

$$\begin{array}{c|c|c}
G & ω & \text{orden}\\\hline
σ_1 & ω & 1\\
σ_2 & ω^2 & 3\\
σ_3 & ω^3 & 6\\
σ_4 & ω^4 & ¿2?\\
σ_5 & ω^5 & ¿3?\\
σ_6 & ω^6 & 2
\end{array}$$

Para calcular el orden de $σ_2$:
$$σ_2^2(ω) = σ_2(σ_2(ω)) = ω^4$$
$$σ_2^3 (ω) = σ_2(σ_2^2(ω)) = σ_2(ω^4) = ω^8 = 1$$

Se deja como ejercicio para el lector la comprobación. 

\textbf{Útil:} Solo hay 4 posibilidades para el orden del elemento: $1,2,3,6$, porque tienen que dividir al orden del grupo. Es interesante también darnos cuenta de que:
\[1 = σ_3^6 = (\underbrace{σ_3^2}_{σ_2})^3 = (\underbrace{σ_3^3}_{σ_6})^2 = 1 \implies ord(σ_6) = 2\]

\paragraph{Subgrupos/cuerpos fijos} Hay tantos cuerpos fijos como subgrupos. 

El cuerpo fijo correspondiente a $σ_1=id$ es $K$.

El cuerpo fijo correspondiente a $K^{\gen{σ_3}}$ es $ℚ$ 

Vamos a comprobarlo (por puro amor a las cuentas): $$K^{\gen{σ_3}} = \{x∈K \tq σ_3(x) = x\}$$

Podríamos utilizar la base $\mathcal{B} = \{1,ω,ω^2,...,ω^5\}$. Para este caso es más conveniente utilizar la base $\mathcal{B} = \{ω,ω^2,...,ω^5,ω^6\}$, teniendo entonces  $$K^{\gen{σ_3}} = \{x = a_0 ω + a_1ω^2 + a_2ω^3+...+a_5ω^6\quad x∈K \tq σ_3(x) = x\}$$

Aplicamos $σ_3(x) = σ_3(a_0 ω + a_1ω^2 + a_2ω^3+...+a_5ω^6) = σ_3(...) = ... = ... $

Resolvemos el sistema de ecuaciones $σ(x) = x$.

\[
σ_3(x) = x \dimplies \left\{
\begin{array}{cc}
a_1 = a_3 = a_2 = a_6 = a_4 = a_5
\end{array}
\right.
\]

Entonces: $$K^{σ_3} = \{ x = q(ω+ω^2 + ... + ω^6), q∈ℚ\} = \{x = -a, a∈ℚ\} = ℚ$$.

$$\textbf{Subgrupos de } G = \left\{
\begin{array}{c}
\gen{id}\\\gen{σ_3^3=σ_6}\\\gen{σ_3^2 = σ_2}\\G=\gen{σ_3}
\end{array}
\right\} \dimplies \left\{
\begin{array}{c}
K^{\gen{id}} = K\\
K^{\gen{σ_6}} = (1) \\
K^{\gen{σ_2}} = (2) \\
K^{\gen{σ_3}} = ℚ 
\end{array}
\right\}$$

Vamos a calcular (1) y (2):

Tomamos: $x = a_1 ω + a_2 ω^2 + a_3ω^3 + a_4ω^4 + a_5ω^5+a_6ω^6$


$$σ_6(x) = a_1ω^6 + a_2ω^5 + a_3ω^4 + a_4ω^3 + a_5ω^2 + a_6ω$$
$$σ_6(x) = x\dimplies \left\{
\begin{array}{c}
a_1 = a_6\\a_2=a_5\\a_3=a_4
\end{array}
\right. \dimplies x = a_1(ω+ω^6) + a_2(ω^2 + ω^5) + a_3(ω^3+ω^4)$$

Por la teoría de Galois, sabíamos antes de empezar, que el grado de la extensión $[K^{\gen{σ_6}}:ℚ] = \frac{6}{2} = 3$, que es lo que hemos obtenido.

Otra cosa que nos dice el teorema, es que existe un elemento primitivo. En este caso lo lógico es apostar porque sea $(ω^2+ω^5)$, es decir, ¿se cumple $K^{\gen{σ_6}} = ℚ(ω^2 + ω^5)$? Si este no lo fuera, probaríamos con el resto.

Sea $γ = ω^2 + ω^5$. Si $γ∈ℚ(γ) \implies γ^2 ∈ℚ(γ)$. En este caso, $γ^2 = ω^4 + ω^3 + 2\implies (ω^3+ω^4) ∈ℚ(γ)$.

Si $γ∈ℚ(γ) \implies γ^3 ∈ℚ(γ)$. En este caso, $γ^3 = ω^6 + 3ω^2 + 3ω^5 + ω \implies (ω+ω^6)∈ℚ(γ)$.

Se propone al lector, para afianzar los conocimientos de la materia, calcular el polinomio mínimo de esta extensión.

\end{problem}