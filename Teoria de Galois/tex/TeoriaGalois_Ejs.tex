% -*- root: ../TeoriaGalois.tex -*-
\section{Hoja 1}

\begin{problem} Encuentra el grado de las siguientes extensiones.
\solution
\begin{enumerate}
\item $[ℚ(\sqrt{2}) : ℚ ]= 2$.
\item $[ℚ(\sqrt{6}) : ℚ ]= 2$ con $p(t) = t^2 - 6$: su solución es $\sqrt{6}$ que no está en $ℚ$. El criterio de Eisenstein también funciona con $p=2$: $p$ divide al término independiente pero no al de mayor grado.
\item $[ℚ(\sqrt{16}) : ℚ ]= 1$ con $p(t) = t-4$.
\item $[ℚ(\sqrt{p}) : ℚ] = 2$ con $p$ primo, siendo $p(t) = t^2 - p$.
\item $[ℚ\left(e^{\frac{2πi}{n}}\right) : ℚ]$ con $n=2,3,4,5,6,8$.
\begin{enumerate}
\item Para $n=2$, $e^{\frac{2πi}{n}} = -1$ y $[ℚ(-1):ℚ] =  1$.

\item Para $n=3$, el polinomio que lo anula es $p(t) = t^3 - 1$, pero no es irreducible porque tiene como raíz $1$, y entonces $p(t) = (t-1) (t^2 + t + 1)$. Luego $e^{\frac{2πi}{3}}$ es raíz de $t^2+t+1$, que tiene que ser irreducible (si una raíz,$e^{\frac{2πi}{n}}$, es compleja, la otra es su conjugado, que también es complejo). $[ℚ\left(e^{\frac{2πi}{3}}\right) : ℚ] = 2$.

\item Para $n=4$, $e^{\frac{2πi}{4}} = i$, luego $[ℚ(i) : ℚ ] = 2$ con $p(t) = t^2 + 1$. Si no fuese $p(t)$ irreducible, tendríamos de nuevo que $[ℚ(i) : ℚ]=1$ y $ℚ(i)$ sería igual a $ℚ$, lo cual es absurdo porque $i$ no está en $ℚ$.

\item Para $n=5$, está claro que $h(t) = t^5 - 1$ anula a $e^{\frac{2πi}{5}}$, luego el grado de la extensión es menor o igual que 5. Desde luego no es irreducible porque 1 es raíz $h(t) = (t-1)(t^4 + t^3 + t^2 + t + 1)$. $e^{\frac{2πi}{5}}$ debe ser raíz de $t^4 + t^3 + t^2 + t + 1$. ¿Es este polinomio irreducible? Desarrollaremos un método más general para encontrar los polinomios irreducibles de las raíces de la unidad, pero para este nos quedamos con una sugerencia: sí es irreducible y $[ℚ\left(e^{\frac{2πi}{5}}\right): ℚ ] = 4$. Sólo hay que estudiar la aplicación $ℚ[t] \mapsto ℚ[t]$ que manda $t$ a $t+1$ y por lo tanto $\sum a_i t^i \mapsto \sum a_i (t+1)^i$, que si es un isomorfismo de anillos entonces $p(t)$ es irreducible si y sólo si $p(t+1)$ lo es. En este caso podríamos usar el criterio de Eisenstein para probarlo. Esto se resolvi\'o en un ejemplo de teor\'ia (\ref{Teoria_H1.E1.A5.S_Af})

\item Para $n=6$, $h(t) = t^6 - 1$ lo anula, pero podemos descomponerlo como $h(t) = (t^3 - 1)(t^3 + 1)$. $e^{\frac{2πi}{6}}$ es raíz sólo de $t^3 + 1$, pero este polinomio tampoco es irreducible: $t^3 + 1 = (t-1)(t^2 - t + 1)$. $e^{\frac{2πi}{6}}$ es raíz de $t^2-t+1$, que por el mismo argumento que en el caso $n=4$ tiene que ser irreducible, o si no un número complejo estaría en $ℚ$.

\item Por último, cuando $n=8$, sabemos que $h(t) = t^8 - 1$ anula a $e^{\frac{2πi}{8}}$. Pero descomponiendo de nuevo, $h(t) = (t^4 -1)(t^4 +1)$, $e^{\frac{2πi}{8}}$ es raíz de $t^4 +1$, el grado de la extensión es menor o igual que cuatro. Por suerte, $t^4 + 1$ es irreducible: las raíces deberían ser enteras y dividir a $1$, y ni $1$ ni $-1$ son raíces.
\end{enumerate}
\end{enumerate}
\end{problem}


\begin{problem}Encuentra el grado de las siguientes extensiones

\solution

\begin{itemize}

\item $[\rac(\sqrt{6},i):\rac]$.

 Para resolver estos ejercicios, el procedimiento es escribirlo desmenuzadas las inclusiones.

$\rac \to \rac(\sqrt{6}) \to \rac(\sqrt{6})(i) = \rac(\sqrt{6},i)$

Y vamos caso por caso:

\begin{itemize}
\item $[\rac: \rac(\sqrt{6})] = 2$
\item $[\rac(i): \rac(\sqrt{6}) ] = 2 (i \notin \rac(\sqrt{6}))$
\end{itemize}

Entonces: $[\rac(\sqrt{6},i):\rac]= 2\cdot 2 = 4$.

\item $[\rac(\sqrt[n]{p},\sqrt{-m}):\rac] = 2n$. Se resuelve generalizando la misma idea del anterior.

\item $[ℚ(\sqrt[3]{2},e^{\frac{2 \pi i}{3}}\cdot \sqrt[3]{2}):ℚ]$

Lo primero que nos tenemos que dar cuenta es de que \\
 $ℚ(\sqrt[3]{2},e^{\frac{2 \pi i}{3}}\cdot \sqrt[3]{2}) = ℚ(\sqrt[3]{2},e^{\frac{2 \pi i}{3}})$

Ahora escribimos la cadena de inclusiones:

\begin{gather*}
[ℚ : ℚ(\sqrt[3]{2})] = 3, p(t)=t³-2\\
[ℚ(\sqrt[3]{2}) : ℚ(\sqrt[3]{2})(e^{\frac{2\pi i}{3}})]  [ℚ(e^{\frac{2\pi i}{3}})] = 2; p(t) =  t²+t+1\\
[ℚ(\sqrt[3]{2}) : ℚ(\sqrt[3]{2})(e^{\frac{2\pi i}{3}})] = 2; e^{\frac{2\pi i}{3}}∉ℚ(\sqrt[3]{2})
\end{gather*}

Entonces deducimos que $[ℚ(\sqrt[3]{2},e^{\frac{2 \pi i}{3}}\cdot \sqrt[3]{2}):ℚ] = 2·3 = 6$
\end{itemize}

\end{problem}


\begin{problem}[3]
Sean $E₁$ y $E₂$ dos extensiones distintas de un cuerpo $F$ de grados $p$ y $p'$ respectivamente, donde $p$ y $p'$ son
primos distintos. Demostrar que $E_1 ∩ E_2 = F$.

\solution

Está claro que $E_1 ∩ E_2 ⊆ E_1$, así que podemos considerar la cadena de extensiones \[ E_1 \hookrightarrow E_1 ∩ E_2 \hookrightarrow F \]

Entonces, $p = [E_1 : F] = [E_1 : E_1 ∩ E_2] · [E_1 ∩ E_2 : F]$. Como $p$ es primo, sólo puede ser $E_1 = E_1 ∩ E_2$ o bien $E_1 ∩ E_2 = F$. Lo primero no puede ocurrir: si $E_1 ∩ E_2 = E_1$ entonces $E_1 ⊆ E_2$, así que tendríamos $[E_2:F] = p' = [E_2 : E_1] · [E_1 : F]$, imposible porque $p'$ es primo.

\end{problem}

\begin{problem}[4]
Sea $K$ un cuerpo finito. Demostrar que $\card{K} = p^m$ para algún primo $p$.

Demuestra que además, ese $p$ es la característica $\mop{char}(K)$ de $K$.
\solution

Lo primero es recordar la \concept[Característica]{característica}: el número de veces que tienes que sumar la unidad para llegar al 0 en cuerpos finitos.

Por ser $K$ un cuerpo finito, podemos definir el siguiente homomorfismo φ de anillos que nos lleve de $ℤ$ al cuerpo $K$:

\begin{align*}
ℤ &\longmapsto K \\
1 &\longmapsto 1_K \\
α &\longmapsto α·1_K = 1_K + \dotsb + 1_K
\end{align*}

Estudiemos el núcleo de φ. Son los elementos de $ℤ$ cuya imagen es la unidad en $K$, esto es, tomando $n = \mop{char}(K)$, \[ \ker φ = \set{a·n \tq a ∈ ℤ} = nℤ \]

El segundo teorema de isomorfía de grupos\footnote{Ver magníficos apuntes de Pedro Valero de Estructuras Algebraicas.} nos dice que existe un isomorfismo entre la imagen de $φ$ y el grupo cociente $\quot{ℤ}{\ker φ}$, al que llamaremos $γ$:

\[ \appl{γ}{\quot{ℤ}{nℤ} = \fd_n}{\img φ ⊆ K} \]

Afirmamos que $n≠0$: $K$ es finito y $\quot{ℤ}{0ℤ} = ℤ$, así que no puede existir un isomorfismo entre ambos.

Afirmamos que $n$ es primo, y vamos a demostrarlo por reducción al absurdo. Supongamos $n = ab$. Si operamos en $\fd_n$, tenemos que $n\equiv 0 \mod n$ y entonces $ab=0$. Aplicamos $γ$ a ambos lados y tendríamos que $γ(a)·γ(b) = γ(0) = 0_K$, contradicción ($K$ es un cuerpo así que no puede tener divisores de $0$).

Sea $p=n$, entonces \[ \mathbb{F}_p = \quot{ℤ}{nℤ}  \simeq \img φ \] es un subcuerpo de $K$, luego $K$ es un espacio vectorial sobre $\mathbb{F}_p$. Es decir, que si $[K:\mathbb{F}_p] = m$, entonces $\{x₁, \dotsc, x_m\}$ es base de $K$ sobre $\mathbb{F}_p$, luego \[ \card{K} = \card{\fd_p}^{[K:\fd_p]} = p^m\] para $m ≥ 1$, y ya tenemos lo que buscábamos.

Una observación importante: el cuerpo finito de $n$ elementos \textbf{no es} $\quot{ℤ}{nℤ}$. Los enteros módulo $n$ sólo son un cuerpo cuando $n$ es primo. Es decir, que en general \[ \fd_n ≠ \quot{ℤ}{nℤ} \].

Los cuerpos finitos hay que construirlos específicamente como el cociente de un cuerpo finito de orden primo con el ideal generado por un polinomio irreducile. Por ejemplo, podríamos preguntarnos si hay un cuerpo con 9 elementos. Existe, porque $9=3^2$.

El problema 4 de la hoja 2 (que incluye al 6 de la hoja 1) va precisamente de construir este cuerpo.
\end{problem}


%%%%%%%%%% Malvado 2-10-2014 %%%%%%%%%%%%%
\section{Hoja 2}

% Ejercicio 1
\begin{problem}[1] Inmersiones de $ℚ(\alpha)$ en $ℂ$
\solution
\begin{enumerate}
	\item $\alpha \eq \sqrt{2}$.\\
	{\inputtikz{ejercicios/img1-ej1-h2}} \\
	$\sigma(\sqrt{2}) \eq \sqrt{2}  \Rightarrow \sigma_1 \eq Id$ \\
	$\sigma_2(\sqrt{2}) \eq -\sqrt{2}$

	\item $\alpha \eq i \eq \sqrt{2}$ \\
	{\inputtikz{ejercicios/img2-ej1-h2}} \\
	$p(t) = t^2 + 1$\\
	$\sigma_1(i) \eq i \Rightarrow \sigma_1 \eq Id$\\
	$\sigma_2(i) \eq -i \Rightarrow \sigma_2 \eq$ conjugación compleja\\
	$\sigma_2(a+bi) \eq a + b(-i) = a + bi$

	\item $\alpha \eq \sqrt[3]{2}$\\
	{\inputtikz{ejercicios/img3-ej1-h2}} \\
	polinomio mínimo de $\sqrt[3]{2}$ es $p(t) \eq t^3 - 2$\\
	raíces $\sqrt[3]{2}, e^{\frac{2πi}{3}}\sqrt[3]{2}, e^{\frac{4πi}{3}}\sqrt[3]{2}$\\
	$\sigma_1(\sqrt[3]{2}) \eq \sqrt[3]{2} \implies \sigma_1 \eq Id$\\
	$\sigma_2(\sqrt[3]{2}) \eq e^{\frac{2πi}{3}}\sqrt[3]{2}$\\
	$\sigma_3(\sqrt[3]{2}) \eq e^{\frac{4πi}{3}}\sqrt[3]{2}$

	E.G $x \eq 7 + 2\sqrt[3]{2} + 5(\sqrt[3]{2})^2 \in ℚ(\sqrt[3]{2})$\\
	$\sigma_2(x) = 7 + 2e^{\frac{2πi}{3}}\sqrt[3]{2} + 5e^{\frac{4πi}{3}}(\sqrt[3]{2})^2$

	¿Cuáles son entonces?
	$\sigma_2(ℚ(\sqrt[3]{2})) \eq ℚ(e^{\frac{2πi}{3}}\sqrt[3]{2}) \neq ℚ(\sqrt[3]{2})$ con $\sigma_2(ℚ(\sqrt[3]{2})) \subset \real$
\end{enumerate}
\end{problem}


% Ejercicio 2
\begin{problem}[2]

\ppart Escribir todos los automorfismos de $ℚ(e^{\frac{2πi}{8}})$ que extienden el automorfismo $\sigma: ℚ(i) \rightarrow ℚ(i)$ que envía $i$ a $-i$.

\ppart Escribir todos los automorfismos de $ℚ(e^{\frac{2πi}{8}})$.

\solution

\spart

\begin{wrapfigure}{R}{0.4\textwidth}
\centering
\inputtikz{ejercicios/img1-ej2-h2}
\caption{Esquema de las inmersiones que buscamos, con $σ$ tal que $σ(i) = -i$.}
\end{wrapfigure}

Lo primero que buscamos es el polinomio irreducible de $e^{\frac{2πi}{8}}$ sobre $ℚ(i)$. Vemos que \[ (t^4 + 1)(t^4 - 1) = t^8 - 1 \] anula a $e^{\frac{2πi}{8}}$.

$t^4 + 1$ era el polinomio irreducible de $e^{\frac{2πi}{8}}$ sobre $ℚ$; y como sobre $ℚ(i)$ $t^4 + 1 \eq (t^2 - i)(t^2 + i)$, el polinomio irreducible de $e^{\frac{2πi}{8}}$ sobre $ℚ(i)$ es $p(t) = t^2 - i$, pues \[ \left(e^{\frac{2πi}{8}}\right)^2 \eq
e^{\frac{πi}{2}} \eq i \]

El polinomio mínimo es de grado dos, así que tenemos que encontrar dos extensiones, que serán

\begin{align*}
	\sigma_1(e^{\frac{2πi}{8}}) &\eq w_1 \neq e^{\frac{2πi}{8}} \\
	\sigma_2(e^{\frac{2πi}{8}}) &\eq w_2 \neq e^{\frac{2πi}{8}} \\
\end{align*}
donde $w_1$ y $w_2$ son raíces de $p^{\sigma}(t) \eq t^2 + i$.

\spart

\begin{wrapfigure}{R}{0.4\textwidth}
\centering
\begin{tabular}{r|c|c}
$\;$  & $i$ & α 	\\ \hline
$σ_1$ & $i$ & $α$  	\\
$σ_2$ & $i$ & $-α$  \\ \hline
$σ_3$ & $-i$ & $α$ 	\\
$σ_4$ & $-i$ & $-α$	\\
\end{tabular}
\caption{Automorfismos de $ℚ(α)$.}
\label{tblH2E2}
\end{wrapfigure}

Los automorfismos deben dejar fijos los elementos de $ℚ$ y llevar raíces del polinomio mínimo de $e^{\frac{2πi}{8}} = α$ a otras raíces del mismo polinomio. En el apartado anterior habíamos visto que este polinomio era $t^4 + 1 = (t^2 - i)(t^2 + i)$, así que los automorfismos se caracterizarán por la imagen de $i$ y $α$. Los tenemos definidos en la tabla lateral \ref{tblH2E2}.

\end{problem}

\begin{problem}[3] Encontrar todas las inmersionces de $ℚ(α,β)$ en $ℚ$ donde

\ppart $α=\sqrt{2},\; β=\sqrt{3}$.
\ppart $α = \sqrt{6},\; β = i = \sqrt{-1}$.
\ppart $α = \sqrt[3]{2},\; β = i$.

\solution

\spart
\spart
\spart

\begin{wrapfigure}{R}{0.4\textwidth}
\centering
\begin{tabular}{r|c|c}
$\;$  & $i$ & α \\\hline
$σ_1$ & $i$ &  $α$  \\
$σ_2$ & $i$ & $ω_3α$  \\
$σ_3$ & $i$ & $ω_3^2α$\\\hline
$σ_4$ & $.i$ & $α$ \\
$σ_5$ & $-i$ & $ωα$\\
$σ_6$ & $-i$ & $ω^2α$
\end{tabular}
\caption{Tabla de inmersiones de $ℚ\left(\sqrt[3]{2}, i\right)$ en $ℚ$.}
\label{tblH2E3}
\end{wrapfigure}

Es fácil de comprobar que \[ [ℚ(\sqrt[3]{2},i):ℚ] = 6 \]

Ahora vamos a añadirle algo de complejidad/interés al problema calculando el elemento primitivo de esta extensión.

Un elemento primitivo de $ℚ(\sqrt[3]{2},i) / ℚ$ sería $γ = i + \sqrt[3]{2}$.
Con el fin de simplificar la notación representaremos $ω=\sqrt[3]{2}$
Tenemos varias maneras de comprobar que el elemento es primitivo, podemos utilizar el algoritmo de ir elevando a potencias y acabar calculando el polinomio irreducile y ver su grado, o escribir todas las inmersiones que es lo que vamos a utilizar (tabla \ref{tblH2E3}). Lo mismo hicimos en \ref{tblGaloisT4-2}.

(Donde $ω$ serían números complejos que nos permiten obtener el resto de raíces del polinomio irreducible de α)

Ahora comprobamos que $σ_k (γ) ≠ σ_l(γ) \dimplies k≠l\; k,l=1,\dotsc,6$.

\end{problem}

\paragraph{Interesante:}

Las extensiones no tienen polinomios. Las extensiones tienen elementos y son éstos los que tienen polinomios.  Si ese elemento es primitivo, el grado del polinomio del elemento será el grado de la extensión.

%Ejercicio 4
\begin{problem}[4]
 \label{H1.E6}
 Construir un cuerpo con 9 elementos (como un cociente adecuado del anillo de polinomios $\mathbb{F}_3 [x]$ ).

\solution

Como veíamos en el ejercicio 4 de la hoja 1, como $9=3^2$, entonces existe un cuerpo $K$ de nueve elementos que contiene a $\fd_3$.

Sabemos que $\quot{\mathbb{F}_3[t]}{(p(t))}$ es un cuerpo si $p(t) \in \mathbb{F}_3[t]$ es irreducible. Además, es una extensión de grado $\deg p(t)$ sobre $\fd_3$.

Por lo tanto, necesitamos un polinomio $p(t)$ con $\deg p(t) = 2$ para que $\card{K} = 3^2 = 9$.

Tomemos por ejemplo $p(t) = 1+t+t^2 ∈ \mathbb{F}_3$. ¿Este polinomio es irreducible? no, porque 1 es raíz: $\overline{1}+\overline{1}+\overline{1}^2 = \overline{3} = \overline{0}$.

Vamos a buscar otro: sea $p(t) = 1+t^2$. Esta vez sí es irreducible: operando con los tres elementos de $\fd_3$ tenemos que $p(0) = 1,\, p(1) = 2,\, p(2) = 2$, ninguno es raíz.

Entonces \[ K = \quot{\mathbb{F}_3[t]}{(t^2+1)} \] es un cuerpo por ser $p(t)$ irreducible. \footnote{Igual que $\displaystyle\quot{ℝ}{(x^2+1)} = ℂ$ es un cuerpo, por ejemplo}

Sea entonces $α$ un símbolo con la propiedad $α^2 = -1$. Su polinomio mínimo sobre $\fd_3[t]$ es $p(t) = t^2 + 1$. Como el grado de este polinomio es dos, el grado de la extensión $[\fd_3(α) : \fd_3]$ será $2$, luego $\card{\fd_3 (α)} = 3^2 = 9$, el cuerpo que buscábamos.

¿Cómo es $K$? Es un anillo cociente, luego es el conjunto de todas las clases de equivalencia \[ [f(t)] = \set{f(t) + g(t) · p(t) \tq g(t) ∈ \fd_3[t]}\] con $f(t) ∈ \fd_3[t]$. Dicho de otra forma, igual que considerábamos la clase de equivalencia de $a$ en $ℤ_n$ como todos los enteros cuyo resto al ser divididos por $n$ es $a$, aquí la clase de equivalencia de $f(t)$ son todos los polinomios con coeficientes en $\fd_3$ cuyo resto al dividir por $p(t)$ es $f(t)$.

Podemos simplificar la construcción anterior para hacerla más manejable. Como $p(α) = 0$, las clases de equivalencia se caracterizan por su resto al dividirlas por $p(t)$, ya que tenemos que $f(α) + g(α) · p(α) = f(α) \; ∀g(t) ∈ \fd_3[t]$.

Así, podemos decir que los elementos de $K$ son los restos de dividir polinomios de $\fd_3 [t]$ entre $p(t)$. Es decir, serán elementos de la forma \[ x = a + b α\] con $a,b ∈ \fd_3$. Precisamente así es muy fácil ver que, como podemos escoger $a$ y $b$ de tres formas distintas cada uno, tenemos en total $3^2 = 9$ elementos.
\end{problem}

\section{Hoja 3}

\begin{problem}[1] Encontrar un elemento primitivo y su correspondiente polinomio irreducible de la extensión $\quot{ℚ(α,β)}{ℚ}$ donde
\ppart $α = \sqrt{2},\; β = i$.
\ppart $α = \sqrt{2},\; β = \sqrt{3}$.
\ppart $α = \sqrt{2},\; β = 7 + \sqrt{2}$.
\solution

\spart

El elemento primitivo puede ser $\sqrt{2} + i$, ya que podemos operar y ver que obtenemos $\sqrt{2}$ e $i$ (ejemplo de esto en la definición de elemento primitivo, \ref{DefElemPrimitivo}). Su polinomio mínimo será \[ p(t) = 9 - 2t^2 + t^4\]

\spart

Este vamos a hacerlo completo. Supongamos que el elemento primitivo es $γ = \sqrt{2} + \sqrt{3} = α + β$. Primero tenemos que ver que $ℚ(γ) = ℚ(α,β)$, y para esto tenemos que ver que $α, β ∈ ℚ(γ)$. Empezamos obteniendo las potencias de $γ$, que nos servirán más tarde para el polinomio mínimo:

\begin{align*}
γ &= α + β \\
γ^2 &= 5 + 2αβ \\
γ^3 &= 11α + 9β \\
γ^4 &= 49 + 20αβ
\end{align*}

Si operamos, tenemos que $γ^3 - 9γ = 2α$. Como $2∈ℚ$, tiene que ser $α ∈ ℚ(γ)$, y con esto podemos sacar que $β ∈ ℚ(γ)$.

Ahora falta ver el polinomio mínimo. Buscamos $p(t) ∈ ℚ[t]$ tal que $p(γ) = 0$, esto es \[ 0 = a γ^4 + b γ^3 + c γ^2 + d γ + e \]

Si sustituimos las potencias que hemos calculado antes y agrupamos por los elementos de la base de $ℚ(γ)$ (es decir, $\set{1, α, β, αβ}$) tenemos que

\begin{align*}
0 &= a 49 + a20αβ + b11α + b9β + c5 + c2αb + dα + dβ + e = \\
&= (e + 49a + 5c)·1 + (11b + d)·α + (9b + d)·β + (20a + 2c) ·αβ = 0
\end{align*}

Cada uno de los factores tiene que ser cero, así que resolvemos el sistema

\begin{align*}
0 &= 49a + 5c + e \\
0 &= 11b + d \\
0 &= 9b + d \\
0 &= 20a + 2c
\end{align*}

De la segunda y tercera ecuaciones sacamos que $b = d = 0$. De la última, que $c = -10a$, y sustituyendo con esto en la primera, que $e = a$. Como buscamos que $p(t)$ sea mónico, tomamos $a = 1$ y entonces nos queda

\[ p(t) = t^4 - 10t^2 + 1 \]

\spart

Dado que $ℚ(\sqrt{2}) = ℚ(7 + \sqrt{2})$ al estar $7 ∈ ℚ$, el elemento primitivo es $\sqrt{2}$ con polinomio mínimo $p(t) = t^2 - 2$.
\end{problem}

\begin{problem}[3]

\ppart Calcular el grado de la extensión $\mathbb{F}_3(x, y)/\mathbb{F}_3(x^3, y^3)$.

\ppart Encontrar todas la extensiones de la inclusión $\mathbb{F}_3 (x^3 , y^3) \hookrightarrow \mathbb{F}_3(x, y)$.

\ppart Probar que esta extensión no tiene elementos primitivos.

\solution

Pequeño anexo interesante: $\mathbb{F}_9 ≠ ℤ_9$, $\frac{\mathbb{F}_3[x]}{x^2+1}$

\spart Calculamos el grado de la extensión a partir de la siguiente cadena de inclusiones

\[
\begin{array}{ccccc}
\mathbb{F}_3(x^3,y^3) & \hookrightarrow
	& \mathbb{F}_3(x^3,y^3)(x) & \hookrightarrow
	& \mathbb{F}_3(x,y^3)(y) \\
& \downarrow & & \downarrow & \\
& p_1(t) = t^3-x^3 & & p_2(t) = t^3-y^3 &
\end{array}
\]

$p_1$, $p_2$ serán irreducibles sii no tienen raíces en $\mathbb{F}_3(x^3,y^3)$, y en ese caso tendremos que $[\fd_3(x,y): \fd_3(x^3,y^3)] = 9$.

Esto es sencillo de comprobar puesto que sabemos que $t^3-x^3 = (t-x)^3$ en $\mathbb{F}_3(x,y)$, y este polinomio solo tiene una raíz, que es $x$.

Ahora la única posibilidad de que no fuese irreducible es que resulte que $x\in \mathbb{F}_3(x^3,y^3)$. En ese caso resultaría que la extensión tendría grado 1 pues estaríamos 'añadiendo' un elemento que ya está en el cuerpo.

Si $x$ perteneciera a $\mathbb{F}_3(x^3,y^3)$, entonces podríamos escribir:
\[x = \frac{\sum a_{ij}(x^3)^i(y^3)^j}{\sum b_{ij}(x^3)^i(y^3)^j}; \quad a_{ij},b_{ij} ∈ \mathbb{F}_3\] que es equivalente a que
\[ \sum b_{ij} x^{3i+1}y^{3j} = \sum a_{ij} x^{3i}y^{3j}\]

Para que estos dos polinomios sean iguales debemos igualar sus coeficientes uno a uno. Para cada $j$ el coeficiente de $y^{3j}$ debería ser igual a un lado y al otro de la igualdad, es decir:
\[b_{ij}x^{3i+1}=a_{ij}x^{3i}\]

Y vemos fácilmente que para $i$ que fijemos no existen unas constantes $a,b$ que satisfagan la ecuación.

\spart Esto es paradigmático. Este es un buen ejemplo de algo que veremos más adelante.

En la cadena:
$$\begin{array}{ccccc}\mathbb{F}_3(x^3,y^3) &\subset& \mathbb{F}_3(x^3,y^3)(x) &\subset& \mathbb{F}_3(x,y^3)(y)\\
\end{array}$$

las únicas extensiones posibles en ambas inclusiones son la identidad, ya que por fuerza, $σ(x) = x$ porque $σ$ manda raíces de polinomios en raíces de polinomios.

Podemos concluir que esta extensión es normal, pero no es separable. Se deja como ejercicio para el lector la comprobación de estas afirmaciones.

\spart Se trata de un ejercicio de examen salvo por un pequeño detalle, y es que en el examen daban una pista extra:
\[γ∈\mathbb{F}_3(x,y) \implies γ^3 ∈\mathbb{F}_3(x^3,y^3)\]

Supongamos que esto es verdad. Entonces:

\[[\mathbb{F}_3(x^3,y^3)(γ) : \mathbb{F}_3(x^3,y^3)] ≤ 3\]

porque $γ$ satisface el polinomio $t^3 -γ^3 ∈ \mathbb{F}_3(x^3,y^3)[t]$

\[γ = \frac{\sum a_{ij}(x^3)^i(y^3)^j}{\sum b_{ij}(x^3)^i(y^3)^j}; a_{ij},b_{ij} ∈ \mathbb{F}_3\]

Vamos a demostrar la pista que, en el examen, nos daba el enunciado.

Para ello, fjémonos en el valor de $γ^3$. Como el cuerpo en el que nos movemos tiene $Ch(\mathbb{F}) = 3$ se cumple que $(a+b)^{3} = a^3 + b^3$.

Por tanto:
\[γ^3 = \frac{\sum a_{ij}^3(x^3)^i(y^3)^i}{\sum b_{ij}^3(x^3)^i(y^3)^i}; a_{ij},b_{ij} ∈ \mathbb{F}_3 \implies γ^3 ∈ \mathbb{F}_3 \implies [\mathbb{F}_3(x^3,y^3)(γ) : \mathbb{F}_3(x^3,y^3)] ≤ 3\]

\paragraph{Nota:} Esta respuesta estaría también bien:
$$γ^3 = \frac{\sum a_{ij}(x^3)^i(y^3)^i}{\sum b_{ij}(x^3)^i(y^3)^i}; a_{ij},b_{ij} ∈ \mathbb{F}_3 \implies γ^3 ∈ \mathbb{F}_3 \implies [... : ...] ≤ 3$$

Es exactamente la misma respuesta solo que aplicamos el pequeño teorema de Fermat (que dice que en $Ch(\mathbb{F}_p) \implies a^p = a, ∀a∈\mathbb{F}_p$)


\paragraph{Conclusión:} No tiene elementos primitivos, por que añadiendo el $γ$ que añada, siempre el grado de la extensión va a ser $≤3$. Si $γ$ fuera primitivo, tendríamos que tener que el grado de esa extensión fuera 9 (por definición de elemento primitivo de una extensión \ref{DefElemPrimitivo}).

\end{problem}


\begin{problem}[Parcial 2]

Demostrar que $K= \rac(ω_n = e^{\frac{2πi}{n}})$ es una extensión galoisiana de grado 2 de $E=ℚ(ω_n+ω_{n}^{-1})$

\solution
Si nos fijamos un poco podemos ver sencillamente que si cojo un polinomio de la forma:
\[p(x)=x-(ω_n+ω^{-1}_n)\]
al evaluarlo en $ω_n$ nos quedará $ω_n^{-1}$. Por tanto si cogemos el polinomio
\[p(x)=(x-(ω_n+ω^{-1}_n))x\]
al evaluarlo en $ω_n$ obtendremos un 1. Así, el candidato a polinomio irreducible sería:
\[p(x)=(x-(ω_n+ω^{-1}_n))x -1\]
%Debemos darnos cuenta de que $ℚ(ω_n) =  ℚ(ω_n+ω_{n}^{-1})(ω_n)$.


%Como tenemos grado 2 y $ω_n$ es raíz, entonces: $$p(t) = (t-ω_n) \cdot \underbrace{(...)}_{\text{Grado 1}}$$

%$$p(t) = (t-ω_n)(t-a) ∈ℚ(ω_n + ω_n^{-1})$$
%$$p(t) = t^2 - (ω_n +a)t + ω_na ∈ℚ(ω_n + ω_n^{-1})$$

%Es lógico probar con $a = ω_n^{-1}$, para que los coeficientes del polinomio pertenezcan al ¿cuerpo?.

Con esto hemos visto que $[ℚ(ω_n) : ℚ(ω_n+ω_{n}^{-1})] ≤ 2$, ya que hemos encontrado un polinomio de grado 2, que no sabemos si es irreducible en el cuerpo pequeño.

Si fuese de grado 1 entonces los 2 cuerpos serían iguales y eso no puede ser porque $ω_n = cos\left(\frac{2π}{n}\right) + i sen (\frac{2π}{n})$. Si calculamos $\bar{ω_n} = cos(\frac{2π}{n}) + isen(\frac{2π}{n})$ tenemos que $ω_n + ω_n^{-1} = ω_n+\bar{ω_n} = 2cos(\frac{2π}{n}) ∉ ℝ$ y $ℚ(ω_n + ω_n^{-1}) \subset ℝ$.


\end{problem}

\section{Hoja 4}

\begin{problem}[4]
$$K = ℚ(ω), ω = e^{\frac{2πi}{7}}$$

Vamos a estudiar la teoría de Galois de esta extensión.

\solution

1) $[K:ℚ] = deg(p(t))$, siendo $p(t)$ el polinomio mínimo de $ω$ sobre $ℚ$.

Sabemos que si tenemos $h(t) = t^7 -1$, $h(ω) = 0$, es decir, $ω≠1$ es raíz de $h(t)$.

Vamos a contruir el polinomio irreducible $h(t) = t^7 - 1 = (t-1)\underbrace{(t^6 + t^5 + t^4 + t^3 + t^2 + t + 1)}_{p(t)}$.

Este polinomio $p(t)$ ya es irreducible sobre $ℚ$ porque si tomamos el isomorfismo de anillos $\appl{σ}{ℚ[t]}{ℚ[t]}$ tal que $ σ(p(t)) = p(t+1)$ tenemos que $p(t+1)$ es irreducible por Einsenstein. Por tratarse de un isomorfismo un elemento será irreducible sii lo es su imagen, por lo que concluimos que $p(t)$ es irreducible. (Ya aplicamos este criterio con más detalle en otros ejercicos).

Concluimos por lo tanto $[K:ℚ] = 6$.

1.2) $ℚ(ω) / ℚ$ es galoisiana, porque es el cuerpo de descomposición de $h(t)$ lo que nos asegura normalidad en la extensión. Sabemos que es separable por $Char(K)=0$.

2) Vamos a calcular el grupo de galois: $Gal(K/ℚ)$.

$$\begin{array}{c|c}
G & ω\\\hline
σ_1 & ω\\
σ_2 & ω^2\\
\vdots&\vdots\\
σ_6 & ω^6
\end{array}$$

El orden del grupo $G = Gal(K/ℚ)$ es 6, por ser 6 el grado de la extensión.


Vamos a estudiar un poco cómo son los elementos de este grupo de 6 elementos. Es interesante saber si es isomorfo a $ℤ_6$ o a $ℤ_2×ℤ_3$. Para ello vamos a estudiar el orden de los elementos. Si hay algún elemento de orden 6 tendrá que ser isomorfo a $ℤ_6$, y sino a $ℤ_2×ℤ_3$ (no hay más posibilidades).

$$\begin{array}{c|c|c}
G & ω & \text{orden}\\\hline
σ_1 & ω & 1\\
σ_2 & ω^2 & 3\\
σ_3 & ω^3 & 6\\
σ_4 & ω^4 & ¿2?\\
σ_5 & ω^5 & ¿3?\\
σ_6 & ω^6 & 2
\end{array}$$

Para calcular el orden de $σ_2$:
$$σ_2^2(ω) = σ_2(σ_2(ω)) = ω^4$$
$$σ_2^3 (ω) = σ_2(σ_2^2(ω)) = σ_2(ω^4) = ω^8 = ω$$

Se deja como ejercicio para el lector la comprobación.

\textbf{Útil:} Solo hay 4 posibilidades para el orden del elemento: $1,2,3,6$, porque tienen que dividir al orden del grupo. Es interesante también darnos cuenta de que:
\[1 = σ_3^6 = (\underbrace{σ_3^2}_{σ_2})^3 = (\underbrace{σ_3^3}_{σ_6})^2 = 1 \implies ord(σ_6) = 2\]

\paragraph{Subgrupos/cuerpos fijos} Hay tantos cuerpos fijos como subgrupos.

El cuerpo fijo correspondiente a $σ_1=id$ es $K$.

El cuerpo fijo correspondiente a $K^{\gen{σ_3}}$ es $ℚ$

Vamos a comprobarlo (por puro amor a las cuentas): $$K^{\gen{σ_3}} = \{x∈K \tq σ_3(x) = x\}$$

Podríamos utilizar la base $\mathcal{B} = \{1,ω,ω^2,...,ω^5\}$. Para este caso es más conveniente utilizar la base $\mathcal{B} = \{ω,ω^2,...,ω^5,ω^6\}$, teniendo entonces  $$K^{\gen{σ_3}} = \{x = a_0 ω + a_1ω^2 + a_2ω^3+...+a_5ω^6\quad x∈K \tq σ_3(x) = x\}$$

Aplicamos $σ_3(x) = σ_3(a_0 ω + a_1ω^2 + a_2ω^3+...+a_5ω^6) = σ_3(...) = ... = ... $

Resolvemos el sistema de ecuaciones $σ(x) = x$.

\[
σ_3(x) = x \dimplies \left\{
\begin{array}{cc}
a_1 = a_3 = a_2 = a_6 = a_4 = a_5
\end{array}
\right.
\]

Entonces: $$K^{<σ_3>} = \{ x = q(ω+ω^2 + ... + ω^6), q∈ℚ\} = \{x = -a, a∈ℚ\} = ℚ$$.

$$\textbf{Subgrupos de } G = \left\{
\begin{array}{c}
\gen{id}\\\gen{σ_3^3=σ_6}\\\gen{σ_3^2 = σ_2}\\G=\gen{σ_3}
\end{array}
\right\} \dimplies \left\{
\begin{array}{c}
K^{\gen{id}} = K\\
K^{\gen{σ_6}} = (1) \\
K^{\gen{σ_2}} = (2) \\
K^{\gen{σ_3}} = ℚ
\end{array}
\right\}$$

Vamos a calcular (1) y (2):

Tomamos: $x = a_1 ω + a_2 ω^2 + a_3ω^3 + a_4ω^4 + a_5ω^5+a_6ω^6$


$$σ_6(x) = a_1ω^6 + a_2ω^5 + a_3ω^4 + a_4ω^3 + a_5ω^2 + a_6ω$$
$$σ_6(x) = x\dimplies \left\{
\begin{array}{c}
a_1 = a_6\\a_2=a_5\\a_3=a_4
\end{array}
\right. \dimplies x = a_1(ω+ω^6) + a_2(ω^2 + ω^5) + a_3(ω^3+ω^4)$$

Por la teoría de Galois, sabíamos antes de empezar, que el grado de la extensión $[K^{\gen{σ_6}}:ℚ] = \frac{6}{2} = 3$, que es lo que hemos obtenido.

Otra cosa que nos dice el teorema, es que existe un elemento primitivo. En este caso lo lógico es apostar porque sea $(ω^2+ω^5)$, es decir, ¿se cumple $K^{\gen{σ_6}} = ℚ(ω^2 + ω^5)$? Si este no lo fuera, probaríamos con el resto.

Sea $γ = ω^2 + ω^5$. Si $γ∈ℚ(γ) \implies γ^2 ∈ℚ(γ)$. En este caso, $γ^2 = ω^4 + ω^3 + 2\implies (ω^3+ω^4) ∈ℚ(γ)$.

Si $γ∈ℚ(γ) \implies γ^3 ∈ℚ(γ)$. En este caso, $γ^3 = ω^6 + 3ω^2 + 3ω^5 + ω \implies (ω+ω^6)∈ℚ(γ)$.



Vamos a calcular el polinomio mínimo de la extensión. Tenemos $H=\gen{σ^3} \quad K^{\gen{σ_3}} = ℚ(ω + ω^{-1})$.

$$|H| = 2 \implies ℚ \underset{3}{\subset}K^H \underset{2}{\subset} K = ℚ(ω)$$.

El polinomio irrducible de $ω$ sobre $K^H$ será de forma $q(t) = (t-ω) (t-ω^{-1})  = t^2 - ωt-ω^{-1}t + 1$, por lo que $[K^H: K]=2$

El polinomio irreducible de $γ=ω+ω^{-1}$ sobre $ℚ$ es el que queremos calcular, pero sabemos que $[ℚ:K^H] = 3$ (por el teoremade Galois).


Para afianzar los conocimientos de la materia y un poco por amor al arte: vamos a calcular el polinomio irreducible de $γ=ω+ω^{-1}$ sobre $K^H$ y lo llamamos $f(t)$. Buscamos un polinomio $f(t) = a₀+a₁t + a₂t^2 + t^3$ con $f(γ) = 0$. Este polinomio sabemos que es de grado 3 porque es el grado de la extensión.


Para ello, resolvemos el sistema de ecuaciones que surge de $f(γ) = 0$

\[\begin{array}{lcl}
γ^0 &=& 1\\
γ = ω+ω^{-1} &=& ω+ω^6\\
γ^2  &=& ω^2 + ω^5 + 2\\
γ^3 &=& ω^3 + 3ω + 3ω^6 + ω^4
\end{array}\]


Aquí tenemos un pequeño problema. La forma normal de concluir que los coeficientes de los $\omega^i$ son 0 es basándonos en que los $ω$ forman una base vectorial. En este caso, $\{1,ω,...,ω^6\}$ que son los elementos con los que trabajamos no forman una base... (son 7 elementos en un espacio vectorial de dimensión 6. No pueden ser linealmente independientes). Es por ello que vamos a intentar reescribirlo para quitarnos el $ω^6$.

Sustituimos $ω^6$ por su combinación lineal respecto del resto de elementos de la base, que es $ω^6 = -1-ω-ω^2-ω^3-ω^4-ω^5$.

Razón: $0 = p(ω) = 1+ω+ω^2 + ω^3 + ω^4+ω^5+ω^6$ y despejamos.


\[\begin{array}{lcc}
γ^0 = 1&\\
γ = ω+ω^{-1} = ω -1-ω-ω^2-ω^3-ω^4-ω^5 &=& -1 -ω^2-ω^3-ω^4-ω^5 \\
γ^2 = &=& ω^2 + ω^5 + 2\\
γ^3 = ω^3 + 3ω + 3(-1-ω-ω^2-ω^3-ω^4-ω^5) + ω^4 &=& -3ω^2 -2ω^3 -2ω^4-3ω^5
\end{array}\]

Buscamos $a_i$ tal que $a₀ + a₁γ + a₂γ^2 + a_3 = 0$.

$$f(γ) = a_0 + a_1 (-1 -ω^2-ω^3-ω^4-ω^5) + a_2 ( ω^2 + ω^5 + 2) + (-3ω^2 -2ω^3 -2ω^4-3ω^5)$$

Podemos resolverlo como siempre, agrupando los coeficientes que multiplican a cada $ω^i$ y obteniendo los $a_i$.

En este caso, por ser sencillo se puede hacer de cabeza y, si no nos hemos equivocado, obtenemos:
\[\begin{array}{l}
a_1 = -2\\2+a_2 -3 = 0 \implies a_2 = 1\\a₀ + 2 + 2 -3 = 0 \implies a₀=-1
\end{array}\]


\textbf{Conclusión: } El polinomio irreducible del elemento $ω+ω^{-1}$ sobre $K^H$ es $f(t) = 1-2t+t^2+t^3$
\end{problem}


\begin{problem}[3]
$\mathcal{K}$: cuerpo de descomposición de $p(t) = t^4 + 30t^2 + 45$ sobre $ℚ$.

Nos piden demostrar que el grado de la extensión es $4$.


\solution
Vamos a buscar raices del polinomio $p(t) = (t^2)^2 + 30(t^2) + 45$, resolviendo la ecuación bicuadrática.

En este caso: $$t =\pm\left[ i\sqrt{3(5\pm2\sqrt{5})}\right]$$.

\[\begin{array}{cc}
γ_1 =  i\sqrt{3(5+2\sqrt{5})}\\
γ_2 =  i\sqrt{3(5-2\sqrt{5})}\\
γ_3 = -i\sqrt{3(5+2\sqrt{5})}\\
γ_4 = -i\sqrt{3(5-2\sqrt{5})}
\end{array}\]

Vemos que $∀i, γ_i∉ℚ \implies p(t)$ irreducible en $ℚ$.

Por definición, el cuerpo de descomposición $\mathcal{K} = ℚ(γ_1,γ_2,γ_3,γ_4)$.

$\mathcal{K} = ℚ(γ_1,γ_2)$, porque si $a∈ \mathcal{K} \dimplies -a ∈ \mathcal{K}$ y en este caso $γ_1 = -γ_3$ y $γ_2 = γ_4$.



\paragraph{$[ℚ(γ_1):ℚ]$}
El grado de esta extensión tiene que ser $\leq4$, por ser de grado 4 el polinomio $p(t)$.

Tenemos que demostrar: $[\mathcal{K}:ℚ] = 4\dimplies \mathcal{K} =ℚ(γ_1) \dimplies γ_2∈ℚ(γ_1)$


\textbf{Sugerencia: } Multiplicar $γ_1 · γ_2$ (nos lo sugiere el enunciado)

$$γ_1γ_2 = ... = 3\sqrt{5} \implies γ_1·γ_2 ∈ℚ(\sqrt{5})\overset{?}{\subset} ℚ(γ_1)$$

Esa inclusión la tenemos porque $γ_1^2 = -15 + 6\sqrt{5}$, luego $ℚ(\sqrt{5}) = ℚ(γ_1)^2 \subset ℚ(γ_1)$.\


Como el producto $γ_1 · γ_2∈ℚ(γ_1) \implies γ_2∈ℚ(γ_1)$.

\textbf{Conclusión:} $\mathcal{K} = ℚ(γ_1,γ_2) = ℚ(γ_1) = ℚ(γ_2)$ (es decir, las 4 raíces son elementos primitivos.)


\paragraph{Subgrupos/subcuerpos}


Como tiene grado 4 sólo tenemos estas 2 posibilidades:
$$G = Gal(\mathcal{K}/ℚ), |G| = 4 \implies
\left\{ \begin{array}{c}
G = C_4\\
G = C_2×C_2
\end{array}\right.$$

Tenemos que ver el orden de los elementos, para lo cual, tenemos que encontrarlos.

Sea $σ∈G \dimplies σ(γ_1) = -γ_1 \quad σ^2(γ_1) = γ_1$.

Sea $τ∈G \dimplies τ(γ_1) = γ_2.$ y calculamos
$$τ(γ_2) = τ\left(\frac{3\sqrt{5}}{γ_1}\right) = \frac{τ(3\sqrt{5})}{τ(γ_2)} = \frac{\pm\sqrt{5}}{γ_2}$$

Como $\sqrt{5} = \frac{γ_1^2 + 15}{6} \implies τ(\sqrt{5}) = \frac{τ(15) + τ(γ_1^2)}{τ(6)} = \frac{15 + (-15-6\sqrt{5})}{6} = -\sqrt{5}$

Hay un error con un 3 que nos hemos dejado.

Como $γ_1γ_2 ≠ -3\sqrt{5}$ concluimos que $|τ| = 4$,por lo que $G = C_4$.

\end{problem}


\begin{problem}[5] (Versión hecha en clase distinta de la de las hojas).
$$\mathcal{K} = ℚ(\sqrt{2},\sqrt{3},α)\; α = \sqrt{(9-5\sqrt{3})(2-2\sqrt{2})}$$

Demostrar que $[K:ℚ] = 8$.
\solution

Descomponemos la extensión para poder calcular su grado.

$$[K:ℚ] = [K: ℚ(\sqrt{2},\sqrt{3})]\underbrace{[ℚ(\sqrt{2},\sqrt{3})(α):ℚ]}_{=4}$$

Es decir, tenemos que calcular probar si $[K: ℚ(\sqrt{2},\sqrt{3})]= 2$. Para ello, calculamos el polinomio mínimo.

Como α es raíz de $h(t) = t^2-α^2 ∈ℚ(\sqrt{2},\sqrt{3})[t]$, lo que tenemos que comprobar es que $α∉ℚ(\sqrt{2},\sqrt{3})$.

Vamos a demostrarlo por reducción al absurdo. \textbf{Supongamos} $α∈ℚ(\sqrt{2},\sqrt{3})$, entonces, $α = a+b\sqrt{2} + c \sqrt{3} + d\sqrt{2}\sqrt{3}\quad a,b,c,d∈ℚ$, ya que $\{1,\sqrt{2}.\sqrt{3},\sqrt{6}\}$ es una base.

Por un lado, utilizando el enunciado: $α^2 =  ... = 18-18\sqrt{2} - 10\sqrt{3} + 10\sqrt{6}$.

Por otrol lado, utilizando la hipótesis:
$α^2 =(a^2+2b^2 + 3c^2 + 6d^2) + (2ab + 6cd) \sqrt{2} + (2ac + 4bd)\sqrt{3}+(2ad + 2bc) \sqrt{6}$.

Igualando obtenemos el sistema:

\[
\begin{array}{cc}
a^2+2b^2 + 3c^2 + 6d^2 &= 18\\2ab + 6cd &= -18 \\ 2ac + 4bd &= -10\\ 2ad + 2bc &= 10
\end{array}
\]

Resolviendo este sistema deberíamos obtener una contradicción.


\ppart

$K_1 = ℚ(\sqrt{2},\sqrt{3})$.

$K_2 = ℚ(\sqrt{2},\sqrt{3},α)$.


Vamos a calcular $Gal(K_1/K_2)$ (aunque no sepamos seguro si es Galoisiana o no).

Sabemos: $$\begin{array}{c|c|c}
&\sqrt{2}&\sqrt{3}\\\hline
σ_1 & \sqrt{2} & \sqrt{3}\\
σ_2 & - \sqrt{2} & \sqrt{3}\\
σ_3 & \sqrt{2} & -\sqrt{3}\\
σ_4 & - \sqrt{2} & -\sqrt{3}\end{array}
$$

Tenemos que calcular las extensiones de los $σ's$. Habrá 2 extensiones de cada $σ$, es decir:

$$\begin{array}{c|c|c|c}
&\sqrt{2}&\sqrt{3} & α\\\hline
σ_{11} & \sqrt{2} & \sqrt{3} & α_1\\
σ_{12} & \sqrt{2} & \sqrt{3} & -α_1\\\hline

σ_{21} & \sqrt{2} & -\sqrt{3} &  α_2\\
σ_{22} & \sqrt{2} & -\sqrt{3} & -α_2\\\hline

σ_{31} & -\sqrt{2} & \sqrt{3} & α_3\\
σ_{32} & -\sqrt{2} & \sqrt{3} & -α_3\\\hline

σ_{41} & -\sqrt{2} & -\sqrt{3} & α_4\\
σ_{42} & -\sqrt{2} & -\sqrt{3} & -α_4\\
\end{array}
$$

Donde:
\begin{itemize}
\item $α_2$ es una raíz de  $h^{σ_2}(t) = t^2 - (9+5\sqrt{3})(2-2\sqrt{2})$, es decir $α_2 = \pm\sqrt{(9+5\sqrt{3})(2-2\sqrt{2})}$

\item $α_3$ es una raíz de $h^{σ_3}(t) = t^2 - (9-5\sqrt{3})(2+2\sqrt{2})$, es decir, $α_3 = \pm\sqrt{(9-5\sqrt{3})(2+2\sqrt{2})}$

\item $α_4$ es una raíz de $h^{σ_4}(t) = t^2 - (9+5\sqrt{3})(2+2\sqrt{2})$, es decir, $α_4 = \pm\sqrt{(9+5\sqrt{3})(2+2\sqrt{2})}$
\end{itemize}

Esto se debe a que las inmersiones tienen que llevar raíces de polinomios en polinomios.



Nos pedían las inmersiones que es lo que hemos calculado, pero nos queda comprobar: $$K/ℚ \text{ Galoisiana } \dimplies α_2,α_3,α_4 ∈K$$

Estamos demostrando si la extensión es normal (porque al trabajar siempre con Char=0 tenemos normal $\implies$ galoisiana).


Vamos a utilizar un truco ya empleado en la resolución del ejercicio anterior (o el anterior del anterior (o el anterior el anterior del anterior (o el anterior del anterior del anterior del anterior (...)))).

$$αα_2 = \sqrt{...}\sqrt{...} = \sqrt{(81 - 75)·(2-2\sqrt{2})^2} = \sqrt{6}(2-2\sqrt{2}) ∈ ℚ(\sqrt{2},\sqrt{3}) \subset K \implies α_2 ∈ K$$.

$$αα_3 = \sqrt{...}\sqrt{...} =(9-5\sqrt{3})\sqrt{4-8} \overset{?}{\implies}α_3 ∈K \dimplies \sqrt{-4}∈K \dimplies i∈K$$

Si $i∈K \implies$ Galoisiana. Si $i∉K \implies$ la extensión no es galoisiana.

Vamos a comprobarlo: $αα_3 = 2i(9-5\sqrt{3})$. No basta con comprobar que no está en los reales\footnote{$[ℚ(\sqrt{2}i):ℚ] = 2 ≠ 4$. Sería 4 si $i,\sqrt{2}∈ℚ(\sqrt{2}i)$}

Supongamos que $i∈K = ℚ(\sqrt{2},\sqrt{3})(α)$, entonces:

$$i = a+bα\quad a,b∈ℚ(\sqrt{2},\sqrt{3})\quad α = i
\underbrace{\sqrt{(9-5\sqrt{3})(2\sqrt{2}-2)}}_{r∈ℝ} \implies$$
$$\implies i = a+bri\quad a,br∈ℝ\implies \left\{
\begin{array}{cc}
a = 0&\\ b = \frac{1}{r} &\implies r ∈ℚ(\sqrt{2},\sqrt{3})
\end{array}
\right.$$

Hemos reducido el problema a algo más sencillo de demostrar: ¿$r∈ℝ$?. Para ello vamos a calcular:

$$\frac{σ_{31}(r^2)}{r^2} = \frac{σ_{31}\left((9-5\sqrt{3})(2-2\sqrt{2})\right)}{(9-5\sqrt{3})(2\sqrt{2}-2)} = \frac{(9-5\sqrt{3})(-2-2\sqrt{2})}{(9-5\sqrt{3})(2\sqrt{2}-2)}$$

Oh vaya, esta cuenta no nos sirve para nada. Vamos a hacer otra a ver si nos sale:

$$\frac{σ_{21}(r^2)}{r^2} = \frac{σ_{21}\left((9-5\sqrt{3})(-2+2\sqrt{2})\right)}{(9-5\sqrt{3})(2\sqrt{2}-2)} = \frac{(9+5\sqrt{3})(-2+2\sqrt{2})}{(9-5\sqrt{3})(2\sqrt{2}-2)}=\frac{(9-5\sqrt{3})^2}{\sqrt{6}^2}$$
$$\implies σ_{21}(r^2) = ... \left(\frac{9+5\sqrt{3}}{\sqrt{6}}\right)^2 \implies σ_{21}(r) = \pm \frac{9+5\sqrt{3}}{\sqrt{6}}$$

Para buscar la contradicción, calculamos $σ_{21}^2(r)$.

$$σ_{21}^2(r) =  ... = ... = \left(\frac{9-5\sqrt{3}}{-\sqrt{6}} \frac{9+5\sqrt{3}}{\sqrt{6}}r\right) = ... = -r$$.

Tenemos que $σ_{21}(r) = -r$ y esto contradice que el orden de $σ_{21}$ es 2.



Seguimos con el ejercicio:

$$αα_4 = ... = ... \implies α_4 ∈K$$

Hemos demostrado que la extensión es normal (porque las imágenes de las inmersiones no se salen del cuerpo) y por lo tanto es galoisiana. Ponemos $HG = Gal(K/ℚ) = Inm(K/ℚ)$.


Ahora, al tener un grupo de orden 8, vamos a calcular cómo son los elementos del grupo, si es isomorfo a $C_4×C_2$ o si es cíclico o lo que sea. Se da por supuesto que, llegados a este punto, somos capaces de calcular el orden de las ¿inmersiones?.

$$\begin{array}{c|c|c|c|c}
&\sqrt{2}&\sqrt{3} & α & \text{orden}\\\hline
σ_{11} & \sqrt{2} & \sqrt{3} & α_1& 1 \\
σ_{12} & \sqrt{2} & \sqrt{3} & -α_1& 2 \\\hline

σ_{21} & \sqrt{2} & -\sqrt{3} &  α_2& 4\\
σ_{22} & \sqrt{2} & -\sqrt{3} & -α_2& \\\hline

σ_{31} & -\sqrt{2} & \sqrt{3} & α_3& \\
σ_{32} & -\sqrt{2} & \sqrt{3} & -α_3& \\\hline

σ_{41} & -\sqrt{2} & -\sqrt{3} & α_4& \\
σ_{42} & -\sqrt{2} & -\sqrt{3} & -α_4& \\
\end{array}
$$


% % Si cambias esto, cambia el ejercicio siguiente, que hace referencia a esto.
Vamos a calcular (porque es la sugerencia del enunciado) $σ(α^2)$ y $σ^2(α)$ (utilizando $σ = σ_2$)

$$\frac{σ(α^2)}{α^2} = \frac{(9+5\sqrt{3})(2-2\sqrt{2})}{(9-5\sqrt{3})(2-2\sqrt{2})} = \frac{9+5\sqrt{3}}{9-5\sqrt{3}} = \left(\frac{9+5\sqrt{3}}{\sqrt{6}}\right) ^2$$

Tenemos: $$σ(α^2) = \left(\frac{9+5\sqrt{3}}{\sqrt{6}}α\right)^2 \implies σ(α) = \pm \frac{9+5\sqrt{3}}{\sqrt{6}} α$$


Vamos a calcular ahora $σ^2(α)$:

$$σ^2(α) = σ(σ(α)) = σ\left(\pm\frac{9+5\sqrt{3}}{\sqrt{6}}α \right) = \pm\frac{9-5\sqrt{3}}{-\sqrt{6}} \left( \pm\frac{9+5\sqrt{3}}{\sqrt{6}} α \right) = - α$$

Con estos 2 cálculos, llegamos a demostrar que $α\notinℚ(\sqrt{2},\sqrt{3})$, debido a que $σ^2 = id$ y con $α$ no se cumple. Cocluimos entonces que $α∉ℚ(\sqrt{2},\sqrt{3})$

\end{problem}

\begin{problem}[5] (Versión oficial de la hoja)

$$K = ℚ(\sqrt{2},\sqrt{3},α) \quad α = \sqrt{(9-5\sqrt{3})(2-\sqrt{2})} \subseteq ℝ$$.

\solution

Para demostrar que $[K:ℚ] = 8$, vale el mismo razonamiento seguido al final del ejercicio anterior.


Vamos a calcular $\gal(K/G)$, como pide el enunciado:

$$\begin{array}{c|c|c|c}
&\sqrt{2}&\sqrt{3} & α \\\hline
σ_{11} & \sqrt{2} & \sqrt{3} & α_1 \\
σ_{12} & \sqrt{2} & \sqrt{3} & -α_1 \\\hline

σ_{21} & \sqrt{2} & -\sqrt{3} &  α_2\\
σ_{22} & \sqrt{2} & -\sqrt{3} & -α_2 \\\hline

σ_{31} & -\sqrt{2} & \sqrt{3} & α_3 \\
σ_{32} & -\sqrt{2} & \sqrt{3} & -α_3 \\\hline

σ_{41} & -\sqrt{2} & -\sqrt{3} & α_4 \\
σ_{42} & -\sqrt{2} & -\sqrt{3} & -α_4 \\
\end{array}
$$



Donde:
\begin{itemize}
\item $α_2$ es una raíz de  $h^{σ_2}(t) = t^2 - (9+5\sqrt{3})(2-\sqrt{2})$, es decir $α_2 = \sqrt{(9+5\sqrt{3})(2-\sqrt{2})}$

\item $α_3$ es una raíz de $h^{σ_3}(t) = t^2 - (9-5\sqrt{3})(2+\sqrt{2})$, es decir, $α_3 = \sqrt{(9-5\sqrt{3})(2+\sqrt{2})}$

\item $α_4$ es una raíz de $h^{σ_4}(t) = t^2 - (9+5\sqrt{3})(2+\sqrt{2})$, es decir, $α_4 = \sqrt{(9+5\sqrt{3})(2+\sqrt{2})}$
\end{itemize}


Nos piden calcular que las inmersiones me llevan elementos de dentro del cuerpo a elementos de dentro del cuerpo. Eso es ver que $K/Q$ es normal, pero al trabajar con Char=0, es lo mismo que demostrar que es galoisiana.


$$αα_2 = \sqrt{...}\sqrt{...} = \sqrt{(81 - 75)·(2-\sqrt{2})^2} = \sqrt{6}(2-\sqrt{2}) ∈ ℚ(\sqrt{2},\sqrt{3}) \subset K \implies α_2 ∈ K$$

$$αα_3 = \sqrt{...}\sqrt{...} =(9-5\sqrt{3})\sqrt{2}∈ ℚ(\sqrt{2},\sqrt{3}) \subset K   \implies α_3 ∈K$$


$$αα_4 = \sqrt{...} \sqrt{...} = \sqrt{6}\sqrt{2} ∈ ℚ(\sqrt{2},\sqrt{3}) \subset K \implies α_4 ∈K$$

\textbf{Conclusión: } $K/G$ es galoisiana.

Vamos a ver el orden de las inmersiones, para ver a qué grupo es homeomorfo el grupo de las inmersiones.


$$\begin{array}{c|c|c|c|c}
&\sqrt{2}&\sqrt{3} & α & \text{orden}\\\hline
σ_{11} & \sqrt{2} & \sqrt{3} & α_1& 1 \\
σ_{12} & \sqrt{2} & \sqrt{3} & -α_1& 2 \\\hline

σ_{21} & \sqrt{2} & -\sqrt{3} &  α_2& 4\\
σ_{22} & \sqrt{2} & -\sqrt{3} & -α_2& 4 \\\hline

σ_{31} & -\sqrt{2} & \sqrt{3} & α_3& 4\\
σ_{32} & -\sqrt{2} & \sqrt{3} & -α_3& 4\\\hline

σ_{41} & -\sqrt{2} & -\sqrt{3} & α_4& 4\\
σ_{42} & -\sqrt{2} & -\sqrt{3} & -α_4& 4\\
\end{array}
$$


$$σ_{21}^2(α) = σ_{21}(α_2) = σ_{21}\left( \frac{αα_2}{α_2}\right) = σ_{21} \left(\frac{\sqrt{6}(2-\sqrt{2})}{α}\right) = \frac{-\sqrt{6}(2-\sqrt{2})}{α_2} = -α \implies \ord(σ_{21}) = 4$$

$$ σ_{42}^2(α) = - σ_{42}(α_4) = ... = σ_{42}\left(\frac{4\sqrt{3}}{α}\right) = ... = -α \implies \ord(σ_{42}) = 4$$

Las cuentas del orden del resto de las inmersiones son muy parecidas. Se deja como ejercicio para el lector (o para el revisor) cerciorarse. Además, podemos ser optimistas e inventarnos un poco a ojo que $σ_{21}^2 = σ_{22}^2 = ... = σ_{42}^2 = σ_{12}$.


Estos órdenes de los elementos se corresponden con el grupo de los cuaternios, formado por $\{±1,±i,±j,±k\}$ que cumple $i²= j² = k² = -1$ y $ij = k, jk = i, ki = j$ y algunas cosas más, como que el grupo no es abeliano ($ij = -ji$).

Llamando a $σ_{11}=1,σ_{12} = -1,σ_{21} = i,σ_{22}=-i,σ_{31}=j,σ_{32}=-j,σ_{41}=k,σ_{42}=k$.

Nos tenemos que asegurar que las igualdades anteriores son ciertas, es decir, como $ij=k$ entonces tendríamos que tener $σ_{21}σ_{31} = σ_{41}?$.

$$σ_{21}σ_{31} = \left\{\begin{array}{ccc}
\sqrt{2} &\overset{σ_{31}}{\to} -\sqrt{2} &\overset{σ_{21}}{\to} -\sqrt{2}\\
\sqrt{3} &\overset{σ_{31}}{\to} \sqrt{3} &\overset{σ_{21}}{\to} -\sqrt{3}\\
α &\overset{σ_{31}}{\to} α_3 &\overset{σ_{21}}{\to} α_4
\end{array}\right.$$

Vamos a calcular $$σ_{21}σ_{31}(α) = σ_{21}\left(\frac{\sqrt{2}(9-5\sqrt{3})}{α}\right) = \frac{\sqrt{2}(9+5\sqrt{3})}{α_2} = \pm α_4$$

Ese resultado tiene que ser $\pm α_4$, ya que $σ_{21}σ_{31}$ tiene que ser $σ_{42}$ o $σ_{41}$ porque cambia de signo $\sqrt{2}$ y $\sqrt{3}$.

Vamos a calcular $α_4α_2$ $$α_4α₂= \frac{\sqrt{6}{\sqrt{2}}}{α} \frac{\sqrt{6}(2-\sqrt{2})}{α} = ... = \sqrt{2}(9+5\sqrt{3})$$

Entonces $$\left.\begin{array}{c}
α_4α_2 = \sqrt{2}(9+5\sqrt{3})\\
\frac{\sqrt{2}(9+5\sqrt{3})}{α_2} = \pm α_4
\end{array}\right\} \implies + α_4 \implies σ_{21}σ_{31} = σ_{41}$$


\paragraph{Cuerpos (a ojo)}

$$ℚ,\underbrace{ℚ(\sqrt{2}),ℚ(\sqrt{3}),ℚ(\sqrt{2}\sqrt{3})}_{(1)},\underbrace{ℚ(\sqrt{2},\sqrt{3})}_{\gen{σ_{12}}},K$$

(1): Fijos por $σ_{21},σ_{31},σ_{41}$ respectivamente. Podemos comprobar que: $[K^H : ℚ] = \frac{|G|}{|H|} = 2$.


\obs Todas estas extensiones son normales, porque todos los subgrupos son normales.
Justificación:

Recordamos que cualquier subgrupo de orden la mitad que el total es normal. Además, $σ_{12} = -1$ está en el centro del grupo, es decir, conmuta con todos los elementos, y lo mismo $σ_{11} = 1$.


Por fin lo hemos acabado!!
\end{problem}


\section{Primer Parcial}

\begin{problem}[2]
Tenemos que encontrar: $\mathbb{F}_5 \underset{2}{\subset} K$
\solution

Si tomamos $K = \frac{\mathbb{F}_5}{(p(x))}$, con $p(x)$ irreducible y de grado 2.

Empezamos a mirar:

$p(x) = x^2 + 1$ no es irreducible, por $p(2) = 0$

$p(x) = x^2 + 2$ sí es irreducible (hacemos la cuenta de $p(i) ≠ 0, ∀i ∈ \{0,1,2,3,4\}$


Entonces: $$K = \mathbb{F}_5(x), x = \bar{X}$$

\end{problem}

\section{Hoja 6}

\begin{problem}[1]
noseque
\solution
$p(t) ∈ℚ[t]$ irreducible de grado 3. Las raíces son $γ₁,γ₂,γ_3$. Tomamos el cuerpo de descomposición $K=ℚ(γ_1,γ₂,γ_3)$.

1) $G = \gal(K/ℚ)$, siempre vamos a tener que $G \subseteq S_3 = \{Biyecciones(\{γ_i\})\}$

2) $|G| = [K:ℚ] = 3,6$ porque $ℚ\underbrace{\subset}_{3} ℚ(γ_i) \subset K$

3) $[K:ℚ] = 3 \dimplies K = ℚ(γ_1) \dimplies G = A_3=\gen{(1,2,3)}$

4) Sea $Δ := (γ₁-γ₂)(γ₁ - γ_3)(γ₂ - γ_3)$. Se pide demostrar que $Δ²∈ℚ$.

Está claro que $Δ²∈K$ ($Δ²$ es el discriminante del polinomio). Además,  $ Δ^2$ queda fijo por $G\subset S_3 \implies Δ^2∈K^G=ℚ$. Queda fijo porque si permutamos las $γ_i$ con el generador de $G$ queda lo mismo, tal vez con signos distintos, pero al tomar el cuadrado tenemos siempre positivo.


5) $[K:ℚ] = 6 \dimplies G=S_3 \dimplies Δ∉ℚ=K^G$.

Si $∀∉ℚ$, tenemos $ℚ\overset{\subset}{≥2}ℚ(Δ) \overset{\subset}{a} ℚ(Δ,γ₁) \subseteq K$. Además, $ℚ\overset{\subset}{3} ℚ(γ₁) \overset{\subseteq}{b} ℚ(Δ,γ₁)$.

Entonces, $[ℚ(Δ,γ₁):ℚ] ≥ 6 \implies K = ℚ(Δ,γ₁)$. Esta implicación se debe a que $[ℚ(Δ,γ₁) : ℚ]  = \left\{\begin{array}{c}
= 2a\\>3b
\end{array}\right\} \implies [ℚ(Δ,γ₁):ℚ] = 6$.

Ahora vamos a demostrar la implicación hacia la derecha. $G=S_3 \implies Δ∉ℚ$. Esto se debe a que $Δ$ no queda fijo por todos los elementos de $S_3$ (basta tomar la permutación del 1 con el 2 y entonces $Δ$ cambia de signo), entonces no queda fijo por $G$, es decir $∉K^G$, es decir $∉ℚ$

6) $K=ℚ(Δ,γ₁)$ en cualquier caso. La extensión de galois de un polinomio de grado 3 está generada por la raíz del discriminante y una de las raíces.

7) $G=S_3\dimplies Δ^2$ no es un cuadrado en $ℚ$.

8) Un polinomio de grado 3 se puede escribir en la forma $p(t) = t^3+at+b$ (después de u cambio de la forma $t\to t+c$)\footnote{$p(t) =  a₀ + a₁t+a₂t^2 + a_3t^3$. Tomamos $p(t+c) = a₀ + a₁(t+c) + a_2(t+c)^2 + a_3(t+c)^3$, desarrollamos y sustituimos para eliminar coeficientes de $t^2$.}

9.1) Si $p(t) = t^3+at+b$, entonces sus raíces satisfacen 3 cosas. Desarrollando $(t-γ_1)(t-γ_2)(t-γ_3)$ sale sencillo.

9.2)  $∆^2 = −4a^3 − 27b^2$. Las cuentas salen, se dejan como ejercicio para el paciente lector.

\end{problem}

\begin{problem}[2]
Usando el apartado anterior calcular $K$ y $G$ para los polinomios $p(t)$ siguientes:

\ppart $t^3 − 3t + 1$
\ppart $t^3 − t + 1$
\ppart $t^3 − 5t + 7$
\ppart $t^3 + 2t + 2$
\ppart $t^3 + 3$
\ppart $t^3 − t − 1$

\solution

Introducimos un pequeño concepto. $x∈ℚ^2 \dimplies \sqrt{x}∈ℚ$. Es importante utilizar las cosas demostradas en el ejercicio anterior, sobretodo: $∆^2 = −4a^3 − 27b^2$

\spart $p(t) = t^3 − 3t + 1$.

Calculamos $Δ^2 = -4(-3)^3-27·1^2 = 3^4 \implies Δ^2 ∈ℚ^2 \implies K=ℚ(γ₁) \implies G=A_3$

\spart $p(t) = t^3 − t + 1$

Calculamos $Δ^2 = -4(-1)^3-27·1^2 = -23 \implies Δ∉ℚ^2 \implies K = ℚ(Δ,γ₁) \implies G=S_3$
\end{problem}

\section{Segundo parcial año pasado}

$\mathbb{F}_p\subset \mathbb{F}_q, q=p^n$

Para este ejercicio es necesaria la siguiente definición:
\begin{defn}[Automorfismo\IS de Frobenius]
$\appl{φ}{\mathbb{F}_p}{\mathbb{F}_q}$ con $φ(x) = x^p$.
Esta aplicación cumple:
\begin{itemize}
\item $φ(x+y) = ... = φ(x) + φ(y)$.
\item Por ser de grupos, $φ(x·y) = φ(x)·φ(y)$
\item Por ser finito es suprayectiva.
\item Por ser de cuerpos, es inyectiva.
\end{itemize}


Sea que $\ord(φ) = d$. Esto es,$φ^d = Id$. Utilizando: $φ(x) = x^p$ caluclamos $φ(x^2) = ... = x^{p^2}$. En general, $φ^d(x) = x^{p^d}$.
\end{defn}

\begin{prop}
$\ord(φ) = d \implies $ todos los elementos de $\mathbb{F}_q$ son raíces del polinomio $f(t) = (t^{pd}-d)\mathbb{F}_p[t]$. Esto quiere decir $\gal(\mathbb{F}_q/\mathbb{F}_p) = <φ>$

\end{prop}

Parece que no vamos a demostrar, pero sí a ejemplificar la utilidad.

Sea $F_{5^3}/F_5$. Entonces: $$F_{5^3} = \frac{F_5[x]}{x^3+x+1} = F_5[\bar{x}]$$ siendo $\bar{x}$ una raíz de $t^3+t+1$. Las otras raíces han de ser $φ(\bar{x}) = (\bar{x})^5$ y $φ(\bar{x})^2 = (\bar{x})^{5^2}$.

Vamos a poner un ejemplo de esta última propiedad
