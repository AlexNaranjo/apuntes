\documentclass{apuntes}

\title{Teoría de Galois}
\author{Víctor de Juan}
\date{14/15 C1}

% Paquetes adicionales

% --------------------

\begin{document}
\pagestyle{plain}
\maketitle

\tableofcontents
\newpage

\chapter{???}
\section{???}

Notación: si $F$ y $E$ son cuerpos tales que $F⊆E$ diremos que $E$ es una extensión de $F$. Se escribe $F⊆E$ (obviamente) o también $E/F$.

\begin{defn}[Elemento\IS algebraico]
Sea $E/F$ una extensión de cuerpos (conmutativos). Se dice que un elemento $α∈E$ es algebraico sobre $F$ si existe un polinomio $h(t) ∈ F[t]$ tal que $h(α) = 0$.
\end{defn}

Por ejemplo, $ℚ⊆ℝ$ o $ℚ⊆ℂ$ son extensiones de $ℚ$. $\sqrt{2}$ es algebraico sobre $ℚ$ porque si tomo $h(t) = t^2 - 2$, entonces $h(\sqrt{2}) = 0$.

Es muy difícil encontrar números (reales o complejos) que no sean algebraicos (o trascendentes). Sin embargo hay muchísimos (es un conjunto no numerable).

\begin{prop} SI $α∈E$ es algebraico sobre $F$ existe un único polinomio mónico de grado mínimo $p(t) ∈ F[t]$ tal que $p(α) = 0$. 

A $p(t)$ se le llama el polinomio mínimo o irreducible de $α$.

Además, $p(t)$ es un polinomio irreducible.
\end{prop}

\begin{proof}
Sea $h(t) ∈ F[t]$ mónico tal que $h(α) = 0$. Dividiendo $h(t)$ entre $p(t)$ podemos escribir \[ h(t) = q(t) · p(t) + r(t) \] donde $\deg(r(t)) < \deg(p(t))$. Si evalúo en $α$, tenemos que \[ h(α) = q(α)p(α) + r(α) \] pero sabemos que $h(α) = p(α) = 0$, luego obligatoriamente tiene que ser $r(α) = 0$.

De ahí se deduce que $r(t)$ es cero, porque habíamos dicho que $p(t)$ era el polinomio de grado mínimo que anulaba a α. La única forma de que $r(α)$ sea 0 y que además tenga menor grado que $p(t)$ es que $r(t) = 0$.

Por otra parte, tenemos que $h(t) = q(t)p(t)$, y si suponemos que $h(t)$ es de grado mínimo entre los que anulan a α, entonces $h(t) = p(t)$ luego $p$ es único.

Además, es irreducible porque si existiese $p(t) = p_1(t) p_2(t)$, tendríamos que $0 = p(α) = p_1(α) p_2(α)$. Si suponemos $p_1(α) = 0$, entonces $\deg p(t) = \deg p_1(t)$.
\end{proof}

\paragraph{Ejemplos:} 

\begin{itemize}
\item Polinomio mínimo de $\sqrt{2}$ sobre $\rac \to p(t) = t²-2$ Es irreducible en $\rac$ (podemos aplicar también el criterio de Einsestein)

\item El polinomio mínimo de $x\in\mathbf{F_3}(x) = $ cuerpo de funciones racionales sobre $\frac{\mathbf{Z}}{3\mathbf{Z}}$

La respuesta es $p(t) = t³ - x³ \in \mathbf{F_3}(x³)[t]$.

Desde luego, $p(x) = x³-x³ = 0 $.
Ahora tenemos que comprobar si $p(t)$ es irreducible, que es preguntarse si tiene raices en $\mathbf{F_3}(x³)$.

Los polinomios de $\mathbf{F_3}(x)$ son $
\frac{\sum a_ix^{3i}}{\sum b_ix^{31}} | a_i,b_i \in \mathbf{F}$

Tenemos: $p(t) = t³ - x³ = (t - x)³$ \footnote{Aplicando el binomio de Newton y utilizando que el 3 es el 0 en este cuerpo, los términos $3a²b$ y $3ab²$ se cancelan}. $\implies$ p(t) tiene una única raíz $\mathbf{F_3}(x)$ que es x, pero $x\notin \mathbf{F_3}(x³) \implies p(t)$ no tiene raices en $\mathbf{F_3}(x) \implies p(t)$ irreducible.

\end{itemize}


\begin{corol} El ideal \[ I = \{ h(t) ∈ F[t] \tq h(α) = 0 \} ⊆ F[t] \] coincide con $(p(t))$, es decir, con el ideal generado por los múltiplos de $p(t)$.
\end{corol}

Considero el homomorfismo $\appl{φ}{F[t]}{E}$, con \[ \img φ = K[α] = \left\lbrace\sum a_i α^i \tq a_i ∈ F \right\} ⊆ E \]

$k(α)$ es el subcuerpo de $E$ más pequeño que contiene a $F$ y a $α∈E$. De hecho \[ k(α) = \left\{ \frac{\sum a_i α^i}{\sum b_i α^i} \tq a_i, b_i ∈ F \right\} ⊆ E \]

\begin{theorem} Si $α$ es algebraico, entonces $k(α) = k[α]$, es decir, $k[α]$ es un cuerpo.
\end{theorem}

\begin{proof} Consideramos el homomorfismo $F[t] \mapsto^φ E$. Según el primer teorema de isomorfía, $F[t] / \ncl φ $ es isomorfo a $\img φ = k[α]$. 

Pero $\ncl φ = I = (p(t))$. 
\end{proof}

Por ejemplo, $ℚ[\sqrt{2}]$ es un cuerpo, de hecho $ℚ[\sqrt{2}] = \frac{ℚ[t]}{(t^2 -2)}$. Pero, ¿cuál es el inverso de $1 + \sqrt{2}$? Es $\frac{1}{1+\sqrt{2}}$ que está en $ℚ[\sqrt{2}]$ porque lo podemos expresar como $\sqrt{2} - 1$.

En la extensión $\fd_3(t) / \fd_3 (t^3)$, con $\fd_3 = ℤ/3ℤ$, tenemos que $\fd_3(t^3) ⊆ \fd_3(t)$, y que $t$ es algebraico sobre $\fd_3(t^3)$, el polinomio irreducible es $p(x) = x^3 - t^3$.

\begin{defn}[Grado\IS de una extensión]
Sea $E/F$ una extensión de cuerpos.
El grado es la dimensión de $E$ como espacio vectorial sobre $F$, es decir, \[[E:F] = \dim_FE \]
\end{defn}

\paragraph{Ejemplos}
\begin{itemize}
\item $[\mathbb{C} : \real] = 2$
\obs $\mathbb{C} = \real(i)$ ¿Cuál es el polinomio mínimo de $i$ sobre $\real$? 

$p(t) = t² + 1$
\item $[\rac(\sqrt{2}) : \rac)] = 2$, y una base es $\{1,\sqrt{2}\}$ y entonces el  polinomio mínimo es $p(t) = t²-2$.
\end{itemize}

En estos ejemplos se puede comprobar la siguiente proposición:

\begin{prop}
Sea $\alpha$ algebraico sobre $F$ y sea $p(t)$ su polinomio mínimo (que es también irreducible). Entonces, $[F(\alpha) : F] = n = \deg(p(t))$.

Además, $\{1,\alpha,\alpha²,\alpha³,...,\alpha^{n-1}\}$ es una base de $F(\alpha)$ sobre $F$.
\end{prop}

\begin{proof}
Tenemos una combinación lineal dependiente: $a_0 + a_1\alpha + \dotsb + a_{n-1}\alpha^{n-1} = 0, a_i\in F \implies$ el polinomio $h(t) = a_0 + a_1t + \dotsb. + a_{n-1}t^{n-1}$ anula a $\alpha \implies a_i=0, \forall i$ lo que es una contradicción, porque el grado mínimo tiene que ser $n$ y en este caso es $n-1$.


Además tienen que ser generadores. 

$E=\gen{1,\alpha,\alpha^2,\dotsc,\alpha^{n-1},\alpha^{n},\alpha^{n+2},\dotsc}$. Vamos a ver que de este conjunto de generadores me sobran a partir de $\alpha^n$.

Como $0 = p(\alpha) = a_0 + a_1\alpha + ... + a_{n-1}\alpha^{n-1} + \alpha^n \implies \alpha^n = -(a_0 + a_1\alpha + \dotsb + a_{n-1}\alpha^{n-1})$, es decir, podemos expresar los grados superiores como combinación lineal de los anteriores. 

Hemos demostrado que  $\{1,\alpha,\alpha^2,\dotsc,\alpha^{n-1}\}$ son generadores e independientes por lo que son una base.
\end{proof}

Vamos a afianzar un poco el concepto de dimensión. \begin{itemize}
\item ¿Qué significa $[E:F] = 1$? $\dimplies E=F$.
\item ¿$[\rac(\sqrt{2} : \rac]$? 2, por ser $p(t) = t^2 -2$ el polinomio mínimo.
\item $[\rac (\sqrt{2},\sqrt{3}) % %}_{\text{ Minimo cuerpo que contiene \uparrow}}
:\rac]$
Para resolver esto vamos a utilizar la siguiente proposición:
\end{itemize}

\begin{prop}
$F \subset E_1 \subset E_2$ extensiones finitas.

Entonces $[E_2 : F] = [E_2 : E_1][E_1 : F]$.

Además, $\{y_i\}_{i=1}^n$ es una base de $E_2 / E_1$ y $\{x_j\}_{i=1}^n$  es una base de $E_1 / F$. Entonces $\{x_jy_i\}$ es una base de $E_2/F$
\end{prop}

\begin{proof}
\begin{enumerate}
\item $\{x_jy_i\}$ son generadores de $E_2 / F$.

$$\forall \alpha \in E_2 \implies \alpha = \sum_{i=1}^n b_iy_i; b_i\in E_1 \implies \alpha \sum_{i=1}^n \left(\sum_j a_{ij}x_j\right) y_i = \sum a_{ij}x_jy_i; a_{ij}\in F$$

\item $\{x_jy_i\}$ son independientes sobre $F$.

\[0 = \sum a_{ij}x_iy_i = \sum_i \left(\sum_j a_{ij}x_j\right)y_i \implies \forall \sum_i a_{ij}x_j = 0 \overset{(1)}{\implies} a_{ij} = 0
\]

(1) Por ser $\{x_j\}$ base de $E_1$.
\end{enumerate}
\end{proof}


Para el caso anterior tenemos:

\[
\rac \subset \rac(\sqrt{2}) \subset \rac(\sqrt{2},\sqrt{3}) = \rac(\sqrt{2})(\sqrt{3})
\]

¿Podría ser $[\rac(\sqrt{2})(\sqrt{3}): \rac(\sqrt{2}) = 1 ]$?

Si lo fuera $\implies \sqrt{3} = a+b\sqrt{2} ;\, a,b\in\rac \implies 3=a^2+2b^2+2ab\sqrt{2} \implies \sqrt{2}\in \rac$.

Luego \[[\rac(\sqrt{2},\sqrt{3}): \rac] = [\rac(\sqrt{2},\sqrt{3}): \rac(\sqrt{2})] [\rac(\sqrt{2}): \rac] = 4\]

Veamos otro \textbf{ejemplo}

$[\rac(\sqrt{2} + \sqrt{3}) : \rac] \leq 4$

Utilizando $\gamma = \sqrt{2} + \sqrt{3}$ vamos a intentar calcular el polinomio mínimo, que sabemos que su grado, como máximo, es 4 (porque el grado de la extensión al ser generada por un elemento si coincidirá con el del polinomio mínimo).

$\gamma^2 = 2+3+2\sqrt{6}$

$\gamma^3 = \gamma^2 · \gamma = ... = 11 \sqrt{2} + 9\sqrt{3} + $

Al calcular $\gamma^4$ deberíamos encontrar una relación con las anteriores.

$\gamma^4 = ... = 49 + 20\sqrt{6}$


Ahora el procedimiento general sería resolver $a_0 + a_1\gamma + a_2 \gamma^2 + a_3\gamma^3 + \gamma^4 = 0$

En este caso no lo vamos a hacer porque combinando $\gamma^4$ y $\gamma^2$ podemos eliminar los radicales.

$\gamma^4 = -10 \gamma^2 = -1$. Esto quiere decir que $t^4 - 10t^2 + 1$ es el polinomio mínimo.

Vamos a resolverlo también con el método general para otros casos más complicados.

Buscamos $p(t) = a_0+a_1\gamma+a_2\gamma^2 + a_3\gamma^3 + \gamma^4, a_i\in\rac$.

Sustituyendo  los $\gamma^i$ calculados anteriormente obtenemos: $a_0 + a_1() + a_2() + a_3 + $

\[
\begin{array}{ccc}
(1) & 0 =& a_0 + 5a_2 + 4\\
(\sqrt{2}) & 0 =& a_1 + 11a_3\\
(\sqrt{3}) & 0 =& a_1 + 9a_3\\
(\sqrt{2}\sqrt{3}) & 0 =& 2a_2 + 20
\end{array}
\]

Si resolvemos este sistema (comprobarlo) nos debería dar $a_0 = 1, a_1=a_3=0,a_2 = -10$

\paragraph{Conclusiones}
De este cálculo se deduce que \[\rac(\sqrt{2} + \sqrt{3}) \subseteq \rac(\sqrt{2},\sqrt{3}) \impliedby (\gamma^3 - 9\gamma = 2\sqrt{2} \implies \sqrt{2}\in\rac(\sqrt{2} + \sqrt{3}))\]

Como los cuerpos son iguales $\implies [\rac(\sqrt{2}+\sqrt{3}) : \rac] = 4 \implies p(t)$ es irreducible.


\begin{defn}[Extensión algebraica]
$F\subset E$ es algebraica cuando todos los elementos de $E$ son algebraicos sobre $F \dimplies \forall \alpha \in E [F(\alpha) : F] < \infty$.
\end{defn}
\obs
$[E:F]<\infty \implies E/F (F \subset E)$ es algebraica.

El recíproco no es cierto, es decir, puede haber extensiones algebraicas de dimensión infinita. 

\paragraph{Contraejemplo} $K = \rac(\sqrt{2},\sqrt[3]{2},\sqrt[5]{2},...,\sqrt[p]{2},...)$

Esta extensión $K/\rac$ es algebraica porque cualquier extensión $F\subset F(\alpha_1,\alpha_2,...) = \bigcup_{n=1}^{\infty} F(\alpha_1,...,\alpha_n)$ con $\alpha_i$ algebraico sobre F, es algebraico.

\textbf{Razón} $\forall \alpha \in K  = \bigcup_{n=1}^{\infty} F(\alpha_1,...,\alpha_n) \implies \exists n \tlq \alpha \in F(\alpha_1,...,\alpha_n)$.

$$[F(\alpha) : F] \leq$$
$$[F(\alpha_1,...,\alpha_n) : F] = [F(\alpha_1,..,\alpha_n): F(\alpha_1,..,\alpha_{n-1})] [ F(\alpha_1,..,\alpha_{n-1}) : F(\alpha_1,..,\alpha_{n-2})]...[F(\alpha_n):F] $$
$$\leq\footnote{$F(\alpha_i)\subseteq F(\alpha_1,...,\alpha_{n-i})$. Un cuerpo está contenido en el otro, entonces los grados del polinomio mínimo de cada uno cumplirán $p_1(t) \leq p_2(t)$} \prod [F(\alpha_i):F] < \infty$$

Una vez hemos demostrado (no se muy bien cómo) que es algebraica, vamos a demostrar que su dimensión es infinita.

Si tomamos $[K:\rac] \geq [ \rac(\sqrt[p]{2}):\rac] = p$ pues el polinomio irreducible de $\sqrt[p]{2}$ es $t^p - 2$.

Tenemos que $[K:\rac] \geq M$ donde $M$ es arbitrariamente grande, porque hay infinitos primos.


\begin{example}
\begin{itemize}
\item $[\rac(\sqrt{2}) : \rac] = 2$
\item \[[\rac(\sqrt{2},\sqrt[3]{2})] = \underbrace{[\rac(\sqrt{2},\sqrt[3]{2}):\rac(\sqrt{2})]}_{\leq 3 (1)} \underbrace{[\rac(\sqrt{2}):\rac]}_{= 2} = 2\cdot 3\]

(1) $\to [\rac(\sqrt[3]{2}): \rac] = 3 \implies  [\rac(\sqrt{2},\sqrt[3]{2}): \rac(\sqrt{2})] \leq 3$ porque el polinomio mínimo puede ser que contenga raices en $\sqrt{2}$ (entonces será menor) o puede ser que no, entonces será igual y el polinomio será $t^3-2$

\begin{itemize}
\item $[\rac : \rac(\sqrt[3]{2})] = 3$
\item $[\rac(\sqrt[3]{2}):\rac(\sqrt{2},\sqrt[3]{2})] = m \leq 2$
\item $[\rac : \rac(\sqrt{2})] = 2$
\item $[\rac(\sqrt{2}):\rac(\sqrt{2},\sqrt[3]{2})] = l \leq 3$
\end{itemize}
\[m=l \implies 3m=2l \implies\left\{ \begin{array}{cc}
m=2\\l=3\end{array}\right.\]
\end{itemize}
\end{example}

\begin{example}
$[\rac(\sqrt{2},\sqrt[3]{2},\sqrt[5]{2}):\rac] = 2\cdot 3\cdot5$

Se deja como ejercicio para el lector la comprobación siguiendo el mismo razonamiento que antes.

\end{example}

\paragraph{Conclusión:} Existen extensiones infinitas algebraicas.

\begin{corol}
Una extensión generada por elementos algebraicos es algebraica.
\end{corol}

\begin{defn}[Cuerpo algebraicamente cerrado]
Sea $F$ un cuerpo.

$F$ es algebraicamente cerrado si todo polinomio $h(t) \in F(t), deg(h) \geq 1$, tiene alguna raiz en $F$.
\end{defn}
\begin{example}
$\rac,\rac(\sqrt{2}),\real$ no son algebraicamente cerrados, porque $t^2+1$ no tiene raices. En cambio, $\mathbb{C}$ si es algebraicamente cerrado (esto es el teorema Fundamental del Álgebra).
\end{example}

\begin{prop}
Sea $F\subset E$ una extensión de cuerpos tal que $E$ es algebraicamente cerrado. 

Consideremos $\bar{F} = \{x\in E \tq x \textbf{es algebraico sobre } F$

Entonces:
\begin{enumerate}
\item $\bar{F}$ es un cuerpo.
\item $\bar{F} / F$ es algebraica.
\item $\bar{F}$ es algebraicamente cerrado.
\end{enumerate}
\end{prop}

\begin{proof}

\begin{enumerate}
\item Sean $x,y\in\bar{F}$. Vamos a comprobar las propiedades de cuerpos: $x+y\in \bar{F},xy\in\bar{F}x^{-1}\in\bar{F}$


\begin{gather*}
x,y \in \bar{F} \implies 
\begin{cases}
[F(x):F]≤ \infty\\
[F(y):F]≤ \infty
\end{cases} \implies \\
 [F(x,y) :F] = \underbrace{[F(x,y) :F(x)]}_{\leq [F(x):F]} [F(x,y):F(y)] < \infty \end{gather*}

Tenemos que $F(x,y)/F$ es algebraico. Además, $x+y,xy,x^{-1}\in F(x,y)$, por lo que $\bar{F}$ es un cuerpo.

\item $\bar{F}$ está formado por elementos algebraicos, entonces es algebraica.
\end{enumerate}
\end{proof}


\end{document}