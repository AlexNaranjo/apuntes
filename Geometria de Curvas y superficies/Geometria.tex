\documentclass[nochap]{apuntes}

\usepackage{changepage}
\title{Geometría de curvas y superficies}
\author{Por definir}
\date{13 / 14 C2}

\usepackage{tikz}

\begin{document}

\maketitle
\newpage

\section{Introducción}

\subsection{Parametrización de curvas}
Existen varias formas de representar curvas en $\real^2$.

\begin{defn}[Curva\IS parametrizada diferenciabble]
Es una aplicación diferenciable $\appl{sigma}{I=(a,b)}{\real^n}$
\end{defn}
Vamos a ver varios ejemplos de curvas parametrizadas:

\begin{example}
Definimos la curva $x^3+y^3 = xy$ y la parametrizamos utilizando la indicación $t = \frac{x}{y}$.

Tras unos cuantos cálculos llegamos a \[\sigma(t)=\left(\frac{t}{1+t^3},\frac{t^2}{1+t^3}\right) \]

Nos damos cuenta que en $t=-1$ no está definida. Vamos a estudiar a ver que sale. Nos fijamos en el intervalo $(-1,1]$.Si $t\to-1^{-1}$ Estudiamos la asíntota


Además es interesante estudiar la simetría. Suponemos que puede ser simétrica respecto de la recta $y=x$. Para ver si una curva parametrizada es simétrica respecto de la recta estudiamos qué pasa con $\sigma\left(\frac{1}{t}\right)$. 

Sustituímos y vemos que $(x(t),y(t)) = (y(1/t),x(1/t))$ con lo que concluimos que es simétrica.

\end{example}
Folium de Descartes (análisis completo)

\[ x^3+y^3 = 3axy \]

con $a$ número real, que corresponde a la asíntota que tiene la función. Podemos obtener una parametrización del folium a través de una aplicación $\appl{F}{I⊆ℝ}{ℝ^2}$. El conjunto de partida $I$ se divide en dos subintervalos que tienen su correspondiente imagen: $I = I_1 \cup I_2 = (-∞, -1) \cup (-1,∞)$. La parametrización sería

\[ φ(t) = \left(\frac{t}{1+t^3} ,\frac{t^2}{1+t^3}\right) \]

El folium tiene una asíntota, de ecuación \[0=x+y+\frac{1}{3} = \frac{1}{9} \frac{(t+1)^2}{t^2-t+1}\]. También es simétrica por la recta $y=x$. Por otra parte, cuando $t\to ∞$, $σ(t)\to (0,0)$.

\todo{Dibujitos de lo que vale para cada $t$ y los intervalos.}

Es una parametrización regular de la curva, ya que $σ'$ es distinta de cero siempre.

Otro ejemplo:

\begin{example}
Vamos a comparar las dos curvas siguientes: 

\begin{gather*}
C_1=\{ (x,y)\tq e^x-y^3=0 \}\\
C_2=\{(x,y)\tq e^x-y^2=0\}
\end{gather*}

Podemos parametrizar $C_1$ con una aplicación $\appl{σ}{ℝ}{ℝ^2}$ tal que $σ(x)=(x,e^{x/3})$.

\todo{Dibujito de $C_1$}

En el caso de $C_2$, tenemos dos ramas debido a la raíz cuadrada. Aquí tendríamos que $y=\pm e^{x/2}$. Es decir, que a partir de un conjunto podemos sacar una curva conexa o no.
\end{example}

\begin{example}[Parábola semicúbica]
Tomamos la parábola semicúbica con la aplicación 
\begin{align*}
\appl{σ}{ℝ&}{ℝ^2} \\
σ(t) &= (t^3, t^2)
\end{align*}

Esta curva tiene un pincho (\textit{sic}) en el origen, ya que $σ'$ se anula cuando $t=0$.

\end{example}

\begin{defn}[Velocidad\IS de una trayectoria] Dada una curva parametrizada de $ℝ^n$  $\appl{σ}{I}{ℝ^n}$, el vector velocidad de la trayectoria en $σ(t)$ es $σ'(t)$, su derivada.
\end{defn}

\begin{defn}[Rapidez\IS de una trayectoria] Dada una curva parametrizada de $ℝ^n$  $\appl{σ}{I}{ℝ^n}$, la rapidez de la trayectoria es $\md{σ'}$, el módulo su derivada.
\end{defn}

\begin{example}[Cúbica nodal]La cúbica nodal es una curva plana que viene dada por el siguiente conjunto:

\[ C = \{ (x,y)\tq y^2=x^2(x+1) \} \]

\todo{Otro dibujito}

Para hallar la parametrización, consideramos $t=\frac{x}{y}$ como parámetro, y nos queda que

\[ α(t) = (t^2-1,t(t^2-1)) \]

, que es regular ya que $α'$ no se anula. Sin embargo, esta parametrización no es inyectiva (por ejemplo, $α(t) = (0,0) \implies t= \pm 1$). Dividimos la curva en dos partes que se solapan: 

\[ α_1(t) = \eval[2]{α(t)}_{t> -\frac{1}{2}},\qquad α_2(t) = \eval[2]{α(t)}_{t<\frac{1}{2}} \]

\todo{Dibujito de cada una de las ramas. De Juan, esto te pasa por escaquearte y querer hacer los dibujitos xD}

¿Qué ganamos con esto? No lo sé, salvo varias preguntas estúpidas (Parra y Pablo), mucho rato tratando de definir qué es solapamiento y nada útil.
\end{example}

Podemos tener varios \textit{accidentes} en una parametrización:

\begin{enumerate}
\item Que la parametrización no sea inyectiva.
\item Que sea inyectiva pero su inversa no sea continua.
\end{enumerate}

Por lo tanto, buscaremos un tipo de parametrizaciones \textit{buenas}:

\begin{defn}[Parametrización\IS bicontinua] Se dice que una parametrización es bicontinua si ella y su inversa son continuas.
\end{defn}

Por la propia definición, una parametrización regular es localmente bicontinua. 

\begin{defn}[Longitud\IS de arco] Sea $\appl{σ}{(a,b)}{ℝ^N}$ una trayectoria de clase $C^1$. Entonces la longitud de arco de σ está definida por 

\[ L(σ) = \int_a^b \md{σ'(t)} \dif t \]
\end{defn}

La longitud de arco nos permite definir una parametrización para cualquier curva:

\begin{defn}[Parametrización\IS por longitud de arco] Se dice que una curva $\appl{σ}{I}{ℝ^n}$ está parametrizada por longitud de arco si $\md{σ'(s)} = 1\;\forall s∈I$.
\end{defn}

Igualmente, también querremos crear una reparametrización que nos lleve una parametrización regular cualquiera a una parametrización por longitud de arco.

\begin{prop} Para toda curva parametrizada regular $\appl{σ}{I}{ℝ^N}$ existe una transformación de parámetros que conserva la orientación , $h$, tal que $σ○h$ está parametrizada por longitud de arco.
\end{prop}

\begin{proof} Para reparametrizar, buscamos cambiar cómo nos movemos por el parámetro sin variar el conjunto imagen en $ℝ^3$, según el esquema de \ref{figParamLongArco}.

\begin{wrapfigure}{r}{0.4\textwidth}
\centering
\begin{tikzpicture}
\node (I) at (0,0) {$t∈I⊆ℝ$};
\node (J) at (0,-3) {$s∈J⊆ℝ$};
\node (R) at (3,0) {$ℝ^N$};

\draw[->] (J) -- node[left] {$f$} (I);
\draw[->] (I) -- node[above] {$σ$} (R);
\draw[->] (J) -- node[below right] {$β=σ○f$} (R);
\end{tikzpicture}
\caption{Reparametrización por longitud de arco.}
\label{figParamLongArco}
\end{wrapfigure}

Buscamos que $f$ sea un difeomorfismo (esto es, que exista su inversa y que sea diferenciable). 

La reparametrización por arco de $σ$ es la aplicación $β(s)$ construida como en (\ref{figParamLongArco}) tal que $\md{β'(s)}=1$.

Sabemos que 

\[ β'(s) = \deriv{β}{s}(s) = \deriv{σ}{t}(f(s))\cdot \deriv{f}{s}(s) \]

y por lo tanto
\[ \abs{f'(s)} = \frac{1}{\md{α'(f(s))}} \]

Entonces necesitamos que $σ'(t)≠0\; ∀t∈I$ (que sea regular). Además, si $f(s)$ es creciente 

\[ f'(s) = t'(s) = \frac{1}{\md{σ'(t(s)}} \implies s'(t) = \md{σ'(t)} \]

y nos quedaría

\[ s(t) = \int_{t_0}^t \md{σ'(t)}\dif t \]

que es la reparametrización que buscamos.

\end{proof}

\begin{lemma} Sean $\appl{σ_1}{I_1}{ℝ^N}$, $\appl{σ_2}{I_2}{ℝ^N}$ dos parametrizaciones por arco de la misma curva. Entonces la transformación de parámetros correspondiente $\appl{h}{I_1}{I_2}$ tal que $σ_1=σ_2○h$ es de la forma $h(s) = \pm s + s_0$ con $s_0∈ℝ$ constante. $s$ será positivo si la orientación es compatible y negativa si la orientación de ambas parametrizaciones es opuesta.
\end{lemma}

\begin{proof} Sabemos que 

\[ 1 = \md{σ_1'(s)} = \md{(σ_2○h)'(s)} = \md{σ_2'(h(s))\cdot h'(s)} = \md{σ_2'(h(s))} ·  \abs{h'(s)} = \abs{h'(s)} \]

Integrando $\abs{h'(s)}$ nos queda que $h(s) = \pm s + s_0$.
\end{proof}

\section{Curvatura}

\begin{defn}[Curvatura] Dada una curva regular en $ℝ^N$, el campo \mv{k} de los vectores curvatura de la misma es el formado por la derivada segunda $β''(s)$, siendo β una parametrización por longitud de arco.
\end{defn}

De forma trivial, se verifica que la derivada segunda \textbf{no depende de la parametrización} por arco elegida, ya que dos parametrizaciones distintas sólo diferían en una constante y en la orientación: $β(s) = α(c\pm s)$ con $c$ constante.

Si además consideramos el campo \mv{t} de las tangentes unitarias de la curva\footnote{$\displaystyle\mv t(s) = \frac{α'(s)}{\md{α'(s)}}$}, tenemos que, independientemente de la parametrización:

\[ \mv k(s) \equiv α''(s) \equiv \mv t'(s) \]

Esto nos lleva a una primera propiedad interesante de la curvatura:

\begin{lemma} Dada una curva regular en $ℝ^N$, su vector curvatura es normal a la curva en cada punto
\end{lemma}

Por otra parte, la curvatura cumple una idea muy sencilla: si su valor es 0, estamos ante una recta. Dicho de otra forma:

\begin{lemma} Una curva regular en $ℝ^N$ es un trozo de recta afín si y sólo si $\mv k = 0$. \end{lemma}

\par 

Hasta ahora hemos tratado la curvatura sobre parametrizaciones por longitud de arco. ¿Cómo podemos obtenerla si estamos tratando con cualquier otro parámetro α?

Sabemos que podemos escribir una reparametrización de α, así que podemos efectivamente "normalizar" la curvatura y el campo de tangentes:

\begin{gather*}
\mv t(s) = \frac{α'(s)}{\md{α'(s)}} \\
\mv k(s) = \frac{α''(s)}{\md{α'(s)}^2} - \left(\mv t \frac{α''(s)}{\md{α'(s)}^2}\right) \mv t
\end{gather*}

En el caso de \mv{k}, lo que hacemos es quitarle a la normalización la componente tangencial, la que tiene la misma dirección que \mv{t}.\footnote{A grandes rasgos, esto equivale a aplicar la ortogonalización de Gram-Schmidt al par de vectores $\displaystyle\left\{\mv t, \frac{α''(s)}{\md{α'(s)}^2}\right\}$.}

\appendix
\chapter{Apéndices}
\documentclass[nochap]{apuntes}

\usepackage{tikz}

\title{Ejercicios de Estructuras Algebraicas}
\date{2013 / 2014}
\author{Pedro Valero, Victor de Juan, Guillermo Julián, Guillermo Ruiz}

\begin{document}
\maketitle
\newpage
\tableofcontents
\newpage
\section{Hoja 1}
\begin{problem}[1]
 Decide de manera razonada si los siguientes conjuntos son grupos con la operación definida.
 
 \begin{itemize}
  \item a) $(\real,+)$
  \item b) Fijado $n \in \mathbb{Z}_{n>0}$, el conjunto de los enteros módulo $n$ con la suma.
  \item c) $(C^*,\cdot)$
  \item d) $(U (n), ·)$, donde $U (n)$ denota los restos módulo $n$ de enteros coprimos con $n$.
  \item e) Dado un conjunto no vacío $X$, el conjunto $G$ de las biyecciones de $X$ con la composición, $(G, \circ)$.Calcula el cardinal de G si X es un conjunto finito.
 \end{itemize}

 \solution

\spart
$$\left. \begin{matrix}
\text{¿Es asociativo? Sí}\\
\text{¿Tiene elmento neutro? Sí: } \\a + 0 = a, \forall a \in \real\\
\text{¿Existe el inverso de todo elemento? Sí: } \\a^{-1} = -a, \forall a \in \real
\end{matrix}\right\} \implies \text{Sí es un grupo.}$$

\spart
$$\left. \begin{matrix}
\text{¿Es asociativo? Sí}\\
\text{¿Tiene elmento neutro? Sí: } \\\gor{a} + \gor{n} = \gor{a}, \forall \gor{a} \in \mathbb{Z}_n\\
\text{¿Existe el inverso de todo elemento? Sí: } \\ \gor{a} + (\gor{n} - \gor{a}) = \gor{n} = e \implies a^{-1} = \gor{n} - \gor{a}, \forall a \in \real
\end{matrix}\right\} \implies \text{Sí es un grupo.}$$

\end{problem}

\begin{problem}[14] Sea $G$ un grupo finito y $x∈G$. Demuestra que existe $n∈ℕ$ tal que $x^n = e$.
\solution

Consideremos la sucesión $\{x,x^2,\dotsc,x^k+1\}$ con $k=\card{G}$. Como $G$ es finito, por el principio del palomar hay un elemento repetido en esa sucesión. Es decir, existe $α,β$ tales que $x^α=x^β$. Operando, $x = x^{α+β}$ y por lo tanto $x^{α+β-1}$ sería el elemento neutro.
\end{problem}

\section{Hoja 2}
\begin{problem}[4]
Sea G un grupo. Demostrar que \[ Z(G) = \{x \in G \tq xy = yx\; \forall Y \in G\} \] es un grupo. 

\ppart Calcula el centro de $D_3$.

\ppart Calcula el centro de $D_4$.

\solution

Demostramos que es un grupo. Primero vemos que $e \in Z(G)$. Buscamos ahora la existencia del inverso.

Dado $x∈Z(G)$, tenemos que $xy = yx$. Operando 

\begin{align*}
xy &= yx \\
\inv{x}xy\inv{x} &= \inv{x}yx\inv{x} \\
y \inv{x} &= \inv{x} y
\end{align*}

Es decir, que $\inv{x}$ también está en $Z(G)$. Demostramos por último que es una operación cerrada. Siendo $x_1,x_2∈Z(G)$:

\[ x_1 x_2 y = x_1 y x_2 = y x_1 x_2 \]

por lo tanto $x_1x_2 ∈ Z(G)$. Hemos demostrado que $Z(G)$ es un grupo.

\spart
$$Z(D_3), D_3 = \left\{\begin{matrix} 1,a,a^2\\b,ab,a^2b\end{matrix}\right\},\left\{\begin{matrix}a^3 = 1 = b^2\\ ba^j = a^{-j}b\end{matrix}\right\}$$

\spart 
$$Z(D_4) =\left\{\begin{matrix} 1,a,a^2,a^3\\b,ab,a^2b,a^3b\\a^4=1=b^2\\ba=a^{-1}b\end{matrix}\right\}$$

$$Z(D_4) = \left\{x \tq x\in D_4 axa^{-1} = x = bxb^{-1}\right\}$$

Sin sentido...
$a^ib \notin Z(D_4). (a^2 \neq 1)$

$a^i \in Z(D_4) \dimplies (a^{2i} = 1)$

$Z(D_4) = \{1,a^2\} = <a^2>$
\end{problem}

\begin{problem}[5]
Sea $a∈G$. Demuestra que \[ C_G(a) = \{ x∈G\tq xa = ax \} \], el centralizador de $a$ en $G$, es un subgrupo de $G$.

\ppart Demuestra que $Z(G) ≤ C_G(a)$.

\ppart Calcula $C_{D_4}(b)$

\ppart Concluye que la inclusión $Z(G) ≤ C_G(a)$ puede ser estricta.
\solution

Para demostrar que es un subgrupo, vemos que es un subconjunto y que además, dados $x,y∈C_G(a)$:

\begin{align*}
 x\inv{y}a &\qeq ax\inv{y} \\
 x\inv{y}ay &\qeq ax\inv{y}y \\
 x\inv{y}ya &\qeq ax \\
 xa &= ax
\end{align*}

Vemos que $x\inv{y}∈C_G(a)$, por lo tanto es un subgrupo.

\spart Ya sabemos que $Z(G)$ es un grupo: sólo falta demostrar que es un subconjunto. Es trivial dada la definición de centro, que son los elementos que conmutan con todos los elementos de $G$.

\spart

$C_{D_4}(b)$. Basta con comprobar la conmutación con $a^j$ y con $a^jb$ siendo $j = 0,1,2,3$, ya que con eso podemos ver la conmutación con todos los elementos. Se puede demostrar la conmutatividad multiplicando a derecha e izquierda por $b$ y $b^{-1}$ y si nos queda $=1$, es conmutativo.

$$\left\{\begin{matrix}b(a^j)b^{-1} = a^{-j}, a^j \in C_{D_4}(b) \dimplies a^2j = 1\\
b(a^jb)b^{-1} = a^{-j} = a^{-j}b, a^jb \in C_{D_4}(b) \dimplies a^2j = 1\end{matrix}\right.$$
\end{problem}

\section{Hoja 3}


\begin{problem}[1] Subgrupos de $(ℤ,+)$.

\ppart Demuestra que todo subgrupo de $ℤ$ es cíclico, de la forma $kℤ$ para algún entero $k$.

\ppart Demuestra que hay tantos subgrupos en $ℤ$ como enteros no negativos.

\ppart Dados dos enteros positivos $r,s$, demuestra que $rℤ\subset sℤ$ si y sólo si $r$ es un múltiplo de $s$

\ppart Halla todos los subgrupos de $ℤ$ que contienen a $6ℤ$.

\solution

\spart Queremos demostrar que todo subgrupo finito de $\mathbb{Z}$ es cíclico. Como $H$ es un subgrupo, tenemos que $0∈H$ y que $-x∈H\;∀x∈H$. Además, como es operación cerrada, para todo $x∈H$ tenemos que $kx∈H\,∀k∈ℤ$. Buscamos el elemento $b$ más pequeño que no se puede expresar como $b = kx$ con $0 < x < b$ y $k>1$. 

Este elemento $b$ es único. Demostrémoslo: supongamos $H≠Z$ y que otro $b'$ cumple esas condiciones. Podemos plantear la ecuación $kb + kb' = 1$, que tiene solución en todo caso. Por ser operación cerrada, $1∈H$ e igualmente $k∈H\,∀k∈ℤ$, y tendríamos que $H=Z$, contradicción.

Como cualquier otro elemento de $H$ se puede expresar como $kb$ con $k∈ℤ$, tenemos que $H=\gen{b}=bℤ$ y por lo tanto es cíclico.

\spart Trivial.

\spart Queremos demostrar $r\mathbb{Z} \subset s\mathbb{Z} \iff s|r$. Empezamos demostrando la implicación a la derecha, que es trivial si vemos que $r∈rℤ$ se puede expresar como $r=ks∈sℤ$ con $k$ entero.

Por otra parte, si $s|r$ (que podemos expresar como $r=αs$ tenemos que 

\[ x = kr ∈ rℤ \implies x = kαs \implies x ∈ sℤ \]

\spart Conociendo la demostración del apartado anterior, tenemos que los posibles subgrupos son los asociados a sus divisores: $\mathbb{Z}, 2\mathbb{Z}, 3\mathbb{Z}, 6\mathbb{Z}$.
\end{problem}

\begin{problem}[3] Propiedades de los subgrupos normales.

\ppart Demuestra que si $H$ y $K$ son normales en $G$, entonces $H\cap K$ también lo es.

\ppart Si $N \lhd G$ y $N⊆H≤G$, demuestra que 
\begin{enumerate}
\item $N\lhd H$
\item $H/N ≤ G/N$
\item $H/N\lhd G/N \iff H\lhd G$
\end{enumerate}

\ppart Supongamos que $N\lhd G$ y $N⊆H≤G$. Demuestra que

\[ \card{G:H} = \card{G/N : H/N} \]

\solution
Otro día.

\end{problem}

\end{document}


\end{document}