\documentclass[nochap]{apuntes}

\title{Geometría de curvas y superficies}
\author{Por definir}
\date{13 / 14 C2}

\begin{document}

\maketitle
\newpage

\section{Introducción}

\subsection{Parametrización de curvas}
Existen varias formas de representar curvas en $\real^2$.

\begin{defn}[Curva\IS parametrizada diferenciabble]
Es una aplicación diferenciable $\appl{sigma}{I=(a,b)}{\real^n}$
\end{defn}
Vamos a ver varios ejemplos de curvas parametrizadas:

\begin{example}
Definimos la curva $x^3+y^3 = xy$ y la parametrizamos utilizando la indicación $t = \frac{x}{y}$.

Tras unos cuantos cálculos llegamos a \[\sigma(t)=\left(\frac{t}{1+t^3},\frac{t^2}{1+t^3}\right) \]

Nos damos cuenta que en $t=-1$ no está definida. Vamos a estudiar a ver que sale. Nos fijamos en el intervalo $(-1,1]$.Si $t\to-1^{-1}$ Estudiamos la asíntota


Además es interesante estudiar la simetría. Suponemos que puede ser simétrica respecto de la recta $y=x$. Para ver si una curva parametrizada es simétrica respecto de la recta estudiamos qué pasa con $\sigma\left(\frac{1}{t}\right)$. 

Sustituímos y vemos que $(x(t),y(t)) = (y(1/t),x(1/t))$ con lo que concluimos que es simétrica.

\end{example}
Folium de Descartes (análisis completo)

\[ x^3+y^3 = 3axy \]

con $a$ número real, que corresponde a la asíntota que tiene la función. Podemos obtener una parametrización del folium a través de una aplicación $\appl{F}{I⊆ℝ}{ℝ^2}$. El conjunto de partida $I$ se divide en dos subintervalos que tienen su correspondiente imagen: $I = I_1 \cup I_2 = (-∞, -1) \cup (-1,∞)$. La parametrización sería

\[ φ(t) = \left(\frac{t}{1+t^3} ,\frac{t^2}{1+t^3}\right) \]

El folium tiene una asíntota, de ecuación \[0=x+y+\frac{1}{3} = \frac{1}{9} \frac{(t+1)^2}{t^2-t+1}\]. También es simétrica por la recta $y=x$. Por otra parte, cuando $t\to ∞$, $σ(t)\to (0,0)$.

\todo{Dibujitos de lo que vale para cada $t$ y los intervalos.}

Es una parametrización regular de la curva, ya que $σ'$ es distinta de cero siempre.

Otro ejemplo:

\begin{example}
Vamos a comparar las dos curvas siguientes: 

\begin{gather*}
C_1=\{ (x,y)\tq e^x-y^3=0 \}\\
C_2=\{(x,y)\tq e^x-y^2=0\}
\end{gather*}

Podemos parametrizar $C_1$ con una aplicación $\appl{σ}{ℝ}{ℝ^2}$ tal que $σ(x)=(x,e^{x/3})$.

\todo{Dibujito de $C_1$}

En el caso de $C_2$, tenemos dos ramas debido a la raíz cuadrada. Aquí tendríamos que $y=\pm e^{x/2}$. Es decir, que a partir de un conjunto podemos sacar una curva conexa o no.
\end{example}

\begin{example}[Parábola semicúbica]
Tomamos la parábola semicúbica con la aplicación 
\begin{align*}
\appl{σ}{ℝ&}{ℝ^2} \\
σ(t) &= (t^3, t^2)
\end{align*}

Esta curva tiene un pincho (\textit{sic}) en el origen, ya que $σ'$ se anula cuando $t=0$.

\end{example}

\begin{defn}[Velocidad\IS de una trayectoria] Dada una curva parametrizada de $ℝ^n$  $\appl{σ}{I}{ℝ^n}$, el vector velocidad de la trayectoria en $σ(t)$ es $σ'(t)$, su derivada.
\end{defn}

\begin{defn}[Rapidez\IS de una trayectoria] Dada una curva parametrizada de $ℝ^n$  $\appl{σ}{I}{ℝ^n}$, la rapidez de la trayectoria es $\md{σ'}$, el módulo su derivada.
\end{defn}

\begin{example}[Cúbica nodal]La cúbica nodal es una curva plana que viene dada por el siguiente conjunto:

\[ C = \{ (x,y)\tq y^2=x^2(x+1) \} \]

\todo{Otro dibujito}

Para hallar la parametrización, consideramos $t=\frac{x}{y}$ como parámetro, y nos queda que

\[ α(t) = (t^2-1,t(t^2-1)) \]

, que es regular ya que $α'$ no se anula. Sin embargo, esta parametrización no es inyectiva (por ejemplo, $α(t) = (0,0) \implies t= \pm 1$). Dividimos la curva en dos partes que se solapan: 

\[ α_1(t) = \eval[2]{α(t)}_{t> -\frac{1}{2}},\qquad α_2(t) = \eval[2]{α(t)}_{t<\frac{1}{2}} \]

\todo{Dibujito de cada una de las ramas. De Juan, esto te pasa por escaquearte y querer hacer los dibujitos xD}

¿Qué ganamos con esto? No lo sé, salvo varias preguntas estúpidas (Parra y Pablo), mucho rato tratando de definir qué es solapamiento y nada útil.
\end{example}

Podemos tener varios \textit{accidentes} en una parametrización:

\begin{enumerate}
\item Que la parametrización no sea inyectiva.
\item Que sea inyectiva pero su inversa no sea continua.
\end{enumerate}

Por lo tanto, buscaremos un tipo de parametrizaciones \textit{buenas}:

\begin{defn}[Parametrización\IS bicontinua] Se dice que una parametrización es bicontinua si ella y su inversa son continuas.
\end{defn}

Por la propia definición, una parametrización regular es localmente bicontinua. 

\todo[inline]{Aquí va la demostración mañana de esto. Espero que no vuelva a repetir otra vez que una parametrización regular es localmente bicontinua. Porque lo ha dicho varias veces.}

\begin{defn}[Longitud\IS de arco] Sea $\appl{σ}{(a,b)}{ℝ^N}$ una trayectoria de clase $C^1$. Entonces la longitud de arco de σ está definida por 

\[ L(σ) = \int_a^b = \md{σ'(t)} \dif t \]
\end{defn}


\end{document}