\subsection{Hoja 5}

\paragraph{Primera forma fundamental} Tenemos una superficie $S$, una forma cuadrática

\begin{align*}
\appl{I_p}{T_pS&}{ℝ} \\
v \mapsto \pesc{v,v}
\end{align*}

y una parametrización $\appl{Φ}{U}{S}$ que lleva un punto $(u,v)$ a $p=Φ(u,v)$. $\{Φ_u(u,v),Φ_v(u,v)\}$ es una base del plano tangente $T_pS$. Para hallar la forma fundamental, hallamos ese producto escalar 

\[ I_p(v) = \pesc{aΦ_u + bΦ_v, aΦ_u + bΦ_v} = a^2\underbrace{\pesc{Φ_u,Φ_u}}_{E(u,v)} + 2ab \underbrace{\pesc{Φ_u, Φ_v}}_{F(u,v)} + b^2 \underbrace{\pesc{Φ_v,Φ_v}}_{G(u,v)} \]


\paragraph{Cálculo del ángulo entre curvas} Si tenemos dos curvas $\appl{α,β}{I}{S}$ que se cortan en un punto $p=α(t_0) = β(s_0)$, entonces el coseno del ángulo $θ$ de corte es

\[ \cos θ = \frac{\pesc{α'(t_0), β'(s_0)}}{\norm{α'(t_0)}\norm{β'(s_0)}} \]

Nos pueden dar la primera forma fundamental de una superficie $Φ(u,v)$ de tal forma que 

\begin{align*}
α(t) &= Φ(u_1(t), v_1(t)) \\
β(t) &= Φ(u_2(s), v_2(s))
\end{align*}

Entonces, podemos expresar las derivadas como

\begin{align*}
α'(t_0) &= u_1'(t_0) Φ_u + v_1'(t_0) Φ_v \\
β'(s_0) &= u_2'(s_0) Φ_u + v_2'(s_0) Φ_v 
\end{align*}

y nos queda

\[ \pesc{α'(t_0),β'(s_0)} = \begin{pmatrix}
u_1' & v_1'
\end{pmatrix} 
\begin{pmatrix}
E & F \\
F & G
\end{pmatrix}
\begin{pmatrix}
u_2' \\ v_2 '
\end{pmatrix} \]

Sólo faltaría hallar las normas, que se haría como  \[ \norm{α'(t_0)}^2 = E(u_1')^2 + 2Fu_1'v_1' + G(v_1')^2 \].

En particular, podemos hallar el ángulo entre las curvas coordenadas $α=Φ(t,v_0)$ y $β=Φ(u_0, s)$. En este caso, 
\begin{align*}
α' &= Φ_u \\
β' &= Φ_v 
\end{align*}

y en el numerador nos queda $Φ_uΦ_v = F$. Es decir, que

\[ \cos θ = \frac{F}{\sqrt{EG}} \]

\begin{problem}[2] Hallar la primera forma fundamental de una superficie con parametrización 

\[ Φ(u,v) = (au\cosh v, bu\sinh v, u^2) \]

\solution Para hallar los coeficientes, derivamos:

\begin{align*}
Φ_u &= (a\cosh v, b\cosh v, 2u) \\
Φ_v &= (au\sinh v, bu\sinh v, 0)
\end{align*}

Ahora tenemos que hallar los coeficientes de la primera forma fundamental haciendo los productos escalares:

\begin{align*}
E(u,v) &= \pesc{Φ_u,Φ_u} &= a^2\cosh^2v + b^2\sinh^2v + 4u^2 \\
F(u,v) &= \pesc{Φ_u,Φ_v} &= (a^2+b^2)u\sinh v \cosh v \\
G(u,v) &= \pesc{Φ_v,Φ_v} &= a^2u^2\sinh^2 v + b^2u^2\cosh^2v 
\end{align*}

\end{problem}

\paragraph{Cálculo del área a través de la primera forma fundamental} Supongamos que queremos calcular el área de una región $R=φ(U)$. Normalmente, escribiríamos

\[ \mop{\'Area}(R) = \iint_U \norm{Φ_u× Φ_v}\id{u,v} \]

Con los coeficientes de la primera forma fundamental, esto coincide con

\[ \mop{\'Area}(R) = \iint_U \sqrt{EG - F^2} \id{u,v} \]

\begin{problem}[3] Hallar el área de la superficie de revolución dada por el giro de la curva PPA

\[ α(t) = (0, ρ(t), h(t)) \]

\solution La parametrización de la superficie de revolución con respecto al eje $Z$ será

\[ Φ(u,v) = (ρ(v)\cos u, ρ(v) \sin u, h(v)) \]

Hallamos la primera forma fundamental, primero derivando

\begin{align*}
Φ_u &= (-ρ(v)\sin u, ρ(v) \cos u, 0 ) \\
Φ_v &= (ρ'(v)\cos u, ρ'(v) \sin u, h'(v))\\
\end{align*}

lo que nos da los coeficientes

\begin{align*}
E = ρ^2(v) \\
F = 0 \\
G = ρ'^2 + h'^2 = \norm{α'}^2 \stackrel{PPA}{=} 1
\end{align*}

Vemos que salen valores \textit{curiosos}. Estudiamos $F$, que era $ \pesc{Φ_u,Φ_v}$. Como vale 0, esto indica que los vectores $Φ_u$ y $Φ_v$ son ortogonales, algo que parece muy lógico si vemos que $Φ_u$ es siempre tangente a la circunferencia de revolución, y $Φ_v$ es tangente a la curva.

Hallamos ahora el área:

\[ \mop{\'Area}(S) = \int_a^b \int_0^{2π} \sqrt{ρ^2(v)} \id{u,v} = 2π \int_a^b ρ(v) \dif v \]


\end{problem}

\begin{problem}[5] Tenemos la siguiente parametrización de una superficie $S$

\[ Φ(u,v) = (u\cos v,u \sin v, \log \cos v + u) \]

con $u∈ℝ$, $v∈\left(\frac{-π}{2}, \frac{π}{2}\right)$. Demostrar que el par de curvas $Φ(u_1,v)$, $Φ(u_2,v)$ cortan a cada curva $Φ(u,v_0)$ con igual longitud.

\solution Supongamos que $v$ es el parámetro \textit{vertical} y $u$ el \textit{horizontal}.

\begin{figure}[htbp]
\centering
\begin{tikzpicture}[scale=3]
\coordinate (A) at (0,0) {};
\coordinate (B) at (1,0) {};
\coordinate (C) at (1,1) {};
\coordinate (D) at (0,1) {};

\draw[-] (-0.1,-0.3) edge[out=90,in=270] (A) (A) -- (D) to[out=90,in=270, looseness=1]  (0.1,1.3) (0.1,1.3)  node[above] {$Φ(u_1, v)$};
\draw[-] (0.9,-0.3) edge[out=90,in=270] (B) (B) -- (C) edge[out=90,in=270]  (1.1,1.3) (1.1,1.3)  node[above] {$Φ(u_2, v)$};
\draw[-] (-0.3,0.4) edge[out=0,in=180] (0,0.5) (0,0.5)-- (1,0.5) edge[out=0,in=180]  (1.3,0.6) (1.3,0.6) node[right] {$Φ(u, v_0)$};

\draw[-,green!20!black,very thick] (0,0.5) -- node[midway,below] {$α_{v_0}$} (1,0.5);

\end{tikzpicture}
\end{figure}

Llamamos $α_{v_0} = Φ(t,v_0)$, con $t∈[u_1, u_2]$ a la curva desde la primera a la segunda curva. Su longitud será entonces

\[ L(α_{v_0}) = \int_{u_1}^{u_2} \norm{α_{v_0}'(t)} \dif t \]

Uno diría que ya ha terminado, pero falta calcular el vector tangente. Pero ese es el camino desesperado, nosotros somos listos y vemos que el vector del que hallamos la norma es $Φ_u(t,v_0)$. Entonces

\[ \norm{α_{v_0}'(t)} \pesc{Φ_u(t,v_0), Φ_u(t,v_0)}^{\frac{1}{2}} = E(t,v_0)^{\frac{1}{2}} \]

No hemos simplificado mucho, pero bueno, integramos y a ver qué pasa.

\[ Φ_u(t,v_0) = (\cos v_0, \sin v_0, 1) \implies E = 2 \]

y por lo tanto

\[ L(α_{v_0}) = \int_{u_1}^{u_2} \sqrt{2} \dif t = \sqrt{2} (u_2 - u_1) \]

No hemos ganado mucho, así que vamos a hacer otra curva.
\end{problem}

\begin{problem}[9] En una parametrización $Φ(u,v)$, la 1FF es 

\[ Q = \dif u^2 + 2(u+v) \id{u,v} + e^v \dif v^2 \]

Introduzco nuevas coordenadas $(λ,μ)$ tales que

\begin{align*}
u &= e^λ + μ \\
v &= μ
\end{align*}

Halla $\tilde{Q}$, la 1FF en las coordenadas $(λ,μ)$.

\solution Tenemos dos formas para solucionar. Derivamos

\begin{gather*}
u = e^λ + μ \\
\dif u = e^λ \dif λ + \dif μ \\
\dif v = \dif μ
\end{gather*}

Entonces

\begin{align*}
\dif u^2 &= e^{2λ} \dif λ^2 + 2e^λ \dif λ \dif μ + \dif μ^2 \\
\dif u \dif v &= e^λ \dif λ \dif μ + \dif μ^2 \\
\dif v^2 &= \dif μ^2
\end{align*}

Si $Q = \dif u^2 + 2(u+v) \id{u,v} + e^v \dif v^2$, sustituimos

\[  \tilde{Q} = e^{2λ} \dif λ^2 + 2e^λ \dif λ \dif μ + \dif μ^2 + 2 + \\ (2e^λ + 2μ)(e^λ \dif λ \dif μ + \dif μ^2) + e^μ \dif μ^2
\]
Agrupamos coeficientes:

\[ \tilde{Q} = \underbrace{e^{2λ}}_{\tilde{E}} \dif λ^2 + \underbrace{(2e^λ + (2e^λ + 2μ)e^λ)}_{\tilde{F}}\dif λ \dif μ + \underbrace{(1+ (2e^λ + 2μ) + e^μ)}_{\tilde{G}} \dif μ^2 \]

\paragraph{Segunda forma}

Tenemos $Φ(u,v)$ que nos da los coeficinetes $E, F, G$. Con el cambio de variables al que llamamos $T$ podemos construir $ψ(λ,μ) = Φ ○ T$. Entonces, por ejemplo, $\tilde{E} = \pesc{ψ_λ,ψ_λ}$. Pero

\[ ψ_λ = Φ_ue^λ + Φ_v · 0 = e^λ Φ_u \]

así que sustituimos:

\[ \tilde{E}(λ,μ) = \pesc{e^{2λ}Φ_u, e^{2λ}Φ_u} = e^{2λ}E(e^λ + μ, μ) = e^2λ \]

Etc\ae tera, etc\ae tera.

\end{problem}

\paragraph{Cálculo de isometrías} Tenemos un abierto $U⊆ℝ^2$, con parámetros $(u,v)$, y dos parametrizaciones sobreyectivas

\begin{align*}
\appl{Φ&}{U}{S} \\
\appl{ψ&}{U}{S'}
\end{align*}

Tenemos también una aplicación $\appl{h}{S}{S'}$ tal que $h(Φ(u,v)) = ψ(u,v)$.

Entonces $h$ es isometría si y sólo si la primera forma fundamental de $S$ en la parametrización $φ$ es la misma que la 1ff de $S'$ en la parametrización $Ψ$. Es decir

\begin{align*}
E_Φ (u,v) &= E_ψ(u,v) \\
F_Φ (u,v) &= F_ψ(u,v) \\
G_Φ (u,v) &= G_ψ(u,v) 
\end{align*}.

El esquema de lo que ocurre está en la figura \ref{imgH5E1}. Es un diagrama conmutativo.

\begin{figure}[hbtp]
\centering
\begin{tikzpicture}[x=2cm,y=2cm]
\node (S) at (0,1) {$Φ(u,v) ∈ S$};
\node (S2) at (2,1) {$ψ(u,v) ∈ S'$};
\node (U) at (1,0) {$(u,v) ∈ U$};

\draw[->] (S) -- node[above] {$h$} (S2);
\draw[->] (U) -- (S);
\draw[->] (U) -- (S2);
\end{tikzpicture}
\label{imgH5E1}
\caption{Diagrama de las aplicaciones definidas.}
\end{figure}

Podemos ver que si $\appl{Φ}{U}{S}$ es inyectiva, tengo una inversa $\appl{\inv{Φ}}{S}{U}$ y entonces podríamos reescribir $h$ como $h = ψ ○ \inv{Φ}$.

\begin{problem}[1] Consideramos dos parametrizaxiones

\begin{align*}
\appl{Φ&}{ℝ^2}{S} \\
\appl{ψ&}{ℝ^2}{S'}
\end{align*}
con
\begin{align*}
Φ(u,v) &= \left(u^3+3u, \frac{3}{5}, \frac{3}{5}v^5 - 3v, 4u^2 + \frac{10}{3}v^3 \right)\\
\end{align*}

Demuestra que $\appl{h}{S}{S'}$ con $h(Φ(u,v)) = ψ(u,v)$ existe y es una isometría.

\solution

Tenemos que demostrar que existe la inversa de Φ, así que como $S=Φ(ℝ^2)$ sólo tenemos que demostrar la inyectividad. Buscamos resolver $Φ(u,v) = Φ(\bar{u},\bar{v})$:

\begin{align*}
u^3+3u &= \bar{u}^3 + 3\bar{u} \\
\frac{3}{5}v^5 - 3v &= \frac{3}{5}\bar{v}^5 - 3\bar{v} \\
4u^2 + \frac{10}{3}v^3 &= 4\bar{u}^2 + \frac{10}{3}\bar{v}^3
\end{align*}

¿Cómo lo hacemos? En lugar de resolver el sistema, vemos que en la primera ecuación, la primera derivada es $3u^2 + 3$, que es siempre positivo. Por lo tanto, la primera coordenada es siempre creciente e inyectiva y entonces $u=\bar{u}$. Sustituyendo en la última ecuación, eliminamos la $u$ y nos queda que $\bar{v} = v$.

Para que $h$ sea isometría tiene que ser un difeomorfismo y que preserve la primera forma fundamental. Por lo tanto, necesitamos ver que $ψ$ es biyectiva, que lo es porque sí.

Si hacemos los cálculos de los coeficientes de la 1ff para Φ y ψ, vemos que son iguales y por lo tanto es una isometría.
\end{problem}

