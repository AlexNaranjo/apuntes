\section{Ejercicios}

\subsection{Hoja 1}

\begin{problem}[1] Sean $\appl{α,β}{ℝ}{ℝ^3}$ suaves. Hallar la derivada de cada una de estas funciones

\ppart $\appl{f}{ℝ}{ℝ}$ con $f(t) = \pesc{α(t), β(t)}$.
\ppart $\appl{g}{ℝ}{ℝ^3}$ con $g(t)=  α(t)×β(t)$.
\ppart $\appl{h}{ℝ}{ℝ^3}$ fijada $A$ matriz $3×3$, con $h(t) = Aα(t)$

\solution

\spart Primero desarrollamos $f$ y vemos qué nos sale. Tenemos que

\begin{gather*}
α(t) = \left(α_x(t), α_y(t), α_z(t) \right) \\
β(t) = \left(β_x(t), β_y(t), β_z(t) \right)
\end{gather*}

y por lo tanto tenemos que

\[ f(t) = α_x(t)β_x(t) + α_y(t)β_y(t) +α_z(t)β_z(t) \]

. Derivamos:

\[ f'(t) = α_x'(t)β_x(t) + α_x(t)β_x'(t) + α_y'(t)β_y(t) + α_y(t)β_y'(t) + α_z'(t)β_z(t) + α_z(t)β_z'(t) \]

Sin embargo, queremos reescribir esto en función de las aplicaciones originales α y β. Simplemente nos fijamos que esto es una suma de productos escalares:

\[ f'(t) = \pesc{α'(t),β(t)} + \pesc{α(t),β'(t)} \]

\todo{Apartados b y c}.

\spart

\spart


\end{problem}

\begin{problem}[2] Sea $α(t)$ una curva que no pasa por el origen. Si $α(t_0)$ es el punto de la traza de α más cercano al origen y $α'(t)≠0$, demuestre que el vector de posición $α(t_0)$ es ortogonal a $α'(t_0)$.

\solution

Tenemos que ver que $\pesc{α(t_0),α'(t_0)} = 0$. En $t_0$ tenemos un mínimo global de la distancia al origen, que es $d(t) = \md{α(t)}$. Usando el resultado del primer apartado del ejercicio anterior, $d(t)=\pesc{α(t), α(t)}^{\frac{1}{2}}$. Para ahorrarnos la raíz cuadrada, minimizamos la distancia al cuadrado $c(t)=\pesc{α(t), α(t)}$:

\[ c'(t)= 2\pesc{α'(t), α(t)} \]

Igualamos a cero:

\[ 0 = 2\pesc{α'(t), α(t)} = \pesc{α'(t), α(t)} \]

que es lo que queríamos demostrar. Además, como α no pasa por el origen y su derivada no es nula, sabemos que $d$ es diferenciable en todo punto (la raíz cuadrada no es diferenciable en el 0).
\end{problem}

\begin{problem}[5]
La curva engendrada por un punto P de una circunferencia de radio $r$ que rueda sin deslizar por una recta fija se llama cicloide. Tomando dicha recta como eje de las $X$, y como parámetro $t$ el ángulo orientado $MCP$ ($C$ es el centro de la circunferencia, y $M$ el punto de contacto con el eje), pruebe que la posición de $P$ para cada $t$ es
\[ α(t) = (rt − r \sin t, r − r \cos t) \]

Se ha supuesto que en $t = 0$, $P$ coincide con $M$, y con el origen de coordenadas. Determine los puntos t donde $α′(t) = 0$ (llamados de retroceso). (Nota: "sin deslizar" significa a efectos prácticos que la longitud del arco $MP$ coincide con la longitud del segmento $OM$).

\solution

\easyimg{img/Hoja1_5_Cicloide.png}{Cicloide.}{figEj5_Cicloide}

Viendo el dibujo \ref{figEj5_Cicloide} podemos ver claramente la definición de la parametrización, que es la que nos dan. Calculamos ahora $α'$:

\[ α'(t) = (r-r\cos t, r\sin t) \]

Igualando a 0, tenemos que $\cos t = 1$ y que $\sin t = 0$. Por lo tanto, los puntos en los que $α'(t)=0$ son de la forma $t=2kπ$, es decir, $(2r,0)$.


\end{problem}

\begin{problem}[8] Parametrizamos la espiral logarítima como

\begin{align*}
\appl{α&}{ℝ}{ℝ^2} \\
α(t) &= (e^{bt} \cos t, e^{bt} \sin t)
\end{align*}

con $b < 0$. Demostrar que cuando $t\to ∞$, $α(t)$ se acerca al origen. Calcule, para cada intervalo $[t_0, t_1] ⊆ ℝ$, la longitud de arco de $α$. Halle las ecuaciones de la parametrización por longitud de arco y demuestre que ningún arco suyo es circular.

\solution

\easyimgw{img/Hoja1_8_EspLog.png}{Espiral logarítimica.}{figEspLog}{0.7}

Calculamos $\lim_{t\to ∞} α(t)$. Dado que $\abs{\cos t} ≤ 1\;∀t∈ℝ$,

\[ \abs{e^{bt}\cos t } ≤ \abs{e^{bt}} = e^{bt} \]

y tomando límites, $\lim_{t\to ∞} e^{bt} = 0$ ya que $b$ es negativa. La demostración es análoga para la otra coordenada.

Pasamos ahora a calcular la longitud de cada intervalo. Para $[t_0, t_1] ⊆ ℝ$, tenemos que

\[ L_{t_0}^{t_1}(α)= \int_{t_0}^{t_1}\md{α'(t)} \dif t \]

Para hallar la parametrización por longitud de arco, tomamos un $t_0∈I$ en el dominio de α y calculamos

\begin{equation}\label{eq1_8} s(t) = \int_{t_0}^t \md{α'(τ)} \dif τ \end{equation}

Esta aplicación $s$ va de $I$ a $ℝ$, así que hallamos el intervalo $J=s(I)$. Despejaremos para $t=t(s)$ de (\ref{eq1_8}) y escribo $\appl{β}{J}{ℝ^2}$ con $β(s)=α(t(s))$. Operamos

\begin{align*}
α'(t) &= (be^{bt}\cos t - e^{bt} \sin t, be^{bt}\sin t + e^{bt}\cos t) \\
\md{α'(t)}^2 &= b^2e^{2bt}\cos^2 t + e^{2bt}\sin^2 t - 2be^{bt}\cos t e^{bt}\sin t + \\
& \quad + b^2e^{2bt}\sin^2 t + e^{2bt}\cos^2 t + 2be^{bt}\cos t e^{bt}\sin t = \\
&= b^2e^{2bt} + e^{2bt} = e^{2bt}(b^2+1) \\
\md{α'(t)} &= e^{bt}\sqrt{b^2 +1}
\end{align*}

Una vez que tenemos la expresión del módulo de la derivada, integramos. Elegimos $t_0 = 0$ para facilitar la integración:

\begin{align*}
s(t) &= \int_0^t e^{bτ}\sqrt{b^2 +1} \dif τ = \\
&= \frac{\sqrt{b^2+1}}{b} \left(e^{bt}\right|_0^t = \frac{\sqrt{b^2+1}}{b}(e^{bt}-1)
\end{align*}

Hallamos la imagen:

\[ s(ℝ) = \left(-\frac{\sqrt{b^2+1}}{b}, ∞\right) = J \]

Ahora despejamos $t(s)$:

\begin{align*}
s &= \frac{\sqrt{b^2+1}}{b}(e^{bt}-1) \\
\frac{b}{\sqrt{b^2+1}}s + 1&= e^{bt} \\
\log \left(\frac{b}{\sqrt{b^2+1}}s + 1\right)&= bt \\
t &= \frac{1}{b} \log \left(\frac{b}{\sqrt{b^2+1}}s + 1\right)
\end{align*}

y sustituimos en $α$ para hallar la parametrización por longitud de arco $β$:

\begin{multline*} β(s) = α(t(s))  = \left(\left(\frac{b}{\sqrt{b^2+1}}s + 1\right)\cos \left(\frac{1}{b} \log \left(\frac{b}{\sqrt{b^2+1}}s + 1\right)\right),\right. \\ \left. \left(\frac{b}{\sqrt{b^2+1}}s + 1\right)\cos \left(\frac{1}{b} \log \left(\frac{b}{\sqrt{b^2+1}}s + 1\right)\right)\right) \end{multline*}

\easyimgw{img/Hoja1_8_EspLogArco.png}{Buscamos un arco circular en la espiral logarítmica.}{figEspLogArco}{0.7}

Vamos ahora a comprobar si existe algún arco circular. Buscaremos si existe algún intervalo $I⊆ℝ$ tal que $α(A)$ sea el arco de una circunferencia (no necesariamente centrada en el origen), como se ve en la figura \ref{figEspLogArco}.

Si existiese tal circunferencia, habría algún centro $(a_0,b_0)$ y un radio $r>0$ fijo de tal forma que $∀t∈I\;\md{α(t) - (a_0,b_0)} = r$. La condición se puede expresar mejor como $\md{α(t) - (a_0,b_0)}^2 = r^2$. Operamos:

\[ \md{α(t) - (a_0,b_0)}^2 = (e^{bt}\cos t -a)^2 + (e^{bt}\sin t -b_0)^2 = r^2 \]

Si buscamos que la parte de la izquierda sea constante, derivamos y buscamos cuándo vale 0:

\begin{gather*}
2(e^{bt}\cos t -a_0)(be^{bt}-e^{bt}\sin t) + 2(e^{bt}\sin t - b_0)(be^{bt}\sin t + e^{bt} \cos t) =
\end{gather*}

Y se halla. Normalmente lo haríamos con una función de curvatura, viendo si es constante positiva o negativa.
\end{problem}

\begin{problem}[9] Cisoide.

Copias tú el enunciado.

\solution

Por la forma del problema vamos a escribir todo en polares.

Empezamos con $M_2$. La recta $x=2a$ se reescribe en polares como $r_{M_2}(θ) = \frac{2a}{\cos θ} = 2a \sec θ$.

Por otra parte, para reescribir $M_1$ reescribimos la circunferencia en polares y nos queda que $r_{M_1}(θ) = 2a\cos θ$. 

Para $P$, la distancia es la resta de las anteriores, entonces

\[ r_P(θ) = 2a \sec θ - \cos θ = 2a\left(\frac{1}{\cos θ} - \cos θ\right) \]

con $θ∈(-π/2, π/2)$.  Para hallar la ecuación implícita, 

\end{problem}

\begin{problem}[16] Sea $b$ una constante no nula. Halle una parametrización por longitud de arco de la hélice

\[ α(t) = (a\cos t, a \sin t, bt) \]

\solution

Vemos que las dos primeras coordenadas van a ir describiendo una circunferencia de radio $a$ en el plano $XY$, donde $t$ será el ángulo formado con el eje de las $X$. La última coordenada nos da la altura, que es siempre creciente.

Para reparametrizar, buscamos cambiar cómo nos movemos por el parámetro sin variar el conjunto imagen en $ℝ^3$, según el siguiente esquema:

\begin{figure}[hbtp]
\centering
\begin{tikzpicture}
\node (I) at (0,0) {$t∈I⊆ℝ$};
\node (J) at (0,-3) {$s∈J⊆ℝ$};
\node (R) at (3,0) {$ℝ^3$};

\draw[->] (J) -- node[left] {$f$} (I);
\draw[->] (I) -- node[above] {$α$} (R);
\draw[->] (J) -- node[below right] {$β=α○f$} (R);
\end{tikzpicture}
\caption{Reparametrización.}
\label{figEj1}
\end{figure}

donde buscamos que $f$ sea un difeomorfismo (esto es, que exista su inversa y que sea diferenciable).

La reparametrización por arco de $α$ es la aplicación $β(s)$ construida como en (\ref{figEj1}) tal que $\md{β'(s)}=1$.

Sabemos que

\[ β'(s) = \deriv{β}{s}(s) = \deriv{α}{t}(f(s))\cdot \deriv{f}{s}(s) \]

y por lo tanto
\[ \abs{f'(s)} = \frac{1}{\md{α'(f(s))}} \]

Entonces necesitamos que $α'(t)≠0\; ∀t∈I$ (que sea regular). Además, si $f(s)$ es creciente

\[ f'(s) = t'(s) = \frac{1}{\md{α'(t(s)}} \implies s'(t) = \md{α'(t)} \]

y nos quedaría

\[ s(t) = \int_{t_0}^t \md{α'(t)}\dif t \]

Aplicamos todo esto al ejercicio en el que estamos.

\begin{gather*}
 α'(t) = (-a\sin t, a \cos t, b) \\
 \md{α'(t)} = \sqrt{a^2+b^2}
\end{gather*}

Integramos:

\[ s(t) = \int_0^t\md{α'(t)} \dif t = t\sqrt{a^2+b^2} \]

de tal forma que la inversa es \[ t(s) = \frac{s}{\sqrt{a^2+b^2}} \]

Nuestra reparametrización es

\[ β(s) =α(t(s)) = \left(a\cos \frac{s}{\sqrt{a^2+b^2}},a\sin \frac{s}{\sqrt{a^2+b^2}}, b \frac{s}{\sqrt{a^2+b^2}}\right) \]

\end{problem}

\begin{problem}[19] Si $\appl{α}{I}{ℝ^3}$ es una curva y $\appl{M}{ℝ^3}{ℝ^3}$ es un movimiento rígido de $ℝ^3$, demuestre que las longitudes de $α$ y $M○α$ entre $a$ y $b$ coinciden.

\solution

Un movimiento rígido conserva el movimiento escalar. Pasa algo. Y sale que sí. Hallas productos escalares, longitudes de arco y toda la pesca.

\end{problem}

\subsection{Hoja 2}


\begin{problem}[2] \textbf{Tacnodo vertical.} Sea $C$ la curva definida implícitamente por $y^4=x^2(x+1)$ en todo el plano $xy$. Demuestre que la parte de $C$ cercana al punto $(0,0)$ es la unión 

\[ \{ x= h_1(y)\} \cup \{ x=h_2(y) \} \]

de los grafos de dos funciones suaves $h_1,h_2$, definidas en $-ε<y≤ε$ y tales que $y=0$ es mínimo local de $h_1$ y máximo local de $h_2$.

Indicación: estudie si $φ(x) \equiv x\sqrt{x+1} $ tiene inversa cerca de $x=0$. Dibuje $C$ y demuestre que $C\cap\left( ℝ^2\setminus\{(0,0)\}\right)$ es suave y sin autointersecciones.

\solution

\easyimg{img/Hoja2_2_Tacnodo.png}{Tacnodo vertical}{imgTacnodo}

Consideramos $\appl{F}{ℝ^2}{ℝ}$ dada por 

\[ F(x,y) = y^4 - x^2(x+1) \]

Está claro que $C\cap ℝ^2\setminus (0,0) = \inv{F}(0)$. Calculemos el gradiente de $F$:

\[ \grad F (x,y) = \left(-3x^2-2x, 4y^3\right) =\left(x(-3x-2), 4y^3 \right) \]

Queremos ver si el gradiente se anula en algún punto de nuestra curva. Vemos que $\grad F = 0$ en dos casos:

\[ (x,y) = (0,0);\quad (x,y) = \left(-\frac{2}{3}, 0\right) \]

Sin embargo, $(0,0)$ no está en la curva, y dado que $F \left(-\frac{2}{3}, 0\right) = -\frac{4}{27} ≠ 0$  $\left(-\frac{2}{3}, 0\right)$ tampoco está. Por lo tanto, la función es \textbf{suave y sin autointersecciones}.

Vamos ahora a demostrar la segunda parte del problema. Buscamos dos funciones $h_1$ y $h_2$. Nos piden que las funciones dependan de $y$ y no de $x$, así que tenemos que comprobar que $x\sqrt{x+1}$ tiene una inversa cerca de cero para saber que podemos despejar $x$ en $φ(x) = y$ Calculamos la derivada de $φ$:

\[ φ'(x) = \sqrt{x+1} + x\frac{1}{2\sqrt{x+1}} \]

que no se anula en $φ'(0)$. Por el Teorema de la Función Inversa, existen $δ>0, ε>0$ tales que $\appl{\inv{φ}}{(-ε,ε)}{(-δ,δ)}$ existe y es diferenciable.  

Entonces, si $x≥0$, $(x,y)∈C \dimplies y^2 = φ(x)$, y de la misma forma si $x≤0$, $(x,y) ∈ C \dimplies y^2 = -φ(x)$. Por lo tanto, podemos construir las dos funciones $h_i$ de la siguiente forma:

\[ h_1(y) = \inv{φ}(y^2);\quad h_2(y) = \inv{φ}(-y^2) \]

\end{problem}

\begin{problem}[4] $\appl{α}{(-a,a)}{ℝ^2}$ PPA. Defino $β(s)=α(-s)$, de tal forma que cambiamos la orientación. Verifica que β PPA y halle la curvatura.

\solution

Obvio que β también está PPA.
\end{problem}

\begin{problem}[5] Halle una curva plana parametrizada por longitud de arco tal que 

\[ k_α(s) = \frac{1}{1+s^2} \]

\solution

Hallamos el ángulo:

\[ θ(s) = \int \frac{1}{1+s^2}\dif s = \arctan s \]

Despejamos la arcotangente:

\begin{wrapfigure}{r}{0.4\textwidth}
\begin{tikzpicture}
\coordinate (A) at (0,0);
\coordinate (B) at (4,0);
\coordinate (C) at (4,2);

\draw[-] (A) -- node[above,sloped] {$\sqrt{1+s^2}$} (C);
\draw[-] (A) -- node[below,sloped] {$1$} (B);
\draw[-] (C) -- node[right] {$s$} (B);

\begin{scope}
\path[clip] (A) -- node[below,sloped] {$1$} (B) -- node[right] {$s$} (C) -- node[above,sloped] {$\sqrt{1+s^2}$} (A);
\fill[green, opacity=0.3, draw=black] (A) circle (5mm);
\node at ($(A)+(10:12mm)$) {$\theta(s)$};
\end{scope}
\end{tikzpicture}
\end{wrapfigure}

\begin{gather*}
\cos θ(s) = \frac{1}{\sqrt{1+s^2}} \\
\sin θ(s) = \frac{s}{\sqrt{1+s^2}}
\end{gather*}

Integrando, tenemos que 

\begin{align*}
x(s) &= \log \left(s + \sqrt{s^2+1}\right) \\
y(s) &= \sqrt{1+s^2}
\end{align*}

Añadiendo el movimiento rígido tendríamos todas las curvas que cumplen esa ecuación:

\[ β(s) = \begin{pmatrix}
\cos θ_0 & - \sin θ_0 \\
\sin θ_0 & \cos θ_0
\end{pmatrix}\begin{pmatrix}
 \log \left(s + \sqrt{s^2+1}\right) \\
 \sqrt{1+s^2}
\end{pmatrix} + \begin{pmatrix}
x_0 \\ y_0 
\end{pmatrix} \]
\end{problem}

\begin{problem}[8] Sea $\appl{α}{I}{ℝ^2}$ PPA. Demuestra

\ppart α es segmento de recta si y sólo si $∃p_0∈ℝ^2$ por el cual pasan todas las rectas tangentes.

\ppart α es un arco de circunferencia si y sólo si $∃p_0∈ℝ^2$ que esté en todas las rectas normales.

\solution

\spart Empezamos demostrando la implicación a la derecha. Si $α$ es un segmento de recta, lo podemos escribir como 

\[ α(s) = p_0 + s\vv \]

con $\vv$ unitario. La recta tangente a $α$ en $s=s_0$ será 

 \[ α(s_0) + λα'(s_0) = p_0 + s_0\vv + λ\vv = p_0 + (s_0 + λ)\vv \]
 
y por lo tanto $p_0$ está en todas las rectas tangentes a $α$.

Ahora la implicación a la izquierda. Para $s∈I$, sabemos que la recta tangente a $α$ por $s$ es

\begin{equation}\label{eqH2E8} α(s) + λα'(s)\quad λ∈ℝ \end{equation}

Como $p_0$ está en esa recta, $∃λ(s)\tq p_0 = α(s) + λ(s)α'(s)$. Ahora la idea que se nos viene a la cabeza es derivar, pero no sabemos si λ es derivable. Despejamos y 

\[ p_0 - α(s) = λ(s)α'(s) \]

Ahora no podemos dividir por $α'$ (es un vector) así que mltiplicamos a ambos lados por $α'$:

\[ \pesc{p_α - α(s), α'(s)} =λ(s) \pesc{α'(s), α'(s)} = λ(s) \]

por lo tanto $λ(s)$ es suave en $s$. Podemos derivar, así que derivamos en (\ref{eqH2E8})

\begin{gather*}
 0 = α'(s) + λ'(s) α'(s) + λ(s) k_α(s) \mv{n}_α(s) \\
 0 = \mv{t}_α(s) + λ'(s) \mv{t}_α(s) + λ(s)k_α(s)\mv{n}_α(s) \\
 0 = (1+λ'(s))\mv{t}_α(s) + λ(s)k_α(s)\mv{n}_α(s)
 \end{gather*}
 
Dado que $\{\mv{t}_α,\mv{n}_α\}$ es una base ortonormal del plano, una combinación lineal de ambos dos sólo puede ser cero si los dos coeficientes son cero. Esto nos lleva a las dos siguientes ecuaciones

\[ \left.\begin{matrix}
1+λ'(s) = 0 \\
λ(s)k_α(s) = 0 \\
\end{matrix}\right\} ∀s∈I \]

Como $λ'(s) = -1$, $λ(s) = s_0 - s$ para algún $s_0$. Sustituimos en la segunda ecuación y nos queda que $(s_0-s)k_α(s) = 0$, así que si $s≠s_0$ $k_α(s)=0$. Y como además, $k_α$ es continua, en ese punto $s_0$ también es cero. La curvatura es cero en todo punto $s$, y por lo tanto es un segmento de recta.

\spart Visualmente, vemos que $p_0$ deberá ser el centro. La implicación a la derecha se demuestra fácilmente. Sabemos la parametrización de $α$

\[ α(s) = p_0 + r\left(\cos \frac{s}{r}, \sin \frac{s}{r}\right) \]

Calculamos la recta normal por $α(s)$ y veo que $p_0$ está en ella.

Calculamos ahora la \textbf{implicación a la izquierda}. Queremos demostrar que la curvatura es constante y distinto de 0. Sabemos que 

\[ p_0 = α(s) + λ(s)\mv{n}_α(s) \]

para algún $λ(s)$. Como podemos expresar $λ(s) = \pesc{p_0-α(s),\mv{n}_α(s)}$, es derivable. Así que derivamos.

\begin{gather*}
0 = \mv{t}_α(s) + λ'(s)\mv{n}_α(s) + λ(s)\left(-k_α(s) \mv{t}_α(s)\right) \\
0 = (1-k(s)λ(s))\mv{t}_α(s) + λ'(s)\mv{n}_α(s) 
\end{gather*}

Al igual que veíamos en el anterior apartado, sabiendo que tenemos una base normal sacamos dos ecuaciones:

\begin{gather*}
1-k(s)λ(s) = 0 \\
λ'(s)=0 
\end{gather*}

λ es constante, así que $λ(s) = c_0$ para algún $c_0∈ℝ$. Entonces

\[ 1-k_α(s)c_0 = 0 \implies k_α(s) = \frac{1}{c_0} \]

La curvatura escalar es constante, así que efectivamente trabajamos en un arco de circunferencia.

\end{problem}

\begin{problem}[9] Sea $\appl{α}{I}{ℝ^2}$ PPA. Demuestra que todas las rectas normales equidistan de un $p_0$ dado si y sólo si existen $a,b∈ℝ$ tal que \[ k_α(s) = \pm \frac{1}{\sqrt{as + b}} \]

\solution

Hagamos un dibujito

\begin{figure}[hbtp]
\centering
\begin{tikzpicture}[pnt/.style={draw,shape=circle,fill=white, inner sep=2pt}]
\draw[-] (0,0) -- (6,0);
\node[pnt,label=below:{$α(s)$}] (S) at (2,0) {};
\node[label=below:{$\cn(s)$}] (N) at (3.5,0) {};
\node[pnt] (PA) at (4,0) {};
\draw[thick,->] (S) -- (N);
\draw[->] (S) -- node[left] {$-\mv{t}_α(s)$} (2,2);
\node[pnt,label=right:{$p_0$}] (P) at (4,3) {};
\draw[->] (S) -- node[midway, above, sloped] {$p_0-α(s)$} (P);
\draw[dashed] (P) -- node[right] {$\pesc{p_0-α(s), -\mv{t}_α(s)}$} (PA);
\end{tikzpicture}
\end{figure}

Derivamos la distancia $\pesc{p_0-α(s), -\mv{t}_α(s)}$

\[ 0 = \pesc{-\mv{t}_α(s),-\mv{t}_α(s)} + \pesc{p_0-α(s),-k_α(s) \mv{n}_α(s)} \]

y por lo tanto $k_α(s)\pesc{p_60-α(s),\mv{n}_α(s)} = 1$. Volvemos a derivar (total, es gratis)

\[ \cv(s)\pesc{p_0-α(s)\cn(s)} + \cv(s)\left(\pesc{\ct(s), \cn(s)} + \pesc{p_0-α(s), -\cv(s)\ct(s)}\right) = 0 \]

y dividimos:

\[ \frac{\cv'(s)}{\cv(s)} - c\cv^2(s) = 0 \]

Resolvemos la ecuación diferencial y nos queda que

\[ \frac{-1}{2k^2}=cs + d \]

Por lo tanto

\[ k^2 = \frac{1}{(-2c)s + (-2d)} \]

de tal forma que hemos llegado a la fórmula que nos daban al principio.

Ahora vamos a por la \textbf{otra implicación}. Pero no, que es muy larga.
\end{problem}

\begin{problem}[11] Sea $Γ$ una curva cerrada, simple, en el plano, contenida en el interior de la circunferencia $\{ x^2+y^2=r^2 \}$. Demuestra que existe un punto $p∈Γ$ tal que $\md{k(p)} ≥ \frac{1}{r}$.

\solution

Partimos de que $\frac{1}{r}$ es la curvatura del círculo. Trasladamos la circunferencia en cualquier dirección hasta tocar la curva por primera vez. En ese punto de contacto, la tangente es la misma y por lo tanto la normal también.

Si trasladásemos y rotamos las curvas para que el plano sea el eje $OX$ y el punto de contacto fuese el origen, tendríamos algo parecido a la imagen \ref{imgCirc}.

\easyimg{img/Hoja2_11_CircCurva.png}{Curva y circunferencia}{imgCirc}

Si podemos escribir la curva $α=α(s)$ como grafo de una función $h_1(x)$, entonces

\[ k(α(0)) = \frac{h_1''(x)}{\left(1+h_1'(x)^2\right)^{\frac{3}{2}}} \]

Dado que tenemos un grafo sobre la recta tangente, $h_1'(0) = 0$ y entonces $k_Γ(0)=h_1''(0)$. Por otra parte, sabemos que $k_C(0) = h_2''(0) = \frac{1}{4}$. Además, $h_1(x) ≥ h_2(x)$ en un intervalo $(-δ,δ)$ y las funciones y sus derivadas valen lo mismo en $0$. Por lo tanto, tiene que cumplirse

\[ h_1''(0) ≥ h_2''(0) \implies k_Γ(0) ≥ \frac{1}{r} \]

\end{problem}

\subsection{Hoja 3}

\begin{problem}[2] Tenemos una curva $α=α(s)$ PPA biregular. Demuestra que 

\[ \ctr (s) = \frac{α'(s) × α''(s) \cdot α'''(s)}{\abs{\cv(s)}^2} \]

\solution

Sabemos que 

\begin{align*}
α'(s) &= \ct (s) \\
α''(s) &= \ct'(s) = \cv (s) \cn (s) \\
α'''(s) &= \cv'(s) \cn(s) + \cv(s) \cn'(s) = \\
&= \cv'(s) \cn(s) + \cv(s) \left(-\cv(s)\ct(s) + \ctr(s) \cb(s) \right)
\end{align*}

Mientras arreglaba una cosa de \LaTeX me ha escrito una pizarra y me he perdido. Así que nada.

\end{problem}

\begin{problem}[3] Sea $α(s)$ una curva birregular PPA, con $\cv$ y $\ctr$ curvatura y torsión. Dado $β(s) = \ct(s)$, comprobar que es regular y que 

\[ \cvv[β] = \frac{\ctr}{\cv}\cb(s) - \ct(s) \]


\solution

\spart Defino $β(s) = \ct(s)$. Comprobamos si es regular

\[ β'(s) = \ct'(s) = \cv(s) \cn (s) ≠ 0 \]

ya que $k_α(s)≠0$ para todo $s$ ya que α es birregular.

\spart Queremos demostrar que

\[ \cvv[β] = \frac{\ctr}{\cv}\cb(s) - \ct(s) \]

No sabemos si β está parametrizada por arco. Viendo la ecuación anterior, $\md{β'} = \md{\cv}$, que es distinto de uno si la curvatura no es constante igual a 1 y por lo tanto β no es PPA.

En esta situación tenmos varias posibilidades. O bien primero reparametrizamos β por arco $r$ tal que $β(s(r)) = γ(r)$ y hallo $γ''(r)$ o bien hallo $\cvv[β](s)$  usando la fórmula para curvas no PPA. 

\paragraph{Reparametrización por arco} Sea $r$ el parámetro de arco de β. Entonces

\[ r(s) = \int_{s_0}^s \md{β'(z)}\dif z = \int_{s_0}^s \cv(z) \dif z;\quad r'(s) = \cv(s) \] 

Sea $γ(r) = β(s(r))$ la reparametrización de β por arco $r$. Hallamos las dos derivadas de γ:

\[ γ'(r) = β'(s(r)) · s'(r) = \cv(s(r)) \cn(s(r)) · \frac{1}{\cv(s(r))} = \cn (s(r)) \]

\begin{align*} 
\cv[γ](r) &= γ''(r) = \cn'(s(r)) s'(r) =& \text{(Eqs. Frenet)}\\
  &= \left (-\cv(s(r))\ct(s(r)) + \ctr(s(r))\cb(s(r))\right) \frac{1}{\cv(s(r))} =& \\
  &= \frac{\ctr}{\cv}\cb(s) - \ct(s) &
\end{align*}

\end{problem}

\begin{problem}[4]
Sea $\appl{α}{I}{ℝ^3}$ curva birregular y PPA. Todas sus rectas normales pasan por el mismo punto $p_0$. Demuestre que la traza de α está contenida en una circunferencia.

\solution

Sabemos que una circunferencia dentro de un plano tiene curvatura constante y torsión 0, con el recíproco también cierto. Vamos a demostrar que si todas las rectas normales pasan por el mismo punto, tienen que darse esas dos condiciones sobre la curvatura y torsión.

Escribimos la condición de las rectas normales. Para un punto $s$ en la curva, la recta normal es

\[ α(s) + λ(s)\cn(s) = p_0 \]

con $λ(s)∈ℝ$. Derivamos:

\begin{gather*}
\ct + λ'\cn + λ\cn' = 0 \\
\ct + λ'\cn + λ(-\cv \ct + \ctr \cb) = 0
\end{gather*}

Dado que $\ct$, $\cb$ y $\cn$ forman una base l.i. del diedro ortonormal, para que esa ecuación sea 0 todos los coeficientes han de ser 0. Es decir

\[ 1-\cv λ = 0;\quad λ' = 0;\quad λ\ctr = 0 \]

De $λ'=0$ obtenemos que $λ=c_0≠0$, por lo que 

\begin{align*}
1-\cv λ = 0 &\implies \cv = \frac{1}{c_0} \\
λ\ctr = 0 &\implies \ctr = 0
\end{align*}

Ambos resultados implican que α es un trozo de circunferencia.
\end{problem}

\begin{problem}[7] Sea $\appl{α}{I}{ℝ^3}$ PPA con $\cv > 0$. Demostrar que $α(I)$ es un arco de circunferencia si y sólo si $\cv$ es constante y α está contenida en una esfera.

\solution

La implicación a la derecha es trivial. Muy trivial. 

Para la implicación a la izquierda, tenemos que ver que $\ctr\equiv 0$, ya que por hipótesis nos han dicho que $\cv$ es constante. Dado que α está contenida en la esfera, la condición la escribimos como

\[ \md{α(s) - p_0}^2 = R^2 \]

para $p_0$ el centro de la esfera. Dado que eso es una función de una variable y tiene que ser constante, derivamos porque a este hombre le encanta derivar

\[ f'(s) = 2\pesc{\ct,α(s)-p_0} = 0 \]

\begin{center}\Huge{Más derivadas}\end{center}

\[ \frac{f''(s)}{2} = \pesc{\ct', α(s)-p_0} + \pesc{\ct, \ct} \]

Que podemos reexpresar que 

\[ \pesc{\cn, α(s) - p_0} = \frac{-1}{\cv} \]

Y...

\easyimg{img/DERIVADAS.png}{DERIVAAAAAMOOOOOOOOOOOOOS}{imgDerivadas}

\begin{gather*}
 \pesc{\cn', α(s)-p_0} + \pesc{\cn, \ct} = 0 \\
 \pesc{-\cv\ct + \ctr\cb, α(s)-p_0} = \ctr \pesc{\cb, α(s)-p_0} = 0
 \end{gather*}
 
 Suponemos que existe un $s_0$ donde $\ctr(s_0)≠0$. En un entorno alrededor de ese punto, se tiene que 
 
 \[ \pesc{\cb, α(s)-p_0} = 0 \]
 
 Entonces
 
 \begin{gather*}
  α(s)-p_0 = \frac{1}{\cv}\cn(s) \\
  α(s) = p_0 - \frac{1}{\cv}\cn(s) \\
  \ct = \frac{1}{\cv}\left(-\cv\ct + \ctr \cb\right) \\
  \frac{-\ctr}{\cv}\cb = 0
  \end{gather*}
  
  Lo que nos llevaría a que $\ctr=0$, contradicción.
 
\end{problem}

\subsection{Hoja 4}

\begin{problem}[1] Sea $f(x) = (x+y+z-1)^2$. Halle los valores y puntos críticos de $f$. Decida para qué valores $c$, $\inv{f}(c)$ es una superficie regular.

\solution

Calculamos las derivadas parciales:

\[ f_x = f_y = f_z = 2(x + y + z -1) \]

Igualando a cero, tenemos que los puntos críticos están en el plano

\[ x+ y+z = 1 \]

y el único valor crítico es el 0.

Ahora nos piden buscar para qué valores $c∈ℝ$ $\inv{f}(c)$ es una superficie regular. Consideramos varias opciones.

\begin{itemize}
\item Si $c<0$, $\inv{f}(c)= \emptyset$ y no es superficie regular.
\item Si $c>0$, $\inv{f}(c) ≠\emptyset$. Por ejemplo $P=(\sqrt{c} + 1, 0,0∈\inv{f}(c)$ y como $c$ no es valor crítico, $f^{-1}(c)$ es superficie regular.
\item Si $c=0$, antes habíamos calculado que ese era un valor crítico para los puntos críticos del plano $x+ y+z = 1$, así que efectivamente es una superficie regular.
\end{itemize}

\end{problem}

\begin{problem}[6] Sean $r,R>0$ y $r<R$. Llamamos toro $T$ al conjunto de puntos $(x,y,z)$ obtenidos al girar alrededor del eje $OZ$ la circunferencia $C$ de centro $(0,R,0)$, radio $r$ y situada en el plano $x=0$.

Demuestra que $T$ se puede expresar como un conjunto de nivel $T=\inv{F}(0)$ de una función $\appl{F}{ℝ^3}{ℝ}$, con 

\[ F(x,y,z) = (x^2+y^2+z^2-R^2-r^2)^2-4R^2(r^2-z^2) \]

\solution

Partimos del siguiente dibujo.

\easyimg{img/Hoja4_6_Toro.png}{Planteamiento del problema.}{imgToro}

Proponemos que todo punto de $T$ lo obtengo a partir de un $p_0$ en $C$ tras rotarlo un cierto ángulo θ con respecto al eje $OX$. Entonces, la parametrización es

\(\label{eqH4_6_Param} \begin{matrix}
x &=& (R + r\cos ψ)\cos θ \\
y &=& (R + r\cos ψ)\sin θ \\
z &=& r\sin ψ
\end{matrix} \)

¿Cómo demostramos que ambos conjuntos $T$ y $\inv{F}(0)$ son iguales? Pues lo de siempre: doble contención. Primero vemos si $T⊆\inv{F}(0)$.

Calculamos $F(T)$. Para $(x,y,z)∈T$, 

\[ F(x,y,z) = \dotsb =\footnote{Magia!} 0 \]

Vamos ahora al revés: queremos mirar si $\inv{F}(0)⊆T$. Si $(x,y,z)∈\inv{F}(0)$, sabemos que

\[ (x^2+y^2+z^2-R^2-r^2)^2 = 4R^2(r^2-z^2) \]

Hay que buscar $θ,ψ$ para que se cumpla \eqref{eqH4_6_Param}.

\end{problem}

\subsection{Hoja 5}

\paragraph{Primera forma fundamental} Tenemos una superficie $S$, una forma cuadrática

\begin{align*}
\appl{I_p}{T_pS&}{ℝ} \\
v \mapsto \pesc{v,v}
\end{align*}

y una parametrización $\appl{Φ}{U}{S}$ que lleva un punto $(u,v)$ a $p=Φ(u,v)$. $\{Φ_u(u,v),Φ_v(u,v)\}$ es una base del plano tangente $T_pS$. Para hallar la forma fundamental, hallamos ese producto escalar 

\[ I_p(v) = \pesc{aΦ_u + bΦ_v, aΦ_u + bΦ_v} = a^2\underbrace{\pesc{Φ_u,Φ_u}}_{E(u,v)} + 2ab \underbrace{\pesc{Φ_u, Φ_v}}_{F(u,v)} + b^2 \underbrace{\pesc{Φ_v,Φ_v}}_{G(u,v)} \]


\paragraph{Cálculo del ángulo entre curvas} Si tenemos dos curvas $\appl{α,β}{I}{S}$ que se cortan en un punto $p=α(t_0) = β(s_0)$, entonces el coseno del ángulo $θ$ de corte es

\[ \cos θ = \frac{\pesc{α'(t_0), β'(s_0)}}{\norm{α'(t_0)}\norm{β'(s_0)}} \]

Nos pueden dar la primera forma fundamental de una superficie $Φ(u,v)$ de tal forma que 

\begin{align*}
α(t) &= Φ(u_1(t), v_1(t)) \\
β(t) &= Φ(u_2(s), v_2(s))
\end{align*}

Entonces, podemos expresar las derivadas como

\begin{align*}
α'(t_0) &= u_1'(t_0) Φ_u + v_1'(t_0) Φ_v \\
β'(s_0) &= u_2'(s_0) Φ_u + v_2'(s_0) Φ_v 
\end{align*}

y nos queda

\[ \pesc{α'(t_0),β'(s_0)} = \begin{pmatrix}
u_1' & v_1'
\end{pmatrix} 
\begin{pmatrix}
E & F \\
F & G
\end{pmatrix}
\begin{pmatrix}
u_2' \\ v_2 '
\end{pmatrix} \]

Sólo faltaría hallar las normas, que se haría como  \[ \norm{α'(t_0)}^2 = E(u_1')^2 + 2Fu_1'v_1' + G(v_1')^2 \].

En particular, podemos hallar el ángulo entre las curvas coordenadas $α=Φ(t,v_0)$ y $β=Φ(u_0, s)$. En este caso, 
\begin{align*}
α' &= Φ_u \\
β' &= Φ_v 
\end{align*}

y en el numerador nos queda $Φ_uΦ_v = F$. Es decir, que

\[ \cos θ = \frac{F}{\sqrt{EG}} \]

\begin{problem}[2] Hallar la primera forma fundamental de una superficie con parametrización 

\[ Φ(u,v) = (au\cosh v, bu\sinh v, u^2) \]

\solution Para hallar los coeficientes, derivamos:

\begin{align*}
Φ_u &= (a\cosh v, b\cosh v, 2u) \\
Φ_v &= (au\sinh v, bu\sinh v, 0)
\end{align*}

Ahora tenemos que hallar los coeficientes de la primera forma fundamental haciendo los productos escalares:

\begin{align*}
E(u,v) &= \pesc{Φ_u,Φ_u} &= a^2\cosh^2v + b^2\sinh^2v + 4u^2 \\
F(u,v) &= \pesc{Φ_u,Φ_v} &= (a^2+b^2)u\sinh v \cosh v \\
G(u,v) &= \pesc{Φ_v,Φ_v} &= a^2u^2\sinh^2 v + b^2u^2\cosh^2v 
\end{align*}

\end{problem}

\paragraph{Cálculo del área a través de la primera forma fundamental} Supongamos que queremos calcular el área de una región $R=φ(U)$. Normalmente, escribiríamos

\[ \mop{\'Area}(R) = \iint_U \norm{Φ_u× Φ_v}\id{u,v} \]

Con los coeficientes de la primera forma fundamental, esto coincide con

\[ \mop{\'Area}(R) = \iint_U \sqrt{EG - F^2} \id{u,v} \]

\begin{problem}[3] Hallar el área de la superficie de revolución dada por el giro de la curva PPA

\[ α(t) = (0, ρ(t), h(t)) \]

\solution La parametrización de la superficie de revolución con respecto al eje $Z$ será

\[ Φ(u,v) = (ρ(v)\cos u, ρ(v) \sin u, h(v)) \]

Hallamos la primera forma fundamental, primero derivando

\begin{align*}
Φ_u &= (-ρ(v)\sin u, ρ(v) \cos u, 0 ) \\
Φ_v &= (ρ'(v)\cos u, ρ'(v) \sin u, h'(v))\\
\end{align*}

lo que nos da los coeficientes

\begin{align*}
E = ρ^2(v) \\
F = 0 \\
G = ρ'^2 + h'^2 = \norm{α'}^2 \stackrel{PPA}{=} 1
\end{align*}

Vemos que salen valores \textit{curiosos}. Estudiamos $F$, que era $ \pesc{Φ_u,Φ_v}$. Como vale 0, esto indica que los vectores $Φ_u$ y $Φ_v$ son ortogonales, algo que parece muy lógico si vemos que $Φ_u$ es siempre tangente a la circunferencia de revolución, y $Φ_v$ es tangente a la curva.

Hallamos ahora el área:

\[ \mop{\'Area}(S) = \int_a^b \int_0^{2π} \sqrt{ρ^2(v)} \id{u,v} = 2π \int_a^b ρ(v) \dif v \]


\end{problem}

\begin{problem}[5] Tenemos la siguiente parametrización de una superficie $S$

\[ Φ(u,v) = (u\cos v,u \sin v, \log \cos v + u) \]

con $u∈ℝ$, $v∈\left(\frac{-π}{2}, \frac{π}{2}\right)$. Demostrar que el par de curvas $Φ(u_1,v)$, $Φ(u_2,v)$ cortan a cada curva $Φ(u,v_0)$ con igual longitud.

\solution Supongamos que $v$ es el parámetro \textit{vertical} y $u$ el \textit{horizontal}.

\begin{figure}[htbp]
\centering
\begin{tikzpicture}[scale=3]
\coordinate (A) at (0,0) {};
\coordinate (B) at (1,0) {};
\coordinate (C) at (1,1) {};
\coordinate (D) at (0,1) {};

\draw[-] (-0.1,-0.3) edge[out=90,in=270] (A) (A) -- (D) to[out=90,in=270, looseness=1]  (0.1,1.3) (0.1,1.3)  node[above] {$Φ(u_1, v)$};
\draw[-] (0.9,-0.3) edge[out=90,in=270] (B) (B) -- (C) edge[out=90,in=270]  (1.1,1.3) (1.1,1.3)  node[above] {$Φ(u_2, v)$};
\draw[-] (-0.3,0.4) edge[out=0,in=180] (0,0.5) (0,0.5)-- (1,0.5) edge[out=0,in=180]  (1.3,0.6) (1.3,0.6) node[right] {$Φ(u, v_0)$};

\draw[-,green!20!black,very thick] (0,0.5) -- node[midway,below] {$α_{v_0}$} (1,0.5);

\end{tikzpicture}
\end{figure}

Llamamos $α_{v_0} = Φ(t,v_0)$, con $t∈[u_1, u_2]$ a la curva desde la primera a la segunda curva. Su longitud será entonces

\[ L(α_{v_0}) = \int_{u_1}^{u_2} \norm{α_{v_0}'(t)} \dif t \]

Uno diría que ya ha terminado, pero falta calcular el vector tangente. Pero ese es el camino desesperado, nosotros somos listos y vemos que el vector del que hallamos la norma es $Φ_u(t,v_0)$. Entonces

\[ \norm{α_{v_0}'(t)} \pesc{Φ_u(t,v_0), Φ_u(t,v_0)}^{\frac{1}{2}} = E(t,v_0)^{\frac{1}{2}} \]

No hemos simplificado mucho, pero bueno, integramos y a ver qué pasa.

\[ Φ_u(t,v_0) = (\cos v_0, \sin v_0, 1) \implies E = 2 \]

y por lo tanto

\[ L(α_{v_0}) = \int_{u_1}^{u_2} \sqrt{2} \dif t = \sqrt{2} (u_2 - u_1) \]

No hemos ganado mucho, así que vamos a hacer otra curva.
\end{problem}

\begin{problem}[9] En una parametrización $Φ(u,v)$, la 1FF es 

\[ Q = \dif u^2 + 2(u+v) \id{u,v} + e^v \dif v^2 \]

Introduzco nuevas coordenadas $(λ,μ)$ tales que

\begin{align*}
u &= e^λ + μ \\
v &= μ
\end{align*}

Halla $\tilde{Q}$, la 1FF en las coordenadas $(λ,μ)$.

\solution Tenemos dos formas para solucionar. Derivamos

\begin{gather*}
u = e^λ + μ \\
\dif u = e^λ \dif λ + \dif μ \\
\dif v = \dif μ
\end{gather*}

Entonces

\begin{align*}
\dif u^2 &= e^{2λ} \dif λ^2 + 2e^λ \dif λ \dif μ + \dif μ^2 \\
\dif u \dif v &= e^λ \dif λ \dif μ + \dif μ^2 \\
\dif v^2 &= \dif μ^2
\end{align*}

Si $Q = \dif u^2 + 2(u+v) \id{u,v} + e^v \dif v^2$, sustituimos

\[  \tilde{Q} = e^{2λ} \dif λ^2 + 2e^λ \dif λ \dif μ + \dif μ^2 + 2 + \\ (2e^λ + 2μ)(e^λ \dif λ \dif μ + \dif μ^2) + e^μ \dif μ^2
\]
Agrupamos coeficientes:

\[ \tilde{Q} = \underbrace{e^{2λ}}_{\tilde{E}} \dif λ^2 + \underbrace{(2e^λ + (2e^λ + 2μ)e^λ)}_{\tilde{F}}\dif λ \dif μ + \underbrace{(1+ (2e^λ + 2μ) + e^μ)}_{\tilde{G}} \dif μ^2 \]

\paragraph{Segunda forma}

Tenemos $Φ(u,v)$ que nos da los coeficinetes $E, F, G$. Con el cambio de variables al que llamamos $T$ podemos construir $ψ(λ,μ) = Φ ○ T$. Entonces, por ejemplo, $\tilde{E} = \pesc{ψ_λ,ψ_λ}$. Pero

\[ ψ_λ = Φ_ue^λ + Φ_v · 0 = e^λ Φ_u \]

así que sustituimos:

\[ \tilde{E}(λ,μ) = \pesc{e^{2λ}Φ_u, e^{2λ}Φ_u} = e^{2λ}E(e^λ + μ, μ) = e^2λ \]

Etc\ae tera, etc\ae tera.

\end{problem}

\paragraph{Cálculo de isometrías} Tenemos un abierto $U⊆ℝ^2$, con parámetros $(u,v)$, y dos parametrizaciones sobreyectivas

\begin{align*}
\appl{Φ&}{U}{S} \\
\appl{ψ&}{U}{S'}
\end{align*}

Tenemos también una aplicación $\appl{h}{S}{S'}$ tal que $h(Φ(u,v)) = ψ(u,v)$.

Entonces $h$ es isometría si y sólo si la primera forma fundamental de $S$ en la parametrización $φ$ es la misma que la 1ff de $S'$ en la parametrización $Ψ$. Es decir

\begin{align*}
E_Φ (u,v) &= E_ψ(u,v) \\
F_Φ (u,v) &= F_ψ(u,v) \\
G_Φ (u,v) &= G_ψ(u,v) 
\end{align*}.

El esquema de lo que ocurre está en la figura \ref{imgH5E1}. Es un diagrama conmutativo.

\begin{figure}[hbtp]
\centering
\begin{tikzpicture}[x=2cm,y=2cm]
\node (S) at (0,1) {$Φ(u,v) ∈ S$};
\node (S2) at (2,1) {$ψ(u,v) ∈ S'$};
\node (U) at (1,0) {$(u,v) ∈ U$};

\draw[->] (S) -- node[above] {$h$} (S2);
\draw[->] (U) -- (S);
\draw[->] (U) -- (S2);
\end{tikzpicture}
\label{imgH5E1}
\caption{Diagrama de las aplicaciones definidas.}
\end{figure}

Podemos ver que si $\appl{Φ}{U}{S}$ es inyectiva, tengo una inversa $\appl{\inv{Φ}}{S}{U}$ y entonces podríamos reescribir $h$ como $h = ψ ○ \inv{Φ}$.

\begin{problem}[1] Consideramos dos parametrizaxiones

\begin{align*}
\appl{Φ&}{ℝ^2}{S} \\
\appl{ψ&}{ℝ^2}{S'}
\end{align*}
con
\begin{align*}
Φ(u,v) &= \left(u^3+3u, \frac{3}{5}, \frac{3}{5}v^5 - 3v, 4u^2 + \frac{10}{3}v^3 \right)\\
\end{align*}

Demuestra que $\appl{h}{S}{S'}$ con $h(Φ(u,v)) = ψ(u,v)$ existe y es una isometría.

\solution

Tenemos que demostrar que existe la inversa de Φ, así que como $S=Φ(ℝ^2)$ sólo tenemos que demostrar la inyectividad. Buscamos resolver $Φ(u,v) = Φ(\bar{u},\bar{v})$:

\begin{align*}
u^3+3u &= \bar{u}^3 + 3\bar{u} \\
\frac{3}{5}v^5 - 3v &= \frac{3}{5}\bar{v}^5 - 3\bar{v} \\
4u^2 + \frac{10}{3}v^3 &= 4\bar{u}^2 + \frac{10}{3}\bar{v}^3
\end{align*}

¿Cómo lo hacemos? En lugar de resolver el sistema, vemos que en la primera ecuación, la primera derivada es $3u^2 + 3$, que es siempre positivo. Por lo tanto, la primera coordenada es siempre creciente e inyectiva y entonces $u=\bar{u}$. Sustituyendo en la última ecuación, eliminamos la $u$ y nos queda que $\bar{v} = v$.

Para que $h$ sea isometría tiene que ser un difeomorfismo y que preserve la primera forma fundamental. Por lo tanto, necesitamos ver que $ψ$ es biyectiva, que lo es porque sí.

Si hacemos los cálculos de los coeficientes de la 1ff para Φ y ψ, vemos que son iguales y por lo tanto es una isometría.
\end{problem}

\subsection{Hoja 6}

\begin{problem}[1] Consideramos una superficie $C⊆ℝ^3$ que es un \textit{"cono"}, generado trazando rectas desde un punto fijo (vértice) hasta todos los puntos de una curva cualquiera (no tiene por qué ser una circunferencia.

Sea $S_1$ la esfera unidad centrada en el origen, y α la intersección del cono con $S_1$. $α(u)$ es una parametrización, y entonces tomamos $Φ(u,v) = vα(u)$ con $v>0$.

Comprueba que si $α$ es PPA, la 1FF de $Φ$ es la misma para cualquier cono.

\solution

Calculamos la 1FF:

\[ \begin{cases}
Φ_u = vα'(u) \\
Φ_v = α(u) \\
\end{cases}\implies \begin{cases}
E &= v^2α'^2(u) = v^2 \\
F &= vα'(u)α(u) = 0\\
G &= α^2(u) = 1
\end{cases} \]

Es decir, la curva no influye en ningún momento y por lo tanto la 1FF de $Φ$ es siempre la misma.
\end{problem}

\begin{problem}[3] Sea $Σ$ parametrizada por 

\[ Φ(u,v) = \left(r(u)\cos v,r(u)\sin v,z(u)\right) \].

Además, la normal unitaria es

\[ N = \left(-z'(u) \cos v, -z'(v) \sin v, r'(u) \right) \]

Calcular la primera y segunda forma fundamental, y algo sobre las curvaturas principales.
\solution

Podemos ver que $Σ$ es una superficie de revolución de la curva $(r(u), 0, z(u))$. Además, viendo la normal unitaria tenemos que $z'^2 + r'^2 = 1$, y por lo tanto la curva de revolución está parametrizada por arco. 

Calculamos la 1ff y la 2ff:

\begin{align*}
Φ_u &= \left(r' \cos v, r'\sin v, z'\right) \\
Φ_v &= \left(-r\sin v, r\cos v, 0\right) \\
Φ_{uu} &= \left(r''\cos v, r''\sin v, z''\right) \\
Φ_{uv} &= \left(-r' \sin v,-r'\cos v, 0\right) \\
Φ_{vv} &= \left(-r\cos v, -r \sin v, 0\right)
\end{align*}
lo que nos da los coeficientes
\begin{align*}
E &= r'^2 + z'^2 = 1 \\
F &= 0 \\
G &= r^2 \\
e &= \pesc{N,Φ_{uu}} = -r''z' + r'z''\\
f &= \pesc{N,Φ_{uv}} = rz'\\
g &= \pesc{N,Φ_{vv}} = rz' \\
\end{align*}

Pero además, al haber calculado antes la primera forma fundamental, podríamos sacar que

\[ N = \frac{Φ_u × Φ_v}{\norm{Φ_u×Φ_v}} = \frac{Φ_u × Φ_v}{\sqrt{EG-F^2}} 
\] y que además

\[ e = \pesc{N,Φ_{uu}} = \frac{1}{\sqrt{EG-F^2}} \det [ Φ_u | Φ_v | Φ_{uu} ] \]

De esta forma evitamos realizar cuentas excesivamente complejas.

Siguiendo con el ejercicio, la segunda forma fundamental sería

\[ Π_Φ = (-r''z'+r'z'')\dif u^2 + 2·0·\dif u \dif v + rz'\dif v ^2 \]

Hallemos ahora las \textbf{curvaturas principales}. La primera aplicación es $\appl{?}{T_pS}{T_pS}$ que lleva un vector $v$ a un vector $\Dif_vN$. La matriz de la aplicación está dada por 

\[ \begin{pmatrix}
E & F \\ F & G
\end{pmatrix}^{-1}
\begin{pmatrix}
e & f \\ f & g
\end{pmatrix} = \begin{pmatrix}
-r''z' + r'z'' & 0 \\ 0 & \frac{z'}{r}
\end{pmatrix}\]

Las curvaturas principales son los autovalores de esa matriz, que como es diagonal ya los tenemos escritos.

Si por ejemplo nos hubieran dado la curvatura gaussiana $K$ y la media $H$, podríamos recuperar las curvaturas principales $k_1,k_2$ a través de sus respectivas fórmulas:

\begin{gather*}
 K = k_1k_2 \\
 H = \frac{-(k_1 + k_2)}{2}
\end{gather*}

\end{problem}

Una curva α es asintótica cuando $II_{α(t)}(α'(t)) = 0$. Es decir, la curva siempre va buscando ser tangente a una dirección que anule la 2FF.

\begin{problem}[6] \textbf{Limones localmente isométricos a la esfera.} Tenemos la esfera sin un meridiano dada por la parametrización
\[
Φ(u,v) = (\cos u\cos v,\cos u \sin v, \sin u)\quad u∈\left(-\frac{π}{2},\frac{π}{2}\right),\;v∈(0,2π) 
\]
que es una superficie de revolución de la semicircunferencia dada por $ρ(u) = \cos u$, $h(u) = \sin u$.

\ppart Para cada $c>1$ encuentra $r_c(u), z_c(u)$ con $u∈\left(-\frac{π}{2},\frac{π}{2}\right)$ tal que si \[ ψ^c(u,μ) = \left(r_c(u)\cos μ, r_c(u)\sin μ, z_c(u)\right) \] con $μ∈ℝ$, entonces la aplicación $\appl{h}{S}{S_c}$ que a $Φ(u,v)$ le asigna $ψ^c(u,cv)$ es isometría local.

\solution

Para que $h$ sea isometría, la 1FF de Φ y ψ tienen que ser iguales. Nos molesta la $c$ que están en $ψ(u,cv)$ para poder aplicar lo que veíamos en el problema 1 de la hoja 5, así que vamos a arreglar esto.

Sea $ Γ(u,v) = ψ^c(u,cv)$. Basta ver que los coeficientes de la primera forma fundamental de Φ y de Γ coinciden. Vamos a calcularlos:

\begin{align*}
E_Φ &= 1\\
F_Φ &= 0\quad \text{(sup. de revolución)} \\
G_Φ &= \cos^2 u \\
\end{align*}

Calculamos las derivadas de Γ
\begin{gather*}
Γ_u = (r_c' \cos cv,r_c'\sin cv,z_c') \\
Γ_v = (-r_c'c\sin cv, r_c c \cos cv, 0) 
\end{gather*}
y obtenemos los coeficientes

\begin{align*}
E_Γ &= r_c'^2 + z_c'^2 \\
F_Γ &= 0 \\
G_Γ &= c^2r^2
\end{align*}.

Igualamos los coeficientes y tenemos

\begin{align*}
r_c'^2 + z_c'^2 &= 1 \\
c^2r^2(u) &= \cos^2 u
\end{align*} lo que nos da las dos funciones

\begin{gather*}
z_c(u) = \int\sqrt{1-\frac{1}{c^2}\sin^2u}\dif u \\
r_c(u) = \frac{1}{c}\cos u
\end{gather*} 

El enunciado ahora nos pide ver qué le ocurre al ecuador de esta superficie cuando $c$ aumenta, y lo que vemos es que se acerca al eje del origen.

También nos pide demostrar que la altura del limón es creciente en $c$. La altura del limón es

\[ z_c\left(\frac{π}{2}\right)-z_c\left(\frac{-π}{2}\right) = \int_{-π/2}^{π/2} \sqrt{1-\frac{1}{c^2} \sin^2 s}\dif s = f(c) \]. 

Obtenemos la derivada

\[ f'(c) = \int_{-π/2}^{π/2} \left(1-\frac{1}{c^2}\sin s\right)^{\frac{-1}{2}} · \frac{1}{2} · \frac{2}{c^3} \sin^2 s \dif s 
\]
que es siempre positiva, y por lo tanto es creciente. Cuando $c\to ∞$, converge a \[\lim_{c\to ∞} \int_{-π/2}^{π/2} \sqrt{1-\frac{1}{c^2}\sin^2u}\dif u = \int_{-π/2}^{π/2} 1 = π \].

Vamos ahora con las \textbf{curvaturas principales}. En el ejercicio anterior se dice que una superficie de revolución $(r(u),z(u))$ tiene curvaturas principales \[ r'z'' - r''z',\quad \frac{z'}{r} \]. Queremos saber si conserva la isometría $h_c$ el par no ordenado de curvaturas principales. En nuestro caso, derivando y operando

\begin{gather*}
 r'z'' - z'r'' = \frac{1}{c}\cos u \left(1-\frac{1}{c^2}\sin^2 u\right)^{-\frac{1}{2}} \\
 \frac{z'}{r} = \frac{\left(1-\frac{1}{c}\sin^2 u\right)^{1/2}}{\frac{1}{c}\cos u}
 \end{gather*}

¿Para qué queremos todos estos cálculos? Los limones son localmente isométricos a la esfera. La esfera tiene curvaturas principales iguales a $1$ en todos los puntos. Sin embargo, nuestros limones tienen curvaturas principales distintas. De aquí podemos extraer la sabia lección de que una isometría local no conserva las curvaturas principales.

Pero hay otra cosa: el producto de las curvaturas principales es igual a la curvatura gaussiana. La curvatura gaussiana de la esfera es uno, y la de los limones es también uno. Es decir, \textbf{la curvatura gaussiana} se conserva en las isometrías locales.
\end{problem}

\begin{problem}[7] Sea $α(u)$ una curva birregular en el espacio PPA. Sea $S$ la parte de la superficie tangencial dada por la parametrización \[ Φ(u,v) = α(u) + v \ct(u) \] con $v > 0$.

Describe la normal unitaria $N(u,v)$ de $S$ en términos del triedro de Frenet $\{\ct(u), \cn(u),\cb(u)\}$ de α. A partir de eso, calcula el endomorfismo de Weingarten de $S$ sin pasar por la segunda forma fundamental. Demuestra que cada punto de $S$ es parabólico o plano. ¿Cuándo tiene $S$ puntos planos?

\solution

Calculamos las derivadas

\begin{align*}
Φ_u &= \ct(u) + v \cv(u) \cn(u) \\
Φ_v &= \ct(u)
\end{align*}, el producto vectorial

\[ 
Φ_u×Φ_v = 0 + v\cv(u) (-\cb(u)) 
\]
y por último la normal unitaria
\[
N = \frac{Φ_u×Φ_v}{\norm{Φ_u×Φ_v}} = \frac{-v\cv(u)\cb(u)}{\norm{-v\cv(u)\cb(u)}} = - \cb(u) \].

Vamos a calcular ahora el \textbf{endomorfismo de Weingarten} sin usar la 2FF. ¿Cómo calculamos esta cosa? Tenemos que calcular la aplicación $\appl{\dif N}{T_pS}{T_pS}$ evaluada en $Φ_u$ y en $Φ_v$. Es

\begin{gather*}
 \dif N (Φ_u )= \od{}{s}(N(u+s,v)) = N_u = \od{}{u} ( -\cb(u)) = -\cb'(u) = \ctr(u) \cn (u) \\
 \dif N (Φ_v) = N_v = \od{}{s}(-\cb(u)) = 0
 \end{gather*}

Tenemos que escribir $\dif N(Φ_u)$ en función de $Φ_u$ y $Φ_v$, entonces

\[ \ctr \cn = a Φ_u + b Φ_v =  a \ct + a v \cv \cn + b \ct \] luego $a+b=0$ y además $av\cv = \ctr$, por lo que \begin{align*}
a &= \frac{\ctr}{v\cv} \\ 
b &= \frac{-\ctr}{v\cv}
\end{align*}

y entonces

\[ \dif N = \begin{pmatrix}
\dfrac{\ctr}{v\cv}  & 0\\ 
\dfrac{-\ctr}{v\cv} & 0
\end{pmatrix} \]

\index{Punto!plano}
\index{Punto!elíptico}
\index{Punto!parabólico}
\index{Punto!hiperbólico}
Vamos ahora con la cosa de los puntos que nos decían. Los puntos se dividen en \textbf{elípticos, parabólicos, planos e hiperbólicos}. Los elípticos tienen curvatura gaussiana positiva, los hiperbólicos negativa, y los planos y parabólicos 0. Más concretamente, en los puntos planos ambas curvaturas principales son 0, mientras que en los parabólicos sólo lo es una.

Apliquemos esto a los puntos de nuestra superficie. Tenemos núcleo de $\dif N$ (hemos visto que $\dif N(Φ_v) =0$) por lo que no podemos tener puntos hiperbólicos ni elípticos. Vamos a ver si somos capaces de decir cuándo los puntos son planos.

Para que eso ocurra, tenemos que tener que toda la diferencial $\dif N$ sea 0. Es decir, que si $\ctr = 0$ tenemos puntos planos, y si no, tenemos puntos parabólicos.

Además, al calcular $\dif N$ hemos calculado la segunda forma normal, ya que 

\[ II_p(v) = I_p(v,\pm \dif N(v)) \].

\end{problem}

\begin{problem}[8] Sea $α(u)$ curva birregular en el espacio PPA con torsión constante 1 y con $\cv(u) > 0$. 

\ppart Siendo $\ast$ el triedro de Frenet de α definimos una supercicie algop.

\ppart 

\ppart Hallar $N$ normal unitaria tal que $N\cn(u) > 0$ y demuestra que $II_s$ es un engendro.

\solution

\spart Calculamos la 1FF. 
\[ \begin{cases}
Φ_u &= \ct(u) - v \cn (u) \\
Φ_v&= \cb (u) 
\end{cases} \implies \begin{cases}
E &= 1+v^2 \\
F &= 0  \\
G &= 1
\end{cases} \].

\spart Tenemos $α_1,α_2$ ambas PPA con torsión 1 y curvaturas distintas $k_1,k_2$. Hay que demostrar que la aplicaicón $\appl{h}{S_1}{S_2}$ que lleva $Φ_1(u,v)$ a $Φ_2(u,v)$ es isometría local. EN el anterior apartado hemos calculado la primera forma fundamental, que no dependía de la curvatura. Por lo tanto, como la 1FF coincide es una isometría local.

\spart Calculamos la normal, eligiendo el signo menos para que se cumpla la condición del apartado.

\[ - \frac{Φ_u×Φ_v}{\norm{Φ_u×Φ_v}} = - \frac{-\cn -v\ct }{\sqrt{1+v^2}} \]

Calculamos ahora la segunda forma fundamental \footnote{Estoy empezando a perder capacidades. Dos horas seguidas copiando esto no puede ser bueno.} Calculamos las derivadas segundas

\[
\begin{cases}
Φ_{uu} &= \cv \cn - v (-\cv \ct + \ctr \cb) \\
Φ_{uv} &= -\cn \\
Φ_{vv} &= 0
\end{cases} \implies \begin{cases}
e &= \pesc{Φ_{uu},N} = \cv \sqrt{1+v^2} \\
f &= \pesc{Φ_{uv},N} = -\frac{1}{1+v^2} \\
g &= \pesc{Φ_{vv},N} = 0
\end{cases} \]
que es igual que el engendro que nos decían en el enunciado.

Vamos a hallar ahora las líneas asintóticas\footnote{No tengo ni la más menor idea de qué puñetas es esto}. La condición de las líneas asintóticas es $II(α') = 0 \; ∀t$. Cogemos una curva $α(t) = Φ(u(t),v(t))$. Sabemos que las coordenadas de $α'$ en la base $\{Φ_u,Φ_v\}$ son $(u',v')$. Por lo tanto, sólo tenemos que coger $u'$ y $v'$ y meterlos en la 2FF.Entonces

\[ II (α'(t)) = \sqrt{1+v^2} \cv u'^2 - \frac{2}{\sqrt{1+v^2}}u'v' \]
lo que nos da una ecuación diferencial con esta pinta:
\[ \sqrt{1+v^2} \cv (u) u'^2 - 2u'v'\frac{1}{1+v^2} = 0\].

Podemos sacar $u'$ como factor común y la raíz esa también y 

\[ \frac{u'}{\sqrt{1+v^2}}\left(\cv u'(1+v^2) - 2v'\right) = 0 \], lo que nos da dos posibilidades. Puede ser $u'(t) = 0$ y entonces $u(t) = u_0$, lo que no es nada raro porque es una curva dentro de la superficie y algo asintótico. 

La otra ecuación a resolver es el engendro ese de ahí \footnote{Matadme ya.} \[ ku' = \frac{2v'}{1+v^2} \], que integrando sale algo.
\end{problem}

\subsection{Hoja 7}

\begin{problem}

Sea $\appl{L}{ℝ^2×ℝ^2}{ℝ}$.

\[ L = ((x_1, x_2), (y_1, y_2)) = \frac{1}{2}\left(y_1^2 + e^{x_1}y_2^2\right) \]

\solution

Calculamos el \textbf{funcional integral asociado a $L$}:

\[ L(α) = \int_a^b L(α,α')\dif t \] donde \[ \appl{α}{[a,b]}{ℝ^2};\quad α(t) = (x(t), y(t)) \]. Entonces

\[ F(α) = \int_a^b \frac{1}{2}\left(x'^2 +e^x y'^2\right) \dif t \]

Hallamos ahora la \textbf{primera variación del funcional}. Lo que hacemos es coger varias curvas próximas a α que dependen de un parámetro λ, por ejemplo. La primera variación del funcional nos dirá cómo varía la curva en función de ese segundo parámetro. 

Definimos $α(t, λ)$ como la variación de la curva α y

\[ \tilde{F}(λ) = F(α(t,λ)) = \int_{a}^{b} \frac{1}{2}\left(x'(t,λ)^2 + e^{x(t,λ)} y'(t,λ)^2\right) \dif t \].

La primera variación no será más que $\tilde{F}'(0)$. Derivamos la integral con respecto a λ:

\[ \tilde{F}(λ) = \int_a^b \frac{1}{2}\left(2x'(t,λ) \pd{x'}{λ} (t,λ)  +e^{x(t, λ)} \pd{x}{λ}y'^2 + e^{x(t,λ)} 2y'\pd{y'}{λ}\right) \dif t \]

\begin{wrapfigure}{r}{0.4\textwidth}

\begin{tikzpicture}[font=\small]
\foreach \l/\pos in {-1/below,0/above,1/above}
{
	\draw[green!80!blue, domain=-pi-0.5:pi+0.5, variable=\x, smooth, samples=200] plot ({\x}, {-0.02*\x*\x + \l * (0.2*sin(4*\x r) + 0.1*(\x-pi)*(\x+pi))});
	\node[\pos] at (-1, {\l * 1.1}) {$α(t, \l)$};
}

\draw[blue,domain=-1.2:1.2, variable=\x, smooth] plot ({0.4 + 0.2*\x*\x}, {\x}) node[above] {$α(1, λ)$};

\node[draw, circle, fill, black, inner sep=1pt, label=above:{$V(a) = 0$}] at (-pi, -0.02*pi*pi) {};
\node[draw, circle, fill, black, inner sep=1pt, label=above:{$V(b) = 0$}] at (pi, -0.02*pi*pi) {};
\end{tikzpicture} 
\caption{Variación de las curvas y puntos de inicio y final}
\end{wrapfigure}

Hasta aquí es sólo cálculo de primero. Llamamos ahora $V(t,λ) = \pd{α}{λ}(t,λ)$ al vector que indica hacia dónde varían las curvas. Dicho de otra forma, \[ V(t) = (V_1(t), V_2(t)) = \left(\pd{x}{λ}(t,0), \pd{y}{λ}(t,0)\right) \] y ahora sustituimos:

\[ \tilde{F}'(0) = \int_a^b \frac{1}{2}\left(2x'(t) V_1'(t) + e^{x(t)}V_1 y'^2 + e^x 2y'V_2' \right) \] usando el hecho de que todo sea $C^\infty$ y podamos intercambiar el orden de las derivadas:

\[ \pd{x'}{λ} = \pd{}{λ}\pd{x}{t} = \pd{}{t}\pd{x}{λ} = V_1' \]

Aplicamos ahora integración por partes para dejarlo todo en términos de $V_1$ y $V_2$, y no de sus derivadas. 

\begin{gather*} \int_a^b \underbrace{x'}_u \underbrace{V_1'\dif t}_{\dif v} = \eval{x'V_1}_a^b - \int_a^b V_1 x'' \dif t \\
\int_a^b \underbrace{e^xy'}_u \underbrace{V_2'\dif t}_{\dif v} = \eval{e^xy'V_2}_a^b - \int_a^b V_2(e^xx'y' + e^xy'')\dif t 
\end{gather*}

Finalmente, nos queda

\begin{align*} \tilde{F}'(0) &= \frac{1}{2} \left[ \eval{2x'V_1 + e^x2y'V_2}_a^b + \int_a^b -V_1 2 x'' + e^xV_1y'^2-V_2(ex'2y' + e^x2y'') \dif t \right] = \\
&= \frac{1}{2}\left[\eval{2x'V_1+e^x2y'V_2}_a^b + \int_a^b V_1(e^xy'^2-2x'') + V_2(-e^x2y'-e^x2y'') \dif t \right]
\end{align*}

\paragraph{Ecuaciones de Euler-Lagrange de este Lagrangiano} La curva va a ser un punto crítico del funcional integral. Fijamos los extremos: si todas las curvas salen del mismo punto $a$ y llegan al mismo punto $b$, entonces $V(a) = 0$ y $V(b) = 0$. Es decir, la primera parte de $\tilde{F}'$ desaparece y resolvemos dos ecuaciones:

\[ \begin{cases} e^xy'^2 - 2x'' &= 0 \\
-e^xx'2y' - e^x2y'' &= 0\end{cases} \begin{cases}  x'' &= \frac{1}{2}e^xy'^2 \\ 
y'' &= -x'y' \end{cases}\]

de tal forma que $\tilde{F}'(0) = 0$ para toda forma posible de variar α. 

Para resolverlo, sabemos que si $(x(t), y(t))$ es solución del sistema existen constantes $c_1, c_2$ tales que \[ \begin{cases} x'^2 + e^x y'^2&=c_1 \\
e^xy'&= c_2 \end{cases} \]

Vamos a ello:

\begin{gather*}
(e^xy')' = e^xx'y' + e^xy'' = e^x(x'y'+y'') = e^x(x'y' -x'y') = 0 \\
(x'^2 + e^xy'^2)' = 
\end{gather*}


\end{problem}

\begin{problem}[4] Sea $S$ una superficie, y $α(t)$ una geodésica birregular plana en $S$. 

\ppart Demostrar que el plano que contiene a α es perpendicular a $S$.
\ppart Demuestra que α es línea de curvatura de $S$.
\ppart Geodésicas planas en una superficie de revolución.
\solution

\spart $P$ es normal a $S$ si es normal a $T_αS$, es decir, si sus normales son ortogonales. Por lo tanto, $N_α⊆P$. Tenemos que ver entonces que $\cb (t) \perp N_α$, donde $N$ es la normal a la superficie $S$. 

Sabemos que α es geodésica, por lo tanto está parametrizada por arco. Al ser una geodésica, va con velocidad constante por la superficie. Cualquier aceleración que tenga deberá ser ortogonal a la superficie y por lo tanto $α''(t)$ está en la dirección de la normal: $α'' = λ(t) N_α(t)$. Operamos, y como α era birregular.

\[ α''(t) = \cv(t) \cn(t) = λ(t)N_α \]

\spart Hay que ver que $(N○α)'(t) = λ(t) α'(t)$. Es decir, que su tangente sea siempre una curvatura principal. Operamos sabiendo que $\cn(t) = \pm N_α(t)$:

\[ (N○α)'(t) = \cn' = -\cv \ct + \underbrace{\ctr}_0 \cb = -\cv α'(t) \]

\spart Si $α$ es birregular y geodésica plana en $S$, es una línea de curvatura según el apartado anterior. Al ser una superficie de revolución, son los paralelos o meridianos\footnote{También llamada curva generatriz}. Ahora bien, ¿son todos geodésicas?

Sólo son los meridianos (acordémonos de la esfera: sólo los meridianos son círculos máximos) son geodésicas siempre. Un paralelo será geodésica su la generatriz en ese punto es perpendicular a $N$. Necesito que el vector tangente a la generatriz sea normal a la normal a la superficie. El vector tangente a la generatriz tiene que ser paralelo al eje de giro.

\end{problem}

\begin{problem}[15] \textbf{Paraguas de Whitney}. Tenemos una superficie $S$ dada por la parametrización

\[ Φ(u,v) = (u,uv,v^2) \]

\ppart Comprobar que $Φ$ es regular en todo punto salvo $(0,0)$.
\ppart Hallar $II_Φ$ y comprobar que ciertos puntos son hiperbólicos.
\ppart Comprueba que las curvas $Φ(u,v_0)$ con $v_0$ constante son rectas y además curvas asintóticas.
\ppart Calcula las curvas asintóticas de $S$.

\solution 

\spart Calculamos la diferencial

\[ \Dif Φ(u,v) = \begin{pmatrix}
1 & 0 \\
v & u \\
0 & 2v \\ 
\end{pmatrix} \] y vemos que, efectivamente, tiene rango 2 salvo en $(u,v) = (0,0)$.

\spart Vamos a por la 2FF.

\begin{align*}
Φ_u &= ( 1,v,0) \\
Φ_v &= (0,u,2v) \\
Φ_{uu} &= (0,0,0) \\
Φ_{uv} &= (0,1,0) \\
Φ_{vv} &= (0,0,2) 
\end{align*}

Calculamos ahora los coeficientes. El primero es $0$ ya que $Φ_{uu} = \vec{0}$. Para el segundo necesitaríamos calcular el vector normal, pero podemos retrasarlo llamando $\norm{Φ_u×Φ_v}= l$ y calculando el determinante que nos da $\pesc{Φ_u×Φ_v, Φ_{uv}}$:

\[ lf = \left|\begin{matrix}
1 & 0 & 0 \\
v & u & 1 \\
0 & 2v & 0
\end{matrix}\right| = -2v \]. De la misma forma obtenemos $lg = 2u$.

Calculamos ahora sí $l$:

\[ l = \sqrt{EG - F^2} = \sqrt{u^2 + 4v^2 + 4v^4} 
\]  y la 2FF es 

\[ II_Φ = \frac{1}{ \sqrt{u^2 + 4v^2 + 4v^4} } \left(-4v\dif u \dif v + 2u \dif v^2 \right) \]

Buscamos ahora los puntos hiperbólicos a través de la curvatura gaussiana $K$, viendo dónde ésta sea negativa. 

\[ K = \frac{eg-f^2}{EG-F^2} = \frac{0-(-4v)^2}{l^4} = \frac{1}{l^4} (-16v^2) ≤ 0 \]

Es decir, que todos los puntos con $v≠0$ son hiperbólicos.

\spart Las curvas de la forma $Φ(u,v_0)$ se pueden expresar como

\[ Φ(u,v_0) = (u,uv_0, v_0^2) = u(1,v_0,0) + (0,0,v_0^2) \]
lo que nos dice que efectivamente esas curvas son rectas. Además, son curvas asintóticas. 

Recordamos que las curvas asintóticas cumplen que $II_α(t)(α'(t))=\pesc{α''(t), N}$. Dado que $α$ son rectas en este caso, $α''(t) = 0$ y por lo tanto la segunda forma fundamental se anula y son curvas asintóticas.

Vamos a ver mejor de dónde viene esa expresión de la 2FF. Sabemos que \[ II(α'(t) = \pesc{\dif N(α'(t)),α'(t)} \]. Podemos expresar $N(α'(t))$ como $(N○α)'(t)$. Ahora bien, si recordamos la fórmula de la derivada del producto escalar podemos reexpresar

\[ \underbrace{\pesc{(N○α)'(t),α'(t)}'}_{0} - \pesc{N○α(t),α''(t)} = \pesc{N○α(t), α''(t)} \]

donde la primera parte se anula al ser un producto escalar de un vector tangente a la superficie y otro normal a ella (son perpendiculares).

\spart

Vamos a hallar todas las curvas $α(t) = Φ(u(t), v(t))$ asintóticas, es decir, que cumplan $II_α(t)(α'(t)) = 0$. Podemos ahorrarnos la raíz fea $l$ y buscamos que

\[ lII_{α(t)} (α'(t)) = 0 \iff -4vu'v' +2uv'^2 = 0 \iff 2v'(-2u'v + uv') = 0 \]

La ecuación se anula si $v'\equiv 0$, lo que nos da las rectas que ya habíamos hallado. La segunda posibilidad es que \[ 2u'v = uv' \], una ecuación diferencial de variables separadas que podemos integrar:

\begin{gather*}
 \int \frac{2u'}{u} = \int \frac{v'}{v} \\
 \log u^2 - \log \abs{v} = C \\
 \log \frac{u^2}{\abs{v}} = C \\
 u^2 = \bar{C} \abs{v}
 \end{gather*} 
 
Tenemos que comprobar que no hay más rectas contenidas en el paraguas de Whitney. Si hubiera otra recta β que no hemos considerado, sería asintótica, y tendría la forma

\[ β(t) = Φ(u, \frac{u^2}{\bar{C}}) = \left(u,u\frac{u^2}{\bar{C}}, \frac{u^4}{\bar{C}^2}\right) 
\]
que no es una recta ni de coña. La demostración rigurosa podríamos hacerla o viendo que la curvatura no es constante o bien viendo que su proyección sobre los planos coordenados no es una recta.

\end{problem}

\begin{problem}[4]
Sea $Φ(u,v)$ una parametrización de una superficie con métrica de Riemann \[ Q \equiv e^v\dif u^2 + (1+u^2)\dif v^2 \].  

\ppart Sea $α(s) = Φ(u(s), v(s))$. Halla EL para el funcional de energía.
\ppart Coge la curva $v = 0$, parametrizamos por arco. Hallar el vector de curvatura geodésica a lo largo de $α_0$ dado de forma

\[ \cv[g,α_0,Q] = a_1(s) Φ_u + a_2(s) Φ_v \]
\solution

\spart Hallamos el funcional de energía \[ E(α) = \int_a^b Q_{α(t)} (α'(t)) \dif t = \int_a^b e^v u'^2 + (1+u)v'^2 \dif t \]. El de longitud es igual pero poniendo $\sqrt{Q}$ en lugar de $Q$.

Hallamos el nosequécojonesseráeso

\[ \tilde{E}'(0) = \int_a^b e^v V_2 u'^2 e^v 2u'V_1'+2uV_1v'^2 + (1+u^2)2v'V_2' \dif t \]

Cuando uso partes y supongo que en los extremos $V_1=V_2=0$ queda

\[ \tilde{F}'(0) = \int_a^b V_1(2uv'^2-e^vv'2u'^2-e^v4u'u'') + V_2(e^vu'^2-4uu'v'-2(1+u')v'') \dif t \]

\spart $α_0(s) = Φ(s,0)$. En la métrica que nos dan, 

\[\md{α_0'(s)}^2 =  Q(α_0'(s)) = e^v(s) · 1^2 + (1+s^2) ·0^2 = e^0 = 1 \], vemos que en la métrica que nos dan está parametrizada por arco.

Para hallar los valores, tenemos una formulita:

\[ \begin{pmatrix}
a_1(s) \\ a_2(s) 
\end{pmatrix} = \inv{[Q]} \left(\deriv{}{s}\left([Q]\begin{pmatrix}
u'(s) \\ v'(s)
\end{pmatrix}\right) - \frac{1}{2} \left[\begin{matrix}
Q_u(\ct[α_0]) \\ Q_v(\ct[α_0]) \end{matrix}\right]\right) \] donde

\[ [Q] = \begin{pmatrix}
e^v & 0 \\ 0 & 1+u^2
\end{pmatrix};\quad \begin{matrix}
Q_u &= 0 \dif u ^2 + 2u \dif v^2 \\
Q_v &= e^v\dif u^2 + 0\dif v^2
\end{matrix} \]

Seguimos calculando: \begin{gather*}
Q_v(\ct[α_0]) = e^0 = 1 \\
Q_u(\ct[α_0]) = 0
\end{gather*}

\[ \inv{[Q]} = \begin{pmatrix}
e^{-v} & 0 \\ 0 & \frac{1}{1+u^2}
\end{pmatrix} \], que evaluado en $α_0$ es \[ \inv{[Q]} = \begin{pmatrix} 1 & 0 \\ 0 & \frac{1}{1+s^2} \end{pmatrix} \]

Seguimos con más operaciones

\begin{gather*} [Q]\begin{pmatrix}
u' \\ v
\end{pmatrix} = \begin{pmatrix}u' \\ (1+s^2)v' \end{pmatrix} \\
\deriv{}{s} [Q] \begin{pmatrix}
u' \\ v
\end{pmatrix}  = \begin{pmatrix}
u'' \\ (1+s^2)v'' + 2sv'
\end{pmatrix} \end{gather*}

Juntándolo todo \footnote{Matadme.}

\[
\begin{pmatrix}
a_1(s) \\ a_2(s)
\end{pmatrix} = \begin{pmatrix}
u'' \\ v'' +\frac{2s}{1+s^2}v'
\end{pmatrix} - \frac{1}{2}\begin{pmatrix}
0 \\ \frac{1}{1+s^2}
\end{pmatrix} = \frac{1}{2} \begin{pmatrix} 0 \\ 1+s^2 \end{pmatrix} \] y finalmente

\[ \cv[g,α_0, Q] = \frac{1}{2}(1+s^2) Φ_v \]


\end{problem}

\begin{problem}[9] Tenemos una superficie $S$ y una parametrización $ψ(u,v)$, con métrica \[ Q = u\dif u^2 + u\dif v^2 \] con $u>0$. 

\ppart Hallar la ecuación de la parametrización geodésica.

\solution

\spart Si $ψ(u(t), v(t))$ es geodésica, entonces

\[ \deriv{}{t}\left([Q] \begin{pmatrix} u' \\ v' \end{pmatrix}\right) - \frac{1}{2} \begin{pmatrix} Q_u(α') \\ Q_v(α') \end{pmatrix} = \begin{pmatrix} 0 \\ 0 \end{pmatrix} \]

Me rindo.
\end{problem}

\subsection{Hoja 9}

\begin{problem}[1] Dadas las siguientes superficies $S$ y $Q$ sus métricas Riemannianas, halla la curvatura gaussiana $K$.

\ppart $Q = a^2 \dif u ^2 + b^2 \dif v ^2$.

\solution

Si $S$ estuviese dada como parametrización y $Q$ fuese la primera forma fundamental, ya sabríamos cómo obtener la curvatura. Pero no lo sabemos.

Escribimos la métrica como la suma de dos formas\footnote{Formas de adsafdsf} \[ Q = ω_1^2 + ω_2^2 \], y hallamos un $ω_3$ tal que\[ \dif ω_1 = ω_2 \y ω_3\quad \dif ω_2 = ω_3 \y ω_1 \]. De esta forma, obtendremos 

\[ \dif ω_3 = K (ω_1 \y ω_2 \]

\spart Sea $ω_1 = a\dif u$, $ω_2 = b \dif v$. El paso dos nos exige hallar $\dif ω_1$ y $\dif ω_2$ primero.

\[ \dif ω_1 = \dif (a\dif u) = \dif a \y \dif u = (a_u \dif u + a_v \dif v) \y \dif u = - a_v \dif u \y \dif v \]

Análogamente, nos queda que $\dif ω_2 = b_u \dif u \y \dif v $. Nos falta hallar ahora $ω_3$. Si no la vemos al principio, tenemos que darnos cuenta que $ω_3$ será de la forma $A \dif u + B \dif v$, y sustituimos en las ecuaciones.

\begin{align*}
 \dif ω_1 &= ω_2 \y ω_3 \\
 - a_v \dif u \y \dif v &= b \dif v \y (A\dif u + B \dif v) \\
 - a_v \dif u \y \dif v &= -Ab \dif u \y \dif v \\
 A &= \frac{a_v}{b}
\end{align*}

De la misma forma obtenemos $B = - \frac{b_u}{a}$, y entonces

\[ ω_3 = \frac{a_v}{b} \dif u - \frac{b_u}{a} \dif v \]

Hallamos la diferencial:

\begin{align*}
 \dif ω_3 &= \dif\left(\frac{a_v}{b} \dif u\right) - \dif = \left(\frac{b_u}{a} \dif v \right) = \\
 	&= \left(\frac{a_v}{b}\right)_u \dif u \y \dif u + \left(\frac{a_v}{b}\right)_v \dif v \y \dif u - \left(\frac{b_u}{a}\right)_u \dif u \y \dif v + \left(\frac{b_u}{b}\right)_v \dif v \y \dif v = \\
 	&= \left(-\left(\frac{b_u}{a}\right)_u - \left(\frac{a_v}{b}\right)_v \right)\dif u \y \dif v
\end{align*}

Comparamos ahora ese resultado con $Kω_1 \y ω_2$:

\[ \left(-\left(\frac{b_u}{a}\right)_u - \left(\frac{a_v}{b}\right)_v \right)\dif u \y \dif v = K \left(a\dif u \y b \dif v\right) = Kab \dif u \y \dif v 
\]
y por lo tanto
\[ K = \frac{-1}{ab} \left(\left(\frac{b_u}{a}\right)_u + \left(\frac{a_v}{b}\right)_v\right) \]
\end{problem}

\begin{problem}[2] Dado \[ \frac{\dif x^2 + \dif y^2}{(x^2+y^2+C)^2} \] di en qué lugar del plano $XY$ esto define una métrica, y calcula su curvatura gaussiana.

\solution

Podemos reexpresar la fórmula así:

\[ \frac{1}{(x^2+y^2+C)^2}  \dif x^2 +  \frac{1}{(x^2+y^2+C)^2} \dif y^2\]

Que se parece bastante a la primera forma fundamental $E \dif x^2 + 2F \dif x \dif y + G \dif y^2$. Podemos entonces escribir la matriz de esa forma fundamental: cuando esté definida positiva, será una métrica. Estudiamos

\[Q =  \begin{pmatrix}
 \dfrac{1}{(x^2+y^2+C)^2}  & 0 \\
0 & \dfrac{1}{(x^2+y^2+C)^2} 
\end{pmatrix} \]

Vemos que hay un problema cuando $x^2+y^2+C = 0$. Tenemos varios casos:

\begin{itemize}
\item Si $C > 0$, $x^2+y^2 + C > 0$ y $Q$ es definida positiva.
\item Si $C ≤ 0$, hay puntos $(x,y)$ tales que $x^2 + y^2 + C = 0$ (los de la circunferencia de radio $\sqrt{C}$). Salvo en esa circunferencia, $Q$ es definida positiva.
\end{itemize}

La curvatura gaussiana sale $4C$, por si os apetece hacerlo.

\end{problem}

\begin{problem}[3] Sea $c∈ℝ$ y $S_c$ la superficie parametrizada por

\[ Φ(u,v) = \left(\frac{u^3}{3} - uv^2, \frac{v^3}{3} - u^2v, cv\right) \]

\ppart Halla la primera forma fundamental $I_Φ$.
\ppart Ver que $X = (2cuv, cu^2-cv^2, (u^2+v^2)^2)$ es normal a $S_c$.
\ppart Halla $\md{X}$.
\ppart Calcula la segunda forma fundamental ($e,f,g$).
\ppart Calcula la curvatura gaussiana $K$ de dos formas diferentes.
\ppart Di cuándo es la métrica localmente isométrica al plano.

\solution

\spart Hallamos las derivadas:

\begin{align*}
Φ_u &= (u^2-v^2, -2uv, 0) \\
Φ_v &= (-2uv, v^2-u^2, c) \\
E   &= (u^2-v^2)^2 + 4u^2v^2 = (u^2 + v^2)^2 \\
F	&= 0 \\
G 	&= (u^2 + v^2)^2 + C^2
\end{align*}

Luego \[ I_Φ = (u^2 + v^2)^2 \dif u^2 + \left((u^2+v^2)^2 + C)\right) \dif v^2 \].

Salvo en el $(0,0)$, la 1FF es una métrica riemanniana y podemos calcular la curvatura gaussiana con la formulita del ejercicio anterior o con la otra \footnote{Ver cuál es la otra.}

\spart 

Podemos verlo fácilmente comprobando que $X$ es ortogonal a una base del plano tangente, es decir, viendo que $\pesc{X,Φ_u} = \pesc{X,Φ_v} = 0$.

\spart

Calculamos 

\begin{gather*}
 \md{X}^2 = \cdots = (u^2+v^2)^2\left((u^2+v^2)^2 + C^2\right) \\
 \md{X} = (u^2+v^2) \sqrt{(u^2 + v^2)^2 + C^2} = (u^2+v^2) \sqrt{G} = \sqrt{EG}
 \end{gather*}

\spart 

\begin{align*}
Φ_{uu} &= (2u,-2v,0) \\
Φ_{uv} &= (-2v, -2u, 0) \\
Φ_{vv} &= (-2u, 2v, 0) \\
e &= \frac{2cv}{\sqrt{G}} \\
f &= \frac{-2cu}{\sqrt{G}} \\
g &= \frac{-2cv}{\sqrt{G}}
\end{align*}

\spart La primera forma de calcular $K$ es usar la 1FF $I_Φ$ y lo que habíamos visto en el problema 1. Tenemos

\begin{gather*}
a = (u^2+v^2) \\
b = \sqrt{(u^2+v^2)^2 + C^2} \\
K  = \frac{-1}{ab} \left(\left(\frac{b_u}{a}\right)_u + \left(\frac{a_v}{b}\right)_v\right) = \frac{-4C^2}{G^2} 
\end{gather*}

La otra forma de calcularla es la formulita

\[ K = \frac{eg-f^2}{EG-F^2} = \frac{-4C^2}{G^2} \]

Siempre coinciden ambas formas cuando la métrica riemanniana es la 1FF.

\spart Hay que acordarse de varios detalles. Lo primero es que hay que recordar que $K$ se preserva mediante isometrías locales, ya que estas preservan la 1FF y por lo tanto preservan $K$, que sólo depende de la 1FF. 

Entonces, si $S_c$ es localmente isométrico mediante $\appl{ψ}{A⊆S_c}{B⊆ℝ^2}$ al plano, tenemos que tener que $K_{S_c} (p) = K_{ℝ^2} (ψ(p)) = 0$. Entonces, está claro que cuando $c$ sea $0$ $S_C$ será localmente isométrica al plano.

Ahora bien, esto es sólo una condición necesaria (el teorema egregio de Gauss nos da condición necesaria, no suficiente). Si $c=0$, ¿es $S_c$ localmente isométrica al plano? 

Usamos el teorema de Minding \eqref{thmMinding}, que contesta a la pregunta directamente: si tienen curvaturas constantes, iguales, las dos superficies son localmente isométricas.
\end{problem}

\begin{problem}[4] Sea $Q$ métrica Riemanniana en $S$, $c> 0 ∈ ℝ$. ¿Cuál es la relación entre la $K$ de $Q$ y la de $cQ$?

\solution

A grandes rasgos, multiplicar por la constante $c$ significa \textit{agrandar} la superficie (por ejemplo, de una esfera a una esfera con más radio). En este primer análisis vemos que debería disminuir la curvatura: vamos a probarlo.

Trabajamos de forma paralela en ambas superficies:

\[
\begin{matrix}
K_Q 				& K_{cQ} \\
Q = ω_1^2 + ω_2^2	& cQ = c(ω_1^2+ω_2^2) = (\sqrt{c}ω_1)^2 + (\sqrt{c}ω_2)^2 \\
\dif ω_1 = ω_2 \y ω 3 & \dif \bar{ω}_1 = \sqrt{c} \dif ω_1 = \sqrt{c} ω_2 \y ω_3  \implies \dif \bar{ω}_1 = \bar{ω_2} \y ω_3 \\ 
\dif ω_2 = ω_3 \y ω 2 & \dif \bar{ω}_2 = \sqrt{c} \dif ω_2 = \sqrt{c} ω_3 \y ω_1  \implies \dif \bar{ω}_2 = ω_3 \y \bar{ω}_1 \\
					& ω_3 = \bar{ω}_3 \\
\end{matrix} \]

Seguimos operando:

\[ \dif \bar{ω}_3 = K_{cQ} \bar{ω}_1 \y \bar{ω}_2 = c K_{cQ} ω_1 \y ω_2 \]

Por otra parte, como \[ \dif \bar{ω}_3 = \dif ω_3 = K_Q ω_1 \y ω_2 \] entonces

\[ K_Q = cK_{cQ} \]

\end{problem}

\begin{problem}[5] Dada la siguiente forma fundamental \[ I = \dif u^2 + 2u\dif u \dif v + \dif v^2 \] con $\abs{u} < 1$, demuestra que $S$ es localmente isométrica al plano.

\solution

Buscamos aplicar el teorema de Minding \eqref{thmMinding}: curvatura gaussiana y a ver si es $0$. Buscamos aplicar lo del problema 1 de forma pedestre.

\[ ω_1 = A \dif u + B \dif v\quad ω_2 = C\dif u + D \dif v \]

Calculamos los cuadrados y a ver qué sale.

\begin{align*}
ω_1^2 &= A^2\dif u^2 + 2AB \dif u \dif v  + B^2\dif v^2 \\
ω_2^2 &= C^2\dif u^2 + 2CD \dif u \dif v + D^2 \dif v^2 \\
ω_1^2 +ω_2^2 &= (A^2+C^2) \dif u^2 + 2(AB+CD)\dif u \dif v + (B^2+D^2)\dif v^2 
\end{align*}

lo que nos da un sistema

\[\begin{cases} 1 = A^2+C^2 \\ u = AB +CD \\ 1 = B^2 +D^2 \end{cases} \], que podemos resolver diciendo que $D=0$, y nos queda $A=u$, $B=1$, $C=\sqrt{1-u^2}$. Finalmente

\begin{align*}
ω_1 &= u\dif u + \dif v \\
ω_2 &= \sqrt{1-u^2} \dif u
\end{align*}

Buscamos ahora $\dif ω_1$ y $\dif ω_2$:

\[ \begin{cases}
\dif ω_1 &= 0 \\
\dif ω_2 &= 0 
\end{cases} \implies ω_3 = 0 \implies \dif ω_3 = 0 \implies K = 0 
\]
y ya está.

\end{problem}

\begin{problem}[6]\footnote{DIce que te puedes pasar una vida calculándolo. Miedito.} Sea $S_1$ una superficie dada por la parametrización

\[ Φ(u,v) = (\cos v -u\sin v, \sin v + u \cos v, v) \] y $S_2$ dada por 

\[ ψ(z,θ) = \left(\sqrt{1+z^2}\cos θ, \sqrt{1+z^2}\sin θ, z\right) \] que como se puede ver, es de revolución.

Halla una isometría local $\appl{h}{S_1}{S_2}$ expresada como \[ Φ(u,v) \longmapsto ψ(z(u,v), θ(u,v)) \]

Utiliza además que si $\appl{h}{S_1}{S_2}$ es isometría local, entonces $∀c∈ℝ$ la imagen por $h$ de la curva de nivel $\{K = c\}$ en $S_1$ van a puntos en la misma curva de nivel $\{ K = c \}$ en $S_2$.

\solution

Calculo la $K$ de $S_1$ y $S_2$ y veo qué ocurre con ellas. Como tenemos las parametrizaciones, podemos hallar la primera y segunda forma fundamental y obtener la curvatura. Al hacerlo, tenemos que

\begin{gather*}
K_{S_1}(u,v) = -\frac{1}{1+u^2)^2} \\
K_{S_2}(z,θ) = -\frac{1}{(1+2z^2)^2}
\end{gather*}

Entonces en un punto $(z(u,v),θ(u,v))$ la curvatura es

\[ K_{S_2} = - \frac{1}{(1+2z(u,v)^2)^2} \]

Si $h$ es isometría local, tiene que darse 

\begin{align*}
 K_{S_1}(u,v) &= K_{S_2}(z(u,v),θ(u,v)) \\
 - \frac{1}{(1+u^2)^2} &= -\frac{1}{(1+2z(u,v)^2)^2} \\
 z(u,v) &= \frac{u}{\sqrt{2}}
\end{align*}

Hemos sacado entonces que la pinta de $h$ tiene que ser algo como 
\[ Φ(u,v) \longmapsto ψ\left(\frac{2}{\sqrt{2}}, θ(u,v)\right)\]

Nos falta encontrar el ángulo, y ahora vamos a usar que $h$ es isometría local. En este caso, $\md{Φ_u}^2 = \md{\dif h (Φ_u)}^2$ ya que la norma preserva los vectores. Además, $\pesc{Φ_u, Φ_v} = \pesc{\dif h(Φ_u), \dif h (Φ_v)}$ y análogamente con ψ. Estas ecuaciones nos darán información sobre el ángulo θ.

\[ \md{Φ_u}^2 = E(u,v) = 1 \] y 

\[ \dif h(Φ_u) = ψ_z\left(\frac{u}{\sqrt{2}}, θ\right) · \frac{1}{\sqrt{2}} + ψ_θ\left(\frac{u}{\sqrt{2}}, θ\right) θ_u 
\], luego podemos calcular la norma al cuadrado:

\[ \md{\dif h(Φ_u)}^2 = \md{ ψ_z\left(\frac{u}{\sqrt{2}}, θ\right)}^2·\frac{1}{2} + 2 \frac{1}{\sqrt{2}}\pesc{ ψ_z\left(\frac{u}{\sqrt{2}}, θ\right),  ψ_θ\left(\frac{u}{\sqrt{2}}, θ\right)} θ_u = \bar{E}\left(\frac{u}{\sqrt{2}}, θ\right) \frac{1}{2} + \sqrt{2} θ_u \bar{F}\left(\frac{u}{\sqrt{2}}, θ\right) + \bar{G}\left(\frac{u}{\sqrt{2}}, θ\right) \]

Obtenemos $\bar{E}, \bar{F}, \bar{G}$ y sustituyendo cosas ahí, tenemos que \[ θ_u = \pm \frac{\sqrt{2}}{2+u^2} \], y cogemos el valor positivo por no complicarnos la vida.

Necesitamos también la derivada con respecto a $v$, información que vamos a sacar de la tercera igualdad o de la primera. Usamos que \[ \md{Φ_v}^2 = \md{\dif h(Φ_v)}^2 \]. Entonces

\[ \dif h(Φ_v) = \pd{}{v} h(Φ(u,v)) = \pd{}{v} ψ\left(\frac{u}{\sqrt{2}}, θ\right) = ψ_z\left(\frac{u}{\sqrt{2}}, θ\right) \cdot 0 + ψ_θ\left(\frac{u}{\sqrt{2}}, θ\right)θ_v \]

Aplicamos ahora la condición:

\begin{align*}
\md{Φ_v}^2 &= \md{\dif h(Φ_v)}^2  \\
2+u^2 &= θ_v^2 \md{ψ_θ\left(\frac{u}{\sqrt{2}}, θ\right)} \\
G(u,v) &= θ_v^2 \md{\bar{G}\left(\frac{u}{\sqrt{2}}, θ\right)}^2 \\
θ_v &= \sqrt{2}
\end{align*}

Ahora sólo falta hallar $z$ integrando. Integramos primero $θ_u$

\[ θ = \int \frac{\sqrt{2}}{2+u^2}\dif u = \arctan \frac{u}{\sqrt{2}} + C(v) \], derivamos con respecto a $v$:

\[ θ_v = c'(v) \implies c(v) = \sqrt{2}v + C_0 \] luego
\[ θ(u,v) = \arctan\frac{u}{\sqrt{2}} + \sqrt{2} v + C_0 \]
\end{problem}