\subsection{Hoja 1}

\begin{problem}[1] Sean $\appl{α,β}{ℝ}{ℝ^3}$ suaves. Hallar la derivada de cada una de estas funciones

\ppart $\appl{f}{ℝ}{ℝ}$ con $f(t) = \pesc{α(t), β(t)}$.
\ppart $\appl{g}{ℝ}{ℝ^3}$ con $g(t)=  α(t)×β(t)$.
\ppart $\appl{h}{ℝ}{ℝ^3}$ fijada $A$ matriz $3×3$, con $h(t) = Aα(t)$

\solution

\spart Primero desarrollamos $f$ y vemos qué nos sale. Tenemos que

\begin{gather*}
α(t) = \left(α_x(t), α_y(t), α_z(t) \right) \\
β(t) = \left(β_x(t), β_y(t), β_z(t) \right)
\end{gather*}

y por lo tanto tenemos que

\[ f(t) = α_x(t)β_x(t) + α_y(t)β_y(t) +α_z(t)β_z(t) \]

. Derivamos:

\[ f'(t) = α_x'(t)β_x(t) + α_x(t)β_x'(t) + α_y'(t)β_y(t) + α_y(t)β_y'(t) + α_z'(t)β_z(t) + α_z(t)β_z'(t) \]

Sin embargo, queremos reescribir esto en función de las aplicaciones originales α y β. Simplemente nos fijamos que esto es una suma de productos escalares:

\[ f'(t) = \pesc{α'(t),β(t)} + \pesc{α(t),β'(t)} \]

\spart

\spart


\end{problem}

\begin{problem}[2] Sea $α(t)$ una curva que no pasa por el origen. Si $α(t_0)$ es el punto de la traza de α más cercano al origen y $α'(t)≠0$, demuestre que el vector de posición $α(t_0)$ es ortogonal a $α'(t_0)$.

\solution

Tenemos que ver que $\pesc{α(t_0),α'(t_0)} = 0$. En $t_0$ tenemos un mínimo global de la distancia al origen, que es $d(t) = \md{α(t)}$. Usando el resultado del primer apartado del ejercicio anterior, $d(t)=\pesc{α(t), α(t)}^{\frac{1}{2}}$. Para ahorrarnos la raíz cuadrada, minimizamos la distancia al cuadrado $c(t)=\pesc{α(t), α(t)}$:

\[ c'(t)= 2\pesc{α'(t), α(t)} \]

Igualamos a cero:

\[ 0 = 2\pesc{α'(t), α(t)} = \pesc{α'(t), α(t)} \]

que es lo que queríamos demostrar. Además, como α no pasa por el origen y su derivada no es nula, sabemos que $d$ es diferenciable en todo punto (la raíz cuadrada no es diferenciable en el 0).
\end{problem}

\begin{problem}[5]
La curva engendrada por un punto P de una circunferencia de radio $r$ que rueda sin deslizar por una recta fija se llama cicloide. Tomando dicha recta como eje de las $X$, y como parámetro $t$ el ángulo orientado $MCP$ ($C$ es el centro de la circunferencia, y $M$ el punto de contacto con el eje), pruebe que la posición de $P$ para cada $t$ es
\[ α(t) = (rt − r \sin t, r − r \cos t) \]

Se ha supuesto que en $t = 0$, $P$ coincide con $M$, y con el origen de coordenadas. Determine los puntos t donde $α′(t) = 0$ (llamados de retroceso). (Nota: "sin deslizar" significa a efectos prácticos que la longitud del arco $MP$ coincide con la longitud del segmento $OM$).

\solution

\easyimg{img/Hoja1_5_Cicloide.png}{Cicloide.}{figEj5_Cicloide}

Viendo el dibujo \ref{figEj5_Cicloide} podemos ver claramente la definición de la parametrización, que es la que nos dan. Calculamos ahora $α'$:

\[ α'(t) = (r-r\cos t, r\sin t) \]

Igualando a 0, tenemos que $\cos t = 1$ y que $\sin t = 0$. Por lo tanto, los puntos en los que $α'(t)=0$ son de la forma $t=2kπ$, es decir, $(2r,0)$.


\end{problem}

\begin{problem}[8] Parametrizamos la espiral logarítima como

\begin{align*}
\appl{α&}{ℝ}{ℝ^2} \\
α(t) &= (e^{bt} \cos t, e^{bt} \sin t)
\end{align*}

con $b < 0$. Demostrar que cuando $t\to ∞$, $α(t)$ se acerca al origen. Calcule, para cada intervalo $[t_0, t_1] ⊆ ℝ$, la longitud de arco de $α$. Halle las ecuaciones de la parametrización por longitud de arco y demuestre que ningún arco suyo es circular.

\solution

\easyimgw{img/Hoja1_8_EspLog.png}{Espiral logarítimica.}{figEspLog}{0.7}

Calculamos $\lim_{t\to ∞} α(t)$. Dado que $\abs{\cos t} ≤ 1\;∀t∈ℝ$,

\[ \abs{e^{bt}\cos t } ≤ \abs{e^{bt}} = e^{bt} \]

y tomando límites, $\lim_{t\to ∞} e^{bt} = 0$ ya que $b$ es negativa. La demostración es análoga para la otra coordenada.

Pasamos ahora a calcular la longitud de cada intervalo. Para $[t_0, t_1] ⊆ ℝ$, tenemos que

\[ L_{t_0}^{t_1}(α)= \int_{t_0}^{t_1}\md{α'(t)} \dif t \]

Para hallar la parametrización por longitud de arco, tomamos un $t_0∈I$ en el dominio de α y calculamos

\begin{equation}\label{eq1_8} s(t) = \int_{t_0}^t \md{α'(τ)} \dif τ \end{equation}

Esta aplicación $s$ va de $I$ a $ℝ$, así que hallamos el intervalo $J=s(I)$. Despejaremos para $t=t(s)$ de (\ref{eq1_8}) y escribo $\appl{β}{J}{ℝ^2}$ con $β(s)=α(t(s))$. Operamos

\begin{align*}
α'(t) &= (be^{bt}\cos t - e^{bt} \sin t, be^{bt}\sin t + e^{bt}\cos t) \\
\md{α'(t)}^2 &= b^2e^{2bt}\cos^2 t + e^{2bt}\sin^2 t - 2be^{bt}\cos t e^{bt}\sin t + \\
& \quad + b^2e^{2bt}\sin^2 t + e^{2bt}\cos^2 t + 2be^{bt}\cos t e^{bt}\sin t = \\
&= b^2e^{2bt} + e^{2bt} = e^{2bt}(b^2+1) \\
\md{α'(t)} &= e^{bt}\sqrt{b^2 +1}
\end{align*}

Una vez que tenemos la expresión del módulo de la derivada, integramos. Elegimos $t_0 = 0$ para facilitar la integración:

\begin{align*}
s(t) &= \int_0^t e^{bτ}\sqrt{b^2 +1} \dif τ = \\
&= \frac{\sqrt{b^2+1}}{b} \left(e^{bt}\right|_0^t = \frac{\sqrt{b^2+1}}{b}(e^{bt}-1)
\end{align*}

Hallamos la imagen:

\[ s(ℝ) = \left(-\frac{\sqrt{b^2+1}}{b}, ∞\right) = J \]

Ahora despejamos $t(s)$:

\begin{align*}
s &= \frac{\sqrt{b^2+1}}{b}(e^{bt}-1) \\
\frac{b}{\sqrt{b^2+1}}s + 1&= e^{bt} \\
\log \left(\frac{b}{\sqrt{b^2+1}}s + 1\right)&= bt \\
t &= \frac{1}{b} \log \left(\frac{b}{\sqrt{b^2+1}}s + 1\right)
\end{align*}

y sustituimos en $α$ para hallar la parametrización por longitud de arco $β$:

\begin{multline*} β(s) = α(t(s))  = \left(\left(\frac{b}{\sqrt{b^2+1}}s + 1\right)\cos \left(\frac{1}{b} \log \left(\frac{b}{\sqrt{b^2+1}}s + 1\right)\right),\right. \\ \left. \left(\frac{b}{\sqrt{b^2+1}}s + 1\right)\cos \left(\frac{1}{b} \log \left(\frac{b}{\sqrt{b^2+1}}s + 1\right)\right)\right) \end{multline*}

\easyimgw{img/Hoja1_8_EspLogArco.png}{Buscamos un arco circular en la espiral logarítmica.}{figEspLogArco}{0.7}

Vamos ahora a comprobar si existe algún arco circular. Buscaremos si existe algún intervalo $I⊆ℝ$ tal que $α(A)$ sea el arco de una circunferencia (no necesariamente centrada en el origen), como se ve en la figura \ref{figEspLogArco}.

Si existiese tal circunferencia, habría algún centro $(a_0,b_0)$ y un radio $r>0$ fijo de tal forma que $∀t∈I\;\md{α(t) - (a_0,b_0)} = r$. La condición se puede expresar mejor como $\md{α(t) - (a_0,b_0)}^2 = r^2$. Operamos:

\[ \md{α(t) - (a_0,b_0)}^2 = (e^{bt}\cos t -a)^2 + (e^{bt}\sin t -b_0)^2 = r^2 \]

Si buscamos que la parte de la izquierda sea constante, derivamos y buscamos cuándo vale 0:

\begin{gather*}
2(e^{bt}\cos t -a_0)(be^{bt}-e^{bt}\sin t) + 2(e^{bt}\sin t - b_0)(be^{bt}\sin t + e^{bt} \cos t) =
\end{gather*}

Y se halla. Normalmente lo haríamos con una función de curvatura, viendo si es constante positiva o negativa.
\end{problem}

\begin{problem}[9] Cisoide.

Copias tú el enunciado.

\solution

Por la forma del problema vamos a escribir todo en polares.

Empezamos con $M_2$. La recta $x=2a$ se reescribe en polares como $r_{M_2}(θ) = \frac{2a}{\cos θ} = 2a \sec θ$.

Por otra parte, para reescribir $M_1$ reescribimos la circunferencia en polares y nos queda que $r_{M_1}(θ) = 2a\cos θ$. 

Para $P$, la distancia es la resta de las anteriores, entonces

\[ r_P(θ) = 2a \sec θ - \cos θ = 2a\left(\frac{1}{\cos θ} - \cos θ\right) \]

con $θ∈(-π/2, π/2)$.  Para hallar la ecuación implícita, 

\end{problem}

\begin{problem}[14] Envolvente de una familia de cuvas. 

\ppart Suponga que los puntos singulares forman una curva, trazada por $β(u) = (t(u), λ(u))$, entonces la envolvente es trazada por la parametrización $γ(u) = Φ(β(u))$. Demuestre que la curva $γ(u)$ es tangente a cada curva $α_{λ(u)}$ que se va encontrando.

\ppart Considere la siguiente familia de circunferencias: \[ α_{λ(t)} = (λ,λ) + λ(\sin t, \cos t) \] para $λ>0$. Halle la envolvente y haga un dibujo conjunto de la envolvente y la familia de circunferencias utilizando la identidad \[ \sin t  + \cos t = \sqrt{2} \sin \left(t + \frac{π}{4} \right) \].

\solution

\spart Calculamos las tangentes en el punto en el que se encuentran, las dos curvas. Si operamos, $γ(u) = α_{λ(u)}(t) \implies t = t(u)$.

\begin{align*}
γ'(u) &= \Dif Φ(t(u), λ(u)) \cdot (t'(u), λ'(u)) \\
α_{λ(u)}'(t(u)) &= \Dif Φ(t(u), λ(u)) \cdot (t'(u), λ'(u)) \cdot t'(u)
\end{align*}

Si nos fijamos, son iguales salvo por una constante, esto es, ambas curvas son tangentes.

\spart Hallamos la diferencial:

\[ \Dif Φ(t,λ) = \Dif α_λ(t) = \begin{pmatrix}
λ \cos t & 1 + \sin t \\
- λ \sin t & 1 + \cos t
\end{pmatrix} \], obtenemos su determinante

\begin{multline*}
\det \Dif Φ(t,λ) = λ \cos t (1+\cos t) + λ \sin t(1 + \sin t) = λ\cos t + λ\cos^2 t + λ\sin t + λ \sin^2 t = \\ 
 =  λ(\cos t + \sin t + 1) = λ\left(\sqrt{2} \sin \left(t + \frac{π}{4} \right) + 1\right)
 \end{multline*} y vemos que nos da cero cuando $λ=0$ (que no está en el rango que estudiamos, decíamos que $λ>0$) y cuando
 
\begin{align*}
	\sqrt{2} \sin \left(t + \frac{π}{4} \right) + 1 &= 0 \\
	\sin \left(t + \frac{π}{4} \right) &= \frac{-1}{\sqrt{2}}
\end{align*}

Esto nos da dos posibles soluciones. Por un lado, \[ t + \frac{π}{4} = \frac{-π}{4} \implies t = \frac{-π}{2} \]  y por otro \[ t + \frac{π}{4} = \frac{5π}{4} \implies t = π \]

Es decir, las curvas envolventes son los semiejes positivos $X$ e $Y$\footnote{En realidad con esos ángulos tenemos los semiejes negativos, pero no veo por qué puñetas me salen los ángulos al revés así que lo voy a dejar así.}

\end{problem}


\begin{problem}[16] Sea $b$ una constante no nula. Halle una parametrización por longitud de arco de la hélice

\[ α(t) = (a\cos t, a \sin t, bt) \]

\solution

Vemos que las dos primeras coordenadas van a ir describiendo una circunferencia de radio $a$ en el plano $XY$, donde $t$ será el ángulo formado con el eje de las $X$. La última coordenada nos da la altura, que es siempre creciente.

Para reparametrizar, buscamos cambiar cómo nos movemos por el parámetro sin variar el conjunto imagen en $ℝ^3$, según el siguiente esquema:

\begin{figure}[hbtp]
\centering
\begin{tikzpicture}
\node (I) at (0,0) {$t∈I⊆ℝ$};
\node (J) at (0,-3) {$s∈J⊆ℝ$};
\node (R) at (3,0) {$ℝ^3$};

\draw[->] (J) -- node[left] {$f$} (I);
\draw[->] (I) -- node[above] {$α$} (R);
\draw[->] (J) -- node[below right] {$β=α○f$} (R);
\end{tikzpicture}
\caption{Reparametrización.}
\label{figEj1}
\end{figure}

donde buscamos que $f$ sea un difeomorfismo (esto es, que exista su inversa y que sea diferenciable).

La reparametrización por arco de $α$ es la aplicación $β(s)$ construida como en (\ref{figEj1}) tal que $\md{β'(s)}=1$.

Sabemos que

\[ β'(s) = \deriv{β}{s}(s) = \deriv{α}{t}(f(s))\cdot \deriv{f}{s}(s) \]

y por lo tanto
\[ \abs{f'(s)} = \frac{1}{\md{α'(f(s))}} \]

Entonces necesitamos que $α'(t)≠0\; ∀t∈I$ (que sea regular). Además, si $f(s)$ es creciente

\[ f'(s) = t'(s) = \frac{1}{\md{α'(t(s)}} \implies s'(t) = \md{α'(t)} \]

y nos quedaría

\[ s(t) = \int_{t_0}^t \md{α'(t)}\dif t \]

Aplicamos todo esto al ejercicio en el que estamos.

\begin{gather*}
 α'(t) = (-a\sin t, a \cos t, b) \\
 \md{α'(t)} = \sqrt{a^2+b^2}
\end{gather*}

Integramos:

\[ s(t) = \int_0^t\md{α'(t)} \dif t = t\sqrt{a^2+b^2} \]

de tal forma que la inversa es \[ t(s) = \frac{s}{\sqrt{a^2+b^2}} \]

Nuestra reparametrización es

\[ β(s) =α(t(s)) = \left(a\cos \frac{s}{\sqrt{a^2+b^2}},a\sin \frac{s}{\sqrt{a^2+b^2}}, b \frac{s}{\sqrt{a^2+b^2}}\right) \]
\end{problem}

\begin{problem}[19] Si $\appl{α}{I}{ℝ^3}$ es una curva y $\appl{M}{ℝ^3}{ℝ^3}$ es un movimiento rígido de $ℝ^3$, demuestre que las longitudes de $α$ y $M○α$ entre $a$ y $b$ coinciden.

\solution
 
Un movimiento rígido conserva el movimiento escalar. Pasa algo. Y sale que sí. Hallas productos escalares, longitudes de arco y toda la pesca.

\end{problem}

