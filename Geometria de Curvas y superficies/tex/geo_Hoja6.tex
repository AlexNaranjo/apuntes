\subsection{Hoja 6}

\begin{problem}[1] Consideramos una superficie $C⊆ℝ^3$ que es un \textit{"cono"}, generado trazando rectas desde un punto fijo (vértice) hasta todos los puntos de una curva cualquiera (no tiene por qué ser una circunferencia.

Sea $S_1$ la esfera unidad centrada en el origen, y α la intersección del cono con $S_1$. $α(u)$ es una parametrización, y entonces tomamos $Φ(u,v) = vα(u)$ con $v>0$.

Comprueba que si $α$ es PPA, la 1FF de $Φ$ es la misma para cualquier cono.

\solution

Calculamos la 1FF:

\[ \begin{cases}
Φ_u = vα'(u) \\
Φ_v = α(u) \\
\end{cases}\implies \begin{cases}
E &= v^2α'^2(u) = v^2 \\
F &= vα'(u)α(u) = 0\\
G &= α^2(u) = 1
\end{cases} \]

Es decir, la curva no influye en ningún momento y por lo tanto la 1FF de $Φ$ es siempre la misma.
\end{problem}

\begin{problem}[3] Sea $Σ$ parametrizada por 

\[ Φ(u,v) = \left(r(u)\cos v,r(u)\sin v,z(u)\right) \].

Además, la normal unitaria es

\[ N = \left(-z'(u) \cos v, -z'(v) \sin v, r'(u) \right) \]

Calcular la primera y segunda forma fundamental, y algo sobre las curvaturas principales.
\solution

Podemos ver que $Σ$ es una superficie de revolución de la curva $(r(u), 0, z(u))$. Además, viendo la normal unitaria tenemos que $z'^2 + r'^2 = 1$, y por lo tanto la curva de revolución está parametrizada por arco. 

Calculamos la 1ff y la 2ff:

\begin{align*}
Φ_u &= \left(r' \cos v, r'\sin v, z'\right) \\
Φ_v &= \left(-r\sin v, r\cos v, 0\right) \\
Φ_{uu} &= \left(r''\cos v, r''\sin v, z''\right) \\
Φ_{uv} &= \left(-r' \sin v,-r'\cos v, 0\right) \\
Φ_{vv} &= \left(-r\cos v, -r \sin v, 0\right)
\end{align*}
lo que nos da los coeficientes
\begin{align*}
E &= r'^2 + z'^2 = 1 \\
F &= 0 \\
G &= r^2 \\
e &= \pesc{N,Φ_{uu}} = -r''z' + r'z''\\
f &= \pesc{N,Φ_{uv}} = rz'\\
g &= \pesc{N,Φ_{vv}} = rz' \\
\end{align*}

Pero además, al haber calculado antes la primera forma fundamental, podríamos sacar que

\[ N = \frac{Φ_u × Φ_v}{\norm{Φ_u×Φ_v}} = \frac{Φ_u × Φ_v}{\sqrt{EG-F^2}} 
\] y que además

\[ e = \pesc{N,Φ_{uu}} = \frac{1}{\sqrt{EG-F^2}} \det [ Φ_u | Φ_v | Φ_{uu} ] \]

De esta forma evitamos realizar cuentas excesivamente complejas.

Siguiendo con el ejercicio, la segunda forma fundamental sería

\[ Π_Φ = (-r''z'+r'z'')\dif u^2 + 2·0·\dif u \dif v + rz'\dif v ^2 \]

Hallemos ahora las \textbf{curvaturas principales}. La primera aplicación es $\appl{?}{T_pS}{T_pS}$ que lleva un vector $v$ a un vector $\Dif_vN$. La matriz de la aplicación está dada por 

\[ \begin{pmatrix}
E & F \\ F & G
\end{pmatrix}^{-1}
\begin{pmatrix}
e & f \\ f & g
\end{pmatrix} = \begin{pmatrix}
-r''z' + r'z'' & 0 \\ 0 & \frac{z'}{r}
\end{pmatrix}\]

Las curvaturas principales son los autovalores de esa matriz, que como es diagonal ya los tenemos escritos.

Si por ejemplo nos hubieran dado la curvatura gaussiana $K$ y la media $H$, podríamos recuperar las curvaturas principales $k_1,k_2$ a través de sus respectivas fórmulas:

\begin{gather*}
 K = k_1k_2 \\
 H = \frac{-(k_1 + k_2)}{2}
\end{gather*}

\end{problem}

Una curva α es asintótica cuando $II_{α(t)}(α'(t)) = 0$. Es decir, la curva siempre va buscando ser tangente a una dirección que anule la 2FF.

\begin{problem}[6] \textbf{Limones localmente isométricos a la esfera.} Tenemos la esfera sin un meridiano dada por la parametrización
\[
Φ(u,v) = (\cos u\cos v,\cos u \sin v, \sin u)\quad u∈\left(-\frac{π}{2},\frac{π}{2}\right),\;v∈(0,2π) 
\]
que es una superficie de revolución de la semicircunferencia dada por $ρ(u) = \cos u$, $h(u) = \sin u$.

\ppart Para cada $c>1$ encuentra $r_c(u), z_c(u)$ con $u∈\left(-\frac{π}{2},\frac{π}{2}\right)$ tal que si \[ ψ^c(u,μ) = \left(r_c(u)\cos μ, r_c(u)\sin μ, z_c(u)\right) \] con $μ∈ℝ$, entonces la aplicación $\appl{h}{S}{S_c}$ que a $Φ(u,v)$ le asigna $ψ^c(u,cv)$ es isometría local.

\solution

Para que $h$ sea isometría, la 1FF de Φ y ψ tienen que ser iguales. Nos molesta la $c$ que están en $ψ(u,cv)$ para poder aplicar lo que veíamos en el problema 1 de la hoja 5, así que vamos a arreglar esto.

Sea $ Γ(u,v) = ψ^c(u,cv)$. Basta ver que los coeficientes de la primera forma fundamental de Φ y de Γ coinciden. Vamos a calcularlos:

\begin{align*}
E_Φ &= 1\\
F_Φ &= 0\quad \text{(sup. de revolución)} \\
G_Φ &= \cos^2 u \\
\end{align*}

Calculamos las derivadas de Γ
\begin{gather*}
Γ_u = (r_c' \cos cv,r_c'\sin cv,z_c') \\
Γ_v = (-r_c'c\sin cv, r_c c \cos cv, 0) 
\end{gather*}
y obtenemos los coeficientes

\begin{align*}
E_Γ &= r_c'^2 + z_c'^2 \\
F_Γ &= 0 \\
G_Γ &= c^2r^2
\end{align*}.

Igualamos los coeficientes y tenemos

\begin{align*}
r_c'^2 + z_c'^2 &= 1 \\
c^2r^2(u) &= \cos^2 u
\end{align*} lo que nos da las dos funciones

\begin{gather*}
z_c(u) = \int\sqrt{1-\frac{1}{c^2}\sin^2u}\dif u \\
r_c(u) = \frac{1}{c}\cos u
\end{gather*} 

El enunciado ahora nos pide ver qué le ocurre al ecuador de esta superficie cuando $c$ aumenta, y lo que vemos es que se acerca al eje del origen.

También nos pide demostrar que la altura del limón es creciente en $c$. La altura del limón es

\[ z_c\left(\frac{π}{2}\right)-z_c\left(\frac{-π}{2}\right) = \int_{-π/2}^{π/2} \sqrt{1-\frac{1}{c^2} \sin^2 s}\dif s = f(c) \]. 

Obtenemos la derivada

\[ f'(c) = \int_{-π/2}^{π/2} \left(1-\frac{1}{c^2}\sin s\right)^{\frac{-1}{2}} · \frac{1}{2} · \frac{2}{c^3} \sin^2 s \dif s 
\]
que es siempre positiva, y por lo tanto es creciente. Cuando $c\to ∞$, converge a \[\lim_{c\to ∞} \int_{-π/2}^{π/2} \sqrt{1-\frac{1}{c^2}\sin^2u}\dif u = \int_{-π/2}^{π/2} 1 = π \].

Vamos ahora con las \textbf{curvaturas principales}. En el ejercicio anterior se dice que una superficie de revolución $(r(u),z(u))$ tiene curvaturas principales \[ r'z'' - r''z',\quad \frac{z'}{r} \]. Queremos saber si conserva la isometría $h_c$ el par no ordenado de curvaturas principales. En nuestro caso, derivando y operando

\begin{gather*}
 r'z'' - z'r'' = \frac{1}{c}\cos u \left(1-\frac{1}{c^2}\sin^2 u\right)^{-\frac{1}{2}} \\
 \frac{z'}{r} = \frac{\left(1-\frac{1}{c}\sin^2 u\right)^{1/2}}{\frac{1}{c}\cos u}
 \end{gather*}

¿Para qué queremos todos estos cálculos? Los limones son localmente isométricos a la esfera. La esfera tiene curvaturas principales iguales a $1$ en todos los puntos. Sin embargo, nuestros limones tienen curvaturas principales distintas. De aquí podemos extraer la sabia lección de que una isometría local no conserva las curvaturas principales.

Pero hay otra cosa: el producto de las curvaturas principales es igual a la curvatura gaussiana. La curvatura gaussiana de la esfera es uno, y la de los limones es también uno. Es decir, \textbf{la curvatura gaussiana} se conserva en las isometrías locales.
\end{problem}

\begin{problem}[7] Sea $α(u)$ una curva birregular en el espacio PPA. Sea $S$ la parte de la superficie tangencial dada por la parametrización \[ Φ(u,v) = α(u) + v \ct(u) \] con $v > 0$.

Describe la normal unitaria $N(u,v)$ de $S$ en términos del triedro de Frenet $\{\ct(u), \cn(u),\cb(u)\}$ de α. A partir de eso, calcula el endomorfismo de Weingarten de $S$ sin pasar por la segunda forma fundamental. Demuestra que cada punto de $S$ es parabólico o plano. ¿Cuándo tiene $S$ puntos planos?

\solution

Calculamos las derivadas

\begin{align*}
Φ_u &= \ct(u) + v \cv(u) \cn(u) \\
Φ_v &= \ct(u)
\end{align*}, el producto vectorial

\[ 
Φ_u×Φ_v = 0 + v\cv(u) (-\cb(u)) 
\]
y por último la normal unitaria
\[
N = \frac{Φ_u×Φ_v}{\norm{Φ_u×Φ_v}} = \frac{-v\cv(u)\cb(u)}{\norm{-v\cv(u)\cb(u)}} = - \cb(u) \].

Vamos a calcular ahora el \textbf{endomorfismo de Weingarten} sin usar la 2FF. ¿Cómo calculamos esta cosa? Tenemos que calcular la aplicación $\appl{\dif N}{T_pS}{T_pS}$ evaluada en $Φ_u$ y en $Φ_v$. Es

\begin{gather*}
 \dif N (Φ_u )= \od{}{s}(N(u+s,v)) = N_u = \od{}{u} ( -\cb(u)) = -\cb'(u) = \ctr(u) \cn (u) \\
 \dif N (Φ_v) = N_v = \od{}{s}(-\cb(u)) = 0
 \end{gather*}

Tenemos que escribir $\dif N(Φ_u)$ en función de $Φ_u$ y $Φ_v$, entonces

\[ \ctr \cn = a Φ_u + b Φ_v =  a \ct + a v \cv \cn + b \ct \] luego $a+b=0$ y además $av\cv = \ctr$, por lo que \begin{align*}
a &= \frac{\ctr}{v\cv} \\ 
b &= \frac{-\ctr}{v\cv}
\end{align*}

y entonces

\[ \dif N = \begin{pmatrix}
\dfrac{\ctr}{v\cv}  & 0\\ 
\dfrac{-\ctr}{v\cv} & 0
\end{pmatrix} \]

\index{Punto!plano}
\index{Punto!elíptico}
\index{Punto!parabólico}
\index{Punto!hiperbólico}
Vamos ahora con la cosa de los puntos que nos decían. Los puntos se dividen en \textbf{elípticos, parabólicos, planos e hiperbólicos}. Los elípticos tienen curvatura gaussiana positiva, los hiperbólicos negativa, y los planos y parabólicos 0. Más concretamente, en los puntos planos ambas curvaturas principales son 0, mientras que en los parabólicos sólo lo es una.

Apliquemos esto a los puntos de nuestra superficie. Tenemos núcleo de $\dif N$ (hemos visto que $\dif N(Φ_v) =0$) por lo que no podemos tener puntos hiperbólicos ni elípticos. Vamos a ver si somos capaces de decir cuándo los puntos son planos.

Para que eso ocurra, tenemos que tener que toda la diferencial $\dif N$ sea 0. Es decir, que si $\ctr = 0$ tenemos puntos planos, y si no, tenemos puntos parabólicos.

Además, al calcular $\dif N$ hemos calculado la segunda forma normal, ya que 

\[ II_p(v) = I_p(v,\pm \dif N(v)) \].

\end{problem}

\begin{problem}[8] Sea $α(u)$ curva birregular en el espacio PPA con torsión constante 1 y con $\cv(u) > 0$. 

\ppart Siendo $\ast$ el triedro de Frenet de α definimos una supercicie algop.

\ppart 

\ppart Hallar $N$ normal unitaria tal que $N\cn(u) > 0$ y demuestra que $II_s$ es un engendro.

\solution

\spart Calculamos la 1FF. 
\[ \begin{cases}
Φ_u &= \ct(u) - v \cn (u) \\
Φ_v&= \cb (u) 
\end{cases} \implies \begin{cases}
E &= 1+v^2 \\
F &= 0  \\
G &= 1
\end{cases} \].

\spart Tenemos $α_1,α_2$ ambas PPA con torsión 1 y curvaturas distintas $k_1,k_2$. Hay que demostrar que la aplicaicón $\appl{h}{S_1}{S_2}$ que lleva $Φ_1(u,v)$ a $Φ_2(u,v)$ es isometría local. EN el anterior apartado hemos calculado la primera forma fundamental, que no dependía de la curvatura. Por lo tanto, como la 1FF coincide es una isometría local.

\spart Calculamos la normal, eligiendo el signo menos para que se cumpla la condición del apartado.

\[ - \frac{Φ_u×Φ_v}{\norm{Φ_u×Φ_v}} = - \frac{-\cn -v\ct }{\sqrt{1+v^2}} \]

Calculamos ahora la segunda forma fundamental \footnote{Estoy empezando a perder capacidades. Dos horas seguidas copiando esto no puede ser bueno.} Calculamos las derivadas segundas

\[
\begin{cases}
Φ_{uu} &= \cv \cn - v (-\cv \ct + \ctr \cb) \\
Φ_{uv} &= -\cn \\
Φ_{vv} &= 0
\end{cases} \implies \begin{cases}
e &= \pesc{Φ_{uu},N} = \cv \sqrt{1+v^2} \\
f &= \pesc{Φ_{uv},N} = -\frac{1}{1+v^2} \\
g &= \pesc{Φ_{vv},N} = 0
\end{cases} \]
que es igual que el engendro que nos decían en el enunciado.

Vamos a hallar ahora las líneas asintóticas\footnote{No tengo ni la más menor idea de qué puñetas es esto}. La condición de las líneas asintóticas es $II(α') = 0 \; ∀t$. Cogemos una curva $α(t) = Φ(u(t),v(t))$. Sabemos que las coordenadas de $α'$ en la base $\{Φ_u,Φ_v\}$ son $(u',v')$. Por lo tanto, sólo tenemos que coger $u'$ y $v'$ y meterlos en la 2FF.Entonces

\[ II (α'(t)) = \sqrt{1+v^2} \cv u'^2 - \frac{2}{\sqrt{1+v^2}}u'v' \]
lo que nos da una ecuación diferencial con esta pinta:
\[ \sqrt{1+v^2} \cv (u) u'^2 - 2u'v'\frac{1}{1+v^2} = 0\].

Podemos sacar $u'$ como factor común y la raíz esa también y 

\[ \frac{u'}{\sqrt{1+v^2}}\left(\cv u'(1+v^2) - 2v'\right) = 0 \], lo que nos da dos posibilidades. Puede ser $u'(t) = 0$ y entonces $u(t) = u_0$, lo que no es nada raro porque es una curva dentro de la superficie y algo asintótico. 

La otra ecuación a resolver es el engendro ese de ahí \footnote{Matadme ya.} \[ ku' = \frac{2v'}{1+v^2} \], que integrando sale algo.
\end{problem}

