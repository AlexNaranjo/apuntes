\subsection{Hoja 3}

\begin{problem}[2] Tenemos una curva $α=α(s)$ PPA biregular. Demuestra que 

\[ \ctr (s) = \frac{α'(s) × α''(s) \cdot α'''(s)}{\abs{\cv(s)}^2} \]

\solution

Sabemos que 

\begin{align*}
α'(s) &= \ct (s) \\
α''(s) &= \ct'(s) = \cv (s) \cn (s) \\
α'''(s) &= \cv'(s) \cn(s) + \cv(s) \cn'(s) = \\
&= \cv'(s) \cn(s) + \cv(s) \left(-\cv(s)\ct(s) + \ctr(s) \cb(s) \right)
\end{align*}

Mientras arreglaba una cosa de \LaTeX me ha escrito una pizarra y me he perdido. Así que nada.

\end{problem}

\begin{problem}[3] Sea $α(s)$ una curva birregular PPA, con $\cv$ y $\ctr$ curvatura y torsión. Dado $β(s) = \ct(s)$, comprobar que es regular y que 

\[ \cvv[β] = \frac{\ctr}{\cv}\cb(s) - \ct(s) \]


\solution

\spart Defino $β(s) = \ct(s)$. Comprobamos si es regular

\[ β'(s) = \ct'(s) = \cv(s) \cn (s) ≠ 0 \]

ya que $k_α(s)≠0$ para todo $s$ ya que α es birregular.

\spart Queremos demostrar que

\[ \cvv[β] = \frac{\ctr}{\cv}\cb(s) - \ct(s) \]

No sabemos si β está parametrizada por arco. Viendo la ecuación anterior, $\md{β'} = \md{\cv}$, que es distinto de uno si la curvatura no es constante igual a 1 y por lo tanto β no es PPA.

En esta situación tenmos varias posibilidades. O bien primero reparametrizamos β por arco $r$ tal que $β(s(r)) = γ(r)$ y hallo $γ''(r)$ o bien hallo $\cvv[β](s)$  usando la fórmula para curvas no PPA. 

\paragraph{Reparametrización por arco} Sea $r$ el parámetro de arco de β. Entonces

\[ r(s) = \int_{s_0}^s \md{β'(z)}\dif z = \int_{s_0}^s \cv(z) \dif z;\quad r'(s) = \cv(s) \] 

Sea $γ(r) = β(s(r))$ la reparametrización de β por arco $r$. Hallamos las dos derivadas de γ:

\[ γ'(r) = β'(s(r)) · s'(r) = \cv(s(r)) \cn(s(r)) · \frac{1}{\cv(s(r))} = \cn (s(r)) \]

\begin{align*} 
\cv[γ](r) &= γ''(r) = \cn'(s(r)) s'(r) =& \text{(Eqs. Frenet)}\\
  &= \left (-\cv(s(r))\ct(s(r)) + \ctr(s(r))\cb(s(r))\right) \frac{1}{\cv(s(r))} =& \\
  &= \frac{\ctr}{\cv}\cb(s) - \ct(s) &
\end{align*}

\end{problem}

\begin{problem}[4]
Sea $\appl{α}{I}{ℝ^3}$ curva birregular y PPA. Todas sus rectas normales pasan por el mismo punto $p_0$. Demuestre que la traza de α está contenida en una circunferencia.

\solution

Sabemos que una circunferencia dentro de un plano tiene curvatura constante y torsión 0, con el recíproco también cierto. Vamos a demostrar que si todas las rectas normales pasan por el mismo punto, tienen que darse esas dos condiciones sobre la curvatura y torsión.

Escribimos la condición de las rectas normales. Para un punto $s$ en la curva, la recta normal es

\[ α(s) + λ(s)\cn(s) = p_0 \]

con $λ(s)∈ℝ$. Derivamos:

\begin{gather*}
\ct + λ'\cn + λ\cn' = 0 \\
\ct + λ'\cn + λ(-\cv \ct + \ctr \cb) = 0
\end{gather*}

Dado que $\ct$, $\cb$ y $\cn$ forman una base l.i. del diedro ortonormal, para que esa ecuación sea 0 todos los coeficientes han de ser 0. Es decir

\[ 1-\cv λ = 0;\quad λ' = 0;\quad λ\ctr = 0 \]

De $λ'=0$ obtenemos que $λ=c_0≠0$, por lo que 

\begin{align*}
1-\cv λ = 0 &\implies \cv = \frac{1}{c_0} \\
λ\ctr = 0 &\implies \ctr = 0
\end{align*}

Ambos resultados implican que α es un trozo de circunferencia.
\end{problem}

\begin{problem}[7] Sea $\appl{α}{I}{ℝ^3}$ PPA con $\cv > 0$. Demostrar que $α(I)$ es un arco de circunferencia si y sólo si $\cv$ es constante y α está contenida en una esfera.

\solution

La implicación a la derecha es trivial. Muy trivial. 

Para la implicación a la izquierda, tenemos que ver que $\ctr\equiv 0$, ya que por hipótesis nos han dicho que $\cv$ es constante. Dado que α está contenida en la esfera, la condición la escribimos como

\[ \md{α(s) - p_0}^2 = R^2 \]

para $p_0$ el centro de la esfera. Dado que eso es una función de una variable y tiene que ser constante, derivamos porque a este hombre le encanta derivar

\[ f'(s) = 2\pesc{\ct,α(s)-p_0} = 0 \]

\begin{center}\Huge{Más derivadas}\end{center}

\[ \frac{f''(s)}{2} = \pesc{\ct', α(s)-p_0} + \pesc{\ct, \ct} \]

Que podemos reexpresar que 

\[ \pesc{\cn, α(s) - p_0} = \frac{-1}{\cv} \]

Y...

\easyimg{img/DERIVADAS.png}{DERIVAAAAAMOOOOOOOOOOOOOS}{imgDerivadas}

\begin{gather*}
 \pesc{\cn', α(s)-p_0} + \pesc{\cn, \ct} = 0 \\
 \pesc{-\cv\ct + \ctr\cb, α(s)-p_0} = \ctr \pesc{\cb, α(s)-p_0} = 0
 \end{gather*}
 
 Suponemos que existe un $s_0$ donde $\ctr(s_0)≠0$. En un entorno alrededor de ese punto, se tiene que 
 
 \[ \pesc{\cb, α(s)-p_0} = 0 \]
 
 Entonces
 
 \begin{gather*}
  α(s)-p_0 = \frac{1}{\cv}\cn(s) \\
  α(s) = p_0 - \frac{1}{\cv}\cn(s) \\
  \ct = \frac{1}{\cv}\left(-\cv\ct + \ctr \cb\right) \\
  \frac{-\ctr}{\cv}\cb = 0
  \end{gather*}
  
  Lo que nos llevaría a que $\ctr=0$, contradicción.
 
\end{problem}

