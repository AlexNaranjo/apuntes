\subsection{Hoja 2}


\begin{problem}[2] \textbf{Tacnodo vertical.} Sea $C$ la curva definida implícitamente por $y^4=x^2(x+1)$ en todo el plano $xy$. Demuestre que la parte de $C$ cercana al punto $(0,0)$ es la unión 

\[ \{ x= h_1(y)\} \cup \{ x=h_2(y) \} \]

de los grafos de dos funciones suaves $h_1,h_2$, definidas en $-ε<y≤ε$ y tales que $y=0$ es mínimo local de $h_1$ y máximo local de $h_2$.

Indicación: estudie si $φ(x) \equiv x\sqrt{x+1} $ tiene inversa cerca de $x=0$. Dibuje $C$ y demuestre que $C\cap\left( ℝ^2\setminus\{(0,0)\}\right)$ es suave y sin autointersecciones.

\solution

\easyimg{img/Hoja2_2_Tacnodo.png}{Tacnodo vertical}{imgTacnodo}

Consideramos $\appl{F}{ℝ^2}{ℝ}$ dada por 

\[ F(x,y) = y^4 - x^2(x+1) \]

Está claro que $C\cap ℝ^2\setminus (0,0) = \inv{F}(0)$. Calculemos el gradiente de $F$:

\[ \grad F (x,y) = \left(-3x^2-2x, 4y^3\right) =\left(x(-3x-2), 4y^3 \right) \]

Queremos ver si el gradiente se anula en algún punto de nuestra curva. Vemos que $\grad F = 0$ en dos casos:

\[ (x,y) = (0,0);\quad (x,y) = \left(-\frac{2}{3}, 0\right) \]

Sin embargo, $(0,0)$ no está en la curva, y dado que $F \left(-\frac{2}{3}, 0\right) = -\frac{4}{27} ≠ 0$  $\left(-\frac{2}{3}, 0\right)$ tampoco está. Por lo tanto, la función es \textbf{suave y sin autointersecciones}.

Vamos ahora a demostrar la segunda parte del problema. Buscamos dos funciones $h_1$ y $h_2$. Nos piden que las funciones dependan de $y$ y no de $x$, así que tenemos que comprobar que $x\sqrt{x+1}$ tiene una inversa cerca de cero para saber que podemos despejar $x$ en $φ(x) = y$ Calculamos la derivada de $φ$:

\[ φ'(x) = \sqrt{x+1} + x\frac{1}{2\sqrt{x+1}} \]

que no se anula en $φ'(0)$. Por el Teorema de la Función Inversa, existen $δ>0, ε>0$ tales que $\appl{\inv{φ}}{(-ε,ε)}{(-δ,δ)}$ existe y es diferenciable.  

Entonces, si $x≥0$, $(x,y)∈C \dimplies y^2 = φ(x)$, y de la misma forma si $x≤0$, $(x,y) ∈ C \dimplies y^2 = -φ(x)$. Por lo tanto, podemos construir las dos funciones $h_i$ de la siguiente forma:

\[ h_1(y) = \inv{φ}(y^2);\quad h_2(y) = \inv{φ}(-y^2) \]

\end{problem}

\begin{problem}[4] $\appl{α}{(-a,a)}{ℝ^2}$ PPA. Defino $β(s)=α(-s)$, de tal forma que cambiamos la orientación. Verifica que β PPA y halle la curvatura.

\solution

Obvio que β también está PPA.
\end{problem}

\begin{problem}[5] Halle una curva plana parametrizada por longitud de arco tal que 

\[ k_α(s) = \frac{1}{1+s^2} \]

\solution

Hallamos el ángulo:

\[ θ(s) = \int \frac{1}{1+s^2}\dif s = \arctan s \]

Despejamos la arcotangente:

\begin{wrapfigure}{r}{0.4\textwidth}
\begin{tikzpicture}
\coordinate (A) at (0,0);
\coordinate (B) at (4,0);
\coordinate (C) at (4,2);

\draw[-] (A) -- node[above,sloped] {$\sqrt{1+s^2}$} (C);
\draw[-] (A) -- node[below,sloped] {$1$} (B);
\draw[-] (C) -- node[right] {$s$} (B);

\begin{scope}
\path[clip] (A) -- node[below,sloped] {$1$} (B) -- node[right] {$s$} (C) -- node[above,sloped] {$\sqrt{1+s^2}$} (A);
\fill[green, opacity=0.3, draw=black] (A) circle (5mm);
\node at ($(A)+(10:12mm)$) {$\theta(s)$};
\end{scope}
\end{tikzpicture}
\end{wrapfigure}

\begin{gather*}
\cos θ(s) = \frac{1}{\sqrt{1+s^2}} \\
\sin θ(s) = \frac{s}{\sqrt{1+s^2}}
\end{gather*}

Integrando, tenemos que 

\begin{align*}
x(s) &= \log \left(s + \sqrt{s^2+1}\right) \\
y(s) &= \sqrt{1+s^2}
\end{align*}

Añadiendo el movimiento rígido tendríamos todas las curvas que cumplen esa ecuación:

\[ β(s) = \begin{pmatrix}
\cos θ_0 & - \sin θ_0 \\
\sin θ_0 & \cos θ_0
\end{pmatrix}\begin{pmatrix}
 \log \left(s + \sqrt{s^2+1}\right) \\
 \sqrt{1+s^2}
\end{pmatrix} + \begin{pmatrix}
x_0 \\ y_0 
\end{pmatrix} \]
\end{problem}

\begin{problem}[8] Sea $\appl{α}{I}{ℝ^2}$ PPA. Demuestra

\ppart α es segmento de recta si y sólo si $∃p_0∈ℝ^2$ por el cual pasan todas las rectas tangentes.

\ppart α es un arco de circunferencia si y sólo si $∃p_0∈ℝ^2$ que esté en todas las rectas normales.

\solution

\spart Empezamos demostrando la implicación a la derecha. Si $α$ es un segmento de recta, lo podemos escribir como 

\[ α(s) = p_0 + s\vv \]

con $\vv$ unitario. La recta tangente a $α$ en $s=s_0$ será 

 \[ α(s_0) + λα'(s_0) = p_0 + s_0\vv + λ\vv = p_0 + (s_0 + λ)\vv \]
 
y por lo tanto $p_0$ está en todas las rectas tangentes a $α$.

Ahora la implicación a la izquierda. Para $s∈I$, sabemos que la recta tangente a $α$ por $s$ es

\begin{equation}\label{eqH2E8} α(s) + λα'(s)\quad λ∈ℝ \end{equation}

Como $p_0$ está en esa recta, $∃λ(s)\tq p_0 = α(s) + λ(s)α'(s)$. Ahora la idea que se nos viene a la cabeza es derivar, pero no sabemos si λ es derivable. Despejamos y 

\[ p_0 - α(s) = λ(s)α'(s) \]

Ahora no podemos dividir por $α'$ (es un vector) así que mltiplicamos a ambos lados por $α'$:

\[ \pesc{p_α - α(s), α'(s)} =λ(s) \pesc{α'(s), α'(s)} = λ(s) \]

por lo tanto $λ(s)$ es suave en $s$. Podemos derivar, así que derivamos en (\ref{eqH2E8})

\begin{gather*}
 0 = α'(s) + λ'(s) α'(s) + λ(s) k_α(s) \mv{n}_α(s) \\
 0 = \mv{t}_α(s) + λ'(s) \mv{t}_α(s) + λ(s)k_α(s)\mv{n}_α(s) \\
 0 = (1+λ'(s))\mv{t}_α(s) + λ(s)k_α(s)\mv{n}_α(s)
 \end{gather*}
 
Dado que $\{\mv{t}_α,\mv{n}_α\}$ es una base ortonormal del plano, una combinación lineal de ambos dos sólo puede ser cero si los dos coeficientes son cero. Esto nos lleva a las dos siguientes ecuaciones

\[ \left.\begin{matrix}
1+λ'(s) = 0 \\
λ(s)k_α(s) = 0 \\
\end{matrix}\right\} ∀s∈I \]

Como $λ'(s) = -1$, $λ(s) = s_0 - s$ para algún $s_0$. Sustituimos en la segunda ecuación y nos queda que $(s_0-s)k_α(s) = 0$, así que si $s≠s_0$ $k_α(s)=0$. Y como además, $k_α$ es continua, en ese punto $s_0$ también es cero. La curvatura es cero en todo punto $s$, y por lo tanto es un segmento de recta.

\spart Visualmente, vemos que $p_0$ deberá ser el centro. La implicación a la derecha se demuestra fácilmente. Sabemos la parametrización de $α$

\[ α(s) = p_0 + r\left(\cos \frac{s}{r}, \sin \frac{s}{r}\right) \]

Calculamos la recta normal por $α(s)$ y veo que $p_0$ está en ella.

Calculamos ahora la \textbf{implicación a la izquierda}. Queremos demostrar que la curvatura es constante y distinto de 0. Sabemos que 

\[ p_0 = α(s) + λ(s)\mv{n}_α(s) \]

para algún $λ(s)$. Como podemos expresar $λ(s) = \pesc{p_0-α(s),\mv{n}_α(s)}$, es derivable. Así que derivamos.

\begin{gather*}
0 = \mv{t}_α(s) + λ'(s)\mv{n}_α(s) + λ(s)\left(-k_α(s) \mv{t}_α(s)\right) \\
0 = (1-k(s)λ(s))\mv{t}_α(s) + λ'(s)\mv{n}_α(s) 
\end{gather*}

Al igual que veíamos en el anterior apartado, sabiendo que tenemos una base normal sacamos dos ecuaciones:

\begin{gather*}
1-k(s)λ(s) = 0 \\
λ'(s)=0 
\end{gather*}

λ es constante, así que $λ(s) = c_0$ para algún $c_0∈ℝ$. Entonces

\[ 1-k_α(s)c_0 = 0 \implies k_α(s) = \frac{1}{c_0} \]

La curvatura escalar es constante, así que efectivamente trabajamos en un arco de circunferencia.

\end{problem}

\begin{problem}[9] Sea $\appl{α}{I}{ℝ^2}$ PPA. Demuestra que todas las rectas normales equidistan de un $p_0$ dado si y sólo si existen $a,b∈ℝ$ tal que \[ k_α(s) = \pm \frac{1}{\sqrt{as + b}} \]

\solution

Hagamos un dibujito

\begin{figure}[hbtp]
\centering
\begin{tikzpicture}[pnt/.style={draw,shape=circle,fill=white, inner sep=2pt}]
\draw[-] (0,0) -- (6,0);
\node[pnt,label=below:{$α(s)$}] (S) at (2,0) {};
\node[label=below:{$\cn(s)$}] (N) at (3.5,0) {};
\node[pnt] (PA) at (4,0) {};
\draw[thick,->] (S) -- (N);
\draw[->] (S) -- node[left] {$-\mv{t}_α(s)$} (2,2);
\node[pnt,label=right:{$p_0$}] (P) at (4,3) {};
\draw[->] (S) -- node[midway, above, sloped] {$p_0-α(s)$} (P);
\draw[dashed] (P) -- node[right] {$\pesc{p_0-α(s), -\mv{t}_α(s)}$} (PA);
\end{tikzpicture}
\end{figure}

Derivamos la distancia $\pesc{p_0-α(s), -\mv{t}_α(s)}$

\[ 0 = \pesc{-\mv{t}_α(s),-\mv{t}_α(s)} + \pesc{p_0-α(s),-k_α(s) \mv{n}_α(s)} \]

y por lo tanto $k_α(s)\pesc{p_60-α(s),\mv{n}_α(s)} = 1$. Volvemos a derivar (total, es gratis)

\[ \cv(s)\pesc{p_0-α(s)\cn(s)} + \cv(s)\left(\pesc{\ct(s), \cn(s)} + \pesc{p_0-α(s), -\cv(s)\ct(s)}\right) = 0 \]

y dividimos:

\[ \frac{\cv'(s)}{\cv(s)} - c\cv^2(s) = 0 \]

Resolvemos la ecuación diferencial y nos queda que

\[ \frac{-1}{2k^2}=cs + d \]

Por lo tanto

\[ k^2 = \frac{1}{(-2c)s + (-2d)} \]

de tal forma que hemos llegado a la fórmula que nos daban al principio.

Ahora vamos a por la \textbf{otra implicación}. Pero no, que es muy larga.
\end{problem}

\begin{problem}[11] Sea $Γ$ una curva cerrada, simple, en el plano, contenida en el interior de la circunferencia $\{ x^2+y^2=r^2 \}$. Demuestra que existe un punto $p∈Γ$ tal que $\md{k(p)} ≥ \frac{1}{r}$.

\solution

Partimos de que $\frac{1}{r}$ es la curvatura del círculo. Trasladamos la circunferencia en cualquier dirección hasta tocar la curva por primera vez. En ese punto de contacto, la tangente es la misma y por lo tanto la normal también.

Si trasladásemos y rotamos las curvas para que el plano sea el eje $OX$ y el punto de contacto fuese el origen, tendríamos algo parecido a la imagen \ref{imgCirc}.

\easyimg{img/Hoja2_11_CircCurva.png}{Curva y circunferencia}{imgCirc}

Si podemos escribir la curva $α=α(s)$ como grafo de una función $h_1(x)$, entonces

\[ k(α(0)) = \frac{h_1''(x)}{\left(1+h_1'(x)^2\right)^{\frac{3}{2}}} \]

Dado que tenemos un grafo sobre la recta tangente, $h_1'(0) = 0$ y entonces $k_Γ(0)=h_1''(0)$. Por otra parte, sabemos que $k_C(0) = h_2''(0) = \frac{1}{4}$. Además, $h_1(x) ≥ h_2(x)$ en un intervalo $(-δ,δ)$ y las funciones y sus derivadas valen lo mismo en $0$. Por lo tanto, tiene que cumplirse

\[ h_1''(0) ≥ h_2''(0) \implies k_Γ(0) ≥ \frac{1}{r} \]

\end{problem}

