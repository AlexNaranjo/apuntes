\subsection{Hoja 7}

\begin{problem}

Sea $\appl{L}{ℝ^2×ℝ^2}{ℝ}$.

\[ L = ((x_1, x_2), (y_1, y_2)) = \frac{1}{2}\left(y_1^2 + e^{x_1}y_2^2\right) \]

\solution

Calculamos el \textbf{funcional integral asociado a $L$}:

\[ L(α) = \int_a^b L(α,α')\dif t \] donde \[ \appl{α}{[a,b]}{ℝ^2};\quad α(t) = (x(t), y(t)) \]. Entonces

\[ F(α) = \int_a^b \frac{1}{2}\left(x'^2 +e^x y'^2\right) \dif t \]

Hallamos ahora la \textbf{primera variación del funcional}. Lo que hacemos es coger varias curvas próximas a α que dependen de un parámetro λ, por ejemplo. La primera variación del funcional nos dirá cómo varía la curva en función de ese segundo parámetro. 

Definimos $α(t, λ)$ como la variación de la curva α y

\[ \tilde{F}(λ) = F(α(t,λ)) = \int_{a}^{b} \frac{1}{2}\left(x'(t,λ)^2 + e^{x(t,λ)} y'(t,λ)^2\right) \dif t \].

La primera variación no será más que $\tilde{F}'(0)$. Derivamos la integral con respecto a λ:

\[ \tilde{F}(λ) = \int_a^b \frac{1}{2}\left(2x'(t,λ) \pd{x'}{λ} (t,λ)  +e^{x(t, λ)} \pd{x}{λ}y'^2 + e^{x(t,λ)} 2y'\pd{y'}{λ}\right) \dif t \]

\begin{wrapfigure}{r}{0.4\textwidth}

\begin{tikzpicture}[font=\small]
\foreach \l/\pos in {-1/below,0/above,1/above}
{
	\draw[green!80!blue, domain=-pi-0.5:pi+0.5, variable=\x, smooth, samples=200] plot ({\x}, {-0.02*\x*\x + \l * (0.2*sin(4*\x r) + 0.1*(\x-pi)*(\x+pi))});
	\node[\pos] at (-1, {\l * 1.1}) {$α(t, \l)$};
}

\draw[blue,domain=-1.2:1.2, variable=\x, smooth] plot ({0.4 + 0.2*\x*\x}, {\x}) node[above] {$α(1, λ)$};

\node[draw, circle, fill, black, inner sep=1pt, label=above:{$V(a) = 0$}] at (-pi, -0.02*pi*pi) {};
\node[draw, circle, fill, black, inner sep=1pt, label=above:{$V(b) = 0$}] at (pi, -0.02*pi*pi) {};
\end{tikzpicture} 
\caption{Variación de las curvas y puntos de inicio y final}
\end{wrapfigure}

Hasta aquí es sólo cálculo de primero. Llamamos ahora $V(t,λ) = \pd{α}{λ}(t,λ)$ al vector que indica hacia dónde varían las curvas. Dicho de otra forma, \[ V(t) = (V_1(t), V_2(t)) = \left(\pd{x}{λ}(t,0), \pd{y}{λ}(t,0)\right) \] y ahora sustituimos:

\[ \tilde{F}'(0) = \int_a^b \frac{1}{2}\left(2x'(t) V_1'(t) + e^{x(t)}V_1 y'^2 + e^x 2y'V_2' \right) \] usando el hecho de que todo sea $C^\infty$ y podamos intercambiar el orden de las derivadas:

\[ \pd{x'}{λ} = \pd{}{λ}\pd{x}{t} = \pd{}{t}\pd{x}{λ} = V_1' \]

Aplicamos ahora integración por partes para dejarlo todo en términos de $V_1$ y $V_2$, y no de sus derivadas. 

\begin{gather*} \int_a^b \underbrace{x'}_u \underbrace{V_1'\dif t}_{\dif v} = \eval{x'V_1}_a^b - \int_a^b V_1 x'' \dif t \\
\int_a^b \underbrace{e^xy'}_u \underbrace{V_2'\dif t}_{\dif v} = \eval{e^xy'V_2}_a^b - \int_a^b V_2(e^xx'y' + e^xy'')\dif t 
\end{gather*}

Finalmente, nos queda

\begin{align*} \tilde{F}'(0) &= \frac{1}{2} \left[ \eval{2x'V_1 + e^x2y'V_2}_a^b + \int_a^b -V_1 2 x'' + e^xV_1y'^2-V_2(ex'2y' + e^x2y'') \dif t \right] = \\
&= \frac{1}{2}\left[\eval{2x'V_1+e^x2y'V_2}_a^b + \int_a^b V_1(e^xy'^2-2x'') + V_2(-e^x2y'-e^x2y'') \dif t \right]
\end{align*}

\paragraph{Ecuaciones de Euler-Lagrange de este Lagrangiano} La curva va a ser un punto crítico del funcional integral. Fijamos los extremos: si todas las curvas salen del mismo punto $a$ y llegan al mismo punto $b$, entonces $V(a) = 0$ y $V(b) = 0$. Es decir, la primera parte de $\tilde{F}'$ desaparece y resolvemos dos ecuaciones:

\[ \begin{cases} e^xy'^2 - 2x'' &= 0 \\
-e^xx'2y' - e^x2y'' &= 0\end{cases} \begin{cases}  x'' &= \frac{1}{2}e^xy'^2 \\ 
y'' &= -x'y' \end{cases}\]

de tal forma que $\tilde{F}'(0) = 0$ para toda forma posible de variar α. 

Para resolverlo, sabemos que si $(x(t), y(t))$ es solución del sistema existen constantes $c_1, c_2$ tales que \[ \begin{cases} x'^2 + e^x y'^2&=c_1 \\
e^xy'&= c_2 \end{cases} \]

Vamos a ello:

\begin{gather*}
(e^xy')' = e^xx'y' + e^xy'' = e^x(x'y'+y'') = e^x(x'y' -x'y') = 0 \\
(x'^2 + e^xy'^2)' = 
\end{gather*}


\end{problem}

\begin{problem}[4] Sea $S$ una superficie, y $α(t)$ una geodésica birregular plana en $S$. 

\ppart Demostrar que el plano que contiene a α es perpendicular a $S$.
\ppart Demuestra que α es línea de curvatura de $S$.
\ppart Geodésicas planas en una superficie de revolución.
\solution

\spart $P$ es normal a $S$ si es normal a $T_αS$, es decir, si sus normales son ortogonales. Por lo tanto, $N_α⊆P$. Tenemos que ver entonces que $\cb (t) \perp N_α$, donde $N$ es la normal a la superficie $S$. 

Sabemos que α es geodésica, por lo tanto está parametrizada por arco. Al ser una geodésica, va con velocidad constante por la superficie. Cualquier aceleración que tenga deberá ser ortogonal a la superficie y por lo tanto $α''(t)$ está en la dirección de la normal: $α'' = λ(t) N_α(t)$. Operamos, y como α era birregular.

\[ α''(t) = \cv(t) \cn(t) = λ(t)N_α \]

\spart Hay que ver que $(N○α)'(t) = λ(t) α'(t)$. Es decir, que su tangente sea siempre una curvatura principal. Operamos sabiendo que $\cn(t) = \pm N_α(t)$:

\[ (N○α)'(t) = \cn' = -\cv \ct + \underbrace{\ctr}_0 \cb = -\cv α'(t) \]

\spart Si $α$ es birregular y geodésica plana en $S$, es una línea de curvatura según el apartado anterior. Al ser una superficie de revolución, son los paralelos o meridianos\footnote{También llamada curva generatriz}. Ahora bien, ¿son todos geodésicas?

Sólo son los meridianos (acordémonos de la esfera: sólo los meridianos son círculos máximos) son geodésicas siempre. Un paralelo será geodésica su la generatriz en ese punto es perpendicular a $N$. Necesito que el vector tangente a la generatriz sea normal a la normal a la superficie. El vector tangente a la generatriz tiene que ser paralelo al eje de giro.

\end{problem}

\begin{problem}[15] \textbf{Paraguas de Whitney}. Tenemos una superficie $S$ dada por la parametrización

\[ Φ(u,v) = (u,uv,v^2) \]

\ppart Comprobar que $Φ$ es regular en todo punto salvo $(0,0)$.
\ppart Hallar $II_Φ$ y comprobar que ciertos puntos son hiperbólicos.
\ppart Comprueba que las curvas $Φ(u,v_0)$ con $v_0$ constante son rectas y además curvas asintóticas.
\ppart Calcula las curvas asintóticas de $S$.

\solution 

\spart Calculamos la diferencial

\[ \Dif Φ(u,v) = \begin{pmatrix}
1 & 0 \\
v & u \\
0 & 2v \\ 
\end{pmatrix} \] y vemos que, efectivamente, tiene rango 2 salvo en $(u,v) = (0,0)$.

\spart Vamos a por la 2FF.

\begin{align*}
Φ_u &= ( 1,v,0) \\
Φ_v &= (0,u,2v) \\
Φ_{uu} &= (0,0,0) \\
Φ_{uv} &= (0,1,0) \\
Φ_{vv} &= (0,0,2) 
\end{align*}

Calculamos ahora los coeficientes. El primero es $0$ ya que $Φ_{uu} = \vec{0}$. Para el segundo necesitaríamos calcular el vector normal, pero podemos retrasarlo llamando $\norm{Φ_u×Φ_v}= l$ y calculando el determinante que nos da $\pesc{Φ_u×Φ_v, Φ_{uv}}$:

\[ lf = \left|\begin{matrix}
1 & 0 & 0 \\
v & u & 1 \\
0 & 2v & 0
\end{matrix}\right| = -2v \]. De la misma forma obtenemos $lg = 2u$.

Calculamos ahora sí $l$:

\[ l = \sqrt{EG - F^2} = \sqrt{u^2 + 4v^2 + 4v^4} 
\]  y la 2FF es 

\[ II_Φ = \frac{1}{ \sqrt{u^2 + 4v^2 + 4v^4} } \left(-4v\dif u \dif v + 2u \dif v^2 \right) \]

Buscamos ahora los puntos hiperbólicos a través de la curvatura gaussiana $K$, viendo dónde ésta sea negativa. 

\[ K = \frac{eg-f^2}{EG-F^2} = \frac{0-(-4v)^2}{l^4} = \frac{1}{l^4} (-16v^2) ≤ 0 \]

Es decir, que todos los puntos con $v≠0$ son hiperbólicos.

\spart Las curvas de la forma $Φ(u,v_0)$ se pueden expresar como

\[ Φ(u,v_0) = (u,uv_0, v_0^2) = u(1,v_0,0) + (0,0,v_0^2) \]
lo que nos dice que efectivamente esas curvas son rectas. Además, son curvas asintóticas. 

Recordamos que las curvas asintóticas cumplen que $II_α(t)(α'(t))=\pesc{α''(t), N}$. Dado que $α$ son rectas en este caso, $α''(t) = 0$ y por lo tanto la segunda forma fundamental se anula y son curvas asintóticas.

Vamos a ver mejor de dónde viene esa expresión de la 2FF. Sabemos que \[ II(α'(t) = \pesc{\dif N(α'(t)),α'(t)} \]. Podemos expresar $N(α'(t))$ como $(N○α)'(t)$. Ahora bien, si recordamos la fórmula de la derivada del producto escalar podemos reexpresar

\[ \underbrace{\pesc{(N○α)'(t),α'(t)}'}_{0} - \pesc{N○α(t),α''(t)} = \pesc{N○α(t), α''(t)} \]

donde la primera parte se anula al ser un producto escalar de un vector tangente a la superficie y otro normal a ella (son perpendiculares).

\spart

Vamos a hallar todas las curvas $α(t) = Φ(u(t), v(t))$ asintóticas, es decir, que cumplan $II_α(t)(α'(t)) = 0$. Podemos ahorrarnos la raíz fea $l$ y buscamos que

\[ lII_{α(t)} (α'(t)) = 0 \iff -4vu'v' +2uv'^2 = 0 \iff 2v'(-2u'v + uv') = 0 \]

La ecuación se anula si $v'\equiv 0$, lo que nos da las rectas que ya habíamos hallado. La segunda posibilidad es que \[ 2u'v = uv' \], una ecuación diferencial de variables separadas que podemos integrar:

\begin{gather*}
 \int \frac{2u'}{u} = \int \frac{v'}{v} \\
 \log u^2 - \log \abs{v} = C \\
 \log \frac{u^2}{\abs{v}} = C \\
 u^2 = \bar{C} \abs{v}
 \end{gather*} 
 
Tenemos que comprobar que no hay más rectas contenidas en el paraguas de Whitney. Si hubiera otra recta β que no hemos considerado, sería asintótica, y tendría la forma

\[ β(t) = Φ(u, \frac{u^2}{\bar{C}}) = \left(u,u\frac{u^2}{\bar{C}}, \frac{u^4}{\bar{C}^2}\right) 
\]
que no es una recta ni de coña. La demostración rigurosa podríamos hacerla o viendo que la curvatura no es constante o bien viendo que su proyección sobre los planos coordenados no es una recta.

\end{problem}

\begin{problem}[4]
Sea $Φ(u,v)$ una parametrización de una superficie con métrica de Riemann \[ Q \equiv e^v\dif u^2 + (1+u^2)\dif v^2 \].  

\ppart Sea $α(s) = Φ(u(s), v(s))$. Halla EL para el funcional de energía.
\ppart Coge la curva $v = 0$, parametrizamos por arco. Hallar el vector de curvatura geodésica a lo largo de $α_0$ dado de forma

\[ \cv[g,α_0,Q] = a_1(s) Φ_u + a_2(s) Φ_v \]
\solution

\spart Hallamos el funcional de energía \[ E(α) = \int_a^b Q_{α(t)} (α'(t)) \dif t = \int_a^b e^v u'^2 + (1+u)v'^2 \dif t \]. El de longitud es igual pero poniendo $\sqrt{Q}$ en lugar de $Q$.

Hallamos el nosequécojonesseráeso

\[ \tilde{E}'(0) = \int_a^b e^v V_2 u'^2 e^v 2u'V_1'+2uV_1v'^2 + (1+u^2)2v'V_2' \dif t \]

Cuando uso partes y supongo que en los extremos $V_1=V_2=0$ queda

\[ \tilde{F}'(0) = \int_a^b V_1(2uv'^2-e^vv'2u'^2-e^v4u'u'') + V_2(e^vu'^2-4uu'v'-2(1+u')v'') \dif t \]

\spart $α_0(s) = Φ(s,0)$. En la métrica que nos dan, 

\[\md{α_0'(s)}^2 =  Q(α_0'(s)) = e^v(s) · 1^2 + (1+s^2) ·0^2 = e^0 = 1 \], vemos que en la métrica que nos dan está parametrizada por arco.

Para hallar los valores, tenemos una formulita:

\[ \begin{pmatrix}
a_1(s) \\ a_2(s) 
\end{pmatrix} = \inv{[Q]} \left(\deriv{}{s}\left([Q]\begin{pmatrix}
u'(s) \\ v'(s)
\end{pmatrix}\right) - \frac{1}{2} \left[\begin{matrix}
Q_u(\ct[α_0]) \\ Q_v(\ct[α_0]) \end{matrix}\right]\right) \] donde

\[ [Q] = \begin{pmatrix}
e^v & 0 \\ 0 & 1+u^2
\end{pmatrix};\quad \begin{matrix}
Q_u &= 0 \dif u ^2 + 2u \dif v^2 \\
Q_v &= e^v\dif u^2 + 0\dif v^2
\end{matrix} \]

Seguimos calculando: \begin{gather*}
Q_v(\ct[α_0]) = e^0 = 1 \\
Q_u(\ct[α_0]) = 0
\end{gather*}

\[ \inv{[Q]} = \begin{pmatrix}
e^{-v} & 0 \\ 0 & \frac{1}{1+u^2}
\end{pmatrix} \], que evaluado en $α_0$ es \[ \inv{[Q]} = \begin{pmatrix} 1 & 0 \\ 0 & \frac{1}{1+s^2} \end{pmatrix} \]

Seguimos con más operaciones

\begin{gather*} [Q]\begin{pmatrix}
u' \\ v
\end{pmatrix} = \begin{pmatrix}u' \\ (1+s^2)v' \end{pmatrix} \\
\deriv{}{s} [Q] \begin{pmatrix}
u' \\ v
\end{pmatrix}  = \begin{pmatrix}
u'' \\ (1+s^2)v'' + 2sv'
\end{pmatrix} \end{gather*}

Juntándolo todo \footnote{Matadme.}

\[
\begin{pmatrix}
a_1(s) \\ a_2(s)
\end{pmatrix} = \begin{pmatrix}
u'' \\ v'' +\frac{2s}{1+s^2}v'
\end{pmatrix} - \frac{1}{2}\begin{pmatrix}
0 \\ \frac{1}{1+s^2}
\end{pmatrix} = \frac{1}{2} \begin{pmatrix} 0 \\ 1+s^2 \end{pmatrix} \] y finalmente

\[ \cv[g,α_0, Q] = \frac{1}{2}(1+s^2) Φ_v \]


\end{problem}

\begin{problem}[9] Tenemos una superficie $S$ y una parametrización $ψ(u,v)$, con métrica \[ Q = u\dif u^2 + u\dif v^2 \] con $u>0$. 

\ppart Hallar la ecuación de la parametrización geodésica.

\solution

\spart Si $ψ(u(t), v(t))$ es geodésica, entonces

\[ \deriv{}{t}\left([Q] \begin{pmatrix} u' \\ v' \end{pmatrix}\right) - \frac{1}{2} \begin{pmatrix} Q_u(α') \\ Q_v(α') \end{pmatrix} = \begin{pmatrix} 0 \\ 0 \end{pmatrix} \]

Me rindo.
\end{problem}

