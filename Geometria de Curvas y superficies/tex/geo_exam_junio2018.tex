\subsection{Junio 2018 (11-VI-2018)}

\begin{problem}
	
Sea $\alpha(t)$ una curva en $\mathbb{R}^3$ con rapidez unitaria, triedro de Frenet-Serret $\left\{\textbf{t}(t), \textbf{n}(t), \textbf{b}(t) \right\}$, curvatura $\kappa(t) > 0$ y torsión $\tau(t) \neq 0$ para todo $t$. Suponemos que la curvatura y la torsión de $\alpha(t)$ verifcican la ecuación


\begin{equation*}
\dfrac{\tau(t)}{\kappa(t)} + \left(\dfrac{1}{\tau(t)} \left(\dfrac{1}{\kappa(t)}\right)'\right)' = 0
\end{equation*}

\textbf{Demuestra:}

\ppart $c(t) = \alpha(t) + \dfrac{1}{\kappa(t)}\textbf{n}(t) - \dfrac{\kappa'(t)}{\tau(t)(\kappa(t))^2}\textbf{b}(t) \quad$ es constante.
	
\ppart $\alpha(t)$ pertenece a una esfera. Determina el centro y el radio de la esfera. (Demuestra que no dependen de $t$)


\solution

\end{problem}







\begin{problem}
	
Considera el helicoide $S$ parametrizada por $\textbf{X}(u,v) = (u\cos v, u\sin v, v)$

\ppart Calcula la primera y segunda formas fundamentales.
	
\ppart Determina el coseno del ángulo entre las curvas
\begin{equation*}
\alpha(t) = \textbf{X}\left(t, t+1\right) \quad \text{y} \quad \beta(x) = \textbf{X}\left(\dfrac{x^2}{2}-1, x\right)
\end{equation*}

en el punto de intersección \textbf{X}(1,2).

\ppart Determina la curvatura normal $k_n(t)$ de $\alpha(t) = \textbf{X} \left(t, t\right).$

\ppart Encuentra las curvaturas principales y las direcciones principales de curvatura de la superficie $S$.

\ppart Determina y resuelve la ecuación diferencial que debe cumplir $\gamma(t) = \textbf{X}(u(t), v(t))$ para ser (i) línea de curvatura. (ii) línea asintótica.

\textit{Recuerda:} $(arc\sinh x)' = \dfrac{1}{\sqrt{1+x^2}}$

\solution

\end{problem}






\begin{problem}

Considera el dominio $\Omega = \left\{(u,v) \in \mathbb{R}^2 : 0<u<1\right\}$ con la métrica
\begin{equation*}
Q = \left(\dfrac{1}{u^2} - 1\right)\left((du)^2 + (dv)^2\right)
\end{equation*}

\ppart Encuentra las geodésicas. (Exprésalas implícitamente, con $v$ en función de $u$).

\textit{Sugerencia:} Utiliza el método de conservación del momento.

\ppart Calcula la curvatura geodésica \textit{(escalar)} de la curva $\alpha(t) = (\frac{1}{2}, t^2)$ para $t>0$.

\solution
	
\end{problem}





\begin{problem}
Considera el semiplano superior $\mathbb{H} = \left\{(u,v) \in \mathbb{R}^2 : v > 0\right\}$ con la métrica $Q = \dfrac{(du)^2 + (dv)^2}{v^2}$.

\ppart Calcula la curvatura gaussiana de $\mathbb{H}$ usando formas diferenciales.

\ppart Sea $h:\mathbb{H} \rightarrow \mathbb{H}$ la transformación definida por $h(u,v)=(\lambda(u, v), \nu(u, v))$ con
\begin{equation*}
\lambda(u,v) = \dfrac{u}{u^2+v^2} \quad \text{y} \quad \nu(u, v) = \dfrac{v}{u^2+v^2}
\end{equation*}

Estudia si $h$ es o no una isometría. \textit{Sugerencia:} Demuestra que $\lambda_u = -\nu_v$ y $\lambda_v = \nu_u$.

\solution

\end{problem}
