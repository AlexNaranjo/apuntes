\documentclass[palatino]{apuntes}

\title{Modelizacion2016}
\author{Pablo Pérez Manso}
\date{15/16 C2}

% Paquetes adicionales

% --------------------

\begin{document}
\pagestyle{plain}
\maketitle

\tableofcontents
\newpage
% Contenido.

\chapter{Sistemas dinámicos}

\chapter{Cadenas de Markov}

	\textbf{Ejemplo:} Vamos a lanar una moneda equilibrada. Hago el experimento muchísimas veces. ¿Cuál es el número medio de tiradas que tengo que hacer para que salga CXC por 1º vez? ¿y CCC?

$$N_{CXC} < N_{CCC}$$

Gráfico (1 26-1-2016)

Si yo llamo $\tau_{CXC}$ al número de la 1º tirada en la que se obtiene CXC (variable aleatoria)

$$\tau_{CXC}(CXXCXCCC...)=6$$
$$\tau_{CXC}(CXCCC...)=3$$

Por lo que ahora podemos ver que 

$$N_{CXC} = E[\tau_{CXC}] (esperanza)$$

Para ello, tenemos que buscar que:

$$P(\tau_{CXC} < \tau_{CCC})$$

\chapter{Cálculo ..........}

%% Apéndices (ejercicios, exámenes)
\appendix

\chapter{---}
% -*- root: ../Modelizacion2016.tex -*-

\printindex
\end{document}
