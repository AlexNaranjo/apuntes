% -*- root: ../VariableReal.tex -*-
\section{Hoja 1 - Teoría de la integral y la medida}

\begin{problem} Sea \meas un espacio de medida.

\ppart Demostrar que $\appl{f}{Ω ⊆ X}{ℝ}$ es medible si y sólo si el conjunto $\set{x ∈ Ω \tq f(x) > r}$ es medible $∀r ∈ ℝ$.
\ppart Demostrar que si la sucesión de funciones $\appl{f_n}{Ω}{ℝ}$ es medible para todo $n$ también lo son las funciones $\sup f_n,\ \inf f_n,\ \limsup f_n$ y $\liminf f_n$.
\ppart Demostrar que si $\appl{f_n}{Ω}{ℝ}$ son medibles y $f_n(x) \convs f(x)$ puntualmente, entonces $f$ es medible.
\solution

\spart

Si $f$ es medible, entonces $∀E ⊆ ℝ$ medible se tiene que $\inv{f}(E)$ es medible, y en particular esto valdrá para los conjuntos de la forma $(r, +∞)$.

Para demostrar la implicación al otro lado, vemos ciertas propiedades de la inversa: sabemos que $μ(\inv{f}(E^c)) = μ(X) - μ(\inv{f}(E))$, que $\inv{f}(E ∩ F) = μ(E)∩μ(F)$ y que $\inv{f}(E∪F) = \inv{f}(E) ∪ \inv{f}(F)$. Así, podemos construir cualquier subconjunto de $ℝ$ a partir de intervalos de la forma $(r, +∞)$, que sabemos que son medibles. Esas operaciones son compatibles con la inversa y entonces podremos descomponerlo todo en conjuntos medibles\footnote{Esto merece algo más de explicación pero bueno.}.

\spart\label{ej:H1:SupremosMedibles} \textit{Nota: esto es una demostración extendida de la \fref{prop:SupremoInfimoMedibles}.}

Para demostrarlo vamos a usar el apartado anterior.

En el caso del supremo, sabemos que $\appl{\sup f_n}{Ω}{ℝ}$ será medible si y sólo si el conjunto \[ \set{x∈Ω \tq \sup_{n∈ℕ} f_n (x) > r } = \bigcup_{n ∈ ℕ} \set{x ∈ Ω \tq f_n(x) > r} \] es medible para todo $r ∈ ℝ$, que lo es porque se puede descomponer como unión de conjuntos medibles. En el caso del ínfimo lo tenemos análogamente pero haciendo la intersección de esos conjuntos: \[ \set{x ∈ Ω \tq \inf_{n∈ℕ} f_n(x) > r} = \bigcap_{n∈ℕ} \set{x∈Ω \tq f_n(x) > r } \]

\spart

Si $f_n \convs f$ puntualmente, eso significa que $\limsup f_n = \liminf f_n = f$, luego por lo que hemos visto en el apartado anterior es efectivamente medible.


\end{problem}

\begin{problem} Probar que si $\appl{f}{ℝ}{ℂ}$ es continua y $f(x) = 0$ en casi todo punto con respecto a la medida de Lebesgue, entonces $f(x) = 0\; ∀x ∈ ℝ$.

\solution

Si $f$ es nula en casi todo punto con respecto a la medida de Lebegue, entonces sólo podemos tener puntos aislados que difieran de 0. Sin embargo, si esos puntos existen no se cumpliría la definición de continuidad.

Más formalmente y usando la definición topológica de continuidad\footnote{No me apetece pensar continuidad en los complejos.} supongamos que existe $a ∈ ℝ$ tal que $c = f(a) ≠ 0$. Dado que $f$ es nula en casi todo punto, $c$ tiene que ser punto aislado y por lo tanto existe un $r ∈ ℝ$ tal que $0 ∉ \bola_r(c) ⊆ ℂ$, es decir, existe un entorno alrededor de $c$ que no contiene a $0$. Sin embargo, si cogemos un entorno $E = (a - ε, a + ε)$, tenemos que $f(E)$ es nulo en casi todo punto, por lo que $f(E) \nsubseteq \bola_r(c)$ y por lo tanto $f$ no puede ser continua. Contradicción, así que $f(x) = 0\; ∀x∈ℝ$.

\end{problem}

\begin{problem} \label{ej:H1:ConvMonotonaSeries} Sea $\set{f_n}$ una sucesión de funciones medibles con $f_n ≥ 0$. Usar el \nlref{thm:ConvMonotona} para demostrar que \[ \int_Ω\left(\sum_{n=1}^∞ f_n\right) \dif μ = \sum_{n=1}^∞ \int_Ω f_n \dif μ \]
\solution

Lo cierto es que esto es el \nlref{thm:ConvMonotonaSeries}, así que no hay que complicarse mucho la vida. Simplemente hay que definir la sucesión \[ S_N = \sum_{n=1}^N f_n \] que es monótona creciente por ser $f_n ≥ 0$, luego podemos usar el \fref{thm:ConvMonotona}. Lo que tenemos es por lo tanto que  \begin{align*} \int_Ω\left(\sum_{n=1}^∞ f_n\right) \dif μ &= \int_Ω \lim_{N\to \infty} S_N \dif μ \eqexpl{\ref{thm:ConvMonotona}} \lim_{N\to ∞} \int_Ω S_N \dif μ = \\ &= \lim_{N\to ∞} \int_Ω \sum_{n=1}^N f_n \dif μ \eqreasonup{Suma finita} \lim_{N\to ∞} \sum_{n=1}^N \int_Ω f_n \dif μ = \sum_{n=1}^∞ \int_Ω f_n \dif μ \end{align*}

\end{problem}

\begin{problem}[5] Probar la siguiente generalización de la \nref{prop:DesigualdadMarkov}. Sea $\appl{g}{ℝ}{[0, +∞]}$ una función medible y creciente en la imagen de $\appl{f}{\meas}{ℝ}$. Probar que \( μ\left(\set{x ∈ X \tq f(x) > t}\right) ≤ \frac{1}{g(t)} \int_X g(f(x)) \dif μ(x) \label{eq:H1E5} \)

Como consecuencia, probar que si $f_n \convs f$ en $\espL$, entonces $f_n \convs f$ en medida.
\solution

La desigualdad de Markov nos dice que, dada una función medible $f$, entonces \[ μ\left(\set{x∈X \tq \abs{f(x)} > ε }\right) ≤ \frac{1}{ε} \int_X \abs{f} \dif μ \]

Sea $E = \set{x ∈ X \tq f(x) > t}$. Podemos usar esto para descomponer la integral: \[ \frac{1}{g(t)} \int_X g(f(x)) \dif μ = \frac{1}{g(t)} \int_{X \setminus E} g(f(x)) \dif μ + \frac{1}{g(t)} \int_E g(f(x)) \dif μ\]

Como $g$ es creciente en la imagen de $f$, tenemos que $∀a ∈ f(E),\, g(a) > g(t)$, así que podemos simplificar en la segunda integral y nos queda que \begin{multline*}\frac{1}{g(t)} \int_{X \setminus E} g(f(x)) \dif μ + \frac{1}{g(t)} \int_E g(f(x)) \dif μ ≥ \frac{1}{g(t)} \int_{X \setminus E} g(f(x)) \dif μ + \frac{1}{g(t)} \int_E g(t) \dif μ = \\ = \int_{X \setminus E} g(f(x)) \dif μ + μ(E) ≥ μ(E) \end{multline*} y queda probada la desigualdad.

Ahora tenemos que usar esto para demostrar que si $f_n \convs f$ en $\espL$, entonces $f_n \convs f$ en medida o, en otras palabras, que
\[ \int_X \abs{f_n - f} \dif μ \to 0 \implies ∀ε > 0 \quad μ\left(\set{x∈X \tq \abs{f_n(x) - f(x)} > ε}\right) \to 0 \]

Sea $h_n(x) = \abs{f_n - f}$ y $g(t) = εt$. $g$ es medible y creciente en la imagen de $h_n$, así que podemos aplicar \eqref{eq:H1E5}, y como $\int_X h_n \dif μ \to 0$ entonces tendremos que $μ(h_n > ε) \to 0$ igualmente.
\end{problem}

\section{Hoja 2 - Medidas con signo y Teorema de Lebesgue-Radon-Nikodim}

\begin{problem} Sea μ una medida con signo sobre el espacio medible \meds. Definimos \index{Variación!total de una medida}la variación total de una medida como $\abs{μ} = μ_+ + μ_-$, siendo $μ_+, μ_-$ las partes positiva y negativa de una medida respectivamente, dadas por la descomposición de Jordan. Decimos que μ es finita si y sólo si $\abs{μ}$ lo es.

\ppart Probar la siguiente caracterización de la variación total\footnote{En los ejercicios aparece como partición de $X$ pero no le veo mucho sentido a eso.}: \[ \abs{μ} (E) = \sup \set{ \sum_{n=1}^∞ \abs{μ(E_i)} \tq \set{E_n} \text{ es una partición de } E} \]

Demostrar además que \begin{align*} μ_+(E) &= \sup \set{μ(F) \tq F ⊆ E, F ∈ \algb{X}}  \\  μ_-(E) &= - \inf \set{μ(F) \tq F ⊆ E, F ∈ \algb{X}} \end{align*}

Verificar que $μ = μ_+ - μ_-$ y que $\abs{μ} = μ_+ + μ_-$.

\ppart Probar que $\abs{μ}$ es la medida minimal tal que $-\abs{μ} ≤ μ ≤ \abs{μ}$.
\ppart Probar que $E ∈ \algb{X}$ es μ-nulo si y sólo si es $\abs{μ}$-nulo.
\ppart Probar que $\abs{μ_1 + μ_2} ≤ \abs{μ_1} + \abs{μ_2}$.
\ppart Probar que μ es finita si y sólo si $μ_+, μ_-$ lo son.
\ppart Sean $μ_1, μ_2$ medidas con signo. Probar que \[ μ_1 \perp μ_2 \iff μ_1 \perp \abs{μ_2} \iff μ_1\perp μ_{2,+} \iff μ_1 \perp μ_{2,-} \]
\solution

\spart Según el \nref{thm:DescompHahn} y el \nref{thm:DescompJordan}, siempre tendremos una partición de $X = X_+ ∪ X_-$ con medidas asociadas $μ_+, μ_-$ singulares. Al considerar todas las posibles particiones de $E_i$, siempre habrá alguna tal que los $E_i$ sean subconjuntos o bien de $X_+$ o de $X_-$ (es decir, que o bien $E_i ∩ X_+ = ∅$ o $E_i ∩ X_- = ∅$).

Si nos encontramos con una partición de este tipo, tendremos que $\abs{μ(E_i)} = μ_+(E_i)$ o que $\abs{μ(E_i)} = μ_-$, y por lo tanto la suma sobre todos los elementos de la partición será igual a $\abs{μ}(E)$. Sólo nos faltaría demostrar que esta partición es la que tiene valor máximo.

Por suerte, eso lo podemos ver muy fácilmente. Sea $\set{E^*_n}$ otra partición de $E$ que no cumpla la ``exclusividad'', es decir, que existan conjuntos cuyas intersecciones con $X_+$ \textbf{y} con $X_-$ sean no vacías. En ese caso, podemos considerar que cada elemento de la partición se puede descomponer $E^*_i = E_j ∪ E_k$ con $E_j, E_k$ parte de la partición $\set{E_n}$ y con\footnote{Si esto no ocurre porque no tenemos conjuntos suficientemente pequeños, siempre podemos montarnos otra partición que tendrá la misma medida de la suma.} $E_j ⊆ X_+$, $E_k ⊆ X_-$, de tal forma que \[ \abs{μ(E_i^*)} = \abs{μ(E_j) + μ(E_k)} = \abs{μ_+(E_j) - μ_-(E_k)} ≤ μ_+(E_j) + μ_-(E_k) \] donde para algunos $E_i$ la desigualdad será estricta (los que intersequen con $X_+$ y con $X_-$) y para otros habrá igualdad (más que nada sólo intersecarán con $X_+$ ó con $X_-$ y $E_j$ o $E_k$ serán vacíos). En cualquier caso, la suma sobre la partición siempre será menor que la suma sobre la partición ``exclusiva'' $\set{E_n}$.

Con esta demostración, es fácil ver que las definiciones de $μ_+, μ_-$ en base a supremo e ínfimo de subconjuntos de $E$ son válidas. Sólo tenemos que darnos cuenta de que $μ(F)$ valdrá el máximo cuando $F = E ∩ X_+$, y el mínimo (negativo, por eso el cambio de signo) cuando $F = E ∩ X_-$.

\spart

\spart

No veo muy bien cómo hacerlo, más que nada porque si $μ_+(E) = μ_-(E)$ se nos puede estropear la cosa.

\spart

\spart Bastante trivial.

\spart Hay dos tipos de implicaciones: o las que se ven fácilmente o las que me da la sensación de que no funcionan, como lo de $μ_1 \perp μ_{2,+} \iff μ_1 \perp μ_{2,-}$.

\end{problem}

\begin{problem}[6] Sea $\appl{μ}{\mathcal{B}(ℝ)}{[-∞,+∞]}$ una medida con signo sobre los borelianos de $ℝ$. Definimos la función $g(x) = μ([0,x])$. Probar que

\ppart $μ$ es medida continua si y sólo si $g$ es continua.
\ppart $μ$ es medida absolutamente continua si y sólo si $g$ es una función absolutamente continua.
\solution

Vamos con la definición de función absolutamente continua primero.

\begin{defn}[Función\IS absolutamente continua] \label{def:FuncAbsCont} $\appl{f}{ℝ}{ℝ}$ lo es si y sólo si para todo $ε>0$ existe un $δ>0$ tal que \[ \sum_{i=1}^k \abs{f(y_i) - f(x_i)} < ε\] para cualquier familia de intervalos disjuntos $[x_1, y_1], \dotsc, [x_n, y_n]$ tales que \[ \sum_{i=1}^k \abs{y_i - x_i} < δ \]

La continuidad absoluta implica continuidad uniforme (donde la δ no depende del punto que se coja), que a su vez implica continuidad normal (donde δ depende del ε pero también del punto $x_0$ que se valore).

La continuidad absoluta es la clase de funciones más grande tal que el teorema fundamental del cálculo integral con la integral de Lebesgue y la derivada clásica es cierto.
\end{defn}

\spart

Si $g$ es continua en $x_0$, entonces para todo $ε > 0$ existe un $δ > 0$ tal que, si $\abs{x_0 - x} < δ$, entonces $\abs{g(x_0) - g(x)} < ε$. Transformando esa última ecuación y suponiendo que $x_0 < x$, tenemos que $\abs{μ([0, x_0]) - μ([0,x])} = \abs{μ([x_0,x)} < ε$. Igualmente, si tomamos $x_1 > x$ con $\abs{x_1 - x} < δ$, tendremos que $\abs{μ([x, x_1))} < ε$. Usando la desigualdad triangular y operando un poquillo nos quedará que $\abs{μ([x_0, x_1))} < 2ε$.

Podemos hacer ε tan pequeño como queramos, de tal forma que $μ([x_0, x_1)) \to 0$ con $x ∈ [x_0, x_1)$ siempre, y por convergencia dominada tendremos que $μ(\set{x}) = 0$.

Creo que además podemos recorrer esta demostración al revés para sacar la implicación al otro lado.

\spart


\end{problem}

\begin{problem}[8] Si $f ∈ \espL$, con $f \not\equiv 0$, probar que

\ppart Existen $c, R > 0$ tales que la función maximal $Mf(x) ≥ c \abs{x}^{-N}$ para $\abs{x} > R$.
\ppart Como consecuencia, cuando $t > 0$ es pequeño, existe $c'$ tal que \[ m\left(\set{x ∈ ℝ^N \tq Mf(x) ≥ t}\right) ≥ \frac{c'}{t} \]
\ppart Probar que, en general, es falsa la desigualdad \[ \norm{Mf}_{\espL} ≤ C \norm{f}_{\espL} \]

Indicación: describir la función maximal de $f(x) = \ind_{(-1, 1)} (x)$.
\solution

\spart Cogemos $α = \inf Mf(x)$, que está bien definido ($Mf(x) ≥ \abs{f(x)} ≥ 0$) y además no es 0 (es el supremo de los valores medios para $r > 0$, sólo puede ser $0$ si $f \equiv 0$ y ya hemos dicho por hipótesis que eso no ocurre). Por otra parte, $c \abs{x}^{-N} < c R^{-N}$ si $\abs{x} >R$, así que sólo tenemos que coger $c,R$ que cumplan que $α ≥ cR^{-N}$.

\spart Es cierto, sólo hay que aplicar el \nref{thm:Maximal}. No sé, eso sí, como hacerlo como consecuencia de lo otro.

\spart Cuando $x ∈ (-1, 1)$, el valor de la función maximal es $1$. Cuando está fuera, el valor máximo se alcanzará en el primer momento en el que la bola $\bola_r(x)$ cubra $(-1,1)$ (es decir, la bola de menor radio que cubre ese intervalo). Esa bola será el intervalo $(-1, x + (x + 1))$ (si $x < 0$ es análogo), esto es, la bola de radio $x + 1$. Así, tendremos que \[ Mf(x) = \fint\limits_{(-1, 2x + 1)}\ind_{(-1, 1)} (t) \dif t = \frac{m((-1, 1))}{m((-1, 2x + 1))} = \frac{1}{\abs{x}} \quad x ∉ (-1,1) \]

Si ahora hacemos la integral de $Mf$ nos quedará que \[ \norm{Mf}_{\espL} = \int_{-∞}^∞ \abs{Mf(x)} \dif x = 2 \int_{1}^∞ \frac{1}{x} \dif x + \int_{-1}^1 \dif x \] que no converge por el $\frac{1}{x}$, luego la desigualdad no se cumple.

\end{problem}

\begin{problem} A partir del \nref{thm:Maximal} y usando la \nlref{thm:DesigualdadJensen}, probar que si $p ∈ [1,∞)$ existe $C < ∞$ tal que para toda $f ∈ \espLp(ℝ)$ y para todo $λ>0$ se cumple que \[ m\left(\set{x ∈ ℝ \tq Mf(x) > λ}\right) ≤ \frac{C}{λ^p} \int_ℝ \abs{f(y)}^p \dif y \] donde $M$ es la \nlref{def:FuncMaximalHL}.

\solution

Está claro que si $Mf(x) > λ$, entonces $(Mf(x))^p > λ^p$. Operamos con $(Mf(x))^p$: \[ (Mf(x))^p = \left(\sup_{r>0} \fint_{\bola_r(x)} \abs{f} \dif μ\right)^p = \sup_{r>0} \left(\fint_{\bola_r(x)} \abs{f} \dif μ\right)^p\]

Sea entonces $φ(t) = t^p$. Como nos dicen que $p ∈ [1,∞)$, entonces φ es convexa y estamos en las condiciones de la desigualdad de Jensen. Así, podemos decir que \[ \sup_{r>0} \left(\fint_{\bola_r(x)} \abs{f} \dif μ\right)^p ≤ \sup_{r>0} \fint_{\bola_r(x)} \abs{f}^p \dif μ = Mf^p (x)\]

Como $(Mf(x))^p ≤ Mf^p(x)$, entonces\footnote{Por simplificar notación $\set{Mf(x) > λ} \equiv \set{x ∈ ℝ \tq Mf(x) > λ}$.} $m\left(\set{(Mf(x))^p > λ^p}\right) ≤ m\left(\set{Mf^p(x) > λ^p}\right)$. Aplicando el Teorema Maximal a esa segunda medida, tenemos que \[ m\left(\set{(Mf(x))^p > λ^p}\right) ≤ m\left(\set{Mf^p(x) > λ^p}\right) ≤ \frac{C}{λ^p} \int_ℝ \abs{f^p(y)} \dif y \] que es lo que queríamos demostrar (como tenemos el valor absoluto ahí, podemos meter y sacar el exponente $p$ tranquilamente).

\end{problem}

\begin{problem}[13] Probar el siguiente teorema que relaciona la derivada de Radon-Nikodim con el \nref{thm:DifLebesgue}. Sea μ una medida de Borel Regular y sea $μ = μ_S + μ_{AC}$ con $μ_{AC} = f \dif m$.  Entonces, para casi todo punto $x ∈ ℝ^n$, se tiene que \[ \lim_{r \to 0} \frac{μ(E_r)}{m(E_r)} = f(x) \] para toda familia $\set{E_r}$ que tiende suavemente (\fref{def:EncogerBien}) a $x$.

\solution

Siguiendo con la descomposición, tenemos que $μ(E_r) = μ_S(E_r) + μ_{AC} (E_r)$. Tenemos que ver que la parte singular nos la podemos quitar.

Como $E_r$ es una familia que tiende suavemente a $x$, entonces $m(E_r) > α m(\bola_r(x))$ con $α > 0$. En concreto, esto nos dice que $m(E_r) ≠ 0$. Como $μ_S \perp m$, la intersección de sus soportes tiene medida $0$, luego $μ(E_r) = 0$.

Ahora sólo tenemos que ir a por la parte singular continua, que como se tiene que $μ_{AC} = f \dif m$ con $f$ integrable, entonces nos queda que \[ μ_{AC} (E_r)  = \int_{E_r} f \dif m \]

Dividiendo esto por $m(E_r)$, nos queda el valor medio de $f$, así que finalmente tenemos \[ \lim_{r \to 0} \frac{μ(E_r)}{m(E_r)} = \frac{1}{m(E_r)} \int_{E_r} f \dif m = f(x)\] por el \nlref{thm:DifLebesgue}.
\end{problem}

\begin{problem}
Probar que si $f ∈ \espLloc$ y es continua en $x_0 ∈ ℝ^N$, entonces $x_0 ∈ L_f$. Es decir, que cada punto de continuidad es un \nlref{def:PuntoLebesgue}.
\solution

Recordamos que $x_0$ es un punto de Lebesgue de $f$ si y sólo si \[ \lim_{r \to 0} \fint_{\bola_r(x_0)} f \dif m = f(x_0) \]

Si $f$ es continua en $x_0$, entonces $∀ε > 0$ existe un $δ > 0$ tal que, si $\abs{x-x_0} < δ$, entonces $\abs{f(x) - f(x_0)} < ε$. Lo que hacemos entonces es coger bolas de radio $δ \to 0$ e integrar $f(x) - f(x_0)$:
\[ \fint_{\bola_δ(x_0)} \abs{f(x) - f(x_0)} \dif x ≤ \fint_{\bola_δ(x_0)} ε \dif x = ε \]

Si esta integral se va a cero con el valor absoluto cuando $ε \to 0$ (y por lo tanto $δ \to 0$ igualmente), entonces está claro que $\fint_{\bola_δ(x_0)} f(x) - f(x_0) \dif x \to 0$, y podemos sacar el $f(x_0)$ de ahí para quedarnos con la igualdad que teníamos que demostrar.
\end{problem}

\section{Hoja 3 - Espacios $L^p$}

\begin{problem}[2] Usar la \nref{thm:DesigualdadJensen} para demostrar las siguientes desigualdades:

\ppart Desigualdad entre media aritmética y geométrica. Para todo $1 ≤ n ∈ ℕ$ y todo $x_i ∈ [0, +∞)$, \[ (x_1 \dotsb x_n)^{\frac{1}{n}} ≤ \frac{x_1 + \dotsb + x_n}{n} \]

\ppart Desigualdad entre media aritmética y geométrica generalizada: para todo $1 ≤ n ∈ ℕ$ y todo $α_i,x_i ∈ [0, +∞)$ con $α_1 + \dots + α_n = 1$, se tiene que \[ \prod x_i^{α_i} ≤ \sum α_i x_i \]

\solution

\spart Una función que nos permite pasar de productos a sumas es el logaritmo, así que vamos a ver qué pasa si tomamos $φ = \log$ en un espacio discreto. Por ser formales, diremos que $X = ℤ_n$, con $f(z) = x_z$ y la medida dada como $μ(z) = 1\;∀z ∈ ℤ_n$, de tal forma que  $\int f = \sum x_i$ y $μ(ℤ_n) = n$.

En este caso, φ no es convexa sino cóncava. Lo bueno es que entonces $-φ$ es convexa, así que podemos multiplicar por $-1$ en ambos lados de la desigualdad y se nos invierte. Así, lo que tenemos es que \[ \log\left(\frac{1}{n} \sum x_i\right) ≥ \frac{1}{n} \sum \log x_i  = \frac{1}{n} \log \prod x_i = \log \left(\prod x_i\right)^{\frac{1}{n}} \] y quitando los logaritmos ya tenemos lo que nos pedían.

\spart Aquí tenemos que cambiar la medida, y decir que $μ(z) = α_z$, lo que nos queda que $μ(ℤ_n) = 1$, que en el fondo nos va a simplificar algo las cosas. Aplicando de nuevo la desigualdad de Jensen con $φ = \log$, tenemos que \begin{align*}
\log\left(\int f \dif μ\right) &≥ \int \log f \dif μ \\
\log\left(\sum α_i x_i\right) &≥ \sum α_i \log x_i = \log\left(\prod x_i^{α_i}\right)
\end{align*}, que de nuevo se nos queda en lo que nos pedían demostrar si quitamos logaritmos.

\end{problem}

\begin{problem} Probar la \nlref{prop:DesHolder} como consecuencia de la \nlref{thm:DesigualdadJensen}.

\solution

La desigualdad de Hölder nos dice que, siendo $1 ≤ p ≤ + ∞$ y $q = \frac{p}{p - 1} ≥ 1$ de tal forma que $\frac{1}{p} + \frac{1}{q} = 1$, tenemos que \[ \int_X \abs{fg} \dif μ ≤ \norm{f}_p \norm{g}_q \]

Para demostrarla, vamos a hacer un truco que consiste en montar una medida artificial. Sea \[ ν = \frac{f^p}{\norm{f}_p^p} \] una medida, cuya principal ventaja es que $ν(X) = \int_X \frac{f^p}{\norm{f}_p^p} \dif μ = 1$ y por lo tanto $\fint \dif ν\equiv \int \dif ν$.

Sea $φ(z) = z^q$, convexa. Podemos aplicar ahora la desigualdad de Jensen a una función $h$ que ya veremos luego cómo montamos. La desigualdad nos dice que \begin{align*}
\left(\int h \dif ν\right)^q &≤ \int h^q \dif ν \\
\left(\int \frac{hf^p}{\norm{f}_p^p} \dif μ\right)^q &≤ \int \frac{h^qf^p}{\norm{f}_p^p} \dif μ
\end{align*}

Querríamos ahora que $hf^p$ fuese $fg$, así que tomamos $h = \frac{g}{f^{p-1}}$, que además nos da que $h^q f^p = \frac{g^qf^p}{f^{q(p-1)}} = \frac{g^qf^p}{f^p} = g^q$. Sustituyendo, tenemos que
\begin{align*}
\left(\int \frac{hf^p}{\norm{f}_p^p} \dif μ\right)^q &≤ \int \frac{h^qf^p}{\norm{f}_p^p} \dif μ \\
\frac{1}{\norm{f}_p^{pq}} \left(\int fg \dif μ\right)^q &≤ \frac{1}{\norm{f}_p^p} \int g^q \dif μ \\
\int fg \dif μ &≤ \norm{f}_p^{\frac{p(q-1)}{q}} \norm{g}_q \\
\int fg \dif μ &≤ \norm{f}_p \norm{g}_q
\end{align*}, que es la desigualdad de Hölder ya que $p\frac{q - 1}{q} = p\frac{\frac{p}{p-1} - \frac{p - 1}{p-1}}{\frac{p}{p - 1}} = \frac{\frac{1}{p-1}}{\frac{1}{p-1}}$ = 1.
\end{problem}


\begin{problem} Probar que la desigualdad de Hölder es una igualdad si y sólo si $\abs{f}^p = λ\abs{g}^{q}$ CTP para alguna constante $λ > 0$, con $q$ el exponente conjugado de Hölder.

\solution

Si $\abs{f}^p = λ\abs{g}^{q}$, entonces \[ \norm{f}_p = \left(\int_X \abs{f}^p \right)^{\frac{1}{p}} = \left(\int_X λ \abs{g}^q \right)^{\frac{1}{p}} = λ^\frac{1}{p} \norm{g}_q^{\frac{q}{p}} \], y como $\frac{q}{p} + 1 = \frac{q + p}{p} = q$, entonces $\norm{fg}_q = λ^{\frac{1}{p}} \norm{g}_q^q$.

Por otra parte \[ \int_X \abs{fg} \dif μ = \int_X \abs{λ^\frac{1}{p} g^\frac{q}{p} g} \dif μ  = λ^{\frac{1}{p}} \int_X g^q \dif μ = λ^{\frac{1}{p}} \norm{g}_q^q \], así que en este sentido queda demostrado.

Supongamos ahora que $\norm{fg}_1 = \norm{f}_p \norm{g}_q$. Entonces, no sé cómo demostrarlo.

\end{problem}

\begin{problem}[5] Probar por inducción las siguientes generalizaciones de la \nlref{prop:DesHolder}.

\ppart Dados $1 < p_1, p_2, \dotsc, p_n < ∞$ con $f_i ∈ L^{p_i}$ para $i = 1,2, \dotsc, n$, entonces se cumple que \[ \abs{\int \prod f_i \dif μ} ≤ \prod \norm{f_i}_{p_i} \] cuando $\sum\frac{1}{p_i} = 1$.
\solution

\spart Sabemos que para $n = 1,2$ se cumple. Supongamos que se cumple para $n - 1$ y vamos a demostrar que se cumple para $n$. Consideramos $g = \prod_{i=1}^{n-1} f_i$, con $\sum_{i=1}^{n-1}\frac{1}{p_i} = 1$, que sabemos que cumple que $\norm{g}_1 ≤ \prod \norm{f_i}_{p_i}$, luego en concreto $g ∈ L^1$.

\end{problem}

\section{Hoja 4 - Espacios de Hilbert}

\begin{problem}[6]

\ppart Demostrar que la norma $\norm{f}_{∞}$ definida en $L^∞([0,1])$ no puede provenir de un producto escalar.

\ppart Demostrar que $L^2([0,1])$ es el único espacio de Hilbert de entre todos los espacios $L^p([0,1])$, con $0 < p < ∞$.

\textit{Indicación: probar que no se cumple la \nlref{prop:IdParal}.}

\solution

\spart

Para demostrarlo, vamos a estudiar la norma de la suma de dos funciones. Recordamos que en la norma de $L^∞$, tenemos que $\norm{f}_∞ = \essup \abs{f}$.

Consideramos dos funciones $f,g ∈ L^∞$ no nulas pero tales que $fg = 0$ CTP (es decir, si en un punto una no es cero la otra sí lo es) y con $α = \norm{f}_∞ = \norm{g}_∞$. Entonces, será obvio que $\norm{f+g}_∞ = α$.

Si la norma $L^∞$ viniese de un producto escalar, entonces podríamos escribir la norma de la suma como \[ \norm{f+g}_∞^2 = \pesc{f+g,f+g} = \pesc{f,f} + \pesc{f,g} + \pesc{g,f} + \pesc{g,g} = \norm{f}_∞^2 + \norm{g}_∞^2 = 2α^2 \]

Tendríamos que $α^2 = 2α^2$, que sólo se cumple cuando $α = 0$ y por tanto cuando $f$ y $g$ son nulas en casi todo punto, contradicción porque habíamos dicho que no lo eran.

\spart

Supongamos que, en general $L^p$ es un espacio Hilbert. La identidad del paralelogramo (\fref{prop:IdParal}) nos dice que, para $f,g ∈ L^p([0,1])$ se cumple que \[ \norm{f + g}^2_p + \norm{f - g}^2_p = 2\norm{f}^2_p + 2 \norm{g}^2_p \]

En particular, deberá cumplirse para $f = g = \ind_{[0,\sfrac{1}{2}]}, g = \ind_{[\sfrac{1}{2},1]}$. Así, nos quedaría que \begin{align*}
\left(\frac{1}{2} · 2^p\right)^{\frac{2}{p}} + 0 &= 4 · \frac{1}{2^{\sfrac{2}{p}}} \\
\frac{1}{4^{\sfrac{1}{p}}} · 4 &= 4 · \frac{1}{4} \\
\end{align*}, lo que sólo se cumple cuando $p = 1$.

\end{problem}

\begin{problem} Sea $\appl{U}{E_1}{E_2}$ una isometría lineal entre dos espacios de Hilbert, es decir, que $∀\vx,\vy ∈ E_1$ tengamos que \[ \norm{U(\vx) - U(\vy)}_{E_2} = \norm{\vx-\vy}_{E_1}\]

Probar que $U$ es un operador unitario, es decir, que $\pesc{U(\vx), U(\vy)}_{E_2} = \pesc{\vx,\vy}_{E_1}$.

\solution

Lo primero de lo que nos damos cuenta es de que, si tomamos $\vy = 0$, entonces $U(\vy) = 0$ por ser lineal y entonces $\norm{U(\vx)}_{E_2} = \norm{\vx}_{E_1}$. Ahora podemos operar y sale directamente eliminando las normas: \begin{align*}
\norm{U(\vx) - U(\vy)}_{E_2}^2 &= \norm{\vx-\vy}_{E_1}^2 \\
\norm{U(\vx)}^2_{E_2} + \norm{U(\vy)}^2_{E_2} - 2 \pesc{U(\vx),U(\vy)}_{E_2} &=
	  \norm{\vx}^2_{E_1} + \norm{\vy}^2_{E_1} - 2 \pesc{\vx,\vy}_{E_1} \\
\pesc{U(\vx),U(\vy)}_{E_2} &= \pesc{\vx,\vy}_{E_1}
\end{align*}

\end{problem}

\begin{problem}[9]
Sea $(H, \pesc{·,·})$ un espacio de Hilbert.

\ppart Si $E ⊂ H$, demostrar que $E^\perp$ es un subespacio vectorial cerrado de $H$.
\ppart Si $M$ es un subespacio vectorial cerrado de $H$, demostrar que $(M^\perp)^\perp = M$.
\ppart Sea $\mathcal{U} = \set{u_α}$ un sistema ortonormal no finito. Demostrar que $\mathcal{U}$ es un conjunto cerrado y acotado, pero no compacto.
\ppart Si $x_0 ∈ H$ y $M$ es un subespacio vectorial cerrado de $H$, demostrar que \[ \min\set{\norm{x_0 - x} \tq x ∈ M} = \max \set{\abs{\pesc{x_0,y}} \tq y ∈ M^\perp,\;\norm{y} = 1} \]

\solution

\spart Esto es simplemente la \fref{prop:PropsComplOrtogonal}.

\spart De nuevo la \fref{prop:PropsComplOrtogonal}, viendo que si $M$ es cerrado entonces $\adh{M} = M$.

\spart

\spart

Parece intuitivo tomando $y = \frac{x-x_0}{\norm{x-x_0}}$ pero no se me ocurre cómo demostrarlo.

\end{problem}

\section{Hoja 5 - Series de Fourier}

\begin{problem} Hallar la serie de Fourier de las siguientes funciones:
\ppart $f(x) = \frac{(π-x)^2}{4}$ en $0 ≤ x < 2π$.
\ppart $g(x) = \frac{π}{\sin πα} e^{i(π-x)α}$ en $0≤x<2π$ con $α∉ℤ$.
\ppart $h(x) = x(π-x)$ en $[0,π]$ extendida de manera impar a $[-π,0]$.

\solution

En la teoría (\fref{sec:SeriesFourier}) no hemos visto ejemplos de cálculo de series de Fourier, así que lo veremos aquí. Lo más sencillo es tomarlo como un problema de álgebra vectorial. Demostramos que $\set{e^{2πinx}}_{n ∈ ℤ}$ es una base ortonormal de $L^2([0,1])$, que es un espacio Hilbert.  Como en cualquier espacio vectorial, para obtener la coordenada $n$-ésima de un vector (una función en este caso) hacemos el producto escalar de ese vector con el $n$-ésimo elemento de la base. Esto es, dada $f ∈ L^2([0,1])$ tenemos que $f_n = \pesc{e_n, f}$, lo que, teniendo en cuenta cómo hemos definido el producto escalar en espacios de funciones integrables, nos deja la magnífica fórmula que ya habremos visto salir de la nada en otras ocasiones. Cada coordenada será de la forma \[ \hat{f}_n = \int_0^1 f(x) e^{2πinx} \dif x \] y entonces la función se podrá expresar como \[ f = \sum_{n∈ℤ} \hat{f}_n e^{2πinx} =  \sum_{n∈ℤ} e^{2πinx} · \int_0^1 f(t) e^{2πint} \dif t \]

Por supuesto, podremos expresar como una serie cualquier función periódica $f ∈ L^2([a,b])$. Por ejemplo, si $f ∈ L^2([0,2π])$, entonces la base del espacio Hilbert es $\set{e^{inx}}_{n∈ℤ}$, y tendremos que cambiar eso en las fórmulas anteriores (también habrá que cambiar el intervalo de integración).

\spart

En el primer coeficiente, $e^{iπ0} = 1$ y por lo tanto la integral se queda sencillita:
\[ \hat{f}_0 = \frac{1}{2π}\int_0^{2π} \frac{(π-x)^2}{4} \dif x =  \eval{\frac{-1}{2π}·\frac{(π-x)^3}{12}}_{x=0}^{2π} =\frac{-1}{2π} · \left(\frac{-π^3}{12} - \frac{π^3}{12}\right) = \frac{π^2}{12} \]

Para los siguientes hay que hacer algo más, integrando por partes dos veces. Hacemos primero la integral impropia \begin{align*}
\int \underbracket{\frac{(π-x)^2}{4}}_u \underbracket{e^{inx} \dif x}_{\dif v} &=
	\frac{(π-x)^2}{4} \frac{e^{inx}}{in} + \int \underbracket{\frac{π-x}{2}}_u \underbracket{\frac{e^{inx}}{in}\dif x}_{\dif v} = \\
	&= \frac{(π-x)^2}{4} \frac{e^{inx}}{in} + \frac{π-x}{2} \frac{e^{inx}}{-n^2} + \int \frac{1}{2} \frac{e^{inx}}{-n^2} \dif x = \\
	&= \frac{(π-x)^2}{4} \frac{e^{inx}}{in} + \frac{π-x}{2} \frac{e^{inx}}{-n^2}  - \frac{e^{inx}}{2in^3} = \\
	&= e^{inx} \left(\frac{(π-x)^2}{4in} - \frac{π-x}{2n^2} - \frac{1}{2in^3}\right)
\end{align*}

Evaluando ahora en $[0,2π)$ y reescalando (tenemos que multiplicar por $\frac{1}{2π}$ el resultado de la integral) tenemos que \[ \hat{f}_n = \frac{1}{2π} \left( - \frac{π-2π}{2n^2} + \frac{π-0}{2n^2} \right) = \frac{1}{2n^2} \]

Juntándolo todo, tenemos que $f(x)$ se puede expresar como \[ f(x) = \frac{π^2}{12} + \sum_{n≠0} \frac{1}{2n^2} e^{inx} \]

\spart

Está bien que nos digan que $α ∉ ℤ$ más que nada porque nos evitamos que esa fracción $\frac{π}{\sin πα}$ se anule y se vaya a infinito. En cualquier caso, como es constante y no depende de $x$, nos lo podemos quitar y calcular los coeficientes de Fourier de $e^{i(π-x)α}$: \begin{align*}
\hat{g}_n &= \frac{1}{2π} \int_0^{2π} e^{i(π-x)α} e^{inx} \dif x
	= \frac{1}{2π} \int_0^{2π} e^{xi(n-α)}e^{iπα} \dif x = \frac{e^{iπα}}{2π} \eval{\frac{e^{xi(n-α)}}{i(n-α)}}_{x=0}^{2π} = \\
	&= \frac{e^{iπα}}{2πi(n-α)} \left(e^{2πin}e^{-2πiα} - 1\right) = \frac{e^{-iπα} - e^{iπα}}{2πi(n-α)} =
	\frac{\sin πα}{π(n-α)}
\end{align*}

Multiplicando ahora por el $\frac{π}{\sin πα}$ que nos habíamos dejado antes, tendremos que \[ g(x) = \sum_{n ∈ ℤ} \frac{1}{n-α} e^{inx} \]

\end{problem}

\begin{problem}[4] Se considera la función $\appl{f}{[-π,π]}{ℝ}$ dada por \[ f(x) = \begin{cases} 0 & \abs{x} > δ \\ 1 - \frac{\abs{x}}{δ} & \abs{x} ≤ δ\end{cases} \]

Dibujar la gráfica de $f$ y demostrar que \[ f = \frac{δ}{2π} + 2 \sum_{n=1}^∞ \frac{1-\cos nδ}{πn^2δ} \cos nx \]
\solution

\begin{wrapfigure}{r}{0.4\textwidth}
\centering
\inputtikz{H5_E4}
\caption{Gráfica de la función $f$.}
\label{fig:Hoja5:Ej4}
\end{wrapfigure}

La gráfica de la función la podemos ver en la \fref{fig:Hoja5:Ej4}: es básicamente un triángulo de altura $1$ y base $2δ$. Para demostrar la expresión que nos piden calcularemos su desarrollo de Fourier. Para $n = 0$, tenemos que \[ \hat{f}_0 = \frac{1}{2π} \int_{-δ}^δ 1 - \frac{\abs{x}}{δ} \dif x = \frac{δ}{2π} \]

Para $n ≠ 0$, calculamos la integral: \[ \hat{f}_n = \frac{1}{2π} \int_{-δ}^δ \left(1 - \frac{\abs{x}}{δ}\right) e^{inx} \dif x = \frac{1}{2π} \left( \eval{\frac{e^{inx}}{in}}_{x = -δ}^δ - \frac{1}{δ} \underbracket{\int_{-δ}^δ e^{inx} \abs{x} \dif x}_{I} \right)\]

Calculamos la integral por separado \[ I = \int_{-δ}^δ \abs{x} e^{inx}\dif x = \int_{-δ}^0 -x e^{inx} \dif x + \int_0^δ xe^{inx} \dif x \]

Resolvemos la integral impropia (es la misma en ambos casos): \[ \int \underbracket{x}_u \underbracket{e^{inx} \dif x}_{\dif v} = \frac{xe^{inx}}{in} - \int \frac{e^{inx}}{in} \dif x  = \frac{xe^{inx}}{in} + \frac{e^{inx}}{n^2} \]

Sustituyendo ahora: \begin{align*}
I &= - \left(\frac{1}{n^2} - \frac{-δe^{-inδ}}{in} - \frac{e^{-inδ}}{n^2}\right) + \left(\frac{δe^{inδ}}{in} + \frac{e^{inδ}}{n^2} - \frac{1}{n^2}\right) = \\
	&= \frac{δe^{inδ} - δe^{-inδ}}{in} + \frac{e^{inδ} + e^{-inδ}}{n^2} = \\
	&= \frac{δ 2 \sin nδ}{n} + \frac{2 \cos nδ}{n^2}
\end{align*}

Por otra parte, \[ \eval{\frac{e^{inx}}{in}}_{x = -δ}^δ = \frac{e^{inδ} - e^{-inδ}}{in} = \frac{2\sin nδ}{n} \], así que juntándolo todo tenemos que \[ \hat{f}_n = \frac{- \cos nδ}{δπn^2} \]

Y bueno, ya supongo que la función esa que decían sale. No me apetece hacer más cuentas.

\end{problem}


\begin{problem} Usando el ejercicio anterior y la identidad de Parseval \eqref{eq:IdParseval}, demostrar lo siguiente:

\ppart $\sum_{n=0}^∞ \frac{1}{(2n+1)^4} = \frac{π^4}{96}$.
\ppart $\sum_{n=1}^∞ \frac{1}{n^4} = \frac{π^4}{90}$.

\solution
\end{problem}

\section{Hoja 6 - Transformada de Fourier}

\begin{problem} Para una función $\appl{f}{ℝ^N}{ℝ}$ se define la traslación por $y ∈ ℝ^N$ como $(T_yf) (x) = f(x-y)$, la modulación por $z ∈ ℝ^N$ como $(M_zf)(x) = e^{2πix·z}f(x)$ y la dilatación por una matriz de orden $N$ no singular $A$ como $(D_A f)(x) = f(Ax)$. Demostrar las siguientes identidades:

\ppart $\widehat{(T_yf)}(ξ) = (M_{-y}\hat{f})(ξ)$.
\ppart $\widehat{(M_zf)}(ξ) = (T_{z}\hat{f})(ξ)$.
\ppart $\widehat{(D_Af)}(ξ) = \frac{1}{\abs{\det A}}(D_{(\inv{A})^T}\hat{f})(ξ)$.

\solution

A modo de recordatorio, la \nref{def:TransFourier} se define como \[ \hat{f}(ξ) = \int_{ℝ^N} f(x) e^{-2πixξ} \dif x\quad ξ∈ℝ^N \]

\spart

\[ \widehat{(T_yf)}(ξ) = \int_ℝ f(x-y) e^{-2πixξ} \dif x = e^{2πiξ·(-y)} \int_ℝ f(x-y) e^{-2πi(x - y)ξ} \dif x = (M_{-y}\hat{f})(ξ) \]

\spart

\[ \widehat{(M_z f)}(ξ) = \int_ℝ e^{2πix·z} f(x) e^{-2πixξ} \dif x = \int_ℝ f(x) e^{-2πix(ξ-z)} \dif x = (T_z\hat{f}) (ξ) \]

\spart

\[ \widehat{(D_Af)}(ξ) = \int_ℝ f(Ax) e^{-2πixξ} \dif x \]

\end{problem}


\begin{problem}[6] Si $λ > 0$ y $x ∈ ℝ$, calcular \[ \lim_{R\to ∞} \int_{-R}^R \frac{\sin λt}{t} e^{itx} \dif t \]

Sugerencia: escribir $\sfrac{\sin λt}{t}$ como transformada de Fourier de una función característica, usar Fubini y después pasar al límite. En el último paso quizás necesites usar $\int_0^{∞} \sfrac{\sin x}{x} \dif x = \sfrac{π}{2}$.

\solution


\end{problem}
