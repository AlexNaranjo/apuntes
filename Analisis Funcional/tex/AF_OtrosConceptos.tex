% -*- root: ../AnalisisFuncional.tex -*-
En esta sección se pueden encontrar algunos conceptos extra de otras asignaturas (principalmente de Variable Real).


\begin{theorem}[Teorema\IS de convergencia dominada] \citep[Teorema I.18]{ApuntesVarReal} \label{thm:ConvDominada} Dada una sucesión $f_n \convs f$ en casi todo punto, si existe una función $g$ integrable y no negativa tal que $\abs{f_n} ≤ g$ entonces \[ \int_X \lim_n f_n \dif μ = \lim_n \int f_n \dif μ\]
\end{theorem}

\begin{theorem}[Teorema\IS de convergencia monótona] \citep[Teorema I.14]{ApuntesVarReal} \label{thm:ConvMonotona}
Dado un espacio de medida \meas y una sucesión monótona de funciones $0 ≤ f_1 ≤ \dotsb ≤ f_n ≤ \dotsb$ medibles, entonces  \[ \int_X \lim_n f_n \dif μ = \lim_n \int_X f_n \dif μ \]

Se puede extender a sucesiones decrecientes simplemente cambiado el signo.
\end{theorem}

\begin{defn}[Medida\IS σ-finita] \citep[Def. II.9]{ApuntesVarReal} \label{def:MedidaSFinita} Dado un espacio medible \meds y una medida μ, se dice que μ es σ-finita si y sólo si existe un conjunto numerable $\algb{U} ⊆ \algb{X}$ tal que su unión recubre todo $X$ y $μ(U) < ∞$ para todo $U ∈ \algb{U}$. \end{defn}

\begin{theorem}[Teorema\IS de Tychonoff] \label{thm:Tychonoff} El producto cartesianos de espacios topológicos compactos es compacto con respecto de la topología producto.

Recordamos que en la topología producto los abiertos son producto cartesiano de abiertos en las topologías originales.
\end{theorem}
