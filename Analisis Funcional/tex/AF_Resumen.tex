% -*- root: ../AnalisisFuncional.tex -*-
\section{Espacios métricos completos}

En esta sección dedicamos un pequeño repaso a la topología de espacios métricos, principalmente a lo que es una \nlref{def:Metrica}, una \nlref{def:SucesionCauchy} y un \nlref{def:EspacioMetricoCompleto} (espacio donde toda sucesión de Cauchy converge en el espacio). que serán los dos conceptos principales que manejemos durante el curso. Una vez visto esto, se trabajan teoremas adicionales que se usan para las demostraciones sucesivas.

El primero es el \nref{thm:CategoriaBaire} para espacios $X$ métricos completos, que dice que la intersección numerable de abiertos densos es densa en $X$, y que además $X$ no es unión numerable de conjuntos diseminados (\fref{def:ConjuntoDiseminado}, conjuntos cuyo complementario del cierre es denso en $X$). Una aplicación del teorema es la existencia de funciones continuas que no son diferenciables en ningún punto (\fref{sec:FuncionesNoDiferenciables}).

El segundo teorema es el \nref{thm:AscoliArzela}, que dice que el cierre de una familia de funciones continuas es compacto si y sólo si la familia es equiacotada y equicontinua\footnote{Ver \fref{def:FamiliaUnifEquicont}, a grandes rasgos simplemente pedimos extender la definición de continuidad para que un mismo $δ$ valga para toda la familia de funciones.}. Para la prueba del teorema se desarrollan varios resultados, como el \nref{lem:AscoliArzela}, que nos dice lo mismo que el teorema (equiacotación y equicontinuidad) para una sucesión de funciones continuas que convergen uniformemente o el \nref{thm:PuntoFijoBanach} que nos dice que hay exactamente un punto fijo en una aplicación estrictamente contractiva (si no es estricta, el punto no tiene por qué existir, aunque como mucho hay uno).

\section{Espacios de Banach}

Con la topología desarrollada en el primer capítulo pasamos a estudiar lo que es un \nlref{def:EspacioBanach}, que es básicamente un espacio con una \nlref{def:Norma} y completo respecto a la topología inducida por la norma. Lo interesante de estos espacios son las aplicaciones que van de uno a otro: un \nlref{def:MorfismoEspaciosNormados}. Es interesante la distinción entre aplicaciones lineales y aplicaciones lineales y continuas (mejor explicada en la \fref{sec:AplicacionesEspaciosNormados}). La \fref{prop:AppLinealContinua} da algunas condiciones para que las aplicaciones lineales sean continuas.

Además, se puede definir una \nlref{prop:NormaEspacioAppLineales}, que nos lleva a poder decir que el espacio $\linapp$ es Banach si $Y$ es Banach (\fref{prop:AppLinealesEspBanach}). También se define el \nlref{def:DualTopologico} que usaremos para muchas demostraciones, que es una noción equivalente al dual de álgebra lineal pero obligando a que las aplicaciones sean además continuas.

\subsection{Teoremas importantes sobre espacios de Banach}

A lo largo del resto del capítulo se tratan varios teoremas importantes sobre espacios de Banach, a saber:

\begin{itemize}
\item \nref{thm:AcotacionUniforme}: Una colección $\set{T_α}_{α ∈ A} ⊂ \linapp$ con $X$ Banach está o bien uniformemente acotada o bien hay un \nlref{def:ConjuntoGDelta} denso, $B$, para el cual $\sup_α \norm{T_α(x)}_Y = ∞$ cuando $x ∈ B$. El corolario sencillo es que si demostramos que $\sup_α \norm{T_α(x)}_Y$ para $x ∈ X$, entonces la familia está uniformemente acotada.
\item \nref{thm:AppAbierta}: Una aplicación sobreyectiva $T ∈ \linapp$ ($X,Y$ Banach) lleva abiertos en abiertos y en particular $T(\bola_1(0)) ⊃ \bola_δ(0)$ para algún $δ > 0$. El corolario principal es que si $T$ es biyectiva, entonces $\inv{T} ∈ \linapp$.
\item \nref{thm:GraficaCerrada}: Si $X,Y$ son Banach y $\appl{T}{X}{Y}$, entonces si el gráfico $G(T) = \set{(x,T(x))}$ es cerrado entonces $T ∈ \linapp$.
\item \nref{thm:HahnBanach}: Un funcional lineal acotado $λ ∈ E^*$ con $E$ subespacio de $V$ se puede extender a un funcional lineal acotado con $\abs{Λ(x)} ≤ \norm{λ}\norm{x}$.
\end{itemize}

\subsection{Espacio cociente en espacios de Banach}

Una pequeña mezcla entre análisis y álgebra: se puede definir el espacio cociente de un espacio de Banach $\quot{X}{E}$ con $E$ un subespacio cerrado. Lo más relevante es que hay un \nref{thm:IsomorfiaBanach}.

\section{Topologías débiles y dualidad en espacios de Banach}

Para facilitar el estudio de los espacios de Banach y el análisis, introducimos las topologías débiles (\fref{def:TopologiaDebil} y \ref{def:TopologiaDebilStar}), de las cuales la primera propiedad interesante es que son Hausdorff.

Bajo estas topologías se puede estudiar la \nlref{def:ConvergenciaDebil} y la \nlref{def:ConvergenciaDebilStar}, que tienen ambas ciertas propiedades esperables (\fref{prop:ConvDebil}), básicamente que la convergencia en norma es más fuerte, que da una cota para la sucesión (además, $\norm{f} ≤ \liminf \norm{x_n}$) y que si $λ_n \to λ$ en norma, $\pesc{λ_n, x_n} \to \pesc{x_n, x}$.

También es interesante estudiar la \nlref{sec:CompacidadTopDebiles}, donde el principal resultado es el \nref{thm:BanachAlaoglu}, que dice que la bola cerrada unidad de un espacio de Banach dual es compacta en la topología débil-$*$. Además, es metrizable si el predual es separable (\fref{thm:BolaMetrizableDebilStar}).

Para la reflexividad, la proposición más importante es la \nref{prop:MillmanPettis}, que nos dice que todo \nref{def:EspacioUnifConvexo} y Banach es reflexivo; y el \nref{thm:Kakutani} que nos dice que es reflexivo si y sólo si $\bola_X$ es compacta en la topología débil.

\subsection{Determinación de espacios duales}

A lo largo de la \fref{sec:EspaciosDualesLp} se trabaja en concreto con los espacios $L^p$ y sus duales. El teorema importante es el \fref{thm:DualLp}, que nos dice que $\dual[L^p] = L^q$ cuando $1 < p < ∞$ y $\frac{1}{p} + \frac{1}{q} = 1$. Cuando $p= 1$, el dual es $L^∞$ si el espacio es de medida σ-finita (\fref{def:MedidaSFinita}).

Para las demostraciones es especialmente importante la \nref{prop:DesHolder}, que dice que $\norm{fg}_1 ≤ \norm{f}_p \norm{g}_q$ con $p,q$ exponentes conjugados.

En esta sección también se trabaja en la demostración de que $L^p$ es Banach (\fref{thm:EspacioLpBanach}) y en la identificación de funcionales del dual de $L^p$ con funciones $L^q$ de la forma $λ_g(f) = \int gf \dif μ$ (ver \fref{lem:FuncionalDual}).
