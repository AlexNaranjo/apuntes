\documentclass{apuntes}

\title{MNEDO}
\author{Pedro Valero \& El Malvado}
\date{15/15 C1}
% Paquetes adicionales
\usepackage{tikztools}
\usepackage{fastbuild}
\usepackage{tikz-3dplot}
\usepackage{fancysprefs}

\bibliographystyle{plainnat}

\usetikzlibrary{arrows}
% --------------------

\precompileTikz

\begin{document}
\pagestyle{plain}
\maketitle

\tableofcontents


\chapter{Problemas de valor inicial. Preliminares}

\section{Problemas de valor inicial. (PVI)}

A lo largo de este curso estudiaremos sistemas del tipo:
\[y'(x)=f(x,y(x))\]
siendo $y(x)$ una función del tipo
\[\appl{y}{ℝ}{ℝ^d}\]
\[y(x)=(y_1(x), …, y_d(x))\]

Por tanto $f$ quedará definida como:
\[\appl{f}{ℝ×ℝ}{ℝ^d} \]
\[f(x,y) = (x,y(x))\]

Como notación muchas veces consideraremos $y(x)=y$, del mismo modo que $y(a)=y_a$ con $a$ un valor dado.

\begin{remark}
	En el curso trabajaremos con funciones $f∈C\left( [a,b] × ℝ^d \right)$
\end{remark}


\begin{defn}[Función Lipschitz]
	Una función $f$ será Lipschitz en la segunda variable si cumple:
	\[\md{f(x,y_1) - f(x,y_2)} ≤ L\md{y_1-y_2}\]
	\[∀x ∈ [a,b], ∀y_1,y_2 ∈ℝ^d\]
\end{defn}

Los problemas de valor inicial (PVI) que vamos a estudiar son de la forma:
\[
	\begin{cases}
		y'(x)=f(x,y(x)) & x∈[a,b]\\
		y(a)=y_a
	\end{cases}
\]

Vamos a ver en qué situaciones este tipo de problemas tienen solución única:

\begin{theorem}[Teorema de Picard]
	\label{TeoremaPicard}
	Sea $Ω=[a,b]×ℝ^d$, si $f$ es continua en $Ω$ y Lipschitz en la segunda variable; entonces el PVI tiene una única solución $y∈C'\big[[a,b] × ℝ^d\big)$
\end{theorem}

\begin{proof}
	\textbf{Vamos a demostrar primero la existencia}:
	Para ello haremos uso de las iteradas de Picard.

	\[y(x)=y_a+\int_a^x f(s,y(s))ds\]

	Si definimos la aplicación $T_y = y_a + \int_a^x f(s,y(s))ds$, querríamos encontrar un punto fijo de dicha aplicación.

	\begin{enumerate}
		\item \textbf{Construimos la secuencia de funciones $\{ y_n(x) \}_{n=0}^∞$}:
		\[y_{n+1}(x) = y_a + \int_a^x f(s,y_n(s)) ds\]
		\[n=0,1,… \text{\ y \ } y_0(x)=y_a\]
		Cada función $y_n$ es continua (es una constante más una función continua).

		\item $∃y ∈ C([a,b])$ tal que:
		\[y_n \longrightarrow y \text{\ uniformemente}\]
		Que es equivalente a decir que $\lim_{n \to ∞} \md{y_n - y}_{L^∞[a,b]} = 0$. Donde $\md{f}_{L^∞[a,b]} = \max_{x ∈ [a,b]}\md{f(x)}$, denota la norma infinito de $f$.
	\end{enumerate}

	Antes de continuar vamos a recordar el Test M de Weierstrass:

	\begin{theorem}[Test M de Weierstrass]
		\label{TestMWeierstrass}
		Sea $\{f_n\}_{n=0}^∞$ una sucesión de funciones definidas en un dominio $Ω$ con valores en $ℝ$, si existe una sucesión de números $M_n$ tal que:
		\begin{enumerate}
			\item $\abs{f_n} < M_n \text{,\ } ∀x∈Ω$
			\item $\sum_{n=0}^∞ M_n < ∞$
		\end{enumerate}

		Entonces existe una función $g$ tal que:
		\[g_n(x) = \sum_{n=1}^N f_n - f_{n-1}\]
		\[g_n \xrightarrow[n \rightarrow ∞]{} g \text{,\ uniformemente en\ } x\]
	\end{theorem}

	¿Cómo vamos a usar esto para conseguir ver que $y_n \rightarrow y$?.\\
	Primero vamos a reescribir $y_n$ como (sabiendo que tomamos $y_a=y_0$) una suma telescópica:

	\[y_{n+1} = y_a + \sum_{k=0}^n y_{k+1} - y_k\]
	Para el caso $n=1$: $y_2 = y_a + (y_1 - y_0) + (y_2 - y_1) = y_2$.

	Ahora vamos a aplicar el test M de Weierstrass:

	Las $f_k$ que aparecen en el enunciado de \ref{TestMWeierstrass} serán $(y_{k+1} - y_k)$, si demuestro que $∃{M_k}$ tales que:

	\[\abs{y_{k+1}-y_k} < M_k \text{,\ } ∀x∈[a,b]\]
	\[\text{y que\ } \sum_k M_k < ∞ \text{, entonces } ∃y \text{ tal que:}\]
	\[y_n \rightarrow y \text{ uniformemente}\]

	Vamos a encontrar dichos $M_k$:

	\begin{lemma}
		\[\abs{y_{k+1} - y_k} ≤ \frac{C(y_a)}{L}\frac{(L(x-a))^{k+1}}{(k+1)!}\]
	\end{lemma}
	\begin{proof}
		Vamos a proceder a demostrar el lema usando inducción, así que vamos a ver que es cierto para $k=0$:

		\[\abs{y_1-y_0}=\abs{\int_a^x f(s,y_a) ds} ≤ \int_a^x \abs{f(s,y_a)} ds ≤\]
		\[≤ \md{f(x,y_n)}_{L^∞[a,b]} (x-a) = C(y_n) (x-a)\]

		Ahora supondremos que el lema es cierto para $k-1$ para poder demostrarlo para $k$:

		\[\abs{y_{n+1}-y_k} = \abs{\int_a^x f(s,y_k(s)) - \int_a^x f(s,y_{k-1}(s))} ds ≤\]
		\[≤ \int_a^x \abs{f(s,y_k(s)) - f(s,y_{k-1}(s))} ds \underbrace{≤}_{\mathclap{f \text{ Lipschitz en 2ª variable}}} L \int_a^x \abs{y_k(s) - y_{k-1}(s)} ds ≤\]
		\[\underbrace{≤}_{\text{hip. inducción}} L \int_a^x \frac{C(y_a)}{L} \frac{(L(s-a))^k}{k!} ds ≤ C(y_a)\frac{L^k}{k!}\int_a^x (s-a)^k ds ≤\]
		\[≤ C(y_a) \frac{L^k}{(k+1) k!}(x-a)^{k+1}\]
	\end{proof}

	Volviendo a la demostración de \ref{TeoremaPicard} recordemos que queríamos encontrar los $M_k$ necesarios para ver la convergencia de nuestras $y_k$. Gracias al lema que acabamos de demostrar y al test M de Weierstrass \ref{TestMWeierstrass} podemos afirmar:

	\[\abs{y_{k+1}-y_k} ≤ \frac{C(y_a)}{L} ≤ \frac{C(y_a)}{L}\frac{(L(x-a))^{k+1}}{(k+1)!} = M_k\]

	Los siguiente es verificar la segunda hipótesis de \ref{TestMWeierstrass}:

	\[M_k = \frac{C(y_a)}{L}\frac{[L(x-a)]^{k+1}}{(k+1)!} \underbrace{=}_{s=L(x-a)} \frac{C(y_a)}{L}\frac{s^{k+1}}{(k+1)!}\]

	Sabiendo que $e^x = \sum_{k=0}^∞ \frac{x^k}{k!}$:

	\[\sum_{k=0}^∞ M_k = \frac{-C(y_a)}{L} (1 - \frac{C(y_a)}{L} \sum_{k=0}^∞ \frac{s^k}{k!}) = \frac{C(y_a)}{L} (e^s - 1) < ∞\]

	Luego aplicando el test M de Weierstrass \ref{TestMWeierstrass}:
	\[∃y \text{ tal que: } y_n\rightarrow y \text{ uniformemente en} [a,b]\]

	\begin{enumerate}
		\setcounter{enumi}{3}
		\item Por tanto hemos probado que $∃y$ tal que $y_n \rightarrow y$ uniformemente, y por lo tanto $y∈C([a,c])$ (sabemos que la convergencia uniforme de una sucesión de funciones continuas hace que la función a la que se converge sea continua).

		\item
			\[y_{n+1} = y_a + \int_a^x f(s,y_n(s))ds \underbrace{\longrightarrow}_{n \rightarrow ∞} y = y_a + \int_a^x f(s,y(s))ds\]
			Como $f∈C([a,b])$ y $y_n$ son continuas, está claro que $y$ es continua. De modo que \textbf{\underline{hemos demostrado que existe solución}} $y∈C([a,b])$.
	\end{enumerate}

	Lo que \underline{nos queda es demostrar la unicidad} de la solución:

	Supongamos que existen dos soluciones distintas $y,\tilde{y} ∈ C([a,b])$:
	\[y(x) = y_a + \int_a^x f(s,y(s))ds\]
	\[\tilde{y}(x) = y_a + \int_a^x f(s,\tilde{y}(s))ds\]
	\[\abs{y(x) - \tilde{y}(x)} = \int_a^x \abs{f(s,y(s)) - f(s,\tilde{y}(s))} ds ≤\]
	\[≤ L \int_a^x \abs{y(s) - \tilde{y}(s)} ds \]

	Definimos $g(x)=\int_a^x \abs{y(s) - \tilde{y}(s)} ds$:

	\begin{equation}
		\label{eqDemPicard}
		g'(x) = Lg(x); g'(x)e^{-L(x-a)} ≤ Lg(x)e^{-L(x-a)}
	\end{equation}

	Seguimos:
	\[\left( g(x)e^{-L(x-a)} \right)^` = g'(x) e^{-L(x-a)} - Lg(x) e^{-L(x-a)}\]
	\[g'(x) e^{-L(x-a)} = \left( g(x)e^{-L(x-a)} \right)^` + Lg(x) e^{-L(x-a)}\]

	Usando \ref{eqDemPicard}:
	\[\left( g(x)e^{-L(x-a)} \right)^` + Lg(x) e^{-L(x-a)} ≤ Lg(x) e^{-L(x-a)}\]
	\[\left( g(x)e^{-L(x-a)} \right)^` ≤ 0\]

	Si integramos a ambos lados entre $a$ y $X$:
	\[g(X) e^{-L(x-a)} ≤ g(a) = 0 \text{ por como hemos definido } g\]

	Por tanto:
	\[g(X) = \int_a^X \abs{\tilde{y}(x) - y(x)} dx = 0 \implies \tilde{y}=y\]

	Y hemos demostrado la unicidad de la solución.
\end{proof}

Lo siguiente que cabe preguntarse es cómo varía la solución de un problema de valor inicial (PVI) cuando variamos el dato inicial. Y la respuesta es que al variarlo muy poco, la solución también se altera muy poco.

\begin{theorem}[Continuidad con respecto al dato inicial]
	Sea $y' = f(x,y(x))$ un problema de valor inicial (PVI) con dos datos iniciales distintos $y_a, \tilde{y}_a$, con $f$ Lipschitz en la segunda variable y continua en $Ω=[a,b]×ℝ^d$:
	\[\md{y(x)-\tilde{y}(x)}_{L^∞[a,b]} ≤ \abs{y_a - \tilde{y}_a} e^{L(b-a)}\]
\end{theorem}
\begin{proof}
	Tomando como soluciones las que se sacan con la iterada de Picard:

	\[y(x)-\tilde{y}(x)=y_a-\tilde{y}_a + \int_a^x (f(s,y(s)) - f(s,\tilde{y}(s)))ds\]
	\[\abs{y - \tilde{y}} ≤ \abs{y_a - \tilde{y}_a} + \int_a^x \abs{f(s,y(s)) - f(s,\tilde{y}(s))} ds ≤\]
	\[≤ \underbrace{ \abs{y_a - \tilde{y}_a} + L\int_a^b \abs{y(s)-\tilde{y}(s)} ds}_{g(x)} \]

	Por tanto $g'(x) = L\abs{y(x) - \tilde{y}(x)}$. Si sobre esta última igualdad realizamos el procedimiento del factor integrante que hemos llevado a cabo al demostrar la unicidad en el teorema de Picard \ref{TeoremaPicard} llegamos a:

	\[g(x)e^{-L(x-a)} ≤ \abs{\tilde{y}_a - y_a} \implies g(x) ≤ \abs{\tilde{y}_a - y_a}e^{L(x-a)}\]

	Por último:
	\[\abs{y_(x) - \tilde{y}(x)} ≤ g(x)\]
	\[y(x) - \tilde{y}(x) ≤ \abs{\tilde{y}_a - y_a}e^{L(x-a)}\]
	\[\md{y - \tilde{y}}_{L^∞[a,b]} ≤ \abs{y_a - \tilde{y}_a} e^{L(b-a)}\]
\end{proof}

Vamos ahora a ver ejemplos donde aplicar el teorema de Picard \ref{TeoremaPicard}:

\begin{example}
	Para el problema de valor inicial (PVI) $y'(x)=\sqrt{x}$, $y(0)=0$; tenemos $f(x,y(x))=\sqrt(y)$. Esta $f$ no es Lipschitz y por tanto no cumple las hipótesis del teorema de Picard \ref{TeoremaPicard}, de hecho existen dos soluciones distintas:
	\[y_1(x) = 0\]
	\[y_2(x) = x^2\]
\end{example}


\section{Algunos ejemplos de métodos numéricos para PVI.}
En este curso querremos encontrar soluciones a problemas de valor inicial (PVI) del tipo:

\[y'(x) = f(x,y(x)) \text{, } x∈[a,b]\]
\[y(a)=y_a\]

Lo que queremos conocer es cómo se comporta y(x) sin saber encontrar su solución. Para ello podemos usar las iteradas de Picard:

\[y_{n-1}(x) = y_a + \int_a^x f(s,y(s)) ds\]
\[y_0(x) = y_a\]

También cabe recalcar que para los métodos numéricos usaremos versiones discretizadas de las funciones, es decir, el conjunto de valores que tomará una función $y(x)$ será sobre una serie de valores ${x_n}_{n=0}^N$ dentro del intervalo $[a,b]$, donde $x_0=a$ y $x_N=b$.

En este curso la distancia entre cada muestra será igual $x_{n+1}-x_n=h=\frac{b-a}{N}$.

A continuación mencionaremos algunos métodos numéricos:
\begin{itemize}
	\item \textbf{Método de Euler}
	\[y'(x) = f(x,y(x))\]
	\[y'(x_n) = f(x_n,y(x_n))\]
	Pero dado que $x$ toma valores discretos:
	\[\frac{y(x_{n+1}) - y(x_n)}{h} \implies \frac{y_{n+1} - y_n}{h} = f(x_n,y(x))\]
	Así la fórmula de recurrencia del método de Euler queda como:
	\[y_{n+1} = y_n + h·f(x,y_n)\]
	\[y_0=y_a\]

	\item \textbf{Desarrollo en serie de Taylor}
	\[y(x_{n+1}) = y(x_n+h) = y(x_n) + h·y'(x_n) + R_n =\]
	\[= y(x_n) + h·f(x_n,y(x_n)) + R_n\]
\end{itemize}


\end{document}

