\documentclass{apuntes}

\title{MNEDO}
\author{Pedro Valero \& El Malvado}
\date{15/15 C1}
% Paquetes adicionales
\usepackage{tikztools}
\usepackage{fastbuild}
\usepackage{tikz-3dplot}
\usepackage{fancysprefs}

\bibliographystyle{plainnat}

\usetikzlibrary{arrows}
% --------------------

\precompileTikz

\begin{document}
\pagestyle{plain}
\maketitle

\tableofcontents


\chapter{Problemas de valor inicial. Preliminares}

\section{Problemas de valor inicial. (PVI)}

A lo largo de este curso estudiaremos sistemas del tipo:
\[y'(x)=f(x,y(x))\]
siendo $y(x)$ una función del tipo
\[\appl{y}{ℝ}{ℝ^d}\]
\[y(x)=(y_1(x), …, y_d(x))\]

Por tanto $f$ quedará definida como:
\[\appl{f}{ℝ×ℝ}{ℝ^d} \]
\[f(x,y) = (x,y(x))\]

Como notación muchas veces consideraremos $y(x)=y$, del mismo modo que $y(a)=y_a$ con $a$ un valor dado.

\begin{remark}
	En el curso trabajaremos con funciones $f∈C\left( [a,b] × ℝ^d \right)$
\end{remark}


\begin{defn}[Función Lipschitz]
	Una función $f$ será Lipschitz en la segunda variable existe una constante $L$ tal que:
	\[\md{f(x,y_1) - f(x,y_2)} ≤ L\md{y_1-y_2}\]
	\[∀x ∈ [a,b], ∀y_1,y_2 ∈ℝ^d\]
\end{defn}

Los problemas de valor inicial (PVI) que vamos a estudiar son de la forma:
\[
	\begin{cases}
		y'(x)=f(x,y(x)) & x∈[a,b]\\
		y(a)=y_a
	\end{cases}
\]

Vamos a ver en qué situaciones este tipo de problemas tienen solución única:

\begin{theorem}[Teorema de Picard]
	\label{TeoremaPicard}
	Sea $Ω=[a,b]×ℝ^d$, si $f$ es continua en $Ω$ y Lipschitz en la segunda variable; entonces el PVI tiene una única solución $y∈C'\big([a,b] × ℝ^d\big)$
\end{theorem}

\begin{proof}
	\textbf{Vamos a demostrar primero la existencia}:
	Para ello haremos uso de las iteradas de Picard.

	\[y(x)=y_a+\int_a^x f(s,y(s))ds\]

	Si definimos la aplicación $T_y = y_a + \int_a^x f(s,y(s))ds$, querríamos encontrar un punto fijo de dicha aplicación.

	\begin{enumerate}
		\item \textbf{Construimos la secuencia de funciones $\{ y_n(x) \}_{n=0}^∞$}:
		\[y_{n+1}(x) = y_a + \int_a^x f(s,y_n(s)) ds\]
		\[n=0,1,… \text{\ y \ } y_0(x)=y_a\]
		Cada función $y_n$ es continua (es una constante más una función continua).

		\item \textbf{Debemos probar que $∃y ∈ C([a,b])$ tal que}:
		\[y_n \longrightarrow y \text{\ uniformemente}\]
		Que es equivalente a decir que $\lim_{n \to ∞} \md{y_n - y}_{L^∞[a,b]} = 0$. Donde $\md{f}_{L^∞[a,b]} = \max_{x ∈ [a,b]}\md{f(x)}$, denota la norma infinito de $f$.
	\end{enumerate}

	Antes de continuar vamos a recordar el Test M de Weierstrass:

	\begin{theorem}[Test M de Weierstrass]
		\label{TestMWeierstrass}
		Sea $\{f_n\}_{n=0}^∞$ una sucesión de funciones definidas en un dominio $Ω$ con valores en $ℝ$, si existe una sucesión de números $M_n$ tal que:
		\begin{enumerate}
			\item $\abs{f_n} < M_n \text{,\ } ∀x∈Ω$
			\item $\sum_{n=0}^∞ M_n < ∞$
		\end{enumerate}

		Entonces existe una función $g$ tal que:
		\[g_n(x) = \sum_{n=1}^N f_n\]
		\[g_n \xrightarrow[n \rightarrow ∞]{} g \text{,\ uniformemente en\ } x\]
	\end{theorem}

	¿Cómo vamos a usar esto para conseguir ver que $y_n \rightarrow y$?.\\
	Primero vamos a reescribir $y_n$ como (sabiendo que tomamos $y_a=y_0$) una suma telescópica:

	\[y_{n+1} = y_a + \sum_{k=0}^n y_{k+1} - y_k\]
	Para el caso $n=1$: $y_2 = y_a + (y_1 - y_0) + (y_2 - y_1) = y_2$.

	Ahora vamos a aplicar el test M de Weierstrass:

	Las $f_k$ que aparecen en el enunciado de \ref{TestMWeierstrass} serán $(y_{k+1} - y_k)$, si demuestro que $∃{M_k}$ tales que:

	\[\abs{y_{k+1}-y_k} < M_k \text{,\ } ∀x∈[a,b]\]
	\[\text{y que\ } \sum_k M_k < ∞ \text{, entonces } ∃y \text{ tal que:}\]
	\[y_n \rightarrow y \text{ uniformemente}\]

	Vamos a encontrar dichos $M_k$:

	\begin{lemma}
		\[\abs{y_{k+1} - y_k} ≤ \frac{C(y_a)}{L}\frac{(L(x-a))^{k+1}}{(k+1)!}\]
	\end{lemma}
	\begin{proof}
		Vamos a proceder a demostrar el lema usando inducción, así que vamos a ver que es cierto para $k=0$:

		\[\abs{y_1-y_0}=\abs{\int_a^x f(s,y_a) ds} ≤ \int_a^x \abs{f(s,y_a)} ds ≤\]
		\[≤ \md{f(x,y_n)}_{L^∞[a,b]} (x-a) = C(y_n) (x-a)\]

		Ahora supondremos que el lema es cierto para $k-1$ para poder demostrarlo para $k$:

		\[\abs{y_{k+1}-y_k} = \abs{\int_a^x f(s,y_k(s)) - \int_a^x f(s,y_{k-1}(s))} ds ≤\]
		\[≤ \int_a^x \abs{f(s,y_k(s)) - f(s,y_{k-1}(s))} ds \underbrace{≤}_{\mathclap{f \text{ Lipschitz en 2ª variable}}} L \int_a^x \abs{y_k(s) - y_{k-1}(s)} ds ≤\]
		\[\underbrace{≤}_{\text{hip. inducción}} L \int_a^x \frac{C(y_a)}{L} \frac{(L(s-a))^k}{k!} ds ≤ C(y_a)\frac{L^k}{k!}\int_a^x (s-a)^k ds ≤\]
		\[≤ C(y_a) \frac{L^k}{(k+1) k!}(x-a)^{k+1}\]
	\end{proof}

	Volviendo a la demostración de \ref{TeoremaPicard} recordemos que queríamos encontrar los $M_k$ necesarios para ver la convergencia de nuestras $y_k$. Gracias al lema que acabamos de demostrar y al test M de Weierstrass \ref{TestMWeierstrass} podemos afirmar:

	\[\abs{y_{k+1}-y_k} ≤ \frac{C(y_a)}{L} ≤ \frac{C(y_a)}{L}\frac{(L(x-a))^{k+1}}{(k+1)!} = M_k\]

	Los siguiente es verificar la segunda hipótesis de \ref{TestMWeierstrass}:

	\[M_k = \frac{C(y_a)}{L}\frac{[L(x-a)]^{k+1}}{(k+1)!} \underbrace{=}_{s=L(x-a)} \frac{C(y_a)}{L}\frac{s^{k+1}}{(k+1)!}\]

	Sabiendo que $e^x = \sum_{k=0}^∞ \frac{x^k}{k!}$:

	\[\sum_{k=0}^∞ M_k = \frac{-C(y_a)}{L} (1 - \frac{C(y_a)}{L} \sum_{k=0}^∞ \frac{s^k}{k!}) = \frac{C(y_a)}{L} (e^s - 1) < ∞\]

	Luego aplicando el test M de Weierstrass \ref{TestMWeierstrass}:
	\[∃y \text{ tal que: } y_n\rightarrow y \text{ uniformemente en} [a,b]\]

	\begin{enumerate}
		\setcounter{enumi}{3}
		\item Por tanto hemos probado que $∃y$ tal que $y_n \rightarrow y$ uniformemente, y por lo tanto $y∈C([a,c])$ (sabemos que la convergencia uniforme de una sucesión de funciones continuas hace que la función a la que se converge sea continua).

		\item
			\[y_{n+1} = y_a + \int_a^x f(s,y_n(s))ds \underbrace{\longrightarrow}_{n \rightarrow ∞} y = y_a + \int_a^x f(s,y(s))ds\]
			Como $f∈C([a,b])$ y $y_n$ son continuas, está claro que $y$ es continua. De modo que \textbf{\underline{hemos demostrado que existe solución}} $y∈C([a,b])$.
	\end{enumerate}

	Lo que \underline{nos queda es demostrar la unicidad} de la solución:

	Supongamos que existen dos soluciones distintas $y,\tilde{y} ∈ C([a,b])$:
	\[y(x) = y_a + \int_a^x f(s,y(s))ds\]
	\[\tilde{y}(x) = y_a + \int_a^x f(s,\tilde{y}(s))ds\]
	\[\abs{y(x) - \tilde{y}(x)} = \int_a^x \abs{f(s,y(s)) - f(s,\tilde{y}(s))} ds ≤\]
	\[≤ L \int_a^x \abs{y(s) - \tilde{y}(s)} ds \]

	Definimos $g(x)=\int_a^x \abs{y(s) - \tilde{y}(s)} ds$:

	\begin{equation}
		\label{eqDemPicard}
		g'(x) = |y(x) - \tilde{y}(x)| \leq Lg(x); \ g'(x)e^{-L(x-a)} \leq Lg(x)e^{-L(x-a)}
	\end{equation}

	Seguimos:
	\[\left( g(x)e^{-L(x-a)} \right)^{'} = g'(x) e^{-L(x-a)} - Lg(x) e^{-L(x-a)}\]
	\[g'(x) e^{-L(x-a)} = \left( g(x)e^{-L(x-a)} \right)^{'} + Lg(x) e^{-L(x-a)}\]

	Usando \ref{eqDemPicard}:
	\[\left( g(x)e^{-L(x-a)} \right)^{'} + Lg(x) e^{-L(x-a)} ≤ Lg(x) e^{-L(x-a)}\]
	\[\left( g(x)e^{-L(x-a)} \right)^{'} ≤ 0\]

	Si integramos a ambos lados entre $a$ y $X$:
	\[g(X) e^{-L(x-a)} ≤ g(a) = 0 \text{ por como hemos definido } g\]

	Por tanto:
	\[g(X) = \int_a^X \abs{\tilde{y}(x) - y(x)} dx = 0 \implies \tilde{y}=y\]

	Y hemos demostrado la unicidad de la solución.
\end{proof}

Lo siguiente que cabe preguntarse es cómo varía la solución de un problema de valor inicial (PVI) cuando variamos el dato inicial. Y la respuesta es que al variarlo muy poco, la solución también se altera muy poco.

\begin{theorem}[Continuidad con respecto al dato inicial]
	Sea $y' = f(x,y(x))$ un problema de valor inicial (PVI) con dos datos iniciales distintos $y_a, \tilde{y}_a$, con $f$ Lipschitz en la segunda variable y continua en $Ω=[a,b]×ℝ^d$:
	\[\md{y(x)-\tilde{y}(x)}_{L^∞[a,b]} ≤ \abs{y_a - \tilde{y}_a} e^{L(b-a)}\]
\end{theorem}
\begin{proof}
	Tomando como soluciones las que se sacan con la iterada de Picard:

	\[y(x)-\tilde{y}(x)=y_a-\tilde{y}_a + \int_a^x (f(s,y(s)) - f(s,\tilde{y}(s)))ds\]
	\[\abs{y - \tilde{y}} ≤ \abs{y_a - \tilde{y}_a} + \int_a^x \abs{f(s,y(s)) - f(s,\tilde{y}(s))} ds ≤\]
	\[≤ \underbrace{ \abs{y_a - \tilde{y}_a} + L\int_a^b \abs{y(s)-\tilde{y}(s)} ds}_{g(x)} \]

	Por tanto $g'(x) = L\abs{y(x) - \tilde{y}(x)}$. Si sobre esta última igualdad realizamos el procedimiento del factor integrante que hemos llevado a cabo al demostrar la unicidad en el teorema de Picard \ref{TeoremaPicard} llegamos a:

	\[g(x)e^{-L(x-a)} ≤ \abs{\tilde{y}_a - y_a} \implies g(x) ≤ \abs{\tilde{y}_a - y_a}e^{L(x-a)}\]

	Por último:
	\[\abs{y_(x) - \tilde{y}(x)} ≤ g(x)\]
	\[y(x) - \tilde{y}(x) ≤ \abs{\tilde{y}_a - y_a}e^{L(x-a)}\]
	\[\md{y - \tilde{y}}_{L^∞[a,b]} ≤ \abs{y_a - \tilde{y}_a} e^{L(b-a)}\]
\end{proof}

Vamos ahora a ver ejemplos donde aplicar el teorema de Picard \ref{TeoremaPicard}:

\begin{example}
	Para el problema de valor inicial (PVI) $y'(x)=\sqrt{x}$, $y(0)=0$; tenemos $f(x,y(x))=\sqrt(y)$. Esta $f$ no es Lipschitz y por tanto no cumple las hipótesis del teorema de Picard \ref{TeoremaPicard}, de hecho existen dos soluciones distintas:
	\[y_1(x) = 0\]
	\[y_2(x) = x^2\]
\end{example}


\section{Algunos ejemplos de métodos numéricos para PVI.}
En este curso querremos encontrar soluciones a problemas de valor inicial (PVI) del tipo:

\[y'(x) = f(x,y(x)) \text{, } x∈[a,b]\]
\[y(a)=y_a\]

Lo que queremos conocer es cómo se comporta y(x) sin saber encontrar su solución. Para ello podemos usar las iteradas de Picard:

\[y_{n-1}(x) = y_a + \int_a^x f(s,y(s)) ds\]
\[y_0(x) = y_a\]

También cabe recalcar que para los métodos numéricos usaremos versiones discretizadas de las funciones, es decir, el conjunto de valores que tomará una función $y(x)$ será sobre una serie de valores ${x_n}_{n=0}^N$ dentro del intervalo $[a,b]$, donde $x_0=a$ y $x_N=b$.

En este curso la distancia entre cada muestra será igual $x_{n+1}-x_n=h=\frac{b-a}{N}$.

A continuación mencionaremos algunos métodos numéricos:
\begin{itemize}
	\item \textbf{Método de Euler}
	\[y'(x) = f(x,y(x))\]
	\[y'(x_n) = f(x_n,y(x_n))\]
	Pero dado que $x$ toma valores discretos:
	\[\frac{y(x_{n+1}) - y(x_n)}{h} \implies \frac{y_{n+1} - y_n}{h} = f(x_n,y(x))\]
	Así la fórmula de recurrencia del método de Euler queda como:
	\[y_{n+1} = y_n + h·f(x,y_n)\]
	\[y_0=y_a\]

	\item \textbf{Desarrollo en serie de Taylor}
	\[y(x_{n+1}) = y(x_n+h) = y(x_n) + h·y'(x_n) + R_n =\]
	\[= y(x_n) + h·f(x_n,y(x_n)) + R_n\]

	\item \textbf{Método leap-frog}
	\[y(x_{n+2} - y(x_n) = \int_{x_n}^{x_{n+2}} y'(x) = \int_{x_n}^{x_{n+2}} f(x,y(x))\dif x \approx (x_{n+1}-x_n)f(x_{n+1},y(x_{n+1})\]
	donde la aproximación se ha realizado empleando la regla del punto medio.

	Finalmente nos queda:
	\[y(x_{n+2} - y(x_n) = 2h f(x_{n+1},y(x_{n+1}))\]
\end{itemize}

\begin{defn}[Método numérico de $k$ pasos]
Se dice que un método numérico es de $k$ pasos si para conocer $y_{n+k}$ necesitamos conocer los $k$ valores: $y_{n}, y_{n+1},... y_{n+k-1}$
\end{defn}

\begin{defn}[Método numérico convergente]
Se dice que un método numérico es convergente si para todo problema de valor inicial se tiene que:
\[\lim_{N \to \infty} \max_{0 \leq n \leq k-1}\norm{y(x_n)-y_n)}= 0\implies \max_{k \leq n \leq N} \norm{y(x_n)-y_n}=0\]

Evidentemente no nos referimos a cualquier problema imaginable sino a todos aquellos problemas de valor inicial en los que la función $f$ sea continua y Lipschitz en la segunda variable, pues esos son lso únicos problmeas con los que hemos trabajado hasta ahora.
\end{defn}

\begin{theorem}[]
Sea $y(x)$ la solución de un problema de valor inicial y sean $\{y_n\}$ los valores que vamos aproximando en cada iteración del método de Euler, se cumple que $\{y_n\}$ converge a $y(x)$
\end{theorem}
\begin{proof}
Al trabajar con este método obtendremos una serie de funciones de la forma
\[y_{n+1} = y_n + h f(x_n, y_b)\]

Para empezar vamos a definir el residuo que no es más que la diferencia entre el valor aproximado $y_n$ obtenido al aplicar el método en cuestión y el valor real $y(x_n)$. Así, podemos escribir:
\[y(x_{n+1})=y(x_n)+hf(x_n,y(x_n))+R_n \text{ siendo } R_n \text{ el residuo }\]

Despejando podemos ver que
\[R_n = \underbrace{y(x_{n+1})-y(x_n)}_{hy'}-h\underbrace{f(x_n,y(x_n))}_{y'} = \frac{h^2}{2}y''(ε)\]


Según la definición de convergencia, para demostrar que la sucesión converge basta con ver que:
\[|y(x_n)-y_n | = 0 \ \implies \lim_{N \to \infty} \max_n |y(x_n)-y_n| = 0\]


\begin{itemize}
\item Basándonos en la condición de Lipschitz:
\[|y(x_{n+1})-y_{n+1}| \leq (1+Lh)|y(x_n)-y_n)+K \text{ siendo } e_{n+1} \leq (1+Lh)e_n + K \text{ con } K = \max_n |R_n|\]

\begin{proof}
\[e_n = |y(x_n)-y_n| \implies e_{n+1} \leq (1+Lh)e_n+K \implies e_n \leq (1+Lh)^n e_0 + K \frac{1-(1+hL)^n}{1-(1+kL)}\]
que simplificando nos queda:
\[e_n \leq (1+Lh)^n e_0 + \frac{K}{h}\frac{(1+hL)^n-1}{L}\]

Esta fórmula puede ser deducida de forma sencilla estudiando como se comporta el error en cada iteración, con lo que veremos que no es más que una sucesión geométrica. Otra opción posible es hacer la demostración de la misma por inducción.

Ambos procedimientos son muy sencillos y se dejan como ejercicio para el lector.
\end{proof}

Gracias a esto tenemos demostrada la \textbf{estabilidad} pues
\[|y(x_n)-y_n| \leq (1+hL)^n |y(x_0)-y_0| + \frac{(1+hL)^{n+1}-1}{L}\cdot \max_n \frac{|R_n|}{h}\]

Por otro lado tenemos que:
\[\lim_{h \to 0} \frac{|R_n|}{h} = 0\]
con lo que vemos que el residuo lo tenemos controlado
\item \textbf{ Tras esto demostramos por inducción que $C_n$ satisface:}
\[C_n \leq (1+Lh)^n C_0 + K\frac{(1+Lh)^n-1}{Lh}\]
% Jorge, esta parte en teoría es repaso de la clase que tienes que completar tu. Mételo como puedas aquí.

\item Ahora vamos a demostrar que
\[|y(x_n) - y_n| \leq (1+Lh)^n |y(x_n)-y_0|+K\frac{(1+Lh)^n-1}{Lh}\]

Para ello basta con ver que
\[\max_n |y(x_n)-y_n| \leq (1+Lh)^N|y(x_n)-y_0| + K\frac{(1+Lh)^n-1}{Lh}\]

Por otro lado tenemos que
\[h = \frac{b-a}{N}= \left( 1+\frac{L(b-a)}{N}\right)^N|y(x_n)-y_0| + K \frac{\left(1+\frac{L(b-a)}{N} \right)^N-1}{Lh}\]

Además sabemos que
\[\lim_{N \to \infty} \max_n |y(x_n)-y_0| \leq e^{L(b-a)} |y(x_n)-y_0| + \frac{e^{L(b-a)}-1}{L}\lim_{h \to 0} \frac{K}{h}\]

\item \textbf{Control del residuo}

El residuo puede escribirse como:
\[R_n = y(x_{n+1})-y(x_n)-hy'(x_n) = \frac{h^2}{2}y''(ε)\]

De aquí podemos ver que
\[\max_n \frac{|R_n|}{h} \leq C h\]
lo cual nos garantiza que el residuo es 0.

Sin embargo esto no es tan sencillo, puesto que para poder acotar el máximo como hemos hecho nos apoyamos en que la derivada segunda, $y''$, está acotada módulo infinito por $C$. Pero nosotros no sabemos nada acerca de nuestra solución por que no lo podemos realizar este tipo de asunciones.

Pero esto puede mejorarse. Sabemos que
\[\frac{R_n}{h} = \frac{y(x_{n+1})-y(x_n)}{h} = y'(x_n) = y'(ε)-y'(x_n) \text{ siendo } x_n \leq ε \leq x_{n+1}\]
A partir de aquí vemos fácilmente que
\[\lim_{h\to 0} \frac{|R_n|}{h} = 0\]

Esto es sencillo de ver puesto que la derivada segunda de $y$ está acotada.
%Aquí faltan cositas Jorge
\end{itemize}
\end{proof}

\subsection{Orden del método de Euler}
\begin{defn}[Método numérico de orden p]
Se dice que un método numérico es convergente de orden $p$ si $p$ es el mayor de los enteros tales que $f \in C^p$ y
\[\max_{0 \leq n \leq k-1}\norm{y(x_n)-y_n)}= O(n)\implies \max_{k \leq n \leq N} \norm{y(x_n)-y_n}=O(h^p)\]
\end{defn}

Recordemos las definiciones de 'O' y 'o'

\begin{defn}[O grande]
Decimos que $f= O(g)$ en $x \to x_0$ si
\[\exists k,r > 0 \tq \forall x_0 < |x-x_0|< r \text{ se tiene } |f(x)| \leq k |g(x)|\]
\end{defn}

\begin{defn}[O pequeña]
Decimos que $f=o(g)$ en $x \to x_0$ si
\[\exists r > 0 \tq 0 < |x-x_0|<r \text{tal que, donde } g \text{ no se anule, tenemos} \lim_{x\to x_0} \frac{f(x)}{g(x)}=0\]
\end{defn}

Una vez vistas estas definiciones, supongamos que tenemos $f \in C [\real \times \real^d]$. En este caso ya hemos visto que el orden de convergencia del método de Euler es al menos 1.

Para poder confirmar que el método es de orden 1, ahora sólo nos queda comprobar que el orden no puede ser 2. Para comprobarlo nos basta con ver un ejemplo en el que no se cumpla:
\begin{example}
Tomamos el caso $y'(x)=x, \ x \in [0,1]$ con valor inicial $y(0)=0$. En este caso sabemos que la solución es $y(x)=\frac{1}{2}x^2$

Vamos a ver que ocurre si aplicamos el método de Euler. En este caso, tendríamos la relación de recurrencia:
\[y_{n+1} = y_n + h x_n\]
con lo que obtendríamos la sucesión de funciones:
\[y_1 = 0 \; ; \; y_2 = 0 + hx_1 =h^2 \; ; \; y_3 = h^2+2h^2=h^2(1+2+3) \; ; \; ...\]

De donde podemos deducir la fórmula general para $y_n$:
\[y_n = h^2\sum_{i=1}^{n-1} i = h^2\frac{n(n-1)}{2}=\frac{n^2+n}{2} =h^2\frac{x_n^2-x_n}{2}\]

Así podemos ver que:
\[|y(x_n)-y_n| = \left|\frac{x_n^2}{2}-\frac{x_n^2}{2}+\frac{x_nh}{2} \right| = \frac{x_nh}{2} \implies \frac{1}{h}|y(x_n)-y_n|  = \frac{x_n}{2}\]
\end{example}


\textbf{Esperemos que hayáis disfrutado hasta aquí porque nos disponemos a entrar en la parte aburrida del curso.}


\chapter{Algunos Métodos Numéricos para Ecuaciones Diferenciales Ordinarias}
\section{Métodos de Taylor}
Como siempre, tenemos que $y'(x)=f(x,y(x))$ y debemos buscar una relación entre $y(x_{n+1})$ y los $y(x_m)$ con $m < n+1$.

La idea del método de Taylor es emplear la relación:
\[y(x_{n+1})=y(x_n)+hf(x_n,y(x_n))+\frac{h^2}{2}y''(x_n)+R_n\]

Pero tenemos el problema de que no conocemos la derivada segudna de $y$ puesto que no conocemos $y$ (es precisamente eso lo que queremos ser capaces de estimar). Pero lo que podemos hacer es definir $y''$ apoyándonos en la regla de la cadena, con lo que obtenemos:
\[y''(x)=(\partial x f)(x,y(x))+(\partial y f)(x,y(x))y'(x)=f_x(x,y(x))+f_y(x,y(x))f(x,y(x)\]

con lo que el método de Taylor nos queda, finalmente:
\[y_{n+1} = y_n + hf(x_n,y_n)+\frac{h^2}{2}\left(f_x(x_n,y_n)+f(x_n, y_n)f_y(x_n,y_n)\right)\]

\begin{remark}
En la práctica estos métodos no se utiliza nunca puesto que no son demasido eficaces.
\end{remark}

Al hablar de estos métodos en plural hacemos referencia a las diferentes estimaciones posibles, pues podemos usar un desarrollo de Taylor de grado superior, lo que añade complejidad a la fórmula pero aproxima mejor el resultado, pues hace el residuo cada vez más pequeño.

\section{Métodos basados en cuadraturas}
Ya hemos visto un método de este tipo, \textbf{el método de Euler}, en el que se define:
\[y(x_{n+1}) -y(x_n) = \int_{x_n}^{x_{n+1}} y'(x)\dif x = \int_{x_n}^{x_{n+1}} f(x_n,y(x_n)) \dif x \approx f(x_n,y(x_n))h \implies \]
\[\implies y(x_{n+1}) =y(x_n)+f(x_n,y(x_n))h\]

\section{Método del trapecio}
En esta ocasión escribimos:
\[y(x_{n+1})=y(x_n) + \int_{x_n}^{x_{n+1}} f(x,y(x)\dif x \approx y(x_n)+ \frac{h}{2}\left( f(x_n,y(x_n))+f(x_{n+1},y(x_{n+1})) \right)\]

Este método presenta una difernecia radical respecto a los demás, pues el objetivo del método es calcular $y_{x+1}$ en lugar de $y_n$.

Una forma de ver este método es definiendo el operador $T$ de la siguiente forma:
\[T(y) = y_n + \frac{h}{2} \left(f(x_n,y_n)+f(x_{n+1}, y) \right)\]
con lo que el problema a resolver se reduce a encontrar un punto fijo de este operador, es decir, una función $y$ tal que
\[T(y)=y\]

Podemos ver bajo qué condiciones la función es Lipchitz y, y por tanto tiene solución. Para ello vemos que
\[|T(y) - T(\tilde{y})| \leq \frac{h}{2} \left( |f(x_{n+1}, \tilde{y}) - f(x_{n+1},\tilde{y})|\right) \leq \frac{Lh}{2}\left( |y-\tilde{y}|\right)\]

Por tanto, la recurrencia tendrá solución siempre que $\frac{Lh}{2} \leq 1$

\section{Métodos basados en cuadraturas}
\subsection{Método Predictor corrector Euler-trapecio}
En este método partimos de la fórmula del trapecio
\[y_{n+1} = y_n + \frac{h}{2} \left(f(x_n,y_n)+f(x_{n+1}, y_{n+1}) \right)\]

Lo que vamos a hacer ahora es aproximar el $y_{n+1}$, empleado a la derecha de la fórmula, por el método de Euler:
\[y^*_{n+1} = y_n + h f(x_n,y_n)\]

Una vez hemos empleado el método de Euler para hacer una predicción de $y_{n+1}$ empleamos el método del trapecio para corregir el valor.

\subsection{Método de Euler modificado}
Este método emplea la relación:
\[y(x_{n+1}) = y(x_n) + \int_{x_n}^{x_{n+1}} f(x,y(x))\dif x \approx y(x_n) + h f(x+\frac{h}{2},y(x+\frac{h}{2}))\]

El problema que tenemos ahora es que necesitamos aproximar $y(x+\frac{h}{2})$, cosa que haremos mediante la ecuación:
\[y(x+\frac{h}{2}) = y(x_n)+\frac{h}{2}f(x_n, y(x_n))\]

Este método es realmente importante pues es el padre de una serie de métodos denominados \textbf{Métodos de Runge-Kutta}, que veremos a continuación

\section{Métodos de Runge-Kutta}
Estos métodos se caracterizan porque vamos a empezar calculando una serie de valores $K_i$ como sigue:
\[K_i = f\left(x_n+c_ih, y_n+h \sum_{j=1}^sa_{ij}K_j\right)\]
\[y_{n+1} = y_n+ h \sum_{i=1}^s b_i K_i\]

La idea de este método es que
\[y(x_{n+1}) = y(x_n) + \int_{x_n}^{x_{n+1}} f(x,y(x))\dif x \approx y(x_n)+h \sum_{j}^{s} b_j f(x_n+hc_j, y(x_n+hc_j))\]

Las diferentes formas de aproximar la función $y(x_n+hc_j)$ nos dan las diferentes $K_i$ empleadas en la definición del método.

De entre los diferentes métodos Runge-Kutta el más empleado, por ser uno de los más eficientes es el método Runge-Kutta de cuarto orden.

\subsection{Método Runge-Kutta de cuarto orden}
\[y_{n+1} = y_n +\frac{h}{6}(K_1+2K_2+2K_3+K_4)\]
siendo:
\[K_1 = f(x_n,y_n)\]
\[K_2 = f(x_n+\frac{h}{2}, y_n+\frac{h}{2}K_1)\]
\[K_3 = f(x_n+\frac{h}{2}, y_n + \frac{h}{2}K_2\]
\[K_4 = f(x_n+h, y_n + h K_3)\]

\textbf{Ejercicio punto extra: ¿De dónde salen estos números del Runge Kutta de cuarto orden?}
\textit{Pista: Considerar la $y$ como una función escalar y apoyarnos en los métodos de Taylor}

\section{Método leap-frog}
En este caso empelamos la relación:
\[y(x_{n+2}) = y(x_n) + \int_{x_n}^{x_{n+1}} f(x,y(x)) \dif x \approx y(x_n) + 2h f(x_{n+1},y_{n+1})\]

La diferencia entre este método y los anteriores es que este método consta de dos pasos, pues debemos aproximas $y_1$ ya que para calcular cada $y_n$ necesitamos conocer $y_{n-1}$ e $y_{n-2}$
\end{document}

