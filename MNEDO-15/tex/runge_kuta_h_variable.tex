\chapter{Métodos Runge Kutta}

\section{Tablero de Butcher}
Como ya vimos en la sección \ref{sec:runge-kurra} los métodos Runge Kutta se caracterizan por la necesidad de calcular unos valores $K_i$ como sigue:
\[K_i = f\left(x_n+c_ih, y_n+h \sum_{j=1}^ia_{ij}K_j\right)\]
para escribir después el método en si mismo:
\[y_{n+1} = y_n+ h \sum_{i=1}^s b_i K_i\]

La idea de este método se apoya en la aproximación de la integral como nos muestra la siguiente ecuación.
\[y(x_{n+1}) = y(x_n) + \int_{x_n}^{x_{n+1}} f(x,y(x))\dif x \approx y(x_n)+h \sum_{j}^{s} b_j f(x_n+hc_j, y(x_n+hc_j))\]

Las diferentes formas de aproximar la función $y(x_n+hc_j)$ nos dan las diferentes $K_i$ de los distintos métodos Runge Kutta.

Una manera de representar de forma compacta el método es mediante el \concept{Tablero de Butcher}. Este tablero nos permite representar de forma cómoda los valores de las constantes que intervienen en el método. Tiene la siguiente estructura:

\begin{center}
\begin{tabular}{c|cccccc}
$c_1$ & $a_{11}$ & $a_{12}$ & $a_{13}$ & $\cdots$ & $a_{1,s}$\\
$c_2$ & $a_{21}$ & $a_{22}$ & $a_{23}$ & $\cdots$ & $a_{2,s}$\\
$\vdots$ & $\cdots$ & $\cdots$ & $\cdots$ & $\ddots$ & $\vdots$ \\
$c_s$ & $a_{s1}$ & $a_{s2}$ & $a_{s3}$ & $\cdots$ & $a_{ss}$\\
\hline
 & $b_1$ & $b_2$ & $b_3$ & $\cdots$ & $b_s$\\
\end{tabular}
\end{center}

Si recordamos la definición, decimos que un método de Runge Kutta es explícito si cada $K_i$ puede calcularse explícitamente a partir de los valores anteriormente calculados. Si atendemos al tablero de Butcher, esto se traduce en que la matriz $A = a_{ij}$ sea triangular inferior estricta.

Evidentemente el método será implícito siempre que la matriz no sea triangular inferior estricta.

No obstante, existe un subconjunto dentro de los métodos de Runge Kutta que hay que destacar. Se trata de los métodos semi-implícitos que se caracterizan por tener la matriz $A$ triangular inferior pero con elementos en la diagonal.

Estos métodos son algo más complejos que los explícitos, pues tenemos que hacer aproximaciones y emplear métodos como el del punto fijo para calcular algunas $K_i$, pero siguen siendo mejores que los métodos con matrices que tengan elementos no nulos por encima de la diagonal pues nos obliga a resolver un sistema de ecuaciones implícito.

Aunque el tablero de Butcher lo estamos estudiando dentro del contexto de los métodos Runge-Kutta, puede utilizarse con cualquier método como muestra el siguiente ejemplo.

\begin{example}
Si consideramos el método de Euler mejorado tenemos el siguiente tablero:

\begin{center}
\begin{tabular}{c|cc}
$0$ & & \\
$\frac{1}{2}$ & $\frac{1}{2}$ & \\
\hline
 & $0$ & $1$\\
\end{tabular}
\end{center}
\end{example}

\begin{example}
Si consideramos el método RK-Raday IA de dos etapas, el tablero queda de la forma:

\begin{center}
\begin{tabular}{c|cc}
$0$ & $1/4$ & $-1/4$\\
$2/3$ & $1/4 $ & $5/12$ \\
\hline
 & $1/4$ & $3/4$\\
\end{tabular}
\end{center}
\end{example}

\begin{example}
Si consideramos el método RK-Lobatto IIIA de dos etapas, el tablero queda de la forma:

\begin{center}
\begin{tabular}{c|cc}
$0$ & $0$ & $0$\\
$1$ & $1/2 $ & $1/2$ \\
\hline
 & $1/2$ & $1/2$\\
\end{tabular}
\end{center}
\end{example}

\obs En general es frecuente encontrarse con tableros que tienen huecos vacíos en la matriz $A$. Estos huecos representan 0s.

\section{Hipótesis de los métodos numéricos}

Los métodos de Runge-Kutta son métodos de un sólo paso con función de incremento:
\[\phi_f(x_n,y_n;h) = \sum_{i=1}^sb_ik_i(x_n,y_n;h)\]

Puesto que las etapas $k_i$ son evaluaciones de la función $f$, no es difícil convencerse de que $\phi_f$ satisface la condición de Lipschitz con respecto a su segunda variable si $f$ también lo hace.

\begin{example}
Vamos a comprobar que la función incremento es Lipschitz en la segunda variable en el método de Euler modificado.
\[\norm{k_2(x_n,y_n;h)-k_2(x_n,\tilde{y}_n;h)} \leq L\left( \norm{y_n-\tilde{y}_n}+\frac{h}{2}\norm{k_1(x_n,y_n;h)-k_1(x_n,\tilde{y}_n;h)}\right) \leq \atop L\left(1+\frac{h}{2}\right) \norm{y_n-\tilde{y}_n}\]
de donde se obtiene inmediatamente la condición de Lipschitz deseada observando que en este caso $\phi_f(x_n,y_n;h)=k_2(x_n,y_n;h)$
\end{example}

\obs En general todo método Runge-Kutta satisface las hipótesis para métodos numéricos si tomamos un $h$ suficientemente pequeño.

Podemos ver que si tomamos $h=0$ tenemos
\[k_i(x_n,y_n;0)=f(x_n,y_n)\]
lo que nos permite escribir:
\[\phi_f(x,y;0)=\sum_{i=1}^sf(x,y)\]
Aplicando el criterio de consistencia \ref{theorem:consist_iif_relaciones} vemos que los métodos Runge-Kutta son consistentes si y sólo si
\[\sum_{i=1}^sb_i = 1\]

Por otro lado podemos comprobar que el criterio de la raíz \ref{theorem:criterio_raiz} siempre se satisface para los métodos Runge-Kutta lo que nos garantiza que siempre serán 0-estables.
