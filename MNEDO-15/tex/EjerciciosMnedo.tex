% -*- root: ../mnedo.tex -*-
\section{Hoja 1}
\begin{problem}[1]
Considerar el PVI:
\[\left\{ y'=f(t) \ t \in [t_0,T] \atop
y(t_0)=0
\right.\]

Probar que utilizando el método de Euler:
\[y_N=\sum_{k=0}^{N-1} f(t_k)h\]
\solution

Recordamos que el método de Euler consiste en emplear
\[y_{n+1} = y_n + h f(t_n)\]

Por inducción podemos resolverlo:

TODO
\end{problem}

\begin{problem}[5]
Dado $α \in [0,1]$ encontrar el orden del método:
\[y_{n+1} = y_n +hf(t_n+(1-α)h, αy_n+(1-α)y_{n+1})\]

\solution

Lo que debemos hacer es calcular el residuo cuya definición, recordemos, era:

\begin{defn}[Residuo]
Cuánto de lejos etá la solución de satisfacer en un paso la fórmula de recurrencia
\end{defn}

Según esta definición, nuestro residuo será:
\[R_n = y(t_{t+1}) - y(t_n) -hf(t_n+(1-α)h, αy(t_n)+(1-α)y(t_{n+1}))\]

Para conocer el orden de este residuo vamos a calcular los siguientes desarrollos de Taylor:
\[y(t_{n+1}) = y(t_n)+y'(t_n)h + y''(t_n)\frac{h^2}{2} + o(h^3)\]

\[h\left( f(t_n,y(t_n))+\underbrace{f_x(t_n,y(t_n))\cdot (1-α)h +\underbrace{f_y(t_n,y(t_n))(1-α)h\left( y'(t_n)+y''(t_n)\frac{h}{2}+o(h^2)\right)}_{f_y(t_n,y(t_n))y'(t_n)(1-α)h + o(h^2)}}_{(1-α)h y''(t_n) + o(h^2)}\right) = \]

\[=hy'(t_n)+(1-α)h^2y''(t_n)+o(h^3)\]

Finalmente llegamos a
\[R_n = y''(t_n)h^2 \left(\frac{1}{2}-(1-α) \right)+o(h^3)\]

Ahora tenemos dos casos posibles:
\begin{itemize}
\item α=1/2
\[R_n=o(h^3) \implies \text{ orden } \geq 1\]
\item α$\neq$1/2
\[R_n = y''(t_n)h^2 \left(\frac{1}{2}-(1-α) \right) \implies \text{ orden } 1\]
\end{itemize}

\end{problem}