\documentclass[nochap]{apuntes}
\title{Métodos numéricos para EDO}
\author{Pedro Valero}
\date{20-09-2015}

% Paquetes adicionales
\usepackage{tikztools}
\usepackage{fastbuild}
\usetikzlibrary{arrows}

\begin{document}
%\maketitle
\pagestyle{plain}

%%%%%%%%%%%%%%%%%%%%%%%%%%%%%%%%%%%%%%%%%%%%%%%%%%%%%%%%%%
%%
%%         HOJA 1
%%         Problema 5
%%
%%%%%%%%%%%%%%%%%%%%%%%%%%%%%%%%%%%%%%%%%%%%%%%%%%%%%%%%%%
\textbf{Deducir cuál es la procedencia de los coeficientes empleadas pen la fórmula de $y_{n+1}$ en el método Runge-Kutta de cuarto orden}

Los métodos de Runge-Kutta consisten en la estimación de la función $y(x)$ mediante la recursión:
\[y_{n+1} = y_n+ h \sum_{i=1}^s b_i K_i\]
donde
\[K_i = f\left(x_n+c_ih, y_n+h \sum_{j=1}^ia_{ij}K_j\right)\]

En concreto, el método Runge-Kutta de cuarto orden consiste en el empleo de la fórmula:
\[y_{n+1} = y_n +\frac{h}{6}(K_1+2K_2+2K_3+K_4)\]
siendo:
\[
\begin{array}{ll}
K_1 = & f(x_n,y_n)\\
K_2 = & f(x_n+\frac{h}{2}, y_n+\frac{h}{2}K_1)\\
K_3 = & f(x_n+\frac{h}{2}, y_n + \frac{h}{2}K_2)\\
K_4 = & f(x_n+h, y_n + h K_3)
\end{array}
\]

La idea del método es que la fórmula dada para $y_{n+1}$ coincida con el desarrollo de Taylor de esta función.

Puesto que tenemos que determinar 4 variables: $b_i|_1^4$, parece razonable escribir el desarrollo de Taylor de orden 4 de $y_{n+1}$ así como los desarrollo de Taylor de orden tres de las $K_i$ a fin de poder establecer una relación.

Así tenemos

\[y_{n+1} = y(x_{n+1}) = y(x_n) +h y'(x_n)+\frac{h^2}{2}y''(x_n)+\frac{h^3}{6}y'''(x_n) + \frac{h^4}{24}y''''(x_n)= \atop y_n + h f(x_n,y_n)+\frac{h^2}{2}\left(f_x(x_n,y_n)+f_y(x_n,y_n)\underbrace{f(x_n,y_n)}_{y_n'}\right)  + \]
\[+\frac{h^3}{6}\left( f_{x^2}(x_n,y_n)+2f_{xy}(x_n,y_n)f(x_n,y_n)+f_{y^2}(x_n,y_n)f(x_n,y_n)^2\right) +\atop \frac{h^4}{24} \left(f_{x^3}(x_n,y_n)+3f_{x^2y}(x_n,y_n)f(x_n,y_n)+3f_{xy^2}(x_n,y_n)f(x_n,y_n)^2+f_{y^3}(x_n,y_n)f(x_n,y_n)^3\right)\]

Vamos ahora a calcular los desarrollos de Taylor de orden 3 de los $K_i$ en torno a $x_n$:

\[K_1 = f(x_n,y_n)\]

\begin{remark}
A partir de ahora, para abreviar, escribiremos $g()$ para representar $g(x_n,y_n)$, sea cual sea la función $g$
\end{remark}

\[K_2 = f\left(x_n+\frac{h}{2},y_n+\frac{h}{2}K_1\right)=f()+f_x()\frac{h}{2}+f_y()\frac{h}{2}K_1 +\]
\[+ \frac{1}{2}\left( f_{x^2}()\left(\frac{h}{2}\right)^2 +2 f_{xy}()\frac{h}{2}\frac{h}{2}K_1 +f_{y^2}()\left(\frac{h}{2}K_1\right)^2\right) +\]
\[+   \frac{1}{6}\left( f_{x^3}()\left(\frac{h}{2}\right)^3 + 3f_{x^2y}\left(\frac{h}{2}\right)^2\frac{h}{2}K_1 + 3f_{xy^2}\frac{h}{2}\left(\frac{h}{2}K_1\right)^2 + f_{y^3}\left(\frac{h}{2}K_1\right)^3\right)\]

Si sustituimos ahora $K_1$ por su valor obtenemos:
\[K_2 = f\left(x_n+\frac{h}{2},y_n+\frac{h}{2}K_1\right)=f()+f_x()\frac{h}{2}+f_y()\frac{h}{2}f() +\]
\[+ \frac{1}{2}\left( f_{x^2}()\left(\frac{h}{2}\right)^2 +2 f_{xy}()\frac{h}{2}\frac{h}{2}f() +f_{y^2}()\left(\frac{h}{2}f()\right)^2\right) +\]
\[+   \frac{1}{6}\left( f_{x^3}()\left(\frac{h}{2}\right)^3 + 3f_{x^2y}\left(\frac{h}{2}\right)^2\frac{h}{2}f() + 3f_{xy^2}\frac{h}{2}\left(\frac{h}{2}f()\right)^2 + f_{y^3}\left(\frac{h}{2}f()\right)^3\right)\]

A fin de simplificar un poco la notación vamos a definir:
\[\begin{array}{l}
Df = f_x+f_yf\\
D^2f = f_{x^2}+2ff_{xy}+f_{y^2}\\
D^3f = f_{x^3} + 3f_{x^2y}f + 3f_{xy^2}f^2 + f_{y^3}f^3
\end{array}\]

Con esta nueva notación, podemos reescribir $K_2$ como sigue:
\[K_2 = f() + \frac{h}{2}Df + \frac{h^2}{8}D^2f + \frac{h^3}{48}D^3f\]

Vamos ahora a por $K_3$

\[K_3 = f\left(x_n+\frac{h}{2},y_n+\frac{h}{2}K_2\right)=f()+f_x()\frac{h}{2}+f_y()\frac{h}{2}K_2 +\]
\[+ \frac{1}{2}\left( f_{x^2}()\left(\frac{h}{2}\right)^2 +2 f_{xy}()\frac{h}{2}\frac{h}{2}K_2 +f_{y^2}()\left(\frac{h}{2}K_2\right)^2\right) +\]
\[+   \frac{1}{6}\left( f_{x^3}()\left(\frac{h}{2}\right)^3 + 3f_{x^2y}\left(\frac{h}{2}\right)^2\frac{h}{2}K_2 + 3f_{xy^2}\frac{h}{2}\left(\frac{h}{2}K_2\right)^2 + f_{y^3}\left(\frac{h}{2}K_2\right)^3\right)\]

Ahora sustituimos $K_2$ por el valor que hemos calculado anteriormente, apoyándonos en la nueva notación definida siempre que sea posible y teniendo en cuenta que estamos manipulando el polinomio de taylor de orden 4 y, por tanto, todos los términos con potencia de $h$ superior a la cuarta desaparecerán.

Así tenemos
\[K_3 =f()+f_x()\frac{h}{2}+f_y()\frac{h}{2}\left( f()+\frac{h}{2}Df + \frac{h^2}{8}D^2f\right) +\]
\[+ \frac{1}{2}\left( f_{x^2}()\left(\frac{h}{2}\right)^2 +2 f_{xy}()\frac{h}{2}\frac{h}{2}\left(f() + \frac{h}{2}Df \right) +f_{y^2}()\left(\frac{h²}{4}f^2()+\frac{h^3}{4}f()Df\right)\right) +\]
\[+\frac{1}{6}\left( f_{x^3}()\left(\frac{h}{2}\right)^3 + 3f_{x^2y}\left(\frac{h}{2}\right)^2\frac{h}{2}f()+ 3f_{xy^2}\frac{h}{2}\left(\frac{h}{2}f()\right)^2 + f_{y^3}\left(\frac{h}{2}f()\right)^3\right)\]

Agrupando los térmnos llegamos a que
\[K_3 = f()+\left(\frac{h}{2}+f_y()\frac{h^2}{4}+\underbrace{2f_{xy}()\frac{h^3}{8}+f_{y^2}()\frac{h^3}{4}f()}_{\frac{h^3}{4}Df_y}\right)Df+\left(\frac{h^2}{8} + f_y\frac{h^3}{16}\right)D^2f +\left(\frac{h^3}{48}\right)D^3f\]

Por último, nos queda
\[K_4 = f\left(x_n+h,y_n+hK_3\right)=f()+f_x()h+f_y()hK_3 +\]
\[+ \frac{1}{2}\left( f_{x^2}()h^2 +2 f_{xy}()hhK_3 +f_{y^2}()\left(hK_3\right)^2\right) +\]
\[+   \frac{1}{6}\left( f_{x^3}()h^3 + 3f_{x^2y}h^2hK_3 + 3f_{xy^2}h\left(hK_3\right)^2 + f_{y^3}\left(hK_3\right)^3\right)\]

De nuevo, sustituimos $K_3$ por su valor, teniendo en cuenta que no nos interesan aquellos términos con potencia en $h$ mayor a 4 lleganod a:
\[K_4 = f()+f_x()h+f_y()h\left( f()+\left(\frac{h}{2}+f_y()\frac{h^2}{4}\right)Df + \frac{h^2}{8}D^2f\right) +\]
\[+ \frac{1}{2}\left( f_{x^2}()h^2 +2 f_{xy}()hh\left(f()+\frac{h}{2}Df\right) +f_{y^2}()\left(hf()\right)^2\right) +\]
\[+   \frac{1}{6}\left( f_{x^3}()h^3 + 3f_{x^2y}h^3f() + 3f_{xy^2}h\left(hf()\right)^2 + f_{y^3}\left(hf()\right)^3\right)\]

Agrupando los términos llegamos a que:
\[K_4 = f() + \left(h +f_yh\left(\frac{h}{2}+f_y()\frac{h^2}{4} \right)+f_{xy}\frac{h^3}{2}\right)Df + \left( \frac{h^2}{2} + f_y()\frac{h^3}{8}+\right) D^2f + \left( \frac{h^3}{6}\right) D^3f\]
\textcolor{red}{Se que la ecuación superior no es correcta pero no encuentro el problema}

De donde podemos simplificar aún más escribiendo
\[K_4 = f() + \left(h +f_y\frac{h^2}{2}+\left(\frac{h^3}{2}\right)Df_y + \frac{h^3}{4}f_y^2\right)Df + \left( \frac{h^2}{2} + f_y()\frac{h^3}{8}+\right) D^2f + \left( \frac{h^3}{6}\right) D^3f\]

Ahora procedemos a reescribir $y_{n+1}$ con la nueva notación empleada al simplificar los $K_i$.

Así nos queda que
\[y_{n+1}=y_n+hf() + \frac{h^2}{2}Df+\frac{h^3}{6}D^2f + \frac{h^4}{24}D^3f\]
\textcolor{red}{Se que la ecuación superior no es correcta pero no encuentro el problema}

Ahora debemos igualar los coeficientes del polinomio de taylor en los dos desarrollos planteados de $y_{n+1}$, es decir, el mostrado en el párrafo superior y el obtenido al combinar las $K_i$ según la ecuación $y_{n+1} = y_n + aK_1+bK_2+cK_3+dK_4$. Con esto llegamos al sistema de ecuaciones:
\[
(a+b+c+d)f= 6\cdot f \iff a+b+c+d = 6 \]
\[\frac{b}{2}Df + \frac{c}{2}Df + dDf= 6\cdot Df  \iff b+c+2d = 12\]

\[\frac{b}{8}D^2f +\frac{c}{4}f_y()Df+\frac{c}{8}D^2f + \frac{d}{2}f_yDf+\frac{d}{2}D^2f= 6\cdot Df^2 \]
\[\frac{b}{48}D^3f + \frac{c}{4}Df_yDf+\frac{c}{16}f_yD^f+\frac{c}{48}D^3f+\frac{d}{4}f^2_yDf+\frac{d}{8}f_yD^2f + \frac{d}{6}D^3f= 6\cdot Df^3\]

%\[
%\b%egin{array}{ll}%
%K_%1 = & f(x_n+c_1h,y_n)\\%
%K_%2 = & f(x_n+c_2h, y_n+a%_{21}K_1)\\%
%K_%3 = & f(x_n+c_3h, y_n +% a_{31}K_1+%a_{32}K_2)\\%
%K_%4 = & f(x_n+c_4h, y_n +% a_{41}K_1+%a_{42}K_2+a_%{43}K_3)%
%\e%nd{array}%
%\]%
%si%endo
%\[%y_{n+1}=y_n+h\left(α_1K_1+α_2K_2+α_3K_3+α_4K_4\right)\]%
%
%Nuestro objetivo ahora es determinar los valores $a_{ij}$, $α_k$ que hacen que la %fórmula dada para $y_{n+1}$ coincida con su desarrollo de taylor.%
%
%Empezaremos manipulando las $K_i$ para escribirlas como desarroll%os de taylor de la %función $f(x_n,y_n)$ de grado 3, para finalmente igualar la fórmu%la de $y_{n+1}$ con% %su desarrollo de taylor de orden 4.%
%
%Para simplificar la notación, escribiremos $\text{algo}()$ para representar $\text{%algo}(x_n,y_n)$, es decir, suprimiremos los argumentos de las funciones. %Además, %por simplificar las fórmulas que irán apareciendo a continuación, definimos los %siguientes operadores diferenciales:%
%
%%\[\begin{array}{ll}%
%%Df = & f_x+ff_y%
%%D^2f = & f_{xx}%+2ff_{xy}+f^2f_{yy}%
%%D^3f = & f_{xxx%}+3ff_{xxy}+3f^2f_{x%yy}+f^3f_{yyy}
%%\end{array}\]%
%\begin{remark}%
%Evaluar el pol%inomio de Taylor e $(x%_n,y_n)$ implica que estamos calculando el %polinomio en t%orno a $h=0$%
%\end{remark}%
%\small%
%\begin%{itemi%ze}
%\item %\textb%f{$K_1 =  f(x_n+c_1h,y_n)$}%
%
%\[K_1 = f(x_n+c_1h, y_n)=f()+c_1hf_x()+\frac{(c_1h)^2}{2}f_{x^2}()\]
%\textbf{Con esto ya hemos deducido la fórmula de $K_1$}
%
%\item \textbf{$K_2 =  f(x_n+c_2h, y_n+a_{21}K_1)$}
%
%\[K_2 = f(x_n+c_2h, y_n+a_{21}K_1) = f()+c_2hf_x()+a_{21}K_1f_y()+ \frac{1}{2}\left( %2(a_{21}K_1)f_{yx}()+(c_2h)^2f_{x^2}()\atop+(a_{21}K_1)^2f_{y^2}()\right)\]%
%
%\item \textbf{$K_3 =  f(x_n+c_3h, y_n + a_{31}K_1+a_{32}K_2)$}%
%\[K_3 = f(x_n+c_3h, y_n+a_{31}K_1+a_{32}K_2) = f() + c_3hf_x()+(a_{31}K_1+a%_{32}K_2)%f_y()+ \atop\frac{1}{2}\left( 2(a_{31}K_1+a_{32}K_2)f_{yx}()+(c_3h)^2f_{x^2%}()+(a_{3%1%}K_1+a_{32}K_2)^2f_{y^2}()\right)\]%
%
%\item \textbf{$K_4 =  f(x_n+c_4h, y_n + a_{41}K_1+a_{42}K_2+a_{43}K_3$)}
%\[K_4 = f(x_n+c_4h, y_n+(a_{41}K_1+a_{42}K_2+a_{43}K_3)) = f() + c_4hf_x()+(a_{41}%K_1+a_{42}K_2+a_{43}K_3)f_y() \atop+\frac{1}{2}\left( 2(c_4h+(a_{41}K_1+a_{42}K_2+%a_{%43}K_3))f_{yx}()+(c_4h)^2f_{x^2}()+((a_{41}K_1+a_{42}K_2+a_{43}K_3))^2f_{y^2}()\ri%ght%)\]%
%\en%d{itemize}
%\no%rmalsize
%
%Por otro lado tenemos que $y_{n+1}$ podrá expresarse como:
%\[y_{n+1} = y(x_{n+1}) = y(x_n) +h y'(x_n)+\frac{h^2}{2}y''(x_n)+\frac{h^3}{6}y'''(%x_n)= \atop y_n + h f(x_n,y_n)+\frac{h^2}{2}\left(f_x(x_n,y_n)+f_y(x_n,y_n)\%underbrace{f(x_n,y_n)}_{y_n'}\right)  + \]%
%\[+\frac{h^3}{6}\left( f_{x^2}(x_n,y_n)+2f%_{xy}(x_n,y_n)f(x_n,y_n)+f_{y^2}(x%_n,y_n)f(%x_n,y_n)^2\right) \]%
%
%Lo primero que podem%os comprobar es el coeficiente del término $h$ del desarrollo de %Taylor de $y_{n+1}$ %es $f(x_n,y_n)$.%
%
%Por otro lado, cuando sustituyamos las $K_i$ por sus respectivas fórmulas en la %ecuación $y_{n+1}=y_n+h\left(α_1K_1+α_2K_2+α_3K_3+α_4K_4\right)$, el coeficiente% del %término $h$ será $(α_1+α_2+α_3+α_4)f(x_n,y_n)$, de donde podemos deducir la ecua%ción:
%\begin{eqnarray}%
%α_1+α_2+α_3+α_4 %= 1%
%\end{eqnarray}%
%
%Veamos ahora cuales serán los coeficientes de $h^2$ en los dos desarrollos de $y_{n+1%}$ para poder establecer una relación:%
%\begin{eqnarray}%
%\frac{1}{2}\left(f_x(x_n,y_n)+f_y(x_n,y_n)\underbrace{f(x_n,y_n)}_{y_n'}\right) = \%frac{c_1}{2}f_{x^2}()(1+a_{21}f_y+a_{21}f_{xy}+)+
%\end{eqnarray}

\end{document}
