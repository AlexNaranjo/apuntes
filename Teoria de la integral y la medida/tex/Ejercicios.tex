\section{Ejercicios}
\subsection{Hoja 1}
\begin{problem}[5]
Dada una sucesión $\lbrace f_n \rbrace \in R([a,b])$ que converge uniformemente a $f$, se pide demostrar que $f$ es ingrable Riemann y que:
\[ lim \int_{a}^{b} f_n = \int_{a}^{b} f \]

\solution
Supongamos que f es integrable Riemann, entonces tenemos que ver que:
\[ \forall \epsilon > 0 \ , \exists N \tq \forall n> N \ \abs{\int_a^b f_n - \int_a^b f} < \epsilon\]

Sabemos que:
\[\abs{\int_a^b f_n - \int_a^b f} \leq \abs{\int_a^b \abs{f_n -f}} dx\]

Recordemos la definición de convergencia uniforme

\begin{defn}[Convergencia\IS uniforme]
\[f_n \xrightarrow{uniforme} f \Leftrightarrow \forall \epsilon < 0, \exists N_{\epsilon} \tq \forall x \in [a,b]  \forall n \geq N_{\epsilon}, \abs{f_n (x) - f(x)} < \epsilon\]
\end{defn}

Si $n \geq N_{\frac{\epsilon}{b-a}}$ entonces, usando la definición de convergencia uniforme:

\[\int_a^b \abs{f_n(x) - f(x) dx} \leq \int_a^b \frac{\epsilon}{b-a}dx = \epsilon\]

Por tanto queda claro que si $f$ es integrable Riemann podemos conmutar el límite con la integral. Ahora queda ver por qué $f$ es integrable Riemann.

$f$ será integrable Riemann sii:
\[\forall \epsilon > 0 \ \exists P \tq \forall P' \prec P \]
\[\overline{J}_{P'}(f) - \underline{J}_{P'}(f) < \epsilon\]

Vamos a probarlo:
\[\overline{J}_P(f) - \underline{J}_P(f) = \sum(\sup_k f - \underset{k}{inf} f)\abs{I_k} \leq\]
\[\leq \sum\left( \abs{\sup_{k}f_n(x) - \sup_{k}f(x)} + \abs{\sup_{k}f_n(x) - \underset{k}{inf}f_n(x)} +  \abs{\underset{k}{inf}f_n(x) - \underset{k}{inf}f_n(x)} \right)\abs{I_k} \leq \]
Puesto que $f_n$ converge uniformemente a $f$ habrá un $n$ a partir del cual la distancia máxima entre $f_n$ y $f$ sea $\frac{\epsilon}{6}$ y por tanto la distancia máxima entre el supremo y el ínfimo de $f_n$ será menor que $\frac{\epsilon}{3}$
\[\leq \sum\left( \sup_{k} \abs{f_n(x) - f(x)} + \frac{\epsilon}{3} + \sup_k \abs{f_n(x) - f(x)} \right)\abs{I_k} \leq\]
Aplicando de nuevo la convergencia uniforme y tomando el máximo entre este $n$ y el calculado en el paso interior nos queda:
\[\leq \sum \left(\frac{\epsilon}{3} + \frac{\epsilon}{3} + \frac{\epsilon}{3} \right)\abs{I_k} = \sum\epsilon\abs{I_k}\]

Como hay convergencia uniforme entre $f_n$ y $f$ podemos hacer que los supremos sean tan pequeños como queramos y hacer así que el interior sumatorio quede menor que $\epsilon \abs{I_k}$

Puesto que $\epsilon$ es un número cualquiera podemos hacerlo tan pequeño como queramos haciendo que el último sumatorio escrito tienda a 0.

\end{problem}

\begin{problem}[6]

Sea $\lbrace f_n \rbrace$ una sucesión monótona creciente de funciones continuas en un intervalo $I=[a,b]$ que convergen en dicho intervalo a otra función continua $f$. Demuestra que entonces:
\[ lim \int_{a}^{b} f_n(x) = \int_{a}^{b} f(x) \]
\solution
Por ser funciones continuas son integrables Riemann. Si conseguimos demostrar que convergen uniformemente podemos emplear el ejercicio anterior y lo tendríamos hecho.
\end{problem}

\begin{problem}[7]
Dada la sucesión $I_k = (a_k, b_k)$ tales que $\bigcup_{k=1}^{N}~I_k~=~[a,b]$

Demostrar que:
\[b-a \leq \sum_{k=1}^N (b_k - a_k)\]

\solution
Vamos a utilizar la integral de Riemann como recomienda el ejercicio, utilizando la función indicatriz de cada intervalo $\ind_{I_k}$

Está claro que la función indicatriz del intervalo I es menor o igual que la suma de las funciones indicatrices de los intervalos. Lo cual es obvio, ya que si $x$ está en el intervalo, $x$ estará también en al menos uno de los intervalos $I_k$. Es decir:
\[\ind_{[a,b]} \leq \sum_{k=1}^{N} \ind_{I_k}\]

Utilizando la monotoreidad de la integral de Riemann podemos ``integrar a ambos lados`` obteniendo:

\[\int_{a}^{b} \ind_{[a,b]} \leq \int_{a_k}^{b_k} \sum_{k=1}^{N} \ind_{I_k}\]

Por la linealidad de la integral, podemos incluso meter la integral denro del sumatorio
\[\int_{a}^{b} \ind_{[a,b]} \leq \sum_{k=1}^{N} \int_{a_k}^{b_k} \ind_{I_k}\]


Conociendo la integral de la función indicatriz tenemos el resulado de forma inmediata.
\[b-a \leq \sum_{k=1}^{N} ( b_k - a_k )\]

\end{problem}

\begin{problem}[9]
Dado $O\subset (a,b)$ unión numerable de intervalos disjuntos, se pide demostrar:
\[m^*(O)=\sum_{n=1}^{\infty} \abs{I_k}\]

\solution
Está claro que:
\[m^*(O) \leq \sum_{n=1}^{\infty} \abs{I_k}\]

Tenemos que demostrar la desigualdad contraria para poder concluir la igualdad.
Vamos a probar que:
\[m^*(O) \geq \sum_{n=1}^{\infty} \abs{I_k} - \epsilon\]

Puesto que sabemos que la suma infinita tiene un resultad finito (ya que $O$ es finito)
\[\forall \epsilon > 0 \exists N \tq \sum_{n=1}^N\abs{I_n} \geq \sum_{n=1}^{\infty} \abs{I_k} - \frac{\epsilon}{2}\]

Vamos a definir los intervalos $I'_n$ que serán un poco más pequeños que los $I_n$
\[\forall n=1,2,...,N \text{ definimos } I'_n=[a_n+\frac{\frac{\epsilon}{2}}{2^{n+1}}, b-\frac{\frac{\epsilon}{2}}{2^{n+1}}]\]

Ahora tenemos que:
\[\sum_{n=1}^N \abs{I'_n}=\sum_{n=1}^N\abs{I_n} - \frac{\epsilon}{2}\sum_{n=1}^{N}\frac{1}{2^n} \geq \sum_{n=1}^N\abs{I_n} - \frac{\epsilon}{2}\]

Definimos ahora el compacto:
\[K_n = \bigcup_{n=1}^N I'_n\]

Sean $J_n$ intervalos abiertos tales que $O \subset \bigcup_{n=1}^{\infty}J_n \Rightarrow K_n \subset \bigcup_{n=1}^{\infty}J_n$.

Por tanto
\[\exists N' \tq K_n \subset \bigcup_{n=1}^{N'}J_n=A_N\]

Vamos a fijarnos ahora en la medida exterior de $O$:

\[ m^*(O) = \inf\lbrace \sum_{n=1}^{\infty}\abs{J_n} \rbrace \geq \inf\lbrace \sum_{n=1}^{N'}\abs{J_n} \rbrace \geq \]
\[ \geq \inf\lbrace \int \ind_{A_N} \rbrace \geq \inf\lbrace \int \ind_{K_N} \rbrace \geq \inf\lbrace \sum_{n=1}^N\abs{I'_n} \rbrace \geq \]
\[ \geq \inf\lbrace \sum_{n=1}^N\abs{I_n} -\frac{\epsilon}{2}\rbrace \geq \inf\lbrace \sum_{n=1}^N\abs{I_n} -\epsilon \rbrace \]

Y obtenemos así la desigualdad buscada.
\end{problem}

\begin{problem}[10]
Dado un compacto K contenido en el intervalo (a,b), nos piden demostrar que:
\[m(K)<b-a\]

\solution
Consideremos la sucesión: $(a+\frac{1}{n}, b - \frac{1}{n})$, con $n >\frac{2}{b-a}$. Obviamente:
\[K \subset \bigcup_{n=n_0}^{\infty}(a+\frac{1}{n}, b - \frac{1}{n})\]

Como K es compacto:
\[\exists N \tq K \subset \bigcup_{n=n_0}^{N}(a+\frac{1}{n}, b - \frac{1}{n}) = (a+\frac{1}{N}, b - \frac{1}{N}) \]

Y llegamos a:
\[m(K) \leq b-a-\frac{2}{N} < b-a\]

\end{problem}

\begin{problem}[11]
Sea $C_n$ una sucesión creciente de subconjuntos medibles contenidos en (a,b) y sea $C$ la unión de estos subconjuntos. Se pide demostrar que:
\[m(C_n) \rightarrow m(C)\]

\solution
Vamos a definir la sucesión $D_n$ como:
\[D_1=C_1 \ D_2 = C_2 \setminus D_1 \ D_3 = C_3 \setminus (C_1 \bigcup C_2) \ ...\]

Teniendo así C expresado como unión de los $D_n$, que son disjuntos. Así, la medida de $C$ queda expresada como:
\[m(C)=\sum_{n=1}^{\infty}m(D_n)\]
sabiendo que:
\[m(C_N)=\sum_{n=1}^{N}m(D_n)\]
\end{problem}

\begin{problem}[12]
Sea $C_n$ una sucesión decreciente de subconjuntos medibles contenidos en (a,b) y sea $C$ la intersección de estos subconjuntos. Se pide demostrar que:
\[m(C) \rightarrow m(C)\]
\solution

Tomando los complementarios, que también son medibles, tenemos una sucesión creciente de subconjuntos medibles contenidos en (a,b). Aplicando el ejercicio anterior llegamos a que:
\[(b-a)-m(C_n) \rightarrow (b-a)-m(C)\]
De donde puede deducirse que $m(C_n)$ decrece hacia $m(c)$.
\end{problem}

\begin{problem}[14]
Dada una sucesión $A_n$ contenia en (a,b), se pide demostrar que:
\[\lim_n m(A_0\Delta A_n)=0 \Rightarrow \lim_n m(A_n)=m(A_0) \]

Recordemos que:
\[ A_n \Delta A_0 = (A_n \setminus A_0) \cup (A_0 \setminus A_n) =
(A_n \cup A_0)\cap(A_n^c \cup A_0^c) \]
\solution
\[\lim_n m(A_0\Delta A_n)=0 \Rightarrow \lim_n m(A_0^c \cap A_n)=0 \wedge \lim_n m(A_0\cap A_n^c)=0\]

Escribimos ahora $A_n$ y $A_0$ como:
\[A_n = (A_n \cap A_0) \bigcup (A_n \cap A_0^c)\]
\[A_0 = (A_0 \cap A_n) \bigcup (A_0 \cap A_n^c)\]

De aquí podemos ver que:
\[m(A_n) = m(A_n \cap A_0) + m(A_n \cap A_n^c)\]
\[m(A_0) = m(A_0 \cap A_n) + m(A_0 \cap A_n^c)\]

Y restando llegamos a:
\[m(An) = m(A_0) -m(A_0 \cap A_n^c) + m(A_n \cap A_0^c)\]

Y sabemos que las medidas de estas intersecciones tienden a 0.
\end{problem}

\subsection{Hoja 2}

\begin{problem}[2]
Sea $X=\{a,b,c,d\}$ y sea $\algb{E}=\{\{a\},\{b\}\}$. Se pide construir la $\salgb$ generada por $\algb{E}$

\solution
Sabemos que en un conjunto finito toda álgebra es una $\salgb$, ya que la diferencia entre álgebra y $\salgb$ era el cierre por uniones infinitas numerables. Si un conjunto es finito, $\algb{P}(X)$ será finito y no habrá posibilidad de hacer uniones infinitas.

Vamos a construir la mínima álgebra que contiene a $\algb{E}$ a pelo, forzando que se cumplan las propiedades de un álgebra de conjuntos:
\[\algb{M}(\algb{E})=\{\emptyset, X, \{a\}, \{b\}, \{a,b\}, \{b,c,d\}, \{a,c,d\}, \{c,d\}\}\]

Con este ejemplo podemos comprobar sencillamente que:
\[\algb{M}(A_1 \cup A_2) \neq \algb{M}(A_1) \cup \algb{M}(A_2)\]
Tomando $A_1=\{a\} \ y \ A_2 = \{b\}$

\end{problem}

\begin{problem}[4]
Dada una función $\appl{g}{X}{Y}$ y sea $\algb{A}$ una $\salgb$ de X, demostrar que:
\[\algb{B}=\{E\subset Y \tq g^{-1}(E)\in \algb{A}\}\]
es una $\salgb$

\solution
Basta con ver que se cumplen las siguientes dos propiedades
\begin{enumerate}
\item 
\[g^{-1}(E_1 \cup E_2)=g^{-1}(E_1) \cup g^{-1}(E_2)\]
\item
\[g^{-1}(Y \setminus E) = X - g^{-1}(E)\]
\end{enumerate}


\begin{proof}
\begin{enumerate}
\item
\[x\in g^{-1}(E_1 \cup E_2) \iff g(x) \in E_1 \cup E_2 \iff  \exists i=1,2 \ g(x) \in E_i \iff \]
\[\iff \exists i=1,2 \ x\in g^{-1}(E_i) \iff x\in g^{-1}(E_1) \cup g^{-1}(E_2) \]
\item
\[x\in g^{-1}(Y \setminus E) \iff g(x) \in Y \setminus E \iff g(x) \notin E \iff x \notin g^{-1}(E) \iff x \in X\setminus g^{-1}(E)\]
\end{enumerate}

\end{proof}
Así, por la primera propiedad queda claro que $\algb{B}$ es cerrado por uniones y con la segunda propiedad vemos que es cerrado por complementación. Es decir:
\[\{E_i\}_{i_1}^{\infty} \in \algb{B} \implies \bigcup_{i=1}^{\infty} E_i \in \algb{B}\]
ya que $g^{-1}(\bigcup_{i=1}^{\infty} E_i) = \bigcup_{i=1}^{\infty} g^{-1}(E_i) \in A$

Podemos hacer lo mismo para ver el cierre por complementación

Además, queda claro que tanto el vacío como el total se contienen en $\algb{B}$, ya que $X$ y $\emptyset \in A \implies g^{-1}(Y)=X \in A, g^{-1}(\emptyset)=\emptyset \in A \implies Y, \emptyset \in \algb{B}$ por lo que se trata de una $\salgb$
\end{problem}

\begin{problem}[5]
Dada una función $\appl{g}{X}{Y}$ y sea $\algb{B}$ una $\salgb$ de Y, demostrar que:
\[\algb{A}=\{g^{-1}(E) \tq E \in \algb{B}\}\]
es una $\salgb$

\solution
Vamos a comprobar las propiedades de cierre por unión y por complementación:
\begin{enumerate}
\item. Debemos ver que:
\[A_1, A_2\in \algb{A} \Rightarrow A_1 \cup A_2 \in A\]
Lo cual es cierto ya que, como demostramos en el ejercicio anterior:
\[g^{-1}(E_1 \cup E_2)=g^{-1}(E_1) \cup g^{-1}(E_2)\]
Si tomamos $A_i=g^{-1}(E_i)$ queda claro que:
\[A_1 \cup A_2 = g^{-1}(E_1) \cup g^{-1}(E_2) = g^{-1}(E_1\cup E_2) \Rightarrow g^{-1}(E)\]
para algún $E\in \algb{B}$ por ser una $\salgb$.

\item Ahora tenemos que ver que:
\[A \in \algb{A} \Rightarrow X\setminus A \in \algb{A}\]
Con la segunda propiedad del ejercicio anterior queda claro que es cierto
\end{enumerate}

Por tanto, $\algb{A}$ cumple todas las propiedades de $\salgb$.
\end{problem}

\begin{problem}[0-0]
\textbf{Inventado por el profesor.}

Vamos a ver que la imagen directa de una $\salgb$ no tiene por qué ser una $\salgb$, es decir:
dada una función $\appl{f}{X}{Y}$ y sea $\algb{A}$ una $\salgb$ de X, demostrar que:
\[\algb{B}=\{g(E) \tq E \in \algb{A}\}\]
no es necesariamente una $\salgb$

\solution
Esto es sencillo puesto que si la función $f$ no es suprayectiva, entonces $Y \neq g(E)$ para cualquier $E$, por lo que $\algb{B}$ no contiene al total y, por tanto no sería $\salgb$.

Supongamos ahora que es suprayectiva la función $f$. En este caso seguríamos teniendo problemas si $f$ no es inyectiva. 

Tomemos por ejemplo $g(x)=x^2$. En este caso, $g((-\infty, 0] \cup [0, \infty))=g(0)=0 \neq g((-\infty, 0]) \cup g([0, \infty))$.
\end{problem}

\begin{problem}[6]
Demuestra que una álgebra $\algb{A}$ es una $\salgb$ si y sólo si es cerrada para uniones numerables crecientes.

\solution
Vamos a demostrar las dos direcciones de la implicación:
\begin{itemize}
\item $\Rightarrow$
Es obvio ya que una $\salgb$ es un álgebra cerrada por uniones numerables y por tanto, es cerrada para el caso concreto de uniones numerables crecientes.
\item $\Leftarrow$
\textcolor{red}{Hecho por mi. No es seguro al 100\%}

Tenemos que demostrar que dada una unión numerable cualquiera $\{A_i\}_{i=1}^{\infty} \in \algb{A}$, se cumple que $\bigcup_{n=1}^{\infty} A_i \in \algb{A}$. Vamos a ello.

Dada mi unión numerable, construyo otra de la siguiente forma:
\[B_n = \bigcup_{i=1}^{n} A_i\]
de tal forma que $B_i \subset B_j \forall i<j$.

Ahora, sabiendo que la unión de los $B_i$ se contiene en la $\salgb$, tenemos que:
\[\bigcup_{n=1}^{\infty} A_i=\bigcup_{n=1}^{\infty} B_i \in \algb{A}\]

\end{itemize}
\end{problem}

\begin{problem}[7]
Determina el álgebra $\algb{A}$ generada por la colección de los subconjuntos finitos de un conjunto X no-numerable. Determina la $\salgb$ generada por $\algb{A}$. Estudiar el mismo problema en caso de que el conjunto X sea infinito numerable

\solution
Dado el conjunto de todos los subconjuntos finitos, para convertirlo en una $\salgb$ debemos asegurarnos de que sea cerrado por uniones numerables y por complementación.

Para hacerlo cerrado por uniones numerables, debemos incluir todos los conjuntos numerables, ya que cualquiera de estos puede obtenerse como unión de conjuntos finitos.

Además, si queremos que sea cerrado por complementación, tendremos que incluir los complementarios de todos los conjuntos mencionados anteriormente, es decir, debemos incluir todos aquellos conjuntos cuyo complementario sea numerable.

Así, vamos a construir de forma directa la $\salgb$ pedida:
\[\algb{M}(\algb{E})=\{ E\subset X \tq E \text{ numerable ó } E^c \text{ numerable}\}\]

Podemos comprobar fácilmente que es una $\salgb$ observando que cumple las propiedades necesarias siguiendo el mismo razonamiento que el realizado para construirla. (está hecho en teroía). Es la mínima por construcción.

Si X es infinito numerable entonces 
\[\algb{M}(\algb{E})=\algb{P}(X)\]
aplicando la misma regla de construcción, ya que todos los subconjuntos de $\algb{P}(X)$ son numerables.
\end{problem}

\begin{problem}[8]
La $\salgb$ de (0,1] engendrada por:
\[\algb{E}= \{(0, \frac{1}{n}]: n=1,2,...\}\]
está formada por uniones finitas o numerables de intervalos (a,b]. Estudia cómo son estos intervalos.

\solution
Queremos ver como son los intervalos (a,b] tales que:
\[\algb{M}(\algb{E})=\{\bigcup_{i=1}^{\infty}(a_i, b_i]\}\]

Vamos a ver cuáles son los elementos que tenemos en $\algb{M}(\algb{E})$. Aquí, además de los propios elementos de $\algb{E}$ tenemos:
\begin{itemize}
\item \textbf{Complementarios} Son de la forma $(\frac{1}{n}, 1]$

\item \textbf{Uniones} Son de la forma $(\frac{1}{m}, \frac{1}{n}]$ con m>n

\item \textbf{Intersecciones} Intersecando dos elementos seguimos estando en $\algb{E}$, así que no ganamos nada nuevo.
\end{itemize}

Puesto que las uniones contienen a los complementarios, tenemos que los intervalos que forman la $\salgb$ son de la forma:
\[\algb{M}(\algb{E})=\{\bigcup_{i=1}^{\infty}(a_i, b_i]: a_i =\frac{1}{m}, \ b_i = \frac{1}{n} \ m>n\}\]
\end{problem}

\begin{problem}[9]
Describe la $\salgb$ generada por:
\[\algb{E}= \{N \subset \nat \tq \forall n \in \nat, \ 2n \in N\}\]

\solution
Vemos que el conjunto $\algb{E}$ está formado por todos los conjuntos que contienen a todos los pares.

Esta claro que la unión y la intersección de conjuntos de $\algb{E}$ pertenecen a $\algb{E}$ ya que contendrán a todos los pares.

Los complementarios serán los subconjuntos de $\nat$ que sólo contengan números impares.

Si combinamos todos estos conjuntos formaremos la mínima $\salgb$ que lo contenga todo. Es decir:
\[\algb{M}(\algb{E})=\{N \subset \nat \tq \text{Todos los n son impares, o N contiene a todos los pares}\}\]
\end{problem}

\begin{problem}[10]
Sea $\algb{M}$ una $\salgb$ de cardinal infinito. Demuestra que tiene cardinal no numerable.
\solution
Si la $\salgb$ es infinita podemos encontrar una colección numerable y disjunta de elementos de la misma. (Lo podemos hacer tomando una colección numerable y haciéndola disjunta mediante la eliminación en cada elemento de la unión de los anteriores).

Denotamos a esta colección como: $\{A_n:n\in \nat\}$. 

Ahora vamos a construir una colección infinita de $A_x$ como sigue:
\[\forall x \in (0,1) \text{ obtenemos el desarrollo en base 2 de } x=0,x_1,x_2,x_3,...\]
\[A_x=\bigcup_{n=1}^{\infty}A'_n \text{ donde } A'_n=\left\{ \begin{array}{lcc}
             A_n &   si  & x_n = 1 \\
             \\ \emptyset &  si  & x_n \neq 1
             \end{array}
   \right.\]

Como no puede darse el caso de dos $A_x$ iguales, es necesario tomarlos todos para construir $\algb{M}$ y puesto que el número de $A_x$ es no numerable (uno por cada real en el intervalo (0,1)), tenemos que el número de elementos de $\algb{M}$ es no numerable.

\end{problem}

\begin{problem}[11]
Hallar una cota superior al número de elementos que puede tener una $\salgb$ $\algb{M}$ generada a partir de un conjunto con n elementos:
\solution
Está claro que el caso peor es cuando los elementos de $\algb{E}$ son disjuntos dos a dos. En este caso tendremos $2^n$, por un argumento similar al del ejercicio anterior.

\textcolor{red}{
Aquí Miguel se sacó la chorra y meó encima del profesor. Fue épico
}

\end{problem}

\begin{problem}[12]
Sellama $\sigma$-anillo de suconjuntos de un conjunto $X$ a toda familia no vacía $\algb{F}$ de subconjuntos de $X$ cerrada para uniones numerables y para las diferencias. Demuestra que todo $\sigma$-anillo es también cerrado para intersecciones numerables. Demuestra que todod $\sigma$-anillo de $X$ es una $\salgb \iff X \in \algb{F}$
\solution
Vamos a probar que un $\sigma$-anillo es cerrado para intersecciones numerables.
Dada una familia infinita numerable $A_i \in \algb{F}$ sabemos que $\bigcup A_i \in \algb{F}$ y lo denotamos por $A$.
Entonces:
\[A \setminus \bigcap A_i = \bigcup(A\setminus A_i) \in \algb{F} \Rightarrow A\setminus (A \setminus \bigcap A_i)=\bigcap A_i \in \algb{F}\]

La condición que le falta a un $\sigma$-anillo para ser una $\salgb$ es contener al total. Por tanto queda clara la última demostración.
\end{problem}

\begin{problem}[13]
Se llama ``clase monótona'' de un conjutno $X$ a toda familia no vacía $\algb{M}$ de subconjuntos de $X$ que sea cerrada para las uniones crecientes y para las intersecciones decrecientes (es decir, si $\forall i=1,2,.. \ C_i \in \algb{M}$ y $C_i \subset C_{i+1}$ o $C_i \supset C_{i+1}$ entonces $\cup_iC_i \in \algb{M}$ o $\cap_iC_i \in \algb{M}$, respectivamente). 

Demuestra que toda $\salgb$ es clase monótona. Da un ejemplo de una clase monótona que no sea $\salgb$
\solution
Antes de nada vamos a definir lo que es una clase monótona.

\begin{defn}[Clase monótona]
$\algb{M} \subset \algb{P}(X)$ es clase monótona si para toda sucesión $\{A_i\}$ de elementos de $\algb{M}$ se cumple que:
\begin{enumerate}
\item \[\forall i A_i \subset A_{i+1} \Rightarrow \bigcup_{n=1}^{\infty}A_i \in \algb{M}\]
\item \[\forall i A_i \supset A_{i+1} \Rightarrow \bigcap_{n=1}^{\infty}A_i \in \algb{M}\]
\end{enumerate}
\end{defn}
Basta con que provemos la primera propiedad ya que la segunda sale por complementación.

No hay nada que demostrar por que esta primera propiedad es parte de la definición de $\salgb$. Es obvio que toda $\salgb$ es clase monótona. La parte interesante del ejercicio es la que sigue.

Vamos a buscar el ejemplo de clase monótona que no sea $\salgb$.
Para el ejemplo basta con encontrar una cadena finita de conjuntos crecientes lo que nos garantiza el cumplimiento de las propiedades de una clase monótona, pero dista mucho de ser una $\salgb$.

Un ejemplo concreto sería:
\[\nat \supset \{2n, \ \forall n \in \nat\}\supset \{4n, \ \forall n \in \nat\}\supset \{8n, \ \forall n \in \nat\}\supset \{16n, \ \forall n \in \nat\}\supset \emptyset\]
La clase monótona sería una subclase de $\algb{P}(\nat)$ formada por todos los conjuntos aquí descritos y no es una $\salgb$, ya que no es cerrada por complementación.
\end{problem}

\begin{problem}[14]
Demuestra que la mínima clase monótona que contiene un álgebra dada $\algb{A}$ es también una $\salgb$

\solution
Vamos a observar dos cosas:
\begin{enumerate}
\item Existe al menos una clase monótona que contiene a un conjunto $\algb{E}$, que es $\algb{P}(X)$
\item La intersección de clases monótonas es clase monótona. Demostración trivial.
\end{enumerate}

Tras estas observaciones podemos hablar de la mínima clase monótona que contiene a $\algb{E}$ como la intersección de todas las clases monótonas que lo contienen y sabemos que la intersección no es vacía por que hay al menos una.

Vamos a tener que hacer dos observaciones más:
\begin{enumerate}
\item Si una clase de conjuntos es álgebra y clase monótona, entonces es $\salgb$.
\item Sea $\algb{A}$ es un álgebra y $\algb{C}(\algb{A})$ la mínima clase monótona que contiene a $\algb{A}$. Demostremos que $\algb{C}(\algb{A})$ es un álgebra.
\end{enumerate}
\textcolor{red}{Hecho por mi. No fiarse al 100\%}
\begin{proof}
\begin{itemize}
\item Si nos apoyamos en el ejercicio 6, vemos clara esta demostración. Por ser clase monótona será cerrada para uniones numerables crecientes. Apoyándonos en el ejercicio 6, tenemos que un álgebra cerrada por uniones crecientes numerables es una $\salgb$

\item La mínima clase monótona que contiene un álgebra $\algb{A}$ contiene, además de todos los elementos de $\algb{A}$, a la unión de toda sucesión creciente de subconjutnos.

Entonces, tenemos un álgebra, a la que le añadimos las condiciones necesarias para que sea cerrada por uniones crecientes numerables. Es decir, hemos añadido conjuntos que son límite de una sucesión creciente de subconjuntos. Por tanto se siguen cumpliendo las propiedades de álgebra.
\textcolor{red}{pichí-pichá}

\end{itemize}
\end{proof}
\end{problem}