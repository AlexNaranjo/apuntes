\documentclass{apuntes}

\title{Teoria de la integral y la medida}
\author{Pedro Valero y Jorge Martín}
\date{14/15 C1}

% Paquetes adicionales

% --------------------

\begin{document}
\pagestyle{plain}
\maketitle

\tableofcontents
\newpage

\chapter{Sobre la asignatura}
\section{Evaluación}
Hay tres posibles itinerarios de evaluación:
\begin{itemize}
\item \textbf{A}: $\frac{1}{3}T_1+\frac{1}{3}EP+\frac{1}{3}T_2$
\item \textbf{B}: $\frac{2}{5}EP+ \frac{3}{5}EF$
\item \textbf{C}:$EF$
\end{itemize}
Donde $T_1$  y $T_2$ son ``trabajos" que habrá que entregar y se realizará un examen en las fechas indicadas. El ``trabajo" consistirá en la resolución y entrega individual de ciertos ejercicios y el examen se basará en esos ejercicios con algunas variaciones.

\section{Fechas de exámenes}
Las fechas de los exámenes/trabajos serán:
\begin{itemize}
\item $T_1$: 7-Oct
\item $T_2$: 17-Dic
\item $EP$: 10:14-Nov
\item $EF$: 12-Ene
\end{itemize}

\section{Bibliografía}
Durante el curso se seguirán dos libros fundamentalmente:
\begin{enumerate}
\item El libro de Folland: \textit{Real Analysis}
\item El libro de Taylor: \textit{Measure theory and integration} (Serivirá en algunos temas en concreto)
\end{enumerate}

\chapter{}
\end{document}
