\documentclass{apuntes}

\title{Teoria de la integral y la medida}
\author{Pedro Valero y Jorge Martín}
\date{14/15 C1}

% Paquetes adicionales
\usepackage{tikztools}
\usepackage{fastbuild}
\usetikzlibrary{arrows}
% --------------------
\begin{document}
\pagestyle{plain}
\maketitle

\tableofcontents
\newpage

\chapter{Sobre la asignatura}
\section{Evaluación}
Hay tres posibles itinerarios de evaluación:
\begin{itemize}
\item \textbf{A}: $\frac{1}{3}T_1+\frac{1}{3}EP+\frac{1}{3}T_2$
\item \textbf{B}: $\frac{2}{5}EP+ \frac{3}{5}EF$
\item \textbf{C}:$EF$
\end{itemize}
Donde $T_1$  y $T_2$ son ``trabajos" que habrá que entregar y se realizará un examen en las fechas indicadas. El ``trabajo" consistirá en la resolución y entrega individual de ciertos ejercicios y el examen se basará en esos ejercicios con algunas variaciones.

\section{Fechas de exámenes}
Las fechas de los exámenes/trabajos serán:
\begin{itemize}
\item $T_1$: 7-Oct
\item $T_2$: 17-Dic
\item $EP$: 10:14-Nov
\item $EF$: 12-Ene
\end{itemize}

\section{Bibliografía}
Durante el curso se seguirán dos libros fundamentalmente:
\begin{enumerate}
\item El libro de Folland: \textit{Real Analysis}
\item El libro de Taylor: \textit{Measure theory and integration} (Serivirá en algunos temas en concreto)
\end{enumerate}

\chapter{Repaso}
\begin{theorem}[Teorema fundamental del Cálculo]
Dada una función $f$ integrable sobre el intervalo [a,b], definimos $F$ sobre [a,b] por:
\[F(x)=\int_{a}^{x}f(t)dt\]
Si $f$ es continua en $c\in (a,b)$ entonces $F$ es derivable en C y $F'(c)=f(c)$
\end{theorem}

\section{Integral de Riemann}
La construcción de esta integral se realiza sobre un intervalo $[a,b]$, en $\mathbb{R}^2$ y basándonos en una función continua $y=f(x)$.

\begin{defn}[Partición]
Una partición P de $[a,b]$ es una colección finita de subintervalos cerrados $I_i$=$[x_{i-1}, x_i]$ con i=1,2,...,N tal que $a=x_0<x_1<\cdots < x_N=b$.

$\abs{I_i}$ = $x_i-x_{i-1}$

$\abs{P}$=max$\lbrace \abs{I_i} \rbrace$
\end{defn}


Sea $f:[ a,b ] \rightarrow \mathbb{R}$ acotada definimos:
\begin{itemize}
\item \begin{defn}[Suma superior]
$\overline{J}_p(f)=\sum_{i=1}^{N}sup(f(x))|I_i|$
\end{defn}

\item \begin{defn}[Suma inferior]
$\underline{J}_p(f)=\sum_{i=1}^{N}inf(f(x))|I_i|$
\end{defn}
\end{itemize}

La suma inferior de Riemman será la correspondiente a tomar para cada intervalo $[x_{i-1}, x_i]$ la altura mínima de f(x) y construir un rectángulo de dicha altura para posteriormente sumar las áreas de esos rectángulos

La suma superior de Riemann es equivalente a la inferior salvo que en cada intervalo $[x_{i-1}, x_i]$ tomamos la altura máxima de f(x).

La suma superior nos dará un área mayor que el área real encerrada por la curva mientras que la suma inferior nos dará un área menor que la real.

\begin{defn}[Finura]
Dadas dos particiones P,Q de [a,b] P es más fina que Q (Q$\prec$~P) si el conjunto de extremos de subintervalos de Q está contenido en el conjunto de extremos de subintervalos de P.

Es decir, P tiene más subintervalos que Q.
\end{defn}

Si Q$\prec$P entonces $\overline{J}_P(f) \leq \overline{J}_Q(f)$ y $\underline{J}_Q(f) \leq \underline{J}_P(f)$.

Dadas dos particiones $P_1,P_2$ siempre es posible obtener una partición $P$ que refina a ambas. Esta partición será aquella que una todos los extremos de subintervalos de las dos particiones dadas.

Entonces $\underline{J}_{P_1(f)} \leq \underline{J}_P(f) \leq \overline{J}_P(f) \leq \overline{J}_{P_2}$. Es decir, cualquier suma inferior de la función f es menor o igual que cualquier suma superior de dicha función

Por tanto podemos garantizar la existencia de un supremo de las sumas inferiores y un ínfimo de las sumas superiores que serán finitos. Obviamente este ínfimo será mayor que el supremo.

\begin{defn}[Función Integrable\IS Riemann]
Una función f:$[ a,b] \rightarrow \mathbb{R}$ acotada es integrable Riemann ($f \in R([a,b])$) si el supremo y el ínfimo anteriores coinciden, en cuyo caso se escribe:
$J(f) = \int_a^b f = \underline{J}(f) = \overline{J}(f) = \int_a^b f(x) dx$
\end{defn}

\subsection{Propiedades}
$f,g \in R([a,b]), \alpha \in \mathbb{R}$
\begin{enumerate}
\item La suma es integrable Riemann
\[f + g \in R([a,b])\]\[ J(f+g) = J(f) + J(g)\]
\item El producto por una constante es integrable Riemann
\[\alpha f \in R([a,b])\] \[ J(\alpha f) = \alpha J(g)\]
\item Si además $f\geq 0$ entonces $J(f)\geq 0$
\item $\forall c \in (a,b)$ $f\in R([a,c])$  y $f\in R([c,b])$ se cumple
$\int_a^b f = \int_a^c f + \int_c^b f$
\end{enumerate}

\begin{example} $ $

Veamos un ejemplo de función acotada en $[0,1]$ que NO ES integrable Riemann.
\[
 \ind_{\mathbb{Q}  \cap  [0,1]} (x) =
  \begin{cases}
   1 &  x \in \mathbb{Q} \cap [0,1] \\
   0       &  x \in [0,1] \backslash \mathbb{Q}
  \end{cases}
\]

Para comprobar que esta función no es integrable Riemann tenemos que ver que las sumas inferiores no coinciden con las superiores.
\[\forall P, \overline{J}_P(\ind_{\mathbb{Q} \cap [0,1]}) = 1\]
\[\forall P, \underline{J}_P(	\ind_{\mathbb{Q} \cap [0,1]}) = 0\]


\begin{proof}
La suma inferior da siempre 0 en este caso ya que en cualquier intervalo de la partición hay números irracionales. Por otro lado la suma superior da 1 ya que en todo intervalo hay racionales. Así el ínfimo de las sumas superiores es 1 mientras que el supremo de las sumas inferiores es 0.
\end{proof}
\end{example}

\subsection{Problema de la Integral de Riemann}
Existe una sucesión de funciones $f_n$, tales que $f_n\in R([ 0, 1 ])$, creciente y uniformemente acotada que converge a $f(x)$ tal que:
\[lim \int_0^1 f_n(x) dx \neq \int_0^1 f(x) dx\]

\begin{example} $ $
%TODO completar esto en condiciones por que da pena.

Sea E = $\lbrace q_1, q_2, q_3, \cdots \rbrace$ una enumeración los racionales comprendidos entre 0 y 1.

Definimos $f_n(x)=
\begin{cases}
   1 &  x \in E \\
   0       &  x \in [0,1] \backslash E \rbrace
\end{cases}$
  
El límite de estas funciones sería $f(x)=
\begin{cases}
   1 &  x \in \mathbb{Q} \cap [0,1] \\
   0       &  x \in [0,1] \backslash \mathbb{Q}
\end{cases}$

Si calculamos las integrales de Riemann superior e inferior vemos que todas las funciones $f_n(x)$ son integrables Riemann pero no lo es $f(x)$, como ya vimos en el ejemplo anterior.
 Veamos como cada $f_n$ es integrable Riemann:
 %TODO Añadir etiqueta


Definimos $f_{q_n}(x) =
\begin{cases}
	1 & x = q_n\\
	0 & x \in [0,1] \backslash \{q_n\}
\end{cases}$

Tomando la partición
$P = \{ [0, q_n - \frac{\varepsilon}{2}], [q_n - \frac{\varepsilon}{2}, q_n + \frac{\varepsilon}{2}], [q_n + \frac{\varepsilon}{2}, 1] \}$

% Dibujo de la recta asociada a la partición
\inputtikz{recta-de-particion}

Podemos ver que $f_{q_n}$ es integrable Riemman, ya que:
\[ \overline{J}_P (f_{q_n}) =
\varepsilon \underset{\varepsilon \rightarrow 0}{\implies} \overline{J} (f_{q_n}) = 0 = \underline{J} (f_{q_n}) \]

Como $f_n = f_{q_1} + \ldots + f_{q_n}$, y toda $ f_{q_i}$ es integrable Riemann, por la propiedad 1 sabemos que $f_n$ será integrable Riemann.
\end{example}

\begin{defn}[Contenido]
A toda integral sobre un intervalo se puede asociar una forma de medir subconjuntos del intervalo.

En el caso de la integral de Riemann esa ``medida" suele llamarse contenido de Jordan.

Sea C$\subset$[a,b] se define:
\[cont(C) = \int_a^b \ind_{C}(x) dx\]
siempre que $\ind_C \in R([a,b])$

En general, aunque $\ind_C \notin R([a,b])$ se puede hablar de contenido exterior y de contenido interior:
\[cont^+(C)=\overline{J}(\ind_C)\]
\[cont^-(C)=\underline{J}(\ind_C)\]

Suele decirse que un conjunto ``tiene contenido" cuando $cont^+(C)=cont^-(C)$
\end{defn}

El problema que abordaremos próximamente es que la unión numerable de conjuntos con contenido no siempre tiene contenido.

\obs $f\in R([a,b]) \Leftrightarrow \forall \epsilon \exists P \tq \overline{J}_P(f)-\underline{J}_P(f) < \epsilon$


\textbf{En moodle hay un resumen de este tema}

\begin{example}
\textbf{Ejercicio 1 de la hoja 1}

El objetivo del ejericio es demostrar que $f\in R([a,b]) \Rightarrow \abs{f}\in R([a,b])$

Para ellos definimos: 
$f^+(x) = max\lbrace f(x), 0 \rbrace$

$f(x) = f^+(x) - f^-(x)$

De donde podemos deducir:

$f^-(x) = f^+(x)-f(x)=max \lbrace -f(x), 0 \rbrace$

Y las funciones tienen el siguiente aspecto:

\begin{center}
	% Función f
	\inputtikz{img1-ej1-h1}
	% Función f+
	\inputtikz{img2-ej1-h1}
	% Funcion f-
	\inputtikz{img3-ej1-h1}
\end{center}


Así podemos escribir 
$\abs{f(x)} = f^+(x)+f^-(x)$.

Gracias a esta afirmación el problema se reduce a demostrar que la parte positiva ($f^+$) es integrable Riemann, ya que entonces lo serán  y $f^-$  y $\abs{f(x)}$  por ser la suma/resta de dos funciones integrables Riemann.


Vamos a calcular $f^+(x)$:
\[\overline{J}_P(f^+)-\underline{J}_P(f^+) = \sum_{k=1}^n\left(\sup_{x\in I_k}(f^+(x))\right)\abs{I_k} - \sum_{k=1}^n\left(\underset{x\in I_k}{inf}(f^+(x))\right)\abs{I_k}=\]
\[= \sum_{k=1}^n\left(\sup_{x\in I_k}(f^+(x))-\underset{x\in I_k}{inf}(f^+(x))\right)\abs{I_k} \leq
\sum_{k=1}^n\left(\sup_{x\in I_k}(f(x))-\underset{x\in I_k}{inf}(f(x))\right)\abs{I_k} =\] \[=\overline{J}_P(f)-\underline{J}_P(f) = 0\]

Y sabemos que f es integrable
\end{example}

\begin{example}
\textbf{Ejercicio 2 hoja 1}

El ejercicio nos pide demostrar:
\[ f,g \in R([a,b]), h(x) = max\lbrace f(x), g(x) \rbrace \Rightarrow h \in R([a,b]) \]

Vamos a comprobar que la diferencia entre las integrales Riemann superior e inferior es 0.
Pero antes fijémonos en la \underline{indicación}:

\newpage
\[\sup\big\{ \max\{f(x), g(x)\}\big\} - inf\big\{ \max\{f(x), g(x)\} \big\} \leq \]

\[ \leq \max\big\{ \sup_{x \in I_k} (f(x)) - \underset{x \in I_k}{inf} (f(x)), \sup_{x \in I_k} (g(x)) - \underset{x \in I_k}{inf} (g(x)) \big\} \leq \]

\[ \leq \left(\sup_{x \in I_k} (f(x)) - \underset{x \in I_k}{inf} (f(x))\right) + \left(\sup_{x \in I_k} (g(x)) - \underset{x \in I_k}{inf} (g(x))\right) \]


Ahora podemos probar que:
\[\overline{J}_P(h)-\underline{J}_P(h)
= \sum\left(\sup_{x\in I_k}(h(x))-\underset{x\in I_k}{inf}(h(x))\right)\abs{I_k} \leq\]

\[\leq \sum\left( \sup_{x \in I_k} (f(x)) - \underset{x \in I_k}{inf} (f(x)) \right) \abs{I_k} + 
 \sum\left( \sup_{x \in I_k} (g(x)) - \underset{x \in I_k}{inf} (g(x)) \right) \abs{I_k} \]

 
 %TODO ¿Esto está completo?
\end{example}

\chapter{Teoría de la medida}
Para resolver los problemas de la integral de Riemann va a aparecer la integral de Lebesgue pero para poder usarla correctamente necesitamos medir. Aquí nos surgen dos problemas: ¿Con qué quieres medir? y ¿Qué quieres medir?

Veamos un caso sencillo. 
\begin{example}
Tomemos el intervalo [0,1) y supongamos que queremos medir un conjunto $A\subset [0,1)$. Esta medida debería cumplir algunas propiedades básicas. Por ejemplo:
\begin{itemize}
\item $0\leq m(A) \leq 1$
\item La medida de un intervalo deberá coincidir con su longitud $m(I)=\abs{I}$
\item Invariable por traslaciones $m(A+t) = m(A)$.
\item Dada una familia de conjuntos disjuntos $A_i\subset [0,1)$, $m(\cup_{1}^{\infty} A_i) = \sum_{1}^{\infty}m(A_i)$
\end{itemize}

¿Puedo asignar a cualquier $A~\subset~[0,1)~un~m(A)$? La respuesta es \textbf{No}

La forma de solucionar este problema consiste en dejar de querer medir todos los conjuntos posibles en el intervalo [0,1) y limitarnos a trabajar con aquellos conjuntos que si podemos medir.

Vamos a definir una relación de equivalencia en el intervalo [0,1): xRy $\Leftrightarrow$ x-y~$\in~\mathbb{Q}$. R divide a [0,1) en clases de equivalencia E=$\lbrace$ un elemento de cada clase $\rbrace$

\[E_q = \lbrace E\cap [0, 1-q) +q \rbrace \cup \lbrace E\cap (1-q, 1) + q - 1\rbrace\]
Si $q \neq q' \ q,q'\in \mathbb{Q} \Rightarrow E_q\cap E_{q'} = \emptyset$

Visto esto podemos escribir [0,1)=$\cup_{q\in \mathbb{Q}}E_q$. Es decir, escribimos el intervalo [0,1) como una unión infinita de conjuntos disjuntos y trasladados. Según las propiedades que hemos pedido para una medida necesitaríamos $1 = \sum m(E_q)$ siendo $m(E_q)$ igual $\forall q$  pero esto es imposible. Si $m(E_q) = 0$ la suma es 0. En caso contrario una suma infinita nos dará infinito.

Por tanto el intervalo [0,1) no podemos medirlo siguiendo este criterio

\end{example}

Podemos pensar que la cuarta propiedad que hemos pedido para la medida es demasiado. ¿Y si restringimos esa propiedad a uniones (y por tanto sumas) finitas?. Aún con esta simplificación de las condiciones no podríamos medir todo.

\begin{example}
Paradoja de Banach-Tarsky

La paradoja de Banach-Tarski es en realidad un teorema que afirma que es posible dividir una esfera (llena) de radio 1 en ocho partes disjuntas dos a dos, de modo que, aplicando movimientos oportunos a cinco de ellas, obtengamos nuevos conjuntos que constituyan una partición de una esfera (llena) de radio 1, y lo mismo ocurra con las tres partes restantes.

En palabras más sencillas, se supone que es posible fabricar un rompecabezas tridimensional de un total de ocho piezas, las cuales, combinadas de una determinada manera, formarían una esfera completa y rellena (sin agujeros) y, combinadas de otra manera, formarían dos esferas rellenas (sin agujeros) del mismo radio que la primera.

El teorema de Banach–Tarski recibe el nombre de paradoja por contradecir nuestra intuición geométrica básica. Las operaciones básicas que se realizan preservan el volumen siempre que los fragmentos sean medibles, pero precisamente las ocho partes citadas en el teorema son conjuntos no medibles. La construcción de estos conjuntos hace uso del axioma de elección para realizar una cantidad no numerable de elecciones arbitrarias.

\end{example}

\begin{defn}[Compacto]
En cualquier topología un compacto es un conjunto tal que de todo recubrimiento mediante abiertos se puede extraer un recubrimiento finito.
\end{defn}

\begin{theorem}
En $\mathbb{R}$ todo compacto es cerrado y acotado
\end{theorem}

Vamos  a movernos en I=(a,b) con a<b. Dado $S \subset (a,b)$  vamos a definir la medida interior y la medida exterior. Para aquellos conjuntos en los que estas dos medidas coincidan definiremos la medida y consideraremos esos conuntos como medibles.

\section{Medida exterior}
\begin{defn}[Medida exterior ($m^*$)] Es el ínfimo de la suma de las longitudes de los conjuntos que forman todos los recubrimientos posibles.
\[m^*(S) = inf \lbrace \sum_{k=1}^{\infty} \abs{I_k}, S\subset \bigcup_{k=1}^{\infty} I_k \rbrace\]

Es decir, tomamos el conjunto de todas las familias recubridoras, asignamos a cada familia un número (suma de las longitudes de cada conjunto de la familia) y tomamos el ínfimo.
\end{defn}

Veamos algunas propiedades de la medida exterior:
\begin{enumerate}
\item \textbf{Monotonía} Dados dos conjuntos $S_1 \subset S_2 \Rightarrow m^*(S_1) \leq m^*(S_2)$
\item \textbf{Subaditividad} 
\[\forall O \ abierto \ O = \bigcup_{1}^{\infty}I_k, \ I_k \ abiertos \ y \ disjuntos \ \Rightarrow m^*(O)=\sum_{1}^{\infty}\abs{I_k}\]
\item \[m^*(S) = inf \lbrace m^*(O)\tq S \subset O \ abierto \rbrace\]
\item Dado un conjunto $S \subset I$  numerable, $m^*(S)=0$
\item Dado un intervalo $I_k \Rightarrow M^*(I_k)=\abs{I_k}$


\end{enumerate}

De la tercera propiedad puedes extraerse que $ m^*(S) = m^*(O)$
\begin{proof}
\begin{enumerate}
\item Es trivial. Puesto que $S_1$ se contiene en $S_2$ todo recubrimiento de $S_2$ será recubrimiento de $S_1$ por tanto, al tomar el ínfimo de todos los recubrimientos de $S_1$ tendremos un resultado, al menos, tan pequeño como el ínfimo de los recubrimientos de $S_2$, es decir:
\[m^*(S_1) \leq m^*(S_2)\]
\item Si tenemos expresado O como unión de abiertos disjuntos, no puede haber un recubrimiento menor que el formado por esos mismos abiertos.
\item Con un argumento similar al empleado para demostrar la primera propiedad, queda claro que
\[m^*(S) \leq inf \lbrace m^*(O)\tq S \subset O \ abierto \rbrace \]

Supongamos que es posible $m^*(S) < inf\left\{m^*(O)\right\}$.

En este caso, si tomo un $a$ tal que $ m^*(S) < a < inf\left\{ m^*(O)\right\}$ entonces existirían unos $\lbrace I_k\rbrace$ tales que $m^*(S) \leq \sum \abs{I_k} \leq a$, ya que si no el ínfimo sería $a$.

Pero por ser los $I_k$ abiertos su unión es un abierto, que llamaremos $O_1$. Entonces $m^*(O_1) \leq a$ pero esto nos lleva a una contradicción ya que habíamos dicho que $a < inf\left\{m^*(O)\right\}$.

Por tanto queda claro que no puede ser $m^*(S) < inf\left\{m^*(O)\right\}$.
\end{enumerate}
\end{proof}
%TODO COMPLETAR LA DEMOSTRACIÓN
\begin{example}
	$m^*(a,b) = b - a$
\end{example}

\begin{example}
	$m^*\{a\} = 0$, porque tomaremos el intervalo más pequeño que contenga al punto a. Este intervalo puede ser de tamaño $\epsilon$.
\end{example}

Dada una familia de conjuntos $\lbrace S_n\rbrace_{n=1}^{\infty}$  entonces:
\[m^*(\bigcup S_n) \leq \sum m^*(S_n)\]


\begin{defn}[Medida exterior de un compacto]
Dado un conjunto $K$ compacto y una familia $\lbrace I_k \rbrace$ tal que $K \subset \bigcup I_j$ entonces podemos extraer una colección finita $\lbrace I_k^{'} \rbrace$ tal que $K \subset \bigcup_1^N I_k^{'}$
\[m^*(K) = inf\left\{\sum_1^N \abs{I_j^{'}} \tq K\subset \bigcup_1^N I_j^{'} \right\}\]
\end{defn}

\begin{theorem}
Si K es un compacto, dado un $\epsilon > 0, \ \exists O$ abierto, unión finita de intervalos abiertos disjuntos, que recubre a K ($K\subset O$) y tal que
\[m^*(O\smallsetminus K) < \epsilon\]

Es decir, podemos hacer que la diferencia entre el compacto K y el abierto O que lo recubre sea todo lo pequeña que queramos.
\end{theorem}

\begin{proof}
La medida de un conjunto es siempre menor que la medida de los abiertos que lo contienen. Por tanto, existe un abierto A con $K \subset A$ tal que:
\[m^*(K) \leq m^*(A) \leq m^*(K) + \epsilon\]

$A \smallsetminus K = A \cap K^c$ y por ser intersección de abiertos es un abierto y podemos expresarla como unión finita de abiertos disjuntos:
\[A \smallsetminus K = \bigcup_1^{\infty} I_n\]
Y tenemos que:
\[\sum_1^{\infty}\abs{I_n} < \infty\]
puesto que los $I_n$ son disjuntos y completan $A \smallsetminus K$ que era finito

Por ser una suma finita:
\[ \forall \epsilon \ge 0,  \exists N : \sum_{n > N}^{\infty}\abs{I_n} < \frac{\epsilon}{2}\]

Llamo ahora $I_n^{'}=\overline{I_n} \ para \ n=1,2,...$ y tenemos obviamente $\abs{I_n^{'}}=\abs{I_n}$ por ser $I_n^{'}$ el cierre de $I_n$

Le quitamos a A una unión finita de cerrados
\[ O = A \smallsetminus \bigcup_1^N I_n^{'} = A \bigcap (\bigcup_1^{\infty}I_n^{'})^c = \ abierto \]
Por tanto O es una unión finita de abiertos y tiene una medida exterior que es $\frac{\epsilon}{2}$ menor que la de A.

Pero por construcción tenemos que:
\[m^*(A \smallsetminus K) \leq m^{*}(A)-m^{*}(K) \leq \epsilon\]

Y tomando el O que hemos construido llegamos a
\[m^*(O \smallsetminus K) \leq m^*(O)-m^*(K) = m^*(A) - \frac{\epsilon}{2} - m^*(K) \leq \epsilon - \frac{\epsilon}{2} = \frac{\epsilon}{2}\]

Puesto que igual que hemos tomado $\frac{\epsilon}{2}$ podríamos haber tomado cualquier otro cociente de $\epsilon$, queda probado que podemos hacer que $m^*(O\smallsetminus K)$ fuera tan pequeño como quisiésemos.
\end{proof}


\begin{theorem}
Sea I=(a,b), K un compacto contenido en I, entonces se cumple que:
\[m^*(K)+m^*(I \smallsetminus K) = \abs{I}\]
\end{theorem}

\begin{proof}
Por el teorema anterior, dado $\epsilon > 0, \exists O$, unión finita de intervalos abiertos disjuntos tenemos:
\[m^*(O \smallsetminus K) < \epsilon\]

Tenemos que, lógicamente $O \cup I \smallsetminus O = I$

Por tanto
\[m^*(O)+m^*(I \smallsetminus O) = \abs{I}\]

Sabiendo que $I \smallsetminus K  \subset  \left( I \smallsetminus O \right) \cup \left( O \smallsetminus K \right)$, deducimos:
\[m^*\left(I \smallsetminus K\right) \leq m^*\left(I \smallsetminus O\right) + m^*\left(O \smallsetminus K\right)\]

Si usamos que $m^*\left(K\right) \leq m^*\left(O\right)$, a partir de la desigualdad anterior obtenemos:
\[m^*\left(K\right) + m^*\left(I \smallsetminus K\right) \leq m^*\left(O\right) + m^*\left(I \smallsetminus O\right) + m^*\left(O \smallsetminus K\right) \leq \abs{I} + \epsilon\]

\end{proof}

%TODO Patricio ha dado esto como propiedad de la medida, o modificamos las propiedades o cambio esto.
\begin{theorem}
Sea $\lbrace O_n\rbrace_{n=1}^{\infty}$ una colección numerable de abiertos disjuntos. Consideremos el conjunto $O=\bigcup_{n=1}^{\infty}O_n$ entonces:
\[m^*(O)=\sum_1^{\infty}m^*(O_n)\]
Es decir, la medida exterior es numerablemente aditiva sobre abiertos disjuntos.
\end{theorem}

\begin{proof}
Cada $O_n=\bigcup_{k=1}^{\infty}I_n^k$ donde los $I_n^k$ son abiertos disjuntos, por ser los $O_n$ abiertos.

Por tanto O es la unión numerable de intervalos abiertos disjuntos, cada uno de los cuales es a su vez unión numerable de intervalos disjuntos, es decir:
\[O= \bigcup_{n=1}^{\infty}O_n = \bigcup_{n,k=1}^{\infty} I_n^k\]
de donde, como ya hemos demostrado anteriormente, podemos extraer:
\[m^*(O)=\sum_{n,k=1}^{\infty}I_n^k = \sum_{n=1}^{\infty}\sum_{k=1}^{\infty}I_n^k=\sum_{n=1}^{\infty} m^*(O_n)\]
\end{proof}

\begin{theorem}
Dados dos conjuntos $C_1, C_2 \subset I_1$  tales que la distancia entre $C_1$  y $C_2$ es mayor que 0, entonces:
\[m^*(C_1 \cup C_2) = m^*(C_1) + m^*(C_2)\]

Considerando la distancia como:
\[dist(C_1, C_2) =  \underset{ }{inf}\lbrace \abs{x-y}, x\in C_1, y\in C_2 \rbrace \]
\end{theorem}

\begin{proof}
Dado $\epsilon > 0$, tenemos que $m^*(C_1) + m^*(C_2) \leq m^*(C_1 \cup C_2) +  \epsilon$

Si comprobamos que esto se cumple para cualquier $\epsilon$ tendremos la igualdad, puesto que podremos tomar un $\epsilon$ infinitamente pequeño.

Dado $\epsilon > 0 \ \exists O$ abierto tal que $C_1 \cup C_2 \subset O$, entonces, como ya comprobamos, tenemos que:
\[m^*(O) \leq m^*(C_1 \cup C_2) + \epsilon  \ \forall \epsilon\] 

Ahora definimos: 
\[\rho = dist(C_1, C_2)\]
\[O_1 = O \cap \lbrace x \tq dist(x, C_1) < \frac{\rho}{2} \rbrace\]
\[O_2 = O \cap \lbrace x \tq dist(x, C_2) < \frac{\rho}{2} \rbrace\]
Siendo estos dos conjuntos disjuntos y abiertos.

Como $C_1 \subset O_1$ y $C_2 \subset O_2$ por construcción, tenemos:
\[m^*(C_1) + m^*(C_2) \leq m^*(O_1) + m^*(O_2) \underbrace{=}_{por \ ser \ disjuntos} m^*(O_1 \cup O_2) \leq m^*(O) < m^*(C_1 \cup C_2) + \epsilon \ \forall \epsilon\]
Por tanto 
\[m^*(C_1) + m^*(C_2) \leq m^*(C_1 \cup C_2) + \epsilon \ \forall \epsilon \]
\end{proof}

\begin{theorem}
La medida exterior es numerablemente aditiva sobre los compactos, es decir:
\[\lbrace K_i \rbrace_{i=1}^{\infty}\ compactos \ disjuntos \ dos \ a \ dos \Rightarrow m^*(\bigcup_1^{\infty}K_i) = \sum_1^{\infty}m^*(K_i)\]
\end{theorem}

\begin{proof}
La demostración es simple, por ser compactos y disjuntos dos a dos, sabemos que dados dos conjuntos :
\[m^*(K_1) + m^*(K_2) = m^*(K_1 \cup K_2)\]
Y esto puede ser extendido a cualquier número de conjuntos N:
\[m^*(\bigcup_1^{N}K_i) = \sum_1^{N}m^*(K_i)\]
Puesto que esta unión finita se contiene en una unión infinita de esos conjuntos, tenemos:
\[m^*(\bigcup_1^{N}K_i) \leq m^*(\bigcup_1^{\infty}K_i)\]
Y por tanto:
\[m^*(\bigcup_1^{\infty}K_i) \geq \sum_1^{N}m^*(K_i)\]

Pero, ya habíamos visto que:
\[m^*(\bigcup_1^{\infty}K_i) \leq \sum_1^{\infty}m^*(K_i)\]
de donde podemos deducir la igualdad
\end{proof}

\section{Medida interior}
\begin{defn}[Medida interior ($m_*$)]
Dado un conjunto $C \subset I$ la medida interior de C es la diferencia entre la medida de I y la medida exterior del ``complementario'' de C respecto a I.
\[m_*(C) = \abs{I} - m^*(I \smallsetminus C)\]
\end{defn}

\begin{lemma}
Si K es compacto la medida exterior y la interior coinciden. 
\end{lemma}
\begin{proof}
Por ser K compacto sabemos que:
\[m^*(K) + m^*(I\smallsetminus K) = \abs{I}\]
De aquí podemos despejar la medida superior obteniendo la misma fórmula que para la medida inferior, de donde podemos deducir la igualdad.
\end{proof}

\begin{theorem}
\[m_*(C) \leq m^*(C)\]
\end{theorem}
\begin{proof}
Está claro que:
\[C \cup (I \smallsetminus C) = I\]
Por tanto tenemos que:
\[\abs{I} \leq m^*(I \smallsetminus C) + m^*(C) \Rightarrow m^*(C) \geq \abs{I} - m^*(I \smallsetminus C) \geq m_*(C)\]
\end{proof}

\begin{lemma}
La medida interior también puede definirse como:
\[m_*(C) = \sup \lbrace  m^*(K) \tq  K \ compacto, \ K \subset C \rbrace \]
\end{lemma}
\begin{proof}
Dado $\epsilon \geq 0, \exists K \subset C$ tal que $m^*(K) \geq m_*(C) - \epsilon$

Ya que
\[m^*(K) \leq m^*(C) \leq m^*(K) + \epsilon 
\Rightarrow m^{*}(K) \geq m^*(C) - \epsilon \geq m_*(C) - \epsilon\]
Por tanto $m_*(C) = \sup\left\{m^*(K)\right\}$
\end{proof}

\begin{defn}[Medibilidad]
Un conjunto $C \subset I$ es medible cuando
\[m_*(C)=m^*(C)\]
y entonces escribimos que la medida $m(C)$ es uno de esos dos valores.
\end{defn}

\begin{lemma}
\[m_*(\bigcup_{n=1}^{\infty}K_n) \geq m_*(\bigcup_{n=1}^{N}K_n)= m(\sum_{n=1}^{N}K_n) \geq \sum_{n=1}^{\infty}m(K_n)-\epsilon = m^*(\bigcup_{n=1}^{\infty}K_n) - \epsilon\]
Con esto hemos probado que la suma interior es mayor o igual que la exterior menos un $\epsilon$, para cualquier $\epsilon$ y por tanto es cierta para $\epsilon = 0$. Es decir, la suma interior de una unión numerable de compactos es mayor que la suma exterior.

Pero ya sabemos que esto ocurre también al revés. 

Por tanto, podemos comprobar que son iguales, lo que implica que la unión numerable de compactos disjuntos dos a dos es medible.
\end{lemma}

\obs Un conjunto es medible si y sólo si su complementario es medible.

\end{document}



