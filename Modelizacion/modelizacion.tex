\documentclass{apuntes}

\newcommand{\theauthor}{}
\newcommand{\thetitle}{Modelización\\ Porque los apuntes de Chamizo son demasiado fáciles}
\newcommand{\rightheader}{Modelización}
\newcommand{\leftheader}{UAM - 2014/2015}

% Si no compila y el directorio tikzgen está creado, quitar estas dos sentencias.
%\precompileImages
%\precompileTikz

\title{Modelización}
\author{Guillermo Guridi Mateos\\Cristina Kasner Toruné}
\date{Curso 2014 - 2015 C2}

% Paquetes adicionales

% --------------------

\begin{document}
\pagestyle{plain}
\maketitle
%\abstract{Porque los apuntes de Chamizo son demasiado fáciles}

\tableofcontents
\newpage
% Contenido.

\chapter{Introducción}

\section{¿De qué trata la modelización?}

La modelización trata de dar descripciones de sistemas (situaciones) reales con un lenguaje matemático. Resulta particularmente últil en Física.

Entre los siglos XVII y XVIII se compilaban tablas astronómicas muy precisas con medidas acerca de los ángulos con los que podía verse cada planeta en un instante determinado. Las tablas más precisas fueron creadas por T.Brahe y eran realmente codiciadas por Kepler.

Kepler desarrolló un modelo matemático que sostenía, entre otras cosas, que los planetas se movían en elipses. Kepler estaba muy interesado en obtener las tablas de T.Brahe pues quería analizarlas con el fin de poder probar sus teorías. Finalmente, con la muerte de su creador, Kepler las heredó y pudo establecer una relación entre los cuadrados de los períodos y el cubo de los radios de giro de los planteas.

Posteriormente llegó Newton, que modelizó el movimiento de los planetas con la fórmula:

$$F = - \frac{GMm}{r^2}$$

Esta fórmula en su momento tuvo una gran importancia filosófica, pues permitía explicar toda la teoría acerca del movimiento de los planetas a partir de una ecuación muy simple. No obstante, la utilidad matemático-física de esta ecuación en el momento de su descubrimiento era prácticamente nula.

Newton no conocía el valor de G y si consideramos las interacciones entre los planetas la fórmula se complica mucho. Sin embargo, esta fórmula seguía (y sigue) suponiendo una aproximación bastante acertada.

Así funciona la modelización, a partir de datos surge una idea o explicación matemática que más tarde es complementada con un modelo matemático.


\chapter{Gravitación y leyes de kepler}
\section{Leyes de Kepler}

Kepler, a principios del siglo XVII, enunció unas leyes experimentales (con datos de Tycho Brahe):

\begin{lemma}[1ª Ley de Kepler]
	Las órbitas de los planetas son elipses con el Sol en uno de los focos.
\end{lemma}

\begin{lemma}[2ª Ley de Kepler]
	La linea que une un planeta y el Sol barre áreas iguales en tiempos iguales.
\end{lemma}

\begin{lemma}[3ª Ley de Kepler]
	El cuadrado del periodo orbital es proporcional al cubo del semieje mayor de la órbita.
\end{lemma}

\section{Newton}

Gracias a Newton, en 1985, las leyes de Kepler pudieron demostrarse matemáticamente suponiendo que la fuerza gravitacional obedece la ecuación
\[ F = \frac{-GMm}{r^2} \]
en la dirección del vector radio.

No existe una forma sencilla de realizar esta demostración a menos que demos por sentados muchos principios físicos que no eran conocidos en aquella época. Vamos a tratar de realizar la demostración utilizando los mínimos ``trucos'' posibles.

Para empezar debemos recordar que toda fuerza puede calcularse como:
$$ \overrightarrow{F} = m\ga$$

La aceleración se calcula a partir de la posición de una partícula, que viene dada por: $\vr = (x(t), y(t), z(t))$. La aceleración se define como la derivada segunda de la posición con respecto al tiempo. Es decir:
\[ \ga = \frac{d^2\vr}{d t^2} \]

Esta fórmula describe, por ejemplo, la atracción que ejerce el Sol sobre la Tierra, por ello el signo negativo. En el caso de el Sol y la Tierra, la fuerza recíproca no se tiene en cuenta ya que el Sol es demasiado pesado como para ser influido apreciablemente por la gravedad de la Tierra. Hay que hacer mediciones muy finas para poder detectar estas perturbaciones.

Para poder seguir con la demostración de forma relativamente sencilla, \textbf{despreciamos la fuerza de los planetas sobre el Sol y de los planetas entre ellos.}

Por lo tanto, modelamos tomando sólo un planeta y asumiendo un Sol fijo.

Igualando las dós fórmulas para el cálculo de la fuerza que hemos indicado, obtenemos:

$$ \frac{-GMm}{{||r||}^2} \frac{\vr}{{||r||}^2}  =  m \frac{d^2\vr}{d t^2}$$


En primer lugar vamos a probar que las soluciones $\vr = \vr(t)$ de esta ecuación diferencial satisfacen las leyes de Kepler (con las condiciones iniciales de los planetas)

%Esto es un sistema con una llave
$$
\begin{cases}
 x'' = \frac{-GMx}{(x^2 + y^2 + z^2)^{3/2}}\\
 y'' = \frac{-GMy}{(x^2 + y^2 + z^2)^{3/2}}\\
 z'' = \frac{-GMz}{(x^2 + y^2 + z^2)^{3/2}}\\
\end{cases}
$$


 \begin{obs}
 No se pueden calcular explícitamente $x$, $y$ y $z$, en función de $t$. $\vr = \vr(t)$ no tiene una fórmula cerrada (en términos de funciones elementales\footnote{Las de la calculadora (sin, cos...)}). Aun así se puede demostrar que la fórmula es una elipse.
 \end{obs}


 %Un método es un truco que sirve varias veces
Vamos a probar que la curva $\vr = \vr(t)$ está contenida en un plano.

Un posible método es usar que una curva es plana si y sólo $τ \equiv 0$, donde τ es la torsión de la curva. Dicha fórmula involucra un determinante con una fila $r$ y otra $r''$, como son proporcionales, la torsión sale 0.


Nuestro método más sencillo se basa en tomar $\overrightarrow{L}(t) = \vr(x) \times \frac{d\vr}{dt}$.

Derivando obtenemos:
$$\frac{d\overrightarrow{L}(t)}{dt} = \frac{d\vr}{dt} \times \frac{d\vr}{dt} + \vr(t) + \frac{d^2\vr}{dt^2} = \overrightarrow{0}$$

de donde podemos deducir que $\overrightarrow{L}$ es un vector constante $\Rightarrow \vr \perp \overrightarrow{L} \Rightarrow \vr(t)$ está contenida en el plano $\overrightarrow{L}(x,y,z) = 0 $.
(pensar $\overrightarrow{L} = \overrightarrow{0}$).

Por comodidad, vamos a girar el plano en que se encuentra la órbita del planeta, $\vr \rightarrow G\vr$, siengo G un giro (matriz ortogonal), para convertirlo en el plano $z=0$.

Podemos comprobar que este giro no afecta a las cuentas puesto que, en definitiva, estamos multiplicando por una constante.

$$\frac{d^2\vr}{dt^2} = \text{cte}\frac{\vr}{||\vr||^3} \Leftrightarrow \frac{d^2(G\vr)}{dt^2} = \text{cte}\frac{G\vr}{||G\vr||^3}$$

En definitiva "girando la cabeza" (aplicando el cambio $\vr \rightarrow G\vr$) podemos suponer que la curva $\vr = \vr(t)$ está contenida en el plano $z = 0$, así que simplemente supondremos que $z(t) = 0$. El sistema nos quedaría así:


%Esto es un sistema con una llave
$$
\begin{cases}
 x'' = k\frac{x}{(x^2 + y^2)^{3/2}}\\
 y'' = k\frac{y}{(x^2 + y^2)^{3/2}}\\
\end{cases}
$$

\subsection{Demostración de las leyes de Kepler}

Vamos a probar las leyes de Kepler para $t \rightarrow (x(t), y(t))$ utilizando los resultados de Newton.

La forma de la solución será una elipse.
Si escribimos $x$ e $y$ en polares, el sistema nos quedará más sencillo.

% % align, gather y equation.
$$\begin{array}{c}
x = r\cos(\theta)\\
y = r\sin(\theta) \\
r = r(t) \\
\theta = \theta(t)
\end{array}
$$

(r y $\theta$ no se pueden calcular eplícitamente.)

Y con esto sacamos las ecuaciones de gravitación:
\begin{gather}
r''\cos\theta - 2r'\theta'\sin\theta - r(\theta')^2\cos\theta - r\theta''\sin\theta = cte \frac{\cos\theta}{r^2} \label{eq1_kepler}\\
r''\sin\theta + 2r'\theta'\cos\theta - r(\theta')^2\sin\theta + r\theta''\cos\theta = cte \frac{\sin\theta}{r^2} \label{eq2_kepler}
\end{gather}
$\eqref{eq1_kepler}\cdot\cos\theta + \eqref{eq2_kepler}\cdot\sin\theta \rightarrow K= r^3(\theta')^2 - r^2*r'' \rightarrow$ 1ª ecuación de gravitación
$$-\eqref{eq1_kepler}\cdot\sin\theta + \eqref{eq2_kepler}\cdot\cos\theta \rightarrow 0= r \theta'' + 2r' * \theta' \rightarrow \text{2ª ecuación de gravitación}$$


\subsubsection{2ª ley de Kepler}
\paragraph{Recordemos:}
2ª ley de Kepler : Un planeta recorre áreas iguales en tiempos iguales.

Vamos a estudiar cual es la fórmula para el área.
Primero nos vamos al caso general, para cualquier curva que tengamos en polares.

En este caso el radio depende de $\theta$ \\
Cogemos una curva R.\\

$$A(R) = \int\int_R 1 \dif x \dif y = \int_0^{\theta_0} \int_0^r(\theta) r \dif r \dif \theta = 1/2 \int_0^{\theta_0} r^2(\theta) \dif\theta$$

La idea es coger pequeños triángulos e ir calculando su área

En el caso de la \textbf{ley de Kepler}, si en el tiempo t=0 estamos en $\theta = 0$ y en el tiempo T estamos en $\theta_0 \implies$ la fórmula para el área en función de T sería:

$$A(T) = \frac{1}{2} \int_1^T r^2(t) \theta'(t) \dif t$$


de donde podemos deducir que $\frac {\dif A(T)}{\dif T} = cte$

Es lo mismo que decir $A(T)$ es lineal (a tiempos iguales tengo áreas iguales)

\[A(x + T) - A(x) = A(T) - A(0)\]

Vamos a traducir la 2ª ley de Kepler en algo más familiar (derivadas):

\[\frac{\dif A}{\dif T} = cte \iff r^2\theta' = cte  \text{ (no depende de T) }\]
que puede verse como la conservación del momento angular

Además, $r^2\theta' = cte  \implies (r^2\theta')' = 0 \iff 2rr'\theta' + r²\theta'' =0$ con lo que obtenemos la 2ª ecuación de gravitación.

Por tanto, queda probada la 2ª ley de Kepler.

\subsubsection{1ª ley de Kepler}
Ahora vamos a intentar manipular la 1ª ecuación para probar la \textbf{1ª ley de Kepler} (las óbitas de los planetas son elipses).

Escribimos:
\[h= r^2\theta'=cte \text{ (para cada planeta, } h \text{ es diferente)}\]

1ª ecuación de gravitación:
\[K= r^3(\theta')^2 -r^2*r''\]

como $h = r^2*\theta'$ podemos eliminar $\theta'$ en la 1ªecuación, obteniendo:

$$K= r^3 \frac{h^2}{r^4} - r^2*r''$$
y, simplificando, llegamos a:
$$K= \frac{h^2}{r} - r^2*r''  \rightarrow  \text{Es una EDO ($r = r(t)$) y no aparece explícitamente la t.}$$

\obs En el curso de EDO se ven métodos para pasar de 2 orden ($r''$) a 1 orden ($r$) y se podría expresar r en términos de la imagen inversa de una integral, pero no se puede calcular en términos elementales.

Para resolver la ecuación debemos estar hábiles y realizar un cambio de variable adecuado. En este caso, el cambio de variable consistirá en escribir $r=r(\theta)$ en vez de $r=r(t)$

Hacemos el siguiente cambio de variable:
$$U(\theta(t)) = \frac{1}{r(t)} \implies U'\theta' = \frac{-r}{r^2} \stackrel{h=r^2\theta' }{\iff} hU' = -r' \stackrel{U(\theta(t))}{\implies} hU''\theta' = -r''$$

Sustituyendo en la EDO:

$$K= h^2U + h^2U'' \rightarrow \text{ecuación del movimiento armónico simple}$$
viendo que $\cos'' = -\cos$ y $\sin'' = -\sin$:

$$U'' + U = \frac{K}{h^2} \implies U= \frac{K}{h^2} + \lambda\cos\theta + \mu\sin\theta \iff U= \frac{K}{h^2} + A\cos(\theta - \theta_0)$$
\begin{mdframed}
\obs Al último resultado hemos llegado utilizando que
$$ \lambda\cos\theta + \mu\sin\theta = \sqrt{\lambda^2 + \mu^2}\left(\frac{\lambda}{\sqrt{\lambda^2 + \mu^2}}\cos\theta  +\frac{\mu}{ \sqrt{\lambda^2 + \mu^2}}\sin\theta\right)$$
LLamamos $\cos\theta_0$ a $\frac{\lambda}{ \sqrt{\lambda^2 + \mu^2}}$ y $\sin\theta_0$ a $\frac{\mu}{ \sqrt{\lambda^2 + \mu^2}}$ ya que ambos están entre 0 y 1.
Nos queda que
$$\lambda\cos\theta + \mu\sin\theta = \sqrt{\lambda^2 + \mu^2}(\cos\theta_0\cos\theta + \sin\theta_0\sin\theta) \stackrel{A=\sqrt{\lambda^2 + \mu^2}}{=} A\cos(\theta - \theta_0)$$
\end{mdframed}
Con un giro podemos suponer $\theta_0 = 0$, porque un giro en polares no es más que sumarle una constante al ángulo ($\theta \rightarrow \theta + cte$)

Volviendo al cambio de variable inicial, tenemos
\[U = \frac{1}{r} \implies r=\frac{h^2/K}{1 + B\cos\theta}\]

\textbf{Hecho matemático:} La ecuación general de una cónica en coordenadas polares centradas en un foco con la "orientación habitual" es:

$$r(\theta) = \frac{l}{1+e\cos\theta} \rightarrow
\begin{cases}
l>0\\
e\text{ (excentricidad) }\ge0\\
\end{cases}
$$

\textbf{Recordemos:}

$$\text{excentricidad} \rightarrow
\begin{cases}
0 \le e \le 1 \rightarrow \text{elipse , caso particular} (e=0) \rightarrow \text{circunferencia}\\
e>1 \rightarrow \text{hipérbola(masas no capturadas por el sol)}\\
e=0 \rightarrow \text{parábola}
\end{cases}
$$
Los \textbf{datos astronómicos} para los planetas(excepto mercurio) muestran $B<0.1$ por lo tanto las órbitas son elipses(1ª ley de Kepler) y parecidas a circunferencias.

\begin{center}
\begin{tabular}{| c | c |}
	\hline
	Planetas & B \\
	\hline
	Tierra & 0.016 \\
	\hline
	Venus & 0.0067 \\
	\hline
	Mercurio & 0.20\\
	\hline
\end{tabular}
\end{center}


\subsubsection{3ª ley de Kepler}
%(Dibujo de la elipse con los ejes)
Tomamos $a$ como el semieje mayor y $b$ como semieje menor de la elipse de forma que:
$$2a = \frac{l}{1+e} + \frac{l}{1-e} \implies a= \frac{l}{1-e^2}$$
y
$$b = \sqrt{al}$$
Ahora vamos a calcular área de la elipse basándonos en la fórmula conocida para el área de la circunferencia.

Escribimos la circunferencia como
\[\frac{x^2}{a^2} + \frac{y^2}{a^2} = 1\]
y su área es
\[\pi a^2\]

Si hacemos en cambio de
\[y \rightarrow \frac{a}{b}y\]
nos queda
\[\frac{x^2}{a^2} + \frac{y^2}{b^2} = 1\]
que es la fórmula de la elipse. Por tanto, el área de la elipse será $\pi ab$.

Ahora calculamos el área de la órbita del planeta:
$$T_o = \text{periodo orbital}$$

$$r= \frac{h^2/K}{1 + B\cos\theta}$$

$$\text{Área de la órbita} = A(T_0) = 1/2 \int_0^{T_0} \underbrace{r^2(t)\theta'(t)}_{\text{h}}\dif t = 1/2hT_0$$

Igualando el área de la órbita con  el área de la elipse tenemos:
$$\frac{1}{2}hT_0 = \pi ab= \pi a \sqrt{al} = \pi a\sqrt{a\frac{h^2}{K}} \implies T_0 = cte*a^{3/2}$$
Y queda demostrada la 3ª ley de Kepler.

\obs Para las demostrciones de las tres leyes, en física se ayudan de dos teoremas:
\begin{enumerate}
\item\textbf{Conservación del momento angular}:
$\overrightarrow{L}$ es cte siendo:
\[\overrightarrow{L} = \overrightarrow{r} \times m \overrightarrow{v} (\overrightarrow{r} \rightarrow  \text{ posición },  \overrightarrow{v} \rightarrow \text{ velocidad}).\]


\item\textbf{Conservación de la energía}:\\
\[\frac{1}{2}m||\overrightarrow{v}||^2 + \frac{GMm}{||\overrightarrow{r}||} \rightarrow \text{ es cte}.\]
\end{enumerate}

\subsection{El cabo suelto de Newton}
A Newton le quedó un cabo suelto que vamos a intentar resolver:

\begin{figure}[hbtp]
	\centering
	\inputtikz{ProblemaFGravitacion}
	\caption{¿Se cumple $\md\vf = \frac{GMm}{r^2}$ ?}
\end{figure}
Vemos que en la fórmula, cuando $r \rightarrow 0 \implies \vf \rightarrow \infty$ y esto , en el caso del dibujo, no es real.

Si una masa M está compuesta por varias masas con densidad constante, $\md\vf = \frac{GMm}{r^2}$, con $r =$distancia al punto medio puede cumplirse.

El problema es ver cómo se comporta $\vf$ cuando la densidad de M no es constante.......\\
(.......)\\
Aqui va razonamiento que no he entendido bien y que me va a explicar Miguel cuando pueda.
El caso es que puedo intentar calcular la fuerza que ejerce el Sol sobre la Tierra igual que si fuera sobre una partícula con densidad uniforme.\\
\newpage
\begin{figure}[hbtp]
\centering
\inputtikz{KasnerNoSabePonerNombresANada}
\caption{Divido la esfera con masa M en pequefas particulas $\partial M$}
\label{figaaa}
\end{figure}

Entonces para ver si $\md\vf = \frac{GMm}{r^2}$  es válida, habrá que calcular una integral triple complicada y ver el efecto que tiene en cada partícula.
$$\int\frac{-Gm\overrightarrow{x}}{\md {\overrightarrow{x}}^3}\dif M$$

Hay una manera de calcular esta integral y probar que  $\md{\overrightarrow{\vf}}$ es válida para objetos con simetría esférica, usando el teorema de la divergencia:

\begin{prop}[Ley de Gauss]
	Dada una región sólida B y un campo $\overrightarrow{E} = \frac{\overrightarrow{x}}{\md {\overrightarrow{x}}^3}$ tenemos que:
	\[\int_{\partial B} \overrightarrow E \dif \overrightarrow S =
	\left\{ \begin{array}{lcc}
	     4\pi & si & \overrightarrow 0 \in Int(B) \\
	  \\ 0 & si & \overrightarrow 0 \notin Int(B)
	\end{array} \right.\]
	Siendo $\partial B$ la frontera de B
\end{prop}
\begin{proof}
	\begin{itemize}
	\item Si $\overrightarrow 0$ está fuera de B, el campo $\overrightarrow E$ es regular ($C^{\infty}$) en B.\\
	$$\int_{\partial B} \overrightarrow E \dif \overrightarrow S \stackrel{Tª divergencia}{=} \int_B div \overrightarrow E = 0$$

	\item Si $\overrightarrow 0$ pertenece a Int(B):\\
	En lugar de B consideramos $B - B_{\delta}$ con $B_{\delta} = \{ \md{\overrightarrow x} \le \delta \} $\\
	$$\overline{B} = B- B_\delta \rightarrow \int_{\partial {\overline{B}}} \overrightarrow E \dif \overrightarrow S = \int_B div \overrightarrow E = 0$$
	Como $\partial{\overline{B}} = \partial B \cup \partial{B_\delta}$ con orientaciones distintas $\rightarrow$ $$ \int_{\partial B} \overrightarrow E \dif \overrightarrow S - \int _{\partial{B_\delta}} \overrightarrow E \dif \overrightarrow S = 0$$
	$$ \int_{\partial B} \overrightarrow E \dif \overrightarrow S = \int _{\partial{B_\delta}} \overrightarrow E \dif \overrightarrow S \overbrace{=}^{esfericas} \int_0^{\pi} \int_0^{2\pi} \frac{\delta ^2 \sin\theta}{\delta ^3} \delta \dif \phi \dif \theta = 4\pi$$
	\end{itemize}
\end{proof}
\newpage
Por lo tanto, tenemos que:

\begin{figure}[hbtp]
	\centering
	\inputtikz{ProbGravitacion2}
	\caption{No hace falta que B y B' sean regulares, pero aún no se dibujar otra cosa}
\end{figure}
$$\overrightarrow F = - \frac{GMm}{\md{\overrightarrow x - \overrightarrow y}}(\overrightarrow x - \overrightarrow y)$$
 Si $\overrightarrow y $ se mueve en $B'$
 $$\int_{\partial B} \overrightarrow F \dif \overrightarrow S =
 \begin{cases}
 0\\
 -4\pi GMm\\
 \end{cases}
 \rightarrow \text{Dependiendo de si} \overrightarrow x \text{está fuera o dentro de B'}$$

 Si tenemos muchas masas, también es cierta la proposición.\\

\begin{figure}[hbtp]
	\centering
	\inputtikz{ProbGravitacion3}
	\caption{$\int_{\partial B} \vf \dif \vec{S} = -4\pi GMm $ Con M suma de las m en el interior}
\end{figure}
\newpage
\paragraph{Conclusión}
En el caso de los planetas, $\overrightarrow F$ tiene que ser una función radial (por la simetría de la esfera), si cogemos B una "esfera ficticia" de radio $r = \md{\overrightarrow y} \implies$
\begin{figure}[hbtp]
	\centering
	\inputtikz{ProbGravitacion4}
\end{figure}
$$\int_{\partial B} \overrightarrow F \dif \overrightarrow S = 4 \pi r^2 |\overrightarrow F|$$
Como $\int_{\partial B} \overrightarrow F \dif \overrightarrow S = -4\pi GMm  \implies$
$$ - 4\pi r^2 |\overrightarrow F| = -4\pi GMm \implies |\overrightarrow F| = \frac{GMm}{r^2}$$

Por lo tanto, $|\overrightarrow F| = \frac{GMm}{r^2}$ , que es la fórmula de Newton, es válida con objetos con simetría radial y por lo tanto válida para los planetas.



\subsection{Un poco de mecánica analítica}
Lagrange escribió un libro llamado "Mecánica Analítica" publicado en 1788. Este libro describía la mecánica de manera muy matemática. Entre sus ventajas estaba la unificación de la mecánica. Todo se regía en base a un principio con adición de teoremas y era muy matemático. Además la mecánica sufrio una simplificación, como por ejemplo la libretad para escoger las coordenadas libremente.

\begin{prop}[Cálculo de variaciones]
Dada L regular de $2n + 1$ variables. Supongamos que la integral:

 $$\int_a^b L(q_1(t),...,q_n(t),\dot{q_1}(t),...,\dot{q_n}(t),t) dt$$

 (donde $\dot{q_i} = \frac{dq_i}{dt}$.)

 alcanza un máximo o mínimo para ciertas $q_i 1\leq i \leq n$ regulares con valores $q_i(a)$, $q_i(b)$ fijados. Entonces dichas $q_i$ satisfacen:


\begin{op}{Ecuaciones de Euler-Lagrange}
 \frac{d}{dt}(\frac{\sigma L}{\sigma \dot{q_i}}) = \frac{\sigma L}{\sigma q_i}    1 \leq i \leq n
\end{op}

Además si L no depende de la última variable (la t suelta) la energúa $ E = \sum_{i=1}^n \dot{q_i} \frac{\sigma L}{\sigma \dot{q_i}} - L$ es constante para las $q_i$ solución.

Las propiedades minimizantes o maximizantes no dependen de las coordenadas elegidas (polares, cartesianas, etc…).

\end{prop}

En los  siguientes ejemplos damos por supuesto la existencia:

\begin{example}
Hallar la curva $\sigma (t) = (t, y(t))$ más corta que conecta $(0,0)$ y $(1,1)$.

(FIGURA)

$$ \text{longitud} = \int_0^1 || \sigma'(t) || dt = \int_0^1 \sqrt{1 + (y'(t))^2} dt$$

$$y = q \; y' = \dot{q} \; n=1 \; L = \sqrt{1 + \dot{q}^2}$$

Pasamos a las ecuaciones de Euler-Lagrange:

$$\frac{d}{dt}(\frac{\sigma L}{\sigma \dot{q}}) = \frac{\sigma L}{\sigma q} \; \frac{d}{dt}(\frac{2 \dot{q}}{2 \sqrt{1 + \dot{q}^2}}) = 0$$

$$\Rightarrow \frac{q}{\sqrt{1 + \dot{q}^2}} = \text{cte} \Rightarrow \frac{\dot{q}^2}{1 + \dot{q}^2} = \text{cte} \Rightarrow \dot{q} = \text{cte} \Rightarrow q(t) = At + B $$

$$ y = 1 \; q(0) = 0 \; q(1) = 1 \Rightarrow q(t) = y(t) = t \Rightarrow \sigma (t) = (t,t) \text(una recta)$$

\begin{obs}
$$E = \dot{q} \frac{\sigma L}{\sigma \dot{q}} - L \Rightarrow \text{cte} = \dot{q} \frac{\dot{q}}{\sqrt{1 + \dot{q}^2}} - \sqrt{1 + \dot{q}^2} = \frac{-1}{\sqrt{1 + \dot{q}^2}} \Rightarrow \dot{q} = \text{cte}$$
\end{obs}

\end{example}




\begin{example}
(Braquistocrona, J.Bernouilli, Newton, Euler, Lagrange) Hallar el tobogán más rápido que une $(0,0)$ y $(\frac{\pi}{2},-1)$.

FIGURA

h = altura medida desde el origen.

Vamos a parametrizar la curva en función de la altura. Ponemos la x en función de la y.


En mecánica la velocidad vertical viene dada por:

$$\frac{dh}{dt} = \frac{\sqrt{2gh}}{\sqrt{1+(x'(h))^2}}$$

Para calcular el tiempo total integramos:

$$\int_0^1 \frac{dt}{dh} dh = \int_0^1 \frac{\sqrt{1 + (x')^2}}{\sqrt{2gh}} dh \;\; g = 9.8$$

$$ n=1 \; t=h \; q=x \;  \dot{q}=x' \; L = \frac{\sqrt{1+\dot{q}^2}}{\sqrt{2gt}} $$


%Aquí hay un = 0 encima de la fracción \frac{\sigma L}{\sigma q}  la primera vez que aparece
$$\frac{d}{dt} (\frac{\sigma L}{\sigma \dot{q}}) = \frac{\sigma L}{\sigma q} \; \frac{\sigma L}{\sigma \dot{q}} = \text{cte} \Rightarrow \frac{\dot{q}}{\sqrt{t(1+\dot{q}^2)}} = \text{cte}$$

$$ \Rightarrow^{\text{EDO (un poco dificil)}} q(t) = \frac{u - \sin(u)}{2} \; t \sin^2(\frac{u}{2}) $$

$$ \Rightarrow \text{Tobogán:} \sigma (u) = ( \frac{u - \sin{u}}{2}, -\sin^2(\frac{u}{2}) ) $$

(Figura cicloide)




\end{example}





%% Apéndices (ejercicios, exámenes)
\appendix

\chapter{---}
% -*- root: ../modelizacion.tex -*-
\section{Hoja 1}
\begin{problem}[1]
Una \textbf{fuerza central} es una fuerza que en $\overrightarrow{x}$ tiene módulo que sólo depende de $||\overrightarrow{x}||$. Por tanto $\overrightarrow F=m\cdot \overrightarrow a$ lleva a una ecuación del tipo
\[\overrightarrow x '' =g(||\overrightarrow x||)\overrightarrow x\]
Calcula la derivada de $\overrightarrow x\times \overrightarrow  x''$ (el momento angular) y deduce de ello que cada curva solución está contenida en un plano. Otra forma (más complicada) de proceder es probar directamente que la torsión de la curva es nula. Investiga este procedimiento usando la fŕomula de la torsión

\solution
	Definimos la fuerza central (la fuerza que va del origen hasta x) de la siguiente forma:$$\overrightarrow F = g(||\overrightarrow x||) \overrightarrow x$$

	Como $\overrightarrow F = m \cdot \overrightarrow a$ entonces $m\overrightarrow x'' = g(||\overrightarrow x||) \overrightarrow x$\\
	De esto último deducimos que $\overrightarrow x \times \overrightarrow x'' = \overrightarrow 0$ puesto que son vectores paralelos.

	Por lo tanto $\frac{\dif}{\dif t}(\overrightarrow x \times \overrightarrow x') = \overrightarrow x \times \overrightarrow x + \overrightarrow x \times \overrightarrow x '' = \overrightarrow 0 + \overrightarrow 0.$

	Y con esto ya tenemos que $\overrightarrow x $ está en un plano ya que:

	$$\overrightarrow x \times \overrightarrow x' = \overrightarrow V_0 \cdot cte \implies \overrightarrow V_0 \cdot \overrightarrow x = 0 \stackrel{V_0 \neq 0}{\implies} \overrightarrow x \text{está en un plano}$$
	¿Qué ocurriría con $V_0 = 0$?

	$x'$ y $x$ serían paralelos, es decir, la velocidad iría en la dirección de x $\implies$ $\overrightarrow x$ está en una recta.

	La explicación de la última implicación se deja como ejercicio.

	Otra forma de ver que $\overrightarrow x$ está en un plano es utilizando la torsión:

	Definimos la curva
	\[t \mapsto \overrightarrow{x(t)}\]
	y la torsión
	\[T = \frac{(\overrightarrow x' \times \overrightarrow x'') \cdot \overrightarrow x'''}{||\overrightarrow x' \times \overrightarrow x ''||}\]

	Por la fórmula de la $\overrightarrow F$ puesdo escribir:
	$$\overrightarrow x'' = \frac{1}{m} g(||\overrightarrow x||)\overrightarrow x $$

	Y si dervivamos:

	$$\overrightarrow x''' = \frac{1}{m} g'(||\overrightarrow x||) \frac{\overrightarrow x \cdot \overrightarrow x'}{||\overrightarrow x||} \overrightarrow x + \frac{1}{m}g(||\overrightarrow x||)\overrightarrow x' = \frac{g'(||\overrightarrow x||)}{g(||\overrightarrow x||)}\cdot \frac{\overrightarrow x \cdot \overrightarrow x'}{||\overrightarrow x||} \overrightarrow x'' + \frac{1}{2}g(||\overrightarrow x||) \overrightarrow x''$$

	Con esto podemos ver que $\overrightarrow x'''$ es combinación lineal de $\overrightarrow x''$ y $\overrightarrow x'$ y por tanto:
	$$(\overrightarrow x' \times \overrightarrow x'' \times \overrightarrow x''') = det(\overrightarrow x', \overrightarrow x'',\overrightarrow x''') = 0 \stackrel{\implies}{\overrightarrow x' \times \overrightarrow x'' \neq 0} T = 0 \implies \text{curva plana}$$
\end{problem}

\begin{problem}[2]

La estación espacial internacional orbita a unos 400 km de la superficie dela Tieraa. Calcula en qué proporción ha disminuido la fuerza de la gravedad a esa altura. ¿Cuánto pesaría una persona de 80kg a esa altura? ¿Por qué entonces las imágenesque nos llegan muestran astronautas y objetos flotando ingrávidos?

\solution
	Queremos ver el peso de una persona a 400 km de la Tierra\\Llamamos M a las masa de la tierra, m al peso de una persona y R al radio de la Tierra.\\ Vemos la relación entre la fuerza que hay sobre la persona en la superficie de la Tierra y la fuerza a 400 km:\\
	$$\frac{\frac{GMm}{R^2}}{\frac{GMm}{(R + 400000 m)^2}} \implies \frac{(R + 4\cdot 10^5)^2}{R^2} = \left(\frac{67.8}{63.8}\right)^2$$
	Por lo tanto, si m = 80, el peso a 400 km sería:
	$$\frac{80}{\left(\frac{67.8}{63.8}\right)^2} = 70 .84 Kg$$

	\textbf{Pregunta del profesor:} Los astronautas están flotando alrededor de la Tierra por la fza.centrífuga. ¿Porqué no ocurre lo mismo en la superficie?
\end{problem}

\begin{problem}[3]
Prueba que la curva en polares $r(\theta)= \frac{a^{-1} \cdot b^2}{1 + e\cos\theta}$ siendo
\[ e = \frac{c}{a} = \sqrt{1 - \frac{b^2}{a^2}}\]
describe la elipse $\frac{x^2}{a^2} + \frac{y^2}{b^2} = 1$ cuando el origen de las coordenadas polares está en uno de los focos.

Prueba también que la curva en polares
\[r(\theta)=\frac{l}{1+e\cos(\theta)}\]
con $e \geq 0$ y $l>0$ constantes, describe una circunferencia si $e=0$, una elipse si $0<e<1$, una parábola si $e=1$ y una hipérbola si $e>1$.

\textbf{Indicación:} Escribe $r(1+e\cos(\theta)$ en cartesianas.

\solution

	Tenemos la elipse:$$r(\theta)= \frac{a^{-1} \cdot b^2}{1 + e\cos\theta} \text{  siendo  } e = \frac{c}{a} = \sqrt{1 - \frac{b^2}{a^2}}$$
		Que es la elipse $$\frac{x^2}{a^2} + \frac{y^2}{b^2} = 1$$ trasladando el origen a uno de sus focos.

		Podemos escribir la ecuación como

		$$ r + er\cos\theta = a^{-1} b^2 \rightarrow
		\begin{cases}
		r = \sqrt{x^2 + y^2}\\
		r\cos\theta = x\\
		\end{cases}$$

		Despejando obtenemos:
		\[(\sqrt{x^2 + y^2})^2 = (\frac{a^{-1}b^2}{1+e\cos\theta})^2 \implies x^2 + y^2 = a^{-2} b^4 + e^2x^2 - 2a^{-1}b^2ex\]
		Como $1-e^2 = \frac{b^2}{a^2}$ tenemos que

		$$\frac{x^2}{a^2}b^2 + y^2 = a^{-2} b^4 - 2a^{-1}b^2ex$$
		Y dividiendo entre $b^2$ y completando cuadrados llego a que:
		$$\frac{(x + a\cdot e)^2}{a^2} + \frac{y^2}{b^2} = 1$$
\end{problem}

\begin{problem}[4]
Sea $V$ el potencial de una fuerza $\overrightarrow F$, esto es $\overrightarrow F = -\nabla V$, que satisface div$\overrightarrow F = 0$. Demuestra que si $V$ es una función radial (sólo depende de la distancia al origen), entonces necesariamente $\overrightarrow F=K||\overrightarrow x||^{-3}\overrightarrow x$, como con la ley de gravitación universal

\solution
\textcolor{blue}{Hecho por De Juan. No fiarse al 100\%}

Podemos descomponer el potencial en una composición de 2 funciones:

$\appl{d}{ℝ^3}{ℝ}$, con $d(x,y,z) = \norm{\overrightarrow{x}}$.

$\appl{f}{ℝ}{ℝ}$.

Con esta construcción, para alguna $f$ tenemos $V(\vx) = f(d(\vx))$.

Si logramos demostrar que $f$ tiene que ser de la forma $f(x) = \frac{-KM}{x}$, ya tendremos el ejercicio hecho puesto que su gradiente nos daría una fuerzo con fórmula como la indicada.

Ahora sólo hay que derivar con cuidado:

$$-\overrightarrow{F} = \grad V(\vx) = f'(d(\vx)) · \grad d(\vx) = f'\left(\norm{\vx}\right)\frac{1}{\norm{\vx}} \vx$$

\obs Hay que prestar especial atención al hecho de que al escribir $\vx$ hacemos referencia a un vector y no a una variable.

Vemos que $\appl{F}{ℝ^3}{ℝ^3}$. Calculamos la divergencia de $\overrightarrow{F}$

$$div \overrightarrow{F} = \frac{\partial{F}}{\partial{x}} + \frac{\partial{F}}{\partial{y}} + \frac{\partial{F}}{\partial{z}}
\overset{hip.}{=} 0$$


Basándonos en la expresión de la fuerza calculada anteriormente tenemos:

$$\frac{\partial{F(\vx)}}{\partial{x}} = \left(\underbrace{\dpa{f'(d(\vx))}{x}}_{(1)} · d(\vx) - f'(d(\vx))\dpa{d(\vx)}{x} \right)\frac{x}{(d(\vx))^2} + \frac{f'(d(\vx))}{d(\vx)}$$

$(1) = \dpa{d(\vx)}{x} · f''(d(\vx))$ por la regla de la cadena.

Simplificando y utilizando el cálculo anterior, obtenemos:

$$\dpa{F(\vx)}{x} = \frac{x^2}{(d(\vx))^3}\left(f''(d(\vx)) d(\vx) - f'(d(\vx))\right) + \frac{f'(\vx)}{d(\vx)}$$

El cálculo es análogo para las derivadas respecto de $y,z$, con lo que la divergencia queda:

$$div \overrightarrow{F} = \left(f''(d(\vx)) d(\vx) - f'(d(\vx))\right) \cdot \frac{x^2 + y^2 + z^2}{(d(\vx))^3} + \frac{3f'(d(\vx))}{d(\vx)}$$

Tomamos $d(\vx) = r$ y utilizamos $x^2+y^2+z^2 = \left(\sqrt{x^2+y^2+z^2}\right)^2 = d(\vx)^2$

$$ div \overrightarrow{F} = 0 = \left((f''(r)r - f'(r)) + 3f'(r)\right) \frac{1}{r} = 0 \dimplies r f''(r) = -2 f'(r)$$

Hemos llegado a una EDO cuya solución es $g(x) = \frac{k}{x}$, porque:

$$g(x) = \frac{k}{x} \implies g'(x) = \frac{-k}{x^2} \implies g''(x) = \frac{2k}{x^3}$$

Vemos que $x g''(x) = x\frac{2K}{x^3} = \frac{2K}{x^2} = -2 \frac{-K}{x^2} = -2 g'(x)$ que es la EDO que teníamos.

Por tanto ya tenemos el ejercicio resuleto, pues la $f$ obtenida coincide exactamente con lo que esperábamos obtener. Veamos por que está $f$ garantiza la fórmula para la fuerza que indica el enunciado.
$$V(\vx) = f(d(\vx)) = \frac{K}{d(\vx)} = \frac{K}{\norm{\vx}}$$

Y para hallar la $\overrightarrow{F}$, utilizamos $$\overrightarrow{F} = - \grad V = -K\left(\frac{-x}{\norm{\vx}^{3}},\frac{-y}{\norm{\vx}^{3}},\frac{-z}{\norm{\vx}^{3}} \right)$$

Y reescribimos : $$\overrightarrow{F} = - \grad V = \frac{K}{\norm{\vx}^3} \vx$$

\end{problem}

\begin{problem}[5]
Explica por qué en el punto más cercano al Sol de la órbita de una planeta, digamos a distancia $r_p$, la velocidad $v_p$ debe cumplir $v_p=r_p\theta'$. Recuerda que en el movimiento de un planeta $h=r^2\theta$ es constante y que $a(1-e^2)=h^2/GM$ con $a$ el semieje mayor y  $e$ la excentricidad. Deduce de todo ello que $v_p=br^{-1}_p\sqrt{GM/a}$

\solution
Para probar que $v_p=r_p\dot{\theta}$ recordamos que estamos tomando $v_p = ||\overrightarrow v||_p$.

Cogemos la fórmula general de $v$ : $$ v = ||(\dot{x}, \dot{y})|| \implies v= \sqrt{\dot{x}^2 + \dot{y}^2}$$
Hacemos el cambio de variables
$\begin{cases}
x = r\cos\theta\\
y = r\sin\theta
\end{cases}$ de forma que
$$v = \sqrt{\dot{r}^2 + r^2\dot{\theta}^2}$$
Como en el punto más cercano al sol la distancia alcanza un mínimo $\implies \dot r = 0$, entonces
$$v= r\dot{\theta} \implies v_p = r_p\dot{\theta}$$

Ahora vamos a deducir que $v_p = \frac{b}{r_p}\cdot \sqrt{\frac{GM}{a}}$

Por el enunciado sabemos que
$$v_p = \frac{h}{r_p} = \frac{\sqrt{GMa(1-e^2)}}{r_p}$$
Solo nos queda probar que $\sqrt{a(1-e^2)}= \frac{b}{\sqrt{a}}$ , o l que es lo mismo
$$a(1-e^2) = \frac{b^2}{a}$$
Utilizamos la relación de la excentricidad (e) con los semiejes (a,b).

Sabemos que $e = \frac{c}{a}$, siendo c la distancia focal $\implies c= \sqrt{a^2-b^2}$ entonces $e = \frac{\sqrt{a^2 - b^2}}{a}$.

Sustituyendo:
$$a(1-e^2) = \frac{b^2}{a} \implies a(1-\frac{a^2 - b^2}{a^2}) = \frac{b^2}{a}$$
Vemos que es cierto, por lo tanto $v_p = \frac{b}{r_p}\cdot \sqrt{\frac{GM}{a}}$
%Esto lo comento por que creo que no vale para nada, pero me da pena tirarlo por si acaso
%Por el principio básico de la mecánica (principio de Hamilton) sabemos que la trayectoria del sistema da un extremo de
%\[\int_{t_0}^T L = \int_{t_0}^T E_{cinetica}-E_{potencial} = \int_{t_0}^T \frac{1}{2}mv^2-\frac{GMm}{r}\]
%Sabiendo que la velocidad es la derivada de la posición y que las órbitas son planas podemos escribir:
%\[\int_{t_0}^T L = \int_{t_0}^T \frac{1}{2}m\left(\dot{x}^2+\dot{y}^2\right)-\frac{GMm}{\sqrt{x^2+y^2}}\]
%Puesto que no depende de $t$ tenemos que la energía es constante. Es decir:
%\[E = \dot{x}\frac{\partial L}{\partial \dot{x}}+\dot{y}\frac{\partial L}{\partial \dot{y}}-\frac{1}{2}m\left(\dot{x}^2+\dot{y}^2\right)-\frac{GMm}{\sqrt{x^2+y^2}} = cte\]
%Vamos a trabajar esta ecaución a fin de llegar al resultado buscado:
%\[\dot{x}\dot{x}'+\dot{y}\dot{y}'-\frac{1}{2}\left(\dot{x}^2+\dot{y}^2\right)-\frac{GM}{\sqrt{x^2+y^2}} = cte\]
\end{problem}

\begin{problem}[6]
Prueba que si el semieje mayor de la elipse de un planeta es $a$ entonces su velocidad cuando está a distancia $r$ del Sol es $\sqrt{2GM/r-GM/a}$.

\textbf{Indicación:} Utiliza el problema anterior y la conservación de la energía.
\solution
\textbf{Idea:} En $v_p = \frac{b}{r_p}\cdot \sqrt{\frac{GM}{a}}$ (ejercicio anterior) hay información redundante, ya que $b,a\text{ y } r_p$ están relacionadas, no son independientes.

La conservación de la energía me dice que
$$\frac{1}{2}m v^2 - \frac{GMm}{r} \text{ es cte}$$

Como tengo la velocidad en un punto ($v_p$), me queda
$$\frac{1}{2} v^2 - \frac{GM}{r} = \frac{1}{2} v_p^2 - \frac{GM}{r}$$
Entonces
$$v = \sqrt{\frac{2GM}{r} + v_p^2 - \frac{2GM}{r_p}}$$
¿Cómo simplifico esto? mirando lo que queremos demostrar sólo queda probar que:
$$\frac{2GM}{r_p} - v_p^2 = \frac{GM}{a}$$
Sustituimos $v_p$ por lo que teníamos en el ejercicio anterior:
$$\frac{2GM}{r_p} - v_p^2 =\frac{2GM}{r_p} -\frac{GMb^2}{ar_p^2} = \frac{GM}{r_p}(2-\frac{b^2}{a r_p}) $$
Para que la expresión se parezca más a lo que queremos demostrar pensamos qué relación hay entre $r_p$ y a.


DIBUJO ELIPSE:


Como $e = \frac{c}{a}$ y $r_p = a-c$, sustituyendo:$$\frac{GM}{a(a-e) \cdot (2-\frac{b^2}{a^2(1 - e)})}$$
Para terminar bastaría comprobar que
$$\frac{1}{1-e} \cdot (2- \frac{b^2}{a^2(1-e)}) = 1$$
Es fácil viendo que
$$e^2 = \frac{c^2}{a^2} = 1-\frac{b^2}{a^2} \implies b^2 = a^2(1-e^2)$$
Finalmente
$$\frac{1}{1-e} \cdot (2 - \frac{1 - e^2}{1 - e}) = \frac{1}{1-e} \cdot (2-(1 + e)) = 1$$
Y ya tenemos que $v=\sqrt{\frac{2GM}{r}- \frac{GM}{a}}$
\end{problem}

\begin{problem}[7]
El cometa Halley tiene distancias máxima y mínima al Sol dadas por $5.25\cdot 10^{12} m$ y por $8.77\cdot10^{10}$ m, respectivamente. Calcula la fórmula de su elipse en coordenadas cartesianas, su valor de $r^2\theta'$ y sus velocidades máxima y mínima.

\solution
\textcolor{blue}{Hecho por Pedro. No fiarse al 100\%}

Conociendo los dos semiejes podemos afirmar que la ecuación de la óribta es de la forma
\[\frac{x^2}{5.25\cdot 10^{12}}+\frac{y^2}{8.77\cdot 10^{10}} = 1 \text{ y } z=0\]

Para calcular el valor de $r^2 \theta'$ basta con acordarnos de que ese valor era constante ($h$) y que, como recordamos en el ejercicios 5,
\[h = \sqrt{GMa(1-e^2)}\]
de donde conocemos todas las constantes.

Por último, para calcular sus velocidades máxima y mínima podemos apoyarnos en el ejercicio anterior, que nos dice que la velocidad a distancia $r$ del Sol viene dada por la ecuación:
\[\sqrt{\frac{2GM}{r}-\frac{GM}{a}}\]
donde nos basta con sustituir el radio por el máximo y el mínimo y listo.

Otra opción para el cálculo de las velocidades habría sido calcular la derivada de la velocidad y su máximo y mínimo.
\end{problem}

\begin{problem}[8]
La excentricidad de la órbita de la Tierra es aproximadamente $e=0.017$. Si en un libro de texto vemos la órbita dibujada con un eje mayor de $20cm$, ¿cuánto debería medir el eje menor?. Suponemos, consecuentemente, la órbita de la Tierra circular. Si en una galaxia lejana hay un planeta hermano de la Tierra con la misma órbita pero recorrida sólo en tres meses, ¿qué relación hay entre la masa de su estrella y la de nuestro Sol?

\solution
\textcolor{blue}{Hecho por Pedro. No fiarse al 100\%}

Puesto que la excentricidad sea calcula como $e=c/a$ siendo $c$ la mitad de la distancia focal y $a$ el semieje mayor de la elipse, podemos deducir fácilmente que la semidistancia focal es
\[c= 0.34cm\]
y sabiendo ahora que $a^2 = c^2+b^2$, con $b$ es semieje menor, podemos despejar y obtener que
\[b = \sqrt{a^2-c^2}=\sqrt{399.8844} = 19.99711 cm\]

Puesto que la diferencia entre los semiejes de la elipse es mínima, tiene sentido considerar la órbita como circular.

Ya sabemos que podemos calcular el área de la órbita de un planeta como
\[A = \frac{1}{2}h T \text{ siendo } h = cte \text{ depende del planeta }\]
Puesto que nuestro nuevo planeta y la Tierra tienen la misma órbita, aunque la recorran a distintas velocidades, su área será la misma.

Por tanto, puesto que nuestro nuevo planeta tiene un período 4 veces menor, tenemos que $h_N=h_T \cdot 4$.

Si recordamos el ejercicio 1.5 ya vimos que
\[a(1-e^2)=\frac{h^2}{GM}\]
puesto que $a,e,G$ no cambian al pasar de estudiar la Tierra a este nuevo planeta, tenemos que
\[h_N^2=\sqrt{a(1-e^2)GM_{NS}}=4h_T=\sqrt{2a(1-e^2)GM} \implies \sqrt{M_{NS}}=\sqrt{2M}\]
es decir, el nuevo Sol tendrá el doble de masa que el nuestro.

\end{problem}

\begin{problem}[9]
Sabiendo que $GM=3.99\cdot 10^{14}m^3s^{-2}$ con $M$ la masa de la Tierra, calcula a qué distancia de su superficie orbitan los satélites goestacionarios: los que están siempre sobre el mismo punto geográfico porque giran a al par que la Tierra, una vez cada 24 horas.
\solution


\end{problem}

\begin{problem}[10]
Usando la ley de Gauss, prueba que en un planeta esférico homogéneo hueco no hay gravedad en el interior.
\solution

\textcolor{blue}{Hecho por Pedro. No fiarse al 100\%}

Según la ley de Gauss la fuerza de la gravedad en un punto de una superficie será proporcional al flujo del campo gravitatorio a través de dicha superficie.

Una vez tomamos un punto en el interior del planeta hueco, consideramos la esfera que lo contiene y que, a su vez, se contiene dentro del planeta.

Para calcular la fuerza de la gravedad en ese punto (o en cualquiera de la esfera que hemos construido, pues todos comparten el mismo valor de gravedad) basta con calcular el flujo del campo gravitatorio en torno a la esfera.

Puesto que esta esfera está hueca y las líneas de campo vienen desde el infinito hasta la superficie del planeta, no hay líneas de campo dentro del planeta por lo que no hay líneas de campo que atraviesen nuestra esfera, por lo que no hay flujo y por tanto, por el teorema de Gauss, no habrá gravedad.
\end{problem}

\begin{problem}[11]
Newton resolvió el problema anterior con un bello argumento geométrico: Fijado un
punto interior se considera un doble cono que lo tiene como vértice. El cono corta a la superficie interior del planeta en dos regiones que cuando se reducen a tamaño infinitesimal ejercen la misma atracción. Intenta elaborar este argumento hasta que te suene convincente.

\solution

\end{problem}

\begin{problem}[12]
Utilizando las ecuaciones de Euler-Lagrange con coordenada generalizada la distancia desde el punto de partida, halla las ecauciones de movimiento de un objeto de masa $m$ que cae por un plano inclinado de ángulo α partiendo del reposo. La energía potencia gravitatoria es $mgy$ donde $g=9.8ms^{-2}$ e $y$ es la altura.

\solution
\textcolor{blue}{Hecho por Pedro. Fiarse -1}
Tomemos nuestra variable $z(t)$ que define la distancia en línea recta desde el punto de partida hasta la posición actual de la masa.

El Lagrangiano sería:
\[\int_a^b L = \int_a^b E_c-E_p = \int_a^b \frac{1}{2} m \left(z'(t)\right)^2-mgz(t)\cos(α)\]
aplicando las ecuaciones de Euler-Lagrange llegamos a
\[\frac{\partial}{\partial t} mz'(t) = -mg\cos(α)\]
de donde podemos deducir que
\[z'(t)=g\cos(α)t \implies z(t)=\frac{1}{2}g \cos(α)t^2\]
Así, la ecuación del movimiento de la partícula sería de la forma
\[\left(x(t),y(t)\right) = \left(x_0+\frac{1}{4}g\sen(2α)t^2,y_0+\frac{1}{2}g \cos^2(α)t^2 \right)\]

\end{problem}

\begin{problem}[13]
Sea $G_α$ un giro en $\real^3$ de ángulo α alrededor de un eje dado por un vector unitario $\vn$ y sea $f(α)=G_α(\vx)$ para un $\vx\in \real^3$. Prueba que $f'(0)=\vn \times \vx$ y utiliza el teorema de Noether para deducir que si $L=\frac{1}{2}m||\vv||^2-V(||\vx||)$, entonces el momento angular $\vx \times m \vv$ se conserva.
\solution

\end{problem}


\begin{problem}[14]
Sabiendo que entre las superficies de revolución cuyos bordes son $\{x^2+y^2 = 4, \ z= \pm 1\}$ hay una de área mínima, prueba que es $\sqrt{x^2+y^2}=C\cosh(z/C)$ con $C \approx 1.69$

\textbf{Indicación:} Recuerda (o demuestra) que si la gráfica de $y=y(x)$, $a\leq x \leq b$ gira alrededor del eje $X$ el área de la superficie resultante es $2\pi \int_a^by\sqrt{1+)y')^2}dx$. Después aplica el cálculo de variaciones preferentemente valiéndote de la energía.
\solution

\end{problem}

\begin{problem}[15]
15) Para lagrangianos unidimensionales L(x, $\dot{x}$), prueba directamente, únicamente con diversas formas de la regla de la cadena, que las ecuaciones de Euler-Lagrange son invariantes por cambios de variable y = y(x).

\solution

\end{problem}
\printindex
\end{document}
