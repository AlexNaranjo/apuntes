% -*- root: ../modelizacion.tex -*-
\chapter{Teoría de la información}


\section{Definición e idea intuitiva de entropía}

	\textit{``Una teoría matemática de la comunicación''} (Claude E. Shannon, 1948)

	Partimos de un conjunto finito $S = \{ S_1, ... , S_n \}$ que además es un espacio de probabilidad, siendo $p_i$ la posibilidad de escoger $s_i$. La idea general es que los elementos de $S$ sean cualesquiera fuentes de información (la notación $S$ proviene de \textit{source}).

	Más adelante consideraremos $S$ como un conjunto de ``caracteres'' que compondrán mayores unidades de información, como mensajes o ficheros.

	\begin{defn}[Entropía]
		Shanon definió una función $ H(p_1, \dotsc, p_n)$ que mide la cantidad de ``información contenida'' en $S$, y que se puede interpretar también como la ``incertidumbre'' al extraer una muestra aleatoria de $S$.
	\end{defn}

	El siguiente ejemplo nos da una idea más intuitiva de este concepto de \textbf{Entropía}

	\begin{example}
		Sea un conjunto $S$ formado por dos elementos y sean las probabilidades $p_1 = 1$ y $p_i = 0$ si $i \neq 1$. Entonces, no hay incertidumbre ya que sabemos el resultado. Es decir, tomando un elemento al azar del conjunto $S$ sabemos a ciencia cierta que será $s_1$.

		Sin embargo, si para el mismo conjunto $S$ tenemos las probabilidades $p_i = \frac{1}{n}$ tendremos mucha incertidumbre (es máxima). Es decir, tomando un elemento al azar del conjunto $S$ no podremos intuir qué elemento será.
	\end{example}


	Shannon, al estudiar el concepto de la entropía, forzó una serie de propiedades que debería cumplir la función $H$ que buscaba:
	\begin{itemize}
		\item $H$ deberá ser continua. Si realizo pequeñas variaciones en la distribución de las probabilidades, es lógico esperar pequeñas variaciones en la entropía del conjunto.

		\item $H$ ha de ser creciente en $n$: Es decir, si dada una función
		\[H\left(\frac{1}{n},..(\text{n veces})..,\frac{1}{n}\right)  \]
		 aumentamos el valor de la $n$ la entropía debería ser mayor, pues tendremos aún más cosas, lo que nos dará una mayor incertidumbre.
		\item La cantidad de información no puede variar si subdividimos $S$ en subconjuntos más pequeños de tamaño $b_i$, es decir:
		\[H\left(\frac{1}{n},..(\text{n veces})..,\frac{1}{n}\right) =  H\left(\frac{b_1}{n},..(\text{k veces})..,\frac{b_k}{n}\right) + \sum^{k}_{i = 1} \frac{b_k}{n} H\left(\frac{1}{b_i},..(b_i \text{ veces})..,\frac{1}{b_i}\right) \] para cualquier $b_i \in \mathbb{Z}^+, \sum^{k}_{i = 1} b_i = n$.
	\end{itemize}


	Así mismo, Shannon probó que las únicas funciones con estas propiedades son aquellas de la forma:

	$$H(p1, ..., p_n) = \text{Cte} \sum^{n}_{i=1} p_i \log(p_i) \quad \text{ con } \text{Cte} < 0 $$

	Eligió $\text{Cte} = -\frac{1}{\log 2}$ pensando en comunicaciones digitales.

	\begin{defn}[Entropia]
		Se define la entropía a partir de la fórmula:
		$$H(p_1, ..., p_n) = -\sum^{n}_{i = 1} p_i \log_2 p_i $$

		Se define $0 \log_2 0 = 0$ puesto que $\lim_{x \rightarrow 0} x \log x = \lim_{x \rightarrow 0^+} \frac{\log(x)}{1/x} = \lim_{x \rightarrow 0} \frac{1/x}{-1 / x^2} = \lim_{x \rightarrow 0} \frac{-x^2}{x} = 0$
	\end{defn}

	Veamos un pequeño ejemplo sobre cómo calcular esta entropía

	\begin{example}

		Sea $S$ = \{ bytes \} el conjunto de todos los posibles bytes y supongamos una distribución equiprobable entre estos bytes, es decir:
		\[p_1 = \frac{1}{2^8}, ..., p_{256} = \frac{1}{2^8}\]

		La entropía se calcularía como:
		$$  H = - \sum_{1}^{256} \frac{1}{256} \log_2 (\frac{1}{256}) = - \sum_{1}^{256} \frac{1}{256} (-8) = \sum_{1}^{256} \frac{1}{32} = 2^8 2^{-5} = 2^3 = 8 bits$$

	\end{example}

\section{Códigos prefijo}

	\begin{defn}[Código]
		Definimos un código (binario únicamente descifrable) como una función $\appl{C}{S}{c\text{cadena de bits}}$ (generalmente ceros y unos) que es inyectiva cuando actúa sobre cadenas de elementos de $S$ como:

		$$ S_{i_1} S_{i_2} ... S_{i_N} \rightarrow C(S_{i_1}) C(S_{i_2}) ... C(S_{i_n}) $$

	\end{defn}

	Tenemos dos necesidades prácticas:

	\begin{enumerate}
		\item Crear códigos que añaden redundancia para poder corregir errores cuando se transmiten por un canal poco fiable (CD, DVD, telecomunicaciones).

		\item Crear códigos que no tengan ninguna información redundante para poder comprimir.
	\end{enumerate}

	En este curso vamos a centrarnos en la segunda necesidad. Para ello definimos el concepto de longitud media.

	\begin{defn}[Longitud media]
		Se define la longitud media de las cadenas de bits $C(s_i)$ como:

		$$ l(C) = \sum_i l(C(s_i))$$

		Siendo $l(C(s_i))$ la longitud de $C(s_i)$ o su número de bits.
	\end{defn}

	Si los elementos de S forman ``palabras'' (cadenas) $s_{i_1} s_{i_2} ... s_{i_N}$ apareciendo como variables aleatorias \textbf{independientes} entonces la longitud media de la codificación de una cadena de $N$ elementos de $S$ es:

	$$l_{N}(C) = \sum_{i_1,...,i_N = 1}^{n} p_{i_1} p_{i_2} ... p_{i_N} l(C(s_{i_1} s_{i_2} ... s_{i_N}))$$

	Siendo $l(C(s_{i_1} s_{i_2} ... s_{i_N}))$ la longitud en bits al codificar de la cadena.

	No obstante esta última suposición no es realista pues en la mayoría de los idiomas, la probabilidad de encontrar una letra depende en gran medida de la letra anterior. Por ejemplo, en castellano, después de una consonante es muy esperable encontrar una vocal.

\section{Source Coding Theorem, Noiseless Coding Theorem o Primer Teorema de Shannon}

	Con estos tres nombres, entre los que el último es el menos habitual, nos referimos a un mismo teorema, que se muestra a continuación.

	\begin{theorem}[Source Coding Theorem]
	\label{SourceCoding}
		Sea $l^*$ el mínimo de $l(C)$ entre todos los posibles códigos (con $S$ y sus probabilidades fijados), entonces:

		$$ H \leq l^* \leq H + 1 $$

		donde $H$ es la entropía de $S$.

	\end{theorem}

	Por supuesto, comprimir o codificar carácter a caracter podría ser absurdo si acumulan poca información, por ejemplo si son bits. Con un truco sencillo se pasa al caso general de cadenas. El truco se recoge en el siguiente corolario.

	\begin{corol}
		\label{corolario minima longitud}
		Sea $l^{*}_N$ el mínimo de $l_N(C)$ para códigos definidos sobre N-cadenas de $S$ ($S_N$) se cumple:
		%es, para calcular el $l^{*}_N$ , el conjunto $\widetilde{S} = \{ S_{i1}  S_{i2}  .... S_{iN}\}^n_{i_1...i_N = 1}$  :

		$$ H \leq \frac{l^*}{N} < H + \frac{1}{N} \text{ , en particular  } \lim_{N \rightarrow \infty} \frac{l^*_N}{N} = H$$

		Este resultado asegura que el mejor código permite, en media, codificar (comprimir) un fichero de $N$ caracteres en $NH$ bits.

	\end{corol}

	La demostración del Teorema es bastante compleja y queda fuera de los contenidos de este curso. No obstante si que podemos llevar a cabo la demostración del corolario suponiendo la veracidad del Teorema.

\begin{proof}
	Aplicamos el teorema cambiando $S = \{ S_1,S_2...S_n \}$ por $\widetilde{S} = \{S_{i1}, S_{i2} .... S_{iN} \} =$ {N-cadenas de elementos de S} con probabilidades asociadas $p_{i1}.... p_{iN}$ con lo que obtenemos.

	$$H _{\text{para } \widehat{S}} \leq l^{*}_N < H _{\text{para } \widehat{S}} +1 $$

	Si conseguimos $H _{\text{para } \widehat{S}} = NH$ obtendremos directamente las desigualdades del corolario que queremos probar. Vamos a ello.

	$$H _{\text{para } \widehat{S}} = - \sum_{i_1...i_N} p_{i1}....p_{iN} \log_2 (p_{i1}...p_{iN})$$

	Vemos que $\log_2 (p_{i1}...p_{iN}) = \log_2 p_{i1} + .... + \log_2 p_{iN}$

	El coeficiente de $\log_2 p_1$ una vez desarrollada la suma será:
	$$- \sum_{k = 1}^{N}\sum_{i_1...i_N = 1} p_{i1} ... p_{iN} = - p_1 \cdot \sum_{k=1}^{N} \sum_{j_1 ... j_{N-1} = 1}^{n} p_{j1} ... p_{jN-1}$$

	Como $\sum_{j_1 ... j_{N-1} = 1}^{n} p_{j1} ... p_{jN-1} = 1$ tendremos

	$$- p_1 \cdot N \cdot 1 = - p_1 \cdot N \cdot ( p_1 + ... + p_n)^{N-1} = -N\cdot p_1$$

	Entonces, en general

	$$\begin{cases}
	\text{coef de } \log p_1 \rightarrow -Np_1\\
	\text{coef de } \log p_j \rightarrow -Np_j\\
	\end{cases}  \implies H _{\text{para } \widehat{S}} = -N \sum_{j=1}^{n} p_j \log p_j = N\cdot H$$

	Por tanto tenemos:
	\[H _{\text{para } \widehat{S}} \leq l^{*}_N < H _{\text{para } \widehat{S}} +1  \implies HN \leq l^{*}_N < H N +1\implies H \leq \frac{l^*}{N} <H+\frac{1}{N} \]


\end{proof}

	\textbf{Recordatorio} $C$ está en principio definido en $S$ pero la definición se extiende a cadenas de símbolos concatenando resultados. Exigimos que $C$ sea inyectiva actuando sobre cadenas.

	\begin{example}

		$S = \{A,B,C\} \quad C(A) = 0 \quad C(B) = 1 \quad C(C) = 10$

		C no es un código válido ya que $C(BA) = C(C) = 10$.

		$\widetilde{C}(A) = 01 \quad \widetilde{C}(B) = 001 \quad \widetilde{C}(C) = 0001 \Rightarrow \widetilde{C}(AC) = 010001 $ si es un código válido.

	\end{example}


	\begin{example}


		%$S = \{A,B\} \quad p_1 = p_2 = \frac{1}{2} \quad H = -\frac{1}{2} \log_2 \frac{1}{2} - \frac{1}{2} \log_2 \frac{1}{2} = 1$

		$N = 10^6$: el fichero en ASCII ocupa $10^6$ bytes.

		Del corolario \ref{corolario minima longitud} obtenemos que el mejor código sólo permitiría llegar a $l^*_N$ hasta $NH$ bits $= 10^6 1$ bits $\simeq 125000$ bytes.

		Chamizo hizo unas pruebas en su ordenador en las que con archivos aleatorios los resultados fueron los siguientes:


		\begin{table}[h]
			\centering
			\begin{tabular}{r|c}
			algoritmo & tamaño fichero comprimido \\ \hline
			gzip & 159068 bits \\
			bzip & 160213 bits \\
			rar & 161495 bits \\
			zip & 159210 bits
			\end{tabular}
			\caption{Todos son aproximadamente un 30\% peores que el óptimo.}
		\end{table}



	\end{example}


	\begin{example}

		$S = \{0,1\} \quad p_1 = p_2 = \frac{1}{2}$

		$$ 1 = H \leq \frac{l^*_N}{N} \Rightarrow l^*_N \geq N \Rightarrow \text{no hay compresión} $$

	\end{example}

\begin{example}
 ¿Cuanto se puede comprimir un fichero con $N$ bits con el 75\% de ceros y 25\% de 1s?

 	En este caso tendríamos el conjunto $S= \{0,1\}$ con las probabilidades asociadas $ p_1 = 3/4, p_2 = 1/4$
 	Podemos calcular su entropía a partir de las fórmulas vistas anteriormente con lo que obtenemos
 	\[H = 0.811 \rightarrow  l^*_N \simeq 0.8111 N \]

	Digamos que un fichero tiene tamaño M = 1000 bits y su mejor codificación tiene un tamaño de R bits..

	Si codificamos bit a bit tendremos N = 1 al aplicar el corolario\footnote{Como es lógico, en el caso N=1 podemos aplicar tanto el corolario como el teorema} visto antes nos queda:

	$$0.811 < \frac{l^{*}_1}{1} < 1.811$$

	Como $R = M \cdot l^{*}_1 \implies 811 \leq R < 1811$

	Si codificamos de 10 en 10 bits N=10 y aplicamos nuevamente el corolario, entonces
	$$0.811 < \frac{l^{*}_1}{10} < 0.911$$

	Como $R = \frac{M}{10} \cdot l^{*}_1 \implies 811 \leq R < 911$

	\textbf{Por lo tanto tomando bloques de bits grandes (lo cual en la práctica tiene límites) la "tasa de compresión" se acerca cuanto queramos a H}
\end{example}


\section{Codificación de Huffman}

La prueba del \textit{source coding theorem} es constructiva, es decir, en principio permitiría diseñar un código para el que $l^*_N$ satisfaga la desigualdad que anuncia el mismo. Huffman, siendo un estudiante de doctorado, creó un algoritmo fácilmente programable para diseñar un código que verifique $H \leq l(C) < H +1$. De hecho $l(C) = L^{*}$

Se basa en la contrucción de un árbol binario partiendo del conjunto $S_0 = S$ y evolucionando hasta llegar al conjunto $S_{n-1}$ que sólo tiene un elemento.

Las reglas para la construcción del árbol son:
\begin{enumerate}
	\item Se toman los dos elementos de $S_i$ con menos probabilidad y se ponen en el árbol como hijos de un padre al que se le asugna la suma de probabilidades.
	\item Recursivamente definimos el conjunto:
	\[S_{i+1} = \left(S_i \setminus \{\text{los hijos de antes}\} \right)\cup \{\text{padre}\}\]
\end{enumerate}
\newpage
Veamos un ejemplo de la construcción de este árbol
\begin{example}
	$$S =\{ a, b , c, d, e , f \}$$

	Siendo $a = \frac{1}{2}$ , $b, c,d = \frac{1}{8}$ y $e, f = \frac{1}{16}$

	Vemos que el padre de e y f ($\widehat{ef}$) tiene probabilidad $\frac{1}{8}$

	Nos queda
	$$S_1 = \{a,b,c,d,\widehat{ef}\}$$


	Ahora vemos el padre de d con $\widehat{ef}$, nos queda:
	$$S_2\{ a, b,c,\widehat{df}\}$$

	De forma que $\widehat{df}$ tiene probabilidad $\frac{1}{4}$

	Siguiendo estos pasos llegamos a:
	$$S_4 = \{ a , \widehat{bf}\}$$

	con las probabilidades de a y $\widehat{bf} = \frac{1}{2}$ y la raiz tendría probabilidad 1.

	Una vez que ya tenemos el árbol, partiendo de la raíz hasta las hojas codificamos con $0$ ir a la izquierda y con $1$ ir a la derecha.

	$$\begin{cases}
	a \rightarrow 0 \\
	b \rightarrow 100 \\
	c \rightarrow 101 \\
	d \rightarrow 110 \\
	e \rightarrow 1110 \\
	f \rightarrow 1111 \\
	\end{cases} \implies \begin{cases}
	l(C) = \frac{17}{8}\\
	H = \frac{17}{8}
	\end{cases}$$
\end{example}

\begin{example}
	\[\text{Tenemos el conjunto } S = \{ a , b , c , d\} \text{ con probabilidades }
	\begin{cases}
	a = 0.35\\b = 0.25\\ c= 0.24\\ d = 0.16
	\end{cases}\]

	Tomamos los dos que tengan probabilidades más pequeñas y los juntamos como hermanos de forma que el padre de c y d ($\widehat{cd}$) tiene probabilidad 0.4.

	Luego agrupamos los dos siguientes elementos con probabilidades más pequeñas de forma que nos queda $\widehat{ab}$ con probabilidad 0.6

	El padre de todos ($\widehat{abcd}$) tiene probabilidad 1.

	La codificación quedaría:
	 $$\begin{cases}
	 a \rightarrow 00\\ b \rightarrow 01 \\c \rightarrow 10 \\ d \rightarrow 11
	 \end{cases}$$
	 con longitud media $l(C) = 2$ y $H = 1.9472$
\end{example}


\begin{example}
	Vamos a hacer una pequeña variación sobre el ejemplo anterior.
	\[\text{Tomamos ahora }\begin{cases}
	a = 0.45 \\ b = 0.25\\c = 0.24 \\ d = 0.06
	\end{cases}\]

	Empezamos igual uniendo, c y d. Nos queda $\widehat{cd}$ con probabilidad 0.30.

	Ahora unimos $\widehat{cd}$ con b. Entonces probabilidad de $ \widehat{bcd}$ es 0.55. Y finalmente unimos el nodo a.

	La codificación quedaría: $\begin{cases}
	a \rightarrow 0 \\ b \rightarrow 10 \\ c \rightarrow 110\\ d \rightarrow 111
	\end{cases}$

	La longitud media (lo que mide típicamente una cadena de bits) se calcularía:
	$$l(C) = 0.45 \cdot 1 + 0.25 \cdot 2 + 0.24 \cdot 3 + 0.06 \cdot 3 = 1.91$$

	Y la entropía $H = 1.7560$
\end{example}

Llegados a este punto nos quedan dos cosas que demostrar:
\begin{enumerate}
\item Demostrar que este algoritmo nos lleva siempre a la obtención de un código válido
\item Demostrar que el código obtenido es óptimo.
\end{enumerate}

\subsection{La codificación de Huffman da lugar a un código válido}

La codificación de Huffman verdaderamente da lugar a un código válido (descodificable de manera única, C es inyectiva actuando sobre cadenas)

Vamos a verlo con el ejemplo anterior:
\begin{example}
	Recordemos que teníamos la siguiente codificación
	 $$\begin{cases}
	 a \rightarrow 0 \\ b \rightarrow 10 \\ c \rightarrow 110\\ d \rightarrow 111
	 \end{cases}$$
	 De esta forma podemos cifrar la cadena $ abacd \rightarrow 010011010$ y podemos comprobar que no hay ambigüedad a la hora de llevar a cabo el proceso inverso.
\end{example}

Esto se debe básicamente a que se trata de un código prefijo:

\begin{defn}[Código prefijo]
	Sea una función $C : S \rightarrow \{ \text{cadena de bits}\}$ tal que si $b_1 b_2 .... b_N \in Im C$ entonces $b_1 b_2 .... b_k \notin Im C$ para $k < N$ se dice que es un \textbf{código prefijo}

	Es decir, dada una cadena de bits perteneciente a la imagen del código, si quitamos su último bit la cadena resultante no pertenece a la imagen del mismo. Un ejemplo de código prefijo es el \textbf{UTF-8}
\end{defn}


Podemos comprobar de manera sencilla que, basándonos en la definición anterior, \textbf{un código prefijo siempre será un código}
\begin{proof}
$$C (S_{i1} S_{i2} .... S_{ik}) = C(S_{j1} S_{j2} .... S_{jk}) \implies C(S_{i1}) = C(S_{j1})$$
ya que de lo contrario se violaría la propiedad de prefijo, por tanto $S_{i1} = S_{j1}$, e iterando vamos obteniendo $S_{i2} = S_{j2}$ ...

Es decir, queda claro que se trata de una función inyectiva.
\end{proof}

Con esto acabamos de ver porqué los codigos prefijos son códigos. Vamos ahora a ver \textbf{porqué los códigos de Huffman son códigos}.

Los códigos prefijos generan y provienen de árboles binarios (no necesariamente completos. Un padre puede tener un solo hijo). En concreto, la codificación Huffman procede de un árbol binario en el que interpretamos como las ramas como 0s o 1s según:
$$0 \rightarrow \text{rama a la izquierda}$$
$$1 \rightarrow \text{rama a la derecha}$$

\begin{example}
Dada la siguiente codificación:
\[\begin{cases}
a \rightarrow 01\\ b \rightarrow 001\\ c \rightarrow 0001\\ d \rightarrow 0000\\
\end{cases}\]

El árbol que nos queda es:


\end{example}
Con esta correspondencia tenemos que la codificación de Huffman es un código prefijo (porque proviene de un árbol)

\subsection{La codificación de Huffman es óptima}
Vamos a ver que esta codificación es óptima entre el subconjunto de códigos prefijo. Como mencionaremos más adelante, en realidad siempre un código se puede transformar en código prefijo sin cambiar las longitudes. Por tanto \textbf{la codificación Huffman también minimiza entre todos los códigos.}

\begin{theorem}
	El mínimo de $l(C)$ sobre todos los códigos prefijo se alcanza con la codificación de Huffman
\end{theorem}

\begin{theorem}[Desigualdad de Kraft]
	Dado un código prefijo
	\[C : S = \{S_1, ..., S_n\} \rightarrow  \{\text{cadenas de bits\}} \]
	y sea $l_i = l (C(S_i))$ entonces
	$$\sum_{i=1}^{n} 2 ^{-li} \leq 1$$
	Además , dados $l_i \in \ent^{+}$ cualesquiera que cumplan la desigualdad, existe un código prefijo tal que $l_i = l(C(S_i))$ con $1\leq i \leq n$
\end{theorem}
\begin{proof}
	Vamos a llevar a cabo la demostración del problema probando las dos direcciones de la doble implicación marcada por el mismo:
	\begin{itemize}

	\item
	\[l_i = l (C(S_i)) \implies	\sum_{i=1}^{n} 2 ^{-li} \leq 1\]

	Digamos que $l_1 \leq l_2 ... \leq l_r < l_{r+1} = l_{r+2} = ... = l_n$ y consideramos $B =$\{cadenas de $l_n$ bits\} de forma que
	$$2^{l_n} = |B| = |B \cap Im C| + |B \setminus B \cap Im C| = n-r + \sum_{i=1}^{r} 2^{l_n - l_i} = \sum_{i=1}^{n} 2^{l_n - l_i}$$

	La ecuación anterior procede del hecho de que $B \cap Im C$ no contiene ninguna de las $2^{l_n - l_i}$ cadenas de bits que comienzan por $C(S_i)$
	%Es decir, cuenta todas las cadenas que empiezan %por $C(S_i)$ , que son $2^{l_n - l_i}$ cadenas, %y ve que es imposible que todas ellas sean un %código y dice que

	\item
	\[\sum_{i=1}^{n} 2 ^{-li} \leq 1 \implies l_i = l (C(S_i)) \]
	Esta parte de la demostración se basa en el algoritmo ilustrado con el siguiente ejemplo:
	$$l_1 = 1 , l_2=2, l_3=3$$
	$$2^{-1} + 2^{-2} + 2 ^{-3} \leq 1$$
	Queremos un código prefijo con estas longitudes.
	Tomamos la longitud mayor $l_3 = 3$ y dibujamos el árbol binario completo de altura esa logitud.


	\begin{center}
		\Tree[ [.0 [.00 ] [.01 010 ] ] [.1 ] ]
	\end{center}


	El algoritmo se basa en podar este árbol de la siguiente forma:
	\begin{enumerate}
		\item En $l_1 = 1$ borramos los descendientes de un vértice que estén en altura/profundidad 1.

		\item Luego hago lo mismo con $l_2$, con lo cual borramos los descendientes que estén en altura/profundidad 2.

		\item Una vez hecho esto, si sobran algunas hojas que no corresponden a las longitudes, también las borramos (en nuestro caso nos quedan dos con longitud 3, y solo queremos 1, por lo tanto borramos la otra)

		\item Las codificaciones que nos quedarían serían $\begin{cases}
		1\\00\\010
		\end{cases}$
	\end{enumerate}
	Necesitamos alguna condición para garantizar que el algoritmo se pueda llevar a cabo sin que se nos acaben los vértices.

	Analizando con cuidado el ejemplo podemos observar que lo que hemos ido haciendo es comenzar con un árbol de 8 hojas al que le hemos quitado 4 y luego hemos quitado 2 para al final quedarnos con $8-4-2 = 2$ hojas con lo que podíamos escoger el $l_3$ que quisiéramos de altura 1.

	En general, con la notación anterior $l_1 \leq l_2 \leq .... \leq l_{r} < l_{r+1} = .. = l_n$ el algoritmo parte de un árbol con $2^{l_n}$ hojas. De ellas borramos primero $2^{l_n - l_1}$ y en sucesivos pasos $2^{l_n - l_2} , .... , 2^{l_n - l_r}$ y deben sobrarnos, al menos, $n-r$ de las hojas iniciales.

	Entonces necesitamos que $2^{l_n} - \sum_{i=1}^{r} 2^{l_n - l_i} \geq n-r$ para que el algoritmo funcione. Por tanto es necesario que

	\[2^{l_n} \geq \sum_{i=1}^{n} 2^{l_n - l_i} \iff 1 \geq  \sum_{i=1}^{n} 2^{- l_i} \]

\end{itemize}
\end{proof}

\obs La generalización de MCMillan afirma que para cualquier código se cumple $\sum_{i=1}^{n} 2^{-l_i} \leq 1$ aunque no sea prefijo.

Por tanto, cualquier código dado satisface la desigualdad de Kraft y por el recíproco de Kraft sabemos que existe un código prefijo con las mismas longitudes.
Así, cualquier teorema que hable de las longitudes y sea cierto para códigos prefijo, lo será para todos.

Una vez que hemos visto esto podemos probar el teorema del Source Coding \ref{SourceCoding}. Para la demostración vamos a utilizar la desigualdad de Kraft:

\begin{proof}
	\textbf{Demostración del Source coding theorem}

	Sea la función
	$$f(x_1 , x_2 , ...., x_n) = -\sum_{i=1}^{n} p_i \log x_i$$
	si la restringimos al \textit{simplex}
	$$\sum_{i=1}^{n} x_i = 1 \text{   con   } 0 \leq x_i \leq 1$$
	entonces f alcanza un mínimo cuando $x_i = p_i$

	Con la notación de la desigualdad de Kraft tomemos
	\[x_i = \frac{2^{-l_i}}{\sum_{j=1}^{n} 2^{-l_j}}\]

	Entonces
	$$H= f(p_1,...,p_n) \leq f(x_1,x_2,...,x_n) \leq f \left(\frac{2^{-l_1}}{\sum_{j=1}^{n} 2^{-l_j}} .... , \frac{2^{-l_n}}{\sum_{j=1}^{n} 2^{-l_j}}\right)$$
	y, por la desigualdad de Kraft, llegamos a que:
	$$f \left(\frac{2^{-l_1}}{\sum_{j=1}^{n} 2^{-l_j}} .... , \frac{2^{-l_n}}{\sum_{j=1}^{n} 2^{-l_j}}\right)\leq f(2^{-l_1} , .... , 2^{-l_n}) = l(C)$$

	Con esto ya tenemos la cota inferior para la longitud media. Vamos ahora a obtener la cota superior. Para ello tomamos $l_i \in \mathbb{Z}^+ : -\log_2 p_i \leq l_i < 1 - \log_2 p_i$. Así tenemos que

	$$\sum^{n}_{i = 1} 2^{l_i} \leq \sum^{n}_{i = 1} 2^{\log_2 p_i} = \sum^{n}_{i = 1} p_i = 1 $$

	El teorema de la desigualdad de Kraft nos garantiza que $\exists \text{ código prefijo con } l_i = l(C(S_i))$

	$$l(C) = \sum^{n}_{i = 1} p_i l_i < \sum^{n}_{i = 1} p_i - \sum^{n}_{i = 1} p_i \log_{2} p_i = 1 + H$$

	Con lo que ya tenemos la acotación de la longitud media que buscábamos:
	\[H \leq l(C) \leq H+1\]

\end{proof}

\section{Métodos de diccionario}
	En español los caracteres tienen una entropía aproximada de $4.09$. Lo que significa que dado un texto de $M$ caracteres, puesto que:

	\[4.09 = H \leq l^{*} (C) \quad \text{ carácter } \rightarrow  1 \text{ byte} = 8 \text{ bits}\]

	cabe esperar que la mejor compresión posible sea $4.09M$ bits $= \frac{4.09}{8}$ bytes $\simeq$ la mitad de los bytes que teníamos en el texto original.

	En cambio si comprobamos empíricamente con varios compresores comprobamos que superan la compresión óptima:

	$M = 1869610$

	\begin{table}[h]
		\centering
		\begin{tabular}{r|c}
		algoritmo & tamaño fichero comprimido \\ \hline
		gzip & 730122 bytes \\
		bzip & 534165 bytes \\
		rar & 609319 bytes \\
		zip & 730264 bytes
		\end{tabular}
	\end{table}

	Supongamos un tipo de ficheros en que cada byte tiene un 99\% de posibilidades de ser igual al siguiente y un 1\% de cambiar aleatoriamente\footnote{Como una imagen rastrerizada de un dibujo. Los cambios de color son raros porque hay grandes bloques iguales, como el fondo.}.

	En un fichero grande de este tipo cada carácter aparecerá con probabilidad $\simeq \frac{1}{256}$ lo que nos da una entropía:
	\[H = \sum\limits^{256}_{i = 1} - \frac{1}{256} log_{2} \frac{1}{256} = 8\]

	que, por el \textit{Source Coding Theorem} indica que no hay compresión posible, cada byte requiere 8 bits.

	Sin embargo hay un método muy fácil para comprimir este tipo de ficheros: RLE (run-length encoding). Este método consiste en sustituir los datos por grupos de dos bytes. En cada grupo está el número de repeticiones de un caracter y el caracter.

	\begin{example}
	Veamos un pequeño ejemplo de este algoritmo

	50A200B = AA... 50 veces ...ABB... 200 veces ...B

	200A100A200B = AA... 300 veces ...AB... 200 veces ...B
	\end{example}

	Con este método típicamente (por las probabilidades mencionadas anteriormente) pasaremos de 100 bytes a 2 bytes.

	En el teorema $S$ está fijado $S$ = \{ posibles bytes \}. Si en su lugar empleásemos $S$ = \{ posibles cadenas de 100 bytes \} el teorema sí aseguraría que hay compresión.

	\begin{obs} Aplicar el corolario con N = 100 no refleja la situación de este tpo de ficheros porque allí se suponía la independencia entre los elementos de $S$.

	Según el modelo del corolario: Prob (A... 100 veces ...A) = $(1/256)^{100}$

	Y en nuestro problema: (A... 100 veces ...A) = $(1 - 0.01)^{100} \simeq{1/e}$

	\end{obs}

	En principio Huffman con $S =$ \{ cadenas de 100 bytes que aparecen en el fichero \} sería ventajoso. Pero este $S$ se basa en un conocimiento previo acerca de las estadísticas de aparición de las cadenas en el fichero. Es por esto que el algoritmo de Huffman no se suele aplicar en compresores ``universales''. En general los principales problemas de este algoritmo son:

	\begin{itemize}
		\item Requiere un estudio estadístico previo del fichero (para decidir $S$ y $p_i$) que es costosa.

		\item Hay que almacenar el árbol para comunicarlo al compresor y podría ser grande.

	\end{itemize}
La idea ahora es buscar métodos "universales" creando diccionarios de forma dinámica que contengan referencias a las palabras.

Estos diccionarios son virtuales; no se almacenan separadamente al fichero, de hecho equivalen a él, y sólo aparecen explícitamente al ejecutar los algotirmos de compresión y descompresión.


	Los algoritmos que se usan parten de un trabajo de J.Zip, A.Lempel de 1997 y dan lugar a tres variantes principales: LZ77, LZ78 y LZW.

	En la práctica los 3 se usan con diferentes trucos y variantes, pero están todos en uso. A parte de en compresores, se usan en diferentes formatos de imágen.
	\newpage
	\subsection{Algunos algoritmos de compresión de imagen}
	\begin{center}
		\begin{table}[h]
			\centering
			\begin{tabular}{r|c}
			algoritmo & funcionamiento \\ \hline
			bmp & sin compresión o RLE \\
			png & LZ777 y Huffman (con filtrados previos) \\
			gif & Fija una paleta de 256 colores (o más pequeñas) y LZW \\
			\end{tabular}
		\end{table}
	\end{center}

	\subsection{El algoritmo LZ78}

	Para simplificar, en los ejemplos pensamos en cadenas de caracteres que terminan con un símbolo de "fin de fichero" que escribiremos como \# \footnote{Esta simplificación no es algo que nos inventemos. En la práctica se considera un carácter de final de fichero para determinar el fin de cada cadena.}. El algoritmo se basa en los siguientes principios:

	\begin{enumerate}
	\item Dividimos la cadena que se quiere comprimir en frases.

	\item Cada frase consiste en añadir un caracter a una frase ya existente, sin repetirlas.

	\item Inicialmente sólo tenemos la frase vacía.
	\end{enumerate}

	Vemos el ejemplo de la palabra \textbf{\textit{rellena}}. En este caso tenemos un problema y es que, después de añadir la primera \textbf{\textit{l}}, no puedo añadir la seguna, porque no puedo añadir frases repetidas. Por lo tanto añado \textbf{\textit{le}} como frase.

	Estas frases en una lista numerada, comenzando por $0 \rightarrow \emptyset$ , y siguiendo el orden, son el "diccionario".

	\begin{example}
		Vamos a ver como sería la lista resultante de: \textbf{"rellena la encuesta"}. Tenemos que recordar que al final tenemos el fin de documento , \#.


		$\begin{cases}
			0 \rightarrow \emptyset\\
			1\rightarrow r\\
			2 \rightarrow e\\
			3\rightarrow l\\
			4 \rightarrow le\\
			.\\
			.\\
			.\\
			10 \rightarrow nc\\
			11\rightarrow u\\
			12 \rightarrow es\\
			13 \rightarrow t\\
			14 \rightarrow a\#
		\end{cases}$

	\end{example}

	\begin{example}
		Hacemos lo mismo con la frase \textbf{"salsa salada"}


		$\begin{cases}
		0 \rightarrow \emptyset\\
		1\rightarrow s\\
		2 \rightarrow a\\
		3\rightarrow l\\
		4 \rightarrow sa\\
		5 \rightarrow \text{ (el espacio)}\\
		6\rightarrow sal\\
		7 \rightarrow ad\\
		8 \rightarrow a\#
		\end{cases}$

	\end{example}


Para codificar se sustituye cada frase por un par formado por un número y un carácter según la relación:

frase $\rightarrow$ (n. de frase en el diccionario sin el último carácter, último carácter de la frase)

\begin{example}
	Vamos a ver la codificación de \textbf{salsa salada}.

	$\begin{cases}
	s \rightarrow (0,s)\\
	a \rightarrow (0,a)\\
	l \rightarrow (0,l)\\
	sa \rightarrow (1,a)\\
	\text{ (espacio)} \rightarrow (0,espacio)\\
	sal \rightarrow (4,l)\\
	ad \rightarrow (2,d)\\
	a\# \rightarrow (2, \#)
	\end{cases}$

	En este ejemplo, si suponemos que los elementos ocupan 1 byte cada uno:
	$$\text{cadena original} \rightarrow 126 \text{ bytes}$$
	$$\text{codificación} \rightarrow 8\cdot 2 =16 \text{ bytes}$$
\end{example}

Un ejemplo un poco más realista es el que hizo Chamizo con el texto de \textit{La Regenta}.

El texto consta de 1869610 caracteres, el diccionario le salió de 281192 frases. Entonces:

$2^{19} > 281192 > 2^{18}\implies$ Necesitamos 19 bits para los números y 1 byte = 8 bits para el caracter. Así, cada frase ocupa 27 bits al codificar. La codificación completa ocupa 949023 bytes por lo que, efectivamente, \textbf{Hay compresión}

Para \textbf{descomprimir} vamos recuperando la lista del diccionario, partiendo de $0 \rightarrow \emptyset$, según leemos la codificación.

\begin{example}
Vamos a descodificar la siguiente cadena:

(0,e) (0,r) (2,e) (0, \textit{espacio}) (0,t) (3,s) (0,\#)

Siempre empezamos por $0 \rightarrow \emptyset$
$\begin{cases}
 (0,e) \implies 1 \rightarrow e\\
  (0,r) \implies 2 \rightarrow r\\
  (2,e) \implies 3 \rightarrow re\\
   (0, \textit{espacio}) \implies 4 \rightarrow \textit{espacio}\\
   (0,t) \implies 5 \rightarrow t\\
    (3,s) \implies 6 \rightarrow res\\
     (0,\#) \implies 1 \rightarrow e
\end{cases}$

Frase final: \textbf{erre tres}
\end{example}

\subsubsection{Nivel de compresión del algoritmo LZ78}

El nivel de compresión depende del número de frases asociadas a la cadena/fichero inicial.

Por ejemplo en la cadena \textbf{\textit{rellenala ya}} tenemos 10 frases. Las posiciones relativas a 10 frases ocupan 4 bits. Puesto que un carácter en ASCII ocupa 8 bits cada par está compuesto por la posición y un carácter. Puesto que la codificación requerirá 10 frases, la longitud en bits será $10 \cdot (4 +8)$

En general la longitud en bits al codificar una cadena \textbf{s} con \textbf{LZ78} es
$$l_{LZ} (s) = F \cdot ( \lceil\log_2 F\rceil  + b)$$
Siendo b el número de bits de un carácter.


¿Qué ocurre al aplicar el algoritmo a $s = s_{i1}.... s_{iN}$ donde $s_{ij} \in S = {s_1...s_n}$ ?

\begin{theorem}
	Consideremos la longitud media
	$$l^{*}_{LZ}(N) = \sum_{i_1,...,i_N} p_{i1}...p_{iN} l_{LZ}(s_{i1}....s_{iN})$$

	Entonces
	$$\lim_{N\rightarrow \infty} \frac{l^{*}_{LZ}(N)}{N} = H$$

	Con H entropía de s
\end{theorem}

\obs Asintóticamente LZ78 es tan bueno como el algoritmo óptimo en el corolario del \textit{source coding theorem}

\subsection{Algoritmo LZW}

Es una variante de LZ78 que comienza predefiniendo un diccionario con todos los posibles caracteres (que pueden aparecer en la cadena).

La división de frases, un poco diferente a la de LZ78, se basa en los siguientes principios
\begin{enumerate}
	\item Los caracteres individuales no forman frases porque ya están en el diccionario.
	\item Cada caracter del final de una frase se incluye en el inicio de la siguiente.
\end{enumerate}

\textbf{Codificación}: referencias de las frases sin el último carácter.


\begin{example}
	Tenemos la siguiente cadena:
	$$01110100001\#$$
	Vamos a ver como será la división en frases.Primero vemos que en el diccionario ya tenemos
	$$\begin{cases}
	0 \rightarrow 0\\
	1 \rightarrow 1
	\end{cases}$$

	Construimos el resto de frases de forma que el diccionario queda:
		$$\begin{cases}
		0 \rightarrow 0\\
		1 \rightarrow 1\\
		2 \rightarrow 01\\
		3 \rightarrow 11\\
		4 \rightarrow 110\\
		5 \rightarrow 010\\
		6 \rightarrow 00\\
		7 \rightarrow 000\\
		8 \rightarrow 01\#\\
		\end{cases}$$

	\textbf{Codificación:} posición de cada frase al suprimir el último caracter.
	$$0,1,3,2,0,6,2$$
\end{example}

\textbf{Descodificación}: En cada paso añadimos al diccionario la última descodificación más el primer carácter de la que está en curso.

Aunque, como veremos, hay una excepción.

\begin{example}
	Supongamos el diccionario inicial:

	$$\begin{cases}
	0 \rightarrow \text{ espacio}\\
	1 \rightarrow e\\
	2 \rightarrow q\\
	3 \rightarrow r\\
	4 \rightarrow u
	\end{cases}$$

	Y nos dan las siguiente cadena a descodificar:
	$$1,3,3,1,0,2,4,8,5,7$$

	Empezamos a descodificar:

	El $1 \rightarrow e$ , lo siguiente es $3 \rightarrow r$ pero si vemos la definición que hemos dado de cómo se descodifica, ahora tenemos que añadir al diccionario la última descodificación más el primer carácter de la que está en curso. por lo tanto también tenemos que añadir al diccionario $5\rightarrow er$

	Vamos a seguir descodificando. Nos vuelve a tocar $3 \rightarrow r$ y además añadimos al diccionario $6 \rightarrow rr$

	El siguiente es $1 \rightarrow e$ y añadimos al diccionario $7 \rightarrow re$

	Pasamos al $0 \rightarrow \text{espacio}$ y añadimos al diccionario $8 \rightarrow \text{espacio}$

	Luego leemos $2 \rightarrow q$ y añadimos $9 \rightarrow \text{espacio } q$

	Y asi seguimos hasta terminar la cadena.

	Para la descodificación , basta leer las frases descodificadas con cada número.

	Finalmente nos queda la frase \textbf{erre que erre}
\end{example}

Antes hemos dicho que este método a veces no funciona, vamos a ver un ejemplo en el que no nos valga este algoritmo:

\begin{example}
	En el ejemplo de ayer...

	011....01\#

	Esto cuando lo codificamos queda
	$$0,1,3,2,0,6,2$$

	Si intentamos descodificarlo vemos que $0 \rightarrow 0$ , $1 \rightarrow 1$ y añadimos $2 \rightarrow 01$ y llegamos al 3... Pero el 3 no está en el diccionario.

	¿Cómo seguimos? Pues vamos a creernos que $3 \rightarrow 11$

	Y sigo descodificando.

	Leo $2 \rightarrow 01$ y añado al diccionario $4 \rightarrow 110$.

	Lo siguiente es $0 \rightarrow 0$ y añado $5 \rightarrow 010$

	Y toca leer un 6, pero aún no tenemos el 6 en e diccionario.

	Vamos a creernos que $6 \rightarrow 00$

	Y continuo hasta el final.
\end{example}

\textbf{¿Cómo hemos sacado la codificación del 3 y del 6 en el ejemplo anterior?}

Siempre que no podamos crear una entrada de diccionario (acabamos de crear la entrada $k-1$ y nos toca descodificar la $k$) entonces escribimos :
$$k \rightarrow \text{última descodificación + primer carácter de esa descodificación}$$

Es decir, en el caso del 6, justo antes habíamos descodificado el 0, miramos el diccionario y vemos que $0\rightarrow 0$, por lo tanto la última descodificación es 0. Como la descodificación de 0 tiene sólo un carácter , el primer carácter de la descodificación vuelve a ser 0.
\section{El algoritmo ID3 de aprendizaje automático}

