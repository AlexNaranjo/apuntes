\chapter{Gravitación y leyes de kepler}
\section{Leyes de Kepler}

Kepler, a principios del siglo XVII, enunció unas leyes experimentales (con datos de Tycho Brahe):

\begin{lemma}[1ª Ley de Kepler]
	Las órbitas de los planetas son elipses con el Sol en uno de los focos.
\end{lemma}

\begin{lemma}[2ª Ley de Kepler]
	La linea que une un planeta y el Sol barre áreas iguales en tiempos iguales.
\end{lemma}

\begin{lemma}[3ª Ley de Kepler]
	El cuadrado del periodo orbital es proporcional al cubo del semieje mayor de la órbita.
\end{lemma}

\section{Newton}

Gracias a Newton, en 1985, las leyes de Kepler pudieron demostrarse matemáticamente suponiendo que la fuerza gravitacional obedece la ecuación
\[ F = \frac{-GMm}{r^2} \]
en la dirección del vector radio.

No existe una forma sencilla de realizar esta demostración a menos que demos por sentados muchos principios físicos que no eran conocidos en aquella época. Vamos a tratar de realizar la demostración utilizando los mínimos ``trucos'' posibles.

Para empezar debemos recordar que toda fuerza puede calcularse como:
$$ \overrightarrow{F} = m\ga$$

La aceleración se calcula a partir de la posición de una partícula, que viene dada por: $\vr = (x(t), y(t), z(t))$. La aceleración se define como la derivada segunda de la posición con respecto al tiempo. Es decir:
\[ \ga = \frac{d^2\vr}{d t^2} \]

Esta fórmula describe, por ejemplo, la atracción que ejerce el Sol sobre la Tierra, por ello el signo negativo. En el caso de el Sol y la Tierra, la fuerza recíproca no se tiene en cuenta ya que el Sol es demasiado pesado como para ser influido apreciablemente por la gravedad de la Tierra. Hay que hacer mediciones muy finas para poder detectar estas perturbaciones.

Para poder seguir con la demostración de forma relativamente sencilla, \textbf{despreciamos la fuerza de los planetas sobre el Sol y de los planetas entre ellos.}

Por lo tanto, modelamos tomando sólo un planeta y asumiendo un Sol fijo.

Igualando las dós fórmulas para el cálculo de la fuerza que hemos indicado, obtenemos:

$$ \frac{-GMm}{{||r||}^2} \frac{\vr}{{||r||}^2}  =  m \frac{d^2\vr}{d t^2}$$


En primer lugar vamos a probar que las soluciones $\vr = \vr(t)$ de esta ecuación diferencial satisfacen las leyes de Kepler (con las condiciones iniciales de los planetas)

%Esto es un sistema con una llave
$$
\begin{cases}
 x'' = \frac{-GMx}{(x^2 + y^2 + z^2)^{3/2}}\\
 y'' = \frac{-GMy}{(x^2 + y^2 + z^2)^{3/2}}\\
 z'' = \frac{-GMz}{(x^2 + y^2 + z^2)^{3/2}}\\
\end{cases}
$$


 \begin{obs}
 No se pueden calcular explícitamente $x$, $y$ y $z$, en función de $t$. $\vr = \vr(t)$ no tiene una fórmula cerrada (en términos de funciones elementales\footnote{Las de la calculadora (sin, cos...)}). Aun así se puede demostrar que la fórmula es una elipse.
 \end{obs}


 %Un método es un truco que sirve varias veces
Vamos a probar que la curva $\vr = \vr(t)$ está contenida en un plano.

Un posible método es usar que una curva es plana si y sólo $τ \equiv 0$, donde τ es la torsión de la curva. Dicha fórmula involucra un determinante con una fila $r$ y otra $r''$, como son proporcionales, la torsión sale 0.


Nuestro método más sencillo se basa en tomar $\overrightarrow{L}(t) = \vr(x) \times \frac{d\vr}{dt}$.

Derivando obtenemos:
$$\frac{d\overrightarrow{L}(t)}{dt} = \frac{d\vr}{dt} \times \frac{d\vr}{dt} + \vr(t) + \frac{d^2\vr}{dt^2} = \overrightarrow{0}$$

de donde podemos deducir que $\overrightarrow{L}$ es un vector constante $\Rightarrow \vr \perp \overrightarrow{L} \Rightarrow \vr(t)$ está contenida en el plano $\overrightarrow{L}(x,y,z) = 0 $.
(pensar $\overrightarrow{L} = \overrightarrow{0}$).

Por comodidad, vamos a girar el plano en que se encuentra la órbita del planeta, $\vr \rightarrow G\vr$, siengo G un giro (matriz ortogonal), para convertirlo en el plano $z=0$.

Podemos comprobar que este giro no afecta a las cuentas puesto que, en definitiva, estamos multiplicando por una constante.

$$\frac{d^2\vr}{dt^2} = \text{cte}\frac{\vr}{||\vr||^3} \Leftrightarrow \frac{d^2(G\vr)}{dt^2} = \text{cte}\frac{G\vr}{||G\vr||^3}$$

En definitiva "girando la cabeza" (aplicando el cambio $\vr \rightarrow G\vr$) podemos suponer que la curva $\vr = \vr(t)$ está contenida en el plano $z = 0$, así que simplemente supondremos que $z(t) = 0$. El sistema nos quedaría así:


%Esto es un sistema con una llave
$$
\begin{cases}
 x'' = k\frac{x}{(x^2 + y^2)^{3/2}}\\
 y'' = k\frac{y}{(x^2 + y^2)^{3/2}}\\
\end{cases}
$$

\subsection{Demostración de las leyes de Kepler}

Vamos a probar las leyes de Kepler para $t \rightarrow (x(t), y(t))$ utilizando los resultados de Newton.

La forma de la solución será una elipse.
Si escribimos $x$ e $y$ en polares, el sistema nos quedará más sencillo.

% % align, gather y equation.
$$\begin{array}{c}
x = r\cos(\theta)\\
y = r\sin(\theta) \\
r = r(t) \\
\theta = \theta(t)
\end{array}
$$

(r y $\theta$ no se pueden calcular eplícitamente.)

Y con esto sacamos las ecuaciones de gravitación:
\begin{gather}
r''\cos\theta - 2r'\theta'\sin\theta - r(\theta')^2\cos\theta - r\theta''\sin\theta = cte \frac{\cos\theta}{r^2} \label{eq1_kepler}\\
r''\sin\theta + 2r'\theta'\cos\theta - r(\theta')^2\sin\theta + r\theta''\cos\theta = cte \frac{\sin\theta}{r^2} \label{eq2_kepler}
\end{gather}
$\eqref{eq1_kepler}\cdot\cos\theta + \eqref{eq2_kepler}\cdot\sin\theta \rightarrow K= r^3(\theta')^2 - r^2*r'' \rightarrow$ 1ª ecuación de gravitación
$$-\eqref{eq1_kepler}\cdot\sin\theta + \eqref{eq2_kepler}\cdot\cos\theta \rightarrow 0= r \theta'' + 2r' * \theta' \rightarrow \text{2ª ecuación de gravitación}$$


\subsubsection{2ª ley de Kepler}
\paragraph{Recordemos:}
2ª ley de Kepler : Un planeta recorre áreas iguales en tiempos iguales.

Vamos a estudiar cual es la fórmula para el área.
Primero nos vamos al caso general, para cualquier curva que tengamos en polares.

En este caso el radio depende de $\theta$ \\
Cogemos una curva R.\\

$$A(R) = \int\int_R 1 \dif x \dif y = \int_0^{\theta_0} \int_0^r(\theta) r \dif r \dif \theta = 1/2 \int_0^{\theta_0} r^2(\theta) \dif\theta$$

La idea es coger pequeños triángulos e ir calculando su área

En el caso de la \textbf{ley de Kepler}, si en el tiempo t=0 estamos en $\theta = 0$ y en el tiempo T estamos en $\theta_0 \implies$ la fórmula para el área en función de T sería:

$$A(T) = \frac{1}{2} \int_1^T r^2(t) \theta'(t) \dif t$$


de donde podemos deducir que $\frac {\dif A(T)}{\dif T} = cte$

Es lo mismo que decir $A(T)$ es lineal (a tiempos iguales tengo áreas iguales)

\[A(x + T) - A(x) = A(T) - A(0)\]

Vamos a traducir la 2ª ley de Kepler en algo más familiar (derivadas):

\[\frac{\dif A}{\dif T} = cte \iff r^2\theta' = cte  \text{ (no depende de T) }\]
que puede verse como la conservación del momento angular

Además, $r^2\theta' = cte  \implies (r^2\theta')' = 0 \iff 2rr'\theta' + r²\theta'' =0$ con lo que obtenemos la 2ª ecuación de gravitación.

Por tanto, queda probada la 2ª ley de Kepler.

\subsubsection{1ª ley de Kepler}
Ahora vamos a intentar manipular la 1ª ecuación para probar la \textbf{1ª ley de Kepler} (las óbitas de los planetas son elipses).

Escribimos:
\[h= r^2\theta'=cte \text{ (para cada planeta, } h \text{ es diferente)}\]

1ª ecuación de gravitación:
\[K= r^3(\theta')^2 -r^2*r''\]

como $h = r^2*\theta'$ podemos eliminar $\theta'$ en la 1ªecuación, obteniendo:

$$K= r^3 \frac{h^2}{r^4} - r^2*r''$$
y, simplificando, llegamos a:
$$K= \frac{h^2}{r} - r^2*r''  \rightarrow  \text{Es una EDO ($r = r(t)$) y no aparece explícitamente la t.}$$

\obs En el curso de EDO se ven métodos para pasar de 2 orden ($r''$) a 1 orden ($r$) y se podría expresar r en términos de la imagen inversa de una integral, pero no se puede calcular en términos elementales.

Para resolver la ecuación debemos estar hábiles y realizar un cambio de variable adecuado. En este caso, el cambio de variable consistirá en escribir $r=r(\theta)$ en vez de $r=r(t)$

Hacemos el siguiente cambio de variable:
$$U(\theta(t)) = \frac{1}{r(t)} \implies U'\theta' = \frac{-r}{r^2} \stackrel{h=r^2\theta' }{\iff} hU' = -r' \stackrel{U(\theta(t))}{\implies} hU''\theta' = -r''$$

Sustituyendo en la EDO:

$$K= h^2U + h^2U'' \rightarrow \text{ecuación del movimiento armónico simple}$$
viendo que $\cos'' = -\cos$ y $\sin'' = -\sin$:

$$U'' + U = \frac{K}{h^2} \implies U= \frac{K}{h^2} + \lambda\cos\theta + \mu\sin\theta \iff U= \frac{K}{h^2} + A\cos(\theta - \theta_0)$$
\begin{mdframed}
\obs Al último resultado hemos llegado utilizando que
$$ \lambda\cos\theta + \mu\sin\theta = \sqrt{\lambda^2 + \mu^2}\left(\frac{\lambda}{\sqrt{\lambda^2 + \mu^2}}\cos\theta  +\frac{\mu}{ \sqrt{\lambda^2 + \mu^2}}\sin\theta\right)$$
LLamamos $\cos\theta_0$ a $\frac{\lambda}{ \sqrt{\lambda^2 + \mu^2}}$ y $\sin\theta_0$ a $\frac{\mu}{ \sqrt{\lambda^2 + \mu^2}}$ ya que ambos están entre 0 y 1.
Nos queda que
$$\lambda\cos\theta + \mu\sin\theta = \sqrt{\lambda^2 + \mu^2}(\cos\theta_0\cos\theta + \sin\theta_0\sin\theta) \stackrel{A=\sqrt{\lambda^2 + \mu^2}}{=} A\cos(\theta - \theta_0)$$
\end{mdframed}
Con un giro podemos suponer $\theta_0 = 0$, porque un giro en polares no es más que sumarle una constante al ángulo ($\theta \rightarrow \theta + cte$)

Volviendo al cambio de variable inicial, tenemos
\[U = \frac{1}{r} \implies r=\frac{h^2/K}{1 + B\cos\theta}\]

\textbf{Hecho matemático:} La ecuación general de una cónica en coordenadas polares centradas en un foco con la "orientación habitual" es:

$$r(\theta) = \frac{l}{1+e\cos\theta} \rightarrow
\begin{cases}
l>0\\
e\text{ (excentricidad) }\ge0\\
\end{cases}
$$

\textbf{Recordemos:}

$$\text{excentricidad} \rightarrow
\begin{cases}
0 \le e \le 1 \rightarrow \text{elipse , caso particular} (e=0) \rightarrow \text{circunferencia}\\
e>1 \rightarrow \text{hipérbola(masas no capturadas por el sol)}\\
e=0 \rightarrow \text{parábola}
\end{cases}
$$
Los \textbf{datos astronómicos} para los planetas(excepto mercurio) muestran $B<0.1$ por lo tanto las órbitas son elipses(1ª ley de Kepler) y parecidas a circunferencias.

\begin{center}
\begin{tabular}{| c | c |}
	\hline
	Planetas & B \\
	\hline
	Tierra & 0.016 \\
	\hline
	Venus & 0.0067 \\
	\hline
	Mercurio & 0.20\\
	\hline
\end{tabular}
\end{center}


\subsubsection{3ª ley de Kepler}
%(Dibujo de la elipse con los ejes)
Tomamos $a$ como el semieje mayor y $b$ como semieje menor de la elipse de forma que:
$$2a = \frac{l}{1+e} + \frac{l}{1-e} \implies a= \frac{l}{1-e^2}$$
y
$$b = \sqrt{al}$$
Ahora vamos a calcular área de la elipse basándonos en la fórmula conocida para el área de la circunferencia.

Escribimos la circunferencia como
\[\frac{x^2}{a^2} + \frac{y^2}{a^2} = 1\]
y su área es
\[\pi a^2\]

Si hacemos en cambio de
\[y \rightarrow \frac{a}{b}y\]
nos queda
\[\frac{x^2}{a^2} + \frac{y^2}{b^2} = 1\]
que es la fórmula de la elipse. Por tanto, el área de la elipse será $\pi ab$.

Ahora calculamos el área de la órbita del planeta:
$$T_o = \text{periodo orbital}$$

$$r= \frac{h^2/K}{1 + B\cos\theta}$$

$$\text{Área de la órbita} = A(T_0) = 1/2 \int_0^{T_0} \underbrace{r^2(t)\theta'(t)}_{\text{h}}\dif t = 1/2hT_0$$

Igualando el área de la órbita con  el área de la elipse tenemos:
$$\frac{1}{2}hT_0 = \pi ab= \pi a \sqrt{al} = \pi a\sqrt{a\frac{h^2}{K}} \implies T_0 = cte*a^{3/2}$$
Y queda demostrada la 3ª ley de Kepler.

\obs Para las demostrciones de las tres leyes, en física se ayudan de dos teoremas:
\begin{enumerate}
\item\textbf{Conservación del momento angular}:
$\overrightarrow{L}$ es cte siendo:
\[\overrightarrow{L} = \overrightarrow{r} \times m \overrightarrow{v} (\overrightarrow{r} \rightarrow  \text{ posición },  \overrightarrow{v} \rightarrow \text{ velocidad}).\]


\item\textbf{Conservación de la energía}:\\
\[\frac{1}{2}m||\overrightarrow{v}||^2 + \frac{GMm}{||\overrightarrow{r}||} \rightarrow \text{ es cte}.\]
\end{enumerate}

\subsection{El cabo suelto de Newton}
A Newton le quedó un cabo suelto que vamos a intentar resolver:

\begin{figure}[hbtp]
	\centering
	\inputtikz{ProblemaFGravitacion}
	\caption{¿Se cumple $\md\vf = \frac{GMm}{r^2}$ ?}
\end{figure}
Vemos que en la fórmula, cuando $r \rightarrow 0 \implies \vf \rightarrow \infty$ y esto , en el caso del dibujo, no es real.

Si una masa M está compuesta por varias masas con densidad constante, $\md\vf = \frac{GMm}{r^2}$, con $r =$distancia al punto medio puede cumplirse.

El problema es ver cómo se comporta $\vf$ cuando la densidad de M no es constante.......\\
(.......)\\
Aqui va razonamiento que no he entendido bien y que me va a explicar Miguel cuando pueda.
El caso es que puedo intentar calcular la fuerza que ejerce el Sol sobre la Tierra igual que si fuera sobre una partícula con densidad uniforme.\\
\newpage
\begin{figure}[hbtp]
\centering
\inputtikz{KasnerNoSabePonerNombresANada}
\caption{Divido la esfera con masa M en pequefas particulas $\partial M$}
\label{figaaa}
\end{figure}

Entonces para ver si $\md\vf = \frac{GMm}{r^2}$  es válida, habrá que calcular una integral triple, complicada y ver el efecto que tiene en cada partícula.
$$\int\frac{-Gm\overrightarrow{x}}{\md {\overrightarrow{x}}^3}\dif M$$

Hay una manera de calcular esta integral y probar que  $\md{\overrightarrow{\vf}}$ es válida para objetos con simetría esférica, usando el teorema de la divergencia:

\begin{prop}[Ley de Gauss]
	Dada una región sólida B y un campo $\overrightarrow{E} = \frac{\overrightarrow{x}}{\md {\overrightarrow{x}}^3}$ tenemos que:
	$$\int_{\partial B} \overrightarrow E \dif \overrightarrow S = 
	\begin{cases}
	4\pi \text{ si } \overrightarrow 0 (pertenece) Int(B)\\
	0 \text{ si } \overrightarrow 0 \text{está fuera de B}\\
	\end{cases}$$
	Siendo $\partial B$ la frontera de B
\end{prop}
\begin{proof}
	\begin{itemize}
	\item Si $\overrightarrow 0$ está fuera de B, el campo $\overrightarrow E$ es regular ($C^{\infty}$) en B.\\
	$$\int_{\partial B} \overrightarrow E \dif \overrightarrow S \stackrel{Tª divergencia}{=} \int_B div \overrightarrow E = 0$$
	
	\item Si $\overrightarrow 0$ pertenece a Int(B):\\
	En lugar de B consideramos $B - B_{\delta}$ con $B_{\delta} = \{ \md{\overrightarrow x} \le \delta \} $\\
	$$\overline{B} = B- B_\delta \rightarrow \int_{\partial {\overline{B}}} \overrightarrow E \dif \overrightarrow S = \int_B div \overrightarrow E = 0$$
	Como $\partial{\overline{B}} = \partial B \cup \partial{B_\delta}$ con orientaciones distintas $\rightarrow$
	
	\end{itemize}
\end{proof}







