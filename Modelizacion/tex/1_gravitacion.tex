\chapter{Gravitación y leyes de kepler}
\section{Leyes de Kepler}

Kepler, a principios del siglo XVII, enunció unas leyes experimentales (con datos de Tycho Brahc):

\begin{lemma}[1ª Ley de Kepler]
	Las órbitas de los planetas son elipses con el Sol en uno de los focos.
\end{lemma}

\begin{lemma}[2ª Ley de Kepler]
	La linea que une un planeta y el Sol barre áreas iguales en tiempos iguales.
\end{lemma}

\begin{lemma}[3ª Ley de Kepler]
	El cuadrado del periodo orbital es proporcional al cubo del semieje mayor de la órbita.
\end{lemma}

\section{Newton}

Gracias a Newton, en 1985, las leyes de Kepler pudieron demostrarse matemáticamente suponiendo que la fuerza gravitacional obedece la ecuación
\[ F = \frac{-GMm}{r^2} \]
en la dirección del radio vector.

No existe una forma sencilla de realizar esta demostración a menos que demos por sentados muchos principios físicos que no eran conocidos en aquella época. Vamos a tratar de realizar la demostración utilizando los mínimos ``trucos'' posibles.

Para empezar debemos recordar que toda fuerza puede calcularse como:
$$ \vec{F} = m\ga$$

La aceleración se calcula a partir de la posición de una partícula, que viene dada por: $\vr = (x(t), y(t), z(t))$. La aceleración se definión como la derivada segunda de la posición con respecto al tiempo. Es decir:
\[ \ga = \frac{d^2\vr}{d t^2} \]

Esta fórmula describe, por ejemplo, la atracción que ejerce el Sol sobre la Tierra, por ello el signo negativo. En el caso de el Sol y la Tierra, la fuerza recíproca no se tiene en cuenta ya que el Sol es demasiado pesado como para ser influido apreciablemente por la gravedad de la Tierra. Hay que hacer mediciones muy finas para poder detectar estas perturbaciones.

Para poder seguir con la demostración de forma relativamente sencilla, \textbf{despreciamos la fuerza de los planetas sobre el Sol y de los planetas entre ellos.}

Por lo tanto, modelamos tomando sólo un planeta y asumiendo un Sol fijo.

Igualando las dós fórmulas para el cálculo de la fuerza que hemos indicado, obtenemos:

$$ \frac{-GMm}{{||r||}^2} \frac{\vr}{{||r||}^2}  =  m \frac{d^2\vr}{d t^2}$$


En primer lugar vamos a probar que las soluciones $\vr = \vr(t)$ de esta ecuación diferencial satisfacen las leyes de Kepler (con las condiciones iniciales de los planetas)

%Esto es un sistema con una llave
$$
\begin{cases}
 x'' = \frac{-GMx}{(x^2 + y^2 + z^2)^{3/2}}\\
 y'' = \frac{-GMy}{(x^2 + y^2 + z^2)^{3/2}}\\
 z'' = \frac{-GMz}{(x^2 + y^2 + z^2)^{3/2}}\\
\end{cases}
$$


 \begin{obs}
 No se puede calcular explicitamente $x$, $y$ y $z$, en función de $t$. $\vr = \vr(t)$ no tiene una fórmula cerrada (en términos de funciones elementales\footnote{Las de la calculadora (sin, cos...)}). Aun así se puede demostrar que la fórmula es una elipse.
 \end{obs}


 %Un método es un truco que sirve varias veces
Vamos a probar que basta demostrar que la curva $\vr = \vr(t)$ está contenida en un plano.

Un posible método es usar que una curva es plana $\Leftrightarrow \tau \text{(torsión)} = 0$ y aplicar la fórmula de la torsión. Dicha fórmula involucra un determinante con una fila $r$ y otra $r''$, como son proporcionales, la torsión sale 0.


Nuestro método más sencillo se basa en tomar $\vec{L}(t) = \vr(x) \times \frac{d\vr}{dt}$.

Derivando obtenemos:
$$\frac{d\vec{L}(t)}{dt} = \frac{d\vr}{dt} \times \frac{d\vr}{dt} + \vr(t) + \frac{d^2\vr}{dt^2} = \vec{0}$$

de donde podemos deducir que $\vec{L}$ es un vector constante $\Rightarrow \vr \perp \vec{L} \Rightarrow \vr(t)$ está contenida en el plano $\vec{L}(x,y,z) = 0 $.
(pensar $\vec{L} = \vec{0}$).

Por comodidad, vamos a girar el plano en que se encuentra la órbita del planeta, $\vr \rightarrow G\vr$, siengo G un giro (matriz ortogonal), para convertirlo en el plano $z=0$.

Podemos comprobar que este giro no afecta a las cuentas puesto que, en definitiva, estamos multiplicando por una constante.

$$\frac{d^2\vr}{dt^2} = \text{cte}\frac{\vr}{||\vr||^3} \Leftrightarrow \frac{d^2(G\vr)}{dt^2} = \text{cte}\frac{G\vr}{||G\vr||^3}$$

En definitiva "girando la cabeza" (aplicando el cambio $\vr \rightarrow G\vr$) podemos suponer que la curva $\vr = \vr(t)$ está contenida en el plano $z = 0 \Rightarrow$ Suponesmos $z(t) = 0$.


%Esto es un sistema con una llave
$$
\begin{cases}
 x'' = k\frac{x}{(x^2 + y^2)^{3/2}}\\
 y'' = k\frac{y}{(x^2 + y^2)^{3/2}}\\
\end{cases}
$$

Ahora hay que probar las leyes de Kepler para $t \rightarrow (x(t), y(t))$.

La forma de la solución será una elipse.
Si escribimos $x$ e $y$ en polares, el sistema nos quedará más sencillo.

% % align, gather y equation.
$$\begin{array}{c}
x = r\cos(\theta)\\
y = r\sin(\theta) \\
r = r(t) \\
\theta = \theta(t)
\end{array}
$$

(r y $\theta$ no se pueden calcular eplícitamente.)

Y con esto sacamos las ecuaciones de gravitación:
\begin{gather}
r''\cos\theta - 2r'\theta'\sin\theta - r(\theta')^2\cos\theta - r\theta''\sin\theta = cte \frac{\cos\theta}{r^2} \label{eq1_kepler}\\
r''\sin\theta + 2r'\theta'\cos\theta - r(\theta')^2\sin\theta + r\theta''\cos\theta = cte \frac{\sin\theta}{r^2} \label{eq2_kepler}
\end{gather}
$\eqref{eq1_kepler}\cdot\cos\theta + \eqref{eq2_kepler}\cdot\sin\theta \rightarrow K= r^3(\theta')^2 - r^2*r'' \rightarrow$ 1ª ecuación de gravitación
$$-\eqref{eq1_kepler}\cdot\sin\theta + \eqref{eq2_kepler}\cdot\cos\theta \rightarrow 0= r \theta'' + 2r' * \theta' \rightarrow \text{2ª ecuación de gravitación}$$


\paragraph{Recordemos:}
2ª ley de Kepler : Un planeta recorre áreas iguales en tiempos iguales.

Vamos a estudiar cual es la fórmula para el área.
Primero nos vamos al caso general, para cualquier curva que tengamos en polares.

En este caso el radio depende de $\theta$ \\
Cogemos una curva R.\\ 

$$A(R) = \int\int_R 1 \dif x \dif y = \int_0^{\theta_0} \int_0^r(\theta) r \dif r \dif \theta = 1/2 \int_0^{\theta_0} r^2(\theta) \dif\theta$$

La idea es coger pequeños triángulos e ir calculando su área

En el caso de la \textbf{ley de Kepler}, si en el tiempo t=0 estamos en $\theta = 0$ y en el tiempo T estamos en $\theta_0 \implies$ la fórmula para el área en función de T sería:

$$A(T) = \frac{1}{2} \int_1^T r^2(t) \theta'(t) \dif t$$
\\
Vamos a probar las leyes de Kepler:
\begin{itemize}
\item \textbf{2ª ley de Kepler} $\iff$ $\frac {\dif A(T)}{\dif T} = cte$

Es lo mismo que decir $A(T)$ es lineal (a tiempos iguales tengo áreas iguales)

$\rightarrow  A(x + T) - A(x) = A(T) - A(0)$\\

Vamos a traducir la 2ª ley de Kepler en algo más familiar (derivadas):

$\frac{\dif A}{\dif T} = cte \iff r^2\theta' = cte$  (no depende de t) \\
*esto se puede pensar como conservación del momento angular

$\implies (r^2\theta')' = 0 \iff 2rr'\theta' + r²\theta'' =0 \rightarrow$ 2ª ecuación de gravitación \\
Hemos probado la 2ª ley de Kepler

\item Ahora vamos a intentar manipular la 1ª ecuación para probar la \textbf{1ª ley de Kepler}. \\
Escribimos:\\ $h= r^2\theta'\\h=cte$ (para cada planeta, $h$ es diferente)

1ª ecuación de gravitación:
$K= r^3(\theta')^2 -r^2*r''$

$h = r^2*\theta'$ permite eliminar $\theta'$ en la 1ªecuación.

$$K= r^3 \frac{h^2}{r^4} - r^2*r''$$
$$K= \frac{h^2}{r} - r^2*r''  \rightarrow  \text{Es una EDO ($r = r(t)$) y no aparece explícitamente la t.}$$

\obs En el curso de EDO se ven métodos para pasar de 2 orden ($r''$) a 1 orden ($r$) y se podría expresar r en términos de la imagen inversa de una integral, pero no se puede calcular en términos elementales.

¿y cómo lo hacemos?

Idea: escribir $r=r(\theta)$ en vez de $r=r(t)$
\end{itemize}









