\chapter{Cadenas de Markov}
\section{Un modelo para redes}
En este capítulo vamos a ver cómo Google a las páginas que muestra.

Supongamos que tenemos una red de páginas web conectadas por enlaces.\\
Google afirma cubrir unas $3\cdot 10^3$ páginas web.
Vamos a ver las páginas como vértices de un grafo y los enlaces como aristas dirigidas(especificando un sentido).

Matemáticamente esto es in grafo dirigido:
DIBUJO GRAFO
¿Cómo ordenar por relevancia los vértices?¿En qué orden aparecerán las páginas?


\textbf{Idea(Markov):} Suponer un paseo aleatorio por los vértices (los internautas navegan al azar siguiendo los enlaces) y una mayor acumulación en ciertos vértices indica que son más importantes.


Aviso: No funciona literalmente, hay que hacer modificaciones.
DIBUJO
A la larga esto converge...\\
DIBUJOS VARIOS
Como había 360 al principio parece que:


\begin{itemize}
	\item Probabilidad de estar en 1 es $\frac{144}{360} = \frac{2}{5}$
	\item Probabilidad de estar en 2 es $\frac{144}{360} = \frac{2}{5}$
	\item Probabilidad de estar en 3 es $\frac{72}{360} = \frac{1}{5}$
\end{itemize}
	Según esto 1 y 2 son igual de importantes y 3 es la mitad de importante.
	
	La distribución que pensamos que da el límite es estacionaria(no cambia con el tiempo).
	
	DIBUJO.
