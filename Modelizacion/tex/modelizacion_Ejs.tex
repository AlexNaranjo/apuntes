% -*- root: ../modelizacion.tex -*-
\section{Hoja 1}
\begin{problem}[1]
Una \textbf{fuerza central} es una fuerza que en $\overrightarrow{x}$ tiene módulo que sólo depende de $||\overrightarrow{x}||$. Por tanto $\overrightarrow F=m\cdot \overrightarrow a$ lleva a una ecuación del tipo
\[\overrightarrow x '' =g(||\overrightarrow x||)\overrightarrow x\]
Calcula la derivada de $\overrightarrow x\times \overrightarrow  x''$ (el momento angular) y deduce de ello que cada curva solución está contenida en un plano. Otra forma (más complicada) de proceder es probar directamente que la torsión de la curva es nula. Investiga este procedimiento usando la fŕomula de la torsión

\solution
	Definimos la fuerza central (la fuerza que va del origen hasta x) de la siguiente forma:$$\overrightarrow F = g(||\overrightarrow x||) \overrightarrow x$$

	Como $\overrightarrow F = m \cdot \overrightarrow a$ entonces $m\overrightarrow x'' = g(||\overrightarrow x||) \overrightarrow x$\\
	De esto último deducimos que $\overrightarrow x \times \overrightarrow x'' = \overrightarrow 0$ puesto que son vectores paralelos.

	Por lo tanto $\frac{\dif}{\dif t}(\overrightarrow x \times \overrightarrow x') = \overrightarrow x \times \overrightarrow x + \overrightarrow x \times \overrightarrow x '' = \overrightarrow 0 + \overrightarrow 0.$

	Y con esto ya tenemos que $\overrightarrow x $ está en un plano ya que:

	$$\overrightarrow x \times \overrightarrow x' = \overrightarrow V_0 \cdot cte \implies \overrightarrow V_0 \cdot \overrightarrow x = 0 \stackrel{V_0 \neq 0}{\implies} \overrightarrow x \text{está en un plano}$$
	¿Qué ocurriría con $V_0 = 0$?

	$x'$ y $x$ serían paralelos, es decir, la velocidad iría en la dirección de x $\implies$ $\overrightarrow x$ está en una recta.

	La explicación de la última implicación se deja como ejercicio.

	Otra forma de ver que $\overrightarrow x$ está en un plano es utilizando la torsión:

	Definimos la curva
	\[t \mapsto \overrightarrow{x(t)}\]
	y la torsión
	\[T = \frac{(\overrightarrow x' \times \overrightarrow x'') \cdot \overrightarrow x'''}{||\overrightarrow x' \times \overrightarrow x ''||}\]

	Por la fórmula de la $\overrightarrow F$ puesdo escribir:
	$$\overrightarrow x'' = \frac{1}{m} g(||\overrightarrow x||)\overrightarrow x $$

	Y si dervivamos:

	$$\overrightarrow x''' = \frac{1}{m} g'(||\overrightarrow x||) \frac{\overrightarrow x \cdot \overrightarrow x'}{||\overrightarrow x||} \overrightarrow x + \frac{1}{m}g(||\overrightarrow x||)\overrightarrow x' = \frac{g'(||\overrightarrow x||)}{g(||\overrightarrow x||)}\cdot \frac{\overrightarrow x \cdot \overrightarrow x'}{||\overrightarrow x||} \overrightarrow x'' + \frac{1}{2}g(||\overrightarrow x||) \overrightarrow x''$$

	Con esto podemos ver que $\overrightarrow x'''$ es combinación lineal de $\overrightarrow x''$ y $\overrightarrow x'$ y por tanto:
	$$(\overrightarrow x' \times \overrightarrow x'' \times \overrightarrow x''') = det(\overrightarrow x', \overrightarrow x'',\overrightarrow x''') = 0 \stackrel{\implies}{\overrightarrow x' \times \overrightarrow x'' \neq 0} T = 0 \implies \text{curva plana}$$
\end{problem}

\begin{problem}[2]

La estación espacial internacional orbita a unos 400 km de la superficie dela Tieraa. Calcula en qué proporción ha disminuido la fuerza de la gravedad a esa altura. ¿Cuánto pesaría una persona de 80kg a esa altura? ¿Por qué entonces las imágenesque nos llegan muestran astronautas y objetos flotando ingrávidos?

\solution
	Queremos ver el peso de una persona a 400 km de la Tierra\\Llamamos M a las masa de la tierra, m al peso de una persona y R al radio de la Tierra.\\ Vemos la relación entre la fuerza que hay sobre la persona en la superficie de la Tierra y la fuerza a 400 km:\\
	$$\frac{\frac{GMm}{R^2}}{\frac{GMm}{(R + 400000 m)^2}} \implies \frac{(R + 4\cdot 10^5)^2}{R^2} = \left(\frac{67.8}{63.8}\right)^2$$
	Por lo tanto, si m = 80, el peso a 400 km sería:
	$$\frac{80}{\left(\frac{67.8}{63.8}\right)^2} = 70 .84 Kg$$

	\textbf{Pregunta del profesor:} Los astronautas están flotando alrededor de la Tierra por la fza.centrífuga. ¿Porqué no ocurre lo mismo en la superficie?
\end{problem}

\begin{problem}[3]
Prueba que la curva en polares $r(\theta)= \frac{a^{-1} \cdot b^2}{1 + e\cos\theta}$ siendo
\[ e = \frac{c}{a} = \sqrt{1 - \frac{b^2}{a^2}}\]
describe la elipse $\frac{x^2}{a^2} + \frac{y^2}{b^2} = 1$ cuando el origen de las coordenadas polares está en uno de los focos.

Prueba también que la curva en polares
\[r(\theta)=\frac{l}{1+e\cos(\theta)}\]
con $e \geq 0$ y $l>0$ constantes, describe una circunferencia si $e=0$, una elipse si $0<e<1$, una parábola si $e=1$ y una hipérbola si $e>1$.

\textbf{Indicación:} Escribe $r(1+e\cos(\theta)$ en cartesianas.

\solution

	Tenemos la elipse:$$r(\theta)= \frac{a^{-1} \cdot b^2}{1 + e\cos\theta} \text{  siendo  } e = \frac{c}{a} = \sqrt{1 - \frac{b^2}{a^2}}$$
		Que es la elipse $$\frac{x^2}{a^2} + \frac{y^2}{b^2} = 1$$ trasladando el origen a uno de sus focos.

		Podemos escribir la ecuación como

		$$ r + er\cos\theta = a^{-1} b^2 \rightarrow
		\begin{cases}
		r = \sqrt{x^2 + y^2}\\
		r\cos\theta = x\\
		\end{cases}$$

		Despejando obtenemos:
		\[(\sqrt{x^2 + y^2})^2 = (\frac{a^{-1}b^2}{1+e\cos\theta})^2 \implies x^2 + y^2 = a^{-2} b^4 + e^2x^2 - 2a^{-1}b^2ex\]
		Como $1-e^2 = \frac{b^2}{a^2}$ tenemos que

		$$\frac{x^2}{a^2}b^2 + y^2 = a^{-2} b^4 - 2a^{-1}b^2ex$$
		Y dividiendo entre $b^2$ y completando cuadrados llego a que:
		$$\frac{(x + a\cdot e)^2}{a^2} + \frac{y^2}{b^2} = 1$$
\end{problem}

\begin{problem}[4]
Sea $V$ el potencial de una fuerza $\overrightarrow F$, esto es $\overrightarrow F = -\nabla V$, que satisface div$\overrightarrow F = 0$. Demuestra que si $V$ es una función radial (sólo depende de la distancia al origen), entonces necesariamente $\overrightarrow F=K||\overrightarrow x||^{-3}\overrightarrow x$, como con la ley de gravitación universal

\solution

Podemos descomponer el potencial en una composición de 2 funciones: 

$\appl{d}{ℝ^3}{ℝ}$, con $d(x,y,z) = \norm{\overrightarrow{x}}$.

$\appl{f}{ℝ}{ℝ}$.

Con esta construcción, para alguna $f$ tenemos $V(\vx) = f(d(\vx))$.  

Como el potencial sólo depende de la distancia, tenemos que demostrar que la única posibilidad para la $f$ es ser $f(x) = \frac{-KM}{x}$, a partir de que $div \vf = 0$, es decir, $div - \left(\grad V\right) = 0$.

Ahora sólo hay que derivar con cuidado:

$$-\overrightarrow{F} = \grad V(\vx) = f'(d(\vx)) · \grad d(\vx) = f'\left(\norm{\vx}\right)\frac{1}{\norm{\vx}} \vx$$

Vemos que $\appl{F}{ℝ^3}{ℝ^3}$. Calculamos la divergencia de $\overrightarrow{F}$

$$div \overrightarrow{F} = \frac{\partial{F}}{\partial{x}} + \frac{\partial{F}}{\partial{y}} + \frac{\partial{F}}{\partial{z}} 
\overset{hip.}{=} 0$$


Para simplificar la notación, vamos a escribir $r = \norm{\vx}$

$$\frac{\partial{F(\vx)}}{\partial{x}} = \left(\underbrace{\dpa{f'(d(\vx))}{x}}_{(1)} · d(\vx) - f(d(\vx))\dpa{d(\vx)}{x} \right)\frac{x}{(d(\vx))^2} + \frac{f'(d(\vx))}{d(\vx)}$$

$(1) = \dpa{d(\vx)}{x} · f''(d(\vx))$ por la regla de la cadena.

Simplificando y utilizando el cálculo anterior, obtenemos:

$$\dpa{F(\vx)}{x} = \frac{x^2}{(d(\vx))^3}\left(f''(d(\vx)) d(\vx) - f'(d(\vx))\right) + \frac{f'(\vx)}{d(\vx)}$$

El cálculo es análogo para las derivadas respecto de $y,z$, con lo que la divergencia queda:

$$div \overrightarrow{F} = \left(f''(d(\vx)) d(\vx) - f'(d(\vx))\right) \cdot \frac{x^2 + y^2 + z^2}{(d(\vx))^3} + \frac{3f'(d(\vx))}{d(\vx)}$$

Tomamos $d(\vx) = r$ y utilizamos $x^2+y^2+z^2 = \left(\sqrt{x^2+y^2+z^2}\right)^2 = d(\vx)^2$

$$ div \overrightarrow{F} = 0 = \left((f''(r)r - f'(r)) + 3f'(r)\right) \frac{1}{r} = 0 \dimplies r f''(r) = -2 f'(r)$$

Hemos llegado a una EDO cuya solución es $g(x) = \frac{k}{x}$, porque:

$$g(x) = \frac{k}{x} \implies g'(x) = \frac{-k}{x^2} \implies g''(x) = \frac{2k}{x^3}$$

Vemos que $x g''(x) = x\frac{2K}{x^3} = \frac{2K}{x^2} = -2 \frac{-K}{x^2} = -2 g'(x)$ que es la EDO que teníamos.

Ahora que tenemos $f$, obtenemos el potencial con la composición que habíamos construido: $$V(\vx) = f(d(\vx)) = \frac{K}{d(\vx)} = \frac{K}{\norm{\vx}}$$

Y para hallar la $\overrightarrow{F}$, utilizamos $$\overrightarrow{F} = - \grad V = -K\left(\frac{-x}{\norm{\vx}^2},\frac{-y}{\norm{\vx}^2},\frac{-z}{\norm{\vx}^2} \right)$$

Y reescribimos : $$\overrightarrow{F} = - \grad V = \frac{K}{\norm{\vx}^3} \vx$$ 
\textcolor{green}{No se muy bien de dónde sale el 3 la verdad... Si se hace en clase corregid o completad o borrad}

\paragraph{Pedro}
Si $V$ es una función radial podemos escribirlo como
\[V=V(||\overrightarrow x||)\]
dado que $V$ es el potencial de una fuerza $\overrightarrow F$, podemos escribirla como:
\[\overrightarrow F = -\left(\frac{dV(||\overrightarrow x||)}{dx_1},\frac{dV(||\overrightarrow x||)}{dx_2},\frac{dV(||\overrightarrow x||)}{dx_3}\right)\]

Puesto que la divergencia es nula tenemos:
\[\frac{d}{dx_1}\frac{dV(||\overrightarrow x||)}{dx_1}+\frac{d}{dx_2}\frac{dV(||\overrightarrow x||)}{dx_2}+...+\frac{d}{dx_n}\frac{dV(||\overrightarrow x||)}{dx_n} = 0\]
Aplicando la regla de la cadena y organizando un poco las ecuaciones llegamos a:
\[\frac{d}{dx_1}\left(\frac{d||\overrightarrow x||}{dx_1}\frac{dV}{dx_1}\left( || \overrightarrow x || \right)\right)+\frac{d}{dx_2}\left(\frac{d||\overrightarrow x||}{dx_2}\frac{dV}{dx_2}\left( || \overrightarrow x || \right)\right)+...+\frac{d}{dx_n}\left(\frac{d||\overrightarrow x||}{dx_n}\frac{dV}{dx_n}\left( || \overrightarrow x || \right)\right) = 0\]

\textcolor{blue}{Pedro: No tengo claro como seguir ni se si lo hecho está bien. Si lo hacemos en clase revisad y corregís lo necesario o incluso lo tiráis todo xD}
\end{problem}

\begin{problem}[5]
Explica por qué en el punto más cercano al Sol de la órbita de una planeta, digamos a distancia $r_p$, la velocidad $v_p$ debe cumplir $v_p=r_p\theta'$. Recuerda que en el movimiento de un planeta $h=r^2\theta$ es constante y que $a(1-e^2)=h^2/GM$ con $a$ el semieje mayor y  $e$ la excentricidad. Deduce de todo ello que $v_p=br^{-1}_p\sqrt{GM/a}$

\solution
%Esto lo comento por que creo que no vale para nada, pero me da pena tirarlo por si acaso
%Por el principio básico de la mecánica (principio de Hamilton) sabemos que la trayectoria del sistema da un extremo de
%\[\int_{t_0}^T L = \int_{t_0}^T E_{cinetica}-E_{potencial} = \int_{t_0}^T \frac{1}{2}mv^2-\frac{GMm}{r}\]
%Sabiendo que la velocidad es la derivada de la posición y que las órbitas son planas podemos escribir:
%\[\int_{t_0}^T L = \int_{t_0}^T \frac{1}{2}m\left(\dot{x}^2+\dot{y}^2\right)-\frac{GMm}{\sqrt{x^2+y^2}}\]
%Puesto que no depende de $t$ tenemos que la energía es constante. Es decir:
%\[E = \dot{x}\frac{\partial L}{\partial \dot{x}}+\dot{y}\frac{\partial L}{\partial \dot{y}}-\frac{1}{2}m\left(\dot{x}^2+\dot{y}^2\right)-\frac{GMm}{\sqrt{x^2+y^2}} = cte\]
%Vamos a trabajar esta ecaución a fin de llegar al resultado buscado:
%\[\dot{x}\dot{x}'+\dot{y}\dot{y}'-\frac{1}{2}\left(\dot{x}^2+\dot{y}^2\right)-\frac{GM}{\sqrt{x^2+y^2}} = cte\]
\end{problem}

\begin{problem}[6]
Prueba que si el semieje mayor de la elipse de un planeta es $a$ entonces su velocidad cuando está a distancia $r$ del Sol es $\sqrt{2GM/r-GM/a}$.

\textbf{Indicación:} Utiliza el problema anterior y la conservación de la energía.
\solution

\end{problem}

\begin{problem}[7]
El cometa Halley tiene distancias máxima y mínima al Sol dadas por $5.25\cdot 10^{12} m$ y por $8.77\cdot10^{10}$ m, respectivamente. Calcula la fórmula de su elipse en coordenadas cartesianas, su valor de $r^2\theta'$ y sus velocidades máxima y mínima.

\solution

\end{problem}