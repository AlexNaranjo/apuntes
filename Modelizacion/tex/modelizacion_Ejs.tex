% -*- root: ../modelizacion.tex -*-
\section{Hoja 1}
\begin{problem}[1]
Una \textbf{fuerza central} es una fuerza que en $\overrightarrow{x}$ tiene módulo que sólo depende de $||\overrightarrow{x}||$. Por tanto $\overrightarrow F=m\cdot \overrightarrow a$ lleva a una ecuación del tipo
\[\overrightarrow x '' =g(||\overrightarrow x||)\overrightarrow x\]
Calcula la derivada de $\overrightarrow x\times \overrightarrow  x''$ (el momento angular) y deduce de ello que cada curva solución está contenida en un plano. Otra forma (más complicada) de proceder es probar directamente que la torsión de la curva es nula. Investiga este procedimiento usando la fŕomula de la torsión

\solution
	Definimos la fuerza central (la fuerza que va del origen hasta x) de la siguiente forma:$$\overrightarrow F = g(||\overrightarrow x||) \overrightarrow x$$

	Como $\overrightarrow F = m \cdot \overrightarrow a$ entonces $m\overrightarrow x'' = g(||\overrightarrow x||) \overrightarrow x$\\
	De esto último deducimos que $\overrightarrow x \times \overrightarrow x'' = \overrightarrow 0$ puesto que son vectores paralelos.

	Por lo tanto $\frac{\dif}{\dif t}(\overrightarrow x \times \overrightarrow x') = \overrightarrow x \times \overrightarrow x + \overrightarrow x \times \overrightarrow x '' = \overrightarrow 0 + \overrightarrow 0.$

	Y con esto ya tenemos que $\overrightarrow x $ está en un plano ya que:

	$$\overrightarrow x \times \overrightarrow x' = \overrightarrow V_0 \cdot cte \implies \overrightarrow V_0 \cdot \overrightarrow x = 0 \stackrel{V_0 \neq 0}{\implies} \overrightarrow x \text{está en un plano}$$
	¿Qué ocurriría con $V_0 = 0$?

	$x'$ y $x$ serían paralelos, es decir, la velocidad iría en la dirección de x $\implies$ $\overrightarrow x$ está en una recta.

	La explicación de la última implicación se deja como ejercicio.

	Otra forma de ver que $\overrightarrow x$ está en un plano es utilizando la torsión:

	Definimos la curva
	\[t \mapsto \overrightarrow{x(t)}\]
	y la torsión
	\[T = \frac{(\overrightarrow x' \times \overrightarrow x'') \cdot \overrightarrow x'''}{||\overrightarrow x' \times \overrightarrow x ''||}\]

	Por la fórmula de la $\overrightarrow F$ puesdo escribir:
	$$\overrightarrow x'' = \frac{1}{m} g(||\overrightarrow x||)\overrightarrow x $$

	Y si dervivamos:

	$$\overrightarrow x''' = \frac{1}{m} g'(||\overrightarrow x||) \frac{\overrightarrow x \cdot \overrightarrow x'}{||\overrightarrow x||} \overrightarrow x + \frac{1}{m}g(||\overrightarrow x||)\overrightarrow x' = \frac{g'(||\overrightarrow x||)}{g(||\overrightarrow x||)}\cdot \frac{\overrightarrow x \cdot \overrightarrow x'}{||\overrightarrow x||} \overrightarrow x'' + \frac{1}{2}g(||\overrightarrow x||) \overrightarrow x''$$

	Con esto podemos ver que $\overrightarrow x'''$ es combinación lineal de $\overrightarrow x''$ y $\overrightarrow x'$ y por tanto:
	$$(\overrightarrow x' \times \overrightarrow x'' \times \overrightarrow x''') = det(\overrightarrow x', \overrightarrow x'',\overrightarrow x''') = 0 \stackrel{\implies}{\overrightarrow x' \times \overrightarrow x'' \neq 0} T = 0 \implies \text{curva plana}$$
\end{problem}

\begin{problem}[2]

La estación espacial internacional orbita a unos 400 km de la superficie dela Tieraa. Calcula en qué proporción ha disminuido la fuerza de la gravedad a esa altura. ¿Cuánto pesaría una persona de 80kg a esa altura? ¿Por qué entonces las imágenesque nos llegan muestran astronautas y objetos flotando ingrávidos?

\solution
	Queremos ver el peso de una persona a 400 km de la Tierra\\Llamamos M a las masa de la tierra, m al peso de una persona y R al radio de la Tierra.\\ Vemos la relación entre la fuerza que hay sobre la persona en la superficie de la Tierra y la fuerza a 400 km:\\
	$$\frac{\frac{GMm}{R^2}}{\frac{GMm}{(R + 400000 m)^2}} \implies \frac{(R + 4\cdot 10^5)^2}{R^2} = \left(\frac{67.8}{63.8}\right)^2$$
	Por lo tanto, si m = 80, el peso a 400 km sería:
	$$\frac{80}{\left(\frac{67.8}{63.8}\right)^2} = 70 .84 Kg$$

	\textbf{Pregunta del profesor:} Los astronautas están flotando alrededor de la Tierra por la fza.centrífuga. ¿Porqué no ocurre lo mismo en la superficie?
\end{problem}

\begin{problem}[3]
Prueba que la curva en polares $r(\theta)= \frac{a^{-1} \cdot b^2}{1 + e\cos\theta}$ siendo
\[ e = \frac{c}{a} = \sqrt{1 - \frac{b^2}{a^2}}\]
describe la elipse $\frac{x^2}{a^2} + \frac{y^2}{b^2} = 1$ cuando el origen de las coordenadas polares está en uno de los focos.

Prueba también que la curva en polares
\[r(\theta)=\frac{l}{1+e\cos(\theta)}\]
con $e \geq 0$ y $l>0$ constantes, describe una circunferencia si $e=0$, una elipse si $0<e<1$, una parábola si $e=1$ y una hipérbola si $e>1$.

\textbf{Indicación:} Escribe $r(1+e\cos(\theta)$ en cartesianas.

\solution

	Tenemos la elipse:$$r(\theta)= \frac{a^{-1} \cdot b^2}{1 + e\cos\theta} \text{  siendo  } e = \frac{c}{a} = \sqrt{1 - \frac{b^2}{a^2}}$$
		Que es la elipse $$\frac{x^2}{a^2} + \frac{y^2}{b^2} = 1$$ trasladando el origen a uno de sus focos.

		Podemos escribir la ecuación como

		$$ r + er\cos\theta = a^{-1} b^2 \rightarrow
		\begin{cases}
		r = \sqrt{x^2 + y^2}\\
		r\cos\theta = x\\
		\end{cases}$$

		Despejando obtenemos:
		\[(\sqrt{x^2 + y^2})^2 = (\frac{a^{-1}b^2}{1+e\cos\theta})^2 \implies x^2 + y^2 = a^{-2} b^4 + e^2x^2 - 2a^{-1}b^2ex\]
		Como $1-e^2 = \frac{b^2}{a^2}$ tenemos que

		$$\frac{x^2}{a^2}b^2 + y^2 = a^{-2} b^4 - 2a^{-1}b^2ex$$
		Y dividiendo entre $b^2$ y completando cuadrados llego a que:
		$$\frac{(x + a\cdot e)^2}{a^2} + \frac{y^2}{b^2} = 1$$
\end{problem}

\begin{problem}[4]
Sea $V$ el potencial de una fuerza $\overrightarrow F$, esto es $\overrightarrow F = -\nabla V$, que satisface div$\overrightarrow F = 0$. Demuestra que si $V$ es una función radial (sólo depende de la distancia al origen), entonces necesariamente $\overrightarrow F=K||\overrightarrow x||^{-3}\overrightarrow x$, como con la ley de gravitación universal

\solution
\textcolor{blue}{Hecho por De Juan. No fiarse al 100\%}

Podemos descomponer el potencial en una composición de 2 funciones:

$\appl{d}{ℝ^3}{ℝ}$, con $d(x,y,z) = \norm{\overrightarrow{x}}$.

$\appl{f}{ℝ}{ℝ}$.

Con esta construcción, para alguna $f$ tenemos $V(\vx) = f(d(\vx))$.

Si logramos demostrar que $f$ tiene que ser de la forma $f(x) = \frac{-KM}{x}$, ya tendremos el ejercicio hecho puesto que su gradiente nos daría una fuerzo con fórmula como la indicada.

Ahora sólo hay que derivar con cuidado:

$$-\overrightarrow{F} = \grad V(\vx) = f'(d(\vx)) · \grad d(\vx) = f'\left(\norm{\vx}\right)\frac{1}{\norm{\vx}} \vx$$

\obs Hay que prestar especial atención al hecho de que al escribir $\vx$ hacemos referencia a un vector y no a una variable.

Vemos que $\appl{F}{ℝ^3}{ℝ^3}$. Calculamos la divergencia de $\overrightarrow{F}$

$$div \overrightarrow{F} = \frac{\partial{F}}{\partial{x}} + \frac{\partial{F}}{\partial{y}} + \frac{\partial{F}}{\partial{z}}
\overset{hip.}{=} 0$$


Basándonos en la expresión de la fuerza calculada anteriormente tenemos:

$$\frac{\partial{F(\vx)}}{\partial{x}} = \left(\underbrace{\dpa{f'(d(\vx))}{x}}_{(1)} · d(\vx) - f'(d(\vx))\dpa{d(\vx)}{x} \right)\frac{x}{(d(\vx))^2} + \frac{f'(d(\vx))}{d(\vx)}$$

$(1) = \dpa{d(\vx)}{x} · f''(d(\vx))$ por la regla de la cadena.

Simplificando y utilizando el cálculo anterior, obtenemos:

$$\dpa{F(\vx)}{x} = \frac{x^2}{(d(\vx))^3}\left(f''(d(\vx)) d(\vx) - f'(d(\vx))\right) + \frac{f'(\vx)}{d(\vx)}$$

El cálculo es análogo para las derivadas respecto de $y,z$, con lo que la divergencia queda:

$$div \overrightarrow{F} = \left(f''(d(\vx)) d(\vx) - f'(d(\vx))\right) \cdot \frac{x^2 + y^2 + z^2}{(d(\vx))^3} + \frac{3f'(d(\vx))}{d(\vx)}$$

Tomamos $d(\vx) = r$ y utilizamos $x^2+y^2+z^2 = \left(\sqrt{x^2+y^2+z^2}\right)^2 = d(\vx)^2$

$$ div \overrightarrow{F} = 0 = \left((f''(r)r - f'(r)) + 3f'(r)\right) \frac{1}{r} = 0 \dimplies r f''(r) = -2 f'(r)$$

Hemos llegado a una EDO cuya solución es $g(x) = \frac{k}{x}$, porque:

$$g(x) = \frac{k}{x} \implies g'(x) = \frac{-k}{x^2} \implies g''(x) = \frac{2k}{x^3}$$

Vemos que $x g''(x) = x\frac{2K}{x^3} = \frac{2K}{x^2} = -2 \frac{-K}{x^2} = -2 g'(x)$ que es la EDO que teníamos.

Por tanto ya tenemos el ejercicio resuleto, pues la $f$ obtenida coincide exactamente con lo que esperábamos obtener. Veamos por que está $f$ garantiza la fórmula para la fuerza que indica el enunciado.
$$V(\vx) = f(d(\vx)) = \frac{K}{d(\vx)} = \frac{K}{\norm{\vx}}$$

Y para hallar la $\overrightarrow{F}$, utilizamos $$\overrightarrow{F} = - \grad V = -K\left(\frac{-x}{\norm{\vx}^{3}},\frac{-y}{\norm{\vx}^{3}},\frac{-z}{\norm{\vx}^{3}} \right)$$

Y reescribimos : $$\overrightarrow{F} = - \grad V = \frac{K}{\norm{\vx}^3} \vx$$

\end{problem}

\begin{problem}[5]
Explica por qué en el punto más cercano al Sol de la órbita de una planeta, digamos a distancia $r_p$, la velocidad $v_p$ debe cumplir $v_p=r_p\theta'$. Recuerda que en el movimiento de un planeta $h=r^2\theta$ es constante y que $a(1-e^2)=h^2/GM$ con $a$ el semieje mayor y  $e$ la excentricidad. Deduce de todo ello que $v_p=br^{-1}_p\sqrt{GM/a}$

\solution
Para probar que $v_p=r_p\dot{\theta}$ recordamos que estamos tomando $v_p = ||\overrightarrow v||_p$.

Cogemos la fórmula general de $v$ : $$ v = ||(\dot{x}, \dot{y})|| \implies v= \sqrt{\dot{x}^2 + \dot{y}^2}$$
Hacemos el cambio de variables
$\begin{cases}
x = r\cos\theta\\
y = r\sin\theta
\end{cases}$ de forma que
$$v = \sqrt{\dot{r}^2 + r^2\dot{\theta}^2}$$
Como en el punto más cercano al sol la distancia alcanza un mínimo $\implies \dot r = 0$, entonces
$$v= r\dot{\theta} \implies v_p = r_p\dot{\theta}$$

Ahora vamos a deducir que $v_p = \frac{b}{r_p}\cdot \sqrt{\frac{GM}{a}}$

Por el enunciado sabemos que
$$v_p = \frac{h}{r_p} = \frac{\sqrt{GMa(1-e^2)}}{r_p}$$
Solo nos queda probar que $\sqrt{a(1-e^2)}= \frac{b}{\sqrt{a}}$ , o l que es lo mismo
$$a(1-e^2) = \frac{b^2}{a}$$
Utilizamos la relación de la excentricidad (e) con los semiejes (a,b).

Sabemos que $e = \frac{c}{a}$, siendo c la distancia focal $\implies c= \sqrt{a^2-b^2}$ entonces $e = \frac{\sqrt{a^2 - b^2}}{a}$.

Sustituyendo:
$$a(1-e^2) = \frac{b^2}{a} \implies a(1-\frac{a^2 - b^2}{a^2}) = \frac{b^2}{a}$$
Vemos que es cierto, por lo tanto $v_p = \frac{b}{r_p}\cdot \sqrt{\frac{GM}{a}}$
%Esto lo comento por que creo que no vale para nada, pero me da pena tirarlo por si acaso
%Por el principio básico de la mecánica (principio de Hamilton) sabemos que la trayectoria del sistema da un extremo de
%\[\int_{t_0}^T L = \int_{t_0}^T E_{cinetica}-E_{potencial} = \int_{t_0}^T \frac{1}{2}mv^2-\frac{GMm}{r}\]
%Sabiendo que la velocidad es la derivada de la posición y que las órbitas son planas podemos escribir:
%\[\int_{t_0}^T L = \int_{t_0}^T \frac{1}{2}m\left(\dot{x}^2+\dot{y}^2\right)-\frac{GMm}{\sqrt{x^2+y^2}}\]
%Puesto que no depende de $t$ tenemos que la energía es constante. Es decir:
%\[E = \dot{x}\frac{\partial L}{\partial \dot{x}}+\dot{y}\frac{\partial L}{\partial \dot{y}}-\frac{1}{2}m\left(\dot{x}^2+\dot{y}^2\right)-\frac{GMm}{\sqrt{x^2+y^2}} = cte\]
%Vamos a trabajar esta ecaución a fin de llegar al resultado buscado:
%\[\dot{x}\dot{x}'+\dot{y}\dot{y}'-\frac{1}{2}\left(\dot{x}^2+\dot{y}^2\right)-\frac{GM}{\sqrt{x^2+y^2}} = cte\]
\end{problem}

\begin{problem}[6]
Prueba que si el semieje mayor de la elipse de un planeta es $a$ entonces su velocidad cuando está a distancia $r$ del Sol es $\sqrt{2GM/r-GM/a}$.

\textbf{Indicación:} Utiliza el problema anterior y la conservación de la energía.
\solution
\textbf{Idea:} En $v_p = \frac{b}{r_p}\cdot \sqrt{\frac{GM}{a}}$ (ejercicio anterior) hay información redundante, ya que $b,a\text{ y } r_p$ están relacionadas, no son independientes.

La conservación de la energía me dice que
$$\frac{1}{2}m v^2 - \frac{GMm}{r} \text{ es cte}$$

Como tengo la velocidad en un punto ($v_p$), me queda
$$\frac{1}{2} v^2 - \frac{GM}{r} = \frac{1}{2} v_p^2 - \frac{GM}{r}$$
Entonces
$$v = \sqrt{\frac{2GM}{r} + v_p^2 - \frac{2GM}{r_p}}$$
¿Cómo simplifico esto? mirando lo que queremos demostrar sólo queda probar que:
$$\frac{2GM}{r_p} - v_p^2 = \frac{GM}{a}$$
Sustituimos $v_p$ por lo que teníamos en el ejercicio anterior:
$$\frac{2GM}{r_p} - v_p^2 =\frac{2GM}{r_p} -\frac{GMb^2}{ar_p^2} = \frac{GM}{r_p}(2-\frac{b^2}{a r_p}) $$
Para que la expresión se parezca más a lo que queremos demostrar pensamos qué relación hay entre $r_p$ y a.


DIBUJO ELIPSE:


Como $e = \frac{c}{a}$ y $r_p = a-c$, sustituyendo:$$\frac{GM}{a(a-e) \cdot (2-\frac{b^2}{a^2(1 - e)})}$$
Para terminar bastaría comprobar que
$$\frac{1}{1-e} \cdot (2- \frac{b^2}{a^2(1-e)}) = 1$$
Es fácil viendo que
$$e^2 = \frac{c^2}{a^2} = 1-\frac{b^2}{a^2} \implies b^2 = a^2(1-e^2)$$
Finalmente
$$\frac{1}{1-e} \cdot (2 - \frac{1 - e^2}{1 - e}) = \frac{1}{1-e} \cdot (2-(1 + e)) = 1$$
Y ya tenemos que $v=\sqrt{\frac{2GM}{r}- \frac{GM}{a}}$
\end{problem}

\begin{problem}[7]
El cometa Halley tiene distancias máxima y mínima al Sol dadas por $5.25\cdot 10^{12} m$ y por $8.77\cdot10^{10}$ m, respectivamente. Calcula la fórmula de su elipse en coordenadas cartesianas, su valor de $r^2\theta'$ y sus velocidades máxima y mínima.

\solution

\textbf{Solución de clase}

$$ \frac{(x+c)^2}{a^2} + \frac{y^2}{b^2} =  1 \;\;\;\; c = a e $$

$$ \begin{cases}
		a + (1+e) = 5.25 * 10^{12} \\
		a(1-e) = 8.77 * 10^{10}
	\end{cases} \} \Rightarrow \begin{cases}
		a = 2.67 * 10^{12} \\
		e = 0.967
	\end{cases}$$

	Con esto calculamos $c = a e = 2.58 * 10^{12}$ y $b = 6.80 * 10^{11}$

\textbf{b)} clase (ver ej. 5)

$$ \frac{h^2}{GM} = a(1- e^2) \;\;\; h = r^2 \theta' $$
$$ h = r^2 \theta' = \sqrt{GMa(1 - e^2)} = 5.53 * 10^{25} $$

\textbf{c)} $v_{\text{max}} v_{\text{min}}$

Hay que tener en cuenta que la velocidad máxima se alcanza en el perihelio (punto de la órbita con distancia más corta) y la mínima en el afelio (punto opesto al perihelio).

$$ v_{\text{max}} = \sqrt{\frac{2GM}{a(1-e)} - \frac{GM}{a}} =^{\text{a)}} 5.45 * 10^4 m/s $$

$$ v_{\text{min}} = \sqrt{\frac{2GM}{a(1+e)} - \frac{GM}{a}} =^{\text{a)}} 9.14 * 10^2 m/s $$ 


\textcolor{blue}{Hecho por Pedro. No fiarse al 100\%}

Conociendo los dos semiejes podemos afirmar que la ecuación de la óribta es de la forma
\[\frac{x^2}{5.25\cdot 10^{12}}+\frac{y^2}{8.77\cdot 10^{10}} = 1 \text{ y } z=0\]

Para calcular el valor de $r^2 \theta'$ basta con acordarnos de que ese valor era constante ($h$) y que, como recordamos en el ejercicios 5,
\[h = \sqrt{GMa(1-e^2)}\]
de donde conocemos todas las constantes.

Por último, para calcular sus velocidades máxima y mínima podemos apoyarnos en el ejercicio anterior, que nos dice que la velocidad a distancia $r$ del Sol viene dada por la ecuación:
\[\sqrt{\frac{2GM}{r}-\frac{GM}{a}}\]
donde nos basta con sustituir el radio por el máximo y el mínimo y listo.

Otra opción para el cálculo de las velocidades habría sido calcular la derivada de la velocidad y su máximo y mínimo.

\end{problem}

\begin{problem}[8]
La excentricidad de la órbita de la Tierra es aproximadamente $e=0.017$. Si en un libro de texto vemos la órbita dibujada con un eje mayor de $20cm$, ¿cuánto debería medir el eje menor?. Suponemos, consecuentemente, la órbita de la Tierra circular. Si en una galaxia lejana hay un planeta hermano de la Tierra con la misma órbita pero recorrida sólo en tres meses, ¿qué relación hay entre la masa de su estrella y la de nuestro Sol?

\solution
\textcolor{blue}{Hecho por Pedro. No fiarse al 100\%}

Puesto que la excentricidad sea calcula como $e=c/a$ siendo $c$ la mitad de la distancia focal y $a$ el semieje mayor de la elipse, podemos deducir fácilmente que la semidistancia focal es
\[c= 0.34cm\]
y sabiendo ahora que $a^2 = c^2+b^2$, con $b$ es semieje menor, podemos despejar y obtener que
\[b = \sqrt{a^2-c^2}=\sqrt{399.8844} = 19.99711 cm\]

Puesto que la diferencia entre los semiejes de la elipse es mínima, tiene sentido considerar la órbita como circular.

Ya sabemos que podemos calcular el área de la órbita de un planeta como
\[A = \frac{1}{2}h T \text{ siendo } h = cte \text{ depende del planeta }\]
Puesto que nuestro nuevo planeta y la Tierra tienen la misma órbita, aunque la recorran a distintas velocidades, su área será la misma.

Por tanto, puesto que nuestro nuevo planeta tiene un período 4 veces menor, tenemos que $h_N=h_T \cdot 4$.

Si recordamos el ejercicio 1.5 ya vimos que
\[a(1-e^2)=\frac{h^2}{GM}\]
puesto que $a,e,G$ no cambian al pasar de estudiar la Tierra a este nuevo planeta, tenemos que
\[h_N^2=\sqrt{a(1-e^2)GM_{NS}}=4h_T=\sqrt{2a(1-e^2)GM} \implies \sqrt{M_{NS}}=\sqrt{2M}\]
es decir, el nuevo Sol tendrá el doble de masa que el nuestro.

\end{problem}

\begin{problem}[9]
Sabiendo que $GM=3.99\cdot 10^{14}m^3s^{-2}$ con $M$ la masa de la Tierra, calcula a qué distancia de su superficie orbitan los satélites goestacionarios: los que están siempre sobre el mismo punto geográfico porque giran a al par que la Tierra, una vez cada 24 horas.
\solution

La 3ª ley de kepler nos da una relación entre la distancia y el periodo orbital. En concreto $ \frac{T^2}{R^3} = $cte.$ = \frac{4\pi^2}{GM} $.

Para que sea geostacionario $\rightarrow T = 24h = 24 * 3600 s$.

$$ R = \left( \frac{GMT^2}{4\pi^2} \right)^{1/3} = 4.23 * 10^7 m $$

$$ R_{s} = R - \text{radio de la tierra} = 4.23 * 10^{7} - 6.38 * 10^{6} = 3.58 * 10^{7} m $$

Aproximadamente 35.000 Km


\end{problem}

\begin{problem}[10]
Usando la ley de Gauss, prueba que en un planeta esférico homogéneo hueco no hay gravedad en el interior.
\solution

\textcolor{blue}{Hecho por Pedro. No fiarse al 100\%}

Según la ley de Gauss la fuerza de la gravedad en un punto de una superficie será proporcional al flujo del campo gravitatorio a través de dicha superficie.

Una vez tomamos un punto en el interior del planeta hueco, consideramos la esfera que lo contiene y que, a su vez, se contiene dentro del planeta.

Para calcular la fuerza de la gravedad en ese punto (o en cualquiera de la esfera que hemos construido, pues todos comparten el mismo valor de gravedad) basta con calcular el flujo del campo gravitatorio en torno a la esfera.

Puesto que esta esfera está hueca y las líneas de campo vienen desde el infinito hasta la superficie del planeta, no hay líneas de campo dentro del planeta por lo que no hay líneas de campo que atraviesen nuestra esfera, por lo que no hay flujo y por tanto, por el teorema de Gauss, no habrá gravedad.
\end{problem}

\begin{problem}[11]
Newton resolvió el problema anterior con un bello argumento geométrico: Fijado un
punto interior se considera un doble cono que lo tiene como vértice. El cono corta a la superficie interior del planeta en dos regiones que cuando se reducen a tamaño infinitesimal ejercen la misma atracción. Intenta elaborar este argumento hasta que te suene convincente.

\solution

\end{problem}

\begin{problem}[12]
Utilizando las ecuaciones de Euler-Lagrange con coordenada generalizada la distancia desde el punto de partida, halla las ecauciones de movimiento de un objeto de masa $m$ que cae por un plano inclinado de ángulo α partiendo del reposo. La energía potencia gravitatoria es $mgy$ donde $g=9.8ms^{-2}$ e $y$ es la altura.

\solution
\textcolor{blue}{Hecho por Pedro. Fiarse -1}
Tomemos nuestra variable $z(t)$ que define la distancia en línea recta desde el punto de partida hasta la posición actual de la masa.

El Lagrangiano sería:
\[\int_a^b L = \int_a^b E_c-E_p = \int_a^b \frac{1}{2} m \left(z'(t)\right)^2-mgz(t)\cos(α)\]
aplicando las ecuaciones de Euler-Lagrange llegamos a
\[\frac{\partial}{\partial t} mz'(t) = -mg\cos(α)\]
de donde podemos deducir que
\[z'(t)=g\cos(α)t \implies z(t)=\frac{1}{2}g \cos(α)t^2\]
Así, la ecuación del movimiento de la partícula sería de la forma
\[\left(x(t),y(t)\right) = \left(x_0+\frac{1}{4}g\sen(2α)t^2,y_0+\frac{1}{2}g \cos^2(α)t^2 \right)\]

\end{problem}

\begin{problem}[13]
Sea $G_α$ un giro en $\real^3$ de ángulo α alrededor de un eje dado por un vector unitario $\vn$ y sea $f(α)=G_α(\vx)$ para un $\vx\in \real^3$. Prueba que $f'(0)=\vn \times \vx$ y utiliza el teorema de Noether para deducir que si $L=\frac{1}{2}m||\vv||^2-V(||\vx||)$, entonces el momento angular $\vx \times m \vv$ se conserva.
\solution

\end{problem}


\begin{problem}[14]
Sabiendo que entre las superficies de revolución cuyos bordes son $\{x^2+y^2 = 4, \ z= \pm 1\}$ hay una de área mínima, prueba que es $\sqrt{x^2+y^2}=C\cosh(z/C)$ con $C \approx 1.69$

\textbf{Indicación:} Recuerda (o demuestra) que si la gráfica de $y=y(x)$, $a\leq x \leq b$ gira alrededor del eje $X$ el área de la superficie resultante es $2\pi \int_a^by\sqrt{1+)y')^2}dx$. Después aplica el cálculo de variaciones preferentemente valiéndote de la energía.
\solution

Este problema es equivalente a hallar la superficie generada por dos aros con jabón formando una superficie que una ambos perímetros. La membrana generada intentará alcanzar un area mínima para tener tensión superficial mínima también.

La superficie buscada, en cilíndricas es $r = r(z)$ (Las superficies de revolución son unión de circunferencias, basta tener el radio a lo largo del eje).

Área de una superficie de revolución: $ 2\pi \int r\sqrt{1+)r')^2}dz  $.

Para minimizar usamos las ecuaciones de Euler-Lagrange \ref{ecuaciones-euler-lagrange}:

$$ L (r, r') = r \sqrt{1+(r')^2} r = r(z)$$

$$\frac{d}{dz} \left( \frac{\partial L}{\partial r'} \right) = \frac{\partial L}{\partial r} $$

Sale pero la ecuación es un poco fea. Así que probamos de otra manera (conservanción de la energía):

La energía: $E = r' \frac{\partial L}{\partial r'} - L $ se conserva (es cte.).

$$ E = r' r \frac{r'}{\sqrt{1 + (r')^2}} - r \sqrt{1 + (r')^2} =^{\text{milagro}} \frac{-r}{\sqrt{1 + (r')^2}} $$
$$ \Rightarrow r' = \sqrt{\left( \frac{r}{E} \right)^2 - 1}  $$

Y resulta que:
$$ \int \frac{dr}{\sqrt{\left( \frac{r}{E} \right)^2 - 1}} = \int dz $$.

$$ E \text{arc}\cosh{\frac{r}{E}} = z + \text{cte.} $$

$$ r = E \cosh{\frac{z}{E} + \text{cte.}} $$

Adicionalmente, para que pase por las circunferencias:

$$ r(1) = 2 \;\;\;\; r(-1) = 2 \;\;\;\; \text{cte} = 0 $$

$$ 2 = E \cosh{\frac{1}{E}} \Rightarrow^{\text{aproximado con el ordenador}} E = 1.69 $$ 


La constante debe ser nula

\end{problem}

\begin{problem}[15]
Para lagrangianos unidimensionales L(x, $\dot{x}$), prueba directamente, únicamente con diversas formas de la regla de la cadena, que las ecuaciones de Euler-Lagrange son invariantes por cambios de variable y = y(x).

\solution

\end{problem}

\section{Hoja 2}

\begin{problem}[1]
	Una cadena de Markov con tres estados, {1, 2, 3}, tiene probabilidades de transición
	$p_{ij} = i/10$ para $1 ≤ i ≤ 3$ y $j ∈ {1, 2}$. Calcula $Prob(X1 = X2 = 2|X0 = 1)$.
	
	\solution
	Las probabilidades de transición son por definición:
	$$P_{ij} = Prob(X_{n+1} = j | X_n = i)$$
	
	Por Bayes sabemos que:
	$$Prob(B\cap C|A) = Prob(B|A) \cdot Prob(C|A\cap B)$$
	
	Siendo
	\begin{itemize}
		\item 	$Prob(B\cap C|A) = \frac{Prob(B\cap C \cap A)}{Prob (A)}$
		\item  $ Prob(B|A) = \frac{B\cap A}{Prob(A)}$
		\item  $Prob(C|A\cap B)= \frac{Prob(C\cap A \cap B)}{Prob(A\cap B)}$
	\end{itemize}
	
	Llamamos $B={X_1 = 2}$ y $C={X_2 = 2}$ de forma que:
	$$Prob(X1 = X2 = 2|X0 = 1) = Prob(X1 = 2 \cap X2 = 2|X0 = 1) = Prob (X_1=2|X_0=1) \cdot Prob(X_2 =2 |X_0 = 1 \cap X_1 =2)$$
	
	Vemos que $Prob (X_1=2|X_0=1)$ es la probabilidad de transición $p_{12}$

	La parte de $Prob(X_2 =2 |X_0 = 1 \cap X_1 =2)$ se puede escribir como $Prob(X_2 =2 | X_1 =2) = p_{22}$ ya que por definición, una cadena de Markov sólo depende del suceso anterior. 
	
	Por lo que nos queda que 
	$$Prob(X1 = X2 = 2|X0 = 1) = p_{12} \cdot p_{22} = \frac{1}{10} \cdot \frac{2}{10} = \frac{1}{50}$$
\end{problem}




\begin{problem}[2]
	Considera la cadena de Markov con cuatro estados y con probabilidades de transici´on
	$p_{12} = p_{24} = p_{43} = p_{32} = 1$. Halla para qué distribuciones iniciales existe una distribución
	límite.
	\solution
	Primero construimos la matriz de transición:
	
	$P = \begin{matrix}
	0&1&0&0\\0&0&0&1\\0&1&0&0\\0&0&1&0\\
	\end{matrix}$
	
	Podríamos utilizar Jordan para calcular el límite, pero en este caso , al haber tantos ceros, es más fácil hacer las operaciones.
	
	Cogemos una distribución inicial
	
	$(\prod_0)= (x,y,z,t)$
	
	que por ser distribución de probabilidad tiene $x+y+z+t=1$
	
	¿$\exists \lim (\prod_0) P^n $?
	
	Haciendo $(\prod_0) \rightarrow (\prod_0)\cdot P \rightarrow (\prod_2) = (\prod_1) \cdot P = (\prod_0)\cdot P^2 \rightarrow . . . . $
	
	Nos queda que
	$$(x,y,z,t) \rightarrow (0,x+z ,t,y) \rightarrow (0,t,y,x + z)\rightarrow(0,y,x+z,t) \rightarrow . . . $$
	Vemos que siempre va a oscilar de la manera $(0,a,b,c) \rightarrow (0,b ,c,a)$
	
	Por lo tanto la única posibilidad de que converjan es que a,b,c sean iguales. Por lo tanto todas las distribuciones iniciales que estoy buscando tienen que cumplir.
	
	$\begin{cases}
		 x + y + z + t = 1\\
		 x+z=t=y
	\end{cases} \implies (\prod_0) = (x,\frac{1}{3} , \frac{1}{3},\frac{1}{3} -x)$ con $0 \leq x \geq \frac{1}{3}$
	
	Y la distribución límite sería:
	$$(\prod_n) = (0, \frac{1}{3}, \frac{1}{3}, \frac{1}{3})$$
\end{problem}

\begin{problem}[3]
	Lanzamos una moneda al aire y consideramos la variable aleatoria $X_n$ que da el número
	de caras en las tiradas n y n − 1. ¿Es ${Xn}^{\infty}_{n=1}$ una cadena de Markov?
	\solution
	
	Para saber si es una cadena de Markov lo que tenemos que ver es si el resultado del caso $n$ depende de $n-1$ , $n-2$ , $n-3$...(caso en el que \textbf{no} sería Markov) o si solo depende de $n$ y $n-1$
	
	En otras palabras, si fuera cadena de Markov cumpliría:
	$$Prob(X_{n+2} = 0| X_{n+1} = 1 \cap X_n = 2) = Prob(X_{n+2} = 0 | X_{n+1} = 1)$$
	
	Vamos a ver cada parte por separado
	\begin{itemize}
		\item Parte izquierda 
		\begin{itemize}
			\item Que $X_n = 2$ significa que en $n$ y en $n-1$ , salió cara
			\item Que $X_{n+1} = 1$ significa que: o en $n$ salió cara o en $n+1$ salió cara. Como sabemos que en $n$ salió cara tenemos que:
			$$ X_{n+1} = 1 \cap X_n = 2 \implies ccx(\text{ cara , cara , cruz})$$
			\item Como $X_{n+2} = 0$ tenemos que:
			$$Prob(X_{n+2} = 0| X_{n+1} = 1 \cap X_n = 2) = Prob(ccxx| ccx)$$
		\end{itemize}
		\item Parte derecha
		\begin{itemize}
			\item Que $X_{n+1} = 1$ puede significar $xc$ o $cx$ (sucesos $n$ y $n+1$)
			\item $X_{n+2} = 0$ significa que tendríamos $cxx$
			
			
			Por lo que la parte derecha es:
			$$Prob(X_{n+2} = 0 | X_{n+1} = 1) = Prob (cxx| xc \cup cx)$$
		\end{itemize}
		
	\end{itemize}
	
	Haciendo las probabilidades nos queda 
	 $$ \frac{1}{2} = \frac{1}{4}$$
	 
	 Entonces no es Markov
	
\end{problem}

\begin{problem}[4]
	Si la matriz $P$ de probabilidades de transición de una cadena de Markov finita es
	simétrica y con todos sus elementos no nulos, explica por qué $P^n$
	tiende a una matriz con
	todos sus elementos iguales.
	\solution
	
	Recordamos que la suma de los elementos de cada fila de $P$ es 1 y por lo tanto, los de $P^n$ también.
	
	De esto y de que todos los elementos de $P^n$ son iguales deducimos que:
	$$\lim P^n = 1/N \left(\begin{matrix}
	1&.&.&.&.&1\\
	.&&&&&.\\
	.&&&&&.\\
	.&&&&&.\\
	.&&&&&.\\
	1&.&.&.&.&1
	\end{matrix} \right)$$ 
	
	
	
\end{problem}

\begin{problem}[5]
	En una cadena de Markov hay tres estados y se sabe $p_{11} = 3_{p_{13}} = 1/2, p_{21} = 3/4 y
	p_{22} = p_{31} = p_{33} = 0$. Halla el resto de las probabilidades de transición, justifica que hay una distribución límite y hállala.
	\solution
	
	Para calcular $P$ utilizamos que la suma de todos los elementos de cada fila tiene que dar 1.
	$$P = \left( \begin{matrix}
	\frac{1}{2}&\boxed{\frac{1}{3}}&\frac{1}{6}\\
	\frac{3}{4}&0&\boxed{\frac{1}{4}}\\
	0&\boxed{1}&0
	\end{matrix}\right)$$
	Para calcular el límite vamos a ver primero que sea regular porque regular $\implies \exists \lim (\prod_0)\cdot P^n$ y además es la única distribución estacionaria. 
	
	Para ver que es regular vemos que cumpla que $P^k$ tenga todos los elementos > 0 para algún $k$
	
	$$P^2 = \left( \begin{matrix}
	1/2&1/3&1/6\\
	3/8&1/2&1/8\\
	3/4&0&1/4\\
	\end{matrix}\right)$$
	
	Con esto ya está claro que para $P^4$ todos los elementos son > 0 (y probablemente también para $P^3$) por lo tanto es regular.
	
	Ahora vamos a calcular la distribución estacionaria y con eso ya tendríamos el valor del límite.
	
	Sabemos que la distribución estacionaria $(\prod)$ va a cumplir que:
	\begin{itemize}
		\item $(\prod) = (x,y,z)$
		\item $x+y+z = 1$
		\item $(x,y,z) \cdot P = (x,y,z)$
	\end{itemize}
	lo que me da es siguiente sistema
	
	$$\begin{cases}
	\frac{x}{2} + \frac{3}{4}y = x\\
	\frac{x}{3} + z = y\\
	\frac{x}{6} + \frac{y}{4} = z\\
	x + y + z = 1
	\end{cases} \implies \text{Sol : } \lim(\prod_0) \cdot P^n = (\frac{1}{2}, \frac{1}{3}, \frac{1}{6})$$
	
	
\end{problem}

\begin{problem}[6]
	
	\solution
\end{problem}

\begin{problem}[7]
	
	\solution
\end{problem}

\begin{problem}[8]
	
	\solution
\end{problem}

\begin{problem}[1]
	
	\solution
\end{problem}

\begin{problem}[9]
	
	\solution
\end{problem}

\begin{problem}[10]
	
	\solution
\end{problem}

\begin{problem}[11]
	
	\solution
\end{problem}

\begin{problem}[12]
	
	\solution
\end{problem}

\begin{problem}[13]
	
	\solution
\end{problem}

\begin{problem}[14]
	
	\solution
\end{problem}

\begin{problem}[15]
	
	\solution
\end{problem}