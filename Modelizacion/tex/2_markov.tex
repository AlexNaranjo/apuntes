\chapter{Cadenas de Markov}

\section{Un modelo para redes}
En este capítulo vamos a ver cómo Google a las páginas que muestra.

Supongamos que tenemos una red de páginas web conectadas por enlaces.\\
Google afirma cubrir unas $3\cdot 10^3$ páginas web.
Vamos a ver las páginas como vértices de un grafo y los enlaces como aristas dirigidas(especificando un sentido).

Matemáticamente esto es in grafo dirigido:
DIBUJO GRAFO
¿Cómo ordenar por relevancia los vértices?¿En qué orden aparecerán las páginas?


\textbf{Idea(Markov):} Suponer un paseo aleatorio por los vértices (los internautas navegan al azar siguiendo los enlaces) y una mayor acumulación en ciertos vértices indica que son más importantes.


Aviso: No funciona literalmente, hay que hacer modificaciones.
DIBUJO
A la larga esto converge...\\
DIBUJOS VARIOS
Como había 360 al principio parece que:


\begin{itemize}
	\item Probabilidad de estar en 1 es $\frac{144}{360} = \frac{2}{5}$
	\item Probabilidad de estar en 2 es $\frac{144}{360} = \frac{2}{5}$
	\item Probabilidad de estar en 3 es $\frac{72}{360} = \frac{1}{5}$
\end{itemize}
	Según esto 1 y 2 son igual de importantes y 3 es la mitad de importante.
	
	La distribución que pensamos que da el límite es estacionaria(no cambia con el tiempo).
	
	DIBUJO.

% Empieza clase del 4 de Febrero

Recordamos que en la red anterior cuando el tiempo tiende a infinito parece que la distribución de personas se estabiliza.

Parece además que sea cual sea la distribución inicial llegamos a proporciones similares. En este caso llegamos a la distribución límite $(\frac{2}{5},\frac{2}{5},\frac{1}{5})$, que es una distribución estacionaria (no varía con el tiempo).

(DIBUJO DE EVOLUCIÓN POR PASOS TENIENDO EN CUENTA PROBABILIDADES)

\section{Propiedades de los paseos aleatorios}

\textbf{¿Esta simulación con paseos aleatorios siempre sirve para ordenar los vértices (páginas web)?}

Debemos obsevar las siguientes preguntas:

\begin{itemize}
	\item ¿Existe siempre una distribución estacionaria?
	\item Si existe una distribución estacionaria ¿es única?
	\item El procedimeinto, ¿Da lugar siempre a una distribución límite?
	\item Si la distribución límite existe, ¿Es independiente de la distribución inicial?

\end{itemize}

\subsection{Si existe una distribución estacionaria ¿es única?}
\label{P2}

(BIBUJO CON DOS GRAFOS IGUALES AL ANTERIOR CONECTADAS CON UN NODO INTERMEDIO QUE SOLO TIENE LINKS HACIA FUERA DE MANERA QUE QUEDAN INCOMUNICADAS)
% nodo 4 es internet oscura

Como desde un lado de la red no se puede acceder al otro podemos decir que estas dos distribuciones son estacionarias:

$(1, 2, 3, 4, 5, 6, 7)$

$(\frac{2}{5},\frac{2}{5},\frac{1}{5}, 0, 0 , 0, 0)$ es estacionaria.

$(0, 0, 0, 0, \frac{1}{5}, \frac{2}{5}, \frac{2}{5})$ es estacionaria.


\subsection{¿Existe siempre una distribución estacionaria?}

\begin{example}{sencillo}

	(DIBUJO CON TRES NODOS CADA UNO CON UNA ENTRADA Y UNA SALIDA)

	Si pones todos en un nodo oscila:

	$1\; 0\; 0 \rightarrow 0\; 0\; 1 \rightarrow 0\; 1\; 0 \rightarrow 1\; 0\; 0$

	Se puede argumentar que con una distribución inicial uniforme si funciona:

	$\frac{1}{3} \; \frac{1}{3} \; \frac{1}{3} \; \rightarrow \frac{1}{3} \; \frac{1}{3} \; \frac{1}{3} \; $

	Pero esto no es verdad para cualquier grafo: partiendo de la equidistribución no se tiene la convergencia en general.

	(DIBUJO IGUAL QUE EL ANTERIOR PERO AÑADIENDO UN NODO MÁS, QUE SOLO TIENE UNA FLECHA SALIENTE A UNO DE LOS NODOS DEL DIBUJO ANTERIOR)

	$(\frac{1}{4},\frac{1}{4},\frac{1}{4},\frac{1}{4}) \rightarrow (\frac{1}{2},\frac{1}{4},\frac{1}{4}, 0) \rightarrow (\frac{1}{4},\frac{1}{4},\frac{1}{2},0)$ oscila como antes

	\textbf{La respuesta es negativa en general}

\end{example}


\subsection {Si la distribución límite existe, ¿Es independiente de la distribución inicial?}


Basta usar como distribución inicial las distribuciones estacionarias que vimos para la segunda pregunta \ref{P2}.

\begin{obs} 

Para el grafo de \ref{P2} se puede probar que $\exists$ límite:

$$(1-t)(\frac{2}{5},\frac{2}{5},\frac{1}{5},0,0,0,0) + t (0,0,0,0,\frac{1}{5},\frac{2}{5},\frac{2}{5}) \;\;\; 0 \leq t \leq 1 $$

son infinitos contraejemplos como los necesarios para la segunda pregunta \ref{P2} y la cuarta.

\end{obs}


\textbf{Spoiler} P1 es verdad y "perturbando" un poco el grafo (de manera muy sencilla) todas las preguntas tienen respuesta afirmativa.



\section{Cadenas de Markov}

\begin{defn}[Cadena de markov]
	Es una sucesión de variables aleatorias $\{X_n\}_{n=0}^{\infty}$ que toman valores en un conjunto numetarble S (\textbf{conjunto de estados}) tal que 
	$$Prob(X_{n+1} = V | X_n = U) = Prob(X_{n+1} = V | X_n =U , X_{n-1} = U_{n-1} , ... , X_0 = U_0)$$
	para cualesquiera $n \geq 0$ $U,V,U_0,.... ,U_{n-1} \in S$
\end{defn}


\obs Además supondremos que esta probabilidad no depende de n.


\textbf{Idea intuitiva: } n es el tiempo discretizado, lo que ocurra mañana depende con cierta probabilidad de lo que ocurre hoy sin que importe conocer la historia anterior.

Decir que no depende de n es decir que según pasa el tiempo, las reglas no cambian.

\begin{example}
	$X_n =$ suma de puntuaciones el día $n$ al tirar un dado cada día.
	
	El conjunto de todos los estados serían todas las puntuaciones posibles . (S = naturales)
\end{example}

\begin{example}[2]
	S = puntuaciones en el tenis.\\
	$X_n =$ puntuación el en punto n\\
	Suponemos que el jugador A tiene prob $p > 1/2$ de ganar cada punto.\\
	Pensando un poco las puntuaciones en el tenis vemos que S solo puede tener 20 elementos.\\
	S= puntuaciones numéricas, deuce, advantages(A o B) , victoria (A o B).\\
	Se puede comprobar que:
	$$Prob(\text{victoria A}) = \frac{p^4 \cdot(1- 16(1-p)^4)}{p^4 - (1-p)^4}$$
	Esto es como curiosidad, ver que con tener sólo un poco más de probabilidad de ganar un punto, la probabilidad de ganar un juego y un set va creciendo.
\end{example}




Si $|S| = \infty$ entonces la cadena de Markov es infinita y se supone $S = \ent^+ (\nat -\{0\})$


Si $|S| < \infty$ entonces la cadena de Markov es finita y se supone $S =\{1,2,...,N\}$

\begin{defn}[Probabilidad de transición]
	del estado i al j
	$$P_{ij} = Prob(X_{n+1} = j| X_n = i) = Prob (X_1 = j| X_0 = i)$$
\end{defn}

Los $P_{ij}$ forman la \textbf{matriz de transición}.


\textbf{Propiedades}
\begin{enumerate}
	\item $\sum_{j \in S} P_{ij} = 1$
	\item $Prob(X_1 = j)= \sum_{i \in S} Prob(X_0 = i)P_{ij}$\\
\end{enumerate}

\begin{proof}
	\begin{enumerate}
		\item $\sum_{j \in S} P_{ij} = \sum_{j\in S} Prob(X_1 = j | X_0 = i) = Prob (X_1 \in S | X_0 = i) = 1$
		\item Utilizando la ley de la probabilidad total
	
		$$Prob(X_1 = j) = \sum_{i \in S} Prob (X_0 = i) \cdot Prob(X_1 = j| X_0 = i)$$
	\end{enumerate}
\end{proof}

	\begin{defn}[Ley de la probabilidad total]
		Sea  $\{A_1, A_2,...,A_n\}$ una partición de $\Omega$ con $P(A_i)>0 \forall i=1,2,...,n$. Entonces, $\forall B \subset \Omega$ medible (perteneciente a $\algb{M}$):
		\[
		P(B)=\sum_{i=1}^{n}P(B\cap A_i)=\sum_{i=1}^{n}P(B|A_i)P(A_i)
		\]
		(Se obtiene de despejar de la formula de la probabilidad condicionada: $P(A|B)=\frac{P(A \cap B)}{P(B)}$)
	\end{defn}


De las dos propiedades deducimos que:
\begin{itemize}
	\item \textbf{(prop.1)} La matriz de transición $P_{ij} i,j \in S$ tiene elementos $0 \leq P_{ij} \leq 1$ y la suma de los elementos de cada fila es 1.
	\item Escribiendo ($\prod_0$) = ${Prob (X_0 =i)}_{i \in S}$ como vector fila, y ($\prod_n$) = ${Prob (X_n =i)}_{i \in S}$
	
	Entonces por \textbf{(prop.2)} : ($\prod_0$) $ = (\prod_n)\cdot P$ (P = matriz de transición)
\end{itemize}

De la misma forma $(\prod_2) = (\prod_1) \cdot P$, etc...

Iterando nos queda:
$$(\prod_n) = (\prod_0) \cdot P^n$$

Ahora vemos cómo aplicar esto a las páginas web.
\begin{example}
	DIBUJO DEL GRAFO DE SIEMPRE
	
	S= páginas web (vértices)\\
	$P_{ij}$ = probabilidad de , estando en la página i, llegar a la página j en el instante siguiente.
	$$Prob(X_1 = (2)| X_0 = (1)) = 1$$
	$$P =\left(\begin{matrix}
	0 & 1 & 0\\
	1/2 & 0 & 1/2\\
	1 & 0 & 0
	\end{matrix}\right)$$
\end{example}



\begin{example}[Otro]
	DIBUJO DE OTRO TRUÑO DE ESOS QUE PONE EN LA PIZARRA
	
	$$P = \left( \begin{matrix}
	0 & 1 & 0 & 0\\
	0 & 0 & 1 & 0\\
	1 & 0 & 0 & 0\\
	1 & 0 & 0 & 0\\
	\end{matrix}\right)$$
	Este ejemplo habiamos visto que no funcionaba bien porque no convergia a nada, oscilaba.
\end{example}



Teniendo en cuenta que en la realidad la P sería una matriz de $3\cdot 10^{13} \times 3\cdot 10^{13}$ la forma más fácil de calcular $P^n$ es con la forma canónica de Jordan.

Volviendo al primer ejemplo:
\begin{example}[1er caso]
$$P =\left(\begin{matrix}
0 & 1 & 0\\
1/2 & 0 & 1/2\\
1 & 0 & 0
\end{matrix}\right)$$
Hacemos Jordan:
$$P = C^{-1} \left(\begin{matrix}
	1&&\\
	&z&\\
	&&\overline{z}
	\end{matrix}\right) C$$
Haciendo los cálculos nos queda $z = \frac{1}{\sqrt{2}}\cdot e^{\frac{3\pi i}{4}}$

Por lo tanto:
$$P^n =  C^{-1} \left(\begin{matrix}
1&&\\
&z^n&\\
&&\overline{z}^n
\end{matrix}\right) C \stackrel{n\rightarrow \infty}{\rightarrow}  C^{-1} \left(\begin{matrix}
1&&\\
&0&\\
&&0
\end{matrix}\right) C = \left(\begin{matrix}
2/5&2/5&1/5\\
2/5&2/5&1/5\\
2/5&2/5&1/5
\end{matrix}\right) = L$$
Esto implica que:
$$\lim_{n\rightarrow\infty}(\prod_n) = \lim_{n\rightarrow\infty}(\prod_0)\cdot P^n = (\prod_0)\cdot L =( \begin{matrix}
2/5&2/5&2/5
\end{matrix})$$
Que es independiente de $\prod_0$
\end{example}

Vamos a comprobar si esto también pasa para el segundo ejemplo.
\begin{example}[2o caso]
	DIBUJO TRUÑO
	
	$$P = \left( \begin{matrix}
	0 & 1 & 0 & 0\\
	0 & 0 & 1 & 0\\
	1 & 0 & 0 & 0\\
	1 & 0 & 0 & 0\\
	\end{matrix}\right)= C^{-1} \left(\begin{matrix}
	1&&&&\\
	&w&&&\\
	&&\overline{w}&\\
	&&&0
	\end{matrix}
	\right)$$
	$w = e^{\frac{2\pi i}{3}}$ , si seguimos calculando las potencias de $w$ y de $\overline{w}$ vemos que van oscilando entre los valores $w , \overline{w} , 1$, por lo que vemos que en este caso $P^n$ no tiene límite.
	$$\exists(\prod_0) : \nexists \lim (\prod_0)\cdot P^n$$
	Sin embargo con $(\prod_0) = (\begin{matrix}
	1/3&1/3&1/3&0
	\end{matrix}) $ se comprueba $(\prod_0) = (\prod_0)\cdot P$
	
	
	Con lo que para ese $(\prod_0)$ si tenemos límite.Ya veremos más adelante cómo se llaman este tipo de distribuciones.
\end{example}

Lo que hemos comprobado con estos dos ejemplos es que si conocemos la forma canónica de Jordan podemos saber si tenemos convergencia o no.

\paragraph{Resumen de lo visto hasta ahora}
\begin{itemize}
	\item Cadenas de Markov $\begin{cases}
	(\prod_0) = \{P(X_0 = i)\}_{i \in S} \rightarrow \text{\textbf{distribución inicial}}\\
	P = \text{Matriz de transición}
	\end{cases}$
	\item Distribución de probabilidad en el instante 0 = $(\prod_0)P^n$
	\item $P^n$ se calcula con la forma canónica de Jordan
	\item En general nos queremos que exista la \textbf{distribución límite} : $\lim_{n\rightarrow\infty} (\prod_0) P^n$
\end{itemize}

\obs Si existe $(\prod) = \lim (\prod_0) P^n$ entonces $(\prod) = (\prod) P$
\begin{proof}
	$(\prod) = \lim(\prod_0) P^{n+1} = (\lim (\prod_0)P^n)P = (\prod) P$
\end{proof}

\begin{defn}[Distribuciones estacionarias]
	Se dice que $(\prod)$ es una distribución de probabilidad estacionaria si:
	$$(\prod)\cdot P = (\prod)$$	
\end{defn}

\obs Con estas distribuciones podemos calcular el límite y estudiar la convergencia sin utilizar la forma canónica de Jordan.


Recordando las preguntas que nos hicimos con las propiedades de los paseos aleatorios vemos que , experimentalmente, las cadenas de Markov muy interconectadas responden afirmativamente a estas preguntas.

Vamos a ver dos versiones de lo que se entiende por \textbf{conexión} de una cadena de Markov
\begin{itemize}
	\item \begin{defn}[Irreducible]
		Se dice que una cadena de Markov es \textbf{irreducible} si se puede ir de un estado a otro en un número finito de pasos. Es decir
		$$\forall i,j \in S\text{    }\exists k : P(X_k = j| X_0 = i) \neq 0$$
		Otra forma de definirlo es con la matriz de transición:
		$$\forall i,j \in S \text{    }\exists k: (P^k)_{ij} \neq 0$$
	\end{defn}
	\item \begin{defn}[Regular]
		Se dice que una cadena de Markov es \textbf{regular} si existe un número de pasos tal que dándolos se puede pasar de un estado a cualquier otro. Es decir:
		$$ \exists k : P(X_k = j| X_0 = i) \neq 0  \forall i,j \in S $$
		Otra forma de definirlo es con la matriz de transición:
		$$\exists k : \text{ todos los elementos de} P^k \text{ son positivos.}$$
	\end{defn}
\end{itemize}





